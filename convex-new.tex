\chapter{Positive Gauss curvature}




\section*{Closed surfaces}

\begin{thm}{Lemma}\label{lem:gauss=sphere}
Assume $\Sigma$ is a closed smooth surface with positive Gauss curvature.
Then $\Sigma$ is a smooth sphere; that is, $\Sigma$ admits a smooth regular parametrization by $\SS^2$.
\end{thm}

\begin{wrapfigure}{O}{33 mm}
\vskip-0mm
\centering
\includegraphics{mppics/pic-78}
\vskip-0mm
\end{wrapfigure}

\parit{Proof.}
Without loss of generality we can assume that the origin lies in the interior of the convex region $R$ bounded by $\Sigma$.

By convexity of $R$, any half-line starting at the origin intersects $\Sigma$ at a single point;
that is, there is a positive function $\rho\:\SS^2\to\RR$ such that $\Sigma$ is formed by points $q=\rho(\xi)\cdot \xi$ for $\xi\in \SS^2$.

Let us show that $\rho$ is a smooth function.
Fix a point $p=(x_p,y_p,z_p)$ on $\Sigma$.
Consider a local implicit description of $\Sigma$ at $p$ as a solution of equation $h(x,y,z)=0$ with nonvanishing gradient;
so $h(p)=0$.
Note that for any point $q$ in a neighborhood of $p$, we have that 
\[h(q)=0 \iff q\in \Sigma \iff q=\rho(\xi)\cdot\xi\] for some $\xi\in \SS^2$.
In other words $h$ defines implicitly $\rho$ as in the implicit function theorem.



Recall that $\nabla_p h\perp \T_p$.
Since the origin lies in the interior of $R$, it cannot lie on $\T_p$;
that is, $\langle\nabla_p f,p \rangle\ne0$;
or equivalently $D_\eta h(p)\ne 0$, where $\eta=\tfrac{p}{|p|}$ is the unit vector in the direction of $p$ and $D$ denotes the directional derivative.

Fix a $(u,v)$ chart $s$ on $\SS^2$ in a neighborhood of $\eta$;
note that the map 
\[S\:(u,v,w)\mapsto w\cdot s(u,v)\] 
is smooth and regular for $w>0$; that is, the vectors 
\[\tfrac{\partial S}{\partial u},\quad \tfrac{\partial S}{\partial v},\quad  \tfrac{\partial S}{\partial w}=w\] are linearly independent.
Note that the function $h\circ S$ is smooth and
$\tfrac{\partial h\circ S}{\partial \rho}(p)=D_\eta h(p)\ne 0$.
Applying implicit function theorem, we get that $\rho$ is smooth in a neighborhood of $\eta$;
since $\eta$ is arbitrary, $\rho$ is smooth on whole $\SS^2$.
\qeds


If one only needs to show that $\Sigma$ is a topological sphere, then one only needs to show that $\rho$ is continuous.
The latter is a consequence of another classical result in topology --- the so-called \emph{closed graph theorem}.





\section*{Open surfaces}

\begin{thm}{Lemma}\label{lem:graph}
Suppose $\Sigma$ is an open surface with positive Gauss curvature.
Then there is a coordinate system such that 
$\Sigma$ is a graph $z=f(x,y)$ of a convex function $f$ defined on a convex open region of the $(x,y)$-plane.
\end{thm}

\parit{Proof.}
The surface $\Sigma$ is a boundary of an unbounded closed convex region $R$.

We can assume that the origin lies on $\Sigma$.
Consider a sequence of points $x_n\in \Sigma$ such that $|x_n|\z\to \infty$ as $n\to \infty$.
Set $u_n=\tfrac{x_n}{|x_n|}$; this is the unit vector in the direction from $x_n$.
Since the unit sphere is compact, we can pass to a subsequence of $(x_n)$ such that $u_n$ converges to a unit vector $u$.

\begin{wrapfigure}[15]{o}{28 mm}
\vskip-0mm
\centering
\includegraphics{mppics/pic-81}
\vskip-0mm
\end{wrapfigure}

Note that for any $q\in \Sigma$, the directions $v_n=\tfrac{x_n-q}{|x_n-q|}$ converge to $u$ as well.
Indeed, the sequences of vectors $\tfrac{x_n-q}{|x_n-q|}$, $\tfrac{x_n-q}{|x_n|}$ and 
$\tfrac{x_n}{|x_n|}$ have the same limit since $\tfrac{|q|}{|x_n|}\to 0$
and $\tfrac{|x_n|}{|x_n-q|}\to 1$ as $n\to \infty$;
the latter follows since
\begin{align*}
|\tfrac{|x_n|}{|x_n-q|}-1|&=|\tfrac{|x_n|-|x_n-q|}{|x_n-q|}-1|\le
\\
&\le\tfrac{|p-q|}{|x_n-q|}.
\end{align*}

Morover, the half-line from $q$ in the direction of $u$ lies in $R$.
Indeed any point on the half-line is a limit of points on the line segments $[q,x_n]$;
since $R$ is closed, all of these points lie in $R$.


Let us choose the $z$-axis in the direction of $u$.
Note that there is no point $p\in\Sigma$ with vertical tangent plane.
Otherwise $\Sigma$ would contain a vertical half-line starting from $p$ and therefore Gauss curvature of $\Sigma$ would vanish.
It follows that any vertical line can intersect $\Sigma$ at most at one point.
That is, $\Sigma$ is a graph $z=f(x,y)$ of a smooth convex function $f$.

Let $\Omega$ be the domain of definition of $f$.
Note that $\Omega$ is the projection of $\Sigma$ to the $(x,y)$-plane 
which is the same as the projection of $R$.
Since $R$ is convex, so is $\Omega$.

Since no tangent plane is vertical, the projection from $\Sigma$ to the $(x,y)$-plane is regular.
By inverse function theorem, the set $\Omega$ is open.\qeds

 




 

