\chapter{Parallel transport}

\section{Parallel fields}

Let $\Sigma$ be a smooth regular surface in the Euclidean space and $\alpha\:[a,b]\z\to \Sigma$ be a smooth curve.
A smooth vector-valued function $t\mapsto v(t)$ is called a \emph{tangent field} on $\alpha$ if
the vector $v(t)$ lies in the tangent plane $\T_{\alpha(t)}\Sigma$ for each $t$.

A tangent field $v(t)$ on $\alpha$ is called \emph{parallel} if $v'(t)\perp\T_{\alpha(t)}$ for any~$t$.

In general the family of tangent planes $\T_{\alpha(t)}\Sigma$ is not parallel.
Therefore one can not expect to have a truly parallel family $v(t)$ with $v'\equiv 0$.
The condition $v'(t)\perp\T_{\alpha(t)}$ means that this family is as parallel as possible, it rotates together with the tangent plane, but does not rotate inside the plane.

Note that according to Claim~\ref{clm:gamma''}, for any geodesic $\gamma$, the velocity field $v(t)=\gamma'(t)$ is parallel along $\gamma$.


\begin{thm}{Exercise}\label{ex:parallel}
Let $\Sigma$ be a smooth regular surface in the Euclidean space, 
$\alpha\:[a,b]\to \Sigma$ a smooth curve 
and $v(t)$, $w(t)$ parallel vector fields along $\alpha$.
\begin{enumerate}[(a)]
 \item Show that $|v(t)|$ is constant.
 \item Show that the angle $\theta(t)$ between $v(t)$ and $w(t)$ is constant.
\end{enumerate}
\end{thm}

\section{Parallel transport}

Assume $p=\gamma(a)$ and $q=\gamma(b)$.
Given a tangent vector $v\in\T_p$ there is unique parallel field $v(t)$ along $\alpha$ such that $v(a)=v$.
The latter follows from Picard's theorem; the uniqueness also follows from Exercise~\ref{ex:parallel}.

The vector $v(b)\in\T_q$ is called the \emph{parallel transport} of $v$ along $\alpha$ and denoted as $\iota_\alpha(v)$.

As it follow from the Exercise~\ref{ex:parallel}, parallel transport $\iota_\alpha\:\T_p\to\T_q$ is an an isometry;
it depends on the choice of $\alpha$ --- for another curve $\beta$ connecting $p$ to $q$ in $\Sigma$, the parallel transport $\iota_\beta\:\T_p\to\T_q$ might be different.

To interpret the parallel transport physically, 
think of walking along $\alpha$ and carrying a perfectly balanced bike wheel in such a way that you touch only its axis keeping it normal to $\Sigma$.
It should be physically evident that if the wheel is non-spinning at the starting point $p$, then it will not be spinning after stopping at $q$.\footnote{Indeed, by pushing axis one can not produce torque to spin the wheel.}
The map that sends the initial position of the wheel to the final position is  the parallel transport~$\iota_\alpha$.

This physical interpretation was suggested by Mark Levi \cite{levi}; it will be used further.

On a more formal level, one can choose a partition $a=t_0<\dots\z<t_n=b$ of $[a,b]$
and consider the sequence of orthogonal projections $\phi_i\:\T_{\alpha(t_{i-1})}\to\T_{\alpha(t_i)}$.
For a fine partition, the composition 
\[\phi_n\circ\dots\circ\phi_1\:T_p\z\to\T_q\]
gives an approximation of $\iota_\alpha$.\footnote{Each $\phi_i$ does not increase the magnitude of a vector and neither the composition.
It is straightforward to see that if if the partition is sufficiently fine, then it almost preserves the composition almost preserves magnitudes.}

\begin{wrapfigure}{r}{39 mm}
\begin{lpic}[t(-0 mm),b(-4 mm),r(0 mm),l(0 mm)]{pics/unbend(1)}
\lbl[t]{11.5,29;$p$}
\lbl[r]{10.5,37;$p_t$}
\lbl[t]{32.5,26;$q$}
\lbl[t]{26,30;$\gamma(t)$}
\lbl{20,13;{\Large $\Sigma$}}
\end{lpic}
\end{wrapfigure}

\begin{thm}{Advanced exercise}
Let $\Sigma$ be a smooth closed strictly convex surface 
in $\RR^3$ 
and $\gamma\:[0,\ell]\z\to \Sigma$ be a unit-speed minimizing geodesic.
Set $p\z=\gamma(0)$, $q=\gamma(\ell)$ and 
$$p_t=\gamma(t)-t\cdot\gamma'(t),$$ 
where $\gamma'(t)$ denotes the velocity vector of $\gamma$ at $t$.

Show that for any $t\in (0,\ell)$,
one {}\emph{cannot see}  $q$ from $p_t$;
that is, the line segment $[p_tq]$ intersects $\Sigma$ at a point distinct from $q$.%
\footnote{Hint: Show that the concatenation of the line segment $[p_t\gamma(t)]$ and the arc $\gamma|_{[t,\ell]}$ is a minimizing geodesic in the closed set $W$ outside of $\Sigma$.}
\end{thm}

\section{Geodesic curvature}

Suppose $\Sigma$ is a smooth regular surface.
Assume $\Sigma$ is oriented;
in this case terms ``left'' and ``right'' can be used the same sense as in the plane.

\parbf{Broken geodesics.}
For a concatenation of two geodesics in $\Sigma$, let us define signed external angle at their common point;
the sign is positive if it turns left and negative if it turns right. 

A concatenation of minimizing geodesics in $\Sigma$ will be called \emph{broken geodesic}.

The sum of the signed external angles for a broken geodesic $\gamma$ in $\Sigma$ will be called \emph{total geodesic curvature} of $\gamma$; it will be denoted as~$\tgc\gamma$ or $\tgc{\gamma,\Sigma}$ if we need to emphasize that $\gamma$ is a curve in $\Sigma$.

Note that if we change the orientation of the curve, then the total geodesic curvature changes sign.

\parbf{Smooth regular curves.}
The total geodesic curvature can be also defined for a smooth unit-speed curve $\gamma\:[a,b]\to\Sigma$.

Let $\nu\:\Sigma\to \SS^2$ be the Gauss map that defines the orientation on $\Sigma$.
Then for any $t$ the vectors $\nu(t)=\nu(\gamma(t))$ and the velocity vector $\tau(t)=\gamma'(t)$ are unit vectors that are normal to each other.
Denote by $\mu(t)$ the unit vector that is normal to both $\nu(t)$ and $\tau(t)$ that points to the left from $\gamma$.
Note that the triple $\tau(t),\mu(t),\nu(t)$ is an orthogonal basis for any $t$.

Since $\gamma$ is unit-speed, the acceleration $\gamma''(t)$ is perpendicular to $\tau(t)$;
therefore at any parameter value $t$, we have
\[\gamma''(t)=k_g(t)\cdot \mu(t)-k_n(t)\cdot \nu(t),\]
for some real numbers $k_n(t)$ and $k_g(t)$.
The numbers $k_n(t)$ and $k_g(t)$ are called \emph{normal} and \emph{geodesic curvature} of $\gamma$ at $t$ correspondingly.

Note that the geodesic curvature vanishes if $\gamma$ is a geodesic. 
It measures how much a given curve diverges from being a geodesic;
it is positive if $\gamma$ turns left and negative if $\gamma$ turns right.

The total geodesic curvature of $\gamma$ can be defined as the integral of its geodesic curvature
\[\tgc\gamma=\int_a^b k_g(t)\cdot dt.\]

If $\gamma$ is a regular curve then one has to parameterize it by arc length and then apply the definition above.

The given two definitions for regular curve and broken geodesic agree in the following sense.
If $\beta_n$ is a sequence of inscribed broken geodesics in $\gamma$ for finer and finer partitions, then 
\[\tgc{\beta_n}\to\tgc\gamma,\]
where the left and right hand sides are defined using the first and the second definitions above.
The proof is straightforward, it can be done the same way as the case in the plane. %??? is it???

\begin{thm}{Exercise}
Let $\gamma$ be a smooth unit-speed curve in a  smooth regular surface $\Sigma$ with a Gauss map $\nu$.
Show that 
\[k_n(t)=\langle\gamma'(t),\nu'(t)\rangle.\]

\end{thm}

\parbf{Piecewise smooth curves.}
One could also combine both definitions to define total geodesic curvature for 
piecewise smooth curve $\gamma$ in $\Sigma$; that is
a concatenation of smooth regular curves.
We need to add the total geodesic curvature of all the edges of $\gamma$ and the signed external angle at each vertex. 



\begin{thm}{Proposition}\label{prop:pt+tgc}
Assume $\gamma$ is a closed broken geodesic in a smooth oriented surface $\Sigma$ that starts and ends at the point $p$.
Then the parallel transport $\iota_\gamma\:T_p\to\T_p$ is a rotation of the the plane $\T_p$ clockwise by angle $\tgc\gamma$.

Moreover, the same statement holds for smooth closed curves and piecewise smooth curves.
\end{thm}

\begin{wrapfigure}{o}{22 mm}
\vskip-0mm
\centering
\includegraphics{mppics/pic-48}
\vskip-0mm
\end{wrapfigure}

\parit{Proof.}
Assume $\gamma$ is a cyclic concatenation of geodesics $\gamma_1,\dots,\gamma_n$.
Fix a tangent vector $v$ at $p$ and extend it to a parallel vector field along $\gamma$.
Since $w_i(t)=\gamma_i'(t)$ is parallel along $\gamma_i$, the angle $\phi_i$ between $v$ and $w_i$ stays constant on each $\gamma_i$.

If $\theta_i$ denotes the external angle at this vertex of switch from $\gamma_{i}$ to $\gamma_{i+1}$, we have that 
\[\phi_{i+1}=\phi_i-\theta_i \pmod{2\cdot\pi}.\]
Therefore after going around we get that 
\[\phi_{n+1}-\phi_1=-\theta_1-\dots-\theta_n=-\tgc\gamma.\]
Hence the the first statement follows.

For the smooth unit-speed curve $\gamma\:[a,b]\to\Sigma$, the proof is analogous.
If $\phi(t)$ denotes the angle between $v(t)$ and $w(t)=\gamma'(t)$, then 
\[\phi'(t)+k_g(t)\equiv0\]
Whence the angle of rotation 
\begin{align*}
\phi(b)-\phi(a)&=\int_a^b \phi'(t)\cdot dt=
\\
&=-\int_a^b k_g\cdot dt=
\\
&=-\tgc\gamma
\end{align*}

The case of piecewise regular smooth curve is a straightforward combination of the above two cases. 
\qeds
