\chapter{Supporting circlines}

\section{Definitions}

Suppose $\alpha\:[a,b]\to\RR^2$ be a smooth unit-speed plane curve and $t_0\z\in(a,b)$.

\begin{wrapfigure}{r}{43 mm}
\vskip-4mm
\centering
\includegraphics{mppics/pic-28}
\vskip0mm
\end{wrapfigure}

A circline $\gamma$ supports $\alpha$ at $t_0$ if $\alpha(t_0)\in\gamma$
and $\gamma$ lies locally on one side of $\alpha$.
If $p=\gamma(t_0)$ for a single value $t_0$,
then we can also say \emph{$\gamma$ supports $\alpha$ at $p$} without ambiguity.

More precisely assume that there is a spherical neighborhood $U\ni p$ such that for some interval $[a',b']\ni t_0$
the arc $\bar\alpha=\alpha|_{[a',b']}$ has no self-intersection and runs from boundary to boundary of $U$.
In this case $\alpha$ divides $U$ into two sets  $L$ and $R$;
$L$ lies on the left and $R$ lies on the right from~$\bar\alpha$.
If $\gamma\cap U$ contains only points of $\bar\alpha$ and $R$, we say that \emph{$\gamma$ supports $\alpha$ on the right};
if $\gamma\cap U$ contains only points of $\bar\alpha$ and $L$, we say that \emph{$\gamma$ supports $\alpha$ on the left}.



Note that a circle supports itself on the right and left at the same time.

Suppose a unit-speed circline $\gamma$ supports a smooth unit-speed plane curve $\alpha$ at $t_0$.
Without loss of generality we can assume that $\gamma(0)\z=\alpha(t_0)$. 
Then $\gamma'(0)=\pm\alpha'(t_0)$.
If not, then the curve $\alpha$ would cross $\gamma$ transversely and therefore could not stay at the same side for values close to $t_0$.
Therefore reverting the parametrization of $\gamma$ if necessary we may (and further will) assume that 
\[\gamma'(0)=\alpha'(t_0)\]
holds for any supporting circline $\gamma$ to $\alpha$ at $t_0$.

\section{Supporting test}

The following proposition resembles the second derivative test. 

\begin{thm}{Proposition}\label{prop:supporting-circline}
Assume $\gamma$ is a circle that that supports $\alpha$ at $t_0$ from rigth (correspondingly left).  
Then 
\[\kappa(t_0)\ge \kappa
\quad(\text{correspondingly}\quad\kappa(t_0)\le \kappa).
\] 
where $\kappa$ is the signed curvature of $\gamma$ 
and $\kappa(t_0)$ is the signed curvature of $\alpha$ at $t_0$.

A partial converse also holds.
Namely suppose a unit-speed circline $\gamma$ with signed curvature $\kappa$ starts at $\alpha(t_0)$ in the direction $\alpha'(t_0)$.
Then $\gamma$ supports $\alpha$ at $t_0$ from the right (correspondingly left) if 
\[\kappa(t_0)> \kappa
\quad(\text{correspondingly}\quad\kappa(t_0)< \kappa).
\]

\end{thm}

\parit{Proof.}
We prove only case $\kappa>0$.
The 2 remaining cases $\kappa=0$ and $\kappa<0$ can be done essentially same way.

Since $\kappa\ne0$, the curve $\gamma$ is a circle (it can not be a line).
According to Proposition~\ref{prop:circline},
$\gamma$ has radius $\tfrac1\kappa$ and it is centered at 
\[z=\alpha(t_0)+\tfrac i\kappa\cdot \alpha'(t_0).\]
Consider the function 
\[f(t)=|z-\alpha(t)|^2-\tfrac1{\kappa^2}.\]

Note that $f(t)\le0$ (correspondingly $f(t)\ge0$) 
if an only if $\alpha(t)$ lies on the closed left (correspondingly right) side from $\gamma$.
It follow that 
\begin{itemize}
\item if $\gamma$ supports $\alpha$ at $t_0$ from right, 
then
\[f'(t_0)=0\quad\text{and}\quad f''(t_0)\le 0;\]

\item if $\gamma$ supports $\alpha$ at $t_0$ from  left, 
then 
\[f'(t_0)=0\quad\text{and}\quad f''(t_0)\ge 0;\]

\item if 
\[f'(t_0)=0\quad\text{and}\quad f''(t_0)< 0,\]
then $\gamma$ supports $\alpha$ at $t_0$ from  right;

\item if 
\[f'(t_0)=0\quad\text{and}\quad f''(t_0)> 0,\] then $\gamma$ supports $\alpha$ at $t_0$ from  left;
\end{itemize}

Direct calculations show that
\begin{align*}
f(t_0)&=0;
\\
f'(t_0)&=\left.\langle z-\alpha(t),z-\alpha(t) \rangle'\right|_{t=t_0}=
\\
&=-2\cdot \langle \alpha'(t_0),z-\alpha(t_0) \rangle=
\\&=-2\cdot \langle \alpha'(t_0),\tfrac i\kappa \cdot\alpha'(t_0) \rangle=
\\
&=0;
\\
f''(t_0)&=\langle z-\alpha(t),z-\alpha(t) \rangle''|_{t=t_0}=
\\
&=2\cdot\left( \langle \alpha'(t_0),\alpha'(t) \rangle-\langle \alpha''(t_0),z-\alpha(t) \rangle \right)=
\\
&=2\cdot\left(1-\kappa\cdot \frac1{\kappa(t_0)}\right)
\end{align*}
Hence the result.\qeds


\begin{thm}{Exercise}
Assume $\alpha$ is a closed smooth unit-speed plane curve that runs in a unit disk.
Show that there is a point on $\alpha$ with curvature at least $1$.

Give two proofs, one based on DNA inequality \ref{thm:DNA} and the other based on Proposition~\ref{prop:supporting-circline}.
\end{thm}

\section{Lens lemma}

\begin{thm}{Lemma}\label{lem:lens}
Let $\alpha$ be a smooth regular simple curve that runs from $x$ to $y$.
Assume that $\alpha$ runs on right side (correspondingly left side) of the oriented line $xy$ and only its end points $x$ and $y$ lie on the line.
Then $\alpha$ has a point with positive  (correspondingly negative) curvature.
\end{thm}

\begin{wrapfigure}{r}{35 mm}
\vskip-4mm
\centering
\includegraphics{mppics/pic-22}
\vskip0mm
\end{wrapfigure}

Note that the lemma fails for curves with self-intersections;
the curve $\alpha$ on the diagram turns always right, 
so it has negative curvature everywhere and it lies on the right side of the line $xy$. 

\parit{Proof.}
Choose points $p$ and $q$ on $\ell$
so that the points $p, x, y, q$ appear in the same order on $\ell$.

Consider the smallest disc segment with chord $[pq]$ that contains $\alpha$.
Note that its arc $\gamma$ supports at a point $s=\alpha(t_0)$.

\begin{wrapfigure}{R}{50 mm}
\centering
\includegraphics{mppics/pic-23}
\bigskip
\includegraphics{mppics/pic-24}
\vskip0mm
\end{wrapfigure}

Note that the $\alpha'(t_0)$ is tangent to $\gamma$ at $s$.
Moreover $\alpha'(t_0)$ points in the direction of $q$;
that is, if we go along $\gamma$ in the direction  of $\alpha'(t_0)$ then we have to start at $p$ and end at $q$.
If the direction is opposite, then the arc of $\alpha$ from $s$ to $y$ would be trapped in the curvelinear triangle $xsp$ bounded by arcs of $\gamma$, $\alpha$ and the line segment $[px]$.
But this is impossible since $y$ does not belong to this triangle.



It follows that $\gamma$ supports $\alpha$ at $t_0$ from right.
By Proposition~\ref{prop:supporting-circline}, 
\[\kappa(t_0)\ge \kappa,\]
where $\kappa(t_0)$ is signed curvature of $\alpha$ at $t_0$ and $\kappa$ is the curvature of $\gamma$.
Evidently $\kappa>0$, hence the result.
\qeds


\begin{thm}{Exercise}
Assume that a bounded domain $D$ in the plane is bounded by a closed regular smooth simple plane curve $\alpha$.
Show that $D$ is convex if and only if the signed curvature of $\alpha$ does not change the sign.
\end{thm}

\begin{thm}{Exercise} Assume a smooth regular curve $\alpha$ has curvature at most $1$ at any point (that is, $|\kappa_\alpha(t)|\le 1$ for any $t$).
Show that each of two unit circles tangent to $\gamma$ at $t_0$ is supporting. 
\end{thm}

\begin{figure}[h!]
\vskip-0mm
\centering
\includegraphics{mppics/pic-29}
\vskip0mm
\end{figure}

\begin{thm}{Exercise}
Suppose $\alpha$ is a simple smooth regular curve in the plane with positive curvature.
Assume $\alpha$ crosses a line $\ell$ at the points $p_1,p_2,\dots p_n$ and these points appear on $\alpha$ in the same order.
\begin{enumerate}[(a)]
\item Show that $p_2$ can not lie between $p_1$ and $p_3$ on $\ell$.
\item Show that if $p_3$ lies between $p_1$ and $p_2$ on $\ell$ then these points uppear on $\ell$ in the following order:  
\[p_1,p_3,\dots,p_4 ,p_2.\]
\item Try to describe all possible orders in the case if $p_1$ lies between $p_2$ and $p_3$ (see the diagram).
\end{enumerate}
\end{thm}

\begin{thm}{Exercise}
Suppose $\alpha$ is a simple smooth regular curve in the plane.
Show that $\alpha$ lies on one side from one of its tangent lines. 
\end{thm}

If curvature of a curve $\alpha$ vanish at $t_0$, then we say that $t_0$ is inflection value of parameter, and $p=\alpha(t_0)$ is its inflection point;
the later convention might be ambiguous only if $\alpha$ has self-intersection at $p$. 
In other words, $t_0$ is inflection value if osculating circline at at $t_0$ coincides with the tangent line. 

\begin{wrapfigure}{r}{30 mm}
\vskip-4mm
\centering
\includegraphics{mppics/pic-30}
\vskip0mm
\end{wrapfigure}

\begin{thm}{Exercise}
Assume $\alpha\:[a,b]\to\RR^2$ is a smooth simple regular curve such that 
\[\alpha'(a)=\alpha'(b)=\alpha(b)-\alpha(a).\]
In particular the tangent line $\ell$ at $a$ is also tangent at $b$.
\begin{enumerate}[(a)]
\item
Show that it has at least 2 inflection points.
\item
If $\alpha$ crosses the line $\ell$ then $\alpha$ has at least 3 inflection points.
\end{enumerate}
\end{thm}


\section{Four-vertex theorem}



A vertex of a smooth regular curve is defined as a critical point of its curvature;
in particular, any point of local minimum (or maximum) is a vertex.

\begin{thm}{Exercise}
Assume an osculating circle of curve $\alpha$ at $t_0$ supports $\alpha$ at $t_0$.
Show that $t_0$ is a vertex of $\alpha$.
\end{thm}

\begin{thm}{Four-vertex theorem}
Any smooth regular simple plane curve has at least four
vertices.
\end{thm}

\begin{wrapfigure}[8]{r}{20 mm}
\vskip-4mm
\centering
\includegraphics{mppics/pic-26}
\vskip0mm
\end{wrapfigure}

Evidently any closed curve have to have at least two vertexes --- one minimum and one maximum.
On the diagram the vertexes are marked;
the first curve has one self-intersection and exactly two vertexes;
he second curve has exactly four vertexes and no self-intersections.

The four-vertex theorem was first proved by Syamadas Mukhopadhyaya \cite{mukhopadhyaya} for convex curves.
By now it has number of different proofs and generalizations.
We will present a proof given by Robert Osserman \cite{osserman}.

\parit{Proof.}
Fix a simple smooth regular closed plane curve $\alpha$.

Suppose that $2\cdot n$ points $p_1,s_1,\dots,p_n,s_n$ appear on a closed curve $\alpha$ in the same cyclic order.
Fix a real number $\kappa$.
Assume that the curvature of $\alpha$ at $p_i$ is at least $\kappa$ and its curvature at $s_i$ is at most $\kappa$.
Then each of $n$ arcs $p_{n}p_1$, $p_{1}p_2, \dots p_{n-1}p_n$ of
$\alpha$ has a minimum point of curvature in its interior.
Similarly each of $n$ arcs $s_{n}s_1$, $s_{1}s_2, \dots s_{n-1}s_n$ of
$\alpha$ has a maximum point of curvature.

In the case if some of these local minimum coincide with a local maximum,
an arc around this point has constant curvature;
in this case all these points are vertexes and we have infinite number of them.
If they all different, then we have least $2\cdot n$ vertexes.

Therefore it is sufficient to show that

\begin{clm}{}\label{clm-key}
there are at least 4 points $p_1,s_1,p_2,s_2$ with the described properties for some $\kappa$.
\end{clm}


Note that
\begin{clm}{}\label{clm:circumscribed circle}
$\alpha$
admits unique \emph{circumscribed circle} $\gamma$; that is, a circle of minimal radius that encloses $\alpha$.
\end{clm}

Denote by $r$ the least lower bound of circles that enclose $\alpha$.
We can choose a sequence of circles $\gamma_n$ enclosing $\alpha$ such that their radiuses $r_n\to r$.
Note that all the centers of $\gamma_i$ lie on bounded distance from $\alpha$.
Therefore passing to a subsequence we can assume that centers of $\gamma_n$ converge to a point $o$.
Note that the circle $\gamma$ with center $o$ and radius $r$ encloses $\alpha$;
hence the existence of circumscribed circle follows.

If there are two distinct circumscribed circle, then $\alpha$ lies in the intersection of the discs bounded by these circles.
But this intersection is enclosed in a circle of smaller radius --- a contradiction.
Hence Claim \ref{clm:circumscribed circle} follows.


\begin{clm}{}
Assume $\gamma$ is the circumscribed circle of $\alpha$.
Then $\gamma$ touches $\alpha$ at least $2$ points which divide the $\gamma$ in arcs no longer than semicircle.
\end{clm}

\begin{wrapfigure}{R}{38 mm}
\vskip-0mm
\centering
\includegraphics{mppics/pic-271}
\vskip0mm
\end{wrapfigure}


If it would not be the case then one could move $\gamma$ slightly keeping its radius the same so that $\gamma$ will not touch $\alpha$ at all.
But in this case $\alpha$ could be enclosed by a circle of smaller radius --- a contradiction.


Let us orient $\alpha$ and $\gamma$ counterclockwise.
Then at the common points the directions of $\alpha$ and $\gamma$ coincide.
Note that these points appear on $\alpha$ and $\gamma$ in the same order;
otherwise $\alpha$ could not be simple.

Denote by $\kappa$ the signed curvature of $\gamma$, since it is oriented counterclockwise,
$\kappa\z=\tfrac1r>0$.

\begin{wrapfigure}{R}{38 mm}
\vskip-0mm
\centering
\includegraphics{mppics/pic-27}
\vskip0mm
\end{wrapfigure}
  
Fix two common points $p_1$ and $p_2$ of $\alpha$ and $\gamma$.
By Proposition~\ref{prop:supporting-circline}, the curvature of $\alpha$ at $p_1$ and $p_2$ is at least $\kappa$.
Let $\bar\alpha$ be the arc of $\alpha$ from $p_1$ to $p_2$.

We can assume that the circle is centered at the origin 
and the points $p_1$ and $p_2$ lie on the same vertical line in the right halfplane of the coordinate plane.

Let $o$ be the center of $\gamma$.
Consider an other circle $\gamma'$ with the same radius $r$ and the center $o'$ moved to the left along $x$-axis to the last moment it intersects $\bar\alpha$.

Denote by $s_1$ a common point of $\bar\alpha$ and $\gamma'$;
we can assume that $s_1$ is not an end point of $\bar\alpha$.
At the point $s_1$ the directions of $\bar\alpha$ and $\gamma'$ coinside,
otherwise $\alpha$ could not be simple --- the same argument is used in the proof of Lemma~\ref{lem:lens}.
Therefore $\gamma'$ supports $\alpha$ from left at $s_1$.
By Proposition~\ref{prop:supporting-circline}, the curvature of $\alpha$ at $s_1$ is at most $\kappa$.

By repeating the same argument for an other pair of points $p_2,p_3$,\footnote{If $p_1$ and $p_2$ divide $\gamma$ into two semicircles, then we can take $p_3=p_1$.} 
we prove Claim \ref{clm-key}.
Whence the theorem follows.
\qeds

\begin{thm}{Exercise}
Show that any smooth regular curve of constant width has at least 6 vertexes.
\end{thm}

\section{Tennis ball theorem}

Suppose that a curve $\alpha$ runs on the unit sphere 
\[\SS^2=\set{(x,y,z)\in\RR^3}{x^2+y^2+z^2=1}.\] 
Let us denote by $\alpha''(t)^\top$ the projection of vector $\alpha''(t)$ to the tanget plane of the sphere at $\alpha(t)$.

Assume $\alpha$ is a unit speed curve.
By Proposition~\ref{prop:a'-pertp-a''}, $\alpha''(t)\perp\alpha'(t)$ and therefore $\alpha''(t)^\top\perp\alpha'(t)$.
If $w(t)$ denotes by the vector $\alpha'(t)$ rotated
counterclockwise by angle $\tfrac\pi2$ in the tangent plane to the sphere at $\alpha(t)$, then 
\[\alpha''(t)^\top=\kappa_\alpha(t)\cdot w(t)\]
for some real number $\kappa(t)$;
this number is called \emph{signed geodesic curvature of $\alpha$ at $t$}.

The geodesic curvature measures how $\alpha$ diverges from an equator --- the most straight curve on the curved $\SS^2$.
TBC
