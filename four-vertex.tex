\chapter{Four-vertex theorem}

A vertex of a smooth regular curve is defined as a critical point of its curvature;
in particular, any point of local minimum (or maximum) is a vertex.

\begin{thm}{Four-vertex theorem}
Any smooth regular simple plane curve has at least four
vertices.
\end{thm}

\begin{wrapfigure}{r}{20 mm}
\vskip-4mm
\centering
\includegraphics{mppics/pic-26}
\vskip0mm
\end{wrapfigure}

Evidently any closed curve have to have at least two vertexes --- one minimum and one maximum.
The curve on the diagram has one self-intersection and exactly two vertexes, they are marked on the diagram.
There are smooth regular simple plane curves with exactly 4 vertexes, 
ellipse for example.

The four-vertex theorem was first proved by Syamadas Mukhopadhyaya \cite{mukhopadhyaya} for convex curves.
By now it has number of different proofs and generalizations.

We will present a proof given by Robert Osserman \cite{osserman}.

\parit{Proof.}
Fix a smooth regular closed plane curve $\alpha$.

Suppose that $2\cdot n$ points $p_1,s_1,\dots,p_n,s_n$ appear on a closed curve $\alpha$ in the same cyclic order.
Fix a real number $\kappa$.
Assume that the curvature of $\alpha$ at $p_i$ is at least $\kappa$ and its curvature at $s_i$ is smaller than $\kappa$.
Then each of $n$ arcs $p_{n}p_1$, $p_{1}p_2, \dots p_{n-1}p_n$ of
$\alpha$ has a local minimum.
Similarly each of $n$ arcs $s_{n}s_1$, $s_{1}s_2, \dots s_{n-1}s_n$ of
$\alpha$ has a local maximum.
In particular it has at least $2\cdot n$ vertexes.
It is sufficient to show that

\begin{clm}{}\label{clm-key}
there are at least 4 points $p_1,s_1,p_2,s_2$ with the described condition.
\end{clm}


Note that
\begin{clm}{}
$\alpha$
admits unique \emph{circumscribed circle} $\gamma$; that is, $\gamma$ is a circle of minimal radius that encloses $\alpha$.
\end{clm}

Indeed, denote by $r$ the least lower bound of circles that enclose $\alpha$.
We can choose a sequence of circles $\gamma_n$ enclosing $\alpha$ such that their radiuses $r_n\to r$.
Note that all the centers of $\gamma_i$ lie on bounded distance from $\alpha$.
Therefore passing to a subsequence we can assume that centers of $\gamma_n$ converge to a point $o$.
Note that the circle $\gamma$ with center $o$ and radius $r$ encloses $\alpha$;
hence the existence of circumscribed circle follows.

If there are two distinct circumscribed circle, then $\alpha$ lies in the intersection of the discs bounded by these circles.
But this intersection is enclosed in a circle of smaller radius --- a contradiction.

\begin{clm}{}
Assume $\gamma$ is the circumscribed circle of $\alpha$.
Then $\gamma$ touches $\alpha$ at least $2$ points which divide the $\gamma$ in arcs no longer than semicircle.
\end{clm}

Indeed if it would not be the case then one could move $\gamma$ slightly keeping its radius the same so that $\gamma$ will not touch $\alpha$ at all.
But in this case $\alpha$ could be enclosed by a circle of smaller radius --- a contradiction.

Let us orient $\alpha$ and $\gamma$ counterclockwise.
Then at the common points the directions of $\alpha$ and $\gamma$ coincide.
Note that these points appear on $\alpha$ and $\gamma$ in the same order;
otherwise $\alpha$ could not be simple.

Denote by $\kappa$ the signed curvature of $\gamma$, since it is oriented counterclockwise,
$\kappa=\tfrac1r>0$.

Fix two common points $p_1$ and $q$ of $\alpha$ and $\gamma$.
By Proposition~\ref{prop:supporting-circline}, the curvature of $\alpha$ at $p_1$ and $q$ is at least $\kappa$.

Denote by $\bar \gamma$ the arc of $\gamma$ from $p_1$ to $q$.
Assume $\bar\gamma$ does not exceed semicircle.
Let $\bar\alpha$ be the corresponding arc of $\alpha$  from $p_1$ to $q$.

Consider the minimal lens $L_{p_1,q}$ with vertexes $p_1$ and $q$ that contain $\bar\alpha$;
that is a closed region bounded by two arcs from $p_1$ to $q$.
Note that one of the arcs has to be $\bar\gamma$;
denote the other arc by $\bar\gamma'$ and its curvature by $\kappa'$.

If the lens $L_{p_1,q}$ degenerates to $\bar \gamma$, 
then $\bar\alpha=\bar\gamma$ and all the points on $\alpha$ have curvature $\kappa$.
In particular each point of $\bar\alpha$ is a vertex and the theorem follows.

If $L_{p_1,q}$ is nondegenerate, then
\[\kappa'<\kappa.
\eqlbl{eq:k<k}\]
This inequality is evident if $\kappa'\le0$.
If $\kappa'>0$ then $\kappa'=\tfrac1 {r'}$ where $r'$ is the radius of $\gamma'$.
Since $\gamma$ does not exceed semicircle, we have that $r'>r$ and therefore \ref{eq:k<k} follows. %???PIC

Note that there is a point $s$ on $\bar\alpha$ common with $\bar\gamma'$;
otherwise $L_{p_1,q}$ would not be minimal.
Moreover at the point $s$ the directions of $\bar\alpha$ and $\bar\gamma'$ coinside,
otherwise $\alpha$ could not be simple --- the same argument is used in the proof of Lemma~\ref{lem:lens}.

By Proposition~\ref{prop:supporting-circline}, the curvature of $\alpha$ at $s$ is at most $\kappa'$.

After repeat the same argument for an other pair of points $p_2,p_3$ (possibly $p_3=p_1$), the claim \ref{clm-key} and therefore the theorem follows.
\qeds

\begin{thm}{Exercise}
Show that any smooth regular curve of constant width has at least 6 vertexes.
\end{thm}

