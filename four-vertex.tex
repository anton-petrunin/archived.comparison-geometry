\chapter{Four-vertex theorem}

A vertex of a smooth regular curve is defined as a critical point of its curvature;
in particular, any point of local minimum (or maximum) is a vertex.

\begin{thm}{Four-vertex theorem}
Any smooth regular simple plane curve has at least four
vertices.
\end{thm}

\begin{wrapfigure}[8]{r}{20 mm}
\vskip-4mm
\centering
\includegraphics{mppics/pic-26}
\vskip0mm
\end{wrapfigure}

Evidently any closed curve have to have at least two vertexes --- one minimum and one maximum.
On the diagram the vertexes are marked;
the first curve has one self-intersection and exactly two vertexes;
he second curve has exactly four vertexes and no self-intersections.

The four-vertex theorem was first proved by Syamadas Mukhopadhyaya \cite{mukhopadhyaya} for convex curves.
By now it has number of different proofs and generalizations.
We will present a proof given by Robert Osserman \cite{osserman}.

\parit{Proof.}
Fix a simple smooth regular closed plane curve $\alpha$.

Suppose that $2\cdot n$ points $p_1,s_1,\dots,p_n,s_n$ appear on a closed curve $\alpha$ in the same cyclic order.
Fix a real number $\kappa$.
Assume that the curvature of $\alpha$ at $p_i$ is at least $\kappa$ and its curvature at $s_i$ is at most $\kappa$.
Then each of $n$ arcs $p_{n}p_1$, $p_{1}p_2, \dots p_{n-1}p_n$ of
$\alpha$ has a minimum point of curvature in its interior.
Similarly each of $n$ arcs $s_{n}s_1$, $s_{1}s_2, \dots s_{n-1}s_n$ of
$\alpha$ has a maximum point of curvature.

In the case if some of these local minimum coincide with a local maximum,
an arc around this point has constant curvature;
in this case all these points are vertexes and we have infinite number of them.
If they all different, then we have least $2\cdot n$ vertexes.

Therefore it is sufficient to show that

\begin{clm}{}\label{clm-key}
there are at least 4 points $p_1,s_1,p_2,s_2$ with the described properties for some $\kappa$.
\end{clm}


Note that
\begin{clm}{}\label{clm:circumscribed circle}
$\alpha$
admits unique \emph{circumscribed circle} $\gamma$; that is, a circle of minimal radius that encloses $\alpha$.
\end{clm}

Denote by $r$ the least lower bound of circles that enclose $\alpha$.
We can choose a sequence of circles $\gamma_n$ enclosing $\alpha$ such that their radiuses $r_n\to r$.
Note that all the centers of $\gamma_i$ lie on bounded distance from $\alpha$.
Therefore passing to a subsequence we can assume that centers of $\gamma_n$ converge to a point $o$.
Note that the circle $\gamma$ with center $o$ and radius $r$ encloses $\alpha$;
hence the existence of circumscribed circle follows.

If there are two distinct circumscribed circle, then $\alpha$ lies in the intersection of the discs bounded by these circles.
But this intersection is enclosed in a circle of smaller radius --- a contradiction.
Hence Claim \ref{clm:circumscribed circle} follows.

\begin{clm}{}
Assume $\gamma$ is the circumscribed circle of $\alpha$.
Then $\gamma$ touches $\alpha$ at least $2$ points which divide the $\gamma$ in arcs no longer than semicircle.
\end{clm}

If it would not be the case then one could move $\gamma$ slightly keeping its radius the same so that $\gamma$ will not touch $\alpha$ at all.
But in this case $\alpha$ could be enclosed by a circle of smaller radius --- a contradiction.

\begin{wrapfigure}{r}{38 mm}
\vskip-4mm
\centering
\includegraphics{mppics/pic-27}
\vskip0mm
\end{wrapfigure}

Let us orient $\alpha$ and $\gamma$ counterclockwise.
Then at the common points the directions of $\alpha$ and $\gamma$ coincide.
Note that these points appear on $\alpha$ and $\gamma$ in the same order;
otherwise $\alpha$ could not be simple.

Denote by $\kappa$ the signed curvature of $\gamma$, since it is oriented counterclockwise,
$\kappa\z=\tfrac1r>0$.
  
Fix two common points $p_1$ and $p_2$ of $\alpha$ and $\gamma$.
By Proposition~\ref{prop:supporting-circline}, the curvature of $\alpha$ at $p_1$ and $p_2$ is at least $\kappa$.
Let $\bar\alpha$ be the arc of $\alpha$ from $p_1$ to $p_2$.

Let $o$ be the center of $\gamma$.
Consider an other circle $\gamma'$ with the same radius $r$ and the center $o'$ moved away from the arc $p_1p_2$ to the last moment it intersects $\bar\alpha$.
(More precisely, $o'$ lies on the line bisecting angle $p_1op_2$ on the side opposite from the arc such that $\gamma'$ intersects $\bar\alpha$ the distance $|o-o'|$ is maximal.)

Denote by $s_1$ a common point of $\bar\alpha$ and $\gamma'$;
we can assume that $s_1$ is not an end point of $\bar\alpha$.
At the point $s_1$ the directions of $\bar\alpha$ and $\gamma'$ coinside,
otherwise $\alpha$ could not be simple --- the same argument is used in the proof of Lemma~\ref{lem:lens}.
Therefore $\gamma'$ supports $\alpha$ from left at $s_1$.
By Proposition~\ref{prop:supporting-circline}, the curvature of $\alpha$ at $s_1$ is at most $\kappa$.

By repeating the same argument for an other pair of points $p_2,p_3$,\footnote{If $p_1$ and $p_2$ divide $\gamma$ into two semicircles, then we can take $p_3=p_1$.} 
we prove Claim \ref{clm-key}.
Whence the theorem follows.
\qeds

\begin{thm}{Exercise}
Show that any smooth regular curve of constant width has at least 6 vertexes.
\end{thm}

