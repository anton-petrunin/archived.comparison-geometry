\chapter{Torsion}

\section*{Frenet frame}
Let $\gamma$ be a smooth regular space curve.
Without loss of generality, we may assume that $\gamma$ has arc length parametrization,
so the velocity vector $\tau(s)=\gamma'(s)$ is unit.

Assume its curvature does not vanish at some time moment $s$;
that is $\gamma''(s)\ne 0$.
Then we can define the so called \emph{normal vector} at $s$ as
\[\nu(s)=\frac{\gamma''(s)}{|\gamma''(s)|}.\]
Note that 
\[\tau'(s)=\gamma''(s)=\kur(s)\cdot\nu(s).\]

According to \ref{prop:a'-pertp-a''}, $\nu(s)\perp \tau(s)$.
Therefore the vector product 
\[\beta(s)=\tau(s)\times \nu(s)\]
is a unit vector which makes the triple $\tau(s),\nu(s),\beta(s)$ an oriented orthonormal basis in $\RR^3$;
in particular, we have that
\[\begin{aligned}
\langle\tau,\tau\rangle&=1,
&
\langle\nu,\nu\rangle&=1,
&
\langle\beta,\beta\rangle&=1,
\\
\langle\tau,\nu\rangle&=0,
&
\langle\nu,\beta\rangle&=0,
&
\langle\beta,\tau\rangle&=0.
\end{aligned}
\eqlbl{eq:orthornomal}
\]

The vector $\beta(s)$ is called \emph{binormal} vector at $s$.
The orthonormal basis $\tau(s),\nu(s),\beta(s)$ is called \emph{Frenet frame} at $s$; the vectors in the frame are called \emph{tangent}, \emph{normal} and \emph{binormal} correspondingly.
Recall that Frenet frame $\tau(s),\nu(s),\beta(s)$ is defined if $k(s)\ne 0$.

The line thru $\gamma(s)$ in the direction of $\tau(s)$ is called \emph{thangent line} at $s$.

The plane thru $\gamma(s)$ spanned by vectors $\tau(s)$ and $\nu(s)$ is called \emph{osculating plane} at $s$;
equivalently it can be defined as a plane thru $\gamma(s)$ that is perpendicular to the binormal vector $\beta(s)$.
This plane has \emph{second order of contact} with $\gamma$ at $s$;
that is, $r(\ell)=o(\ell^2)$, where $r(\ell)$ the distance from $\gamma(s+\ell)$ to the osculating plane at $s$.

\section*{Torsoion}

Let $\gamma$ be a smooth unit-speed space curve
and $\tau(s),\nu(s),\beta(s)$ is its Frenet frame.
The value 
\[\tor(s)=\langle \nu'(s),\beta(s)\rangle\]
is called \emph{torsion} of $\gamma$ at $s$.

Note that the torsion $\tor(s)$ is defined at each $s$ with nonzero curvature.
Indeed, if $k(s)\ne 0$ then Frenet frame $\tau(s),\nu(s),\beta(s)$ is defined at $s$.
Moreover since $k(s)$ is continuous, it is positive in an open ineteval containing $s$ and Frenet frame is also defined in this interval.
Clearly $\tau(s)$, $\nu(s)$ and $\beta(s)$ depend smoothly on $s$ in their domains of definition.
Therefore $\nu'(s)$ is defined and so is the torsion $\tor(s)$.

The torsion measures how fast the osculating plane rotated when one travels along $\gamma$.

\begin{thm}{Exercise}\label{ex:helix-torsion}
Given real numbers $a$ and $b$, calculate curvature and torsion of the helix
\[\gamma_{a,b}(t)=(a\cdot \cos t,a\cdot\sin t, b\cdot t).\]

Conclude that for any $\kur>0$ and $\tor$ there is a helix with constant curvature $\kur$ and torsion $\tor$.
\end{thm}


\section*{Frenet formulas}

Assume the Frenet frame $\tau(s),\nu(s),\beta(s)$ of curve $\gamma$ is defined at $s$.
Recall that 
\[\tau'(s)=\kur(s)\cdot \nu(s).
\eqlbl{eq:frenet-tau}\]
Let us  express the remaining derivatives $\nu'(s)$ and $\beta'(s)$ using the frame $\tau(s),\nu(s),\beta(s)$.

First let us show that
\[\nu'(s)=-\kur(s)\cdot\tau(s)+\tor(s)\cdot\beta(s).\eqlbl{eq:frenet-nu}\]

Since the frame $\tau(s),\nu(s),\beta(s)$ is orthonormal it is equivalent to the following three identities:
\begin{align*}
\langle \nu',\tau\rangle&=-\kur,
&
\langle \nu',\nu\rangle&=0,
&
\langle \nu',\beta\rangle&=\tor,
\end{align*}
The last identity follows from the definition of torsion.
Differentiating $\langle \nu,\nu\rangle\z=1$ in \ref{eq:orthornomal}, we get that
\[2\cdot\langle \nu',\nu\rangle=0;\]
whence the second identity follows.
Let us differentiate the identity $\langle\tau,\nu\rangle=0$ in \ref{eq:orthornomal};
we get that
\[\langle\tau',\nu\rangle+\langle\tau,\nu'\rangle=0.\]
Applying \ref{eq:frenet-tau}, we get that
\begin{align*}
\langle \nu',\tau\rangle&=-\langle\tau',\nu\rangle=
\\
&=-\langle\kur\cdot\nu,\nu\rangle=
\\
&=-\kur.
\end{align*}
It proves the first equality $\langle \nu',\tau\rangle=-\kur$, 
whence \ref{eq:frenet-nu} follows.

Taking derivatives of the third identity in \ref{eq:orthornomal}, we get that $\beta'\perp\beta$.
Further taking derivatives of the other identities with $\beta$ in \ref{eq:orthornomal}, we get that 
\begin{align*}
\langle\beta',\tau\rangle&=-\langle\beta,\tau'\rangle=-\langle\beta,\kur\cdot\nu\rangle=0
\\
\langle\beta',\nu\rangle&=-\langle\beta,\nu'\rangle=\tor
\end{align*}
That is,
\[\beta'(s)=-\tor(s)\cdot\nu(s).\eqlbl{eq:frenet-beta}\]

The equations \ref{eq:frenet-tau}, \ref{eq:frenet-nu} and \ref{eq:frenet-beta} are called Frenet formulas.
They could be written in a matrix formulas
\[
\begin{pmatrix}
\tau'
\\
\nu'
\\
\beta'
\end{pmatrix}
=
\begin{pmatrix}
0&\kur&0
\\
-\kur&0&\tor
\\
0&-\tor&0
\end{pmatrix}
\cdot
\begin{pmatrix}
\tau
\\
\nu
\\
\beta
\end{pmatrix}.
\]

\begin{thm}{Exercise}
Deduce the formula \ref{eq:frenet-beta} from  \ref{eq:frenet-tau} and \ref{eq:frenet-nu} by differentiating the identity
$\beta=\tau\times \nu$.
\end{thm}

\begin{thm}{Exercise} 
Let $\gamma$ be a regular space curve with nonvanishing curvature.
Show that $\gamma$ lies in a plane if and only if its torsion vanishes.
\end{thm}

\parit{Hint:} 
Show and use that the binormal vector is constant.


