\chapter{Torsion}

This section provides a useful practice in computations.

Except for the definition in Section \ref{sec:frenet-frame},
the result in this chapter will not be used further in the sequel.


\section{Frenet frame}\label{sec:frenet-frame}

Let $\gamma$ be a smooth regular space curve.
Without loss of generality, we may assume that $\gamma$ has an arc-length parametrization,
so the velocity vector $\tan(s)=\gamma'(s)$ is unit.

Assume its curvature does not vanish at some time $s$;
in other words, $\gamma''(s)\ne 0$.
Then we can define the so-called \index{normal vector}\emph{normal vector} at $s$ as
\[\norm(s)=\frac{\gamma''(s)}{|\gamma''(s)|}.\]
Note that 
\[\tan'(s)=\gamma''(s)=\kur(s)\cdot\norm(s).\]\index{$\norm$}\index{$\tan$}\index{$\bi$}

According to \ref{prop:a'-pertp-a''}, $\norm(s)\perp \tan(s)$.
Therefore the vector product 
\[\bi(s)=\tan(s)\times \norm(s)\]
is a unit vector which makes the triple $\tan(s),\norm(s),\bi(s)$ an oriented orthonormal basis in $\RR^3$;
in particular, we have that
\[\begin{aligned}
\langle\tan,\tan\rangle&=1,
&
\langle\norm,\norm\rangle&=1,
&
\langle\bi,\bi\rangle&=1,
\\
\langle\tan,\norm\rangle&=0,
&
\langle\norm,\bi\rangle&=0,
&
\langle\bi,\tan\rangle&=0.
\end{aligned}
\eqlbl{eq:orthornomal}
\]

The orthonormal basis $\tan(s),\norm(s),\bi(s)$ is called \index{Frenet frame}\emph{Frenet frame} at $s$; the vectors in the frame are called \index{tangent}\emph{tangent}, \index{normal}\emph{normal} and \index{binormal}\emph{binormal} respectively.
Note that the frame $\tan(s),\norm(s),\bi(s)$ is defined only if $k(s)\ne 0$.

The plane $\Pi_s$ thru $\gamma(s)$ spanned by vectors $\tan(s)$ and $\norm(s)$ is called \index{osculating plane}\emph{osculating plane} at $s$;
equivalently it can be defined as a plane thru $\gamma(s)$ that is perpendicular to the binormal vector $\bi(s)$.
This is the unique plane that has \index{order of contact}\emph{second order of contact} with $\gamma$ at $s$;
that is, $\rho(\ell)=o(\ell^2)$, where $\rho(\ell)$ denotes the distance from $\gamma(s+\ell)$ to $\Pi_s$.

\section{Torsion}

Let $\gamma$ be a smooth unit-speed space curve
and $\tan(s),\norm(s),\bi(s)$ is its Frenet frame.
The value 
\[\tor(s)=\langle \norm'(s),\bi(s)\rangle\]\index{$\tor$}
is called the \index{torsion}\emph{torsion} of $\gamma$ at $s$.

Note that the torsion $\tor(s)$ is defined if $\kur(s)\z\ne0$.
Indeed, if $\kur(s)\z\ne 0$, then Frenet frame $\tan(s),\norm(s),\bi(s)$ is defined at $s$.
Moreover since the function $s\mapsto \kur(s)$ is continuous, it must be positive in an open interval containing $s$;
therefore the Frenet frame is also defined in this interval.
Clearly $\tan(s)$, $\norm(s)$ and $\bi(s)$ depend smoothly on $s$ in their domains of definition.
Therefore $\norm'(s)$ is defined and so is the torsion $\tor(s)=\langle \norm'(s),\bi(s)\rangle$.

The torsion measures how fast the osculating plane rotates when one travels along $\gamma$.

\begin{thm}{Exercise}\label{ex:helix-torsion}
Given real numbers $a$ and $b$, calculate curvature and torsion of the helix
\[\gamma_{a,b}(t)=(a\cdot \cos t,a\cdot\sin t, b\cdot t).\]

Conclude that for any $\kur>0$ and $\tor$ there is a helix with constant curvature $\kur$ and torsion $\tor$.
\end{thm}


\section{Frenet formulas}

Assume the Frenet frame $\tan(s),\norm(s),\bi(s)$ of a curve $\gamma$ is defined at $s$.
Recall that 
\[\tan'(s)=\kur(s)\cdot \norm(s).
\eqlbl{eq:frenet-tau}\]
It is convenient to write the remaining derivatives $\norm'(s)$ and $\bi'(s)$ in the frame $\tan(s),\norm(s),\bi(s)$.

First let us show that
\[\norm'(s)=-\kur(s)\cdot\tan(s)+\tor(s)\cdot\bi(s).\eqlbl{eq:frenet-nu}\]

Since the frame $\tan(s),\norm(s),\bi(s)$ is orthonormal, the above formula is equivalent to the following three identities:
\[\begin{aligned}
\langle \norm',\tan\rangle&=-\kur,
&
\langle \norm',\norm\rangle&=0,
&
\langle \norm',\bi\rangle&=\tor,
\end{aligned}\eqlbl{eq:<N',?>}\]
The last identity follows from the definition of torsion.
The second one comes from differentiating $\langle \norm,\norm\rangle\z=1$ in \ref{eq:orthornomal}. 
Differentiating the identity $\langle\tan,\norm\rangle\z=0$ in \ref{eq:orthornomal};
we get 
\[\langle\tan',\norm\rangle+\langle\tan,\norm'\rangle=0.\]
Applying \ref{eq:frenet-tau}, we get the first one.

Differentiating the third identity in \ref{eq:orthornomal}, we get that $\bi'\perp\bi$.
Further taking derivatives of the other identities with $\bi$ in \ref{eq:orthornomal}, we get that 
\begin{align*}
\langle\bi',\tan\rangle&=-\langle\bi,\tan'\rangle=-\kur\cdot\langle\bi,\norm\rangle=0
\\
\langle\bi',\norm\rangle&=-\langle\bi,\norm'\rangle=\tor
\end{align*}
Since the frame $\tan(s),\norm(s),\bi(s)$ is orthonormal, it follows that
\[\bi'(s)=-\tor(s)\cdot\norm(s).\eqlbl{eq:frenet-beta}\]

The equations \ref{eq:frenet-tau}, \ref{eq:frenet-nu} and \ref{eq:frenet-beta} are called \index{Frenet formulas}\emph{Frenet formulas}.
All three can be written as one matrix identity:
\[
\begin{pmatrix}
\tan'
\\
\norm'
\\
\bi'
\end{pmatrix}
=
\begin{pmatrix}
0&\kur&0
\\
-\kur&0&\tor
\\
0&-\tor&0
\end{pmatrix}
\cdot
\begin{pmatrix}
\tan
\\
\norm
\\
\bi
\end{pmatrix}.
\]

\begin{thm}{Exercise}\label{ex:beta-from-tau+nu}
Deduce the formula \ref{eq:frenet-beta} from  \ref{eq:frenet-tau} and \ref{eq:frenet-nu} by differentiating the identity
$\bi=\tan\times \norm$.
\end{thm}

\begin{thm}{Exercise}\label{ex:torsion=0}
Let $\gamma$ be a regular space curve with nonvanishing curvature.
Show that $\gamma$ lies in a plane if and only if its torsion vanishes.
\end{thm}


\begin{thm}{Exercise} \label{ex:frenet}
Let $\gamma$ be a smooth regular space curve, $\kur$ and $\tor$ its curvature and torsion,
and $\tan,\norm,\bi$ its Frenet frame.
Show that 
\[\bi=\frac{\gamma'\times\gamma''}{|\gamma'\times\gamma''|}.\]
Use this formula to show that
\[\tor=\frac{\langle\gamma'\times\gamma'',\gamma'''\rangle}{|\gamma'\times\gamma''|^2}.\]

\end{thm}

\section{Curves of constant slope}

We say that a smooth regular space curve $\gamma$ has \index{constant slope}\emph{constant slope} if its velocity vector makes a constant angle with a fixed direction.
The following theorem was proved by Michel Ange Lancret~\cite{lancret} more than two centuries ago.

\begin{thm}{Theorem}\label{thm:const-slope}
Let $\gamma$ be a smooth regular curve;
denote by $\kur$ and $\tor$ its curvature and torsion.
Suppose $\kur(s)>0$ for all $s$.
Then $\gamma$ has constant slope if and only if the ratio $\tfrac\tor\kur$ is constant.
\end{thm}

The theorem can be proved using the Frenet formulas.
The following exercise will guide you thru the proof of the theorem. 

\begin{thm}{Exercise} \label{ex:lancret}
Let $\gamma$ be a smooth regular space curve with nonvanishing curvature, $\tan,\norm,\bi$ 
its Frenet frame and $\kur$, $\tor$ its curvature and torsion.

\begin{subthm}{ex:lancret:a}
Assume that  $\langle \vec w,\tan\rangle$ is constant for a fixed nonzero vector $\vec w$.
Show that 
\[\langle \vec w, \norm\rangle =0.\]
Use it to show that 
\[\langle \vec w, -\kur\cdot\tan+\tor\cdot \bi\rangle =0.\]
Use these two identities to show that $\tfrac\tor\kur$ is constant;
it proves the ``only if'' part of the theorem.
\end{subthm}

\begin{subthm}{ex:lancret:b} Assume $\tfrac\tor\kur$ is constant, show that the vector $\vec w=\tfrac\tor\kur\cdot \tan+\bi$ is constant.
Conclude that $\gamma$ has constant slope; it proves the ``if'' part of the theorem.
\end{subthm}

\end{thm}

Let $\gamma$ be a smooth unit-speed curve and $s_0$ a fixed real number. 
Then the curve 
\[\alpha(s)=\gamma(s)+(s_0-s)\cdot \gamma'(s)\]
is called the \index{evolvent}\emph{evolvent} of $\gamma$.
Note that if $\ell(s)$ denotes the tangent line to $\gamma$ at $s$,
then $\alpha(s)\in \ell(s)$ and $\alpha'(s)\perp \ell$ for all $s$.

\begin{thm}{Exercise}\label{ex:evolvent-constant-slope}
Show that the evolvent of a constant slope curve is a plane curve.
\end{thm}

\section{Spherical curves}

\begin{thm}{Theorem}
A smooth regular space curve $\gamma$ lies in a unit sphere if and only if 
the following identity 
\[\left|\frac{\kur'}{\tor}\right|=\kur\cdot\sqrt{\kur^2-1}.\]
holds for its curvature $\kur$ and torsion $\tor$.
\end{thm}

Note that the identity implicitely implies that the torsion $\tor$ of the curve is nonzero;
otherwise the left hand side would be undefined while right hand side is defined.
The proof is another application of the Frenet formulas;
we present it in form of a guided exercise:

\begin{thm}{Exercise}\label{ex:spherical-frenet}
Suppose $\gamma$ is a smooth unit-speed space curve.
Denote by $\tan,\norm,\bi$ its Frenet frame and by $\kur$, $\tor$ its curvature and torsion.

\smallskip

Assume that $\gamma$ is spherical; that is, $|\gamma(s)|=1$ for any $s$.
Show that

\begin{enumerate}[(a)]
\item\label{ex:spherical-frenet:tau} $\langle\tan,\gamma\rangle=0$; conclude that $\langle\norm,\gamma\rangle^2+\langle\bi,\gamma\rangle^2=1$.
\item\label{ex:spherical-frenet:nu} $\langle\norm,\gamma\rangle=-\tfrac1\kur$;
\item\label{ex:spherical-frenet:beta} $\langle\bi,\gamma\rangle'=\tfrac\tor\kur$.
\item\label{ex:spherical-frenet:beta+}
Use (\ref{ex:spherical-frenet:beta}) to show that if $\gamma$ is closed, then $\tor(s)=0$ for some~$s$.
\item\label{ex:spherical-frenet:kur-tor} Use (\ref{ex:spherical-frenet:tau})--(\ref{ex:spherical-frenet:beta}) to show that 
\[\left|\frac{\kur'}{\tor}\right|=\kur\cdot\sqrt{\kur^2-1}.\]
\setcounter{lastnumber}{\value{enumi}}
\end{enumerate}
It proves the ``only if'' part of the theorem.

\smallskip

Now assume that $\gamma$ is a space curve that satisfies the idenity in (\ref{ex:spherical-frenet:kur-tor}).
\begin{enumerate}[(a)]
\setcounter{enumi}{\value{lastnumber}}
\item Show that $p=\gamma+\tfrac1\kur\cdot \norm+\tfrac{\kur'}{\kur^2\cdot\tor}\cdot\bi$ is constant; conclude that $\gamma$ lies in the unit sphere centered at $p$.
\end{enumerate}
It proves the ``if'' part of the theorem.
\end{thm}

For a unit-speed curve $\gamma$ with nonzero curvature and torsion at $s$,
the sphere $\Sigma_s$ with center
\[p(s)=\gamma(s)+\tfrac1{\kur(s)}\cdot \norm(s)+\tfrac{\kur'(s)}{\kur^2(s)\cdot\tor(s)}\cdot\bi(s)\]
and passing thru $\gamma(s)$ is called the \index{osculating sphere}\emph{osculating sphere} of $\gamma$ at $s$.
This is the unique sphere that has \index{order of contact}\emph{third order of contact} with $\gamma$ at $s$;
that is, $\rho(\ell)=o(\ell^3)$, where $\rho(\ell)$ denotes the distance from $\gamma(s+\ell)$ to $\Sigma_s$.
 
\section{Fundamental theorem of space curves}

\begin{thm}{Theorem}\label{thm:fund-curves}
Let $\kur(s)$ and $\tor(s)$ be two smooth real valued functions defined on a real interval $\mathbb{I}$.
Suppose $\kur(s)>0$ for all $s$.
Then there is a smooth unit-speed curve $\gamma\:\mathbb{I}\to\RR^3$ with curvature $\kur(s)$ and torsion $\tor(s)$ for every $s$.
Moreover $\gamma$ is uniquely defined up to a rigid motion of the space.
\end{thm}

The proof is an application of the theorem on existence and uniqueness of a solution of ordinary differential equation (\ref{thm:ODE}).

\parit{Proof.}
Fix a parameter value $s_0$, a point $\gamma(s_0)$ and an oriented orthonormal frame $\tan(s_0)$, $\norm(s_0)$, $\bi(s_0)$.

Consider the following system of differential equations
\[
\begin{cases}
\gamma'=\tan,
\\
\tan'=\kur\cdot\norm,
\\
\norm'=-\kur\cdot\tan+\tor\cdot\bi,
\\
\bi'=-\tor\cdot\norm.
\end{cases}
\]
with the initial condition $\gamma(s_0)$ and an oriented orthonormal frame $\tan(s_0)$, $\norm(s_0)$, $\bi(s_0)$.
(The system of equations has four vector equations, so it can be rewritten as a system of 12 scalar equations.)

By \ref{thm:ODE}, this system has a unique solution which is defined in a maximal subinterval $\mathbb{J}\subset \mathbb{I}$ containing $s_0$; we need to show that actually $\mathbb{J}= \mathbb{I}$.

Note that 
\begin{align*}
\langle\tan,\tan\rangle'
&=
2\cdot\langle\tan,\tan'\rangle
=
2\cdot\kur\cdot \langle\tan,\norm\rangle
=
0,
\\
\langle\norm,\norm\rangle'
&=
2\cdot\langle\norm,\norm'\rangle
=
-
2\cdot\kur\cdot\langle\norm,\tan\rangle
+
2\cdot\tor\cdot\langle\norm,\bi\rangle
=
0,
\\
\langle\bi,\bi\rangle'
&=
2\cdot\langle\bi,\bi'\rangle
=
-2\cdot\tor\langle\bi,\norm\rangle
=
0,
\\
\langle\tan,\norm\rangle'
&=
\langle\tan',\norm\rangle
+
\langle\tan,\norm'\rangle
=
\kur\cdot\langle\norm,\norm\rangle
-
\kur\cdot\langle\tan,\tan\rangle
+
\tor\cdot\langle\tan,\bi\rangle
=
0,
\\
\langle\norm,\bi\rangle'
&=
\langle\norm',\bi\rangle+\langle\norm,\bi'\rangle
=0,
\\
\langle\bi,\tan\rangle'
&=
\langle\bi',\tan\rangle+\langle\bi,\tan'\rangle
=
-\tor\cdot \langle\norm,\tan\rangle
+\kur\cdot\langle\bi,\norm\rangle
=0.
\end{align*}
That is, the values 
$\langle\tan,\tan\rangle$,
$\langle\norm,\norm\rangle$,
$\langle\bi,\bi\rangle$,
$\langle\tan,\norm\rangle$,
$\langle\tan,\norm\rangle$,
$\langle\bi,\tan\rangle$
are constant functions of $s$.
Since we choose $\tan(s_0)$, $\norm(s_0)$, $\bi(s_0)$ to be an oriented orthonormal frame, we have that the triple $\tan(s)$, $\norm(s)$, $\bi(s)$ is an oriented orthonormal for any $s$. In particular, $|\gamma'(s)|=1$ for all $s$.

Assume $\mathbb{J} \varsubsetneq \mathbb{I}$. Then an end of $\mathbb{J}$, say $a$, lies in the interior of $\mathbb{I}$.
By Theorem~\ref{thm:ODE}, at least one of the values $\gamma(s)$, $\tan(s)$, $\norm(s)$, $\bi(s)$
escapes to infinity as $s\to a$.
But this is impossible since the vectors $\tan(s)$, $\norm(s)$, $\bi(s)$ remain unit and $|\gamma'(s)|=|\tan(s)|=1$ --- a contradiction.
Hence $\mathbb{J}= \mathbb{I}$.

It remains to prove the last statement.

Assume there are two curves $\gamma_1$ and $\gamma_2$ with the given curvature and torsion functions.
Applying a motion of the space we can assume that the $\gamma_1(s_0)=\gamma_2(s_0)$ and the Frenet frames of the curves coincide at $s_0$.
Then $\gamma_1=\gamma_2$ by uniqueness of  solutions of the system (\ref{thm:ODE}).
\qeds

\begin{thm}{Exercise}\label{ex:cur+tor=helix}
Assume a curve $\gamma\:\RR\to\RR^3$ has constant curvature and torsion.
Show that $\gamma$ is a helix, possibly degenerate to a circle;
that is, in a suitable coordinate system we have
\[\gamma(t)=(a\cdot \cos t,a\cdot\sin t, b\cdot t)\]
for some constants $a$ and $b$.
\end{thm}


\begin{thm}{Advanced exercise}\label{ex:const-dist}
Let $\gamma$ be a smooth regular space curve such that the distance $|\gamma(t)-\gamma(t+\ell)|$ depends only on $\ell$.
Show that $\gamma$ is a helix, possibly degenerate to a line or a circle.
\end{thm}



\parbf{\ref{ex:helix-torsion}.} 
The arc-length parameter $s$ is already found in   \ref{ex:arc-length-helix}.
It remains to find Frenet frame and calculate curvature and torsion.
The latter can be done by straightforward calculations.

\begin{wrapfigure}{r}{20 mm}
\vskip0mm
\centering
\begin{lpic}[t(-0mm),b(0mm),r(0mm),l(0mm)]{asy/helix(1)}
\lbl[br]{8,24;$\norm$}
\lbl[b]{2,26;$\bi$}
\lbl[wl]{15,25;$\tan$}
\end{lpic}
\end{wrapfigure}

If the calculations done right, then you should see that curvature $\kur$ and torsion $\tau$ do not depend on time and given.
Moreover for any $\kur>0$ and $\tor$ one can find $a$ and $b$ so that the helix $\gamma_{a,b}$ has curvature $\kur$ and torsion~$\tor$.

One may also see it geometrically using that the helix is maped to itself by one-parameter family of glide rotations around $z$-axis.
Therefore, for the $t$-parametrization, Frenet frame rotates around $z$-axis with the angular velocity~$1$.
It remains rewrite it for the arc-length parametrization and note that 
\begin{align*}
\tan(0)&=(0,\cos\theta, \sin\theta),
\\
\norm(0)&=(-1,0,0),
\\
\bi(0)&=(0,\sin\theta, -\cos\theta),
\end{align*}
where $\tg\theta\z=b/a$ if $a>0$. 
