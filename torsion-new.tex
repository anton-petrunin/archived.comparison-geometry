\chapter{Torsion}

\section*{Frenet frame}
Let $\gamma$ be a smooth regular space curve.
Without loss of generality, we may assume that $\gamma$ has arc length parametrization,
so the velocity vector $\tau(s)=\gamma'(s)$ is unit.

Assume its curvature does not vanish at some time moment $s$;
in other words, $\gamma''(s)\ne 0$.
Then we can define the so-called \emph{normal vector} at $s$ as
\[\nu(s)=\frac{\gamma''(s)}{|\gamma''(s)|}.\]
Note that 
\[\tau'(s)=\gamma''(s)=\kur(s)\cdot\nu(s).\]

According to \ref{prop:a'-pertp-a''}, $\nu(s)\perp \tau(s)$.
Therefore the vector product 
\[\beta(s)=\tau(s)\times \nu(s)\]
is a unit vector which makes the triple $\tau(s),\nu(s),\beta(s)$ an oriented orthonormal basis in $\RR^3$;
in particular, we have that
\[\begin{aligned}
\langle\tau,\tau\rangle&=1,
&
\langle\nu,\nu\rangle&=1,
&
\langle\beta,\beta\rangle&=1,
\\
\langle\tau,\nu\rangle&=0,
&
\langle\nu,\beta\rangle&=0,
&
\langle\beta,\tau\rangle&=0.
\end{aligned}
\eqlbl{eq:orthornomal}
\]

The orthonormal basis $\tau(s),\nu(s),\beta(s)$ is called \emph{Frenet frame} at $s$; the vectors in the frame are called \emph{tangent}, \emph{normal} and \emph{binormal} correspondingly.
Note that the frame $\tau(s),\nu(s),\beta(s)$ is defined only if $k(s)\ne 0$.

The plane $\Pi_s$ thru $\gamma(s)$ spanned by vectors $\tau(s)$ and $\nu(s)$ is called \emph{osculating plane} at $s$;
equivalently it can be defined as a plane thru $\gamma(s)$ that is perpendicular to the binormal vector $\beta(s)$.
This a unique plane that has \emph{second order of contact} with $\gamma$ at $s$;
that is, $\rho(\ell)=o(\ell^2)$, where $\rho(\ell)$ denotes the distance from $\gamma(s+\ell)$ to $\Pi_s$.

\section*{Torsion}

Let $\gamma$ be a smooth unit-speed space curve
and $\tau(s),\nu(s),\beta(s)$ is its Frenet frame.
The value 
\[\tor(s)=\langle \nu'(s),\beta(s)\rangle\]
is called \emph{torsion} of $\gamma$ at $s$.

Note that the torsion $\tor(s)$ is defined at each $s$ with nonzero curvature.
Indeed, if $k(s)\ne 0$ then Frenet frame $\tau(s),\nu(s),\beta(s)$ is defined at $s$.
Moreover since the function $s\mapsto k(s)$ is continuous, it must be positive in an open interval containing $s$;
therefore Frenet frame is also defined in this interval.
Clearly $\tau(s)$, $\nu(s)$ and $\beta(s)$ depend smoothly on $s$ in their domains of definition.
Therefore $\nu'(s)$ is defined and so is the torsion $\tor(s)$.

The torsion measures how fast the osculating plane rotated when one travels along $\gamma$.

\begin{thm}{Exercise}\label{ex:helix-torsion}
Given real numbers $a$ and $b$, calculate curvature and torsion of the helix
\[\gamma_{a,b}(t)=(a\cdot \cos t,a\cdot\sin t, b\cdot t).\]

Conclude that for any $\kur>0$ and $\tor$ there is a helix with constant curvature $\kur$ and torsion $\tor$.
\end{thm}


\section*{Frenet formulas}

Assume the Frenet frame $\tau(s),\nu(s),\beta(s)$ of curve $\gamma$ is defined at $s$.
Recall that 
\[\tau'(s)=\kur(s)\cdot \nu(s).
\eqlbl{eq:frenet-tau}\]
Let us write the remaining derivatives $\nu'(s)$ and $\beta'(s)$ in the frame $\tau(s),\nu(s),\beta(s)$.

First let us show that
\[\nu'(s)=-\kur(s)\cdot\tau(s)+\tor(s)\cdot\beta(s).\eqlbl{eq:frenet-nu}\]

Since the frame $\tau(s),\nu(s),\beta(s)$ is orthonormal it is equivalent to the following three identities:
\begin{align*}
\langle \nu',\tau\rangle&=-\kur,
&
\langle \nu',\nu\rangle&=0,
&
\langle \nu',\beta\rangle&=\tor,
\end{align*}
The last identity follows from the definition of torsion.
Differentiating $\langle \nu,\nu\rangle\z=1$ in \ref{eq:orthornomal}, we get that
\[2\cdot\langle \nu',\nu\rangle=0;\]
whence the second identity follows.
Differentiating the identity $\langle\tau,\nu\rangle\z=0$ in \ref{eq:orthornomal};
we get that
\[\langle\tau',\nu\rangle+\langle\tau,\nu'\rangle=0.\]
Applying \ref{eq:frenet-tau}, we get that
\begin{align*}
\langle \nu',\tau\rangle&=-\langle\tau',\nu\rangle=
\\
&=-\kur\cdot\langle\nu,\nu\rangle=
\\
&=-\kur.
\end{align*}
It proves the first equality $\langle \nu',\tau\rangle=-\kur$ and
whence \ref{eq:frenet-nu} follows.

Taking derivatives of the third identity in \ref{eq:orthornomal}, we get that $\beta'\perp\beta$.
Further taking derivatives of the other identities with $\beta$ in \ref{eq:orthornomal}, we get that 
\begin{align*}
\langle\beta',\tau\rangle&=-\langle\beta,\tau'\rangle=-\kur\cdot\langle\beta,\nu\rangle=0
\\
\langle\beta',\nu\rangle&=-\langle\beta,\nu'\rangle=\tor
\end{align*}
Since the frame $\tau(s),\nu(s),\beta(s)$ is orthonormal, it follows that
\[\beta'(s)=-\tor(s)\cdot\nu(s).\eqlbl{eq:frenet-beta}\]

The equations \ref{eq:frenet-tau}, \ref{eq:frenet-nu} and \ref{eq:frenet-beta} are called Frenet formulas.
All three can be written as one matrix identity:
\[
\begin{pmatrix}
\tau'
\\
\nu'
\\
\beta'
\end{pmatrix}
=
\begin{pmatrix}
0&\kur&0
\\
-\kur&0&\tor
\\
0&-\tor&0
\end{pmatrix}
\cdot
\begin{pmatrix}
\tau
\\
\nu
\\
\beta
\end{pmatrix}.
\]

\begin{thm}{Exercise}\label{ex:beta-from-tau+nu}
Deduce the formula \ref{eq:frenet-beta} from  \ref{eq:frenet-tau} and \ref{eq:frenet-nu} by differentiating the identity
$\beta=\tau\times \nu$.
\end{thm}

\begin{thm}{Exercise} 
Let $\gamma$ be a regular space curve with nonvanishing curvature.
Show that $\gamma$ lies in a plane if and only if its torsion vanishes.
\end{thm}

\parit{Hint:} 
Show and use that the binormal vector is constant.

\begin{thm}{Exercise} 
Let $\gamma$ be a smooth regular space curve and $\tau,\nu,\beta$ its Frenet frame.
Show that 
\[\beta=\frac{\gamma'\times\gamma''}{|\gamma'\times\gamma''|}.\]
Use this formula to show that its torsion is
\[\tor=\frac{\langle\gamma'\times\gamma'',\gamma'''\rangle}{|\gamma'\times\gamma''|^2}.\]

\end{thm}

\section*{Curves of constant slope}

We say that a smooth regular space curve $\gamma$ has \emph{constant slope} if its velocity vector makes a constant angle with a fixed direction.
The following theorem was proved by Michel Ange Lancret~\cite{lancret} more than two centuries ago.

\begin{thm}{Theorem}\label{thm:const-slope}
Let $\gamma$ be a smooth regular curve;
denote by $\kur$ and $\tor$ its curvature and torsion.
Suppose $\kur(s)>0$ at any $s$.
Then $\gamma$ has constant slope if and only if the ratio $\tfrac\tor\kur$ is constant.
\end{thm}

Note that the assumption in the theorem implicitely implies that $\kur\ne 0$;
otherwise $\tfrac\tor\kur$ is undefined.

The proof is an application of Frenet formulas.
The following exercise will guide you thru the proof of the theorem. 

\begin{thm}{Exercise} \label{ex:lancret}
Suppose that $\gamma$ is a smooth regular space curve with nonvanishing curvature, $\tau,\nu,\beta$ 
is its Frenet frame and $\kur$, $\tor$ are its curvature and torsion.
\begin{enumerate}[(a)]
\item\label{ex:lancret:a} 
Assume that  $\langle w,\tau\rangle$ is constant for a fixed nonzero vector $w$.
Show that 
\[\langle w, \nu\rangle =0.\]
Use it to show that 
\[\langle w, -\kur\cdot\tau+\tor\cdot \beta\rangle =0.\]
Use these two identities to show that $\tfrac\tor\kur$ is constant;
it proves the ``only if'' part of the theorem.

\item Assume that $\tfrac\tor\kur$ is constant, show that the vector $w=\tfrac\tor\kur\cdot \tau+\beta$ is constant.
Conclude that $\gamma$ has constant slope; it proves the ``if'' part of the theorem.
\end{enumerate}
\end{thm}

Assume $\gamma$ is a smooth unit speed curve and $s_0$ is a fixed real number. %???unit-speed/unit speed
Then the curve 
\[\alpha(s)=\gamma(s)+(s_0-s)\cdot \gamma'(s)\]
is called \emph{evolvent} of $\gamma$.
Note that if $\ell(s)$ denotes the tangent line of $\gamma$ at $s$,
then $\alpha(s)\in \ell(s)$ and $\alpha'(s)\perp \ell$ for any $s$.

\begin{thm}{Exercise} 
Show that evolvent of a constant slope curve lies in a plane.
\end{thm}

\parit{Hint:} 
Show that $\langle w,\alpha\rangle$ is constant if $\gamma$ makes constant angle with a fixed vector $w$ and $\alpha$ is the evolvent of $\gamma$.

\section*{Spherical curves}

\begin{thm}{Theorem}
A smooth regular space curve $\gamma$ lies in a unit sphere if and only if 
the following identity 
\[\left|\tfrac{\kur'}{\tor}\right|=\kur\cdot\sqrt{\kur^2-1}.\]
holds for its curvature $\kur$ and torsion $\tor$.
\end{thm}

Note that the identity implicitely implies that the torsion $\tor$ of the curve is nonzero;
otherwise the left hand side would be undefined while right hand side is defined.
The proof is another application of Frenet formulas;
we present it in a form of guided exercise:

\begin{thm}{Exercise}\label{ex:spherical-frenet}
Suppose $\gamma$ is a smooth unit-speed space curve.
Denote by $\tau,\nu,\beta$ 
is its Frenet frame and $\kur$, $\tor$ its curvature and torsion.

\smallskip

Assume that $\gamma$ is spherical; that is, $|\gamma(s)|=1$ for any $s$.
Show that

\begin{enumerate}[(a)]
\item\label{ex:spherical-frenet:tau} $\langle\tau,\gamma\rangle=0$; conclude that $\langle\nu,\gamma\rangle^2+\langle\beta,\gamma\rangle^2=1$.
\item\label{ex:spherical-frenet:nu} $\langle\nu,\gamma\rangle=-\tfrac1\kur$;
\item\label{ex:spherical-frenet:beta} $\langle\beta,\gamma\rangle'=\tfrac\tor\kur$; conclude that if $\gamma$ is closed, then $\tor(s)=0$ for some~$s$.
\item\label{ex:spherical-frenet:kur-tor} Use (\ref{ex:spherical-frenet:tau})--(\ref{ex:spherical-frenet:beta}) to show that 
\[\left|\tfrac{\kur'}{\tor}\right|=\kur\cdot\sqrt{\kur^2-1}.\]
\setcounter{lastnumber}{\value{enumi}}
\end{enumerate}
It proves the ``only if'' part of the theorem.

\smallskip

Now assume that $\gamma$ is a space curve that satisfies the idenity in (\ref{ex:spherical-frenet:kur-tor}).
\begin{enumerate}[(a)]
\setcounter{enumi}{\value{lastnumber}}
\item Show that $p=\gamma+\tfrac1\kur\cdot \nu+\tfrac{\kur'}{\kur^2\cdot\tor}\cdot\beta$ is constant; conclude that $\gamma$ lies in a unit sphere the centered at $p$.
\end{enumerate}
It proves the ``if'' part of the theorem.
\end{thm}

For a unit speed curve $\gamma$ with nonzero curvature and torsion at $s$,
the sphere $\Sigma_s$ with the center
\[p(s)=\gamma(s)+\tfrac1{\kur(s)}\cdot \nu(s)+\tfrac{\kur'(s)}{\kur^2(s)\cdot\tor(s)}\cdot\beta(s)\]
that pass thru $\gamma(s)$ is called \emph{osculating sphere} of $\gamma$ at $s$.
This a unique sphere that has \emph{third order of contact} with $\gamma$ at $s$;
that is, $\rho(\ell)=o(\ell^3)$, where $\rho(\ell)$ denotes the distance from $\gamma(s+\ell)$ to $\Sigma_s$.
 
\section*{Fundamental theorem of curves}

\begin{thm}{Theorem}\label{thm:fund-curves}
Let $\kur(s)$ and $\tor(s)$ be two smooth real valued functions defined on a real interval $\mathbb{I}$.
Suppose $\kur(s)>0$ for any $s$.
Then there is a smooth unit-speed curve $\gamma\:\mathbb{I}\to\RR^3$ with curvature $\kur(s)$ and torsion $\tor(s)$ for every $s$.
Moreover $\gamma$ is uniquely defined up to a rigid motion of the space.
\end{thm}

\parit{Proof.}
Fix a parameter value $s_0$, a point $\gamma(s_0)$ and an oriented orthonormal frame $\tau(s_0)$, $\nu(s_0)$, $\beta(s_0)$.
Consider the following initial value problem: 
\[
\begin{cases}
\gamma'=\tau,
\\
\tau'=\kur\cdot\nu,
\\
\nu'=-\kur\cdot\tau+\tor\cdot\beta,
\\
\beta'=-\tor\cdot\nu.
\end{cases}
\]
It has four vector equations, so it can be rewritten as a system of 12 scalar equations.
By \ref{thm:ODE}, it has a solution which is defined in a maximal subinterval $\mathbb{J}$ containing $s_0$.

Note that 
\begin{align*}
\langle\tau,\tau\rangle'
&=
2\cdot\langle\tau,\tau'\rangle
=
2\cdot\kur\cdot \langle\tau,\nu\rangle
=
0,
\\
\langle\nu,\nu\rangle'
&=
2\cdot\langle\nu,\nu'\rangle
=
-
2\cdot\kur\cdot\langle\nu,\tau\rangle
+
2\cdot\tor\cdot\langle\nu,\beta\rangle
=
0,
\\
\langle\beta,\beta\rangle'
&=
2\cdot\langle\beta,\beta'\rangle
=
-2\cdot\tor\langle\beta,\nu\rangle
=
0,
\\
\langle\tau,\nu\rangle'
&=
\langle\tau',\nu\rangle
+
\langle\tau,\nu'\rangle
=
\kur\cdot\langle\nu,\nu\rangle
-
\kur\cdot\langle\tau,\tau\rangle
+
\tor\cdot\langle\tau,\beta\rangle
=
0,
\\
\langle\nu,\beta\rangle'
&=
\langle\nu',\beta\rangle+\langle\nu,\beta'\rangle
=0,
\\
\langle\beta,\tau\rangle'
&=
\langle\beta',\tau\rangle+\langle\beta,\tau'\rangle
=
-\tor\cdot \langle\nu,\tau\rangle
+\kur\cdot\langle\beta,\nu\rangle
=0.
\end{align*}
That is, the values 
$\langle\tau,\tau\rangle$,
$\langle\nu,\nu\rangle$,
$\langle\beta,\beta\rangle$,
$\langle\tau,\nu\rangle$,
$\langle\tau,\nu\rangle$,
$\langle\beta,\tau\rangle$
are constant functions of $s$.
Since we choose $\tau(s_0)$, $\nu(s_0)$, $\beta(s_0)$ to be an oriented orthonormal frame, we have that the $\tau(s)$, $\nu(s)$, $\beta(s)$ is oriented orthonormal for any $s$.

In particular $|\gamma'(s)|=1$ for any $s$.

Assume $\mathbb{J}\ne \mathbb{I}$.
Then an end of $\mathbb{J}$, say $a$, lies in the interior of $\mathbb{I}$.
By Theorem~\ref{thm:ODE}, at least one of the values $\gamma(s)$, $\tau(s)$, $\nu(s)$, $\beta(s)$
escapes to infinity as $s\to a$.
But this is impossible since the vectors $\tau(s)$, $\nu(s)$, $\beta(s)$ remain unit and $|\gamma'(s)|=|\tau(s)|=1$ --- a contradiction.
Whence $\mathbb{J}= \mathbb{I}$.

Assume there are two curves $\gamma_1$ and $\gamma_2$ with the given curvature and torsion functions.
Applying a motion of the space we can assume that the $\gamma_1(s_0)=\gamma_2(s_0)$ and the Frenet frames of the curves coincide at $s_0$.
Then $\gamma_1=\gamma_2$ by uniqueness of solution of the system (\ref{thm:ODE}).
That is, the curve is unique up to a rigid motion of the space.
\qeds

\begin{thm}{Exercise}\label{ex:cur+tor=helix}
Assume a curve $\gamma\:\RR\to\RR^3$ has constant curvature and torsion.
Show that $\gamma$ is a helix, possibly degenerate to a circle;
that is, in a suitable coordinate system we have
\[\gamma(t)=(a\cdot \cos t,a\cdot\sin t, b\cdot t)\]
for some constants $a$ and $b$.
\end{thm}

\parit{Hint:} Use the second statement in \ref{ex:helix-torsion}.


\begin{thm}{Advanced exercise}
Let $\gamma$ be a smooth regular space curve such that the distance $|\gamma(t)-\gamma(t+\ell)|$ depends only on $\ell$.
Show that $\gamma$ is a helix, possibly degenerate to a line or a circle.
\end{thm}

\parit{Hint:} Note that the function
\[\rho(\ell)=|\gamma(t+\ell)-\gamma(t)|^2=\langle \gamma(t+\ell)-\gamma(t),\gamma(t+\ell)-\gamma(t)\rangle\] 
is smooth and does not depend on $t$.
Express speed, curvature and torsion of $\gamma$ in terms of derivatives $\rho^{(n)}(0)$
and apply \ref{ex:cur+tor=helix}.
