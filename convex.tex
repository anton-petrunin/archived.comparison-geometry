\chapter{Convex surfaces}


\section{Embedded surfaces}
A set $X$ in the Euclidean space is called strictly convex if for any two points $x,y\in X$, any point $z$ between $x$ and $y$ lies in the interior of $X$.
Clearly any open convex set is stricly convex;
the closed cube (as well as any convex polyhedron) gives an example of a convex set which is not strictly convex.

\begin{thm}{Exercise}\label{ex:loc-convex}
Let $\Sigma$ be a surface with positive Gauss curvature.
Show that for any point $p\in \Sigma$ and all sufficiently small $\eps>0$,
the surface $\Sigma$ divides the ball $B(p,\eps)$ in two regions, one of which is strictly convex.
\end{thm}

The following theorem gives a global version of the above exercise.

\begin{thm}{Theorem}\label{thm:convex-embedded}
Assume $\Sigma$ is a closed or open smooth connected surface with positive Gauss curvature.
Then $\Sigma$ bounds a convex region $R$.
Moreover, if $\Sigma$ is closed then it is a sphere; that is, $\Sigma$ admits a smooth regular parametrization by $\SS^2$.

\end{thm}

\parit{Proof.} 
By Exercise~\ref{ex:loc-convex}, one of the regions, say $R$, bounded by $\Sigma$ is locally strictly convex;
that is, at any point $p$ of $R$, the intersection of $R$ with a sufficiently small ball centered at $p$ is strictly convex.

Since $\Sigma$ is connected, so it $R$.
Moreover any two points $x$ and $y$ in the interior of $R$ can be connected by a polygonal line $\beta$ in the interior of $R$.


Arguing by contradiction, assume the line segment $[xy]$ does not lie in the   in the interior of $R$.
Let $y_0$ be the first point on $\beta$ so that the line segment $[xy_0]$ intersects $\Sigma$; assume $[xy_0]$ touches $\Sigma$ at a point $z$.

\begin{wrapfigure}{r}{43 mm}
\vskip-0mm
\centering
\includegraphics{mppics/pic-37}
\vskip-0mm
\end{wrapfigure}

By Exercise \ref{ex:loc-convex}, $R\cap B(z,\eps)$ is strictly convex for all sufficiently small $\eps>0$.
On the other hand $z$ lies between two points both in the line segment $[xy_0]$ and $R\cap B(z,\eps)$ --- a contradiction.

It remains to parameterize $\Sigma$ by $\SS^2$.

Fix a point $p$ in the interior of $R$.
By strict convexity of $R$, for any point $x\in \SS^2$ there is unique point $x'\in \Sigma$ that lies on the halfline $px$;
moreover, the map $h\:x\mapsto x'$ describes a bijection $\SS^2\to \Sigma$.

Applying inverse function theorem in local coordinates of $\SS^2$ and $\Sigma$,
we get that the map $h$ is smooth and regular.
Hence the result.
\qeds

\section{Immersed surfaces}

The following theorem was proved by Jacques Hadamard \cite{hadamard}. %??? generalizations???

\begin{thm}{Theorem}\label{thm:convex-immersed}
Any closed connected immersed surface with positive Gauss curvature is embedded.
\end{thm}



\begin{wrapfigure}{r}{20 mm}
\vskip-0mm
\centering
\includegraphics{mppics/pic-35}
\vskip-0mm
\end{wrapfigure}

In other words such surface can not have self-intersections.
Note that an analogous statement does not hold in the plane;
on the diagram you can see a closed curve with a self-intersection and positive curvature at all points.
Exercise~\ref{ex:convex+2pi} gives a condition that guarantees simplicity of a locally convex plane curve;
it will be used in the following proof.



Before going into the proof, note that theorems \ref{thm:convex-immersed} and \ref{thm:convex-embedded}
imply the following:

\begin{thm}{Corollary}
Any closed connected immersed surface with positive Gauss curvature is an embedded sphere that bounds a convex region.
\end{thm}

In the following sections we will give one complete proof and sketch an alternative proof.

The first proof uses a \emph{Morse-type argument} for the height function;
that is, we study how the part of the surface that lies below a plane changes when we move the plane upward.
Little more careful analysis of this changes would imply the corollary above directly, without using Theorem~\ref{thm:convex-embedded}.

The sketch use equidistants surfaces and the Gauss map.
We will not proove a topological statement relying on intuition.

In the proof we abuse notation slightly;
we say a \emph{point of the immersed surface} instead of a \emph{point in the parameter domain of the immersed surface}.
So each point of self-intersection is considered as two or more ``distinct'' points of the surface.

\section{Morse-type proof}

Let $\Sigma$ be a closed surface with positive Gauss curvature, possibly with self-intersections. 

Fix a horizontal plane $\Pi_h$ defined by the equation $z=h$ in an $(x,y,z)$-coordinate system.
Note that the intersection $W_h=\Sigma\cap\Pi_h$ is formed by a finite collection of closed curves and isolated points.
(These curves and isolated points might intersect in the Euclidean space, but they are disjoint in the domain of parameters of $\Sigma$.)

Indeed, if $\T_p=\Pi_h$, then, since the principle curvatures are positive, $p$ is a local minimum or local maximum of the height function.
In both cases, $p$ is an isolated point of $W_h$ in $\Sigma$.
If the tangent plane $\T_p$ is not $\Pi_h$, then it is not perpendicular to $(x,z)$-plane or $(y,z)$-plane.
Therefore by Proposition~\ref{prop:perp}, the surface can be written locally as a graph $x=f(y,z)$ or $y=f(x,z)$;
in both cases $p$ lies on the curve $x=f(y,h)$ or correspondingly $y=f(x,h)$.

Summarizing, the closed set $W_h\subset \Sigma$ locally looks like a curve or an isolated point.
Since $\Sigma$ is compact, so is $W$.
Therefore $W$ is a finite disjoint collection of isolated points and closed simple curves in $\Sigma$.

Assume $\alpha_{h_0}$ is a closed curve in $W_{h_0}$.
Note that its neighborhood is swept by curves $\alpha_h$ in $W_{h}$ for $h\approx h_0$.
Indeed a neighbohood of $\alpha_{h_0}$ in $\Sigma$ can be covered by a finite number of graphs of the type $x=f(y,z)$ (or $y=f(x,z)$) and the curves $\alpha_h$ can be described locally as curves of the form $t\mapsto (f(t,h),t,h)$ (or correspondingly $t\z\mapsto(t,f(t,h),h)$) for $h\approx h_0$.

As $\alpha_h$ is the intersection of a locally convex surface with a plane,
the curvature of $\alpha_h$ has fixed sign;
so if we choose an orientation for the curves properly, we can assume that they all have positive curvature.

The family $\alpha_h$ depends smoothly on $h$ and the same holds for its tangent indicatrix.
Therefore the total signed curvature $K_h$ of $\alpha_h$ depends continuously on $h$.
If $K_h=2\cdot\pi$ for some $h$, then $K_h=2\cdot\pi$ for every $h$.
It follows since, the function $h\mapsto K_h$ is continuous and its value is a multiple of $2\cdot\pi$.
In this case, by Exercise~\ref{ex:convex+2pi}, all curves $\alpha_h$ are simple and each one bounds a convex region in the plane $\Pi_h$.

Summarizing, if one of the curves in the constructed family $\alpha_{h}$ is simple,
then each curve in the family is simple and each $\alpha_{h}$ bounds a convex region in the plane $\Pi_h$. 

Choose a point $p\in \Sigma$ that minimizes the height function $z$.
Without loss of generality we may assume that $p$ is the origin and therefore the surface lies in the upper half-space.

Fix $h>0$.
The intersection of the set $z\le h$ with the surface may contain several connected components;
one of them contains $p$, denote this component by $\Sigma_h$.\footnote{These components might intersect in the space, but they are disjoint in the domain of parameters. Note also that from the corollary, it follows that there is only one component $\Sigma_h$, but we can not use it before the theorem is proved.}


From above, $\Sigma_h$ is a surface with possibly nonempty boundary.
Indeed it might be bounded only by few closed curves in $W_h$;
any isolated point of $W_h$ either lies in $\Sigma_h$ together with its neighborhood or does not lie in $\Sigma_h$.


Note that for small values of $h$, the surface $\Sigma_h$ is an embedded disc.
Indeed, if $z=f(x,y)$ is a graph represetation of $\Sigma$ around $p$,
then $\Sigma_h$ is a graph of $f$ over 
\[\Delta=\set{(x,y)\in\RR^2}{f(x,y)\le h}.\]
Since the Gauss curvature is positive, the function $f$ is convex and therefore $\Delta$ is convex and bounded by a smooth curve;
any such set can be parameterized by a disc.

\begin{wrapfigure}{r}{30 mm}
\vskip-0mm
\centering
\includegraphics{mppics/pic-36}
\vskip-0mm
\end{wrapfigure}

Let $H>0$ be the maximal value such that $\Sigma_h$ has no self-intersections for any $h<H$.
For a sequence $h_n\to H^-$, choose a point $q_n$ on the boundary of $\Sigma_h$ and pass to a partial limit $q$ of $q_n$ in $\Sigma$;
that is, $q$ is a limit of a subsequence of~$(q_n)$.

If the tangent plane at $q$ is \emph{not} horizontal, 
then there is a closed curve $\alpha_H$ in $\Sigma$ that passes thru $q$ and lies on the plane $z=H$.
From the above discussion, the curve $\alpha_H$, as well as all $\alpha_h$ with $h\approx H$ are closed embedded convex curves.
Hence $\Sigma_h$ has no self-intersections for some $h>H$ --- a contradiction.

If the tangent plane at $q$ is horizontal,
then the surface $\Sigma_H$ has no boundary.
Since $\Sigma$ is connected, $\Sigma_H=\Sigma$.
Since $\Sigma_h$ has no self-intersections for $h<H$, we get that $\Sigma$ is an embedded surface.
\qeds

\begin{thm}{Exercise}
Show that any open immersed surface with positive Gauss curvature is embedded.\footnote{Hint: Modify the proof of the theorem.}
\end{thm}

\section{Proof via equidistant surfaces}

Recall that a surface $\Sigma$ is called \emph{orientable} if one can choose at each point $p$ of the surface
a unit normal vector $\nu(p)$  in such a way that the function $p\mapsto \nu_p$ is continuous in every chart of $\Sigma$.
For immersed surfaces we should say that $\nu$ is a continuous function defined on the parameter domain of the surface.
The map $\nu$ is called a \emph{Gauss map} of the surface.

\begin{thm}{Claim}
Assume $\Sigma$ is a closed immersed surface with positive Gauss curvature, then it is orientable.
\end{thm}


\parit{Proof.} Indeed we can choose the unit normal vector $\nu(p)$ in such a way that both principle curvatures are positive. 
Iin this case the surface lies locally on the side of tangent plane $\T_p$ which is opposite from $\nu(p)$.

Evidently this choice is  continuous.
\qeds

The unit normal described in the proof of the claim will be called \emph{outer normal}.

\begin{thm}{Lemma}\label{lem:gauss-inverse}
Assume $\Sigma$ is a closed connected immersed surface with positive Gauss curvature.
Then the Gauss map $\nu\:\Sigma \to \SS^2$ has a smooth regular inverse;
in particular, $\Sigma$ is a sphere.
\end{thm}

This lemma follows from two facts:
(1) if Gauss curvature does not vanish then the  Gauss map is regular, in particular this map has a local inverse at each point
and
(2) the sphere $\SS^2$ is \emph{simply connected};
that is, $\SS^2$ is connected any closed curve in $\SS^2$ can be deformed continuously into a trivial curve that stays at one point.
The proof is standard in topology, we hope that the statement is intuitively obvious.
The reader might be able to reinvent the theory by trying to prove that if the map $\phi\:\SS^2\to\SS^2$ is smooth and regular then it has an inverse.

\parbf{Equidistant surfaces.}
Assume $\nu\:\Sigma\to \SS^2$ is a Gauss map of a smooth surface $\Sigma$.
Fix a real number $R$ and consider the map $h_R\:\Sigma\z\to\RR^3$ defined by $h_R\:p\mapsto p+R\cdot\nu(p)$.
The map $h_R$ describe the so called \emph{equidistant surface};
it is smooth by definition, but in general it does not have to be regular.

\begin{thm}{Lemma}\label{lem:curc<1/R}
Suppose $\nu\:\Sigma\to \SS^2$ is a Gauss map of a surface $\Sigma$.
Assume the corresponding principle curvatures are nonnegative at all points. 
Then the equidistant surface $\Sigma_R$ is regular and its principle curvatures are positive and strictly smaller than $\tfrac1R$.
\end{thm}

\parit{Proof.}
To prove regularity, let us use the special representation of $\Sigma$ as a graph $z=f(x,y)$ with the $x$ and $y$ axis in the principle directions of $\Sigma$ at $p$.\footnote{If we assume that $\nu(p)$ points in the direction of the $z$-axis, then $\Sigma$ is given in parametric form 
$h_0\:(x,y)\mapsto (x,y,f(x,y))$,
where $f\z=-\tfrac{k_1}2\cdot x^2-\tfrac{k_2}2\cdot y^2+o(x^2+y^2)$.}

Due to the choice of directions of $x$ and $y$ axis,
for the Gauss map $g(x,y)$, we have 
\begin{align*}
\tfrac{\partial}{\partial x}g(0,0)&=(k_1,0,0),
\\
\tfrac{\partial}{\partial y}g(0,0)&=(0,k_2,0).
\end{align*}
Then $h_R=h_0+R\cdot g$; therefore
\begin{align*}
\tfrac{\partial}{\partial x} h_R&=(1+R\cdot k_1,0,0),
\\
\tfrac{\partial}{\partial y} h_R&=(0,1+R\cdot k_2,0)
\end{align*}
which are linearly independent if $R\ge0$ and $k_1,k_2\ge 0$. 

If $\Sigma$ bounds a convex closed set $K$.
Then $\Sigma_R$ bounds $K_R$ --- the closed $R$-neighborhood of $K$;
that is, $K_R$ is the set of all points at distance at most $R$ from $K$.

Since $\Sigma$ is smooth it is supported at each point $p$ from inside by a small ball $B_\eps(o)$.
Then the ball $B_{R+\eps}(o)$ lies in $K_R$ and touches its boundary at the point corresponding to $p$.
Hence the principle curvatures at $p$ are at least $\tfrac1{R+\eps}$.

In the general case, a local chart of $\Sigma$ can be modified so that it has a piece of the original surface around $p$  and bounds a convex set.
Here is one way to do this:

\begin{wrapfigure}{r}{30 mm}
\vskip-0mm
\centering
\includegraphics{mppics/pic-38}
\vskip-0mm
\end{wrapfigure}

Choose a smooth function $\phi(x)$ that is convex increasing and such that for sufficiently small $\eps>0$ we have $\phi(x)=x$ if $x<\eps$ and $\phi(x)\to\infty$ as $x\to 2\cdot\eps$.
(Such functions do exist; moreover an explicit formula can be written, but we leave it without a proof.)

Assume $z=f(x,y)$ is a special representation of $\Sigma$ around $p$ by some convex function $f$.
Direct computations show that $h=\phi\circ f(x,y)$ is still convex.
The surface $\Sigma'$ described as the graph $z=h(x,y)$ bounds a convex closed set $K$ and the part of $\Sigma'$ described by 
the parameters $\set{(x,y)}{f(x,y)<\eps}$ coincide with a neighborhood of $p$ in $\Sigma$.
Hence the general case follows.\qeds

\parit{Proof assembling.}
Let $s\:\SS^2\z\to\RR^3$ be the parametization of $\Sigma$ provided by Lemma~\ref{lem:gauss-inverse}.
Then the equidistant surface $\Sigma_R$ can be parametrized by $s_R(u)= s(u)+R\cdot u$ for $u\in\SS^2$.
Rescaling $s_R$ by a  factor of $\tfrac1r$ we get the map $u\mapsto \tfrac1R\cdot s(u)+u$ which converges smoothly to the indentity map on the sphere $\SS^2$.
Therefore $\Sigma_R$ is embedded for sufficiently large $R$.

\begin{wrapfigure}{r}{47 mm}
\vskip-0mm
\centering
\includegraphics{mppics/pic-39}
\vskip-0mm
\end{wrapfigure}

Applying Theorem~\ref{thm:convex-embedded}, we get that $\Sigma_R$ bounds a convex set.

By Lemma~\ref{lem:curc<1/R}, the principle curvatures of $\Sigma_R$ are smaller than $\tfrac1R$.
Therefore the same idea as in Exercise~\ref{ex:convex-lagunov} shows that any point $p$ of $\Sigma_R$ can be supported by a ball of radius $R$ from inside.
Note that the center $p'$ of such ball has to lie on $\Sigma$;
indeed it lies at distance $R$ in the normal direction.
In other words, the map $s_0(u)=s_R(u)-R\cdot u$ is injective, or equivalently $\Sigma$  has no self-intersection.
\qeds

