\section*{Immersed surfaces}

It seems that first formulation and proof of the following theorem was given by James Stoker \cite{stoker} who attributed it to Jacques Hadamard, who proved a closely relevant statement in \cite[item 23]{hadamard}.


\begin{thm}{Theorem}\label{thm:convex-immersed}
Any closed connected immersed surface with positive Gauss curvature is embedded.
\end{thm}

\begin{wrapfigure}{o}{20 mm}
\vskip-0mm
\centering
\includegraphics{mppics/pic-35}
\vskip-0mm
\end{wrapfigure}

In other words such surface can not have self-intersections.
Note that an analogous statement does not hold in the plane;
on the diagram you can see a closed curve with a self-intersection and positive curvature at all points.
Exercise~\ref{ex:convex+2pi} gives a condition that guarantees simplicity of a locally convex plane curve;
it will be used in the following proof.



Before going into the proof, note that theorems \ref{thm:convex-immersed} and \ref{thm:convex-embedded}
imply the following:

\begin{thm}{Corollary}
Any closed connected immersed surface with positive Gauss curvature is an embedded sphere that bounds a convex region.
\end{thm}

In the following sections we will give one complete proof and sketch an alternative proof.

The first proof uses a \emph{Morse-type argument} for the height function;
that is, we study how the part of the surface that lies below a plane changes when we move the plane upward.
Little more careful analysis of this changes would imply the corollary above directly, without using Theorem~\ref{thm:convex-embedded}.

The sketch use equidistants surfaces and the Gauss map.
We will not proove a topological statement relying on intuition.

In the proof we abuse notation slightly;
we say a \emph{point of the immersed surface} instead of a \emph{point in the parameter domain of the immersed surface}.
So each point of self-intersection is considered as two or more ``distinct'' points of the surface.

\section*{Morse-type proof}

Let $\Sigma$ be a closed surface with positive Gauss curvature, possibly with self-intersections. 

Fix a horizontal plane $\Pi_h$ defined by the equation $z=h$ in an $(x,y,z)$-coordinate system.
Note that the intersection $W_h=\Sigma\cap\Pi_h$ is formed by a finite collection of closed curves and isolated points.
(These curves and isolated points might intersect in the Euclidean space, but they are disjoint in the domain of parameters of $\Sigma$.)

Indeed, if $\T_p=\Pi_h$, then, since the principle curvatures are positive, $p$ is a local minimum or local maximum of the height function.
In both cases, $p$ is an isolated point of $W_h$ in $\Sigma$.
If the tangent plane $\T_p$ is not $\Pi_h$, then it is not perpendicular to $(x,z)$-plane or $(y,z)$-plane.
Therefore by Proposition~\ref{prop:perp}, the surface can be written locally as a graph $x=f(y,z)$ or $y=f(x,z)$;
in both cases $p$ lies on the curve $x=f(y,h)$ or $y=f(x,h)$ respectively.

Summarizing, the closed set $W_h\subset \Sigma$ locally looks like a curve or an isolated point.
Since $\Sigma$ is compact, so is $W$.
Therefore $W$ is a finite disjoint collection of isolated points and closed simple curves in $\Sigma$.

Assume $\alpha_{h_0}$ is a closed curve in $W_{h_0}$.
Note that its neighborhood is swept by curves $\alpha_h$ in $W_{h}$ for $h\approx h_0$.
Indeed a neighborhood of $\alpha_{h_0}$ in $\Sigma$ can be covered by a finite number of graphs of the type $x=f(y,z)$ (or $y=f(x,z)$) and the curves $\alpha_h$ can be described locally as curves of the form $t\mapsto (f(t,h),t,h)$ (or respectively $t\z\mapsto(t,f(t,h),h)$) for $h\approx h_0$.

As $\alpha_h$ is the intersection of a locally convex surface with a plane,
the curvature of $\alpha_h$ has fixed sign;
so if we choose an orientation for the curves properly, we can assume that they all have positive curvature.

The family $\alpha_h$ depends smoothly on $h$ and the same holds for its tangent indicatrix.
Therefore the total signed curvature $K_h$ of $\alpha_h$ depends continuously on $h$.
If $K_h=2\cdot\pi$ for some $h$, then $K_h=2\cdot\pi$ for every $h$.
It follows since, the function $h\mapsto K_h$ is continuous and its value is a multiple of $2\cdot\pi$.
In this case, by Exercise~\ref{ex:convex+2pi}, all curves $\alpha_h$ are simple and each one bounds a convex region in the plane $\Pi_h$.

Summarizing, if one of the curves in the constructed family $\alpha_{h}$ is simple,
then each curve in the family is simple and each $\alpha_{h}$ bounds a convex region in the plane $\Pi_h$. 

Choose a point $p\in \Sigma$ that minimizes the height function $z$.
Without loss of generality we may assume that $p$ is the origin and therefore the surface lies in the upper half-space.

Fix $h>0$.
The intersection of the set $z\le h$ with the surface may contain several connected components;
one of them contains $p$, denote this component by $\Sigma_h$.\footnote{These components might intersect in the space, but they are disjoint in the domain of parameters. Note also that from the corollary, it follows that there is only one component $\Sigma_h$, but we can not use it before the theorem is proved.}


From above, $\Sigma_h$ is a surface with possibly nonempty boundary.
Indeed it might be bounded only by few closed curves in $W_h$;
any isolated point of $W_h$ either lies in $\Sigma_h$ together with its neighborhood or does not lie in $\Sigma_h$.


Note that for small values of $h$, the surface $\Sigma_h$ is an embedded disc.
Indeed, if $z=f(x,y)$ is a graph represetation of $\Sigma$ around $p$,
then $\Sigma_h$ is a graph of $f$ over 
\[\Delta=\set{(x,y)\in\RR^2}{f(x,y)\le h}.\]
Since the Gauss curvature is positive, the function $f$ is convex and therefore $\Delta$ is convex and bounded by a smooth curve;
any such set can be parameterized by a disc.

\begin{wrapfigure}{o}{30 mm}
\vskip-0mm
\centering
\includegraphics{mppics/pic-36}
\vskip-0mm
\end{wrapfigure}

Let $H>0$ be the maximal value such that $\Sigma_h$ has no self-intersections for any $h<H$.
For a sequence $h_n\to H^-$, choose a point $q_n$ on the boundary of $\Sigma_h$ and pass to a partial limit $q$ of $q_n$ in $\Sigma$;
that is, $q$ is a limit of a subsequence of~$(q_n)$.

If the tangent plane at $q$ is \emph{not} horizontal, 
then there is a closed curve $\alpha_H$ in $\Sigma$ that passes thru $q$ and lies on the plane $z=H$.
From the above discussion, the curve $\alpha_H$, as well as all $\alpha_h$ with $h\approx H$ are closed embedded convex curves.
Hence $\Sigma_h$ has no self-intersections for some $h>H$ --- a contradiction.

If the tangent plane at $q$ is horizontal,
then the surface $\Sigma_H$ has no boundary.
Since $\Sigma$ is connected, $\Sigma_H=\Sigma$.
Since $\Sigma_h$ has no self-intersections for $h<H$, we get that $\Sigma$ is an embedded surface.
\qeds

\begin{thm}{Exercise}
Modify the proof of the theorem to show that any open immersed surface with positive Gauss curvature is embedded.
\end{thm}

