\chapter{Convex surfaces}


\section{Embedded surfaces}
A set in $X$ Euclidean space is called strictly convex if for any two points $x,y\in X$ any point $z$ that lies between $x$ and $y$ lies in the interior of $X$.
Clearly any open convex set is stricly convex;
the cube (as well as any convex polyhedron) geves an example of convex set which is not strictly convex.

\begin{thm}{Exercise}\label{ex:loc-convex}
Let $\Sigma$ be a surface with positive Gauss curvature.
Show that for any point $p\in \Sigma$ and all sufficiently small $\eps>0$,
the surface $\Sigma$ divides the ball $B(p,\eps)$ into two regions, one of which is strictly convex.
\end{thm}

The following theorem gives a global version of the exercise above.

\begin{thm}{Theorem}\label{thm:convex-embedded}
Assume $\Sigma$ is a closedor open smooth connected surface with positive Gauss curvature.
Then $\Sigma$ bounds a convex region $R$.
Moreover, if $\Sigma$ is closed then it is a sphere; that is, $\Sigma$ admits a smooth regular parametrization by $\SS^2$.

\end{thm}

\parit{Proof.} 
By Exercise~\ref{ex:loc-convex}, one of the regions, say $R$, bounded by $\Sigma$ is strictly convex locally;
that is intersection of $R$ with a sufficiently small ball centered at a given point is strictly convex.

Since $\Sigma$ is connected, so it $R$.
Moreover any two points $x$ and $y$ in the interior of $R$ can be connected by a polygonal line $\beta$ in the interior of $R$.


Arguing by contradiction, assume the line segment $[xy]$ does not lie in the   in the interior of $R$.
Let $y_0$ be the first point on $\beta$ so that the line segment $[xy_0]$ touches $\Sigma$; assume it touch it at a point $z$.

\begin{wrapfigure}{r}{43 mm}
\vskip-0mm
\centering
\includegraphics{mppics/pic-37}
\vskip-0mm
\end{wrapfigure}

By Exercise \ref{ex:loc-convex}, $R\cap B(z,\eps)$ is strictly convex for all sufficiently small $\eps>0$.
On the other hand $z$ lies between two points common to the line segment $[xy_0]$ and $R\cap B(z,\eps)$ --- a contradiction.

It remains to parameterize $\Sigma$ by $\SS^2$.

Fix a point $p$ in the interior of $R$.
By strict convexity of $R$, for any point $x\in \SS^2$ there is unique point $x'\in \Sigma$ that lies on the halfline $px$;
moreover, the map $h\:x\mapsto x'$ describes a bijection $\SS^2\to \Sigma$.

Applying inverse function theorem in a local coordinates of $\SS^2$ and $\Sigma$,
we get that the map $h$ is smooth and regular.
Hence the result.
\qeds

\section{Immersed surfaces}

\begin{thm}{Theorem}\label{thm:convex-immersed}
Any closed connected immersed surface with positive Gauss curvature is embedded.
\end{thm}



\begin{wrapfigure}{r}{20 mm}
\vskip-0mm
\centering
\includegraphics{mppics/pic-35}
\vskip-0mm
\end{wrapfigure}

In other words such surface can not have self-intersections.
Note that an analogous statement does not hold in the plane;
on the diagram you see a closed curve with a self-intersection and positive curvature at all points.
Exercise~\ref{ex:convex+2pi} gives a condition that guarantees simplicity of locally convex plane curve;
it will be used in the following proof.



Before going into the proof, note that theorems \ref{thm:convex-immersed} and \ref{thm:convex-embedded}
imply the following:

\begin{thm}{Corollary}
Any closed connected immersed surface with positive Gauss curvature is an embedded sphere that bounds a convex region.
\end{thm}

In the following sections we will give one complete proof and sketch an alternative proof.

The first proof use a \emph{Morse-type argument} for the height function;
that is, we study how the part of the surface that lies below a plane changes when we move the plane up.
Little more careful analysis of this changes would imply the corollary above directly, without using Theorem~\ref{thm:convex-embedded}.

The sketch use equidistant surfaces and Gauss map.
We do not proof a topological statement which relying on intuition.

In the proof we abuse notation slightly;
we say a \emph{point of immersed surface} instead of \emph{point in the parameter domain of immersed surface}.
So each point of self-intersection is considered as two or more ``distinct'' points of the surface.

\section{Morse-type proof}

Let $\Sigma$ be a closed surface with positive Gauss curvature, possibly with self-intersections. 

Fix a horizontal plane $\Pi_h$ defined by the equation $z=h$ in an $(x,y,z)$-coordinate system.
Note that the intersection $W_h=\Sigma\cap\Pi_h$ is formed by a finite collection of closed curves and isolated points.
(These curves and isolated points might intersect in the Euclidean space, but they are disjoint in the domain of parameters of $\Sigma$.)

Indeed, if $\T_p=\Pi_h$, then, since the principle curvatures are positive, $p$ is a local minimum or local maximum of the height function.
In both cases, $p$ is an isolated point of $W_h$ in $\Sigma$.
If the tangent plane $\T_p$ is not $\Pi_h$, then it is not perpendicular to $(x,z)$-pane or $(y,z)$-plane.
Therefore by Proposition~\ref{prop:perp}, the surface can be written locally as a graph $x=f(y,z)$ or $y=f(x,z)$;
in both cases $p$ lies on the curve $x=f(y,h)$ or correspondingly $y=f(x,h)$.

Summarizing, the closed set $W_h\subset \Sigma$ locally looks like a curve or an isolated point.
Since $\Sigma$ is compact, so is $W$.
Therefore $W$ is a finite disjoint collection of isolated points and closed simple curves in $\Sigma$.

Assume $\alpha_{h_0}$ is a closed curves in $W_{h_0}$.
Note that its neighborhood is swept by curves $\alpha_h$ in $W_{h}$ for $h\approx h_0$.
Indeed a neighbohood of $\alpha_{h_0}$ in $\Sigma$ can be covered by a finite number of graphs of the type $x=f(y,z)$ (or $y=f(x,z)$) and the curves $\alpha_h$ can be described locally as a curve $t\mapsto (f(t,h),t,h)$ (or correspondingly $t\z\mapsto(t,f(t,h),h)$) for $h\approx h_0$.

As $\alpha_h$ is an intersection of locally convex surface with a plane,
the curvature of $\alpha_h$ has fixed sign;
so if we choose orientation of the curves properly, we can assume that they all have positive curvature.

The family $\alpha_h$ depends smoothly on $h$ and the same holds for its tangent indicatrix.
Therefore the total signed curvature $K_h$ of $\alpha_h$ depends continuously on $h$.
If $K_h=2\cdot\pi$ for some $h$, then $K_h=2\cdot\pi$ for every $h$.
It follows since, the function $h\mapsto K_h$ is continuous and its value is a multiple of $2\cdot\pi$.
In this case, by Exercise~\ref{ex:convex+2pi}, all curves $\alpha_h$ are simple and each bounds a convex region in the plane $\Pi_h$.

Summarizing, if one of the curves in the constructed family $\alpha_{h}$ is simple,
then each curve in the family is simple and each $\alpha_{h}$ bounds a convex region in the plane $\Pi_h$. 

Choose a point $p\in \Sigma$ that minimize the height function $z$.
Without loss of generality we may assume that $p$ is the origin and therefore the surface lies in the upper half-space.

Fix $h>0$.
The intersection of the set $z\le h$ with the surface may contain several connected components;
one of them contains $p$, denote this component by $\Sigma_h$.\footnote{These components might intersect in the space, but they are disjoint in the domain of parameters. Note also that from the corollary, it follows that there is only one component $\Sigma_h$, but we can not use it before the theorem is proved.}


From above, $\Sigma_h$ is a surface with possibly nonempty boundary.
Indeed it might be bounded only by few closed curves in $W_h$;
any isolated point of $W_h$ either lie in $\Sigma_h$ together with it neighborhood or do not lie in $\Sigma_h$.


Note that for small values of $h$, the surface $\Sigma_h$ is an embedded disc.
Indeed, if $z=f(x,y)$ is a graph represetation of $\Sigma$ around $p$,
then $\Sigma_h$ is a graph of $f$ over 
\[\Delta=\set{(x,y)\in\RR^2}{f(x,y)\le h}.\]
Since Gauss curvature is positive, the function $f$ is convex and therefore $\Delta$ is convex and bounded by a smooth curve;
any such set can be parameterized by a disc.

\begin{wrapfigure}{r}{30 mm}
\vskip-0mm
\centering
\includegraphics{mppics/pic-36}
\vskip-0mm
\end{wrapfigure}

Let $H>0$ be the maximal value such that $\Sigma_h$ has no self-intersections for any $h<H$.
For a sequence $h_n\to H^-$, choose a point $q_n$ on the boundary of $\Sigma_h$ and pass to a partial limit $q$ of $q_n$ in $\Sigma$;
that is, $q$ is a limit of a subsequence of~$(q_n)$.

If the tangent plane at $q$ is \emph{not} horizontal, 
then there is a closed curve $\alpha_H$ in $\Sigma$ that passes thru $q$ and lies on the plane $z=H$.
From above curve $\alpha_H$, as well as all $\alpha_h$ with $h\approx H$ are closed embedded convex curves.
Hence $\Sigma_h$ has no self-intersections for some $h>H$ --- a contradiction.

If the tangent plane at $q$ is horizontal,
then the surface $\Sigma_H$ has no boundary.
Since $\Sigma$ is connected, $\Sigma_H=\Sigma$.
Since $\Sigma_h$ has no self-intersections for $h<H$, we get that $\Sigma$ is an embedded surface.
\qeds

\begin{thm}{Exercise}
Show that any open immersed surface with positive Gauss curvature is embedded.\footnote{Hint: Modify the proof of the theorem.}
\end{thm}

\section{Proof via equidistant surface}

\begin{thm}{Claim}
Assume $\Sigma$ is a closed immersed surface, then it is orientable.
\end{thm}


\parit{Proof.} Indeed we can choose the unit normal vector $\nu(p)$ in such a way that both principle curvatures are positive (in this case the surface lies locally on the side of tangent plane $\T_p$ which is opposite from $\nu(p)$).
Evidently this choice is depends continuously from the point on $\Sigma$ (more precisely on the value in the parameter domain).
\qeds

The unit normal described in the proof will be called outer normal.

\begin{thm}{Lemma}\label{lem:gauss-inverse}
Assume $\Sigma$ is a closed connected immersed surface.
Then Gauss map $\nu\:\Sigma \to \SS^2$ has a smooth regular inverse;
in particular $\Sigma$ is a sphere.
\end{thm}

This lemma follows from two facts:
(1) if Gauss curvature does not vanish then the  Gauss map is regular, in particular this map has a local inverse at each point
and
(2) the sphere $\SS^2$ is \emph{simply connected};
that is, $\SS^2$ is connected any closed curve in $\SS^2$ can be deformed continuously into a trivial curve that stays at one point.
The proof is standard in topology, we hope that the statement is intuitively obvious.

\parbf{Equidistant surfaces.}
Assume $\nu\:\Sigma\to \SS^2$ is a Gauss map of a surface $\Sigma$.
Fix a real number $R$ and consider the map $h_R\:\Sigma\to\RR^3$ defined by $h_R\:p\mapsto p+R\cdot\nu(p)$.
The map $h_R$ describe the so called \emph{equidistant surface};
it is smooth by definition, but might be not regular in general.

\begin{thm}{Lemma}\label{lem:curc<1/R}
Suppose $\nu\:\Sigma\to \SS^2$ is a Gauss map of a surface $\Sigma$.
Assume the corresponding principle curvatures are nonnegative at all points. 
Then the equidistant surface $\Sigma_R$ is regular and its principle curvatures are positive and strictly smaller than $\tfrac1R$.
\end{thm}

\parit{Proof.}
To prove regularity, let us use the special representation of $\Sigma$ as a graph $z=f(x,y)$ with $x$ and $y$ axis in the principle directions of $\Sigma$ at $p$.
In other words, $\Sigma$ is given in parametric form 
\[h_0\:(x,y)\mapsto (x,y,f(x,y)).\]

Due to the choice of directions of $x$ and $y$ axis,
for the Gauss map $g$, we have 
\begin{align*}
\tfrac{\partial}{\partial x}g(0,0)&=(k_1,0,0),
\\
\tfrac{\partial}{\partial x}g(0,0)&=(0,k_2,0).
\end{align*}
Then 
\begin{align*}
\tfrac{\partial}{\partial x} h_R&=(1+R\cdot k_1,0,0),
\\
\tfrac{\partial}{\partial x} h_R&=(0,1+R\cdot k_2,0)
\end{align*}
which are linearly independent for if $R>0$ and $k_1,k_2\ge 0$. 

If $\Sigma$ bounds a convex closed set $K$.
Then $\Sigma_R$ bounds $K_R$ --- the closed closed $R$-neighborhood of $K$;
that is, $K$ is the set of all points on the distance at most $R$ from $K$.

Since $\Sigma$ is smooth it is supported at each point $p$ from inside by a ball of small $B_\eps(o)$.
Then the ball $B_{R+\eps}(o)$ lies in $K_R$ and touches its boundary at the point corresponding to $p$.
Hence the principle curvatures at $p$ are at least $\tfrac1{R+\eps}$.

In general case, a local chart of $\Sigma$ can be modified so it has a piece of original surface around $p$  and bounds a convex set.
Here is one way to do this:

Choose a smooth function $\phi(x)$ that is convex increasing and such that for sufficiently small $\eps>0$ we have $\phi(x)=x$ if $x<\eps$ and $\phi(x)\to\infty$ as $x\to 2\cdot\eps$.
(Such functions do exist; moreover an explicit formula can be written, but we leave it without a proof.)

If $z=f(x,y)$ is a special representation of $\Sigma$ around $p$ by some convex function $f$ then the function $h=\phi\circ f(x,y)$ is trill convex, the graph $z=h(x,y)$ bounds a convex closed set and the part of the surface described by parameters $\set{(x,y)}{f(x,y)<\eps}$ coincide with a neighborhood of $p$ in $\Sigma$.
Hence the general case follows.\qeds

\parit{Sketch of proof of \ref{thm:convex-immersed}.}
Let $s\:\SS^2\to\RR^3$ be the parametization of $\Sigma$ provides by Lemma~\ref{lem:gauss-inverse}.
Then the equidistant surface $\Sigma_R$ can be parametrized by map $u\mapsto s(u)+R\cdot u$.
Applying rescaling with factor $\tfrac1r$ we get map $u\mapsto \tfrac1R\cdot s(u)+u$ which converges to the indentity map on the sphere $\SS^2$ together with all its derivatives.
Therefore $\Sigma_R$ for sufficiently large $R$.

Applying Theorem~\ref{thm:convex-embedded}, we get that $\Sigma_R$ bounds a convex set.

By Lemma~\ref{lem:curc<1/R}, principle curvatures of $\Sigma_R$ are smaller than $\tfrac1R$.
Therefore same idea as in the Exercise~\ref{ex:convex-lagunov} shows that any point of $\Sigma_R$ can be touched by a ball of radius $R$ from inside; moreover such ball touches surface at a single point.
The center of this ball has to lie on $\Sigma$ and last property implies that it has no self-intersection.
\qeds

