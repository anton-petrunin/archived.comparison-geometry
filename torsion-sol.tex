\stepcounter{section}
\setcounter{eqtn}{0}

\parbf{\ref{ex:helix-torsion}.} 
The arc-length parameter $s$ is already found in   \ref{ex:arc-length-helix}.
It remains to find the Frenet frame and calculate curvature and torsion.
The latter can be done by straightforward calculations;
the answers are 
\begin{align*}
\tan(t)&=\tfrac{1}{\sqrt{a^2+b^2}}\cdot(-a\cdot\sin t, a\cdot\cos t,b),
&
\norm(t)&=(-\cos t,-\sin t,0),
\\
\bi(t)&=\tfrac{1}{\sqrt{a^2+b^2}}\cdot(b\cdot \sin t,-b\cdot \cos t, a),
&
\kur &\equiv \tfrac{a}{a^2+b^2},\qquad
\tor \equiv \tfrac{b}{a^2+b^2}.
\end{align*}

\begin{wrapfigure}{r}{20 mm}
\vskip0mm
\centering
\begin{lpic}[t(-0mm),b(0mm),r(0mm),l(0mm)]{asy/helix(1)}
\lbl[br]{8,24;$\norm$}
\lbl[b]{2,26;$\bi$}
\lbl[wl]{15,25;$\tan$}
\end{lpic}
\end{wrapfigure}

It remains to show that the map $(a,b) \mapsto (\frac{a}{a^2+b^2}, \frac{b}{a^2+b^2})$ sends bijectively the half plane $a>0$ onto itself.

\parbf{\ref{ex:beta-from-tau+nu}.} By the product rule, we get
\begin{align*}
\bi'&=(\tan\times \norm)'=
\tan'\times \norm+\tan\times\norm'.
\end{align*}
It remains to substitute the values from \ref{eq:frenet-tau} and \ref{eq:frenet-nu} and simplify.



\parbf{\ref{ex:torsion=0}.}
This is a consequence of the equation $\bi' = - \tor\cdot \norm $.

\parbf{\ref{ex:frenet}.} Observe that $\tfrac{\gamma'\times\gamma''}{|\gamma'\times\gamma''|}$ is a unit vector perpendicular to the plane spanned by $\gamma'$ and $\gamma''$, so, up to sign, it has to be equal to $\bi$.
It remains to check that the sign is right.

\parbf{\ref{ex:lancret}}; \ref{SHORT.ex:lancret:a}.
Observe that 
$\langle \vec w,\tan\rangle'=0$.
Show that it implies that $\langle \vec w, \norm\rangle =0$.

By Frenet formulas,  
$\langle \vec w,\norm\rangle'=0$
 implies that 
$\langle \vec w, -\kur\cdot\tan+\tor\cdot \bi\rangle =0$.

\parit{\ref{SHORT.ex:lancret:b}.}
Show that $\vec w'=0$;
it implies that $\langle \vec w,\tan\rangle=\tfrac\tor\kur$.
In particular, the velocity vector of $\gamma$ makes a constant angle with $\vec w$; that is, $\gamma$ has constant slope.

\parbf{\ref{ex:evolvent-constant-slope}.}
Suppose $\alpha$ is an evolvent of $\gamma$ and $\vec w$ is a fixed vector.
Show that $\langle \vec w,\alpha\rangle$ is constant if $\gamma$ makes constant angle with $\vec w$.

\parbf{\ref{ex:spherical-frenet}}.
Part \ref{SHORT.ex:spherical-frenet:tau} follows from the fact that $(  \tan , \norm, \bi  )$ is an orthonormal basis. For \ref{SHORT.ex:spherical-frenet:nu} take the first and second derivatives of the identity $\langle\gamma,\gamma\rangle=1$ and simplify using the Frenet formulas.
Part \ref{SHORT.ex:spherical-frenet:beta} follows from \ref{SHORT.ex:spherical-frenet:nu} and the Frenet formulas.
By \ref{SHORT.ex:spherical-frenet:beta}, $\int\tfrac\tor\kur=0$, hence \ref{SHORT.ex:spherical-frenet:beta+} follows.
Part \ref{SHORT.ex:spherical-frenet:kur-tor} is proved by algebraic manipulations.

\parit{\ref{SHORT.ex:spherical-frenet:f}.}
Use the Frenet formulas to show that $(\gamma+\tfrac1\kur\cdot \norm+\tfrac{\kur'}{\kur^2\cdot\tor}\cdot\bi)'=0$.



\parbf{\ref{ex:cur+tor=helix}.} Use the second statement in \ref{ex:helix-torsion}.

\parbf{\ref{ex:const-dist}.} Note that the function
\[\rho(\ell)=|\gamma(t+\ell)-\gamma(t)|^2=\langle \gamma(t+\ell)-\gamma(t),\gamma(t+\ell)-\gamma(t)\rangle\] 
is smooth and does not depend on $t$.
Express speed, curvature and torsion of $\gamma$ in terms of the derivatives $\rho^{(n)}(0)$.
Be patient, you will need two derivatives for the speed,
four for the curvature and six for the torsion.
Once it is done, apply \ref{ex:cur+tor=helix}.
