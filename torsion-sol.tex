\parbf{\ref{ex:helix-torsion}.} 
The arc-length parameter $s$ is already found in   \ref{ex:arc-length-helix}.
It remains to find Frenet frame and calculate curvature and torsion.
The latter can be done by straightforward calculations.

\begin{wrapfigure}{r}{20 mm}
\vskip0mm
\centering
\begin{lpic}[t(-0mm),b(0mm),r(0mm),l(0mm)]{asy/helix(1)}
\lbl[br]{8,24;$\norm$}
\lbl[b]{2,26;$\bi$}
\lbl[wl]{15,25;$\tan$}
\end{lpic}
\end{wrapfigure}

If the calculations done right, then you should see that curvature $\kur$ and torsion $\tau$ do not depend on time and given.
Moreover for any $\kur>0$ and $\tor$ one can find $a$ and $b$ so that the helix $\gamma_{a,b}$ has curvature $\kur$ and torsion~$\tor$.

One may also see it geometrically using that the helix is maped to itself by one-parameter family of glide rotations around $z$-axis.
Therefore, for the $t$-parametrization, Frenet frame rotates around $z$-axis with the angular velocity~$1$.
It remains rewrite it for the arc-length parametrization and note that 
\begin{align*}
\tan(0)&=(0,\cos\theta, \sin\theta),
&
\norm(0)&=(-1,0,0),
&
\bi(0)&=(0,\sin\theta, -\cos\theta),
\end{align*}
where $\tg\theta\z=b/a$ if $a>0$. 

\parbf{\ref{ex:beta-from-tau+nu}.} By product rule, we get
\begin{align*}
\bi'&=(\tan\times \norm)'=
\tan'\times \norm+\tan\times\norm'.
\end{align*}
It remains to substitute the values from \ref{eq:frenet-tau} and \ref{eq:frenet-nu} and simplify.



\parbf{\ref{ex:torsion=0}.}
Show and use that the binormal vector is constant.

\parbf{\ref{ex:frenet}.} Observe that $\tfrac{\gamma'\times\gamma''}{|\gamma'\times\gamma''|}$ is a unit vector perpendicular to the plane spanned by $\gamma'$ and $\gamma''$, so, up to sign, it has to be equal to $\bi$.
It remains to check that the sign is right.

\parbf{\ref{ex:lancret}}, \ref{SHORT.ex:lancret:a}.
Observe that 
$\langle \vec w,\tan\rangle'=0$.
Show that it implies that
\[\langle \vec w, \norm\rangle =0.\]

Further observe that 
$\langle \vec w,\norm\rangle'=0$.
Show that it implies that 
\[\langle \vec w, -\kur\cdot\tan+\tor\cdot \bi\rangle =0.\]

\parit{\ref{SHORT.ex:lancret:a}.}
Show that $\vec w'=0$;
it implies that $\langle \vec w,\tan\rangle=\tfrac\tor\kur$.
In particular, the velocity vector of $\gamma$ makes a constant angle with $\vec w$; that is, $\gamma$ has constant slope.

\parbf{\ref{ex:evolvent-constant-slope}.}
Show that $\langle w,\alpha\rangle$ is constant if $\gamma$ makes constant angle with a fixed vector $w$ and $\alpha$ is the evolvent of $\gamma$.

\parbf{\ref{ex:spherical-frenet}.}
Suppose $\langle \vec w,\tan\rangle$ is a constant.
Show that $\langle \vec w,\alpha\rangle'=0$.
It follows that $\langle \vec w,\alpha\rangle$ is a constant,
so $\alpha$ lies in a plane perpendicular to $\vec w$.


\parbf{\ref{ex:cur+tor=helix}.} Use the second statement in \ref{ex:helix-torsion}.

\parbf{\ref{ex:const-dist}.} Note that the function
\[\rho(\ell)=|\gamma(t+\ell)-\gamma(t)|^2=\langle \gamma(t+\ell)-\gamma(t),\gamma(t+\ell)-\gamma(t)\rangle\] 
is smooth and does not depend on $t$.
Express speed, curvature and torsion of $\gamma$ in terms of derivatives $\rho^{(n)}(0)$
and apply \ref{ex:cur+tor=helix}.
