\appendix
\chapter{Semisolutions}

\parbf{Exercise~\ref{ex:dido-isop}.}
First let us show that Dido's problem follows from the isoperimetric inequality.

Assume $F$ is a figure bounded by a straight line and a curve of length $\ell$ whose endpoints belong to that line. 
Let $F'$ be the reflection of $F$ in the line.
Note that the union $G=F\cup F'$ is a figure bounded by a closed curve of length~$2\cdot\ell$.

Applying the isoperimetric inequality, we get that the area of $G$ can not exceed the area of round disc with the same circumference $2\cdot\ell$
and the equality holds only if the figure is congruent to the disc.
Since $F$ and $F'$ are congruent, Dido's problem follows.

Now let us show that the isoperimetric inequality follows from the Dido's problem.

Assume $G$ is a convex figure bounded by a closed curve of length $2\cdot\ell$.
Cut $G$ by a line that splits the perimeter in two equal parts --- $\ell$ each.
Denote by $F$ and $F'$ the two parts.
Applying the Dido's problem for each part, we get that that are of each does not exceed the area of half-disc bounded by a half-circle.
The two half-disc could be arranged into a round disc of circumference $\ell$, hence the isoperimetric inequality follows.
