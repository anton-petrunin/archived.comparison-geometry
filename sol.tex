\appendix
\chapter{Semisolutions}

\parbf{Exercise~\ref{ex:dido-isop}.}
First let us show that Dido's problem follows from the isoperimetric inequality.

Assume $F$ is a figure bounded by a straight line and a curve of length $\ell$ whose endpoints belong to that line. 
Let $F'$ be the reflection of $F$ in the line.
Note that the union $G=F\cup F'$ is a figure bounded by a closed curve of length~$2\cdot\ell$.

Applying the isoperimetric inequality, we get that the area of $G$ can not exceed the area of round disc with the same circumference $2\cdot\ell$
and the equality holds only if the figure is congruent to the disc.
Since $F$ and $F'$ are congruent, Dido's problem follows.

Now let us show that the isoperimetric inequality follows from the Dido's problem.

Assume $G$ is a convex figure bounded by a closed curve of length $2\cdot\ell$.
Cut $G$ by a line that splits the perimeter in two equal parts --- $\ell$ each.
Denote by $F$ and $F'$ the two parts.
Applying the Dido's problem for each part, we get that that are of each does not exceed the area of half-disc bounded by a half-circle.
The two half-disc could be arranged into a round disc of circumference $\ell$, hence the isoperimetric inequality follows.




\parbf{Exercise~\ref{ex:fenchel}.} Assume contrary, that is there is a closed smooth regular curve $\alpha$ such that $\tc\alpha<2\cdot\pi$.

The tangent indicatrix $\tau$ of $\alpha$ is a curve in a sphere;
by the definition of total curvature, the length of $\tau$ is the total curvature of $\alpha$; in particular
\[\length \tau <2\cdot\pi.\]

By Exercise~\ref{ex:2pi-sphere}, $\tau$ lies in an open hemisphere.
If $u$ is the center of the hemisphere, then 
\[\langle u,\tau(t)\rangle>0\quad\text{and therefore}\quad \langle u,\alpha'(t)\rangle>0\]
for any $t$.
Therefore the function $t\mapsto \langle u,\alpha(t)\rangle$ is strictly increasing.
In particular, if $\alpha$ is defined on the time interval $[a,b]$, then
\[\langle u,\alpha(a)\rangle<\langle u,\alpha(a)\rangle.\]
But $\alpha$ is closed; that is $\alpha(a)=\alpha(b)$ --- a contradiction.

Now let us prove the equality case.
First note that it is sufficient to show that $\tau$ runs around an equator.

Assume $\tau$ is not an equator, from above we know that $\tau$ can not lie in an open hemisphere.
Note that we can shorten $\tau$ by a small chord.
The obtained curve $\tau'$ is shorter than $2\cdot\pi$ and therefore lies in an open hemisphere.
Applying this construction for shorter and shorter chord and passing to the limit we get that $\tau$ lies in closed hemisphere.
Denote its center by $u$ as before, then
\[\langle u,\tau(t)\rangle\ge 0\quad\text{and therefore}\quad \langle u,\alpha'(t)\rangle\ge0\]
for any $t$.
Since $\alpha$ is closed we have that $\langle u,\alpha(t)\rangle$ is constant;
that is, runs in a plane perpendicular to $u$ and $\tau$ lies in an equator perpendicular to $u$.

So $\tau$ is a curve that runs along equator, has length $2\cdot\pi$ and does not lie in a open hemisphere.
Since $\tau$ is not an equator, it have to run along half-equator back and forth.
In this case $\tau$ lies in an other closed hemisphere and has some points in its interior.
The latter contradicts closeness of $\alpha$ the same way as above. 





