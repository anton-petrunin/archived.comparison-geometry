\chapter{Length}
 
Recall that a sequence 
\[a=t_0 < t_1 < \cdots < t_k=b.\]
is called a \emph{partition} of the interval $[a,b]$.

\begin{thm}{Definition}\label{def:length}
The \emph{length}\index{length of curve} of a $\alpha\:[a,b]\to \mathcal{X}$ is defined as
\begin{align*}
\length \alpha
= 
\sup \{|\alpha(t_0)-\alpha(t_1)|&+|\alpha(t_1)-\alpha(t_2)|+\dots
\\
&\dots+|\alpha(t_{k-1})-\alpha(t_k)|\}. 
\end{align*}
where the exact upper bound is taken over all partitions
\[a=t_0 < t_1 < \cdots < t_k=b.\]

The length of $\alpha$ is a nonnegative real number or infinity;
the curve $\alpha$ is called \emph{rectifiable}\index{rectifiable curve} if its length is finite.
\end{thm}

If $\alpha$ is a space curve, then the above definition says that it length is the exact upper bound of the lengths of polygonal lines $p_0\dots p_k$ \emph{inscribed} in the curve, where $p_i=\alpha(t_i)$ for a  partition $a=t_0 < t_1 < \cdots < t_k=b$.

\begin{thm}{Exercise}\label{ex:integral-length}
Assume $\alpha\:[a,b]\to\RR^2$ is a smooth curve.
Show that
\begin{enumerate}[(a)]
\item\label{ex:integral-length>} $\length \alpha\ge \int_a^b|\alpha'(t)|\cdot dt$,
\item\label{ex:integral-length<} $\length \alpha\le \int_a^b|\alpha'(t)|\cdot dt$.
\end{enumerate}
Conclude that 
\[\length \alpha= \int_a^b|\alpha'(t)|\cdot dt.\]
\end{thm}

\parit{Hints:} For (\ref{ex:integral-length>}), apply the fundamental theorem of calculus for each segment in a given partition. For (\ref{ex:integral-length<}) consider a partition such that $\alpha'$ is nearly constant on each of its segments.

\begin{thm}{Exercise}\label{ex:nonrectifiable-curve}
Construct a nonrectifiable simple curve $\alpha\:[0,1]\z\to\RR^2$.
\end{thm}

\section*{Convex curves}

A closed simple plane curve is called \emph{convex} if it bounds a convex region.

\begin{thm}{Proposition}\label{prop:convex-curve}
Assume a convex figure $A$ bounded by a curve $\alpha$ lies inside a figure $B$ bounded by a curve $\beta$.
Then
\[\length\alpha\le \length\beta.\]
\end{thm}

Note that it is sufficient to show that for any polygon  $P$ inscribed in $\alpha$ there is a polygon $Q$ inscribed in $\beta$ with 
$\perim P\le \perim Q$, where $\perim P$ denotes the perimeter of $P$.

Therefore it is sufficient to prove the following lemma.


\begin{thm}{Lemma}\label{lem:perimeter}
Let $P$ and $Q$ be polygons.
Assume $P$ is convex and $Q\supset P$.
Then $\perim P\le \perim Q$.
\end{thm}


\begin{wrapfigure}{r}{24 mm}
\vskip-4mm
\centering
\includegraphics{mppics/pic-7}
%\caption*{}
\end{wrapfigure}

\parit{Proof.}
Note that by the triangle inequality,
the inequality
\[\perim P\le \perim Q\]

holds
if $P$ can be obtained from $Q$ by cutting it along a chord;
that is, a line segment with ends on the boundary of $Q$ that lies in $Q$.


Note that there is an increasing sequence of polygons 
$$P=P_0\subset P_1\subset\dots\subset P_n=Q$$
such that $P_{i-1}$ obtained from $P_{i}$ by cutting along a chord.
Therefore 
\begin{align*}
\perim P=\perim P_0&\le\perim P_1\le\dots
\\
\dots&\le\perim P_n=\perim Q
\end{align*}
and the lemma follows.
\qeds

\begin{thm}{Corollary}
Any convex closed curve is rectifiable.  
\end{thm}

\parit{Proof.}
Any closed curve is bounded; that is, it lies in a sufficiently large square.


By Proposition~\ref{prop:convex-curve}, the length of the curve can not exceed the perimeter of the square, hence the result.
\qeds



\section*{Semicontinuity of length}

Recall that the lower limit 
of a sequence of real numbers $(x_n)$ is denoted by
\[\liminf_{n\to\infty} x_n.\] 
It is defined as the lowest partial limit; that is, the lowest possible limit of a subsequence of $(x_n)$.
The lower limit is defined for any sequence of real numbers and it lies in the exteded real line $[-\infty,\infty]$


\begin{thm}{Theorem}\label{thm:length-semicont}
Length is a lower semi-continuous with respect to pointwise convergence of curves. 

More precisely, assume that a sequence
of curves $\alpha_n\:[a,b]\to \RR^2$ converges pointwise 
to a curve $\alpha_\infty\:[a,b]\to \RR^2$;
that is, $\alpha_n(t)\z\to\alpha_\infty(t)$ for any fixed $t\in[a,b]$ as $n\to\infty$. 
Then 
$$\liminf_{n\to\infty} \length\alpha_n \ge \length\alpha_\infty.\eqlbl{eq:semicont-length}$$
\end{thm}



\begin{wrapfigure}{r}{20 mm}
\vskip-0mm
\centering
\includegraphics{mppics/pic-6}
\end{wrapfigure}


Note that the inequality \ref{eq:semicont-length} might be strict.
For example the diagonal $\alpha_\infty$ of the unit square 

can be  approximated by a sequence of stairs-like
polygonal curves $\alpha_n$
with sides parallel to the sides of the square ($\alpha_6$ is on the picture).
In this case
\[\length\alpha_\infty=\sqrt{2}\quad
\text{and}\quad \length\alpha_n=2\]
for any $n$.

\parit{Proof.}
Fix a partition $a=t_0<t_1<\dots<t_k=b$.
Set 
\begin{align*}\Sigma_n
&\df
|\alpha_n(t_0)-\alpha_n(t_1)|+\dots+|\alpha_n(t_{k-1})-\alpha_n(t_k)|.
\\
\Sigma_\infty
&\df
|\alpha_\infty(t_0)-\alpha_\infty(t_1)|+\dots+|\alpha_\infty(t_{k-1})-\alpha_\infty(t_k)|.
\end{align*}

Note that $\Sigma_n\to \Sigma_\infty$ as $n\to\infty$
and $\Sigma_n\le\length\alpha_n$ for each $n$.
Hence
$$\liminf_{n\to\infty} \length\alpha_n \ge \Sigma_\infty.\eqlbl{>=Sigma-infty}$$

If $\alpha_\infty$ is rectifiable, we can assume that 
\begin{align*}
\length\alpha_\infty<
|\alpha_\infty(t_0)-\alpha_\infty(t_1)|+\dots+|\alpha_\infty(t_{k-1})-\alpha_\infty(t_k)|+\eps.
\end{align*}
for any given $\eps>0$.
By \ref{>=Sigma-infty} it follows that 
$$\liminf_{n\to\infty} \length\alpha_n > \length\alpha_\infty-\eps$$
for any $\eps>0$; whence \ref{eq:semicont-length} follows.

It remains to consider the case when $\alpha_\infty$ is not rectifiable; 
that is $\length\alpha_\infty=\infty$.
In this case we can choose a partition so that $\Sigma_\infty>L$ for any real number $L$.
By \ref{>=Sigma-infty} it follows that 
$$\liminf_{n\to\infty} \length\alpha_n > L$$
for any $L$; whence $\liminf_{n\to\infty}=\infty$ and \ref{eq:semicont-length} follows.
\qeds

\section*{Length metric}

Let $\spc{X}$ be a metric space.
Given two points $x,y$ in $\spc{X}$, denote by $d(x,y)$ the exact lower bound for lengths of all paths connecting $x$ to $y$; if there is no such path we assume that $d(x,y)=\infty$.
Note that function $d$ satisfies all the axioms of metric except it might take infinite value.

Therefore if any two points in $\spc{X}$ can be connected by a rectifiable curve then $d$ defines a new metric on it; in this case $d$ is called \emph{induced length metric}.

Evidently $d(x,y)\ge |x-y|$ for any pair of points $x,y\in \spc{X}$.
If the equality holds for any pair, then then the metric is called \emph{length metric} and the space is called \emph{length-metric space}.

Note that the Euclidean space is a length-metric space.
Most of the time we consider length-metric spaces.
A subspaces $A$ of length-metric space $\spc{X}$ might be not a lenght-metric space;
the induced length distance between points $x$ and $y$ in the subspace $A$ will be denoted as $|x-y|_A$;
that is $|x-y|_A$ is the exact lower bound for the length of paths in $A$.

\begin{thm}{Exercise}\label{ex:intrinsic-convex}
Let $A\subset \RR^3$ be a closed subset.
Show that $A$ is convex if and only if
\[|x-y|_A=|x-y|_{\RR^3}.\]
\end{thm}

\begin{thm}{Exercise}
Let us denote by $\SS^2$ the unit sphere in the space; that is,
\[\SS^2=\set{(x,y,z)\in\RR^3}{x^2+y^2+z^2=1}.\]
Show that
\[|u-v|_{\SS^2}=\measuredangle(u,v)\df\arccos\langle u,v\rangle.\]

\end{thm}

\parit{Hint:} Use the following map $f\:(r,\theta,\phi)\mapsto (r,\theta,0)$ in spherical coordinates. Note that $f$ a distance nonexpanding map that maps $\RR^3$ to a half-plane and $\SS^2$ to one of its meridians.




