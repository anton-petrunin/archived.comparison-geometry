
\chapter{Spherical map}

\section*{Differential}

Let $f\:\Sigma\to \RR^3$ be a smooth map defined on a smooth surface $\Sigma$;
that is, for any chart $s$ of $\Sigma$ the composition $f\circ s$ is smooth.

Given a smooth curve $\gamma$ in $\Sigma$ consider the smooth curve $\hat \gamma=f\circ\gamma$.
Assume $\gamma$ starts at a point $p\in \Sigma$ with velocity vector $v\in\T_p\Sigma$;
that is, $p=\gamma(0)$ and $v=\gamma'(0)$.
The differential of $v$ at $p$ is defined as 
\[d_pf(v)=\hat\gamma'(0).\eqlbl{eq:differenital}\]
The domain of definition of $d_pf$ is the tangent plane $\T_p\Sigma$.
The differential is an operator that produce a map $d_pf\:\T_p\Sigma\to\RR^3$ for given smooth map $f\:\Sigma\to \RR^3$ and $p\in\Sigma$.

Note that the value $d_pf(v)$ does not depend on the choice of $\gamma$;
that is, if $\gamma_1$ is another curve in $\Sigma$ such that $\gamma_1(0)=p$ and $\gamma'_1(0)=v$,
then 
\[(f\circ\gamma)'(0)=(f\circ\gamma_1)'(0).\eqlbl{eq:df(v)}\]
Indeed,  $v=\gamma'_1(0)=\gamma'(0)$; therefore we have that 
\[|\gamma_1(\eps)-\gamma(\eps)|=o(\eps).\]
Since $f$ is smooth,
\[|f\circ\gamma_1(\eps)-f\circ\gamma(\eps)|=o(\eps);\]
whence \ref{eq:df(v)} follows.

Note that if $\Sigma$ is a plane, then
\[d_pf(v)=D_vf,\]
where $D_v$ denotes the direction derivative.
For general surface $D_vf$ is not well defined since the point $p+t\cdot v$ may not lie in $\Sigma$ for small values $t$.

\begin{thm}{Exercise}\label{ex:differential-range}
Assume $f$ is a smooth map from one surface $\Sigma_0$ to another $\Sigma_1$ and $p\in \Sigma_1$.
Show that the range of $d_pf$ lies in the tangent plane $\T_{f(p)}\Sigma_1$.
\end{thm}

\begin{thm}{Proposition}
The differential is a linear map.
That is, for any smooth map $f\:\Sigma\to \RR^3$ defined on a smooth surface $\Sigma$ and $p\in\Sigma$,
the map $d_pf\:\T_p\to \RR^3$ is linear.
\end{thm}

\parit{Proof.}
Fix a chart $(u,v)\mapsto s(u,v)$ on $\Sigma$ that covers a neighborhood of $p$.
Without loss of generality we may assume that $p=s(0,0)$.

Any smooth curve $\gamma$ that starts at $p$ can be written locally in the chart as $\gamma(t)=s(u(t),v(t))$;
since $\gamma$ stats at $p$, we have $u(0)=v(0)=0$.
Applying the chain rule, we get
\begin{align*}
\gamma'(0)&=\tfrac{\partial s}{\partial u}(0,0)\cdot u'(0)+\tfrac{\partial s}{\partial v}(0,0)\cdot v'(0),
\\
(f\circ\gamma)'(0)&=\tfrac{\partial f\circ s}{\partial u}(0,0)\cdot u'(0)+\tfrac{\partial f\circ s}{\partial v}(0,0)\cdot v'(0).
\end{align*}
The statement follows since $d_p(\gamma'(0))=(f\circ\gamma)'(0)$.
\qeds

\section*{A more conceptual way*}

In this section we introduce a more conceptual way to define tangent vectors.
We will not use this approach in the sequel, but it is better to know about it earlier.

A tangent vector $v\in \T_p$ to a smooth surface $\Sigma$ is a linear functional%
\footnote{Term \emph{functional} is used for functions that take a function as an argument and return a number.}
that takes a smooth function $\phi$ on $\Sigma$, spits a real number denoted by $v\phi$ and satisfies the product rule:
\[v(\phi\cdot\psi)=(v\phi)\cdot \psi(p)+\phi(p)\cdot(v\psi).
\eqlbl{eq:tangent-functional}\]

If $v$ is a velocity vector of a smooth curve $\gamma$ that starts at $p$, then you can set
\[v\phi=(\phi\circ\gamma)'(0).\eqlbl{eq:curve-functional}\]
It follows that any velocity vector $v=\gamma'(0)$ is also a tangent vector in the new definition.
It is not hard to check the converse as well; that is, to any linear functional $v$ satisfying the product rule there is a curve $\gamma$ such that \ref{eq:curve-functional} holds.

The new definition is less intuitive, but it is more convenient to use since it grabs the key algebraic property of tangent vectors.
For example the differential could be defined using the following identity:
\[(d_pf(v))\phi\df v(\phi\circ f);\]
that is, the result of the functional $d_pf(v)$ on the function $\phi\:\RR^3\to\RR$ is defined to be the same as the result of the functional $v$ on the function $\phi\circ f$.

\section*{Spherical map}

Let $\Sigma$ be a smooth oriented surface with unit normal field $\nu$.
The map $\nu\:\Sigma\to \SS^2$ defined by $p\mapsto \nu_p$ is called \emph{spherical map} or \emph{Gauss map}.

Recall that if $(u,v)\mapsto s(u,v)$ is a chart on $\Sigma$ then 
\[\nu(u,v)=\pm \frac{\frac{\partial s}{\partial u}\times \frac{\partial s}{\partial v}}{|\frac{\partial s}{\partial u}\times \frac{\partial s}{\partial v}|}.\]
Since $s$ is a chart $\tfrac{\partial s}{\partial u}\times \tfrac{\partial s}{\partial v}\ne 0$,
therefore the spherical map is smooth.

In particular the differential of the spherical map is defined at any point of $\Sigma$.

\section*{Shape operator}

Suppose $\Sigma$ is an oriented surface with the unit normal field $\nu$;
in other words $\nu\:\Sigma\to\SS^2$ is its spherical map.

Fix a point $p\in \Sigma$.
The \emph{shape operator} at $p$ is defined as 
\[S_p\df-d_p\nu.
\eqlbl{eq:shape}\]
The shape operator $S_p$ is defined on the tangent plane $\T_p\Sigma$ and it returns a vector in the sames plane (otherwise we could not call it \emph{operator}).
The latter is shown in the following proposition, which also gives a reason for the change of sign in \ref{eq:shape}.

\begin{thm}{Proposition}
Suppose $\Sigma$ is an oriented surface.
Then for any $p\in \Sigma$ and any $v\in \T_p$ we have that $S_p(v)\in\T_p$.
Moreover 
\[\langle S_p(v),w\rangle=\langle S_p(w),v\rangle=\II_p(v,w)\]
for any $v,w\in \T_p$.
\end{thm}

\parit{Proof.}
Assume an oriented surface $\Sigma$ is written locally as a graph $z\z=f(x,y)$ in the tangent-normal coordinates at $p\in\Sigma$.
As usual we assume that the normal vector $\nu_p$ points in the direction of $z$-axis,
in this case the normal vector at any point of the graph points up; that is, its $z$-coordinate  is positive.

Consider the corresponding chart  of $\Sigma$:
\[s(x,y)\z=(x,y,f(x,y)).\]
Denote by $\nu(x,y)$ the unit normal vector at $s(x,y)$.

Note that for a tangent vector $v=(a,b,0)\in \T_p$ we have that
\[
\begin{aligned}
S_p(v)&=-D_v\nu(0,0)=
\\
&=-(a\cdot \tfrac{\partial \nu}{\partial x}+b\cdot \tfrac{\partial \nu}{\partial y})(0,0),
\end{aligned}
\eqlbl{eq:S=D}
\]
where $D_v$ denotes the directional derivative along a vector $v$ in the $(x,y)$-plane which is $\T_p$.

Indeed, the first equality follows from \ref{eq:shape} applied for the curve $\gamma(t)=(a\cdot t,b\cdot t, f(a\cdot t,b\cdot t))$ at $t=0$ and the second follow from the chain rule.

Taking partial derivatives of $\langle\nu,\nu\rangle=1$, we get that 
\[\langle\tfrac{\partial\nu}{\partial x},\nu\rangle=\langle\tfrac{\partial\nu}{\partial x},\nu\rangle=0\]
By \ref{eq:S=D} it follows that $S_p(v)\perp \nu_p$, or equivalently $S_p(v)\in\T_p$.

Further, since $\tfrac{\partial s}{\partial x}, \tfrac{\partial s}{\partial y}\in\T_{s(x,y)}\Sigma$
and $\nu(x,y)\perp\T_{s(x,y)}\Sigma$,
we have that
\[\langle \nu,\tfrac{\partial s}{\partial x}\rangle\equiv 0
\quad\text{and}\quad
\langle \nu,\tfrac{\partial s}{\partial y}\rangle\equiv 0.\]
Taking a derivative of these identities, we get that
\[\begin{aligned}
\langle \tfrac{\partial\nu}{\partial x},\tfrac{\partial s}{\partial x}\rangle+\langle \nu,\tfrac{\partial^2 s}{\partial x^2}\rangle&\equiv 0,
\\
\langle \tfrac{\partial\nu}{\partial y},\tfrac{\partial s}{\partial x}\rangle+\langle \nu,\tfrac{\partial^2 s}{\partial y\partial x}\rangle&\equiv 0,
\\
\langle \tfrac{\partial\nu}{\partial x},\tfrac{\partial s}{\partial y}\rangle+\langle \nu,\tfrac{\partial^2 s}{\partial x\partial y}\rangle&\equiv 0,
\\
\langle \tfrac{\partial\nu}{\partial y},\tfrac{\partial s}{\partial y}\rangle+\langle \nu,\tfrac{\partial^2 s}{\partial y^2}\rangle&\equiv 0,
\end{aligned}
\eqlbl{eq:shape=second}
\]

Fix two vectors $v=(a,b,0)$ and $w=(c,d,0)$ in $\T_p$ (which is the $(x,y)$-plane).
Since $\nu(0,0)=(0,0,1)$ we get 
\[f(x,y)\equiv\langle\nu(0,0),s(x,y)\rangle.\]
Therefore by \ref{eq:S=D}, \ref{eq:shape=second} and \ref{eq:DwDv} we get that 
\begin{align*}
\langle S_p(v),w\rangle 
&=-\langle  a\cdot \tfrac{\partial \nu}{\partial x}+b\cdot \tfrac{\partial \nu}{\partial y},c\cdot \tfrac{\partial s}{\partial x}+d\cdot \tfrac{\partial s}{\partial y}\rangle(0,0)=
\\
&=-\bigl(a\cdot c\cdot\langle \tfrac{\partial \nu}{\partial x},\tfrac{\partial s}{\partial x}\rangle 
+a\cdot d\cdot\langle \tfrac{\partial \nu}{\partial x},\tfrac{\partial s}{\partial y}\rangle+
\\&\quad
+b\cdot c\cdot\langle \tfrac{\partial \nu}{\partial y},\tfrac{\partial s}{\partial x}\rangle
+b\cdot d\cdot\langle \tfrac{\partial \nu}{\partial y},\tfrac{\partial s}{\partial y}\rangle\bigr)(0,0)=
\\
&=\bigl(a\cdot c\cdot\langle \nu,\tfrac{\partial^2 s}{\partial x^2}\rangle 
+a\cdot d\cdot\langle \nu,\tfrac{\partial^2 s}{\partial x\partial y}\rangle+
\\&\quad
+b\cdot c\cdot\langle \nu,\tfrac{\partial^2 s}{\partial y\partial x}\rangle
+b\cdot d\cdot\langle \nu,\tfrac{\partial^2 s}{\partial y^2}\rangle\bigr)(0,0)=
\\
&=\bigl(a\cdot c\cdot\tfrac{\partial^2 f}{\partial x^2} 
+a\cdot d\cdot\tfrac{\partial^2 f}{\partial x\partial y}+
\\&\quad
+b\cdot c\cdot\tfrac{\partial^2 f}{\partial y\partial x}
+b\cdot d\cdot\tfrac{\partial^2 f}{\partial y^2}\bigr)(0,0)=
\\
&=\II_p(v,w).
\end{align*}
It remains to apply \ref{eq:II=II}.
\qeds

Recall that 
\[\II_p(w,v)=\langle M_p\cdot v,w\rangle,\]
where $M_p$ is the Hessian matrix at $p$.
It follows that 
\[S_p(v)=M_p\cdot v\]
if $v$ is written in the standard basis of $(x,y)$-plane.
Whence we get the following theorem. 

\begin{thm}{Theorem}\label{thm:rodrigues}
Let $\Sigma$ be a smooth oriented surface and $p\in \Sigma$.
A nonzero tangent vector $v\in \T_p$ points in a principle direction at $p$
if and only if $S(v)\parallel v$ and if so, then the unique coefficient $k$ such that
$S(v)=k\cdot v$ is the principle curvature in this direction.
\end{thm}


\begin{thm}{Exercise}\label{ex:normal-curvature=const}
Let $\Sigma$ be a smooth oriented surface with the unit normal field $\nu$.
Suppose that $\Sigma$ has unit normal curvature at any point in any direction.
\begin{enumerate}[(a)]
 \item Show that $S_p(w)=w$ for any $p\in\Sigma$ and $w\in \T_p\Sigma$.
 \item Show that $p+\nu_p$ is constant; that is the point $c=p+\nu_p$ does not depend on $p\in\Sigma$.
 Conclude that $\Sigma$ is a part of the unit sphere centered at $c$.
\end{enumerate}

\end{thm}


\begin{thm}{Exercise}\label{ex:shape-curvature-line}
Assume that smooth surfaces $\Sigma_1$ and $\Sigma_2$ intersect at constant angle along a smooth regular curve $\gamma$.
Show that if $\gamma$ is a curvature line in $\Sigma_1$ then it is also a curvature line in $\Sigma_2$.

Conclude that if a smooth surface $\Sigma$ intersects a plane or sphere along a smooth curve $\gamma$,
then $\gamma$ is a curvature line of $\Sigma$.
\end{thm}

\parit{Hint:}  Denote by $\nu_1(t)$ and $\nu_2(t)$ the unite normal vectors to $\Sigma_1$ and $\Sigma_2$ at $\gamma(t)$.
Note that $\langle \nu_1(t),\nu_2(t)\rangle$ is constant; take it derivative and apply \ref{thm:rodrigues}.

