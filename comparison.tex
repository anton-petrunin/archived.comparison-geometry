\chapter{Global comparison}

\section{Formulation}

A minimizing geodesic between points $x$ and $y$ in a surface $\Sigma$ will be denoted as $[xy]$ or $[xy]_\Sigma$;
the latter notation is used if we need to emphasise that the geodesic lies in $\Sigma$.
If we write $[xy]$, then we assume that a minimizing geodesic exists and we made a choice of one of them.
If $\Sigma$ is complete, then a minimizing geodesic always exists. %%%???should we prove it???

A \emph{geodesic triangle} in a surface $\Sigma$ is a triple of points $x,y,z\in \Sigma$ with choice of minimizing geodesics $[xy]$, $[yz]$ and $[zx]$.
The points $x,y,z$ are called \emph{vertexes} of the geodesic triangle,
the minimizing geodesics $[xy]$, $[yz]$ and $[zx]$ are called its sides;
the triangle itself is denoted by $[xyz]$.
(The notation $[xyz]$ is just a shortcut for the array $(x,y,z,[xy], [yz], [zx])$.)

The length of one (and therefore any) minimizing geodesic $[xy]_\Sigma$ will be denoted by $|x-y|_\Sigma$; it is called \emph{intrinsic distance} from $x$ to $y$ in $\Sigma$.
If defined, then $|x-y|_\Sigma$ is the exact lower bound on the lengths of curves from $x$ to $y$ in $\Sigma$. 

A triangle $[\~x\~y\~z]$ in the plane is called \emph{model triangle} of the triangle $[xyz]$,
briefly $[\~x\~y\~z]=\~\triangle xyz$, if its corresponding sides are equal;
that is,
\[|\~x-\~y|=|x-y|_\Sigma,
\quad
|\~y-\~z|=|y-z|_\Sigma,
\quad
|\~z-\~x|=|z-x|_\Sigma.
\]

A pair of minimizing geodesics $[xy]$ and $[xz]$ starting from one point $x$ is called \emph{hinge} and denoted as $\hinge xyz$.
The angle between these geodesics at $x$ is denoted by $\measuredangle\hinge xyz$.
The corresponding angle $\measuredangle\hinge {\~x}{\~y}{\~z}$ in the model triangle $[\~x\~y\~z]=\~\triangle xyz$ is denoted by $\angk xyz$.

A surface $\Sigma$ is called \emph{simply connected} if any closed simple curve in $\Sigma$ bounds a disc.
Equivalently any closed curve in $\Sigma$ can be continuously deformed into a trivial curve (trivial means that it stays at one point).
A plane or sphere are examples of simply connected surfaces, while torus or cylinder are not simply connected.


\begin{thm}{Comparison theorem}\label{thm:comp}
Let $\Sigma$ be a complete smooth regular surface with a geodesic triangle $[xyz]$.
\begin{enumerate}[(i)]
 \item\label{thm:comp:toponogov} If $\Sigma$ has nonnegative Gauss curvature, then 
 \[\measuredangle\hinge {x}{y}{z}\ge\angk xyz.\]
 \item\label{thm:comp:cat} If $\Sigma$ is simply connected and has nonpositive Gauss curvature,
 then 
 \[\measuredangle\hinge {x}{y}{z}\le\angk xyz.\]
\end{enumerate}

\end{thm}


Let us make few remarks on the formulation.

The angle $\measuredangle\hinge {x}{y}{z}$ is a number in the interval $[0,\pi]$.
If the triangle $[xyz]$ bounds a disc $\Delta$ and $\theta$ is the external angle at  $x$ which used in Gauss--Bonnet formula, 
then $\measuredangle\hinge {x}{y}{z}=|\pi-\theta|$.
The corresponding internal angle might be $\measuredangle\hinge {x}{y}{z}$ or $2\cdot\pi-\measuredangle\hinge {x}{y}{z}$ depending on which side lies the disc $\Delta$.

Since the angles of any plane triangle sum up to $\pi$,
the part (\ref{thm:comp:toponogov}) of the theorem implies that angles of any triangle in a surface with nonnegative Gauss curvature have sum at least $\pi$.
If the triangle bounds a disc, then by Gauss--Bonnet formula the sum of its internal angles is at least $\pi$.
Note that the triangle may not bound a disc, for example equator on the cyclinder is formed by a geodesic triangle that does not bound a disc.
Also note that (1) Gauss--Bonnet formula gives a lower bound on the sum of its \emph{internal} angles, but does not bound each angle separately (2) if $\alpha$ is the angle in the comparison theorem, then the internal angle might be $\alpha$ or $2\cdot\pi-\alpha$; while Gauss--Bonnet formula gives a lower bound on the sum of internal angles it does not forbid that each of these angles is close to $2\cdot \pi$ which is impossible by the comparison theorem.

On the part (\ref{thm:comp:cat}).
First note that without condition that $\Sigma$ is simply connected, the statement does not hold.
For example the equator $z=0$ of infinite cylinder (which is not simply connected)
\[\set{(x,y,z)\in\RR^3}{x^2+y^2=1}\]
is formed by a triangle with all angles $\pi$; which contradict the comparison.

\begin{thm}{Exercise}
Let $\Sigma$ be a complete smooth regular simply connected surface with nonpositive Gauss curvature.
Show that any two points in $\Sigma$ are connected by unique geodesic.
\end{thm}


\section{Names and history}

Part (\ref{thm:comp:toponogov}) of this theorem is called \emph{Toponogov comparison theorem};
it is was proved by Paolo Pizzetti \cite{pizzetti} and latter independently by Alexandr Alexandrov \cite{alexandrov}; generalizations were obtained by  Victor Toponogov \cite{toponogov}, Mikhael Gromov, Yuri Burago and Grigory Perelman \cite{BGP}.

Part (\ref{thm:comp:cat}) is called \emph{Cartan--Hadamard theorem};
it was proved by 
Hans von Mangoldt \cite{mangoldt} and generalized by Elie Cartan \cite{cartan}, Jacques Hadamard \cite{hadamard},
Herbert Busemann \cite{busemann},
Willi Rinow in \cite{rinow},
Mikhael Gromov \cite[p.119]{gromov},
Stephanie Alexander and Richard Bishop in \cite{a-b:h-c}.

\section{Local part}

First we prove the following local version of comparison theorem and then use it to prove the global version.

\begin{thm}{Theorem}
The comparison theorem (\ref{thm:comp}) holds in a small neighborhood of any point.

That is, if $\Sigma$ be a complete smooth regular surface,
then any point $p\in \Sigma$ admits a neighborhood $U$ such that 
\begin{enumerate}[(i)]
 \item If $\Sigma$ has nonnegative Gauss curvature, then for any geodesic triangle $[xyz]$ in $U$ we have
 \[\measuredangle\hinge {x}{y}{z}\ge\angk xyz.\]
 \item If $\Sigma$ has nonpositive Gauss curvature, then for any geodesic triangle $[xyz]$ in $U$ we have
 \[\measuredangle\hinge {x}{y}{z}\le\angk xyz.\]
\end{enumerate}
\end{thm}

\parit{Proof.}
Assume $y=\exp_xv$ and $z=\exp_xw$ for two small vectors $v,w\in\T_x$.
Note that 
\begin{align*}
\measuredangle \hinge xvw_{\T_x}&=\measuredangle \hinge xyz_\Sigma,
\\
|x-y|_\Sigma&=|x-v|_{\T_x}, 
\\
|x-z|_\Sigma&=|x-w|_{\T_x}.
\end{align*}


If the Gauss curvature is nonnegative,
consider the line segment $\~\gamma$ joining $v$ to $w$ in the tangent plane $\T_x$ and set $\gamma=\exp_x\circ\~\gamma$.
By Rauch comparison theorem (\ref{thm:rauch}), we have
\[\length\gamma\le \length \~\gamma.\]
Since $|v-w|_{\T_x}=\length\~\gamma$ and $|y-z|_{\Sigma}\le \length\gamma$, we get 
\[|v-w|_{\T_x}\ge |y-z|_\Sigma.\]
Therefore
\[\angk xzy\ge \measuredangle\hinge xzy.\]

If the Gauss curvature is nonpositive,
consider a minimizing geodesic $\gamma$ joining $y$ to $z$ in $\Sigma$ and let $\~\gamma$ be the corresponding curve joining $v$ to $w$ in $\T_x$; that is,  $\gamma=\exp_x\circ\~\gamma$.
By Rauch comparison theorem (\ref{thm:rauch}), we have
\[\length\gamma\ge \length \~\gamma.\]
Since $|v-w|_{\T_x}\le\length\~\gamma$ and $|y-z|_{\Sigma}= \length\gamma$, we get 
\[|v-w|_{\T_x}\ge |y-z|_\Sigma.\]
Therefore
\[\angk xzy\ge \measuredangle\hinge xzy.\]
\qedsf
%??? take care of size of the nbhd

\section{Alexandrov's lemma}

It this section we prove the following lemma in the plane geometry.

\begin{thm}{Lemma}
\label{lem:alex}
Assume $[pxyz]$ and $[p'x'y'z']$ be two quadraliterals in the plane with equal corresponding sides.
Assume that the sides $[x'y']$ and $[y'z']$ extend each other; that is, $y'$ lies on the line segment $[x'z']$.
Then the following expressions have the same signs:
\begin{enumerate}[(i)]
 \item $|p-y|-|p'-y'|$;
 \item $\measuredangle\hinge xpy-\measuredangle\hinge {x'}{p'}{y'}$;
 \item $\pi-\measuredangle\hinge ypx-\measuredangle\hinge ypz$;
\end{enumerate}
\end{thm}

\parit{Proof.} 
In the proof we use the following \emph{monotonicity property}:
if two sides adjacent to an angle in a plane triangle are fixed, 
then the angle is increases if the opposite side increase.

\begin{figure}[h!]
\vskip-0mm
\centering
\includegraphics{mppics/pic-50}
\vskip-0mm
\end{figure}

Take 
a point $\bar z$ on the extension of 
$[xy]$ beyond $y$ so that $\dist{y}{\bar z}{}=\dist{y}{z}{}$ (and therefore $\dist{x}{\bar z}{}=\dist{x'}{z'}{}$). 
 
From monotonicity, 
the following expressions have the same sign:
\begin{enumerate}[(i)]
\item $|p-y|-|p'-y'|$;
\item $\measuredangle\hinge{x}{y}{p}-\measuredangle\hinge{x'}{y'}{p'}=\measuredangle\hinge{x}{\bar z}{p}-\measuredangle\hinge{x'}{z'}{p'}$;
\item $|p-\bar z|-|p'-z'|$;
\item $\measuredangle\hinge{y}{\bar z}{p}-\measuredangle\hinge{y'}{z'}{p'}$;
\end{enumerate}
The statement follows since
\[\measuredangle\hinge{y'}{z'}{p'}+\measuredangle\hinge{y'}{x'}{p'}=\pi\]
and
\[\measuredangle\hinge{y}{\bar z}{p}+\measuredangle\hinge{y}{x}{p}=\pi.\]
\qedsf

Further we will use the following reformulation of this lemma that is using language of comparison triangles.

\begin{thm}{Reformulation}\label{lem:alex-reformulation}
\label{lem:alex}
Assume $[pxz]$ be a triangle in a surface $\Sigma$ and 
the point $y$ lies on the side $[xz]$.
Consider its model triangle $[\~p\~x\~z]=\~\triangle pxz$ and let $\~y$ be the corresponding point on the side $[\~x\~z]$.
Then the following expressions have the same signs:
\begin{enumerate}[(i)]
 \item $|p-y|_\Sigma-|\~p-\~y|_{\RR^2}$;
 \item $\angk xpy-\angk {x}{p}{z}$;
 \item $\pi-\angk ypx-\angk ypz$;
\end{enumerate}
\end{thm}

\section{Reformulations of comparison}

In this section we formulate conditions equivalent to the conclusion of the comparison theorem
(which is not yet proved).

A triangle $[xyz]$ in a surface is called \emph{fat} (or correspondingly \emph{thin})
if for any two points $p$ and $q$ on the sides of the triangle and the corresponding points 
$\~p$ and $\~q$ on the sides of its model triangle $[\~x\~y\~z]\z=\~\triangle xyz$ we have
$|p-q|\ge |\~p-\~q|$ (or correspondingly $|p-q|\le |\~p-\~q|$).

Let us reformulate the comparison theorem (\ref{thm:comp}) using thin and fat triangles:

\begin{thm}{Proposition}
Let $\Sigma$ be a complete smooth regular surface.
Then the following three conditions are equivalent:
\begin{enumerate}[(i$^{+}$)]
\item For any geodesic triangle $[xyz]$ in $\Sigma$ we have
 \[\measuredangle\hinge {x}{y}{z}\ge\angk xyz.\]
\item For any geodesic triangle $[pxz]$ in $\Sigma$ and $y$ on the side $[xz]$ we have
 \[\angk xpy \ge \angk xpz.\]
\item Any geodesic triangle in $\Sigma$ is fat.
\end{enumerate}

\medskip

Similarly, following three conditions are equivalent:
\begin{enumerate}[(i$^{-}$)]
\item For any geodesic triangle $[xyz]$ in $\Sigma$ we have
 \[\measuredangle\hinge {x}{y}{z}\le\angk xyz.\]
\item For any geodesic triangle $[pxz]$ in $\Sigma$ and $y$ on the side $[xz]$ we have
 \[\angk xpy \le \angk xpz.\]
\item Any geodesic triangle in $\Sigma$ is thin.
\end{enumerate}

\end{thm}

\parit{Proof; (i$^{+}$)$\Longrightarrow$(ii$^{+}$).}
Note that $\measuredangle\hinge ypx+\measuredangle\hinge ypz=\pi$.
By \textit{(i$^{+}$)}, 
\[\angk ypx+\angk ypz\le \pi.\]
It reamains to apply Alexandrov's lemma \ref{lem:alex-reformulation}.

The implication \textit{(i$^{-}$)$\Longrightarrow$(ii$^{-}$)} can be done the same way.

\parit{(ii$^{+}$)$\Longrightarrow$(iii$^{+}$).}
Applying \textit{(i$^{+}$)} twice, first for $y\in [xz]$ and then for $w\in [px]$, we get that
\[\angk xwy \ge \angk xpy \ge \angk xpz\]
and therefore
\[|w-y|_\Sigma\ge |\~w-\~y|_{\RR^2},\]
where $\~w$ and $\~y$ are the points corresponding to $w$ and $y$ points on the sides of the model triangle. Hence the implication follows.

The implication \textit{(ii$^{-}$)$\Longrightarrow$(iii$^{-}$)} can be done the same way.

\parit{(iii$^{+}$)$\Longrightarrow$(i$^{+}$).}
Since the triangle is fat, we have 
\[\angk xwy \ge \angk xpz\]
for any $w\in \left]xp\right]$ and $y\in \left]xz\right]$.
Note that $\angk xwy\to \measuredangle\hinge xpz$ as $w,y\to x$; hence the implication follows.

The implication \textit{(iii$^{-}$)$\Longrightarrow$(i$^{-}$)} can be done the same way.
\qeds

\begin{thm}{Exercise}
Let $\Sigma$ be a complete smooth regular surface with nonnegative Gauss curvature.
Show that for any four distinct points the following inequality holds:
\[\angk pxy+\angk pyz+\angk pzx\le2\cdot \pi.\]

\end{thm}

\begin{thm}{Exercise}Let $\Sigma$ be a complete smooth regular surface
and $\gamma$ be a unit-speed geodesic in $\Sigma$ and $p\in\Sigma$.

Consider the function
\[h(t)=|p-\gamma(t)|^2-t^2.\]

\begin{enumerate}[(a)]
\item Show that if the Gauss curvature of $\Sigma$ is nonnegative then $h$ is a concave function.
\item Show that if $\Sigma$ is simply connected and the Gauss curvature of $\Sigma$ is nonpositive then $h$ is a convex function.
\end{enumerate}
\end{thm}

\begin{thm}{Exercise}
Let $[\~x_1\dots\~x_n]$ be a convex plane polygon and
$x_1,\dots,x_n$ be points in a complete simply connected surface $\Sigma$ with nonpositive curvature.
Assume that
$|x_i-x_{i-1}|_\Sigma=|\~x_i-\~x_{i-1}|_{\RR^2}$ for each $i$ and
$\measuredangle\hinge{x_{i-1}}{x_i}{x_{i+1}}=\measuredangle\hinge{x_{i-1}}{x_i}{x_{i+1}}$

\end{thm}





\section{Globalization for nonnegative curvature}

In this section we will prove part \textit{(\ref{thm:comp:toponogov})} of the comparison theorem (\ref{thm:comp}) assuming that $\Sigma$ is compact; the general case requre only minor modifications.

Since $\Sigma$ is compact, from local version of the theorem we get that there is $\eps>0$ such that the inequality 
\[\measuredangle\hinge {x}{y}{z}\ge\angk xyz.\]
holds for any hinge $\hinge {x}{y}{z}$ such that $|x-y|+|x-z|<\eps$.
The following lemma states that in this case the same holds for any hinge $\hinge {x}{y}{z}$ such that $|x-y|+|x-z|<\tfrac32\cdot\eps$.
Applying the lemma few times we will get that the comparison holds for arbitrary hinge, which will prove part~\textit{(\ref{thm:comp:toponogov})}.




\begin{thm}{Key lemma}\label{key-lem:globalization}  
Let $\Sigma$ be a complete smooth regular surface.
Assume that the comparison
\[\mangle\hinge x y q
\ge\angk x y q\]
holds for any hinge $\hinge x y q$ with 
$\dist{x}{y}{}+\dist{x}{q}{}
<
\frac{2}{3}\cdot\ell$.
Then comparison
\[\mangle\hinge x p q
\ge\angk x p q\] 
holds for any hinge $\hinge x p q$ with $\dist{x}{ p}{}+\dist{x}{q}{}<\ell$.
\end{thm}

\begin{wrapfigure}{r}{30mm}
\begin{lpic}[t(0mm),b(0mm),r(10mm),l(0mm)]{pics/globalization(1)}
\lbl[l t]{17,2;$\~ p$}
\lbl[l]{16.5,54;$\~ q$}
\lbl[r]{0.3,22;$\~ x$}
\lbl[r b]{8,35;$\~ x'$}
\lbl[l]{15,34;$\ge \mangle\hinge{x'}p q$}
\lbl[lw]{9,22;$\le \mangle\hinge x p q$}
\end{lpic}
\end{wrapfigure}

\parit{Proof.} Let us denote by $\side \hinge x p q$ the third side of a plane triangle
with angle $\measuredangle\hinge x p q$ and two adjacent sides $|x-p|$ and $|x-q|$.
Note that the inequalities 
\[\measuredangle\hinge x p q\ge \angk x p q\quad\text{and}\quad\side \hinge x p q
\ge\dist{p}{q}{}\]
are equivalent.
So it is sufficient to prove that
\[\side \hinge x p q
\ge\dist{p}{q}{}.\eqlbl{eq:thm:=def-loc*}\] 
for any hinge $\hinge x p q$ with $\dist{x}{p}{}+\dist{x}{q}{}<\ell$.

Fix $q$.
Consider a hinge $\hinge x p q$ such that 
\[\tfrac{2}{3}\cdot\ell \le\dist{p}{x}{}\z+\dist{x}{q}{}< \ell.\]
Let us construct a new smaller hinge $\hinge{x'}p q$; that is,
\[
\dist{p}{x}{}+\dist{x}{q}{}\ge\dist{p}{x'}{}+\dist{x'}{q}{}
\eqlbl{eq:thm:=def-loc-fourstar}\]
and such that 
\[\side \hinge x p q
\ge\side \hinge{x'}p q.
\eqlbl{eq:thm:=def-loc-fivestar}\]

%\parit{Construction of $\hinge{x'}p q$.}
Assume $\dist{x}{q}{}\ge\dist{x}{p}{}$, otherwise switch the roles of $p$ and $q$ in the following construction.
Take $x'\in [x q]$ such that 
\[\dist{p}{x}{}+3\cdot\dist{x}{x'}{}
=\tfrac{2}{3}\cdot\ell \eqlbl{3|xx'|}\]
Choose a geodesic $[x' p]$ and consider the  hinge $\hinge{x'}p q$ fromed by $[x'p]$ and $[x' q]\subset [x q]$. 
Then \ref{eq:thm:=def-loc-fourstar} follows since the length of $[x'p]$ can not exceed the length of broken geodesic $x'xp$.

Further, note that 
$\dist{p}{x}{}\z+\dist{x}{x'}{},\dist{p}{x'}{}\z+\dist{x'}{x}{}<\tfrac{2}{3}\cdot\ell $.
In particular, 
\[\mangle\hinge x p{x'}
\ge\angk x p{x'}
\ \ \text{and}\ \ 
\mangle\hinge {x'}p x
\ge\angk {x'}p x.
\eqlbl{eq:thm:=def-loc-threestar}\]


Consider the model triangle
$\trig{\~x}{\~x'}{\~p}=\modtrig x x' p$.
Take $\~ q$ on the extension of $[\~ x\~ x']$ beyond $x'$ such that $\dist{\~x}{\~q}{}=\dist{x}{q}{}$ (and therefore $\dist{\~x'}{\~q}{}=\dist{x'}{q}{}$).
From \ref{eq:thm:=def-loc-threestar},
\[\mangle\hinge x p q
=\mangle\hinge  x p{x'}\ge\angk x p{x'}\ \ \Rightarrow\ \ 
\side \hinge x q p\ge\dist{\~p}{\~q}{}.\]
Since $\mangle\hinge{x'}p x+\mangle\hinge{x'}p q= \pi$,
\ref{eq:thm:=def-loc-threestar} implies
\[
\pi
-\angk{x'}p x
\ge
\pi-\mangle\hinge{x'}p x
\ge
\mangle\hinge{x'}p q.
\]
Therefore
$\dist{\~p}{\~q}{}\ge\side \hinge{x'}q p$ and \ref{eq:thm:=def-loc-fivestar} follows.

\medskip

\begin{center}
 \begin{lpic}[t(0mm),b(0mm),r(0mm),l(0mm)]{pics/px_nq(1)}
\lbl[rb]{1,16;$x_0$}
\lbl[rt]{12,1;$p$}
\lbl[tl]{70,1;$q$}
\lbl[l b]{22,12;$x_1$}
\lbl[l b]{38,9;$x_2$}
\lbl[br]{25,6;$x_3$}
\lbl[t]{47,1.5;$x_4$}
\lbl[t]{30,1;$x_5$}
\end{lpic}
\end{center}

%%%??? WE SHOULD MENTION cat's cradle.
Let us continue the proof.
Set $x_0=x$.
Let us apply inductively the above construction to get a sequence of hinges  $\hinge{x_n}p q$ with $x_{n+1}=x_n'$.

The sequence might terminate at some $n$ only if $\dist{p}{x_n}{}+\dist{x_n}{q}{}\z< \tfrac{2}{3}\cdot\ell $.
In this case, by the assumptions of the lemma, $\side \hinge{x_n}p q\ge\dist{p}{q}{}$.
From \ref{eq:thm:=def-loc-fivestar}, we get that the sequence  $s_n\z=\side \hinge{x_n}p q$ is decreasing.
Hence inequality \ref{eq:thm:=def-loc*} follows.

Otherwise, the sequence $r_n=\dist{p}{x_n}{}+\dist{x_n}{q}{}$ is non-increasing and $r_n\ge\tfrac{2}{3}\cdot\ell$ for all $n$. Hence $r_n\to r$ as $n\to\infty$.
Pass to a subsequence $y_\kay=x_{n_\kay}$ such that $y_\kay'=x_{n_\kay+1}\in [y_\kay q]$.
(In particular, $\dist{y_\kay}{q}{}\ge\dist{y_\kay}{p}{}$.)
Clearly, \[\dist{p}{y_\kay}{}+\dist{y_\kay}{y_\kay'}{}\z-\dist{p}{y_\kay'}{}=r_\kay-r_{\kay+1}\underset{\kay\to\infty} {\to}0.
\eqlbl{trig=}\]
From \ref{3|xx'|}, 
it follows that  $\dist{y_\kay}{y'_\kay}{}\z>\tfrac{\ell}{100}$.
From \ref{3|xx'|} and \ref{trig=}, it follows that 
$\dist{p}{y_\kay}{}\z>\tfrac{\ell}{100}$
for all large $\kay$.
Therefore, $\angk{y_\kay}p{y_\kay'}\to \pi$. 
Since 
$\mangle\hinge{y_\kay}p{y_\kay'}
\ge\angk{y_\kay}p{y_\kay'}$, we have 
$\mangle\hinge{y_\kay}p q=\mangle\hinge{y_\kay}p{y_\kay'}\to \pi$.
Therefore, 
\[\dist{p}{y_\kay}{}+\dist{y_\kay}{q}{}-\side \hinge{y_\kay}p q\to 0.\] 
Together with the triangle inequality
\[
\dist{p}{y_\kay}{}+\dist{y_\kay}{q}{}\ge\dist{p}{q}{}
\]
this yields
\[\lim_{n\to\infty}\side \hinge{y_\kay}p q\ge \dist{p}{q}{}.\]
Applying monotonicity of sequence  $s_n=\side \hinge{y_\kay}p q$ we obtain \ref{eq:thm:=def-loc*}.
\qeds





