\chapter{Global comparison}

\section{Formulation}

A minimizing geodesic between points $x$ and $y$ in a surface $\Sigma$ will be denoted as $[xy]$ or $[xy]_\Sigma$;
the latter notation is used if we need to emphasise that the geodesic lies in $\Sigma$.
If we write $[xy]$, then we assume that a minimizing geodesic exists and we made a choice of one of them.

A \emph{geodesic triangle} in a surface $\Sigma$ is a triple of points $x,y,z\in \Sigma$ with choice of minimizing geodesics $[xy]$, $[yz]$ and $[zx]$.
The points $x,y,z$ are called \emph{vertexes} of the geodesic triangle,
the minimizing geodesics $[xy]$, $[yz]$ and $[zx]$ are called its sides;
the triangle itself is denoted by $[xyz]$.%
\footnote{The notation $[xyz]$ is just a shortcut for the array $(x,y,z,[xy], [yz], [zx]$.}

The length of one (and therefore any) minimizing geodesic $[xy]_\Sigma$ will be denoted by $|x-y|_\Sigma$; it is called \emph{intrinsic distance} from $x$ to $y$ in $\Sigma$.
If defined, then $|x-y|_\Sigma$ is the exact lower bound on the lengths of curves from $x$ to $y$ in $\Sigma$. 

A triangle $[\~x\~y\~z]$ in the plane is called \emph{model triangle} of the triangle $[xyz]$,
briefly $[\~x\~y\~z]=\~\triangle xyz$, if its corresponding sides are equal;
that is,
\[|\~x-\~y|=|x-y|_\Sigma,
\quad
|\~y-\~z|=|y-z|_\Sigma,
\quad
|\~z-\~x|=|z-x|_\Sigma.
\]

A pair of minimizing geodesics $[xy]$ and $[xz]$ starting from one point $x$ is called \emph{hinge} and denoted as $\hinge xyz$.
The angle between these geodesics at $x$ is denoted by $\measuredangle\hinge xyz$.
The corresponding angle $\measuredangle\hinge {\~x}{\~y}{\~z}$ in the model triangle $[\~x\~y\~z]=\~\triangle xyz$ is denoted by $\angk xyz$

A surface $\Sigma$ is called \emph{simply connected} if any closed simple curve in $\Sigma$ bounds a disc.
Equivalently any closed curve in $\Sigma$ can be continuously deformed into a trivial curve (which stays at one point).
A plane or sphere are examples of simply connected surfaces, while torus or cylinder are not simply connected.


\begin{thm}{Comparison theorem}\label{thm:comp}
Let $\Sigma$ be a complete smooth regular surface with a geodesic triangle $[xyz]$.
\begin{enumerate}[(i)]
 \item\label{thm:comp:toponogov} If $\Sigma$ has nonnegative Gauss curvature, then 
 \[\measuredangle\hinge {\~x}{\~y}{\~z}\ge\angk xyz.\]
 \item\label{thm:comp:cat} If $\Sigma$ is simply connected and has nonpositive Gauss curvature,
 then 
 \[\measuredangle\hinge {\~x}{\~y}{\~z}\le\angk xyz.\]
\end{enumerate}

\end{thm}


The angle $\alpha$ between geodesics is a number in the interval $[0,\pi]$;
if $\theta$ is the external angle used in Gauss--Bonnet formula, then the corresponding angle is $\alpha=|\pi-\theta|$.
The internal angle might be $\alpha$ or $2\cdot\pi-\alpha$ depending on which side lies the disc $\Delta$.

Since the angles of any plane triangle sum up to $\pi$,
the part (\ref{thm:comp:toponogov}) of the theorem implies that angles of any triangle in a surface with nonnegative Gauss curvature have sum at least $\pi$.
If the triangle bounds a disc, then by Gauss--Bonnet formula the sum of its internal angles is at least $\pi$.
Note that the triangle may not bound a disc, for example equator on the cyclinder is formed by a geodesic triangle that does not bound a disc.
Also note that (1) Gauss--Bonnet formula gives a lower bound on the sum of its \emph{internal} angles, but does not bound each angle separately (2) if $\alpha$ is the angle in the comparison theorem, then the internal angle might be $\alpha$ or $2\cdot\pi-\alpha$; while Gauss--Bonnet formula gives a lower bound on the sum of internal angles it does not forbid that each of these angles is close to $2\cdot \pi$ which is impossible by the comparison theorem.

On the part (\ref{thm:comp:cat}).
First note that without condition that $\Sigma$ is simply connected, the statement does not hold.
For example the equator $z=0$ of infinite cylinder (which is not simply connected)
\[\set{(x,y,z)\in\RR^3}{x^2+y^2=1}\]
is formed by a triangle with all angles $\pi$; which contradict the comparison.

\begin{thm}{Exercise}
Let $\Sigma$ be a complete smooth regular simply connected surface with nonpositive Gauss curvature.
Show that any two points in $\Sigma$ are connected by unique geodesic.
\end{thm}


\section{Names and history}

Part (\ref{thm:comp:toponogov}) of this theorem is called \emph{Toponogov comparison theorem};
it is was proved by Paolo Pizzetti \cite{pizzetti} and latter independently by Alexandr Alexandrov \cite{alexandrov}; generalizations were obtained by  Victor Toponogov \cite{toponogov}, Mikhael Gromov, Yuri Burago and Grigory Perelman \cite{BGP}.

Part (\ref{thm:comp:cat}) is called \emph{Cartan--Hadamard theorem};
it was proved by 
Hans von Mangoldt \cite{mangoldt} and generalized by Elie Cartan \cite{cartan}, Jacques Hadamard \cite{hadamard},
Herbert Busemann \cite{busemann},
Willi Rinow in \cite{rinow},
Mikhael Gromov \cite[p.119]{gromov},
Stephanie Alexander and Richard Bishop in \cite{a-b:h-c}.

\section{Local part}

First we prove the following local version of comparison theorem and then use it to prove the global version.

\begin{thm}{Theorem}
The comparison theorem (\ref{thm:comp}) holds in a small neighborhood of any point.

That is, if $\Sigma$ be a complete smooth regular surface,
then any point $p\in \Sigma$ admits a neighborhood $U$ such that 
\begin{enumerate}[(i)]
 \item If $\Sigma$ has nonnegative Gauss curvature, then for any geodesic triangle $[xyz]$ in $U$ we have
 \[\measuredangle\hinge {\~x}{\~y}{\~z}\ge\angk xyz.\]
 \item If $\Sigma$ has nonpositive Gauss curvature, then for any geodesic triangle $[xyz]$ in $U$ we have
 \[\measuredangle\hinge {\~x}{\~y}{\~z}\le\angk xyz.\]
\end{enumerate}
\end{thm}

\parit{Proof.}
Assume $y=\exp_xv$ and $z=\exp_xw$ for two small vectors $v,w\in\T_x$.
Note that $\measuredangle \hinge xvw_{\T_x}=\measuredangle \hinge xyz_\Sigma$,
$|x-y|_\Sigma=|x-v|_{\T_x}$, $|x-z|_\Sigma=|x-w|_{\T_x}$.

If the Gauss curvature is nonnegative,
consider the line segment $\~\gamma$ joining $v$ to $w$ and set $\gamma=\exp_x\circ\~\gamma$.
By Rauch comparison theorem (\ref{thm:rauch}), we have
\[\length\gamma\le \length \~\gamma.\]
Since $|v-w|_{\T_x}=\length\~\gamma$ and $|y-z|_{\Sigma}\le \length\gamma$, we get 
\[|v-w|_{\T_x}\ge |y-z|_\Sigma.\]
Therefore
\[\angk xzy\ge \measuredangle\hinge xzy.\]

If the Gauss curvature is nonpositive,
consider a minimizing geodesic $\gamma$ joining $y$ to $z$ in $\Sigma$ and let $\~\gamma$ be the corresponding curve in $\T_x$; that is  $\gamma=\exp_x\circ\~\gamma$.
By Rauch comparison theorem (\ref{thm:rauch}), we have
\[\length\gamma\ge \length \~\gamma.\]
Therefore
\[\angk xzy\ge \measuredangle\hinge xzy.\]
\qedsf
%??? take care of size of the nbhd

\section{Alexandrov's lemma}

It this section we prove the following lemma in the plane geometry.

\begin{thm}{Lemma}
\label{lem:alex}
Assume $[pxyz]$ and $[p'x'y'z']$ be two quadraliterals in the plane with equal corresponding sides.
Assume that the sides $[x'y']$ and $[y'z']$ extend each other; that is, $y'$ lies on the line segment $[x'z']$.
Then the following expressions have the same signs:
\begin{enumerate}[(i)]
 \item $|p-y|-|p'-y'|$;
 \item $\measuredangle\hinge xpy-\measuredangle\hinge {x'}{p'}{y'}$;
 \item $\pi-\measuredangle\hinge pyx+\measuredangle\hinge pyz$;
\end{enumerate}
\end{thm}

\parit{Proof.} 
In the proof we use the following \emph{monotonicity property}:
if two sides adjacent to an angle in a plane triangle are fixed, 
then the angle is increases if the opposite side increase.

Take 
a point $\bar z$ on the extension of 
$[xy]$ beyond $y$ so that $\dist{y}{\bar z}{}=\dist{y}{z}{}$ (and therefore $\dist{x}{\bar z}{}=\dist{x'}{z'}{}$). 

\begin{figure}[h!]
\vskip-0mm
\centering
\includegraphics{mppics/pic-50}
\vskip-0mm
\end{figure}
 
From monotonicity, 
the following expressions have the same sign:
\begin{enumerate}[(i)]
\item $|p-y|-|p'-y'|$;
\item $\measuredangle\hinge{x}{y}{p}-\measuredangle\hinge{x'}{y'}{p'}=\measuredangle\hinge{x}{\bar z}{p}-\measuredangle\hinge{x'}{z'}{p'}$;
\item $|p-\bar z|-|p'-z'|$;
\item $\measuredangle\hinge{y}{\bar z}{p}-\measuredangle\hinge{y'}{z'}{p'}$;
\end{enumerate}
The first statement follows since
\[\measuredangle\hinge{y'}{z'}{p'}+\measuredangle\hinge{y'}{x'}{p'}=\pi\]
and
\[\measuredangle\hinge{y}{\bar z}{p}+\measuredangle\hinge{y}{x}{p}=\pi.\]
\qedsf

\section{Thin and fat triangles}

In this section we discuss some reformulations of the comparison theorem;
it will be proved latter.

A triangle $[xyz]$ in a surface is called \emph{fat} (or correspondingly \emph{thin})
if for any two points $p$ and $q$ on the sides of the triangle and the corresponding points 
$\~p$ and $\~q$ on the sides of its model triangle $[\~x\~y\~z]\z=\~\triangle xyz$ we have
$|p-q|\ge |\~p-\~q|$ (or correspondingly $|p-q|\le |\~p-\~q|$).

Let us reformulate the comparison theorem (\ref{thm:comp}) using thin and fat triangles:

\begin{thm}{Proposition}
Let $\Sigma$ be a complete smooth regular surface with a geodesic triangle $[xyz]$.
\begin{enumerate}[(i)]
 \item If $\Sigma$ has nonnegative Gauss curvature, then $[xyz]$ is fat
 \item If $\Sigma$ is simply connected and has nonpositive Gauss curvature,
 then $[xyz]$ is thin.
\end{enumerate}
\end{thm}

We will give a proof of proposition using the comparison theorem (which is not yet proved).

\parit{Proof.}
Fix two points $p$ and $q$ on the sides of the triangle $[xyz]$.
Without loss of generality, we may assume that $p\in [yz]$ and $q\in [xy]$.
Let $\~p$ and $\~q$ be the corresponding points on the sides of the  model triangle $[\~x\~y\~z]\z=\~\triangle xyz$.

Assume that Gauss curvature is nonnegative.
Then by comparison theorem $\angk pxy\le \measuredangle \hinge pxy$ and $\angk pxz\le \measuredangle \hinge pxz$.
Note that $\measuredangle \hinge pxy+\measuredangle \hinge pxz=\pi$;
therefore
\[\angk pxy +\angk pxz\le pi.\]

Applying Alexandrov's lemma, we get $|x-p|\ge |\~x-\~p|$.
Repeating the same argument for the triangle $[pxy]$ we get $|p-q|\ge |\~p-\~q|$.

Now assume that Gauss curvature is nonpositive.
Then by comparison theorem $\angk pxy\ge \measuredangle \hinge pxy$ and $\angk pxz\ge \measuredangle \hinge pxz$.
Since $\measuredangle \hinge pxy+\measuredangle \hinge pxz=\pi$, we have
\[\angk pxy +\angk pxz\le \pi.\]

Applying Alexandrov's lemma, we get $|x-p|\le |\~x-\~p|$.
Repeating the same argument for the triangle $[pxy]$ we get $|p-q|\ge |\~p-\~q|$.\qeds

\begin{thm}{Exercise}
Exercise assume all triangles in a complete smooth regular surface $\Sigma$ are thin (or correspondingly fat), show that comparison theorem part (i) (or correspondingly par (ii)) holds in $\Sigma$.
\end{thm}

\begin{thm}{Exercise}
Let $\Sigma$ be a complete smooth regular surface with nonnegative Gauss curvature.
Show that for any four distinct points the following inequality holds:
\[\angk pxy+\angk pyz+\angk pzx\le2\cdot \pi.\]

\end{thm}

\begin{thm}{Exercise}Let $\Sigma$ be a complete smooth regular surface
and $\gamma$ be a unit-speed geodesic in $\Sigma$ and $p\in\Sigma$.

Consider the function
\[h(t)=|p-\gamma(t)|^2-t^2.\]

\begin{enumerate}[(a)]
\item Show that if the Gauss curvature of $\Sigma$ is nonnegative then $h$ is a concave function.
\item Show that if $\Sigma$ is simply connected and the Gauss curvature of $\Sigma$ is nonpositive then $h$ is a convex function.
\end{enumerate}
\end{thm}








