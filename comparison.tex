\chapter{Global comparison}

\section{Formulation}

A minimizing geodesic between points $x$ and $y$ in a surface $\Sigma$ will be denoted as $[xy]$ or $[xy]_\Sigma$;
the latter notation is used if we need to emphasise that the geodesic lies in $\Sigma$.
If we write $[xy]$, then we assume that a minimizing geodesic exists and we made a choice of one of them.

A \emph{geodesic triangle} in a surface $\Sigma$ is a triple of points $x,y,z\in \Sigma$ with choice of minimizing geodesics $[xy]$, $[yz]$ and $[zx]$.
The points $x,y,z$ are called \emph{vertexes} of the geodesic triangle,
the minimizing geodesics $[xy]$, $[yz]$ and $[zx]$ are called its sides;
the triangle itself is denoted by $[xyz]$.%
\footnote{The notation $[xyz]$ is just a shortcut for the array $(x,y,z,[xy], [yz], [zx]$.}

The length of one (and therefore any) minimizing geodesic $[xy]_\Sigma$ will be denoted by $|x-y|_\Sigma$; it is called \emph{intrinsic distance} from $x$ to $y$ in $\Sigma$.
If defined, then $|x-y|_\Sigma$ is the exact lower bound on the lengths of curves from $x$ to $y$ in $\Sigma$. 

A triangle $[\~x\~y\~z]$ in the plane is called \emph{model triangle} of the triangle $[xyz]$ if its corresponding sides are equal;
that is,
\[|\~x-\~y|=|x-y|_\Sigma,
\quad
|\~y-\~z|=|y-z|_\Sigma,
\quad
|\~z-\~x|=|z-x|_\Sigma.
\]

A surface $\Sigma$ is called \emph{simply connected} if any closed simple curve in $\Sigma$ bounds a disc.
Equivalently any closed curve in $\Sigma$ can be continuously deformed into a trivial curve (which stays at one point).
A plane or sphere are examples of simply connected surfaces, while torus or cylinder are not simply connected.


\begin{thm}{Comparison theorem}\label{thm:comp}
Let $\Sigma$ be a complete smooth regular surface with a geodesic triangle $[xyz]$.
Suppose $[\~x\~y\~z]$ is a model triangle of $[xyz]$;
denote by $\alpha$ the angle of $[xyz]$ at $x$
and by $\~\alpha$ the angle of $[\~x\~y\~z]$ at $\~x$.
\begin{enumerate}[(i)]
 \item\label{thm:comp:toponogov} If $\Sigma$ has nonnegative Gauss curvature, then $\alpha\ge\~\alpha$.
 \item\label{thm:comp:cat} If $\Sigma$ is simply connected and has nonpositive Gauss curvature,
 then $\alpha\le\~\alpha$.
\end{enumerate}

\end{thm}

The angle $\alpha$ between geodesics is a number in the interval $[0,\pi]$;
if $\theta$ is the external angle used in Gauss--Bonnet formula, then the corresponding angle is $\alpha=\pi-|\theta$.
The internal angle might be $\alpha$ or $2\cdot\pi-\alpha$ depending on which side lies the disc $\Delta$.

Since the angles of any plane triangle sum up to $\pi$,
the part (\ref{thm:comp:toponogov}) of the theorem implies that angles of any triangle in a surface with nonnegative Gauss curvature have sum at least $\pi$.
If the triangle bounds a disc, then by Gauss--Bonnet formula the sum of its internal angles is at least $\pi$.
Note that the triangle may not bound a disc, for example equator on the cyclinder is formed by a geodesic triangle that does not bound a disc.
Also note that (1) Gauss--Bonnet formula gives a lower bound on the sum of its \emph{internal} angles, but does not bound each angle separately (2) if $\alpha$ is the angle in the comparison theorem, then the internal angle might be $\alpha$ or $2\cdot\pi-\alpha$; while Gauss--Bonnet formula gives a lower bound on the sum of internal angles it does not forbid that each of these angles is close to $2\cdot \pi$ which is impossible by the comparison theorem.

On the part \ref{thm:comp:cat}.
First note that without condition that $\Sigma$ is simply connected, the statement does not hold.
For example the equator $z=0$ of infinite cylinder (which is not simply connected)
\[\set{(x,y,z)\in\RR^3}{x^2+y^2=1}\]
is formed by a triangle with all angles $\pi$; which contradict the comparison.

\begin{thm}{Exercise}
Let $\Sigma$ be a complete smooth regular simply connected surface with nonpositive Gauss curvature.
Show that any two points in $\Sigma$ are connected by unique geodesic.
\end{thm}




