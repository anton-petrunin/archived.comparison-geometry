\chapter{Comparison theorems}

\section{Formulation}

A minimizing geodesic between points $x$ and $y$ in a surface $\Sigma$ will be denoted as $[xy]$ or $[xy]_\Sigma$;
the latter notation is used if we need to emphasise that the geodesic lies in $\Sigma$.
If we write $[xy]$, then we assume that a minimizing geodesic exists and we made a choice of one of them.

A \emph{geodesic triangle} in a surface $\Sigma$ is a triple of points $x,y,z\in \Sigma$ with choice of minimizing geodesics $[xy]$, $[yz]$ and $[zx]$.
The points $x,y,z$ are called \emph{vertexes} of the geodesic triangle,
the minimizing geodesics $[xy]$, $[yz]$ and $[zx]$ are called its sides;
the triangle itself is denoted by $[xyz]$.%
\footnote{The notation $[xyz]$ is just a shortcut for the array $(x,y,z,[xy], [yz], [zx]$.}

The length of one (and therefore any) minimizing geodesic $[xy]_\Sigma$ will be denoted by $|x-y|_\Sigma$; it is called \emph{intrinsic distance} from $x$ to $y$ in $\Sigma$.
If defined, then $|x-y|_\Sigma$ is the exact lower bound on the lengths of curves from $x$ to $y$ in $\Sigma$. 

A triangle $[\~x\~y\~z]$ in the plane is called \emph{model triangle} of the triangle $[xyz]$ if its corresponding sides are equal;
that is,
\[|\~x-\~y|=|x-y|_\Sigma,
\quad
|\~y-\~z|=|y-z|_\Sigma,
\quad
|\~z-\~x|=|z-x|_\Sigma.
\]

A surface $\Sigma$ is called \emph{simply connected} if any closed simple curve in $\Sigma$ bounds a disc.
Equivalently any closed curve in $\Sigma$ can be continuously deformed into a trivial curve (which stays at one point).
A plane or sphere are examples of simply connected surfaces, while torus or cylinder are not simply connected.


\begin{thm}{Comparison theorem}\label{thm:comp}
Let $\Sigma$ be a complete smooth regular surface with a geodesic triangle $[xyz]$.
Suppose $[\~x\~y\~z]$ is a model triangle of $[xyz]$;
denote by $\alpha$ the angle of $[xyz]$ at $x$
and by $\~\alpha$ the angle of $[\~x\~y\~z]$ at $\~x$.
\begin{enumerate}[(i)]
 \item\label{thm:comp:toponogov} If $\Sigma$ has nonnegative Gauss curvature, then $\alpha\ge\~\alpha$.
 \item\label{thm:comp:cat} If $\Sigma$ is simply connected and has nonpositive Gauss curvature,
 then $\alpha\le\~\alpha$.
\end{enumerate}

\end{thm}

The angle $\alpha$ between geodesics is a number in the interval $[0,\pi]$;
if $\theta$ is the external angle used in Gauss--Bonnet formula, then the corresponding angle is $\alpha=\pi-|\theta$.
The internal angle might be $\alpha$ or $2\cdot\pi-\alpha$ depending on which side lies the disc $\Delta$.

Since the angles of any plane triangle sum up to $\pi$,
the part (\ref{thm:comp:toponogov}) of the theorem implies that angles of any triangle in a surface with nonnegative Gauss curvature have sum at least $\pi$.
If the triangle bounds a disc, then by Gauss--Bonnet formula the sum of its internal angles is at least $\pi$.
Note that the triangle may not bound a disc, for example equator on the cyclinder is formed by a geodesic triangle that does not bound a disc.
Also note that (1) Gauss--Bonnet formula gives a lower bound on the sum of its \emph{internal} angles, but does not bound each angle separately (2) if $\alpha$ is the angle in the comparison theorem, then the internal angle might be $\alpha$ or $2\cdot\pi-\alpha$; while Gauss--Bonnet formula gives a lower bound on the sum of internal angles it does not forbid that each of these angles is close to $2\cdot \pi$ which is impossible by the comparison theorem.

On the part \ref{thm:comp:cat}.
First note that without condition that $\Sigma$ is simply connected, the statement does not hold.
For example the equator $z=0$ of infinite cylinder (which is not simply connected)
\[\set{(x,y,z)\in\RR^3}{x^2+y^2=1}\]
is formed by a triangle with all angles $\pi$; which contradict the comparison. 


\section{First variation formula}

\begin{thm}{Proposition}\label{prop:first-var}
Assume $(s,t)\mapsto f(s,t)$ be a local parametrization of an oriented smooth regular surface $\Sigma$ such that 
$\tfrac{\partial}{\partial s}f\perp \tfrac{\partial}{\partial t}f$, $|\tfrac{\partial}{\partial s}f|=1$ and the vector $\tfrac{\partial}{\partial s}f$ points to the right from $\tfrac{\partial}{\partial t}f$ at any parameter value $(s,t)$.

Fix an closed real interval $[a,b]$ and consider one parameter family of curves $\gamma_s\:[a,b]\to \Sigma$  defined as the coordinate lines $\gamma_s(t)=f(s,t)$.
Set $\ell(s)=\length \gamma_s$.
Then
\[\ell'(s)=\Theta_{\gamma_s}\]
for any $s$.
\end{thm}

The proof is done by direct calculations.

\parit{Proof.}
Since $\tfrac{\partial}{\partial s}f\perp \tfrac{\partial}{\partial t}f$, we have that
\[\langle\tfrac{\partial}{\partial s}f, \tfrac{\partial}{\partial t}f\rangle=0\]
and therefore
\[\langle\tfrac{\partial^2}{\partial s\partial t}f, \tfrac{\partial}{\partial t}f\rangle
+\langle\tfrac{\partial}{\partial s}f, \tfrac{\partial^2}{\partial t^2}f\rangle=\tfrac{\partial}{\partial t}\langle\tfrac{\partial}{\partial s}f, \tfrac{\partial}{\partial t}f\rangle=0.\]

Note that $|\gamma'_s(t)|=|\tfrac{\partial}{\partial t}f(s,t)|$ and therefore
\begin{align*}
\tfrac \partial {\partial s}|\gamma'_s(t)|&=\tfrac \partial {\partial s}\sqrt{\langle \tfrac{\partial}{\partial t}f(s,t),\tfrac{\partial}{\partial t}f(s,t)\rangle}=
\\
&=\frac{\langle \tfrac{\partial^2}{\partial s\partial t}f(s,t),\tfrac{\partial}{\partial t}f(s,t)\rangle}{\sqrt{\langle \tfrac{\partial}{\partial t}f(s,t),\tfrac{\partial}{\partial t}f(s,t)\rangle}}=
\\
&=-\frac{\langle\tfrac{\partial}{\partial s}f, \tfrac{\partial^2}{\partial t^2}f\rangle}{|\gamma'_s(t)|}=
\\
&=-\frac{\langle\tfrac{\partial}{\partial s}f, \gamma''_s(t)\rangle}{|\gamma'_s(t)|}.
\end{align*}

The values $\ell(s)$ do not change if we reparametrize $\gamma$,
so we can assume that for any fixed value $s$ the curve $\gamma_s$ is unit-speed.
Since $|\tfrac{\partial}{\partial s}f|=1$ and $\tfrac{\partial}{\partial s}f$ points to the right from $\tfrac{\partial}{\partial t}f=\gamma'_s(t)$, the last expression equals to $k_g(s,t)$,
where $k_g(s,t)$ denotes the geodesic curvature of $\gamma_s$ at $t$ and therefore
\begin{align*}
\ell'(s)&= \int_a^b \tfrac \partial {\partial s} |\gamma'_s(t)|\cdot dt =
\\
&= \int_a^b k_g(s,t)\cdot dt=
\\
&=\Theta_{\gamma_s}.
\end{align*}
\qedsf

The parametrization of a surface satisfying the conditions in the proposition are called \emph{semigeodesic coordinates}.
The following exercise explains the reason for this name.

\begin{thm}{Exercise}
Assume $(s,t)\mapsto f(s,t)$ be a local parametrization of an oriented smooth regular surface $\Sigma$ as in the proposition above.
Show that for any fixed $t$ the curve $\gamma_t\:s\mapsto f(s,t)$ is a geodesic.\footnote{Hint: note that it is sufficient to show that $\langle\tfrac{\partial^2}{\partial s^2}f,\tfrac{\partial}{\partial t}f\rangle=0$.}
\end{thm}


\section{Normal coordinates}

Let $\Sigma$ be smooth regular surface and $p\in \Sigma$.
Given a tangent vector $v\in \T_p$ consider a geodesic $\gamma_v$ in $\Sigma$ that runs from $p$ with the initial velocity $v$;  
that is, $\gamma(0)=p$ and $\gamma'(0)=v$.

The point $q=\gamma_v(1)$ is called exponential map of $v$, or briefly $q=\exp_pv$.
The map $\exp_p\:\T_p\to \Sigma$ is defined in a neighborhood of zero.
We assume that it is intuitively obvious that the map $\exp_p$ is smooth;
formally it follows since the solution of the initial value problem for geodesic $\gamma_v$ smoothly depend on the initial data $v$.
Note that the Jacobian of $\exp_p$ at zero is the identity matrix.
Therefore according to inverse function theorem we get the following statement:

\begin{thm}{Proposition}\label{prop:exp}
Let $\Sigma$ be smooth regular surface and $p\in \Sigma$.
Then the exponential map $\exp_p\:\T_p\to \Sigma$ is a smooth regular parametrization of a neighborhood of $p$ in $\Sigma$ by a neighborhood of $0$ in the tangent plane~$\T_p$.

Moreover for any $p\in \Sigma$ there is $\eps>0$ such that for any $x\in \Sigma$ such that $|x-p|_\Sigma<\eps$ the map 
$\exp_x\:\T_x\to \Sigma$ is a smooth regular parametrization of the $\eps$-neighborhood of $x$ in $\Sigma$ by the $\eps$-neighborhood of zero in the tangent plane~$\T_x$.
\end{thm}

Note that if there are two minimizing geodesics between two points $x$ and $y$ in a surface,
then there are two distinct vectors $v,v'\in \T_x$ such that $y=\exp_xv=\exp_xv'$.
Therefore we get the following corollary.

\begin{thm}{Corollary}
Let $\Sigma$ be a smooth regular surface.
Then for any point $p\in\Sigma$ there is $\eps>0$ such that any two points $x$ and $y$ in the $\eps$-neightbohood of $p$ in $\Sigma$ can be connected by a unique minimizing geodesic $[xy]_\Sigma$.
\end{thm}


\section{Polar coordinates}

Proposition~\ref{prop:exp} implies existence of polar coordinates in a neighborhood of any point in $p$ in $\Sigma$.
That is, any point $x$ in $\Sigma$ sufficiently close to $p$ 
can be uniquely described by the distance $|x-p|_\Sigma$ and the direction from $p$ to $x$.

Assume $(\theta,r)$ are the described polar coordinates.
Namely, assume $\~w(\theta,r)$ denotes the tangent vector at $p$ with polar coordinates $(\theta,r)$ and $w( \theta,r)=\exp_p[\~w(\theta,r)]$.
Note that for a fixed $\theta$, the curve $\gamma_\theta(t)=w(\theta,t)$ is a unit-speed geodesic that starts at $p$;
in particular $|\tfrac{\partial}{\partial r}w|=|\gamma_\theta'(r)|=1$ and $\gamma''_\theta(r)\perp\T_{\gamma_\theta(r)}$.

The curve $\sigma_r(t)=w(t,r)$, $t\in[0,2\cdot\pi]$ is a parametrization of the circle of radius $r$ and center at $p$ in $\Sigma$; that is, if $q=\sigma_r(t)$, then $|q-p|_\Sigma=r$.
If the latter is not the case, then a minimizing geodesic $[pq]_\Sigma$ would be shorter than $r$ and therefore $q$ would not be described uniquely in the polar coordinates. 

Note that $\tfrac{\partial}{\partial r}w\perp \tfrac{\partial}{\partial \theta}w$ if $r>0$;
indeed otherwise for small $\eps>0$ the intrinsic distance from $p$ to $w(\theta\pm \eps,r)$ would be shorter than $r$, which contradicts the previous statement.

\begin{thm}{Proposition}
Let $w(\theta,r)$ and $\~w(\theta,r)$ be the polar coordinates of a surface $\Sigma$ at $p$ and its tangent plane $\T_p$ at zero, so $w(\theta,r)\z=\exp_p[\~w(\theta,r)]$.
Given a real interval $[a,b]$ consider the one parameter families of circular arcs $\sigma_r\:[a,b]\to \Sigma$ and $\~\sigma_r\:[a,b]\to \T_p$
$\sigma_r(t)=w(t,r)$ and $\~\sigma_r(t)=\~w(t,r)$.
Set $\ell(r)=\length \sigma_r$ and $\~\ell(r)=\length \~\sigma_r$; note that $\~\ell(r)=r\cdot(b-a)$.

\begin{enumerate}[(i)]
 \item If the Gauss curvature of $\Sigma$ is nonnegative, then 
 \[\ell(r)\le \~\ell(r)\]
 for all small $r>0$.
 \item If the Gauss curvature of $\Sigma$ is nonpositive, then 
 \[\ell(r)\ge \~\ell(r)\]
 for all small $r>0$.
\end{enumerate}

\end{thm}

Taking a limit as $b\to a$, we obtain the following corollary.

\begin{thm}{Corollary}
Let $w(\theta,r)$ and $\~w(\theta,r)$ be the polar coordinates of a surface $\Sigma$ at $p$ and its tangent plane $\T_p$ at zero, so $w(\theta,r)\z=\exp_p[\~w(\theta,r)]$.
\begin{enumerate}[(i)]
 \item If the Gauss curvature of $\Sigma$ is nonnegative, then 
 \[|\tfrac{\partial}{\partial \theta} w|\le |\tfrac{\partial}{\partial \theta} \~w|\]
 for all small $r>0$.
 \item If the Gauss curvature of $\Sigma$ is nonpositive, then 
\[|\tfrac{\partial}{\partial \theta} w|\ge |\tfrac{\partial}{\partial \theta} \~w|\]
 for all small $r>0$.
\end{enumerate}
\end{thm}

\parit{Proof.}
From the above discussion, the polar coordinates $w(\theta,r)$ satisfy the conditions in the first variation formula (\ref{prop:first-var}).
In particular if $\ell(r)=\length \sigma_r$, then
\[\ell'(r)=\Theta_{\sigma_r}\]
for any $r>0$.

By Gauss--Bonnet formula, the last identity can be rewritten as
\[\ell'(r)=2\cdot (b-a) -\iint_{\Delta_r}G,\eqlbl{eq:ell'}\]
where $\Delta_r$ is the sector in $\Sigma$ in the polar coordinates at $p$
\[\set{w(t,s)}{a\le t\le b,\ 0\le s\le r};\]
which is bounded by two geodesics from $p$ with angle $b-a$ 
and a circular arc that meets these geodesics at right angle.

Since the plane has vanishing Gauss curvature, we have
\[\~\ell'(r)=2\cdot (b-a),\eqlbl{eq:tilde-ell'}\]
which agrees with the formula for the length of the arc $\~\ell(r)\z=2\cdot\pi\cdot r$.

If the Gauss curvature of $\Sigma$ is nonnegative,
the equations \ref{eq:ell'} and \ref{eq:tilde-ell'} imply that
\[\ell'(r)\le \~\ell'(r)\]
for any small $r$.

If the Gauss curvature of $\Sigma$ is nonnegative,
the same equations imply that
\[\ell'(r)\ge \~\ell'(r)\]
for any small $r$.

Since $\ell(0)=\~\ell(0)$, integrating the inequalities proves both statements.\qeds

The following exercise provides a stronger statement.
It almost follow from the proof above, but one has to make an extra observation.


\begin{thm}{Exercise}
Assume $\Sigma$ is a smooth regular surface and $p\in\Sigma$,
denote by $\ell(r)$ the circumference of the circle with the center at $p$ and radius $r$ in $\Sigma$
and let $\~\ell(r)=2\cdot\pi\cdot r$ the circumference of the plane circle of radius $r$.

\begin{enumerate}[(i)]
 \item Show that if Gauss curvature of $\Sigma$ is nonnegative, then the function $r\mapsto \ell(r)$ is concave. Conclude that the function $r\mapsto \frac{\ell(r)}{\~\ell(r)}$ is nonincreasing.
\item Show that if Gauss curvature of $\Sigma$ is nonpositive, then the function $r\mapsto \ell(r)$ is convex. Conclude that the function $r\mapsto \frac{\ell(r)}{\~\ell(r)}$ is nondecresing.
\end{enumerate}

\end{thm}
