\chapter{Comparison theorems}
\label{chap:comparison}

This chapter based on material in the book os Stephanie Alexander, Vitali Kapovitch and the first author \cite{alexander-kapovitch-petrunin2027}.

\section{Triangles and hinges}

Recall that a shortest path between points $x$ and $y$ in a surface $\Sigma$ is denoted as $[x,y]$ or $[x,y]_\Sigma$, and
$|x-y|_\Sigma$ denotes the \index{intrinsic distance}\emph{intrinsic distance} from $x$ to $y$ in $\Sigma$.

A \index{geodesic triangle}\emph{geodesic triangle} in a surface $\Sigma$ is defined as a triple of points $x,y,z\z\in \Sigma$ with choice of minimizing geodesics $[x,y]_\Sigma$, $[y,z]_\Sigma$ and $[z,x]_\Sigma$.
The points $x,y,z$ are called {}\emph{vertexes} of the geodesic triangle,
the minimizing geodesics $[x,y]$, $[y,z]$ and $[z,x]$ are called its {}\emph{sides};
the triangle itself is denoted by $[xyz]$, or by $[xyz]_\Sigma$, if we need to emphasize that it lies in $\Sigma$.\index{10aad@$[xyz]$, $[xyz]_\Sigma$}

A triangle $[\tilde x\tilde y\tilde z]$ in the plane $\RR^2$ is called \index{model triangle}\emph{model triangle} of the triangle $[xyz]$
if its corresponding sides are equal;
that is,
\[|\tilde x-\tilde y|_{\RR^2}=|x-y|_\Sigma,
\quad
|\tilde y-\tilde z|_{\RR^2}=|y-z|_\Sigma,
\quad
|\tilde z-\tilde x|_{\RR^2}=|z-x|_\Sigma.
\]
In this case we write $[\tilde x\tilde y\tilde z]=\tilde\triangle xyz$.
\index{10aae@$\tilde\triangle$}

A pair of minimizing geodesics $[x,y]$ and $[x,z]$ starting from one point $x$ is called \index{hinge}\emph{hinge} and denoted as $\hinge xyz$.\index{10aac@$\hinge yxz$}
The angle between these geodesics at $x$ is denoted by $\measuredangle\hinge xyz$.
The corresponding angle $\measuredangle\hinge {\tilde x}{\tilde y}{\tilde z}$ in the model triangle $[\tilde x\tilde y\tilde z]=\tilde\triangle xyz$ is denoted by $\angk xyz$.
\index{10aab@$\angk xyz$}

\section{Formulations}

Part \textit{\ref{SHORT.thm:comp:toponogov}} of the following theorem is called \index{Toponogov comparison theorem}\emph{Toponogov comparison theorem} and sometimes \index{Alexandrov comparison theorem}\emph{Alexandrov comparison theorem};
it was proved by Paolo Pizzetti \cite{pizzetti} and rediscovered by Alexandr Alexandrov \cite{alexandrov}; 
generalizations were obtained by  Victor Toponogov \cite{toponogov1957}, Mikhael Gromov, Yuri Burago and Grigory Perelman \cite{burago-gromov-perelman}.

Part \textit{\ref{SHORT.thm:comp:cat}} is called \index{Cartan--Hadamard theorem}\emph{Cartan--Hadamard theorem};
it was proved by 
Hans von Mangoldt \cite{mangoldt} and generalized by Elie Cartan \cite{cartan}, Jacques Hadamard \cite{hadamard},
Herbert Busemann \cite{busemann},
Willi Rinow \cite{rinow},
Mikhael Gromov \cite[p.~119]{gromov-1987},
Stephanie Alexander and Richard Bishop \cite{alexander-bishop1990}.

Recall that a surface $\Sigma$ is called {}\emph{simply connected} if any closed simple curve in $\Sigma$ bounds a disc.

\begin{thm}{Comparison theorems}\label{thm:comp}
Let $\Sigma$ be a proper smooth regular surface.

\begin{subthm}{thm:comp:cat}
If $\Sigma$ is simply connected and has nonpositive Gauss curvature,
 then 
\[\measuredangle\hinge {x}{y}{z}\le\angk xyz\]
for any geodesic triangle $[xyz]$.
\end{subthm}

\begin{subthm}{thm:comp:toponogov}
If $\Sigma$ has nonnegative Gauss curvature, then 
 \[\measuredangle\hinge {x}{y}{z}\ge\angk xyz\]
for any geodesic triangle $[xyz]$.
\end{subthm}

\end{thm}

The proof of part \ref{SHORT.thm:comp:cat} will be given at the end of Section~\ref{sec:nonpos-comp}.
The proof of part \ref{SHORT.thm:comp:toponogov} will be finished in  Section~\ref{sec:nonneg-comp}.

Let us show that the statement \ref{SHORT.thm:comp:cat} does not hold without assuming that $\Sigma$ is simply connected.
Consider the hyperboloid
\[\set{(x,y,z)\in\RR^3}{x^2+y^2-z^2=1};\]
it has negative Gauss curvature, but it is not simply connected.
The equator $z=0$ of the hyperboloid
forms a triangle with all angles $\pi$, which does not meet the conclusion in~\ref{SHORT.thm:comp:cat}.

Let us discuss the relations between Gauss--Bonnet formula and the comparison theorems.
Suppose that a disc $\Delta$ is bounded by a geodesic triangle $[xyz]$ with internal angles $\alpha$, $\beta$ and $\gamma$.
Then Gauss--Bonnet implies that 
\[\alpha+\beta+\gamma-\pi=\iint_\Delta K;\]
in particular both sides of the equation have the same sign.
It follows that
\begin{itemize}
\item if $K_\Sigma\ge 0$ then $\alpha+\beta+\gamma\ge\pi$, and
\item if $K_\Sigma\le 0$ then $\alpha+\beta+\gamma\le\pi$.
\end{itemize}

Now set 
$\hat\alpha=\measuredangle\hinge {x}{y}{z}$,
$\hat\beta=\measuredangle\hinge {y}{z}{x}$,
and $\hat\gamma=\measuredangle\hinge {z}{x}{y}$.
Since the angles of any plane triangle sum up to $\pi$, from the comparison theorems, we get that
\begin{itemize}
\item if $K_\Sigma\ge 0$ then $\hat\alpha+\hat\beta+\hat\gamma\ge\pi$, and
\item if $K_\Sigma\le 0$ then $\hat\alpha+\hat\beta+\hat\gamma\le\pi$.
\end{itemize}

The angles for comparison theorems can be found using internal angles: 
\begin{align*}
\hat \alpha&=\min\{\,\alpha,2\cdot\pi-\alpha\,\},
&
\hat\beta &=\min\{\,\beta,2\cdot\pi-\beta\,\},
&
\hat\gamma&=\min\{\,\gamma,2\cdot\pi-\gamma\,\}.
\end{align*}
One can use it to see that despite Gauss--Bonnet formula and the comparison theorems are closely related,
this relation is not straightforward.

For example, suppose $K\ge 0$.
Then the Gauss--Bonnet formula does not forbid the internal angles $\alpha$, $\beta$, and $\gamma$ to be close to $2\cdot\pi$.
But if $\alpha$, $\beta$, and $\gamma$ are close to $2\cdot\pi$, then $\hat\alpha$, $\hat\beta$, and $\hat\gamma$ are close to $0$.
The latter is impossible by the comparison theorem.

The following exercise is a lemma from the note of Arseniy Akopyan and the first author \cite{akopyan-petrunin}.

\begin{thm}{Exercise}\label{ex:diam-angle}
Let $p$ and $q$ be points on a closed convex surface $\Sigma$ that lie on maximal intrinsic distance from each other;
that is, $|p\z-q|_\Sigma\z\ge|x-y|_\Sigma$ for any $x,y\in \Sigma$.
Show that 
\[\measuredangle\hinge xpq\ge \tfrac\pi3\]
for any point $x\in \Sigma\backslash\{p,q\}$.
\end{thm}

\begin{thm}{Exercise}\label{ex:sum=<2pi}
Let $\Sigma$ be a closed (or open) regular surface and with nonnegative Gauss curvature.
Show that 
\[\angk pxy+\angk pyz+\angk pzx\le2\cdot \pi.\]
for any four distinct points $p,x,y,z$ in $\Sigma$.
\end{thm}

\section{Local comparisons}\label{sec:loc-comp}

The following local version of comparison theorem follows from the Rauch comparison (\ref{prop:rauch}).
It will be used in the proof of the global version (\ref{thm:comp}).

\begin{thm}{Theorem}\label{thm:loc-comp}
The comparison theorem (\ref{thm:comp}) holds in a small neighborhood of any point.

Moreover, suppose $\Sigma$ be a smooth regular surface without boundary,
then for any $p\in \Sigma$ there is $r>0$ such that if $|p-x|_\Sigma<r$, then $\inj(x)_\Sigma>r$ and the following statements hold:


\begin{subthm}{thm:loc-comp:cba}
If $\Sigma$ has nonpositive Gauss curvature, then 
\[\measuredangle\hinge {x}{y}{z}\le\angk xyz\]
for any geodesic triangle $[xyz]$  in $B(p,\tfrac r4)_\Sigma$.
\end{subthm}

\begin{subthm}{thm:loc-comp:cbb}
If $\Sigma$ has nonnegative Gauss curvature, then 
\[\measuredangle\hinge {x}{y}{z}\ge\angk xyz\]
for any geodesic triangle $[xyz]$ in $B(p,\tfrac r4)_\Sigma$.
\end{subthm}


\end{thm}


\parit{Proof.}
The existence of $r>0$ follows from \ref{prop:exp}.

Note that $y=\exp_x\vec v$ and $z=\exp_x\vec w$ for two vectors $\vec v,\vec w\in\T_x$ such that 
\begin{align*}
\measuredangle\hinge 0{\vec v}{\vec w}_{\T_x}&=\measuredangle\hinge xyz_\Sigma,
\\
|\vec v|_{\T_x}&=|x-y|_\Sigma, 
\\
|\vec w|_{\T_x}&=|x-z|_\Sigma;
\end{align*}
in particular, $|\vec v|, |\vec w|< \tfrac r2$.
\parit{\ref{SHORT.thm:loc-comp:cbb}.}
Consider the line segment $\tilde \gamma$ joining $\vec v$ to $\vec w$ in the tangent plane $\T_x$ and set $\gamma=\exp_x\circ\tilde \gamma$.
By Rauch comparison (\ref{prop:rauch:K>=0}), we have
\[\length\gamma\le \length \tilde \gamma.\]
Since $|\vec v-\vec w|_{\T_x}=\length\tilde \gamma$ and $|y-z|_{\Sigma}\le \length\gamma$, we get 
\[|\vec v-\vec w|_{\T_x}\ge |y-z|_\Sigma.\]
By the angle monotonicity (\ref{lem:angle-monotonicity}), we get
\[\angk xzy\le\measuredangle\hinge 0{\vec v}{\vec w}_{\T_x},\]
whence the result.

\parit{\ref{SHORT.thm:loc-comp:cba}.}
Consider a minimizing geodesic $\gamma$ joining $y$ to $z$ in $\Sigma$.
Since $|x\z-y|,|x\z-z|\z<\tfrac r2$, the triangle inequality implies that $\gamma$ lies in $r$-neighborhood of $x$.
In particular, $\log_x\circ\gamma$ is defined, and the curve
$\tilde \gamma=\log_x\circ\gamma$ lies in a $r$-neighborhood of zero in $\T_x$ that corresponds to $\gamma$.
Note that $\tilde\gamma$ connects $\vec v$ to $\vec w$ in $\T_x$.

By Rauch comparison (\ref{prop:rauch:K=<0}), we have
\[\length\gamma\ge \length \tilde \gamma.\]
Since $|\vec v-\vec w|_{\T_x}\le\length\tilde \gamma$ and $|y-z|_{\Sigma}= \length\gamma$, we get 
\[|\vec v-\vec w|_{\T_x}\ge |y-z|_\Sigma.\]
By angle monotonicity (\ref{lem:angle-monotonicity}), we get
\[\angk xzy\ge\measuredangle\hinge 0{\vec v}{\vec w}_{\T_x}.\]
whence the result.
\qeds

\section{Nonpositive curvature}\label{sec:nonpos-comp}

\parit{Proof of \ref{thm:comp:cat}.}
Sine $\Sigma$ is simply connected, \ref{ex:unique-geod} implies that 
\[\inj(p)_\Sigma=\infty\]
for any $p\in\Sigma$.
Therefore \ref{SHORT.thm:loc-comp:cba} implies \ref{thm:comp:cat}.
\qeds

\section{Nonnegative curvature}\label{sec:nonneg-comp}

We will prove \ref{thm:comp:toponogov}, first assuming that $\Sigma$ is compact.
The general case requires only minor modifications; they are indicated in Exercise \ref{ex:open-comparison} at the end of the section.
The proof is taken from \cite{alexander-kapovitch-petrunin2027} and it is very close to the proof given by Urs Lang and Viktor Schroeder \cite{lang-schroeder}.

\parit{Proof of \ref{thm:comp:toponogov} in the compact case.}\label{proof(thm:comp:toponogov)}
Assume $\Sigma$ is compact. 
From the local theorem (\ref{thm:loc-comp}), we get that there is $\epsilon>0$ such that the inequality 
\[\measuredangle\hinge {x}{p}{q}\ge\angk xpq.\]
holds for any hinge $\hinge {x}{p}{q}$ such that $|x-p|+|x-q|<\epsilon$.
The following lemma states that in this case the same holds true for any hinge $\hinge {x}{p}{q}$ such that $|x-p|+|x-q|<\tfrac32\cdot\epsilon$.
Applying the key lemma (\ref{key-lem:globalization}) a few times we will get that the comparison holds for arbitrary hinge, which proves \mbox{\ref{thm:comp:toponogov}}.
\qeds

\begin{thm}{Key lemma}\label{key-lem:globalization}  
Let $\Sigma$ be a proper smooth regular surface.
Assume that the comparison
\[\measuredangle\hinge x y z
\ge\angk x y z\eqlbl{eq:key-lem:globalization}\]
holds for any hinge $\hinge x y z$ with 
$\dist{x}{y}{}+\dist{x}{z}{}
<
\frac{2}{3}\cdot\ell$.
Then the comparison \ref{eq:key-lem:globalization}
holds for any hinge $\hinge x y z$ with $\dist{x}{y}{}+\dist{x}{z}{}<\ell$.
\end{thm}

\begin{wrapfigure}{r}{35mm}
\centering
\includegraphics{mppics/pic-2308}
\end{wrapfigure}

\parit{Proof.} 
Given a hinge $\hinge x p q$ consider a triangle in the plane
with angle $\measuredangle\hinge x p q$ and two adjacent sides $|x-p|$ and $|x-q|$.
Let us denote by $\side \hinge x p q$ the third side of this triangle; let us call it the \index{model side}\emph{model side} of the hinge.

Note that the inequalities 
\[\measuredangle\hinge x p q\ge \angk x p q\quad\iff\quad\side \hinge x p q
\ge\dist{p}{q}{}.\]
So it is sufficient to prove that
\[\side \hinge x p q
\ge\dist{p}{q}{}.\eqlbl{eq:thm:=def-loc*}\] 
for any hinge $\hinge x p q$ with $\dist{x}{p}{}+\dist{x}{q}{}<\ell$.

\parit{Construction.}
Let us describe a construction that produce a new hinge $\hinge{x'}p q$ for a given hinge $\hinge x p q$ such that 
\[\tfrac{2}{3}\cdot\ell \le\dist{p}{x}{}\z+\dist{x}{q}{}< \ell.\]
The new hinge $\hinge{x'}p q$ will satisfy the following properties:
\[\side \hinge x p q
\ge\side \hinge{x'}p q
\eqlbl{eq:thm:=def-loc-fivestar}\]
and 
$x'\in [x,p]\z\cup [x,q]$. In particular, the triangle inequality implies that 
\[
\dist{p}{x}{}+\dist{x}{q}{}\ge\dist{p}{x'}{}+\dist{x'}{q}{}.
\eqlbl{eq:thm:=def-loc-fourstar}\]

Assume $\dist{x}{q}{}\ge\dist{x}{p}{}$, otherwise switch the roles of $p$ and $q$ in the following construction.
Take $x'\in [x, q]$ such that 
\[\dist{p}{x}{}+3\cdot\dist{x}{x'}{}
=\tfrac{2}{3}\cdot\ell \eqlbl{3|xx'|}\]
Choose a geodesic $[x', p]$ and consider the  hinge $\hinge{x'}p q$ formed by $[x',p]$ and $[x',q]\subset [x,q]$. 

By \ref{3|xx'|}, we have that 
\begin{align*}
\dist{p}{x}{\Sigma}\z+\dist{x}{x'}{\Sigma}&<\tfrac{2}{3}\cdot\ell,
\\
\dist{p}{x'}{\Sigma}\z+\dist{x'}{x}{\Sigma}&<\tfrac{2}{3}\cdot\ell.
\end{align*}
In particular, 
\[\measuredangle\hinge x p{x'}
\ge\angk x p{x'}
\ \ \text{and}\ \ 
\measuredangle\hinge {x'}p x
\ge\angk {x'}p x.
\eqlbl{eq:thm:=def-loc-threestar}\]

\begin{wrapfigure}{r}{35mm}
\vskip -4mm
\centering
\includegraphics{mppics/pic-2310}
\end{wrapfigure}

Consider the model triangle
$[\tilde x\tilde x'\tilde p]\z=\modtrig xx'p$.
Take $\tilde  q$ on the extension of $[\tilde  x,\tilde  x']$ beyond $x'$ such that $\dist{\tilde x}{\tilde q}{}=\dist{x}{q}{}$ (and therefore $\dist{\tilde x'}{\tilde q}{}=\dist{x'}{q}{}$).
From \ref{eq:thm:=def-loc-threestar},
\[\measuredangle\hinge x p q
=\measuredangle\hinge  x p{x'}\ge\angk x p{x'}\ \ \Rightarrow\ \ 
\side \hinge x q p\ge\dist{\tilde p}{\tilde q}{}.\]
Since $\measuredangle\hinge{x'}p x+\measuredangle\hinge{x'}p q= \pi$,
\ref{eq:thm:=def-loc-threestar} implies
\[
\pi
-\angk{x'}p x
\ge
\pi-\measuredangle\hinge{x'}p x
\ge
\measuredangle\hinge{x'}p q.
\]
Therefore
$\dist{\tilde p}{\tilde q}{}\ge\side \hinge{x'}q p$ and \ref{eq:thm:=def-loc-fivestar} follows.
Hence the constructed hinge $\hinge{x'}q p$ satisfies the declared properties.

\begin{figure}[h!]
\centering
\includegraphics{mppics/pic-2315}
\end{figure}

Set $x_0=x$.
Let us apply inductively the above construction to get a sequence of hinges  $\hinge{x_n}p q$ with $x_{n+1}=x_n'$.
By \ref{eq:thm:=def-loc-fivestar} and triangle inequality, both sequences
\[s_n=\side \hinge{x_n}pq\quad\text{and}\quad r_n=\dist{p}{x_n}{}+\dist{x_n}{q}{}\]
are nonincreasing.

The sequence might terminate at some $n$ only if $r_n< \tfrac{2}{3}\cdot\ell $.
In this case, by the assumptions of the lemma, 
\[s_n=\side \hinge{x_n}p q\ge\dist{p}{q}{}.\]
Since sequence $s_n$ is nonincreasing;
\[s_0=\side \hinge{x}p q\ge\dist{p}{q}{},\]
whence inequality \ref{eq:thm:=def-loc*} follows.

If the sequence does not terminate, then $r_n\ge\tfrac{2}{3}\cdot\ell$ for all $n$.
Since $(r_n)$ is nonincreasing, $r_n\to r\ge |p-q|_\Sigma$ as $n\to\infty$.

Let us show that $\measuredangle\hinge{x_n}p q\to \pi$ as $n\to\infty$.

Indeed assume $\measuredangle\hinge{x_n}p q\le \pi-\epsilon$ for some $\epsilon>0$.
Without loss of generality we can assume that $x_{n+1}\in [x_n,q]$;
otherwise switch $p$ and $q$ further.
Note that $|x_n-x_{n+1}|,|p-x_n|>\tfrac\ell{100}$.
Therefore by comparison 
\[|p-x_{n+1}|<\side\hinge{x_n}p{x_{n+1}}<|p-x_n|+|x_n-x_{n+1}|-\delta\]
for some fixed $\delta=\delta(\epsilon)>0$.
Therefore $r_n-r_{n+1}>\delta$.
The latter cannot hold for large $n$, otherwise the sequence $r_n$ would not converge.

It follows that for any $\epsilon>0$ we have that $\measuredangle\hinge{x_n}p q> \pi-\epsilon$ for all large $n$;
that is, $\measuredangle\hinge{x_n}p q\to \pi$ as $n\to\infty$.

Since $\measuredangle\hinge{x_n}p q\to \pi$, we have 
$s_n-r_n\to 0$ as $n\to\infty$;
that is, $s_n\to r$.

Since the sequence $(s_n)$ is nonincreasing and $r\ge |p-q|$, we get
\[s_n\ge |p-q|\]
for any $n$.
In particular
\[\side\hinge spq=s_0\ge |p-q|,\] so we obtain \ref{eq:thm:=def-loc*}.
\qeds

\begin{thm}{Exercise}\label{ex:open-comparison}
Let $\Sigma$ be an open surface with nonnegative Gauss curvature.
Given $p\in\Sigma$, denote by $R_p$ 
(the {}\emph{comparison radius} at $p$) 
the maximal value (possibly $\infty$) such that the comparison 
\[\measuredangle\hinge x p y
\ge\angk x p y\]
holds for any hinge $\hinge x p y$ with $\dist{p}{x}{}+\dist{x}{y}{}<R_p$.

\begin{subthm}{ex:open-comparison:positive}
Show that for any compact subset $K\subset \Sigma$, there is $\epsilon>0$ such that $R_p>\epsilon$ for any $p\in K$.
\end{subthm}

\begin{subthm}{ex:open-comparison:almost-min}
Use part \ref{SHORT.ex:open-comparison:positive} to show that 
there is a point $p\in\Sigma$ such that 
\[R_q>(1-\tfrac1{100})\cdot R_p,\]
for any $q\in B(p,100\cdot R_p)_\Sigma$.
\end{subthm}

\begin{subthm}{ex:open-comparison:proof}
Use \ref{SHORT.ex:open-comparison:almost-min} to extend the proof of \ref{thm:comp:toponogov} (page \pageref{proof(thm:comp:toponogov)}) to open surfaces. 
(Show that $R_p=\infty$ for any $p\in\Sigma$.) 
\end{subthm}


\end{thm}


\begin{wrapfigure}{r}{30 mm}
\vskip4mm
\centering
\includegraphics{mppics/pic-471}
\end{wrapfigure}

\begin{thm}{Advanced exercise}\label{ex:convex-polyhon+self-intersections}
Let $\Sigma$ be a closed smooth regular surface with nonnegative Gauss curvature.
The following sketch shows that a closed geodesic $\gamma$ on $\Sigma$ cannot have self-intersections as shown on the diagram;
in other words, $\gamma$ cannot cut $\Sigma$ into 3 monogons, one quadrangle, and one pentagon. 

Make a complete proof from it.

Arguing by contradiction, suppose that such geodesic exists;
assume that arcs and angles are labeled as on the left diagram.

\begin{figure}[h!]
\begin{minipage}{.38\textwidth}
\centering
\includegraphics{mppics/pic-472}
\end{minipage}\hfill
\begin{minipage}{.58\textwidth}
\centering
\includegraphics{mppics/pic-473}
\end{minipage}
\end{figure}

\begin{subthm}{ex:convex-polyhon+self-intersections:angles}
Apply Gauss--Bonnet formula to show that
\[2\cdot\alpha<\beta+\gamma\]
and 
\[2\cdot\beta+2\cdot \gamma<\pi+\alpha.\]
Conclude that $\alpha <\tfrac \pi 3$.
\end{subthm}

\begin{subthm}{ex:convex-polyhon}
Consider the part of the geodesic $\gamma$ without the arc~$a$.
It cuts from $\Sigma$ a pentagon $\Delta$ with sides and angles as shown on the diagram. 
Show that there is a plane pentagon with convex sides of the same length and angles at most as big as the corresponding angles of $\Delta$.
\end{subthm}

\begin{subthm}{ex:self-intersections-hard}
Arrive to a contradiction using \ref{SHORT.ex:convex-polyhon+self-intersections:angles} and \ref{SHORT.ex:convex-polyhon}. 
\end{subthm}

\end{thm}

\section{Alexandrov's lemma}

A reformulation of the following lemma in plane geometry will be used in the next section to produce a few equivalent formulations of comparison theorems.

\begin{thm}{Lemma}
\label{lem:alex}
Assume $[pxyz]$ and $[p'x'y'z']$ be two quadrilaterals in the plane with equal corresponding sides.
Assume that the sides $[x',y']$ and $[y',z']$ extend each other; that is, $y'$ lies on the line segment $[x',z']$.
Then the following expressions have the same signs:
\begin{enumerate}[(i)]
 \item $|p-y|-|p'-y'|$;
 \item $\measuredangle\hinge xpy-\measuredangle\hinge {x'}{p'}{y'}$;
 \item $\pi-\measuredangle\hinge ypx-\measuredangle\hinge ypz$;
\end{enumerate}
\end{thm}

\parit{Proof.} 
Take 
a point $\bar z$ on the extension of 
$[x,y]$ beyond $y$ so that $\dist{y}{\bar z}{}=\dist{y}{z}{}$ (and therefore $\dist{x}{\bar z}{}=\dist{x'}{z'}{}$). 
 
\begin{figure}[h!]
\vskip-0mm
\centering
\includegraphics{mppics/pic-50}
\vskip-0mm
\end{figure}

From angle monotonicity (\ref{lem:angle-monotonicity}), 
the following expressions have the same sign:
\begin{enumerate}[(i)]
\item $|p-y|-|p'-y'|$;
\item $\measuredangle\hinge{x}{y}{p}-\measuredangle\hinge{x'}{y'}{p'}=\measuredangle\hinge{x}{\bar z}{p}-\measuredangle\hinge{x'}{z'}{p'}$;
\item $|p-\bar z|-|p'-z'|$;
\item $\measuredangle\hinge{y}{\bar z}{p}-\measuredangle\hinge{y'}{z'}{p'}$;
\end{enumerate}
The statement follows since
\[\measuredangle\hinge{y'}{z'}{p'}+\measuredangle\hinge{y'}{x'}{p'}=\pi\]
and
\[\measuredangle\hinge{y}{\bar z}{p}+\measuredangle\hinge{y}{x}{p}=\pi.\]
\qedsf

\section{Reformulations}

In this section we formulate conditions equivalent to the conclusion of the comparison theorem (\ref{thm:comp}).

A triangle $[xyz]$ in a surface is called \index{fat triangle}\emph{fat} (\index{thin triangle}\emph{thin})
if for any two points $p$ and $q$ on the sides of the triangle and the corresponding points 
$\tilde p$ and $\tilde q$ on the sides of its model triangle $[\tilde x\tilde y\tilde z]\z=\tilde\triangle xyz$ we have
$|p-q|\ge |\tilde p-\tilde q|$ (or, respectively, $|p-q|\le |\tilde p-\tilde q|$).


\begin{thm}{Proposition}\label{prop:comp-reformulations}
Let $\Sigma$ be a proper smooth regular surface.
Then the following three conditions are equivalent:

\begin{subthm}{mang>angk}
For any geodesic triangle $[xyz]$ in $\Sigma$ we have
 \[\measuredangle\hinge {x}{y}{z}\ge\angk xyz.\]
\end{subthm}

\begin{subthm}{angk>angk} For any geodesic triangle $[pxz]$ in $\Sigma$ and $y$ on the side $[x,z]$ we have
 \[\angk xpy \ge \angk xpz.\]
 
\end{subthm}

\begin{subthm}{fat}
 Any geodesic triangle in $\Sigma$ is fat.
\end{subthm}


\medskip

Similarly, following three conditions are equivalent:

\begin{subthmA}{mang<angk}
For any geodesic triangle $[xyz]$ in $\Sigma$ we have
 \[\measuredangle\hinge {x}{y}{z}\le\angk xyz.\]
\end{subthmA}

\begin{subthmA}{angk<angk} For any geodesic triangle $[pxz]$ in $\Sigma$ and $y$ on the side $[x,z]$ we have
 \[\angk xpy \le \angk xpz.\]
\end{subthmA}

\begin{subthmA}{thin}
Any geodesic triangle in $\Sigma$ is thin.
\end{subthmA}

\end{thm}

In the proof we will use the following translation of Alexandrov lemma to the language of comparison triangles and angles.

\begin{wrapfigure}{r}{25mm}
\vskip-0mm
\centering
\includegraphics{mppics/pic-2305}
\end{wrapfigure}

\begin{thm}{Reformulation of Alexandrov lemma}\label{lem:alex-reformulation}\\
Assume $[pxz]$ be a triangle in a surface $\Sigma$ and 
the point $y$ lies on the side $[x,z]$.
Consider its model triangle $[\tilde p\tilde x\tilde z]=\tilde\triangle pxz$ and let $\tilde y$ be the corresponding point on the side $[\tilde x,\tilde z]$.
Then the following expressions have the same signs:
\begin{enumerate}[(i)]
 \item $|p-y|_\Sigma-|\tilde p-\tilde y|_{\RR^2}$;
 \item $\angk xpy-\angk {x}{p}{z}$;
 \item $\pi-\angk ypx-\angk ypz$;
\end{enumerate}
\end{thm}

\parit{Proof or \ref{prop:comp-reformulations}.}
We will prove the implications \ref{SHORT.mang>angk}$\Rightarrow$\ref{SHORT.angk>angk}$\Rightarrow$\ref{SHORT.fat}$\Rightarrow$\ref{SHORT.mang>angk}.
The implications \ref{SHORT.mang<angk}$\Rightarrow$\ref{SHORT.angk<angk}$\Rightarrow$\ref{SHORT.thin}$\Rightarrow$\ref{SHORT.mang<angk} can be done the same way.

\parit{\ref{SHORT.mang>angk}$\Rightarrow$\ref{SHORT.angk>angk}.}
Note that $\measuredangle\hinge ypx+\measuredangle\hinge ypz=\pi$.
By \textit{\ref{SHORT.mang>angk}}, 
\[\angk ypx+\angk ypz\le \pi.\]
It remains to apply Alexandrov's lemma (\ref{lem:alex-reformulation}).


\parit{\ref{SHORT.angk>angk}$\Rightarrow$\ref{SHORT.fat}.}
Applying \textit{\ref{SHORT.mang>angk}} twice, first for $y\in [x,z]$ and then for $w\in [p,x]$, we get that
\[\angk xwy \ge \angk xpy \ge \angk xpz\]
and therefore
\[|w-y|_\Sigma\ge |\tilde w-\tilde y|_{\RR^2},\]
where $\tilde w$ and $\tilde y$ are the points corresponding to $w$ and $y$ points on the sides of the model triangle. 

\parit{\ref{SHORT.fat}$\Rightarrow$\ref{SHORT.mang>angk}.}
Since the triangle is fat, we have 
\[\angk xwy \ge \angk xpz\]
for any $w\in \left]xp\right]$ and $y\in \left]xz\right]$.
Note that $\angk xwy\to \measuredangle\hinge xpz$ as $w,y\to x$.
Whence the implication follows.
\qeds

\begin{thm}{Exercise}\label{ex:geod-convexity}
Let $\Sigma$ be a proper smooth regular surface
and $\gamma$ be a unit-speed geodesic in $\Sigma$ and $p\in\Sigma$.

Consider the function
\[h(t)=|p-\gamma(t)|_\Sigma^2-t^2.\]

\begin{subthm}{}
Show that if $\Sigma$ is simply connected and the Gauss curvature of $\Sigma$ is nonpositive, then the function $h$ is convex.
\end{subthm}

\begin{subthm}{} Show that if the Gauss curvature of $\Sigma$ is nonnegative, then the function $h$ is concave.
\end{subthm}

\end{thm}


\begin{thm}{Exercise}\label{ex:midpoints}
Let $\bar x$ and $\bar y$ be the midpoints of minimizing geodesics $[p,x]$ and $[p,y]$ in an open smooth regular surface $\Sigma$.

\begin{subthm}{}
 Show that if $\Sigma$ is simply connected and has nonpositive Gauss curvature, then 
 \[2\cdot |\bar x-\bar y|_\Sigma\le |x-y|_\Sigma.\]
 \end{subthm}
 
\begin{subthm}{} Show that if the Gauss curvature of $\Sigma$ is nonnegative, then 
 \[2\cdot |\bar x-\bar y|_\Sigma\ge |x-y|_\Sigma.\]
\end{subthm}

\end{thm}

\begin{thm}{Exercise}\label{ex:convex-dist}
Assume $\gamma_1$ and $\gamma_2$ are two geodesics in an open smooth regular simply connected surface $\Sigma$ with nonpositive Gauss curvature.
Show that the function
\[h(t)=|\gamma_1(t)-\gamma_2(t)|_\Sigma\]
is convex.
\end{thm}

\begin{thm}{Advanced exercise}\label{ex:dist-to-bry}
Suppose that a smooth curve $\delta$ bounds a convex disc $\Delta$ in a proper smooth surface $\Sigma$;
that is, for any two points $x,y\in\Delta$, any shortest path $[x,y]_\Sigma$ lies in $\Delta$.

Denote by $f\:\Sigma\to\RR$ the intrinsic distance function to $\delta$;
that is, 
\[f(x)=\inf_{y\in\delta}\{\,|x-y|_\Sigma\,\}.\]

Show that

\begin{subthm}{ex:dist-to-bry:-}
If $\Sigma$ is simply connected and $K_\Sigma\le 0$, then the restriction of $f$ to the complement of $\Delta$ is convex;
that is, for any geodesic $\gamma$ in $\Sigma\backslash\Delta$ the function $t\mapsto f\circ\gamma(t)$ is convex.
\end{subthm}

\begin{subthm}{ex:dist-to-bry:+}
If $K_\Sigma\ge 0$, then the restriction of $f$ to $\Delta$ is concave;
that is, for any geodesic $\gamma$ in $\Delta$ the function $t\mapsto f\circ\gamma(t)$ is concave.
\end{subthm}

\end{thm}

\section{Busemann functions} 

A unit-speed geodesic $\lambda\:[0,\infty)\to \spc{X}$ is called a \index{half-line}\emph{half-line} if it is minimizing on each interval $[a,b]\subset [0,\infty)$.


\begin{thm}{Proposition}\label{prop:busemann}
Suppose that $\lambda\:[0,\infty)\to \Sigma$ is a half-line in a smooth regular surface $\Sigma$. 
Then the function 
\[\bus_\lambda(x)=\lim_{t\to\infty}\dist{\lambda(t)}{x}{\Sigma}- t\eqlbl{eq:def:busemann*}\]
is defined.

Moreover,

\begin{subthm}{prop:busemann:all}
$\bus_\lambda$ is a $1$-Lipschitz function and
\[\bus_\lambda\circ\lambda(t)+t=0\] 
for any $t$.
\end{subthm}

\begin{subthm}{prop:busemann-}
If $\Sigma$ is an open simply connected surface with nonpositive Gauss curvature, then $\bus_\lambda$ is convex;
that is, for any geodesic $\alpha$ the real-to-real function 
$t\mapsto \bus_\lambda\circ\alpha(t)$ is convex.
\end{subthm}

\begin{subthm}{prop:busemann+}
If $\Sigma$ is an open surface with nonnegative Gauss curvature, then $\bus_\lambda$ is concave;
that is, for any geodesic $\alpha$ the real-to-real function 
$t\z\mapsto \bus_\lambda\circ\alpha(t)$ is concave.
\end{subthm}

\end{thm}

The function  $\bus_\lambda\:\Sigma\z\to \RR$ as in the proposition is called 
\index{Busemann function}\emph{Busemann function associated to $\lambda$}.
Intuitively the function $\bus_\lambda$ can be described as a distance function to the ideal point on the end of the half-line $\lambda$.

\parit{Proof.}
By the triangle inequality, the function 
\[t\mapsto\dist{\lambda(t)}{x}{}- t\] is nonincreasing in $t$.  
Clearly 
\[\dist{\lambda(t)}{x}{\Sigma}- t\ge-\dist{\lambda(0)}{x}{};\]
that is, for each $x$, the values $\dist{\lambda(t)}{x}{\Sigma}- t$ bounded below.
Thus the limit in \ref{eq:def:busemann*} is defined.

Observe that each function $x\mapsto \dist{\lambda(t)}{x}{\Sigma}- t$ is 1-Lipschitz.
Therefore its limit $x\mapsto\bus_\lambda(x)$ is Lipschitz as well.
The second part of \ref{SHORT.prop:busemann:all} is evident.


It remains to prove the last two statements.
Choose a geodesic~$\alpha$.
Given $t\ge 0$, consider the function $h_t(s)=|{\lambda(t)}-{\alpha(s)}|_{\Sigma}^2-s^2$.


Observe that for any fixed $x\in\Sigma$ we have $\frac{|\lambda(t)-x|}{t}\to 1$ as $t\to\infty$.
Therefore
\begin{align*}
\bus_\lambda\circ\alpha(s)
&=
\lim_{t\to\infty}\frac{h_t(s)}{t}
\end{align*}
According to \ref{ex:geod-convexity}, 
$s\mapsto h_t(s)$ is convex or concave the function assuming the conditions in  \ref{SHORT.prop:busemann-} or \ref{SHORT.prop:busemann+} respectively.
Whence \ref{SHORT.prop:busemann-} and \ref{SHORT.prop:busemann+} follow.
\qeds

\begin{thm}{Exercise}\label{ex:half-line}
Let $\Sigma$ be an open surface and $p\in\Sigma$.

\begin{subthm}{ex:half-line:exist}
Show that there is a half-line $\lambda$ in $\Sigma$ that starts at $p$.

Moreover, if $K\ni p$ is a noncompact closed convex subset of $\Sigma$, then there is a half-line of $\Sigma$ that starts at $p$ and runs in $K$.
\end{subthm}

\begin{subthm}{ex:half-line:suplevel}
Suppose $\Sigma$ has nonnegative Gauss curvature at any point.
Consider the function
\[f(x)=\inf_\lambda\bus_\lambda(x),\]
where the greatest lower bound is taken for all half-lines $\lambda$ that stat at $p$.
Show that $f$ is concave function and its suplevel sets 
\[S_c=\set{x\in\Sigma}{f(x)\ge c}\] 
are compact for any $c\in\RR$.
\end{subthm}

\begin{subthm}{ex:half-line:soul}
Let $s=\max\set{f(x)}{x\in\Sigma}$.
Show that the set $S_s$ is either one-point, a geodesic arc or a closed geodesic.
Show that all these possibilities can occur.
\end{subthm}



\end{thm}

\section{Line splitting theorem}

Let $\Sigma$ be a smooth regular surface.
A unit-speed geodesic $\lambda\:\RR\to\Sigma$ is called a \index{line}\emph{line} if it is length-minimizing on each interval $[a,b]\subset \RR$.

\begin{thm}{Line splitting theorem}\label{thm:splitting}
Let $\Sigma$  be an open smooth regular surface with nonnegative Gauss curvature
and $\lambda$ be a line in $\Sigma$. 
Then $\Sigma$ admits an intrinsic isometry to the Euclidean plane or a circular cylinder $\set{(x,y,z)\in \RR^2}{x^2+y^2=r^2}$ for some $r>0$.

In particular, $\Sigma$ has vanishing Gauss curvature.
\end{thm}


This theorem was proved by Stefan Cohn-Vossen \cite[Satz 5 in][]{convossen}
and it has a sequence of variations in differential geometry:
\begin{itemize}
 \item Victor Toponogov \cite{toponogov-globalization+splitting} proved a version of splitting theorem for Riemannian manifolds with nonnegative sectional curvature;
 \item Jeff Cheeger and Detlef Gromoll \cite{cheeger-gromoll-split} generalized it further to Riemannian manifolds with nonnegative Ricci curvature;
 \item Jost-Hinrich Eshenburg \cite{eshenburg-split} proved a splitting theorem for space-time with nonnegative Ricci curvature in timelike directions.
\end{itemize}


\parit{Proof.} 
Consider two Busemann functions $\bus_+$ and $\bus_-$ associated with half-lines $\lambda:[0,\infty)\to \Sigma$ and $\lambda:(-\infty,0]\to \Sigma$; that is,
\[
\bus_\pm(x)
=
\lim_{t\to\infty}\dist{\lambda(\pm t)}{x}{\Sigma}- t.
\]

\parit{Step 1.}
Let us show and use that
\[
\bus_+(x)+\bus_-(x)= 0
\eqlbl{eq:bus+-=0}
\]
for any $x\in \Sigma$.


Fix $x\in \Sigma$.
Since $\lambda$ is a line, the triangle inequality implies that
\begin{align*}
\dist{\lambda(t)}{x}{\Sigma}+\dist{\lambda(- t)}{x}{\Sigma}
&\ge \dist{\lambda(t)}{\lambda(-t)}{\Sigma}=
\\
&=2\cdot t.
\end{align*}
Passing to the limit as $t\to\infty$, we get
\[\bus_+(x)+\bus_-(x)\ge0.\]

On the other hand, by \ref{ex:geod-convexity}, we have 
$h(t)=|\lambda(t)-x|^2-t^2$ 
is concave.
In particular, 
\[|\lambda(t)-x|_\Sigma\le \sqrt{t^2+at+b}\]
for some constants $a,b\in\RR$. 
Passing to the limit as $t\to\pm\infty$, we get \[\bus_+(x)+\bus_-(x)\le0;\]
whence \ref{eq:bus+-=0} follows.

\parit{Conclusions.}
According to \ref{prop:busemann}, 
both functions $\bus_\pm$ are concave and $\bus_\pm\circ\lambda(t)=\mp t$ for any~$t$.
By \ref{eq:bus+-=0} both functions $\bus_\pm$ are affine;
that is, they are convex and concave at the same time.
It follows that the differential of $\bus_\pm$ is defined at any point $x\in\Sigma$;
that is, there is a linear function $\T_x\to\RR$ that is defined by
$\vec v\mapsto D_{\vec v}\bus_\pm$
for any tangent vector field $\vec v$.

Denote by $\vec u$ the gradient vector field of $\bus_-$;
that is, $\vec u$ is a tangent vector field such that for any tangent field $\vec v$ the following identity holds
\[\langle\vec u,\vec v\rangle=D_{\vec v}(\bus_-).\]

\parit{Step 2.}
Let us show that, the surface $\Sigma$ can be subdivided into lines that run in the direction of $\vec u$.

Fix a point $x$.
Given a real value $a$ choose a shortest path $[x,\lambda(a)]$;
denote by $\vec w^a\in\T_x$ the unit vector in the direction of geodesic $[x,\lambda(a)]$. 
Since $|\vec w^a|=1$ and $\bus_-$ is affine, we get that 
\begin{align*}
|\vec u|&\ge \limsup_{a\to\infty}\langle\vec u,\vec w^a\rangle =
\\
&=\limsup_{a\to\infty} D_{\vec w^a}\bus_-=
\\
&=
\limsup_{a\to\infty}\frac{\bus_-\circ\lambda(a)-\bus_-(x)}{|x-\lambda(a)|_\Sigma}=
\\
&=\limsup_{a\to\infty} \frac{a-\bus_-(x)}{a}=
\\
&=1.
\end{align*}
On the other hand, since $\bus_-$ is $1$-Lipschitz, we have $|\vec u|\le 1$.
Whence 
\[|\vec u|\equiv 1
\quad\text{and}\quad \lim_{a\to\infty}\langle\vec u,\vec w^a\rangle=1.\]
It follows that
\begin{align*}
\lim_{a\to\infty}\measuredangle(\vec u,\vec w^a)&=0;
\intertext{analogously, we get}
\lim_{a\to-\infty}\measuredangle(\vec u,\vec w^a)&=\pi.
\end{align*}

Set $b=\bus_-(x)$.
Consider a unit-speed geodesic $\zeta$
such that $\zeta'(b)\z=\vec u(x)$.
Since $\bus_-$ is affine, we have that $\bus_-\circ\zeta(t)=t$ for any $t$.
Since $\zeta$ is a unit-speed geodesic and $\bus_-$ is 1-Lipschitz, we get
\begin{align*}
|t_1-t_0|&\ge|\zeta(t_1)-\zeta(t_0)|_\Sigma\ge
\\
&\ge |\bus\circ\lambda(t_1)-\bus\circ\lambda(t_0)|=
\\
&=|t_1-t_0|.
\end{align*}
Whence $\zeta$ is a line for any $x$.
Moreover $\zeta$ always runs in the direction of $\vec u$.

\parit{Step 3.}
Let us show that the distances between points on two lines in the direction of $\vec u$ behave the same way as the distances between parallel lines in Euclidean plane; here is a precise formulation:

\begin{clm}{}\label{clm:parallel}
Let $\xi$ and $\zeta$ be two lines in $\Sigma$ that run in the direction of $\vec u$.
Suppose that $\xi$ and $\zeta$ are parametrized so that $\bus_-\circ \xi(t)=\bus_-\circ \zeta(t)=t$ for any $t$;
further set $x_0=\xi(0)$, $z_0=\zeta(0)$, $x_1=\xi(s)$, $z_1=\zeta(t)$ for some $s,t$. 
Then
\[|x_1-z_1|_\Sigma^2=|x_0-z_0|_\Sigma^2+(s-t)^2.\]
\end{clm}

Given $x\in \Sigma$, 
let $t\to\delta^x_a(t)$ be the parametrization of $[x,\lambda(a)]$ by arc-length starting from $x$.
Since $\measuredangle(\vec u,\vec w^a)\to 0$ as $a\to\infty$, we get that
\[\delta^x_a(t)\to\zeta(b+t)\quad\text{as}\quad a\to\infty\]
for any fixed $t\ge0$.

Analogously, since $\measuredangle(\vec u,\vec w^a)\to \pi$ as $a\to-\infty$, and therefore 
\[\delta^x_a(t)\to\zeta(b-t)\quad\text{as}\quad a\to-\infty\]
for any fixed $t\ge0$.

Assume that $s,t\ge 0$, then
\begin{align*}
x_1&=\lim_{a\to\infty}\delta^{x_0}_a(s),
&
z_1&=\lim_{a\to\infty}\delta^{z_0}_a(t).
\end{align*}

Recall that the triangle $[x_0\,\gamma(a)\,z_0]$ is fat (\ref{fat}).
Therefore we get a lower bound for the distance $|\delta^{x_0}_a(s)-\delta^{z_0}_a(t)|_\Sigma$.
Note that if $a$ is large, then the two long sides of the triangle are close to $a$.
Straightforward computations show that passing to the limit as $a\to \infty$, we get
\[|x_1-z_1|_\Sigma^2\ge|{x_0}-{z_0}|_\Sigma^2+(s-t)^2.\]

\begin{figure}
\centering
\includegraphics{mppics/pic-1775}
\end{figure}


Now let us swap ${x_0}$ with $x_1$ and ${z_0}$ with $z_1$,
and repeat the argument above for $a\to-\infty$.
Note that the two longer sides of the triangle are close to  $s-a$ and $t-a$.
Therefore we get the opposite inequality
\[|x_1-z_1|_\Sigma^2\le|{x_0}-{z_0}|_\Sigma^2+(s-t)^2.\]

Whence 
\[|x_1-z_1|_\Sigma^2=|{x_0}-{z_0}|_\Sigma^2+(s-t)^2.\eqlbl{ex:|x-y|=|x-y|}\]
if $s,t\ge0$.
The same argument proves \ref{ex:|x-y|=|x-y|} if $s,t\le 0$.
Applying it couple of times for $s_1=t_1\ge 0$ and $s_2,t_2\le 0$, we get that \ref{ex:|x-y|=|x-y|} holds 
for $s=s_1+s_2$ and $t=t_1+t_2$.
Whence \ref{clm:parallel} follows for any pair $s$ and~$t$.

\parit{Final step.}
Note that since $\bus_-$ is affine, the level set 
\[L=\set{x\in\Sigma}{\bus_-(x)=0}\]
is closed \index{totally convex}\emph{totally convex} and \index{totally geodesic}\emph{geodesic};
that is, if a geodesic $\alpha$ has two common points with $L$, then $\alpha$ lies in $L$.
It follows that $L$ is either closed or both-sides-infinite geodesic.

Choose an arc-length parametrization $v\mapsto\gamma(v)$ of $L$ by a circle (or line).
Denote by $\lambda^v$ the line thru $\gamma(v)$ in the direction of $\vec u$ with the parametrization as above that is $\bus_-\circ\lambda^v(u)=u$ for any $u$ and $v$.
According to \ref{clm:parallel}, $(u,v)\mapsto \lambda^v(u)$ is an intrinsic isometry from a circular cylinder (or, respectively, the Euclidean plane) to $\Sigma$.
\qeds

\begin{thm}{Exercise}\label{ex:line+half-line}
Let $\Sigma$ be an open smooth surface with nonnegative Gauss curvature.
Suppose that $\Sigma$ has a line and a half-line that meet at exactly one point.
Show that $\Sigma$ admits an intrinsic isometry to the Euclidean plane.
\end{thm}

