\chapter{Alternative proofs}

\section*{DNA inequality}

Recall that curvature of a spherical curve is at least $1$
(Exercise~\ref{ex:curvature-of-spherical-curve}).
In particular the length of spherical curve can not exceed its total curvature.
The following theorem shows that the same inequality holds for \emph{closed} curves in a unit ball.

\begin{thm}{Theorem}\label{thm:DNA}
Let $\gamma$ be a nontrivial closed curve that lies in a unit ball.
Then 
\[\tc\gamma\ge \length\gamma.\]

\end{thm}

In the proof we use \ref{def:total-curv-poly} to define for the total curvature;
according to \ref{thm:total-curvature=}, it is more general than the smooth definition on page \pageref{page:total curvature of:smooth-def}.

\parit{Proof.}
We will show that 
\[\tc\gamma> \length\gamma.\]
for any closed polygonal line $\gamma=p_1\dots p_{n}$ in a unit ball.
It implies the theorem since in any nontrivial closed curve we can inscribe a closed polygonal line with arbitrary close total curvature and length.

The indexes are taken modulo $n$, in particular $p_{n}=p_0$, $p_{n+1}=p_1$ and so on.
Denote by $\theta_i$ the external angle of $\gamma$ at $p_i$;
that is,
\[\theta_i=\pi-\measuredangle p_{i-1}p_ip_{i+1}.\]

Denote by $o$ the center of the ball.
Consider a sequence of $n+1$ plane triangles
\begin{align*}
\triangle q_0s_0q_1
&\cong 
\triangle p_0op_1,
\\
\triangle q_1s_1q_2
&\cong 
\triangle p_1op_2,
\\
&\dots
\\
\triangle q_{n}s_nq_{n+1}
&\cong 
\triangle p_nop_{n+1},
\end{align*}
such that the points $q_0,q_1\dots$ lie on one line in that order and all the points $s_0,\dots,s_n$ lie on one side from this line.

\begin{figure}[h!]
\vskip-0mm
\centering
\includegraphics{mppics/pic-16}
\vskip0mm
\end{figure}

Since $p_0=p_n$ and $p_1=p_{n+1}$, we have that
\[\triangle q_{n}s_nq_{n+1}\cong 
\triangle p_nop_{n+1}=\triangle p_0op_1\cong\triangle q_{0}s_0q_1,\]
so $s_0q_0q_ns_n$ is a parallelogram.
Therefore
\begin{align*}
|s_0-s_1|+\dots+|s_{n-1}-s_n|
&\ge|s_n-s_0|=
\\
&=|q_0-q_n|=
\\
&=|p_0-p_1|+\dots+|p_{n-1}-p_n|
\\
&=\length \gamma.
\end{align*}

Note that 
\begin{align*}
\theta_i&=\pi-\measuredangle p_{i-1}p_ip_{i+1}\ge
\\
&\ge\pi-\measuredangle p_{i-1}p_io-\measuredangle op_ip_{i+1}=
\\
&=\pi-\measuredangle q_{i-1}q_is_{i-1}-\measuredangle s_iq_iq_{i+1}=
\\
&=\measuredangle s_{i-1}q_is_i>
\\
&>|s_{i-1}-s_i|;
\end{align*}
the last inequality follows since $|q_i-s_{i-1}|=|q_i-s_i|=|p_i-o|\le 1$.
That is, 
\[\theta_i>|s_{i-1}-s_i|\]
for each $i$.

It follows that
\begin{align*}
\tc \gamma
&=\theta_1+\dots+\theta_n>
\\
&> |s_{0}-s_1|+\dots |s_{n-1}-s_n|\ge 
\\
&\ge\length \gamma.
\end{align*}
Hence the result.
\qeds

This theorem was proved by Don Chakerian \cite{chakerian};
for plane curves it was prved earlier by Istv\'{a}n F\'{a}ry \cite{fary-DNA}.
Few proofs of this theorem are discussed by Serge Tabachnikov~\cite{tabachnikov}.
He also conjectured the following closely related statement:

\begin{thm}{Theorem}
Suppose a closed regular smooth curve $\gamma$ lies in a convex figure with the perimeter $2\cdot \pi$.
Then 
\[\tc\gamma\ge \length\gamma.\]

\end{thm}

It was proved by Jeffrey Lagarias and Thomas Richardson \cite{lagarias-richardso}; latter a simpler proof was given by Alexander Nazarov and Fedor Petrov~\cite{nazarov-petrov}.
The proof is annoyingly difficult; we do not present it here.
