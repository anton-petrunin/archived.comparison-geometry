\chapter{Curvature of space curves}


The term \emph{curvature} is used for something that measures how much 
a geometric object deviates from being \emph{straight};
in particular the curvature of a curve suppose to measure how fast it deviates from a straight line.

\section*{Acceleration of unit-speed curve}

Recall that any regular smooth curve can be parametrized by its arc length.
The obtained curve $\gamma$ remains to be smooth and it has unit speed; 
that is, $|\gamma'(s)|=1$ for all $s$.

The following proposition states that the acceleration vector is perpendicular to the velocity vector if the speed remains constant.

\begin{thm}{Proposition}\label{prop:a'-pertp-a''}
Assume $\gamma$ is a smooth unit-speed space curve.
Then $\gamma'(s)\perp \gamma''(s)$ for any $s$.
\end{thm}

The scalar product (also known as dot product) of two vectors $v$ and $w$ will be denoted by $\langle v,w\rangle$.
Recall that the derivative of a scalar product satisfies the product rule;
that is if $v=v(t)$ and $w=w(t)$ are smooth vector-valued functions of a real parameter $t$, then
\[\langle v,w\rangle'=\langle v',w\rangle+\langle v,w'\rangle.\]

\parit{Proof.}
Since $|\gamma'(s)|=1$, we have
\[\langle\gamma',\gamma'\rangle=1.\]
Differentiating both sides we get
\[2\cdot\langle\gamma'',\gamma'\rangle=\langle\gamma',\gamma'\rangle'=0,\]
hence the result.
\qeds

\section*{Curvature}

For a unit speed space curve $\gamma$ the magnitude of its acceleration $|\gamma''(s)|$ is called its \emph{curvature} at the time $s$.
If $\gamma$ is simple, then we can say that $|\gamma''(s)|$ is the curvature at the point $p=\gamma(s)$ without ambiguity.
The curvature is usually denoted by $k(s)$ or $k_\gamma(s)$ and in the latter case it might be also denoted by $k(p)$ or $k_\gamma(p)$.

Informally the curvature measures how fast the curve turns;
if you drive along a plane curve, curvature tells how much to turn the steering wheel at the given point (note that it does not depend on your speed).

\begin{thm}{Exercise}
Show that any regular smooth spherical curve has curvature at least 1 at each time.
\end{thm}

\parit{Hint:} Differentiate the identity $\langle\gamma(s),\gamma(s)\rangle=1$ a couple of times.



\section*{Tangent indicatrix}

It was convenient to use arc length parametrization to \emph{define} curvature, 
but for calculating the curvature it more convenient to use the original description of the given curve
and its tangent indicatrix described below.

Let $\gamma$ be a regular smooth space curve.
Let us consider another curve 
\[\tau(t)=\tfrac{\gamma'(t)}{|\gamma'(t)|}\eqlbl{eq:tantrix}\] 
that is called \emph{tangent indicatrix} of $\gamma$.
Note that $|\tau(t)|=1$ for any $t$;
that is, $\tau$ is a spherical curve that points in the direction tangent to $\gamma$ at each time.

If $\gamma$ has a unit speed parametrization, then $\tau(t)=\gamma'(t)$.
In this case we have the following expression for curvature: 
$k(t)=|\tau'(t)|=|\gamma''(t)|$.

In general case we have 
\[ k(t)=\tfrac{|\tau'(t)|}{|\gamma'(t)|}.\eqlbl{eq:curvature}\]
Indeed, for an arc length parametrization $s(t)$ we have $s'(t)=|\gamma'(t)|$.
Therefore
\begin{align*}
k(t)&=|\tfrac{d\tau}{ ds}|=
\\
&=|\tfrac{d\tau}{ dt}|/|\tfrac{ds}{ dt}|=
\\
&=\tfrac{|\tau'(t)|}{|\gamma'(t)|}.
\end{align*}



\begin{thm}{Exercise}\label{ex:curvature-formulas}
Use the formulas \ref{eq:tantrix} and \ref{eq:curvature} to show that 
for any smooth regular space curve $\gamma$ we have the following expressions for its curvature:

\begin{enumerate}[(a)]
\item\label{ex:curvature-formulas:a} \[k(t)=\frac{|\gamma''(t)^\perp|}{|\gamma'(t)|^2},\]
where $\gamma''(t)^\perp$ denotes the projection of $\gamma''(t)$ to the perpendicular plane to $\gamma'(t)$;
\item \[k(t)=\frac{|\gamma''(t)\times \gamma'(t)|}{|\gamma'(t)|^{3}},\]
where $\times$ denotes the \emph{vector product} (also known as \emph{cross product}).
\end{enumerate}
\end{thm}

\parit{Hint:} Prove and use the following identities: 
\begin{align*}
\gamma''(t)-\gamma''(t)^\perp&=\tfrac{\gamma'(t)}{|\gamma'(t)|}\cdot\langle\gamma''(t),\tfrac{\gamma'(t)}{|\gamma'(t)|}\rangle,
\\
|\gamma'(t)|&=\sqrt{\langle \gamma'(t),\gamma'(t)\rangle}.\
\end{align*}


\begin{thm}{Exercise}\label{ex:curvature-graph}
Apply the formulas in the previous exercise to show that if $f$ is a smooth real function,
then its graph $y=f(x)$  has curvature
\[k(p)=\frac{|f''(x)|}{(1+f'(x)^2)^{\frac32}}\]
at the point $p=(x,f(x))$.
\end{thm}

\section*{Total curvature}

Let $\gamma\:[a,b]\to\RR^3$ be a regular smooth curve and $\tau$ its \emph{tangent indicatrix}.
Recall that without loss of generality we can assume that $\gamma$ has a unit speed parametrization;
in this case $\tau(s)=\gamma'(s)$ and hence
\[k(s)\df|\gamma''(s)|=|\tau'(s)|,\] 
that is, the curvature of $\gamma$ at time $s$ is the speed of the \emph{tangent indicatrix} $\tau$ at the same time moment.

The integral 
\[\tc\gamma:=\int_a^bk(s)\cdot ds\]
is called \emph{total curvature of} $\gamma$.

\begin{thm}{Exercise}\label{ex:helix-curvature}
Find the curvature of helix $\gamma_a(t)=(a\cdot \cos t,a\cdot \sin t,t)$, its tangent indicatrix and the total curvature of its arc $t\in[0,2\cdot\pi]$.
\end{thm}

\begin{thm}{Observation}\label{obs:tantrix}
The total curvature of a smooth regular curve is the length of its tangent indicatrix.
\end{thm}

\parit{Proof.}
It is sufficient to prove the observation for a unit-speed space curve $\gamma\:[a,b]\to\RR^3$.
Denote by $\tau$ its tangent indicatrix.
Then
\begin{align*}
\tc\gamma&\df\int_a^bk(s)\cdot ds=
\\
&=\int_a^b|\tau'(s)|\cdot ds=
\\
&=\length\tau.
\end{align*}
\qedsf %???other cases --- colsed curves, semiopen intervals...

\begin{thm}{Fenchel's theorem}\label{thm:fenchel}
The total curvature of any closed regular space curve is at least $2\cdot\pi$.
\end{thm}

\parit{Proof.}
Fix a closed regular space curve $\gamma$;
we can assume that it is described by a loop $\gamma\:[a,b]\to \RR^3$;
in this case $\gamma(a)=\gamma(b)$ and $\gamma'(a)=\gamma'(b)$.

Consider its tangent indicatrix $\tau=\gamma'$.
Recall that $|\tau(s)|=1$ for any $s$; that is, $\tau$ is a closed spherical curve.
Let us show that $\tau$ can not lie in a hemisphere.
Assume contrary, without loss of generality we can assume that $\tau$ lies in the north hemisphere defined by the inequality $z>0$ in $(x,y,z)$-coordinates.
It means that $z'(t)>0$ at any time, where $\gamma(t)=(x(t), y(t), z(t))$.
Therefore 
\[z(b)-z(a)=\int_a^bz'(s)\cdot ds>0.\]
In particular, $\gamma(a)\ne \gamma(b)$, a contradiction.

Applying the observation (\ref{obs:tantrix}) and the hemisphere lemma (\ref{lem:hemisphere}), we get that 
\[\tc\gamma=\length \tau\ge2\cdot\pi.\]
\qedsf

\section*{Piecewise smooth curves}

Assume $\alpha\:[a,b]\to \RR^3$ and $\beta\:[b,c]\z\to \RR^3$ are two curves such that $\alpha(b)=\beta(b)$.
Then one can combine these two curves into one $\gamma\:[a,c]\z\to \RR^3$ by the rule $\gamma(t)=\alpha(t)$ for $t\le b$ and $\gamma(t) = \beta(t)$ for $t\ge b$.
The obtained curve $\gamma$ is called the 
\emph{concatenation} of $\alpha$ and $\beta$ and is denoted as $\gamma=\alpha{*}\beta$.

\begin{wrapfigure}{r}{30 mm}
\vskip-0mm
\centering
\includegraphics{mppics/pic-54}
\end{wrapfigure}

The same definition of cancatination $\alpha{*}\beta$ can be applied if $\alpha$ and/or $\beta$ are defied on semiopen intervals 
$(a,b]$ and/or $[b,c)$.

The concatenation can be also defined if the end point of the first curve coincides with the starting point of the second curve; if this is the case, their time intervals can be shifted so that they fit together. 

If in addition $\beta(c)=\alpha(a)$ then we can do cyclic concatination of these curves and obtain a closed curve.

If $\alpha'(b)$ and $\beta'(b)$ are defined then the angle $\theta=\measuredangle(\alpha'(b),\beta'(b))$ is called \emph{external angle} of $\gamma$ at time $b$.
Note that smooth regular curve has zero external angle at each time.

A space curve $\gamma$ is called \emph{piecewise smooth and regular} if it can be presented as a concatination of finite number of smooth regular curves; if $\gamma$ is closed, then the  concatination is assumed to be cyclic.

If $\gamma$ is a concatination of smooth regular arcs $\gamma_1,\dots,\gamma_n$, then the total curvature of $\gamma$ is defined as a sum of total curvatures of $\gamma_i$ and the external angles;
that is, 
\[\tc\gamma=\tc{\gamma_1}+\dots+\tc{\gamma_n}+\theta_1+\dots+\theta_{n-1}\]
where $\theta_i$ is the external angle at the joint $\gamma_i$ and $\gamma_{i+1}$;
if $\gamma$ is closed, then 
\[\tc\gamma=\tc{\gamma_1}+\dots+\tc{\gamma_n}+\theta_1+\dots+\theta_{n},\]
where $\theta_n$ is the external angle at the joint $\gamma_n$ and $\gamma_1$.

\begin{thm}{Generalized Fenchel's theorem}\label{thm:gen-fenchel}
Let $\gamma$ be a piecewise smooth regular space curve.
Then 
\[\tc\gamma\ge2\cdot\pi.\]

\end{thm}

\parit{Proof.}
Suppose $\gamma$ is a cyclic concatenation of $n$ smooth regular arcs $\gamma_1,\dots,\gamma_n$.
Denote by $\theta_1,\dots,\theta_n$ its external angles.
We need to show that
\[\tc{\gamma_1}+\dots+\tc{\gamma_n}+\theta_1+\dots+\theta_n\ge2\cdot\pi.\eqlbl{eq:gen-fenchel}\]

Consider the tangent indicatrix $\tau_1,\dots,\tau_n$ for each arc $\gamma_1,\dots,\gamma_n$;
these are smooth spherical arcs.

The same argument as in the proof of Fenchel's theorem, shows that the curves $\tau_1,\dots,\tau_n$ can not lie in an open hemisphere.

Note that the spherical distance from the end point of $\tau_i$ to the starting point of $\tau_{i+1}$ is equal to the external angle $\theta_i$ (we enumerate modulo $n$, so $\gamma_{n+1}=\gamma_1$).
Let us connect the end point of $\tau_i$ to the starting point of $\tau_{i+1}$ by a short arc of equator;
this way we add $\theta_1+\dots+\theta_n$ to the total length of $\tau_1,\dots,\tau_n$.

By the hemisphee lemma (\ref{lem:hemisphere}), we get 
\[\length\tau_1+\dots+\length\tau_n+\theta_1+\dots+\theta_n\ge 2\cdot\pi.\]
Applying the observation (\ref{obs:tantrix}), we get \ref{eq:gen-fenchel}.
\qedsf


\section*{Polygonal lines}


\begin{thm}{Chord lemma}\label{lem:chord}
Let $\ell$ be the chord to a smooth regular arc $\gamma\:[a,b]\to\RR^3$.
Assume $\gamma$ meets $\ell$ at angles $\alpha$ and $\beta$ at its ends;
that is 
\[\alpha=\measuredangle(w,\gamma'(a))\quad\text{and}\quad \beta=\measuredangle(w,\gamma'(b)),\]
where $w=\gamma(b)-\gamma(a)$.
Then 
\[\tc\gamma\ge \alpha+\beta.\] 

\end{thm}

\begin{wrapfigure}{r}{45 mm}
\vskip-7mm
\centering
\includegraphics{mppics/pic-53}
\vskip0mm
\end{wrapfigure}

\parit{Proof.}
Parameterize $\ell$ from $\gamma(b)$ to $\gamma(a)$ and consider the cyclic concatenation $\bar\gamma$ of $\gamma$ and $\ell$.
The closed curve $\bar\gamma$ has two external angles $\pi-\alpha$ and $\pi-\beta$;
therefore 
\[\tc{\bar\gamma}=\tc\gamma+(\pi-\alpha)+(\pi-\beta).\]
According to the generalized Fenechel's theorem
\[\tc{\bar\gamma}\ge 2\cdot\pi,\]
hence the result.
\qeds

The following exercise states that the estimate in the chord lemma is optimal.

\begin{thm}{Exercise}
Given two points $p, q$ and two vectors $v,w$ in $\RR^3$.
Show that there is a smooth regular curve $\gamma$ that starts at $p$ in the direction of $v$ and ends at $q$ in the direction of $w$ such that 
$\tc\gamma$ is arbitrary close to $\measuredangle(u,v)+\measuredangle(u,w)$, where $u=q-p$.

\end{thm}

Polygonal lines are partial case of piecewise smooth regular curves;
each arc in its concatenation is a line segment.
Since the curvature of a line segment vanish, the total curvature of polygonal line is the sum of its external angles.

\begin{thm}{Exercise}\label{ex:monotonic-tc}
Let $a,b,c,d$ and $x$ be distinct points in $\RR^3$.
Show that the total curvature of polygonal line $abcd$ can not exceed the the total curvature of $abxcd$; that is, 
\[\tc {abcd} \leq \tc {abxcd}.\]

Use this statement to show that any closed polygonal line has curvature at least $2\cdot\pi$.
\end{thm}

\parit{Hint:} For the second part use induction on number of vertexes.


\begin{thm}{Corollary}\label{cor:inscribed-total-curvature}
Assume a polygonal line $\hat\gamma=p_1\dots p_n$ is inscribed in a smooth regular curve $\gamma$.
Then 
\[\tc\gamma\ge \tc{\hat\gamma}.\]
Moreover if $\gamma$ is closed we can assume that the inscribed polygonal line $\hat\gamma$ is also closed.

\end{thm}

\parit{Proof.}
Since curvature of line segment vanish, 
the total curvature of polygonal line is the sum of external angles $\theta_i=\pi-\measuredangle p_{i-1}p_ip_{i+1}$.
In case of closed curve we use indexes modulo $n$, in particular $p_{n+1}=p_1$.

Assume $p_i=\gamma(t_i)$.
Set 
\begin{align*}
w_i&=p_{i+1}-p_i,& v_i&=\gamma'(t_i),
\\
\alpha_i&=\measuredangle (w_i,v_i),&\beta_i&=\measuredangle (w_{i-1},v_i).
\end{align*}


Note that $\theta_i=\measuredangle (w_{i-1},w_i)$.
Therefore 
\[\theta_i\le \alpha_i+\beta_i.\]
By the chord lemma, the total curvature of the arc of $\gamma$ from $p_i$ to $p_{i+1}$ is at least $\alpha_i+\beta_{i+1}$. 
Therefore if $\gamma$ is a closed curve, we have
\begin{align*}
\tc{\hat\gamma}&=\theta_1+\dots+\theta_n\le 
\\
&\le\beta_1+\alpha_1+\dots+\beta_n+\alpha_n = 
\\
&=(\alpha_1+\beta_2)+\dots+(\alpha_n+\beta_1) \le 
\\
&\le \tc\gamma.
\end{align*}
If $\gamma$ is an arc, the argument is analogous:
\begin{align*}
\tc{\hat\gamma}&=\theta_2+\dots+\theta_{n-1}\le 
\\
&\le\beta_2+\alpha_2+\dots+\beta_{n-1}+\alpha_{n-1} \le
\\
&\le (\alpha_1+\beta_2)+\dots+(\alpha_{n-1}+\beta_n) \le 
\\
&\le \tc\gamma.
\end{align*}
\qedsf

\begin{thm}{Exercise}
\begin{enumerate}[(a)]
\item Construct a smooth regular curve $\gamma\:[a,b]\to\RR^3$ which has a self-intersection, but $\tc\gamma<2\cdot\pi$.
\item Show that if a a smooth regular curve $\gamma\:[a,b]\to\RR^3$ has a self-intersection, then $\tc\gamma>\pi$.
\end{enumerate}
\end{thm}

\begin{thm}{Exercise}\label{ex:fenchel=}
\begin{enumerate}[(a)]
 \item Show that the sum of external angles of a space quadraliteral is $2\cdot\pi$ if and only if it is a convex plane qudraliteral; that is if for quadraliteral $abcd$ we have that 
 \[(\pi-\measuredangle dab)+(\pi-\measuredangle abc)+(\pi-\measuredangle bcd)+(\pi-\measuredangle cda)=2\cdot\pi,\]
 then $a,b,c$ and $d$ form a convex quadraliteral in a plane. 

 \item\label{ex:fenchel==} Use the previous part and \ref{cor:inscribed-total-curvature} to show that if the total curvature of a smooth regular space curve is equal to $2\cdot\pi$, then it is a convex plane curve.
\end{enumerate}
\end{thm}

Note that part (\ref{ex:fenchel==}) of the above inequality describes the equality case in the Fenchel's theorem.

\begin{wrapfigure}{r}{30 mm}
\vskip-0mm
\centering
\includegraphics{mppics/pic-20}
\vskip0mm
\end{wrapfigure}

\begin{thm}{Exercise}
Suppose that a closed curve $\gamma$ crosses a line at four points $a$, $b$, $c$ and $d$.
Assume that the points $a$, $b$, $c$ and $d$ appear on the line in that order and they appear on the curve $\gamma$ in the order $a$, $c$, $b$, $d$.
Show that 
\[\tc\gamma\ge 4\cdot\pi.\]

\end{thm}

A line crossing a curve at four points as in the exercise is called \emph{alternating quadrisecants}.
It turns out that any \emph{nontrivial knot} admits an alternating quadrisecants \cite{denne};
it implies the so called F\'ary--Milnor theorem --- the total curvature any knot exceeds $4\cdot \pi$.

\section*{Nonsmooth curves*}

In this section we define total curvature of locally simple curves and discuss its properties.

\begin{thm}{Definition}\label{def:total-curv-poly}
The total curvature of a locally simple curve $\gamma$ is the exact upper bound on the total curvatures of inscribed nondegenerate polygonal lines;
if $\gamma$ is closed then we assume that the inscribed polygonal lines are closed as well.
\end{thm}

\begin{thm}{Theorem}\label{thm:total-curvature=}
For smooth regular curves the two definitions of total curvature agree;
that is, for any regular curve, the length of its tangent indicatrix is equal to the exact upper bound of the total curvatures of inscribed nondegenerate polygonal lines.
\end{thm}

\parit{Proof of \ref{thm:total-curvature=}.}
Let $\gamma\:[a,b]\to\RR^3$ be a smooth curve.
Fix a partition $a\z=t_0<\dots<t_n=b$ and consider the corresponding inscribed polygonal line $\beta=w_0\dots w_n$.
Let $\chi=\xi_1\dots\xi_n$ be a spherical polygonal line
with the vertexes $\xi_i\z=\tfrac{w_i-w_{i-1}}{|w_i-w_{i-1}|}$;
We can assume that it has constant speed on each arc and $\chi(t_i)=\xi_i$ for each $i$. 
The spherical polygonal line $\chi$ will be called tangent indicatrix for $\beta$.

Consider a sequence of finer and finer partitions, denote by $\beta_n$ and $\chi_n$ the corresponding inscribed polygonal line and its tangent indicatrix;
since $\gamma$ is smooth, the $\chi_n$ converge pointwise to $\tau$ --- the thangent indicatrix of $\gamma$.
By semi-continuity of the length functional, we get  
\begin{align*}
\tc\gamma&=\length \tau\le  
\\
&\le \liminf_{n\to\infty}\length \chi_n=
\\
&= \liminf_{n\to\infty}\tc {\beta_n}\le
\\
&\le \sup\{\tc\beta\},
\end{align*}
where the last supremum is taken over all partitions and their corresponding inscribed polygonal lines $\beta$.

By \ref{cor:inscribed-total-curvature}, we get that
\[\tc\gamma\ge \tc\beta\]
for any polygonal line $\beta$ inscribed in $\gamma$.
Hence 
\[\tc\gamma=\sup\{\tc\beta\},\]
where the least upper bound is taken for all inscribed polygonal lines~$\beta$.
\qeds


\begin{thm}{Exercise}
Show that the total curvature is lower semi-continuous with respect to pointwise convergence of curves.
That is, if a sequence
of curves $\gamma_n\:[a,b]\to \RR^3$ converges pointwise 
to a curve $\gamma_\infty\:[a,b]\z\to \RR^3$, then 
\[\liminf_{n\to\infty} \tc{\gamma_n} \ge \tc{\gamma_\infty}.\]
\end{thm}

\parit{Hint:} Modify the proof of semi-continuity of length (Theorem~\ref{thm:length-semicont}).

\begin{thm}{Exercise}
Show that Fenchel's theorem holds for any locally simple closed curve $\gamma$;
that is, 
\[\tc\gamma\ge2\cdot\pi.\]
\end{thm}

\begin{thm}{Exercise} 
Assume that a curve $\gamma\:[a,b]\to\RR^3$ has finite total curvature.
 Show that $\gamma$ is rectifiable.
\end{thm}
