\chapter{Curvature of curves}


In general the term \emph{curvature} is used for something that measures how much 
a geometric object deviates from being \emph{straight};
in particular the curvature of a curve suppose to measure how fast it deviates from a straight line.

\section*{Acceleration of unit-speed curve}

Recall that any regular smooth curve can be parametrized by its length.
The obtained curve $\gamma$ remains to be smooth and it has unit speed; 
that is, $|\gamma'(s)|=1$ for all $s$.

The following proposition essentailly states that the acceleration vector is perpendicular to the velocity vector if the speed remains constant.

\begin{thm}{Proposition}\label{prop:a'-pertp-a''}
Assume $\gamma$ be a smooth unit-speed space curve.
Then $\gamma'(s)\perp \gamma''(s)$ for any $s$.
\end{thm}

The scalar product (also known as dot product) of two vectors $v$ and $w$ will be denoted by $\langle v,w\rangle$.
Recall that the derivative of a scalar product satisfies the product rule;
that is if $v=v(t)$ and $w=w(t)$ are smooth vector-valued functions of a real parameter $t$, then
\[\langle v,w\rangle'=\langle v',w\rangle+\langle v,w'\rangle.\]

\parit{Proof.}
Since $|\gamma'(s)|=1$, we have
\[\langle\gamma'(s),\gamma'(s)\rangle=1.\]
Differentiating both sides we get
\[2\cdot\langle\gamma''(s),\gamma'(s)\rangle=0,\]
hence the result.
\qeds

\section*{Curvature}

For a unit speed space curve $\gamma$ the magnitude of its acceleration $|\gamma''(s)|$ is called its \emph{curvature} at $s$.
If $\gamma$ is simple, then we can say that $|\gamma''(s)|$ is the curvature at the point $p=\gamma(s)$ without ambiguity.
The curvature is usually denoted by $k(s)$ and in the latter case it might be also denoted by $k(p)$.

Informally the curvature measures how fast the curve turns;
if you drive along a plane curve, curvature tells how much to turn the steering wheel at the given point (note that it does not depend on your speed).

\begin{thm}{Exercise}
Show that any regular smooth spherical curve has curvature at least 1 at each point.
\end{thm}

\parit{Hint:} Differentiate the identity $\langle\gamma(s),\gamma(s)\rangle=1$ a couple of times.



\section*{Tangent indicatrix}

It was convenient to use arc length parametrization to define curvature, 
but for finding curvature it more convenient to use the original description of curvature via tangent indicatrix described below.

Let $\gamma$ be a regular smooth space curve.
Let us consider another curve 
\[\tau(t)=\tfrac{\gamma'(t)}{|\gamma'(t)|}\eqlbl{eq:tantrix}\] 
that is called \emph{tangent indicatrix} of $\gamma$.
Note that $|\tau(t)|=1$ for any $t$;
that is, $\tau$ is a spherical curve that points in the direction tangent to $\gamma$ at each time.

If $\gamma$ has a unit speed parametrization, then $\tau(t)=\gamma'(t)$.
In this case we have the following expression for curvature: 
$k(t)=|\tau'(t)|=|\gamma''(t)|$.

In general case we have 
\[ k(t)=\tfrac{|\tau'(t)|}{|\gamma'(t)|}.\eqlbl{eq:curvature}\]
Indeed, for an arc length parametrization $s(t)$ we have $s'(t)=|\gamma'(t)|$.
Therefore
\begin{align*}
k(t)&=|\tfrac{d\tau}{ ds}|=
\\
&=|\tfrac{d\tau}{ dt}|/|\tfrac{ds}{ dt}|=
\\
&=\tfrac{|\tau'(t)|}{|\gamma'(t)|}.
\end{align*}



\begin{thm}{Exercise}\label{ex:curvature-formulas}
Use the formulas \ref{eq:tantrix} and \ref{eq:curvature} to show that 
for any smooth regular space curve $\gamma$ we have the following expressions for its curvature:

\begin{enumerate}[(a)]
\item\label{ex:curvature-formulas:a} \[k(t)=\frac{|\gamma''(t)^\perp|}{|\gamma'(t)|^2},\]
where $\gamma''(t)^\perp$ denotes the projection of $\gamma''(t)$ to the perpendicular plane to $\gamma'(t)$;
\item \[k(t)=\frac{|\gamma''(t)\times \gamma'(t)|}{|\gamma'(t)|^{3}}.\]
\end{enumerate}
\end{thm}

\parit{Hint:} Prove and use the following identities: 
\begin{align*}
\gamma''(t)-\gamma''(t)^\perp&=\tfrac{\gamma'(t)}{|\gamma'(t)|}\cdot\langle\gamma''(t),\tfrac{\gamma'(t)}{|\gamma'(t)|}\rangle,
\\
|\gamma'(t)|&=\sqrt{\langle \gamma'(t),\gamma'(t)\rangle}.\
\end{align*}


\begin{thm}{Exercise}\label{ex:curvature-graph}
Apply the formulas in the previous exercise to show that if $f$ is a smooth real function,
then its graph $\Gamma_f=\{\,(x,f(x))\z\in\RR^2\,\}$  has curvature
\[k(p)=\frac{|f''(x)|}{(1+f'(x)^2)^{\frac32}}\]
at the point $p=(x,f(x))$.
\end{thm}

\section*{Total curvature}

Let $\gamma\:[a,b]\to\RR^3$ be a regular smooth curve and $\tau$ its \emph{tangent indicatrix}.
Recall that without loss of generality we can assume that $\gamma$ has a unit speed parametrization;
in this case $\tau(t)=\gamma'(t)$ and hence
\[k(t)\df|\gamma''(t)|=|\tau'(t)|,\] 
that is, the curvature of $\gamma$ at $t$ is the speed of the \emph{tangent indicatrix} $\tau$ at $t$.

The integral 
\[\tc\gamma:=\int_a^bk(t)\cdot dt\]
is called \emph{total curvature of} $\gamma$.

\begin{thm}{Exercise}\label{ex:helix-curvature}
Find the curvature of helix $\gamma_a=(a\cdot \cos t,a\cdot \sin t,t)$, its tangent indicatrix and the total curvature of its arc $t\in[0,2\cdot\pi]$.
\end{thm}

\begin{thm}{Observation}\label{obs:tantrix}
The total curvature of a smooth regular curve is the length of its tangent indicatrix.
\end{thm}

\parit{Proof.}
It is sufficient to prove the observation for a unit-speed space curve $\gamma\:[a,b]\to\RR^3$.
Denote by $\tau$ its tangent indicatrix.
Then
\begin{align*}
\tc\gamma&\df\int_a^bk(t)\cdot dt=
\\
&=\int_a^b|\tau'(t)|\cdot dt=
\\
&=\length\tau.
\end{align*}
\qedsf %???other cases --- colsed curves, semiopen intervals...

\begin{thm}{Fenchel's theorem}\label{thm:fenchel}
The total curvature of of any closed regular space curve is at least $2\cdot\pi$.
\end{thm}

\parit{Proof.}
Fix a closed regular space curve $\gamma$;
we can assume that it has a unit-speed parametrization $\gamma\:[a,b]\to \RR^3$, so $\gamma(a)=\gamma(b)$.

Consider its tangent indicatrix $\tau=\gamma'$.
By fundamental theorem of calculus we have that 
\[\int_a^b\tau(t)\cdot dt=\gamma(b)-\gamma(a)=0.\eqlbl{eq:zero}\]

Recall that $|\tau(t)|=1$ for any $t$; that is, $\tau$ is a spherical curve.
Let us show that $\tau$ can not lie in a hemisphere.
Assume contrary, denote by $u$ the center of the hemisphere;
so $\langle u,\tau(t)\rangle>0$ for any $t$.
From \ref{eq:zero}, we get that
\begin{align*}
0&=\left\langle u,\int_a^b\tau(t)\cdot dt\right\rangle=
\\
&=\int_a^b\langle u,\tau(t)\rangle\cdot dt>
\\
&>0.
\end{align*}
--- a contradiction.

Applying the observation (\ref{obs:tantrix}) and the hemisphere lemma (\ref{lem:hemisphere}), we get that 
\[\tc\gamma=\length \tau\ge2\cdot\pi.\]
\qedsf



