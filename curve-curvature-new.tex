\chapter{Curvature}

\section{Acceleration of a unit-speed curve}

Recall that any regular smooth curve can be reparametrized by its arc-length (\ref{prop:arc-length-smooth}).
The obtained parametrized curve, say $\gamma$, remains to be smooth and it has unit speed; 
that is, $|\gamma'(s)|=1$ for all $s$.
The following proposition states that in this case
the acceleration vector stays perpendicular to the velocity vector.

\begin{thm}{Proposition}\label{prop:a'-pertp-a''}
Assume $\gamma$ is a smooth unit-speed space curve.
Then $\gamma'(s)\perp \gamma''(s)$ for any $s$.
\end{thm}

The scalar product (also known as dot product) of two vectors $\vec v$ and $\vec w$ will be denoted by $\langle \vec v,\vec w\rangle$.
Recall that the derivative of a scalar product satisfies the product rule;
that is, if $\vec v=\vec v(t)$ and $\vec w=\vec w(t)$ are smooth vector-valued functions of a real parameter $t$, then
\[\langle \vec v,\vec w\rangle'=\langle \vec v',\vec w\rangle+\langle \vec v,\vec w'\rangle.\]

\parit{Proof.}
The identity $|\gamma'|=1$ can be rewritten as $\langle\gamma',\gamma'\rangle=1$.
Differentiating both sides, we get
$2\cdot\langle\gamma'',\gamma'\rangle=\langle\gamma',\gamma'\rangle'=0$;
whence $\gamma''\perp\gamma'$.
\qeds

\section{Curvature}\label{sec:curvature}

For a unit-speed smooth space curve $\gamma$ the magnitude $|\gamma''(s)|$ of its acceleration is called its \index{curvature of curve}\emph{curvature} at  time $s$.
If $\gamma$ is simple, then we can say that $|\gamma''(s)|$ is the curvature at the point $p=\gamma(s)$ without ambiguity.
The curvature is usually denoted by $\kur(s)$ or $\kur(s)_\gamma$ and in the case of simple curves it might be also denoted by $\kur(p)$ or $\kur(p)_\gamma$.

The curvature measures how fast the curve turns;
if you drive along a plane curve, then the curvature describes the position of your steering wheel at the given point.

In general, the term {}\emph{curvature} is used for anything that measures how much a {}\emph{geometric object} deviates from being {}\emph{straight};
for curves, it measures how fast it deviates from a straight line.


\begin{thm}{Exercise}\label{ex:zero-curvature-curve}
Show that a regular simple space curve has 0 curvature at each point if and only if it is a segment of a straight line.
\end{thm}
%This contrasts with the fact that a surface with 0 curvature is not necessarily an (extrinsic) plane.

\begin{thm}{Exercise}\label{ex:scaled-curvature}
Let $\gamma$ be a smooth simple space curve and let $\gamma_{\lambda}$ be a scaled copy of $\gamma$ with factor $\lambda >0$;
that is, $\gamma_{\lambda}(t)=\lambda \cdot\gamma(t)$ for any $t$.
Show that 
\[\kur(\lambda \cdot p )_{\gamma_{\lambda}}
= \frac{\kur(p)_{\gamma}}\lambda\]
for any $p \in \gamma$.
\end{thm}

\begin{thm}{Exercise}\label{ex:curvature-of-spherical-curve}
Show that any regular smooth unit-speed spherical curve has curvature at least 1 at each time.
\end{thm}

\section{Tangent indicatrix}\label{sec:Tangent indicatrix}

Let $\gamma$ be a regular smooth space curve.
The curve 
\[\tan(t)=\tfrac{\gamma'(t)}{|\gamma'(t)|};
\eqlbl{eq:tantrix}\] 
it is called a \index{tangent indicatrix}\emph{tangent indicatrix} of $\gamma$.
Note that $|\tan(t)|=1$ for any $t$;
that is, $\tan$ is a spherical curve.


If $s\mapsto \gamma(s)$ is a unit-speed parametrization, then $\tan(s)=\gamma'(s)$.
In this case we have the following expression for curvature: 
\[\kur(s)\z=|\tan'(s)|\z=|\gamma''(s)|.\]\index{10k@$\kur$}

For general parametrization $t\mapsto \gamma(t)$,
we have instead
\[ \kur(t)=\frac{|\tan'(t)|}{|\gamma'(t)|}.\eqlbl{eq:curvature}\]
Indeed, for an arc-length reparametrization by $s(t)$, we have $s'(t)=|\gamma'(t)|$.
Therefore
\begin{align*}
\kur&=\left|\frac{d\tan}{ ds}\right|=
\\
&=|\tfrac{d\tan}{ dt}|/|\tfrac{ds}{ dt}|=
\\
&=\frac{|\tan'|}{|\gamma'|}.
\end{align*}
It follows that the indicatrix of a smooth regular curve $\gamma$ is regular if the curvature of $\gamma$ does not vanish.

\begin{thm}{Exercise}\label{ex:curvature-formulas}
Use the formulas \ref{eq:tantrix} and \ref{eq:curvature} to show that 
for any smooth regular space curve $\gamma$ we have the following expressions for its curvature:

\begin{subthm}{ex:curvature-formulas:a} 
\[\kur=\frac{|\vec w|}{|\gamma'|^2},\]
where $\vec w=\vec w(t)$ denotes the projection of $\gamma''(t)$ to the plane normal to $\gamma'(t)$;
\end{subthm}

\begin{subthm}{ex:curvature-formulas:b}
\[\kur=\frac{|\gamma''\times \gamma'|}{|\gamma'|^{3}},\]
where $\times$ denotes the \index{vector product}\emph{vector product} (also known as \index{cross product}\emph{cross product}).
\end{subthm}

\end{thm}


\begin{thm}{Exercise}\label{ex:curvature-graph}
Apply the formulas in the previous exercise to show that if $f$ is a smooth real function,
then its graph $y=f(x)$  has curvature
\[\kur(p)=\frac{|f''(x)|}{(1+f'(x)^2)^{\frac32}}\]
at the point $p=(x,f(x))$.
\end{thm}

\begin{thm}{Exercise}\label{ex:approximation-const-curvature}
Show that any smooth regular curve $\gamma\:\mathbb{I}\to\RR^3$ with curvature at most 1 can be approximated by smooth curves with constant curvature 1.

In other words, construct a sequence $\gamma_n\:\mathbb{I}\to\RR^3$ of smooth regular curves  with constant curvature 1 such that $\gamma_n(t)\to \gamma(t)$ for any $t$ as $n\to\infty$.
\end{thm}


\section{Tangent curves}

Let $\gamma$ be a smooth regular space curve and $\tan$ its tangent indicatrix.
The line thru $\gamma(t)$ in the direction of $\tan(t)$ is called the \index{tangent line}\emph{tangent line} of $\gamma$  at~$t$.

The tangent line could be also defined as a unique line that has that has \index{order of contact}\emph{first order of contact} with $\gamma$ at $s$;
that is, $\rho(\ell)=o(\ell)$, where $\rho(\ell)$ denotes the distance from $\gamma(s+\ell)$ to the line.

We say that a smooth regular curve $\gamma_1$ at $s_1$ is \index{tangent curves}\emph{tangent} to a smooth regular curve $\gamma_2$ at $s_2$
if $\gamma_1(s_1)=\gamma_2(s_2)$ and the tangent line of $\gamma_1$ at $s_1$ coincides with the tangent line of $\gamma_2$ at $s_2$;
if both curves are simple we can also say that they are tangent at the point $p=\gamma_1(s_1)\z=\gamma_2(s_2)$ without ambiguity.


\section{Total curvature}

Let $\gamma\:\mathbb{I}\to\RR^3$ be a smooth unit-speed curve and $\tan$ its tangent indicatrix.
The integral 
\[\tc\gamma\df\int_{\mathbb{I}}\kur(s)\cdot ds\]
is called the \index{total curvature of}\emph{total curvature of}\label{page:total curvature of:smooth-def}
$\gamma$.

Rewriting the above integral using a change of variables produce a formula for a general parametrization $t\mapsto \gamma(t)$:
\[\tc\gamma\df\int_{\mathbb{I}}\kur(t)\cdot|\gamma'(t)| \cdot dt.
\eqlbl{eq:tocurv}\]


\begin{thm}{Exercise}\label{ex:helix-curvature}
Find the curvature of the helix \[\gamma_{a,b}(t)=(a\cdot \cos t,a\cdot \sin t,b\cdot t),\] its tangent indicatrix and the total curvature of its arc $\gamma_{a,b}|_{[0,2\cdot\pi]}$.
\end{thm}

Note that for a unit-speed smooth curve,
the speed of its tangent indicatrix equals its curvature.
Therefore \ref{ex:integral-length} implies the following:

\begin{thm}{Observation}\label{obs:tantrix}
The total curvature of a smooth regular curve is the length of its tangent indicatrix.
\end{thm}

\begin{thm}{Fenchel's theorem}\label{thm:fenchel}
The total curvature of any closed regular space curve is at least $2\cdot\pi$.
\end{thm}

\parit{Proof.}
Fix a closed regular space curve $\gamma$.
We can assume that $\gamma$ is described by a unit-speed loop $\gamma\:[a,b]\to \RR^3$;
in this case $\gamma(a)=\gamma(b)$ and $\gamma'(a)=\gamma'(b)$.

Consider its tangent indicatrix $\tan=\gamma'$.
Recall that $|\tan(s)|=1$ for any $s$; that is, $\tan$ is a closed spherical curve.

Let us show that $\tan$ cannot lie in a hemisphere.
Assume the contrary; without loss of generality we can assume that it lies in the north hemisphere defined by the inequality $z>0$ in $(x,y,z)$-coordinates.
In other words, if $\gamma(t)=(x(t), y(t), z(t))$, then $z'(t)>0$ for any $t$.
Therefore 
\[z(b)-z(a)=\int_a^b z'(s)\cdot ds>0.\]
In particular, $\gamma(a)\ne \gamma(b)$, a contradiction.

Applying the observation (\ref{obs:tantrix}) and the hemisphere lemma (\ref{lem:hemisphere}), we get  
\[\tc\gamma=\length \tan\ge2\cdot\pi.\]
\qedsf

\begin{thm}{Exercise}\label{ex:length>=2pi}
Show that a closed space curve $\gamma$ with curvature at most~$1$ cannot be shorter than the unit circle;
that is, 
\[\length\gamma\ge 2\cdot \pi.\]

\end{thm}


\begin{thm}{Advanced exercise}\label{ex:gamma/|gamma|}
Suppose that $\gamma$ is a smooth regular space curve that does not pass thru the origin.
Consider the spherical curve defined as $\sigma(t)=\frac{\gamma(t)}{|\gamma(t)|}$ for any $t$.
Show that 
\[\length \sigma< \tc\gamma+\pi.\]
Moreover, if $\gamma$ is closed, then
\[\length \sigma\le \tc\gamma.\]
\end{thm}

Note that the last inequality gives an alternative proof of Fenchel's theorem.
Indeed, without loss of generality we can assume that the origin lies on a chord of $\gamma$.
In this case the closed spherical curve $\sigma$ goes from a point to its antipode and comes back; 
it takes length $\pi$ each way, 
whence 
\[\length\sigma\ge 2\cdot\pi.\]

Recall that the curvature of a spherical curve is at least $1$
(see \ref{ex:curvature-of-spherical-curve}).
In particular, the length of a spherical curve cannot exceed its total curvature.
The following theorem shows that the same inequality holds for {}\emph{closed} curves in a unit ball.

\begin{thm}{Theorem}\label{thm:DNA}
Let $\gamma$ be a smooth regular closed curve that lies in a unit ball.
Then 
\[\tc\gamma\ge \length\gamma.\]

\end{thm}

The 2-dimensional case of this theorem was proved by Istv\'{a}n F\'{a}ry \cite{fary1950}.
It was generalized by Don Chakerian \cite{chakerian1962} to higher dimensions.
This theorem has many very interesting and very different proofs;
a number of them are collected by Serge Tabachnikov~\cite{tabachnikov}.
The following exercise will guide you thru another proof by Don Chakerian \cite{chakerian1964}:

\begin{thm}{Exercise}\label{ex:DNA}
Let $\gamma\:[0,\ell]\to\RR^3$ be a smooth unit-speed closed curve that lies in the unit ball; that is, $|\gamma|\le 1$.

\begin{subthm}{ex:DNA:c''c>=k}
Show that 
\[\langle\gamma''(s),\gamma(s)\rangle\ge-\kur(s)\]
for any $s$.
\end{subthm}

\begin{subthm}{ex:DNA:int>=length-tc}
Use part \ref{SHORT.ex:DNA:c''c>=k} to show that 
\[\int_0^\ell\langle\gamma(s),\gamma'(s)\rangle'\cdot ds\ge
\length\gamma-\tc\gamma.\]

\end{subthm}

\begin{subthm}{ex:DNA:end}
Suppose that $\gamma(0)=\gamma(\ell)$ and $\gamma'(0)=\gamma'(\ell)$.
Show that 
\[\int_0^\ell\langle\gamma(s),\gamma'(s)\rangle'\cdot ds=0.\]
Use this equality together with  part \ref{SHORT.ex:DNA:int>=length-tc} to prove \ref{thm:DNA}.
\end{subthm}
\end{thm}

\section{Piecewise smooth curves}

Assume $\alpha\:[a,b]\to \RR^3$ and $\beta\:[b,c]\z\to \RR^3$ are two curves such that $\alpha(b)\z=\beta(b)$.
Note that these two curves can be combined into one $\gamma\:[a,c]\z\to \RR^3$ by the rule 
\[\gamma(t)=
\begin{cases}
\alpha(t)&\text{if}\quad t\le b,
\\
\beta(t)&\text{if}\quad t\ge b.
\end{cases}
\]
The obtained curve $\gamma$ is called the 
\emph{concatenation} of $\alpha$ and $\beta$. %briefly $\gamma=\alpha{*}\beta$.
(The condition $\alpha(b)=\beta(b)$ ensures that the map $t\mapsto\gamma(t)$ is continuous.)

The same definition of concatenation can be applied if $\alpha$ and/or $\beta$ are defined on semiopen intervals 
$(a,b]$ and/or $[b,c)$.

\begin{wrapfigure}{o}{25 mm}
\vskip-0mm
\centering
\includegraphics{mppics/pic-54}
\end{wrapfigure}

The assumption that the intervals of definition of $\alpha$ and $\beta$ fit together is not essential --- one can concatenate any of the curves as long as the endpoint of $\alpha$ coincides with the starting point of~$\beta$.
If this is the case, then the time intervals of both curves can be shifted so that they fit together. 

If in addition $\beta(c)=\alpha(a)$, then we can do cyclic concatenation of these curves;
this way we obtain a closed curve.

If $\alpha'(b)$ and $\beta'(b)$ are defined, then the angle $\theta\z=\measuredangle(\alpha'(b),\beta'(b))$ is called the \index{external angle}\emph{external angle} of $\gamma$ at time $b$.
If $\theta=\pi$, then we say that $\gamma$ has a \index{cusp}\emph{cusp} at  time $b$.

A space curve $\gamma$ is called \index{piecewise smooth regular curve}\emph{piecewise smooth and regular} if it can be presented as an iterated concatenation of a finite number of smooth regular curves; if $\gamma$ is closed, then the  concatenation is assumed to be cyclic.

If $\gamma$ is a concatenation of smooth regular arcs $\gamma_1,\dots,\gamma_n$, then the total curvature of $\gamma$ is defined as a sum of the total curvatures of $\gamma_i$ and the external angles;
that is, 
\[\tc\gamma=\tc{\gamma_1}+\dots+\tc{\gamma_n}+\theta_1+\dots+\theta_{n-1}\]
where $\theta_i$ is the external angle at the joint between $\gamma_i$ and $\gamma_{i+1}$;
if $\gamma$ is closed, then 
\[\tc\gamma=\tc{\gamma_1}+\dots+\tc{\gamma_n}+\theta_1+\dots+\theta_{n},\]
where $\theta_n$ is the external angle at the joint between $\gamma_n$ and $\gamma_1$.

{

\begin{wrapfigure}{r}{23 mm}
\vskip-3mm
\centering
\includegraphics{mppics/pic-354}
\end{wrapfigure}

In particular, for a smooth regular loop $\gamma\:[a,b] \z\to \mathbb{R}^3$, the total curvature of the corresponding closed curve $\hat\gamma$ is defined as
\[\tc{\hat\gamma}\df\tc\gamma + \theta,\]
where $\theta=\measuredangle(\gamma'(a),\gamma'(b))$.

}

\begin{thm}{Generalized Fenchel's theorem}\label{thm:gen-fenchel}
Let $\gamma$ be a closed piecewise smooth regular space curve.
Then 
\[\tc\gamma\ge2\cdot\pi.\]

\end{thm}

\parit{Proof.}
Suppose $\gamma$ is a cyclic concatenation of $n$ smooth regular arcs $\gamma_1,\dots,\gamma_n$.
Denote by $\theta_1,\dots,\theta_n$ its external angles.
We need to show that
\[\tc{\gamma_1}+\dots+\tc{\gamma_n}+\theta_1+\dots+\theta_n\ge2\cdot\pi.\eqlbl{eq:gen-fenchel}\]\index{10phi@$\tc{\gamma}$}

Consider the tangent indicatrix $\tan_i$ for each arc $\gamma_i$;
these are smooth spherical arcs.

The same argument as in the proof of Fenchel's theorem, shows that the curves $\tan_1,\dots,\tan_n$ cannot lie in an open hemisphere.

Note that the spherical distance from the endpoint of $\tan_i$ to the starting point of $\tan_{i+1}$ is equal to the external angle $\theta_i$ (we enumerate the arcs modulo $n$, so $\gamma_{n+1}=\gamma_1$).
Let us connect the endpoint of $\tan_i$ to the starting point of $\tan_{i+1}$ by a short arc of a great circle in the sphere.
This way we get a closed spherical curve that is $\theta_1+\dots+\theta_n$ longer then the total length of $\tan_1,\dots,\tan_n$.

Applying the hemisphere lemma (\ref{lem:hemisphere}) to the obtained closed curve, we get that
\[\length\tan_1+\dots+\length\tan_n+\theta_1+\dots+\theta_n\ge 2\cdot\pi.\]
By \ref{obs:tantrix}, the statement follows.
\qedsf

\begin{thm}{Chord lemma}\label{lem:chord}
Let $\gamma\:[a,b]\z\to\RR^3$
be a smooth regular arc, and
$\ell$ be its chord.
Assume $\gamma$ meets $\ell$ at angles $\alpha$ and $\beta$ at $\gamma (a)$ and $\gamma (b)$, respectively;
that is,
\[\alpha=\measuredangle(\vec w,\gamma'(a))\quad\text{and}\quad \beta=\measuredangle(\vec w,\gamma'(b)),\]
where $\vec w=\gamma(b)-\gamma(a)$.
Then 
\[\tc\gamma\ge \alpha+\beta.\eqlbl{tc>a+b}\] 

\end{thm}

%This lemms is due to Alexandr Alexandrov and Yurii Reshetnyak \cite{aleksandrov-reshetnyak}.

\parit{Proof.}
Let us parameterize the chord $\ell$ from $\gamma(b)$ to $\gamma(a)$ and consider the cyclic concatenation $\bar\gamma$ of $\gamma$ and $\ell$.
The closed curve $\bar\gamma$ has two external angles $\pi-\alpha$ and $\pi-\beta$.

\begin{wrapfigure}{r}{45 mm}
\vskip-5mm
\centering
\includegraphics{mppics/pic-53}
\vskip0mm
\end{wrapfigure}

Since the curvature of $\ell$ vanishes, we get 
\[\tc{\bar\gamma}=\tc\gamma+(\pi-\alpha)+(\pi-\beta).\]
According to the generalized Fenechel's theorem (\ref{thm:gen-fenchel}),
$\tc{\bar\gamma}\ge 2\cdot\pi$;
hence \ref{tc>a+b} follows.
\qeds

\begin{thm}{Exercise}\label{ex:chord-lemma-optimal}
Show that the estimate in the chord lemma is optimal.
That is, given two points $p, q$ and two unit vectors $u,v$ in $\RR^3$,
show that there is a smooth regular curve $\gamma$ that starts at $p$ in the direction $\vec u$ and ends at $q$ in the direction $\vec v$ such that 
$\tc\gamma$ is arbitrarily close to $\measuredangle(\vec w,\vec u)+\measuredangle(\vec w,\vec v)$, where $\vec w=q-p$.

\end{thm}

\section{Polygonal lines} 

Polygonal lines are a particular case of piecewise smooth regular curves;
each arc in its concatenation is a line segment.
Since the curvature of a line segment vanishes, the total curvature of a polygonal line is the sum of its external angles.

\begin{thm}{Exercise}\label{ex:monotonic-tc}
Let $a,b,c,d$ and $x$ be distinct points in $\RR^3$.
Show that the total curvature of the polygonal line $abcd$ cannot exceed the total curvature of $abxcd$; that is, 
\[\tc {abcd} \leq \tc {abxcd}.\]

Use this statement to show that any closed polygonal line has curvature at least $2\cdot\pi$.
\end{thm}



\begin{thm}{Proposition}\label{prop:inscribed-total-curvature}
Assume a polygonal line $\beta=p_0\dots p_n$ is inscribed in a smooth regular curve $\gamma$.
Then 
\[\tc\gamma\ge \tc{\beta}.\]
Moreover if $\gamma$ is closed we allow the inscribed polygonal line $\beta$ to be closed.

\end{thm}

\parit{Proof.}
Since the curvature of line segments vanishes, 
the total curvature of the polygonal line is the sum of external angles $\theta_i=\pi-
\measuredangle\hinge {p_i}{p_{i-1}}{p_{i+1}}$.

\begin{wrapfigure}{o}{40 mm}
\vskip-0mm
\centering
\includegraphics{mppics/pic-55}
\vskip0mm
\end{wrapfigure}

Assume $p_i=\gamma(t_i)$.
Set 
\begin{align*}
\vec w_i&=p_{i+1}-p_i,& \vec v_i&=\gamma'(t_i),
\\
\alpha_i&=\measuredangle(\vec w_i,\vec v_i),&\beta_i&=\measuredangle(\vec w_{i-1},\vec v_i).
\end{align*}
In the case of a closed curve we use indexes modulo $n$;
so in this case we have $p_{n+1}\z=p_1$.

Note that $\theta_i=\measuredangle(\vec w_{i-1},\vec w_i)$.
By triangle inequality for angles \ref{thm:spherical-triangle-inq}, we get that
\[\theta_i\le \alpha_i+\beta_i.\]
By the chord lemma, the total curvature of the arc of $\gamma$ from $p_i$ to $p_{i+1}$ is at least $\alpha_i+\beta_{i+1}$. 

Therefore if $\gamma$ is a closed curve, we have
\begin{align*}
\tc{\beta}&=\theta_1+\dots+\theta_n\le 
\\
&\le\beta_1+\alpha_1+\dots+\beta_n+\alpha_n = 
\\
&=(\alpha_1+\beta_2)+\dots+(\alpha_n+\beta_1) \le 
\\
&\le \tc\gamma.
\end{align*}
If $\gamma$ is an arc, the argument is analogous:
\begin{align*}
\tc{\beta}&=\theta_1+\dots+\theta_{n-1}\le 
\\
&\le\beta_1+\alpha_1+\dots+\beta_{n-1}+\alpha_{n-1} \le
\\
&\le (\alpha_0+\beta_1)+\dots+(\alpha_{n-1}+\beta_n) \le 
\\
&\le \tc\gamma.
\end{align*}
\qedsf

The following exercise states that the inequality in \ref{prop:inscribed-total-curvature} is optimal.

\begin{thm}{Exercise}\label{ex:total-curvature=}
Show that for any regular smooth space curve $\gamma$ we have that 
\[\tc\gamma=\sup\{\tc\beta\},\]
where the least upper bound is taken over all polygonal lines~$\beta$ inscribed in $\gamma$
(if $\gamma$ is closed we assume that so is $\beta$).
\end{thm}

This exercise can be used to generalize the notion of total curvature of arbitrary curve $\gamma$.
Namely it can be defined as \emph{the least upper bound on the total curvatures of inscribed nondegenerate polygonal lines inscribed in $\gamma$.}

It is possible to generalize most of the statements in this chapter to the (nonsmooth) curves of finite total curvature.
This theory was developed by Alexandr Alexandrov and Yuri Reshetnyak \cite{aleksandrov-reshetnyak};
a good survey on the subject is written by John Sullivan \cite{sullivan-curves}.

\begin{thm}{Exercise}\label{ex:sef-intersection}

\begin{subthm}{ex:sef-intersection:<2pi}Draw a smooth regular plane curve $\gamma$ that has a self-intersection and such that $\tc\gamma<2\cdot\pi$.
\end{subthm}

\begin{subthm}{ex:sef-intersection:>pi} Show that if a smooth regular curve $\gamma\:[a,b]\to\RR^3$ has a self-intersection, then $\tc\gamma>\pi$.
\end{subthm}

\end{thm}

\begin{thm}{Proposition}\label{prop:fenchel=}
The equality case in Fenchel's theorem holds only for convex plane curves;
that is, if the total curvature of a smooth regular space curve $\gamma$ equals $2\cdot\pi$, then $\gamma$ is a convex plane curve.
\end{thm}

The proof is an application of Proposition~\ref{prop:inscribed-total-curvature}.

\parit{Proof.}
Consider an inscribed quadrangle $abcd$ in $\gamma$.
By the definition of total curvature, we have that
\begin{align*}
\tc{abcd}&=(\pi-
\measuredangle\hinge adb)+(\pi-
\measuredangle\hinge bac)+(\pi-
\measuredangle\hinge cbd)+(\pi-
\measuredangle\hinge dca)=
\\
&=4\cdot\pi -(
\measuredangle\hinge adb
+
\measuredangle\hinge bac
+
\measuredangle\hinge cbd
+
\measuredangle\hinge dca))
\end{align*}


Note that 
\[
\measuredangle\hinge bac
\le
\measuredangle\hinge bad
+ 
\measuredangle\hinge bdc
\quad\text{and}\quad
\measuredangle\hinge dca\le
\measuredangle\hinge dcb
+ 
\measuredangle\hinge dba.
\eqlbl{eq:spheric-triangle}
\]

\begin{wrapfigure}[9]{r}{37 mm}
\vskip-2mm
\centering
\includegraphics{mppics/pic-56}
\vskip0mm
\end{wrapfigure}

The sum of angles in any triangle is $\pi$, so combining these inequalities, we get that 
\begin{align*}
\tc{abcd}\ge 4\cdot \pi 
&- (\measuredangle\hinge adb+\measuredangle\hinge bad+ 
\measuredangle\hinge dba)-
\\
&-(\measuredangle\hinge cbd+\measuredangle\hinge dcb 
+\measuredangle\hinge  bdc)=
\\
=2\cdot\pi.&
\end{align*}

By \ref{prop:inscribed-total-curvature},
\[\tc{abcd}\le \tc\gamma\le 2\cdot\pi.\]
Therefore we have equalities in \ref{eq:spheric-triangle}.
It means that the point $d$ lies in the angle $abc$ 
and the point $b$ lies in the angle $cda$.
That is, $abcd$ is a convex plane quadrangle.

It follows that any quadrangle inscribed in $\gamma$ is a convex plane quadrangle.
Therefore all points of $\gamma$ lie in one plane defined by three points on $\gamma$.
Further, since any quadrangle inscribed in $\gamma$ is a convex,
we get that $\gamma$ is a convex plane curve. 
\qeds

\begin{wrapfigure}{r}{30 mm}
\vskip-0mm
\centering
\includegraphics{mppics/pic-20}
\vskip0mm
\end{wrapfigure}

\begin{thm}{Exercise}\label{ex:quadrisecant}
Suppose that a closed curve $\gamma$ crosses a line at four points $a$, $b$, $c$ and $d$.
Assume that these points appear on the line in the order $a$, $b$, $c$, $d$
and they appear on the curve $\gamma$ in the order $a$, $c$, $b$, $d$.
Show that 
\[\tc\gamma\ge 4\cdot\pi.\]

\end{thm}

Lines crossing a curve at four points as in the exercise are called \index{alternating quadrisecants}\emph{alternating quadrisecants}.
It turns out that any {}\emph{nontrivial knot} admits an alternating quadrisecant \cite{denne};
according to the exercise the latter implies the so-called \index{F\'ary--Milnor theorem}\emph{F\'ary--Milnor theorem} --- the total curvature of any knot exceeds~$4\cdot \pi$.

\section{Bow lemma}

The following lemma was proved by Axel Schur \cite{shur}.

\begin{thm}{Lemma}\label{lem:bow}
Let $\gamma_1\:[a,b]\to\RR^2$ and $\gamma_2\:[a,b] \to\RR^3$ be two smooth unit-speed curves.
Suppose that $\kur(s)_{\gamma_1}\ge\kur(s)_{\gamma_2}$ for any $s$ 
and the curve
$\gamma_1$ is an arc of a convex curve; that is, it runs in the boundary of a convex plane figure.
Then the distance between the ends of $\gamma_1$ cannot exceed the  distance between the ends of $\gamma_2$; that is,
\[|\gamma_1(b)-\gamma_1(a)|\le |\gamma_2(b)-\gamma_2(a)|.\]

\end{thm}

The following exercise states that the condition that $\gamma_1$ is a convex arc is necessary.
It is instructive to do this exercise before reading the proof of the lemma.

{\sloppy 

\begin{thm}{Exercise}\label{ex:anti-bow}
Construct two simple smooth unit-speed plane curves $\gamma_1,\gamma_2\:[a,b]\to\RR^2$ such that 
that $\kur(s)_{\gamma_1}>\kur(s)_{\gamma_2}>0$ for any $s$ and
\[|\gamma_1(b)-\gamma_1(a)|> |\gamma_2(b)-\gamma_2(a)|.\]
\end{thm}

}

\parit{Proof.}
Denote by $\tan_1$ and $\tan_2$ the tangent indicatrixes of $\gamma_1$ and $\gamma_2$, respectively.

Let $\gamma_1(s_0)$ be the point on $\gamma_1$ furthest to the line 
thru $\gamma(a)$ and $\gamma(b)$.
Consider two unit vectors 
\[\vec u_1=\tan_1(s_0)=\gamma_1'(s_0)
\quad\text{and}\quad
\vec u_2=\tan_2(s_0)=\gamma_2'(s_0).\]

Since $\gamma_1$ is an arc of a convex curve, its indicatrix $\tan_1$ runs in one direction along the unit circle.
Suppose $s\le s_0$, then 

{

\begin{wrapfigure}{o}{39 mm}
\vskip14mm
\centering
\includegraphics{mppics/pic-57}
\vskip0mm
\end{wrapfigure}

\vskip-4mm

\begin{align*}
\measuredangle(\gamma'_1(s),\vec u_1)&=\measuredangle(\tan_1(s),\tan_1(s_0))=
\\
&=\length (\tan_1|_{[s,s_0]})=
\\
&=\int_s^{s_0}|\tan_1'(t)|\cdot d t=
\\
&=\int_s^{s_0}\kur_1(t)\cdot d t\ge
\\
&\ge
\int_s^{s_0}\kur_2(t)\cdot d t=
\\
&=\int_s^{s_0}|\tan_2'(t)|\cdot d t= 
\\
&=\length (\tan_1|_{[s,s_0]})\ge
\\
&\ge \measuredangle(\tan_2(s),\tan_2(s_0))=
\\
&= \measuredangle(\gamma'_2(s),\vec u_2).
\end{align*}
That is, 
\[\measuredangle(\gamma'_1(s),\vec u_1)\ge \measuredangle(\gamma'_2(s),\vec u_2)
\eqlbl{<gamma',u}
\]
if $s\ge s_0$.
The same argument shows that \ref{<gamma',u} holds true for $s\ge s_0$.
Therefore the inequality \ref{<gamma',u} holds for any $s$.

}

Since
\[1=|\gamma_1'(s)|=|\gamma_2'(s)|=|\vec u_1|=|\vec u_2|,\] 
the inequality \ref{<gamma',u} implies that 
\[\langle\gamma'_1(s),\vec u_1\rangle\le \langle\gamma'_2(s),\vec u_2\rangle\eqlbl{<gamma',u>}\]
for any $s$.

Further, since $\vec u_1$ is a unit vector parallel to $\gamma_1(b)-\gamma_1(a)$, we have that
\[|\gamma_1(b)-\gamma_1(a)|=\langle \vec u_1,\gamma_1(b)-\gamma_1(a)\rangle,\]
and since $\vec u_2$ is a unit vector, we have that
\[|\gamma_2(b)-\gamma_2(a)|\ge\langle \vec u_2,\gamma_2(b)-\gamma_2(a)\rangle.\]

Integrating \ref{<gamma',u>}, we get 
\begin{align*}
|\gamma_1(b)-\gamma_1(a)|&=\langle \vec u_1,\gamma_1(b)-\gamma_1(a)\rangle=
\\
&=\int_a^b\langle \vec u_1,\gamma'_1(s)\rangle\cdot ds \le 
\\
&\le\int_a^b\langle \vec u_2,\gamma'_2(s)\rangle\cdot ds =
\\
&=\langle \vec u_2,\gamma_2(b)-\gamma_2(a)\rangle \le
\\
&\le |\gamma_2(b)-\gamma_2(a)|.
\end{align*}
\qedsf

\begin{thm}{Exercise}\label{ex:length-dist}
Let $\gamma\:[a,b]\to \RR^3$ be a smooth regular curve and $0\z<\theta\z\le\tfrac\pi2$.
Assume 
\[\tc\gamma\le 2\cdot\theta.\]

\begin{subthm}{ex:length-dist:>} Show that
\[|\gamma(b)-\gamma(a)|> \cos\theta\cdot\length\gamma.\]
\end{subthm}

\begin{subthm}{} Use part \ref{SHORT.ex:length-dist:>} to give another solution of \ref{ex:sef-intersection:>pi}.
\end{subthm}

\begin{subthm}{} Show that the inequality in \ref{SHORT.ex:length-dist:>} is optimal; that is, given 
$\theta$ there is a smooth regular curve $\gamma$ such that $\frac{|\gamma(b)-\gamma(a)|}{\length\gamma}$ is arbitrarily close to $\cos\theta$.
\end{subthm}

\end{thm}

The statement in the following exercise is generally attributed to Hermann Schwarz \cite{shur}.

\begin{thm}{Exercise}\label{ex:schwartz}
Let $p$ and $q$ be points in a unit circle dividing it in two arcs with lengths $\ell_1<\ell_2$.
Suppose a space curve $\gamma$ connects $p$ to $q$ and has curvature at most 1. Show that either
\[\length \gamma\le \ell_1
\quad\text{or}\quad
\length \gamma\ge \ell_2.
\]
\end{thm}

The following exercise generalizes \ref{ex:length>=2pi}.

\begin{thm}{Exercise}\label{ex:loop}
Suppose $\gamma\:[a,b]\to \RR^3$ is a smooth regular loop with curvature at most 1.
Show that 
\[\length\gamma\ge2\cdot\pi.\]

\end{thm}


