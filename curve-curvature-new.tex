\chapter{Space curves}

\section*{Acceleration of unit-speed curve}

Recall that any regular smooth curve can be parametrized by its arc length.
The obtained parametrized curve, say $\gamma$, remains to be smooth and it has unit speed; 
that is, $|\gamma'(s)|=1$ for all $s$.

The following proposition states that the acceleration vector is perpendicular to the velocity vector if the speed remains constant.

\begin{thm}{Proposition}\label{prop:a'-pertp-a''}
Assume $\gamma$ is a smooth unit-speed space curve.
Then $\gamma'(s)\perp \gamma''(s)$ for any $s$.
\end{thm}

The scalar product (also known as dot product) of two vectors $v$ and $w$ will be denoted by $\langle v,w\rangle$.
Recall that the derivative of a scalar product satisfies the product rule;
that is if $v=v(t)$ and $w=w(t)$ are smooth vector-valued functions of a real parameter $t$, then
\[\langle v,w\rangle'=\langle v',w\rangle+\langle v,w'\rangle.\]

\parit{Proof.}
The identity $|\gamma'|=1$ can be rewritten as $\langle\gamma',\gamma'\rangle=1$.
Therefore
\[2\cdot\langle\gamma'',\gamma'\rangle=\langle\gamma',\gamma'\rangle'=0,\]
whence $\gamma''\perp\gamma'$.
\qeds

\section*{Curvature}

For a unit speed smooth space curve $\gamma$ the magnitude of its acceleration $|\gamma''(s)|$ is called its \emph{curvature} at the time $s$.
\label{page:curvature}
If $\gamma$ is simple, then we can say that $|\gamma''(s)|$ is the curvature at the point $p=\gamma(s)$ without ambiguity.
The curvature is usually denoted by $\kur(s)$ or $\kur_\gamma(s)$ %??? do I need it
and in the latter case it might be also denoted by $\kur(p)$ or $\kur_\gamma(p)$. %??? do I need it

The curvature measures how fast the curve turns;
if you drive along a plane curve, curvature tells how much to turn the steering wheel at the given point (note that it does not depend on your speed).
In general, the term \emph{curvature} is used for different types of geometric objects, and it always measures how much it deviates from being \emph{straight};
for curves, it measures how fast it deviates from a straight line.

\begin{thm}{Exercise}\label{ex:curvature-of-spherical-curve}
Show that any regular smooth spherical curve has curvature at least 1 at each time.
\end{thm}

\parit{Hint:} Differentiate the identity $\langle\gamma(s),\gamma(s)\rangle=1$ a couple of times.



\section*{Tangent indicatrix}

Let $\gamma$ be a regular smooth space curve.
Let us consider another curve 
\[\tau(t)=\tfrac{\gamma'(t)}{|\gamma'(t)|}\eqlbl{eq:tantrix}\] 
that is called \emph{tangent indicatrix} of $\gamma$.
Note that $|\tau(t)|=1$ for any $t$;
that is, $\tau$ is a spherical curve.

The line thru $\gamma(s)$ in the direction of $\tau(s)$ is called \emph{thangent line} at $s$.

If $\gamma$ has a unit speed parametrization, then $\tau(t)=\gamma'(t)$.
In this case we have the following expression for curvature: 
$\kur(t)=|\tau'(t)|=|\gamma''(t)|$.

In general case we have 
\[ \kur(t)=\tfrac{|\tau'(t)|}{|\gamma'(t)|}.\eqlbl{eq:curvature}\]
Indeed, for an arc length parametrization $s(t)$ we have $s'(t)=|\gamma'(t)|$.
Therefore
\begin{align*}
\kur(t)&=|\tfrac{d\tau}{ ds}|=
\\
&=|\tfrac{d\tau}{ dt}|/|\tfrac{ds}{ dt}|=
\\
&=\tfrac{|\tau'(t)|}{|\gamma'(t)|}.
\end{align*}
It follows that indicatrix of a smooth regular curve $\gamma$ is regular if the curvature of $\gamma$ does not vanish. 


\begin{thm}{Exercise}\label{ex:curvature-formulas}
Use the formulas \ref{eq:tantrix} and \ref{eq:curvature} to show that 
for any smooth regular space curve $\gamma$ we have the following expressions for its curvature:

\begin{enumerate}[(a)]
\item\label{ex:curvature-formulas:a} \[\kur(t)=\frac{|\gamma''(t)^\perp|}{|\gamma'(t)|^2},\]
where $\gamma''(t)^\perp$ denotes the projection of $\gamma''(t)$ to the normal plane of $\gamma'(t)$;
\item \[\kur(t)=\frac{|\gamma''(t)\times \gamma'(t)|}{|\gamma'(t)|^{3}},\]
where $\times$ denotes the \emph{vector product} (also known as \emph{cross product}).
\end{enumerate}
\end{thm}

\parit{Hint:} Prove and use the following identities: 
\begin{align*}
\gamma''(t)-\gamma''(t)^\perp&=\tfrac{\gamma'(t)}{|\gamma'(t)|}\cdot\langle\gamma''(t),\tfrac{\gamma'(t)}{|\gamma'(t)|}\rangle,
\\
|\gamma'(t)|&=\sqrt{\langle \gamma'(t),\gamma'(t)\rangle}.\
\end{align*}


\begin{thm}{Exercise}\label{ex:curvature-graph}
Apply the formulas in the previous exercise to show that if $f$ is a smooth real function,
then its graph $y=f(x)$  has curvature
\[\kur(p)=\frac{|f''(x)|}{(1+f'(x)^2)^{\frac32}}\]
at the point $p=(x,f(x))$.
\end{thm}

\section*{Total curvature}

Let $\gamma\:\mathbb{I}\to\RR^3$ be a regular smooth curve and $\tau$ its tangent indicatrix.
Recall that without loss of generality we can assume that $\gamma$ has a unit speed parametrization;
in this case $\tau(s)=\gamma'(s)$ and hence
\[\kur(s)\df|\gamma''(s)|=|\tau'(s)|;\] 
that is, the curvature of $\gamma$ at time $s$ is the speed of the tangent indicatrix $\tau$ at the same time moment.

The integral 
\[\tc\gamma:=\int_{\mathbb{I}}\kur(s)\cdot ds\]
is called \emph{total curvature of}\label{page:total curvature of:smooth-def}
$\gamma$.

\begin{thm}{Exercise}\label{ex:helix-curvature}
Find the curvature of the helix \[\gamma_{a,b}(t)=(a\cdot \cos t,a\cdot \sin t,b\cdot t),\] its tangent indicatrix and the total curvature of its arc $t\in[0,2\cdot\pi]$.
\end{thm}

\begin{thm}{Observation}\label{obs:tantrix}
The total curvature of a smooth regular curve is the length of its tangent indicatrix.
\end{thm}

\parit{Proof.}
It is sufficient to prove the observation for a unit-speed space curve $\gamma\:\mathbb{I}\to\RR^3$.
Denote by $\tau$ its tangent indicatrix.
Then
\begin{align*}
\tc\gamma&\df\int_{\mathbb{I}}\kur(s)\cdot ds=
\\
&=\int_{\mathbb{I}}|\tau'(s)|\cdot ds=
\\
&=\length\tau.
\end{align*}
\qedsf %???other cases --- colsed curves, semiopen intervals...

\begin{thm}{Fenchel's theorem}\label{thm:fenchel}
The total curvature of any closed regular space curve is at least $2\cdot\pi$.
\end{thm}

\parit{Proof.}
Fix a closed regular space curve $\gamma$;
we can assume that it is described by a loop $\gamma\:[a,b]\to \RR^3$;
in this case $\gamma(a)=\gamma(b)$ and $\gamma'(a)=\gamma'(b)$.

Consider its tangent indicatrix $\tau=\gamma'$.
Recall that $|\tau(s)|=1$ for any $s$; that is, $\tau$ is a closed spherical curve.

Let us show that $\tau$ can not lie in a hemisphere.
Assume the contrary; without loss of generality we can assume that $\tau$ lies in the north hemisphere defined by the inequality $z>0$ in $(x,y,z)$-coordinates.
It means that $z'(t)>0$ at any time, where $\gamma(t)=(x(t), y(t), z(t))$.
Therefore 
\[z(b)-z(a)=\int_a^bz'(s)\cdot ds>0.\]
In particular, $\gamma(a)\ne \gamma(b)$, a contradiction.

Applying the observation (\ref{obs:tantrix}) and the hemisphere lemma (\ref{lem:hemisphere}), we get that 
\[\tc\gamma=\length \tau\ge2\cdot\pi.\]
\qedsf

\begin{thm}{Exercise}\label{ex:length>=2pi}
Show that a closed space curve $\gamma$ with curvature at most~$1$ can not be shorter than the unit circle;
that is, $\length\gamma\ge 2\cdot \pi$.
\end{thm}


\begin{thm}{Advanced exercise}
Suppose that $\gamma$ is a smooth regular space curve that does not pass thru the origin.
Consider the spherical curve defined as $\sigma(t)=\frac{\gamma(t)}{|\gamma(t)|}$ for any $t$.
Show that 
\[\length \sigma< \tc\gamma+\pi.\]
Moreover, if $\gamma$ is closed, then
\[\length \sigma\le \tc\gamma.\]
\end{thm}


%???Hint: Assume that $\gamma$ is unit-speed; show that $|\alpha'|\le \kur+\theta'$, where $\theta(s)=\angle(\gamma(s),\gamma'(s))$.

Note that the last inequality gives an alternative proof of Fenchel's theorem.
Indeed, without loss of generality we can assume that the origin lies on a chord of $\gamma$;
in this case the spherical curve $\sigma$ passes thru a pair of antipodal points in $\mathbb{S}^2$;
whence 
\[\length\sigma\ge 2\cdot\pi.\]



\section*{Piecewise smooth curves}

Assume $\alpha\:[a,b]\to \RR^3$ and $\beta\:[b,c]\z\to \RR^3$ are two curves such that $\alpha(b)=\beta(b)$.
Then one can combine these two curves into one $\gamma\:[a,c]\z\to \RR^3$ by the rule 
\[\gamma(t)=
\begin{cases}
\alpha(t)&\text{if}\quad t\le b,
\\
\beta(t)&\text{if}\quad t\ge b.
\end{cases}
\]
The obtained curve $\gamma$ is called the 
\emph{concatenation} of $\alpha$ and $\beta$. %briefly $\gamma=\alpha{*}\beta$.
(The condition $\alpha(b)=\beta(b)$ ensures that the map $t\mapsto\gamma(t)$ is continuous.)

\begin{wrapfigure}{r}{30 mm}
\vskip-0mm
\centering
\includegraphics{mppics/pic-54}
\end{wrapfigure}

The same definition of cancatination can be applied if $\alpha$ and/or $\beta$ are defied on semiopen intervals 
$(a,b]$ and/or $[b,c)$.

The concatenation can be also defined if the end point of the first curve coincides with the starting point of the second curve;
if this is the case, then the time intervals of both curves can be shifted so that they fit together. 

If in addition $\beta(c)=\alpha(a)$ then we can do cyclic concatination of these curves;
this way we obtain a closed curve.

If $\alpha'(b)$ and $\beta'(b)$ are defined then the angle $\theta=\measuredangle(\alpha'(b),\beta'(b))$ is called \emph{external angle} of $\gamma$ at time $b$.

A space curve $\gamma$ is called \emph{piecewise smooth and regular} if it can be presented as a concatination of finite number of smooth regular curves; if $\gamma$ is closed, then the  concatination is assumed to be cyclic.

If $\gamma$ is a concatination of smooth regular arcs $\gamma_1,\dots,\gamma_n$, then the total curvature of $\gamma$ is defined as a sum of the total curvatures of $\gamma_i$ and the external angles;
that is, 
\[\tc\gamma=\tc{\gamma_1}+\dots+\tc{\gamma_n}+\theta_1+\dots+\theta_{n-1}\]
where $\theta_i$ is the external angle at the joint $\gamma_i$ and $\gamma_{i+1}$;
if $\gamma$ is closed, then 
\[\tc\gamma=\tc{\gamma_1}+\dots+\tc{\gamma_n}+\theta_1+\dots+\theta_{n},\]
where $\theta_n$ is the external angle at the joint $\gamma_n$ and $\gamma_1$.

\begin{thm}{Generalized Fenchel's theorem}\label{thm:gen-fenchel}
Let $\gamma$ be a closed piecewise smooth regular space curve.
Then 
\[\tc\gamma\ge2\cdot\pi.\]

\end{thm}

\parit{Proof.}
Suppose $\gamma$ is a cyclic concatenation of $n$ smooth regular arcs $\gamma_1,\dots,\gamma_n$.
Denote by $\theta_1,\dots,\theta_n$ its external angles.
We need to show that
\[\tc{\gamma_1}+\dots+\tc{\gamma_n}+\theta_1+\dots+\theta_n\ge2\cdot\pi.\eqlbl{eq:gen-fenchel}\]

Consider the tangent indicatrix $\tau_1,\dots,\tau_n$ for each arc $\gamma_1,\dots,\gamma_n$;
these are smooth spherical arcs.

The same argument as in the proof of Fenchel's theorem, shows that the curves $\tau_1,\dots,\tau_n$ can not lie in an open hemisphere.

Note that the spherical distance from the end point of $\tau_i$ to the starting point of $\tau_{i+1}$ is equal to the external angle $\theta_i$ (we enumerate modulo $n$, so $\gamma_{n+1}=\gamma_1$).
Therefore if we connect the end point of $\tau_i$ to the starting point of $\tau_{i+1}$ by a short arc of a great circle in the sphere, then we add $\theta_1+\dots+\theta_n$ to the total length of $\tau_1,\dots,\tau_n$.

Applying the hemisphee lemma (\ref{lem:hemisphere}) to the obtained closed curve, we get that
\[\length\tau_1+\dots+\length\tau_n+\theta_1+\dots+\theta_n\ge 2\cdot\pi.\]
Applying the observation (\ref{obs:tantrix}), we get \ref{eq:gen-fenchel}.
\qedsf

\begin{thm}{Chord lemma}\label{lem:chord}
Let $\ell$ be the chord to a smooth regular arc $\gamma\:[a,b]\to\RR^3$.
Assume $\gamma$ meets $\ell$ at angles $\alpha$ and $\beta$ at its ends;
that is 
\[\alpha=\measuredangle(w,\gamma'(a))\quad\text{and}\quad \beta=\measuredangle(w,\gamma'(b)),\]
where $w=\gamma(b)-\gamma(a)$.
Then 
\[\tc\gamma\ge \alpha+\beta.\] 

\end{thm}

\begin{wrapfigure}{r}{45 mm}
\vskip-0mm
\centering
\includegraphics{mppics/pic-53}
\vskip0mm
\end{wrapfigure}

%??? is it really due to Reshetnyak???

\parit{Proof.}
Let us parameterize the chord $\ell$ from $\gamma(b)$ to $\gamma(a)$ and consider the cyclic concatenation $\bar\gamma$ of $\gamma$ and $\ell$.
The closed curve $\bar\gamma$ has two external angles $\pi-\alpha$ and $\pi-\beta$.
Since curvature of $\ell$ vanish, we get that 
\[\tc{\bar\gamma}=\tc\gamma+(\pi-\alpha)+(\pi-\beta).\]
According to the generalized Fenechel's theorem (\ref{thm:gen-fenchel}),
\[\tc{\bar\gamma}\ge 2\cdot\pi;\]
hence the result.
\qeds

\begin{thm}{Exercise}\label{ex:chord-lemma-optimal}
Show that the estimate in the chord lemma is optimal.

That is, given two points $p, q$ and two nonzero vectors $u,v$ in $\RR^3$,
show that there is a smooth regular curve $\gamma$ that starts at $p$ in the direction of $u$ and ends at $q$ in the direction of $v$ such that 
$\tc\gamma$ is arbitrary close to $\measuredangle(w,u)+\measuredangle(w,v)$, where $w=q-p$.

\end{thm}

\section*{Polygonal lines} 

Polygonal lines are partial case of piecewise smooth regular curves;
each arc in its concatenation is a line segment.
Since the curvature of a line segment vanish, the total curvature of polygonal line is the sum of its external angles.

\begin{thm}{Exercise}\label{ex:monotonic-tc}
Let $a,b,c,d$ and $x$ be distinct points in $\RR^3$.
Show that the total curvature of polygonal line $abcd$ can not exceed the total curvature of $abxcd$; that is, 
\[\tc {abcd} \leq \tc {abxcd}.\]

Use this statement to show that any closed polygonal line has curvature at least $2\cdot\pi$.
\end{thm}

\parit{Hint:} 
Use that exterior angle of a triangle equals to the sum of the two remote interior angles;
for the second part apply the induction on number of vertexes.

\begin{thm}{Proposition}\label{prop:inscribed-total-curvature}
Assume a polygonal line $\hat\gamma=p_1\dots p_n$ is inscribed in a smooth regular curve $\gamma$.
Then 
\[\tc\gamma\ge \tc{\hat\gamma}.\]
Moreover if $\gamma$ is closed we can assume that the inscribed polygonal line $\hat\gamma$ is also closed.

\end{thm}

\parit{Proof.}
Since the curvature of line segments vanishes, 
the total curvature of polygonal line is the sum of external angles $\theta_i=\pi-\measuredangle p_{i-1}p_ip_{i+1}$.

\begin{wrapfigure}{o}{40 mm}
\vskip-0mm
\centering
\includegraphics{mppics/pic-55}
\vskip0mm
\end{wrapfigure}

Assume $p_i=\gamma(t_i)$.
Set 
\begin{align*}
w_i&=p_{i+1}-p_i,& v_i&=\gamma'(t_i),
\\
\alpha_i&=\measuredangle (w_i,v_i),&\beta_i&=\measuredangle (w_{i-1},v_i).
\end{align*}
In case of closed curve we use indexes modulo $n$, in particular $p_{n+1}\z=p_1$.

Note that $\theta_i=\measuredangle (w_{i-1},w_i)$.
Therefore 
\[\theta_i\le \alpha_i+\beta_i.\]
By the chord lemma, the total curvature of the arc of $\gamma$ from $p_i$ to $p_{i+1}$ is at least $\alpha_i+\beta_{i+1}$. 

Therefore if $\gamma$ is a closed curve, we have
\begin{align*}
\tc{\hat\gamma}&=\theta_1+\dots+\theta_n\le 
\\
&\le\beta_1+\alpha_1+\dots+\beta_n+\alpha_n = 
\\
&=(\alpha_1+\beta_2)+\dots+(\alpha_n+\beta_1) \le 
\\
&\le \tc\gamma.
\end{align*}
If $\gamma$ is an arc, the argument is analogous:
\begin{align*}
\tc{\hat\gamma}&=\theta_2+\dots+\theta_{n-1}\le 
\\
&\le\beta_2+\alpha_2+\dots+\beta_{n-1}+\alpha_{n-1} \le
\\
&\le (\alpha_1+\beta_2)+\dots+(\alpha_{n-1}+\beta_n) \le 
\\
&\le \tc\gamma.
\end{align*}
\qedsf

\begin{thm}{Exercise}\label{ex:sef-intersection}
\begin{enumerate}[(a)]
\item Draw a smooth regular plane curve $\gamma$ which has a self-intersection, such that $\tc\gamma<2\cdot\pi$.
\item\label{ex:sef-intersection:>pi} Show that if a smooth regular curve $\gamma\:[a,b]\to\RR^3$ has a self-intersection, then $\tc\gamma>\pi$.
\end{enumerate}
\end{thm}

\begin{thm}{Proposition}\label{prop:fenchel=}
The equality case in the Fenchel's theorem holds only for convex plane curves;
that is, if the total curvature of a smooth regular space curve $\gamma$ is equal to $2\cdot\pi$, then it is a convex plane curve.
\end{thm}

The proof is an application of Proposition~\ref{prop:inscribed-total-curvature}.

\parit{Proof.}
Consider an inscribed quadraliteral $abcd$ in $\gamma$.
By the definition of total curvature, we have that
\begin{align*}
\tc{abcd}&=(\pi-\measuredangle dab)+(\pi-\measuredangle abc)+(\pi-\measuredangle bcd)+(\pi-\measuredangle cda)=
\\
&=4\cdot\pi -(\measuredangle dab+\measuredangle abc+\measuredangle bcd+\measuredangle cda))
\end{align*}


Note that 
\[
\measuredangle abc\le\measuredangle abd+ \measuredangle dbc
\quad\text{and}\quad
\measuredangle cda\le\measuredangle cdb+ \measuredangle bda.
\eqlbl{eq:spheric-triangle}
\]

The sum of angles in any triangle is $\pi$.
Therefore combining these inequalities, we get that 
\begin{align*}
\tc{abcd}&\ge 4\cdot \pi 
- (\measuredangle dab+\measuredangle abd+ \measuredangle bda)
-(\measuredangle bcd+\measuredangle cdb +\measuredangle dbc)=
\\
=2\cdot\pi.
\end{align*}

\begin{wrapfigure}{r}{40 mm}
\vskip-7mm
\centering
\includegraphics{mppics/pic-56}
\vskip0mm
\end{wrapfigure}

By \ref{prop:inscribed-total-curvature},
\[\tc{abcd}\le \tc\gamma\le 2\cdot\pi.\]
Therefore we have equalities in \ref{eq:spheric-triangle}.
It means that the point $d$ lies in the angle $abc$ 
and the point $b$ lies in the angle $cda$.
That is, $abcd$ is a convex plane quadraliteral.

It follows that any quadraliteral inscribed in $\gamma$ is convex plane quadraliteral.
Therefore all points of $\gamma$ lie in one plane and the domain bounded by $\gamma$ is convex;
that is, $\gamma$ is a convex plane curve.
\qeds

\begin{wrapfigure}{r}{30 mm}
\vskip-0mm
\centering
\includegraphics{mppics/pic-20}
\vskip0mm
\end{wrapfigure}

\begin{thm}{Exercise}\label{ex:quadrisecant}
Suppose that a closed curve $\gamma$ crosses a line at four points $a$, $b$, $c$ and $d$.
Assume that these points appear on the line in the order $a$, $b$, $c$, $d$
and they appear on the curve $\gamma$ in the order $a$, $c$, $b$, $d$.
Show that 
\[\tc\gamma\ge 4\cdot\pi.\]

\end{thm}

A line crossing a curve at four points as in the exercise is called \emph{alternating quadrisecants}.
It turns out that any \emph{nontrivial knot} admits an alternating quadrisecants \cite{denne};
it implies the so called F\'ary--Milnor theorem --- \emph{the total curvature any knot exceeds $4\cdot \pi$}.

\section*{Bow lemma}

\begin{thm}{Lemma}\label{lem:bow}
Let $\gamma_1\:[a,b]\to\RR^2$ and $\gamma_2\:[a,b] \to\RR^3$ be two smooth unit-speed curves;
denote by $\kur_1(s)$ and $\kur_2(s)$ their curvatures at $s$.
Suppose that $\kur_1(s)\ge\kur_2(s)$ for any $s$ 
and the curve
$\gamma_1$ is a simple arc of a convex curve; that is, it runs in the boundary of a covex plane figure.
Then the distance between the ends of $\gamma_1$ can not axceed the  distance between the ends of $\gamma_2$; that is,
\[|\gamma_1(b)-\gamma_1(a)|\le |\gamma_2(b)-\gamma_2(a)|.\]

\end{thm}

\begin{wrapfigure}{r}{36 mm}
\vskip-7mm
\centering
\includegraphics{mppics/pic-57}
\vskip0mm
\end{wrapfigure}

\parit{Proof.}
Denote by $\tau_1$ and $\tau_2$ the tangent indicatrixes of $\gamma_1$ and $\gamma_2$ correspondingly.

Let $\gamma_1(s_0)$ be the point on $\gamma_1$ that maximize the distance to the line 
thru $\gamma(a)$ and $\gamma(b)$.
Consider two unit vectors 
\[u_1=\tau_1(s_0)=\gamma_1'(s_0)
\quad\text{and}\quad
u_2=\tau_2(s_0)=\gamma_2'(s_0).\]
By construction the vector $u_1$ is parallel to $\gamma(b)-\gamma(a)$ in particular
\[|\gamma_1(b)-\gamma_1(a)|=\langle u_1,\gamma_1(b)-\gamma_1(a)\rangle \]

Since $\gamma_1$ is an arc of convex curve, it indicatrix $\tau(s)$ runs in one direction along the unit circle.
Suppose $s\le s_0$, then 
\begin{align*}
\measuredangle(\gamma'_1(s),u_1)&=\measuredangle(\tau_1(s),\tau_1(s_0))=
\\
&=\length (\tau_1|_{[s,s_0]})=
\\
&=\int_s^{s_0}|\tau_1'(t)|\cdot d t=
\\
&=\int_s^{s_0}\kur_1(t)\cdot d t\ge
\\
&\ge
\int_s^{s_0}\kur_2(t)\cdot d t=
\\
&=\int_s^{s_0}|\tau_2'(t)|\cdot d t= 
\\
&=\length (\tau_1|_{[s,s_0]})\ge
\\
&\ge \measuredangle(\tau_2(s),\tau_2(s_0))=
\\
&= \measuredangle(\gamma'_2(s),u_2),
\end{align*}
The same argument shows that 
\[\measuredangle(\gamma'_1(s),u_1)\ge \measuredangle(\gamma'_2(s),u_2)\]
for $s\ge s_0$; therefore the inequality holds for any $s$. 
Since the vectors $\gamma'_1(s),u_1,\gamma'_2(s),u_2$ are unit, it follwos that 
\[\langle\gamma'_1(s),u_1\rangle\le \langle\gamma'_2(s),u_2\rangle.\]
Integrating the last inequality, we get that 
\begin{align*}
|\gamma_1(b)-\gamma_1(a)|&=\langle u_1,\gamma_1(b)-\gamma_1(a)\rangle=
\\
&=\int_a^b\langle u_1,\gamma'_1(s)\rangle\cdot ds \le 
\\
&\le\int_a^b\langle u_2,\gamma'_2(s)\rangle\cdot ds =
\\
&=\langle u_2,\gamma_2(b)-\gamma_2(a)\rangle \le
\\
&\le |\gamma_2(b)-\gamma_2(a)|.
\end{align*}
Hence the result.\qeds

\begin{thm}{Exercise}\label{ex:length-dist}
Let $\gamma\:[a,b]\to \RR^3$ be a smooth regular curve and $0<\theta\le\tfrac\pi2$.
Suppose 
\[\tc\gamma\le 2\cdot\theta.\]
\begin{enumerate}[(a)]
\item\label{ex:length-dist:>} Show that 
\[|\gamma(b)-\gamma(a)|> \cos\theta\cdot\length\gamma.\]
\item Use part (\ref{ex:length-dist:>}) to give another solution of \ref{ex:sef-intersection}\ref{ex:sef-intersection:>pi}.
\item Show that the inequality in (\ref{ex:length-dist:>}) is optimal; that is, given 
$\theta$ there is a smooth regular curve $\gamma$ such that $\frac{|\gamma(b)-\gamma(a)|}{\length\gamma}$ is arbitrary close to $\cos\theta$.

\end{enumerate}

\end{thm}

\parit{Hint:}
Choose a value $s_0\in[a,b]$ that splits the total curvature into two equal parts, $\theta$ in each.
Observe that $\measuredangle(\gamma'(s_0),\gamma'(s))\le \theta$ for any~$s$.
Use this inequality the same way as in the proof of the bow lemma.

\begin{thm}{Exercise}\label{ex:schwartz}
Suppose that two points $p$ and $q$ lie on a unit circle and dividing it in two arcs with lengths $\ell_1<\ell_2$.
Show that if a curve $\gamma$ runs from $p$ to $q$ and has curvature at most 1,
then either
\[\length \gamma\le \ell_1
\quad\text{or}\quad
\length \gamma\ge \ell_2.
\]
\end{thm} %H. A. Schwartz???

\begin{thm}{Exercise}\label{ex:loop}
Suppose $\gamma\:[a,b]\to \RR^3$ is a smooth regular loop with curvature at most 1.
Show that 
\[\length\gamma\ge2\cdot\pi.\]

\end{thm}


\section*{DNA inequality*}

Recall that curvature of a spherical curve is at least $1$
(Exercise~\ref{ex:curvature-of-spherical-curve}).
In particular the length of spherical curve can not exceed its total curvature.
The following theorem shows that the same inequality holds for \emph{closed} curves in a unit ball.

\begin{thm}{Theorem}\label{thm:DNA}
Let $\gamma$ be a smooth regular closed curve that lies in a unit ball.
Then 
\[\tc\gamma\ge \length\gamma.\]

\end{thm}

\parit{Proof.}
Without loss of generality we can assume the curve is described by a loop $\gamma\:[0,\ell]\to\RR^3$ parameterized by its arc length, so $\ell=\length\gamma$.
We can also assume that the origin is the center of the ball.
It follows that
\[\langle\gamma'(s),\gamma'(s)\rangle=1,\qquad |\gamma(s)|\le 1\]
and in particular 
\begin{align*}
\langle\gamma''(s),\gamma(s)\rangle&\ge -|\gamma''(s)|\cdot|\gamma(s)|\ge
\\
&\ge-\kur(s)
\end{align*}
for any $s$, where $\kur(s)=|\gamma''(s)|$ is the curvature of $\gamma$ at $s$.

Since $\gamma$ is closed, we have that
$\gamma'(0)=\gamma'(\ell)$ and $\gamma(0)=\gamma(\ell)$.
Therefore
\begin{align*}
0&=\langle\gamma(\ell),\gamma'(\ell)\rangle
-
\langle\gamma(0),\gamma'(0)\rangle=
\\
&=\int_0^\ell\langle\gamma(s),\gamma'(s)\rangle'\cdot ds=
\\
&=\int_0^\ell\langle\gamma'(s),\gamma'(s)\rangle\cdot ds+\int_0^\ell\langle\gamma(s),\gamma''(s)\rangle\cdot ds\ge
\\
&\ge \ell-\tc\gamma,
\end{align*}
whence the result.
\qeds

This theorem was proved by Don Chakerian \cite{chakerian};
for plane curves it was proved earlier by Istv\'{a}n F\'{a}ry \cite{fary-DNA}.
We present the proof given by Don Chakerian in \cite{chakerian-short};
few other proofs of this theorem are discussed by Serge Tabachnikov~\cite{tabachnikov}.
He also conjectured the following closely related statement:

\begin{thm}{Theorem}
Suppose a closed regular smooth curve $\gamma$ lies in a convex figure with the perimeter $2\cdot \pi$.
Then 
\[\tc\gamma\ge \length\gamma.\]

\end{thm}

It was proved by Jeffrey Lagarias and Thomas Richardson \cite{lagarias-richardso};
latter a simpler proof was found by Alexander Nazarov and Fedor Petrov~\cite{nazarov-petrov}.
The proof is elementary, but annoyingly difficult; we do not present it here.


\section*{Nonsmooth curves*}

\begin{thm}{Theorem}\label{thm:total-curvature=}
For any regular smooth space curve $\gamma$ we have that 
\[\tc\gamma=\sup\{\tc\beta\},\]
where the least upper bound is taken for all polygonal lines~$\beta$ inscribed in $\gamma$;
if $\gamma$ is closed we assume that so is $\beta$.
\end{thm}

\parit{Proof.}
Note that the inequality 
\[\tc\gamma\ge \tc\beta\]
follows from \ref{prop:inscribed-total-curvature};
it remains to show 
\[\tc\gamma\le\sup\{\tc\beta\}. \eqlbl{eq:tc=<tc}\]

Let $\gamma\:[a,b]\to\RR^3$ be a smooth curve.
Fix a partition $a\z=t_0\z<\dots<t_n=b$ and consider the corresponding inscribed polygonal line $\beta=p_0\dots p_n$.
(If $\gamma$ is closed, then  $p_0=p_n$ and $\beta$ is closed as well.)

Let $\chi=\xi_1\dots\xi_n$ be a spherical polygonal line
with the vertexes $\xi_i\z=\tfrac{p_i-p_{i-1}}{|p_i-p_{i-1}|}$.
We can assume that $\chi$ has constant speed on each arc and $\chi(t_i)=\xi_i$ for each $i$. 
The spherical polygonal line $\chi$ will be called tangent indicatrix for $\beta$.

Consider a sequence of finer and finer partitions, denote by $\beta_n$ and $\chi_n$ the corresponding inscribed polygonal lines and their tangent indicatrixes.
Note that since $\gamma$ is smooth, the idicatrixes $\chi_n$ converge pointwise to $\tau$ --- the thangent indicatrix of $\gamma$.
By semi-continuity of the length (\ref{thm:length-semicont}), we get that  
\begin{align*}
\tc\gamma&=\length \tau\le  
\\
&\le \liminf_{n\to\infty}\length \chi_n=
\\
&= \liminf_{n\to\infty}\tc {\beta_n}\le
\\
&\le \sup\{\tc\beta\},
\end{align*}
where the last supremum is taken over all partitions and their corresponding inscribed polygonal lines $\beta$; whence \ref{eq:tc=<tc} follows.
\qeds

The theorem above can be used to define total curvature for arbitrary curves, not necessary (piecewise) smooth and regular. 
We say that a parameterized curve is trivial if it is constant; that is, it stays at one point.

\begin{thm}{Definition}\label{def:total-curv-poly}
The total curvature of a nontrivial parameterized space curve $\gamma$ is the exact upper bound on the total curvatures of inscribed nondegenerate polygonal lines;
if $\gamma$ is closed then we assume that the inscribed polygonal lines are closed as well.
\end{thm}

\begin{thm}{Exercise}
Show that the total curvature is lower semi-continuous with respect to pointwise convergence of curves.
That is, if a sequence
of curves $\gamma_n\:[a,b]\to \RR^3$ converges pointwise 
to a nontrivial curve $\gamma_\infty\:[a,b]\z\to \RR^3$, then 
\[\liminf_{n\to\infty} \tc{\gamma_n} \ge \tc{\gamma_\infty}.\]
\end{thm}

\parit{Hint:} Modify the proof of semi-continuity of length (Theorem~\ref{thm:length-semicont}).

\begin{thm}{Exercise}
Show that Fenchel's theorem holds for any nontrivial closed curve $\gamma$;
that is, 
\[\tc\gamma\ge2\cdot\pi.\]
\end{thm}

\begin{thm}{Exercise} 
Assume that a curve $\gamma\:[a,b]\to\RR^3$ has finite total curvature.
Show that $\gamma$ is rectifiable.

Construct a rectifiable curve $\gamma\:[a,b]\to\RR^3$ that has infinite total curvature.
\end{thm}

A good survey on curves of finite total curvature is written by John Sullivan \cite{sullivan-curves}.


