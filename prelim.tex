\section{Elementary geometry}

\subsection*{Internal angles}

Polygon is defined as a compact set bounded by a closed polygonal line. 
Recall that internal angle of a polygon $P$ at a vertex $v$
is defined as angular measure of the intersection of $P$ with a small small circle centered at $v$.

\begin{thm}{Theorem}\label{thm:sum=(n-2)pi}
The sum of all the internal angles of a simple $n$-gon is $(n-2)\cdot\pi$. 
\end{thm}

While this theorem is well known, it is not easy to find a reference with a proof without cheating.
A clean proof was given by Gary Meisters \cite{meisters}.
It use induction on $n$ and based on the following:

\begin{thm}{Claim}
Suppose $P$ is an $n$-gonwith $n\ge 4$.
Then a diagonal of $P$ lies completely in $P$.
\end{thm}



\subsection*{Angle monotonicity}

The \index{angle measure}\emph{measure} of angle with sides $[px]$ and $[py]$ will be denoted by $\measuredangle\hinge pxy$\index{$\measuredangle\hinge yxz$};
it takes a value in the interval $[0,\pi]$.

The following lemma is very simple and very useful.
It says that the angle of a triangle monotonically depends on the opposite side, assuming the we keep the other two sides fixed.
It follows directly from the cosine rule.

\begin{thm}{Monotonicity lemma}\label{lem:angle-monotonicity}
Let $x$, $y$, $z$, $x^{*}$, $y^{*}$ and $z^{*}$ be 6 points such that $|x-y|\z=|x^{*}-y^{*}|>0$ and $|y-z|=|y^{*}-z^{*}|>0$.
Then 
\[\measuredangle\hinge yxz
\ge
\measuredangle\hinge {y^{*}}{x^{*}}{z^{*}}
\quad\text{if and only if}\quad
|x-z|\ge |x^{*}-z^{*}|.\]
\end{thm}


\subsection*{Spherical triangle inequality}

The following theorem says that the triangle inequality holds for angles between half-lines from a fixed point.
In particular it implies that a sphere with the angle metric is a metric space.

\begin{thm}{Theorem}\label{thm:spherical-triangle-inq}
The following inequality holds for any three line segments $[o,a]$, $[o,b]$ and $[o,c]$ in the Euclidean space:
\[\measuredangle\hinge oab
+
\measuredangle\hinge obc
\ge
\measuredangle\hinge oac\]

\end{thm}

Most of authors use this theorem without mentioning, but the proof is not that simple.
A short elementary proof can be found in the classical textbook in Euclidean geometry by Andrey Kiselyov \cite[\S 47]{kiselyov}.


\subsection*{Area of spherical triangle}

\begin{thm}{Lemma}\label{lem:area-spher-triangle}
Let $\Delta$ be a spherical triangle;
that is, $\Delta$ is the intersection of three closed half-spheres in the unit sphere $\mathbb{S}^2$.
Then 
\[\area\Delta=\alpha+\beta+\gamma-\pi,\eqlbl{eq:area(Delta)}\]
where $\alpha$, $\beta$ and $\gamma$ are the angles of $\Delta$.
\end{thm}

The value $\alpha+\beta+\gamma-\pi$ is called \index{excess of triangle}\emph{excess of the triangle} $\Delta$,
so the lemma says that area of a spherical triangle equals to its excess.

This lemma appears in many texts.
We give its proof here since it is very important in our intuitive proof of Gauss--Bonnet formula.

\begin{wrapfigure}{r}{22 mm}
\vskip-0mm
\centering
\includegraphics{mppics/pic-43}
\vskip2mm
\end{wrapfigure}

\parit{Proof.}
Recall that 
\[\area\mathbb{S}^2=4\cdot\pi.\eqlbl{eq:area(S2)}\]

Note that the area of a spherical slice $S_\alpha$ between two meridians meeting at angle $\alpha$ is proportional to $\alpha$.
Since for $S_\pi$ is a half-sphere, from \ref{eq:area(S2)}, we get $\area S_\pi\z=2\cdot\pi$.
Therefore the coefficient is 2; that is,
\[\area S_\alpha=2\cdot \alpha.\eqlbl{eq:area(Sa)}\]

Extending the sides of $\Delta$ we get 6 slices: two $S_\alpha$, two $S_\beta$ and two $S_\gamma$ which cover most of the sphere once,
but the triangle $\Delta$ and its centrally symmetric copy $\Delta^{*}$ are covered 3 times.
It follows that
\[2\cdot \area S_\alpha+2\cdot \area S_\beta+2\cdot \area S_\gamma
=\area\mathbb{S}^2+4\cdot\area\Delta.\]
Substituting \ref{eq:area(S2)} and \ref{eq:area(Sa)} and simplifying, we get \ref{eq:area(Delta)}.
\qeds





\section{Convex geometry}

A set $X$ in the Euclidean space is called \index{convex set}\emph{convex} if for any two points $x,y\in X$, any point $z$ between $x$ and $y$ lies in $X$.
It is called  \index{strictly convex set}\emph{strictly convex} if for any two points $x,y\in X$, any point $z$ between $x$ and $y$ lies in the interior of $X$.

From the definition, it is easy to see that the intersection of an arbitrary family of convex sets is convex. 
The intersection of all convex sets containing $X$ is called the \index{convex hull}\emph{convex hull} of $X$;
it is the minimal convex set containing the set $X$.

We will use the following corollary of the so-called \index{hyperplane separation theorem}\emph{hyperplane separation theorem}:

\begin{thm}{Lemma}\label{lem:separation}
Let $K\subset \RR^3$ be a closed convex set.
Then for any point $p\notin K$ there is a plane $\Pi$ that separates $K$ from $p$;
that is, $K$ and $p$ lie on opposite open half-spaces separated by $\Pi$.
\end{thm}

These definitions and hyperplane separation should appear on fist few pages of any introductory text in convex geometry;
see for example the book of Roger Webster \cite{webster}.

\section{Linear algebra}

The following theorem can be found in any textbook in linear algebra;
the book of Sergei Treil \cite{treil} will do.

\begin{thm}{Spectral theorem}\label{thm:spectral}
Any symmetric matrix is diagonalizable  by orthogonal matrix.
\end{thm}

We will use this theorem only for $2{\times}2$ matrices.
In this case it can be restated as follows:
Consider a function 
\[f(x,y)=
\begin{pmatrix}
x&y
\end{pmatrix}
\cdot
\begin{pmatrix}
\ell&m
\\
m&n
\end{pmatrix}
\cdot
\begin{pmatrix}
x\\y
\end{pmatrix}
=\ell\cdot x^2+2\cdot m\cdot x\cdot y+n\cdot y^2,\]
that is defined on a $(x,y)$-coordinate plane.
Then after proper rotation of the coordinates, 
the expression for $f$ in the new coordinates will be
\[\bar f(x,y)=
\begin{pmatrix}
x&y
\end{pmatrix}
\cdot
\begin{pmatrix}
k_1&0
\\
0&k_2
\end{pmatrix}
\cdot
\begin{pmatrix}
x\\y
\end{pmatrix}
=k_1\cdot x^2+k_2\cdot y^2.\]

\section{Analysis}\label{sec:analysis}

The following material is discussed in any course of real analysis, the classical book of Walter Rudin \cite{rudin} is one of our favorites.

%??? measure theory+Lebesgue integration

\subsection*{Lipschitz condition}

Recall that a function $f$ between metric spaces is called \index{Lipschitz function}\emph{Lipschitz} if there is a constant $L$ such that 
\[|f(x)-f(y)|\le L\cdot|x-y|\]
for all values $x$ and $y$ in the domain of definition of $f$.
%Although this definition makes sense for maps between metric spaces, we will only use it for real-to-real functions.

The following theorem makes possible to extend number of results about smooth function to Lipschitz functions.
Recall that {}\emph{almost all} means all values, with the possible exceptions in a set of zero {}\emph{Lebesgue measure}.

\begin{thm}{Rademacher's theorem}\label{thm:rademacher}
Let $f\:[a,b]\to\RR$ be a Lipschitz function.
Then the derivative $f'$ of $f$ is a bounded measurable function defined almost everywhere in $[a,b]$ and it satisfies the fundamental theorem of calculus; that is, the following identity 
\[f(b)-f(a)=\int_a^b f'(x)\cdot dx,\]
holds if the integral is understood in the sense of Lebesgue.
\end{thm}

The following theorem makes possible to extend many statements about continuous function to measurable functions.

\begin{thm}{Lusin's theorem}\label{thm:lusin}
Let $\phi\:[a,b]\to \RR$ be a measurable function.
Then for any $\eps>0$, there is a continuous function $\psi_\eps\:[a,b]\to \RR$ that coincides with $\phi$ outside of a set of measure at most $\eps$.
Moreover, if $\phi$ is bounded above and/or below by some constants, then we may assume that so is $\psi_\eps$.  
\end{thm}

\subsection*{Uniform continuity and convergence}

Let $f\:{\spc{X}}\to \spc{Y}$ be a map between metric spaces.
If  for any $\eps>0$ there is $\delta>0$ such that 
\[|x_1-x_2|_{\spc{X}}<\delta\quad\Longrightarrow\quad |f(x_1)-f(x_2)|_\spc{Y}<\eps,\]
then $f$ is called \index{uniformly continuous}\emph{uniformly continuous}.

Evidently every uniformly continuous function is continuous;
the converse does not hold.
For example, the function $f(x)=x^2$ is continuous, but not uniformly continuous.
However the following statement holds true:

\begin{thm}{Heine--Cantor theorem}
Any continuous function defined on a compact metric space is uniformly continuous.
\end{thm}

If the condition above holds for any function $f_n$ in a sequence and $\delta$ depend solely on $\eps$,
then the sequence $(f_n)$ is called \index{uniformly continuous}\emph{uniformly equicontinuous}.
More precisely, 
a sequence of functions $f_n:{\spc{X}}\to \spc{Y}$ is called {}\emph{uniformly equicontinuous} if 
for any $\eps>0$ there is $\delta>0$ such that 
\[|x_1-x_2|_{\spc{X}}<\delta\quad\Longrightarrow\quad |f_n(x_1)-f_n(x_2)|_\spc{Y}<\eps\]
for any $n$.


We say that a sequence of functions $f_i : {\spc{X}} \to \spc{Y}$ \index{uniform convergence}\emph{converges uniformly} to a function $f_{\infty}: {\spc{X}} \to \spc{Y}$ if for any 
$\varepsilon >0$, there is a natural number $N$ such that for all $n \geq N$, we have $| f_{\infty} (x)- f_n (x) | < \varepsilon$
for all $x  \in {\spc{X}}$.

\begin{thm}{Arzel\'{a}--Ascoli Theorem}\label{lem:equicontinuous}
Suppose $\spc{X}$ and $\spc{Y}$ are compact metric spaces. 
Then any uniformly equicontinuous sequence of function $f_n\:\spc{X}\z\to \spc{Y}$ has a subsequence that converges uniformly to a continuous function $f_\infty\:\spc{X}\to \spc{Y}$. 
\end{thm}

\subsection*{Cutoffs and mollifiers}

Here we construct examples of smooth functions that mimic behavior of some model functions.
These functions are used to smooth model objects keeping its shape nearly unchanged.

For example, consider the following functions
\begin{align*}
h(t)&=
\begin{cases}
0&\text{if}\ t\le 0,
\\
t&\text{if}\ t> 0.
\end{cases}
&
f(t)&=
\begin{cases}
0&\text{if}\ t\le 0,
\\
\frac{t}{e^{1\!/\!t}}&\text{if}\ t> 0.
\end{cases}
\end{align*}
\begin{figure}[h]
\begin{minipage}{.48\textwidth}
\centering
\includegraphics{mppics/pic-320}
\end{minipage}\hfill
\begin{minipage}{.48\textwidth}
\centering
\includegraphics{mppics/pic-321}
\end{minipage}
\end{figure}
Note that $h$ and $f$ behave alike ---
both vanish at $t\le 0$ and grows to infinity for positive $t$.
The function $h$ is not smooth --- its derivative at $0$ is undefined.
Unlike $h$, the function $f$ is smooth.
Indeed, the existence of all derivatives $f^{(n)}(x)$ at $x\ne 0$ is evident and direct calculations show that $f^{(n)}(0)=0$ for all $n$.

Other useful examples of that type are the so-called \index{bell function}\emph{bell function} --- a smooth function that is positive in an $\eps$-neighborhood of zero and vanishing outside this neighborhood.
An example of such function can be constructed based using the function $f$ constructed above, say 
\[b_\eps(t)=c\cdot f(\eps^2-t^2);\]
typically one choose the constant $c$ so that $\int b_\eps=1$.

\begin{figure}[h!]
\begin{minipage}{.48\textwidth}
\centering
\includegraphics{mppics/pic-325}
\end{minipage}\hfill
\begin{minipage}{.48\textwidth}
\centering
\includegraphics{mppics/pic-326}
\end{minipage}
\end{figure}

Another useful example is a sigmoid --- nondecreasing function that vanish for $t\le -\eps$ and takes value $1$ for any $t\ge \eps$.
For example the following function \label{page:sigma-function}
\[\sigma_\eps(t)
=
\int_{-\infty}^t b_\eps(x)\cdot dx.\]

%???Borel sets


\section{Multivariable calculus}

The following material is discussed in any course of multivariable calculus, the classical book of Walter Rudin \cite{rudin} is one of our favorites.

\subsection*{Regular value}

A map $\bm{f}\:\RR^m\to\RR^n$ can be thought as an array of functions 
\[f_1,\dots,f_n\:\RR^m\to \RR.\]
The map $\bm{f}$ is called \index{smooth map}\emph{smooth} if each function $f_i$ is smooth;
that is, all partial derivatives of $f_i$ are defined in the domain of definition of $\bm{f}$.

The Jacobian matrix of $\bm{f}$ at $\bm{x}\in\RR^m$ is defined as
\[\Jac_{\bm{x}}\bm{f}=
\begin{pmatrix}
\dfrac{\partial f_1}{\partial x_1} & \cdots & \dfrac{\partial f_1}{\partial x_m}\\
\vdots & \ddots & \vdots\\
\dfrac{\partial f_n}{\partial x_1} & \cdots & \dfrac{\partial f_n}{\partial x_m} \end{pmatrix};\]\index{$\Jac_{\bm{x}}\bm{f}$}
we assume that the right hand side is evaluated at $\bm{x}=(x_1,\dots,x_m)$.

If the Jacobean matrix defines a surjective linear map $\RR^m\to\RR^n$ (that is, if $\rank(\Jac_{\bm{x}}\bm{f})=n$) then we say that 
$\bm{x}$ is a \index{regular point}\emph{regular point} of~$\bm{f}$.

If for some $\bm{y}\in \RR^n$ each point $\bm{x}$ such that $\bm{f}(\bm{x})=\bm{y}$ is regular,
then we say that $\bm{y}$ is a \index{regular value}\emph{regular value} of $\bm{f}$.
The following lemma states that {}\emph{most} values of a smooth map are regular.

\begin{thm}{Sard's lemma}\label{lem:sard}
Given a smooth map $\bm{f}\colon \Omega \z\to \RR^n$ defined on an open set $\Omega\subset \RR^m$, almost all values in $\RR^n$ are regular.
\end{thm}

The words \index{almost all}\emph{almost all} means all values, with the possible exceptions belong to a set with vanishing {}\emph{Lebesgue measure}.
In particular if one chooses a random value equidistributed in an arbitrarily small ball $B\subset \RR^n$, then it is a regular value of $\bm{f}$ with probability 1.

Note that if $m<n$, then any point $\bm{y}=\bm{f}(\bm{x})$ is not a regular value of $\bm{f}$.
Therefore the only regular value of $\bm{f}$ are the points in the complement of the image $\Im \bm{f}$.
In this case, the theorem states that almost all points in $\RR^n$, do {}\emph{not} belong to $\Im \bm{f}$.


\subsection*{Inverse function theorem}

The \index{inverse function theorem}\emph{inverse function theorem} gives a sufficient condition for a smooth map $\bm{f}$ to be invertible in a neighborhood of a given point $\bm{x}$.
The condition is formulated in terms of the Jacobian matrix of $\bm{f}$ at $\bm{x}$.

The \index{implicit function theorem}\emph{implicit function theorem} is a close relative to the inverse function theorem;
in fact it can be obtained as its corollary.
It is used when we need to pass from parametric to implicit description of curves and surfaces.

Both theorems reduce the existence of a map satisfying certain equation to a question in linear algebra.
We use these two theorems only for $n\le 3$.

\begin{thm}{Inverse function theorem}\label{thm:inverse}
Let $\bm{f}=(f_1,\dots,f_n)\:\Omega\to\RR^n$ be a smooth map
defined on an open set $\Omega\subset \RR^n$.
Assume that the Jacobian matrix
$\Jac_{\bm{x}}\bm{f}$
is invertible at some point $\bm{x}\in \Omega$.
Then there is a smooth map $\bm{h}\:\Phi\to\RR^n$ defined in an open neighborhood $\Phi$ of ${\bm{y}}\z=\bm{f}(\bm{x})$ that is a {}\emph{local inverse of $\bm{f}$ at $\bm{x}$};
that is, there is a neighborhood $\Psi\ni \bm{x}$ such that
$\bm{f}$ defines a bijection $\Psi\leftrightarrow \Phi$ and
$\bm{h} \circ \bm{f}$ is an identity map on $\Psi$.

Moreover if an $\Omega$ contains an $\eps$-neighborhood of $\bm{x}$, and the first and second partial derivatives $\tfrac{\partial f_i}{\partial x_j}$, $\tfrac{\partial^2 f_i}{\partial x_j\partial x_k}$ are bounded by a constant $C$ for all $i$, $j$, and $k$, then we can assume that $\Phi$ is a $\delta$-neighborhood of $\bm{y}$, for some $\delta>0$ that depends only on $\eps$ and $C$. 
\end{thm}

\begin{thm}{Implicit function theorem}\label{thm:imlicit}
Let $\bm{f}=(f_1,\dots,f_n)\:\Omega\to\RR^n$ be a smooth map, defined on a open subset $\Omega\subset\RR^{n+m}$, where
$m,n\ge 1$.
Let us consider $\RR^{n+m}$ as a product space $\RR^n\times \RR^m$ with coodinates 
$x_1,\dots,x_n,y_1,\dots,y_m$.
Consider the following matrix 
\[
M=\begin{pmatrix}
\dfrac{\partial f_1}{\partial x_1} & \cdots & \dfrac{\partial f_1}{\partial x_n}\\
\vdots & \ddots & \vdots\\
\dfrac{\partial f_n}{\partial x_1} & \cdots & \dfrac{\partial f_n}{\partial x_n} \end{pmatrix}\]
formed by the first $n$ columns of the Jacobian matrix.
Assume $M$ is invertible at some point $\bm{x}=(x_1,\dots,x_n,y_1,\dots y_m)$ in the domain of definition of $\bm{f}$ and $\bm{f}(\bm{x})=0$.
Then there is a neighborhood $\Psi\ni \bm{x}$
and a smooth function $\bm{h}\:\RR^m\to\RR^n$ defined in a neighborhood $\Phi\ni 0$ such that
for any $(x_1,\dots,x_n,y_1,\dots y_m)\in \Omega$, the equality
\[\bm{f}(x_1,\dots,x_n,y_1,\dots y_m)=0\]
holds if and only if 
\[(x_1,\dots x_n)=\bm{h}(y_1,\dots y_m).\]

\end{thm}

\subsection*{Multiple integral}

Let $\bm{f}\:\RR^n\to\RR^n$ is a smooth map (maybe partially defined).

\[\jac_{\bm{x}}\bm{f}\df|\det[\Jac_{\bm{x}}\bm{f}]|;\index{$\jac_{\bm{x}}\bm{f}$}\]
that is, $\jac_{\bm{x}}\bm{f}$ is the absolute value of the determinant of the Jacobian matrix of $\bm{f}$ at $\bm{x}$.

The following theorem plays the role of a substitution rule for multiple variables.

\begin{thm}{Theorem}\label{thm:mult-substitution} %???do we need measure here???
Let $h\:K\to\RR$ be a bounded measurable function on a measurable subset $K\subset \RR^n$.
Assume $\bm{f}\:K\to \RR^n$ is an injective smooth map.
Then 
\[\idotsint_{\bm{x}\in K} h(\bm{x})\cdot \jac_{\bm{x}}\bm{f}
=
\idotsint_{\bm{y}\in \bm{f}(K)} h\circ \bm{f}^{-1}(\bm{y}).\]

\end{thm}

\subsection*{Convex functions}

The following statements will be used only for $n\le 3$.

Let $f\:\RR^n\to \RR$ be a smooth function (maybe partially defined).
Choose a vector $\vec w\in \RR^n$.
Given a point $p\in\RR^n$ consider the function $\phi(t)=f(p+t\cdot \vec w)$.
Then the \index{directional derivative}\emph{directional derivative} $(D_{\vec w}f)(p)$ of $f$ at $p$ with respect to vector $\vec w$ is defined by
\[(D_{\vec w}f)(p)=\phi'(0).\]

Recall that a function $f$ is called \index{convex function}\emph{convex} if 
its epigraph $z\ge f(\bm{x})$ is a convex set in $\RR^n\times \RR$.

\begin{thm}{Theorem}
A smooth function $f\:K\to \RR$ defined on a convex subset $K\subset\RR^n$ is convex if and only if one of the following equivalent condition holds:

\begin{subthm}{}
The second directional derivative of $f$ at any point in the direction of any vector is nonnegative; that is,
\[(D_{\vec w}^2f)(p)\ge 0\]
for any $p\in K$ and $\vec w\in\RR^n$.
\end{subthm}

\begin{subthm}{}
The so-called \index{Jensen's inequality}\emph{Jensen's inequality}
\[f \left ((1-t)\cdot x_0 + t\cdot x_1 \right ) \le (1-t)\cdot f(x_0)+ t\cdot f(x_1)\]
holds for any $x_0,x_1\in K$ and $t\in[0,1]$.

\end{subthm}

\begin{subthm}{}
For any $x_0,x_1\in K$, we have 
\[f \left (\frac{x_0 + x_1}2 \right ) \le \frac{f(x_0) + f(x_1)}2.\]
\end{subthm}

\end{thm}




\section{Ordinary differential equations}

The following material is discussed at the very beginning of any course of ordinary differential equations; the classical book of Vladimir Arnold \cite{arnold} is one of our favorites.


\subsection*{Systems of first order}

The following theorem guarantees existence and uniqueness of solutions of an initial value problem
for a system of ordinary first order differential equations
\[
\begin{cases}
x_1'&=f_1(x_1,\dots,x_n,t),
\\
&\,\,\vdots
\\
x_n'&=f_n(x_1,\dots,x_n,t),
\end{cases}
\]
where each $t\mapsto x_i=x_i(t)$ is a real valued function defined on a real interval $\mathbb{I}$
and each $f_i$ is a smooth function defined on an open subset $\Omega\subset \RR^n\times \RR$.

The array of functions $(f_1,\dots,f_n)$ can be packed into one vector-valued function 
$\bm{f}\:\Omega\to \RR^n$;
the same way the array $(x_1,\dots,x_n)$ can be packed into a vector  $\bm{x}\in\RR^n$.
Therefore the system can be rewritten as one vector equation 
\[\bm{x}'=\bm{f}(\bm{x}, t).\] 

\begin{thm}{Theorem}\label{thm:ODE}
Suppose $\mathbb{I}$ is a real interval and $\bm{f}\:\Omega\to \RR^n$ is a smooth function defined on an open subset $\Omega\subset \RR^n\times \RR$.
Then for any initial data $\bm{x}(t_0)=\bm{u}$ such that $(\bm{u},t)\in\Omega$ the differential equation 
\[\bm{x}'=\bm{f}(\bm{x},t)\]
has a unique solution $t\mapsto \bm{x}(t)$ defined at a maximal interval $\mathbb{J}$ that contains $t_0$.
Moreover
\begin{enumerate}[(a)]
\item  if $\mathbb{J}\ne \RR$ (that is, if an end $a$ of $\mathbb{J}$ is finite) then $\bm{x}(t)$ does not have a limit point in $\Omega$ as $t\to a$;
\item  the function $(\bm{u},t_0,t)\mapsto \bm{x}(t)$ has open domain of definition in $\Omega\times \RR$ and it is smooth in this domain.
\end{enumerate}

\end{thm}

\subsection*{Higher order}

Suppose we have an ordinary differential equation of order $k$
\[\bm{x}^{(k)}=\bm{f}(\bm{x},\dots,\bm{x}^{(k-1)},t),\]
where $\bm{x}=\bm{x}(t)$ is a function from a real inerval to $\RR^n$.

This equation can be rewritten as $k$ first order equations as follows with $k-1$ new variables $\bm{y}_i=\bm{x}^{(i)}$:
\[
\begin{cases}
\bm{x}'&=\bm{y}_1
\\
\bm{y}_1'&=\bm{y}_2
\\
&\,\,\vdots
\\
\bm{y}_{k-1}'(t)&=\bm{f}(\bm{x},\bm{y}_{1},\dots,\bm{y}_{k-1},t),
\end{cases}
\]

Using this trick one can reduce a higher order ordinary differential equation to a first order equation. 
In particular we get local existence and uniqueness for solutions of higher order equations as in Theorem \ref{thm:ODE}.



\section{Topology}\label{sec:topology}

The following material is covered in any introductory text to topology; 
one of our favorites is a textbook of Czes Kosniowski \cite{kosniowski}.

\subsection*{Compact sets}

A subset $K$ of a metric space is called \index{compact subset}\emph{compact} if any sequence of points $(x_n)$ in $K$ has a subsequence that converges to a point $x_\infty$ in $K$.

The following properties follow directly from the definition:

\begin{itemize}
\item A closed subset of a compact space is compact.
\item A continuous image of a compact space is compact.
\end{itemize}

\begin{thm}{Heine--Borel theorem}\label{thm:Heine--Borel}
A subset of Euclidean space is compact if and only if it is closed and bounded.
\end{thm}


\subsection*{Continuous inverse}

We sometimes use the following characterization of homeomorphisms between compact spaces.

\begin{thm}{Theorem}\label{thm:Hausdorff-compact}
A continuous bijection $f$ between compact metric spaces has a continuous inverse.

In other words, any continuous bijection between compact metric spaces
is a homeomorphism.
\end{thm}

\subsection*{Jordan's theorem}
\index{Jordan's theorem}

The first part of the following theorem was proved by Camille Jordan, the second part is due to Arthur Schoenflies:

\begin{thm}{Theorem}\label{thm:jordan}
The complement of any closed simple plane curve $\gamma$ has exactly two connected components. 

Moreover, there is a homeomorphism $h\:\RR^2\to \RR^2$ that maps the unit circle to $\gamma$.
In particular $\gamma$ bounds a topological disc.
\end{thm}

This theorem is known for its simple formulation and quite hard proof.
By now many proofs of this theorem are known.
For the first statement, a very short proof based on a somewhat developed technique is given by Patrick Doyle \cite{doyle},
among elementary proofs, one of our favorites is the proof given by Aleksei Filippov \cite{filippov}.

We use mostly the smooth case of this theorem which is much simpler.
An amusing proof of this case was given by Gregory Chambers and Yevgeny Liokumovich \cite{chambers-liokumovich}.

\subsection*{Connectedness}

Recall that a continuous map $\alpha$ from the unit interval $[0,1]$ to a Euclidean space is called a \index{path}\emph{path}.
If $p=\alpha (0)$ and $q = \alpha (1)$, then we say that $\alpha$ connects $p$ to $q$.


A set $X$ in the Euclidean space is called \index{path connected set}\emph{path connected} if any two points $x,y\in X$ can be connected by a path lying in $X$.

A set $X$ in the Euclidean space is called \index{connected set}\emph{connected} if one cannot cover $X$ with two disjoint open sets $V$ and $W$ such that both intersections $X\cap V$ and $X\cap W$ are nonempty.

\begin{thm}{Proposition}
Any path connected set is connected.

Moreover, any open connected set in the Euclidean space or  plane is path connected.
\end{thm}

Given a point $x\in X$, the maximal connected subset of $X$ containing $x$ is called the \index{connected component}\emph{connected component} of $x$ in $X$.
