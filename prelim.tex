\addtocounter{chapter}{-1}
\chapter{Preliminaries}

This chapter should be used as a quick reference while reading the rest of the book;
it also contains all necessary references with complete proof.

The first section on metric spaces is an exception;
we suggest to read in before going further.

\section{Metric spaces}\label{sec:metric-spcaes}

We assume that the reader is familiar with the notion of distance in the 
Euclidean space.
In this chapter we briefly discuss its generalization and fix notations that will be used further.

The introductory part of the book by Dmitri Burago, Yuri Burago, and Sergei Ivanov \cite{burago-burago-ivanov} contains all the needed material.

\subsection*{Definitions}

\emph{Metric} is a function that returns a real value $\Dist(x,y)$ for any pair $x,y$ in a given nonempty set $\spc{X}$  and satisfies the following axioms for any triple $x,y,z$: \label{page:def:metric}
\begin{enumerate}[(a)]
\item\label{def:metric-space:a} Positiveness: 
$$\Dist(x,y)\ge 0.$$
\item\label{def:metric-space:b} $x=y$ if and only if 
$$\Dist(x,y)=0.$$
\item\label{def:metric-space:c} Symmetry: $$\Dist(x, y) = \Dist(y, x).$$
\item\label{def:metric-space:d} Triangle inequality: 
$$\Dist(x, z) \le \Dist(x, y) + \Dist(y, z).$$
\end{enumerate}

A set with a metric is called \index{metric space}\emph{metric space} and the elements of the set are called \index{point}\emph{points}.

\subsection*{Shortcut for distance}
Usually we consider only one metric on a set, therefore we can denote the metric space and its underlying set by the same letter, say $\spc{X}$.
In this case we also use the shortcut notations $\dist{x}{y}{}$ or $\dist{x}{y}{\spc{X}}$  for the {}\emph{distance $\Dist(x,y)$ from $x$ to $y$ in $\spc{X}$}.\index{10aaa@$\dist{x}{y}{}$, $\dist{x}{y}{\spc{X}}$}
For example, the triangle inequality can be written as 
$$\dist{x}{z}{\spc{X}}\le \dist{x}{y}{\spc{X}}+\dist{y}{z}{\spc{X}}.$$

The Euclidean space and plane as well as the real line will be the most important examples of metric spaces for us.
In these examples the introduced notation $\dist{x}{y}{}$ for the distance from $x$ to $y$ has perfect sense as the norm of the vector $x-y$.
However, in a general metric space the expression $x-y$ has no meaning, but we use expression $\dist{x}{y}{}$ for the distance anyway.

\subsection*{More examples}

Usually, if we say {}\emph{plane} or {}\emph{space} we mean the {}\emph{Euclidean} plane or space.
However the plane (as well as the space) admits many other metrics, for example the so-called {}\emph{Manhattan metric} from the following exercise.

\begin{thm}{Exercise}\label{ex:ell-infty}
Consider the function
$$\Dist(p,q)=|x_1-x_2|+|y_1-y_2|,$$
where $p=(x_1,y_1)$ and $q=(x_2,y_2)$ are points in the coordinate plane $\mathbb{R}^2$.
Show that $\Dist$ is a metric on $\mathbb{R}^2$.
\end{thm}

Another example: the {}\emph{discrete space} --- an arbitrary nonempty set $\spc{X}$ with the metric defined as $\dist{x}{y}{\spc{X}}=0$ if $x=y$ and $\dist{x}{y}{\spc{X}}=1$ otherwise.

\subsection*{Subspaces}
Any subset of a metric space is also a metric space, by restricting the original metric to the subset;
the obtained metric space is called a {}\emph{subspace}.
In particular, all subsets of the Euclidean space are metric spaces.

\subsection*{Balls}
Given a point $p$ in a metric space $\spc{X}$ and a real number $R\ge 0$, the set of points $x$ on the distance less then $R$ (at most $R$) from $p$ is called the \index{open!ball}\emph{open} (respectively \index{closed!ball}\emph{closed}) {}\emph{ball} of radius $R$ with center $p$.
The {}\emph{open ball} is denoted as $B(p,R)$ or $B(p,R)_{\spc{X}}$;
the second notation is used if we need to emphasize that the ball lies in the metric space $\spc{X}$.
Formally speaking
\[B(p,R)=B(p,R)_{\spc{X}}=\set{x\in \spc{X}}{\dist{x}{p}{\spc{X}}< R}.\]
\index{10b@$B(p,R)_{\spc{X}}$, $\bar B[p,R]_{\spc{X}}$}
Analogously, the {}\emph{closed ball} is denoted as $\bar B[p,R]$ or $\bar B[p,R]_{\spc{X}}$ and
\[\bar B[p,R]=\bar B[p,R]_{\spc{X}}=\set{x\in \spc{X}}{\dist{x}{p}{\spc{X}}\le R}.\]

\begin{thm}{Exercise}\label{ex:B2inB1}
Let $\spc{X}$ be a metric space.

\begin{subthm}{ex:B2inB1:a}
Show that if $\bar B[p,2]\subset \bar B[q,1]$ for some points $p,q\in \spc{X}$, then $\bar B[p,2]\z=\bar B[q,1]$.
\end{subthm}

\begin{subthm}{ex:B2inB1:b} Construct a metric space $\spc{X}$ with two points $p$ and $q$ such that the strict inclusion
$B(p,\tfrac32)\subset B(q,1)$ holds.
\end{subthm}

\end{thm}



\subsection*{Continuity}

\begin{thm}{Definition}
 Let ${\spc{X}}$ be a metric space.
A sequence of points $x_1, x_2, \ldots$ in ${\spc{X}}$ is called \index{convergence of points}\emph{convergent}
if there is 
$x_\infty\in {\spc{X}}$ such that $\dist{x_\infty}{x_n}{}\to 0$ as $n\to\infty$.  
That is, for every $\epsilon > 0$, there is a natural number $N$ such that for all $n \ge N$, we have
\[
\dist{x_\infty}{x_n}{\spc{X}}
<
\epsilon.
\]

In this case we say that the sequence $(x_n)$ {}\emph{converges} to $x_\infty$, 
or $x_\infty$ is the {}\emph{limit} of the sequence $(x_n)$.
Notationally, we write $x_n\to x_\infty$ as $n\to\infty$
or $x_\infty=\lim_{n\to\infty} x_n$.
\end{thm}

\begin{thm}{Definition}\label{def:continuous}
Let $\spc{X}$ and $\spc{Y}$ be metric spaces.
A map $f\:\spc{X}\to \spc{Y}$ is called \index{continuous}\emph{continuous} if for any convergent sequence $x_n\to x_\infty$ in ${\spc{X}}$,
%the sequence $y_n\z=f(x_n)$ converges to $y_\infty=f(x_\infty)$ in $\spc{Y}$.
we have $f(x_n) \to f(x_\infty)$ in $\spc{Y}$.

Equivalently, $f\:\spc{X}\to \spc{Y}$ is continuous if for any $x\in {\spc{X}}$ and any $\epsilon>0$,
there is $\delta>0$ such that 
$$\dist{x}{y}{\spc{X}}<\delta\quad \text{ implies that }\quad \dist{f(x)}{f(y)}{\spc{Y}}<\epsilon.$$

\end{thm}

\begin{thm}{Exercise}\label{ex:shrt=>continuous}
Let ${\spc{X}}$ and $\spc{Y}$ be metric spaces $f\:\spc{X}\to \spc{Y}$ is {}\emph{distance non-expanding map}; that is, 
\[\dist{f(x)}{f(y)}{\spc{Y}}\le \dist{x}{y}{\spc{X}}\]
for any $x,y\in \spc{X}$.
Show that $f$ is continuous.
\end{thm}

\subsection*{Homeomorphisms}

\begin{thm}{Definition}
Let $\spc{X}$ and $\spc{Y}$ be metric spaces.
A continuous bijection $f\:\spc{X}\to \spc{Y}$ 
is called a \index{homeomorphism}\emph{homeomorphism} 
if its inverse $f^{-1}\:\spc{Y}\z\to \spc{X}$ is also continuous.

If there exists a homeomorphism $f\:\spc{X}\to \spc{Y}$,
we say that ${\spc{X}}$ is {}\emph{homeomorphic} to $\spc{Y}$,
or  $\spc{X}$ and $\spc{Y}$ are {}\emph{homeomorphic}.
\end{thm}

If a metric space $\spc{X}$ is homeomorphic to a known space, for example plane, sphere, disc, circle and so on,
we may also say that $\spc{X}$ is a \index{topological}\emph{topological} plane, sphere, disc, circle and so on.

\subsection*{Closed and open sets}

\begin{thm}{Definition}
A subset $C$ of a metric space $\spc{X}$ is called \index{closed!set}\emph{closed} if whenever a sequence $(x_n)$ of points from $C$ converges in $\spc{X}$, we have that $\lim_{n\to\infty} x_n \in C$.

A set $\Omega \subset \spc{X}$ is called \index{open!set}\emph{open} if for any $z\in \Omega$, 
there is $\epsilon>0$ such that $B(z,\epsilon)\subset\Omega$.
\end{thm}

\begin{thm}{Exercise}\label{ex:close-open}
Let $Q$ be a subset of a metric space $\spc{X}$.
Show that $Q$ is closed if and only if its complement $\Omega=\spc{X}\backslash Q$ is open.
\end{thm}

An open set $\Omega$ that contains a given point $p$ is called a \index{neighborhood}\emph{neighborhood of~$p$}.
A closed subset $C$ that contains $p$ together with its neighborhood is called a {}\emph{closed neighborhood of~$p$}

\section{Elementary geometry}

\subsection*{Internal angles}

Polygon is defined as a compact set bounded by a closed polygonal line. 
Recall that the internal angle of a polygon $P$ at a vertex $v$
is defined as an angular measure of the intersection of $P$ with a small circle centered at $v$.

\begin{thm}{Theorem}\label{thm:sum=(n-2)pi}
The sum of all the internal angles of a simple $n$-gon is $(n-2)\cdot\pi$. 
\end{thm}

While this theorem is well known, it is not easy to find a reference with a proof without cheating.
A clean proof was given by Gary Meisters \cite{meisters}.
It uses induction on $n$ and is based on the following:

\begin{thm}{Claim}
Suppose $P$ is an $n$-gon with $n\ge 4$.
Then a diagonal of $P$ lies completely in $P$.
\end{thm}



\subsection*{Angle monotonicity}

The {}\emph{measure} of angle with sides $[p,x]$ and $[p,y]$ will be denoted by $\measuredangle\hinge pxy$\index{10aab@$\measuredangle\hinge yxz$};
it takes a value in the interval $[0,\pi]$.

The following lemma is very simple and very useful.
It says that the angle of a triangle monotonically depends on the opposite side, assuming we keep the other two sides fixed.
It follows directly from the cosine rule.

\begin{thm}{Monotonicity lemma}\label{lem:angle-monotonicity}
Let $x$, $y$, $z$, $x^{*}$, $y^{*}$ and $z^{*}$ be 6 points such that $\dist{x}{y}{}\z=|x^{*}-y^{*}|>0$ and $|y-z|=|y^{*}-z^{*}|>0$.
Then 
\[\measuredangle\hinge yxz
\ge
\measuredangle\hinge {y^{*}}{x^{*}}{z^{*}}
\quad\text{if and only if}\quad
|x-z|\ge |x^{*}-z^{*}|.\]
\end{thm}

%???+inversion

\subsection*{Spherical triangle inequality}

The following theorem says that the triangle inequality holds for angles between half-lines from a fixed point.
In particular it implies that a sphere with the angle metric is a metric space.

\begin{thm}{Theorem}\label{thm:spherical-triangle-inq}
The following inequality holds for any three line segments $[o,a]$, $[o,b]$ and $[o,c]$ in the Euclidean space:
\[\measuredangle\hinge oab
+
\measuredangle\hinge obc
\ge
\measuredangle\hinge oac\]

\end{thm}

Most of authors use this theorem without mentioning, but the proof is not that simple.
A short elementary proof can be found in the classical textbook in Euclidean geometry by Andrey Kiselyov \cite[\S 47]{kiselyov}.


\subsection*{Area of spherical triangle}

\begin{thm}{Lemma}\label{lem:area-spher-triangle}
Let $\Delta$ be a spherical triangle;
that is, $\Delta$ is the intersection of three closed half-spheres in the unit sphere $\mathbb{S}^2$.
Then 
\[\area\Delta=\alpha+\beta+\gamma-\pi,\eqlbl{eq:area(Delta)}\]
where $\alpha$, $\beta$ and $\gamma$ are the angles of $\Delta$.
\end{thm}

The value $\alpha+\beta+\gamma-\pi$ is called \index{excess of triangle}\emph{excess of the triangle} $\Delta$,
so the lemma says that the area of a spherical triangle equals its excess.

This lemma appears in many texts.
We give its proof here since it is very important in our intuitive proof of Gauss--Bonnet formula.

\begin{wrapfigure}{r}{22 mm}
\vskip-0mm
\centering
\includegraphics{mppics/pic-43}
\vskip2mm
\end{wrapfigure}

\parit{Proof.}
Recall that 
\[\area\mathbb{S}^2=4\cdot\pi.\eqlbl{eq:area(S2)}\]

Note that the area of a spherical slice $S_\alpha$ between two meridians meeting at angle $\alpha$ is proportional to $\alpha$.
Since for $S_\pi$ is a half-sphere, from \ref{eq:area(S2)}, we get $\area S_\pi\z=2\cdot\pi$.
Therefore the coefficient is 2; that is,
\[\area S_\alpha=2\cdot \alpha.\eqlbl{eq:area(Sa)}\]

Extending the sides of $\Delta$ we get 6 slices: two $S_\alpha$, two $S_\beta$ and two $S_\gamma$ which cover most of the sphere once,
but the triangle $\Delta$ and its centrally symmetric copy $\Delta^{*}$ are covered 3 times.
It follows that
\[2\cdot \area S_\alpha+2\cdot \area S_\beta+2\cdot \area S_\gamma
=\area\mathbb{S}^2+4\cdot\area\Delta.\]
Substituting \ref{eq:area(S2)} and \ref{eq:area(Sa)} and simplifying, we get \ref{eq:area(Delta)}.
\qeds





\section{Convex geometry}

A set $X$ in the Euclidean space is called \index{convex!set}\emph{convex} if for any two points $x,y\in X$, any point $z$ between $x$ and $y$ lies in $X$.
It is called  {}\emph{strictly convex} if for any two points $x,y\in X$, any point $z$ between $x$ and $y$ lies in the interior of $X$.

From the definition, it is easy to see that the intersection of an arbitrary family of convex sets is convex. 
The intersection of all convex sets containing $X$ is called the \index{convex!hull}\emph{convex hull} of $X$;
it is the minimal convex set containing the set $X$.

We will use the following corollary of the so-called \index{hyperplane separation theorem}\emph{hyperplane separation theorem}:

\begin{thm}{Lemma}\label{lem:separation}
Let $K\subset \mathbb{R}^3$ be a closed convex set.
Then for any point $p\notin K$ there is a plane $\Pi$ that separates $K$ from $p$;
that is, $K$ and $p$ lie on opposite open half-spaces separated by $\Pi$.
\end{thm}

These definitions and hyperplane separation should appear on first few pages of any introductory text in convex geometry;
see for example the book of Roger Webster \cite{webster}.

\section{Linear algebra}

The following theorem can be found in any textbook in linear algebra;
the book of Sergei Treil \cite{treil} will do.

\begin{thm}{Spectral theorem}\label{thm:spectral}
Any symmetric matrix is diagonalizable  by an orthogonal matrix.
\end{thm}

We will use this theorem only for $2{\times}2$ matrices.
In this case it can be restated as follows:
Consider a function 
\[f(x,y)=
\begin{pmatrix}
x&y
\end{pmatrix}
\cdot
\begin{pmatrix}
\ell&m
\\
m&n
\end{pmatrix}
\cdot
\begin{pmatrix}
x\\y
\end{pmatrix}
=\ell\cdot x^2+2\cdot m\cdot x\cdot y+n\cdot y^2,\]
that is defined on a $(x,y)$-coordinate plane.
Then after proper rotation of the coordinates, 
the expression for $f$ in the new coordinates will be
\[\bar f(x,y)=
\begin{pmatrix}
x&y
\end{pmatrix}
\cdot
\begin{pmatrix}
k_1&0
\\
0&k_2
\end{pmatrix}
\cdot
\begin{pmatrix}
x\\y
\end{pmatrix}
=k_1\cdot x^2+k_2\cdot y^2.\]

\section{Analysis}\label{sec:analysis}

The following material is discussed in any course of real analysis, the classical book of Walter Rudin \cite{rudin} is one of our favorites.

%??? measure theory+Lebesgue integration

\subsection*{Lipschitz condition}

Recall that a function $f$ between metric spaces is called \index{Lipschitz function}\emph{Lipschitz} if there is a constant $L$ such that 
\[\dist{f(x)}{f(y)}{}\le L\cdot\dist{x}{y}{}\]
for all values $x$ and $y$ in the domain of definition of $f$.

The following theorem makes it possible to extend a number of results about smooth functions to Lipschitz functions.
Recall that {}\emph{almost all} means all values, with the possible exceptions in a set of zero {}\emph{Lebesgue measure}.

\begin{thm}{Rademacher's theorem}\label{thm:rademacher}
Let $f\:[a,b]\to\mathbb{R}$ be a Lipschitz function.
Then the derivative $f'$ of $f$ is a bounded measurable function defined almost everywhere in $[a,b]$ and it satisfies the fundamental theorem of calculus; that is, the following identity 
\[f(b)-f(a)=\int_a^b f'(x)\cdot dx,\]
holds if the integral is understood in the sense of Lebesgue.
\end{thm}

The following theorem makes it possible to extend many statements about continuous function to measurable functions.

\begin{thm}{Lusin's theorem}\label{thm:lusin}
Let $\phi\:[a,b]\to \mathbb{R}$ be a measurable function.
Then for any $\epsilon>0$, there is a continuous function $\psi_\epsilon\:[a,b]\to \mathbb{R}$ that coincides with $\phi$ outside of a set of measure at most $\epsilon$.
Moreover, if $\phi$ is bounded above and/or below by some constants, then we may assume that so is $\psi_\epsilon$.  
\end{thm}

\subsection*{Uniform continuity and convergence}

Let $f\:{\spc{X}}\to \spc{Y}$ be a map between metric spaces.
If  for any $\epsilon>0$ there is $\delta>0$ such that 
\[\dist{x_1}{x_2}{\spc{X}}<\delta\quad\Longrightarrow\quad \dist{f(x_1)}{f(x_2)}{\spc{Y}}<\epsilon,\]
then $f$ is called \index{uniformly continuous}\emph{uniformly continuous}.

Evidently every uniformly continuous function is continuous;
the converse does not hold.
For example, the function $f(x)=x^2$ is continuous, but not uniformly continuous.
However the following statement holds true:

\begin{thm}{Heine--Cantor theorem}
Any continuous function defined on a compact metric space is uniformly continuous.
\end{thm}

If the condition above holds for any function $f_n$ in a sequence and $\delta$ depends solely on $\epsilon$,
then the sequence $(f_n)$ is called \index{uniformly continuous}\emph{uniformly equicontinuous}.
More precisely, 
a sequence of functions $f_n:{\spc{X}}\to \spc{Y}$ is called {}\emph{uniformly equicontinuous} if 
for any $\epsilon>0$ there is $\delta>0$ such that 
\[\dist{x_1}{x_2}{\spc{X}}<\delta\quad\Longrightarrow\quad \dist{f_n(x_1)}{f_n(x_2)}{\spc{Y}}<\epsilon\]
for any $n$.


We say that a sequence of functions $f_i\: {\spc{X}} \to \spc{Y}$ \index{uniform convergence}\emph{converges uniformly} to a function $f_{\infty}\: {\spc{X}} \to \spc{Y}$ if for any 
$\epsilon >0$, there is a natural number $N$ such that for all $n \ge N$, we have $\dist{f_{\infty}(x)}{f_n (x)}{}<\epsilon$
for all $x  \in {\spc{X}}$.

\begin{thm}{Arzel\'{a}--Ascoli theorem}\label{lem:equicontinuous}
Suppose $\spc{X}$ and $\spc{Y}$ are compact metric spaces. 
Then any uniformly equicontinuous sequence of function $f_n\:\spc{X}\z\to \spc{Y}$ has a subsequence that converges uniformly to a continuous function $f_\infty\:\spc{X}\to \spc{Y}$. 
\end{thm}

\subsection*{Cutoffs and mollifiers}

Here we construct examples of smooth functions that mimic behavior of some model functions.
These functions are used to smooth model objects keeping its shape nearly unchanged.

For example, consider the following functions
\begin{align*}
h(t)&=
\begin{cases}
0&\text{if}\ t\le 0,
\\
t&\text{if}\ t> 0.
\end{cases}
&
f(t)&=
\begin{cases}
0&\text{if}\ t\le 0,
\\
\frac{t}{e^{1\!/\!t}}&\text{if}\ t> 0.
\end{cases}
\end{align*}
\begin{figure}[h]
\begin{minipage}{.48\textwidth}
\centering
\includegraphics{mppics/pic-320}
\end{minipage}\hfill
\begin{minipage}{.48\textwidth}
\centering
\includegraphics{mppics/pic-321}
\end{minipage}
\end{figure}
Note that $h$ and $f$ behave alike ---
both vanish at $t\le 0$ and grow to infinity for positive $t$.
The function $h$ is not smooth --- its derivative at $0$ is undefined.
Unlike $h$, the function $f$ is smooth.
Indeed, the existence of all derivatives $f^{(n)}(x)$ at $x\ne 0$ is evident and direct calculations show that $f^{(n)}(0)=0$ for all $n$.

Other useful examples of that type are the so-called \index{bell function}\emph{bell function} --- a smooth function that is positive in an $\epsilon$-neighborhood of zero and vanishing outside this neighborhood.
An example of such function can be constructed based using the function $f$ constructed above, say 
\[b_\epsilon(t)=c\cdot f(\epsilon^2-t^2);\]
typically one chooses the constant $c$ so that $\int b_\epsilon=1$.

\begin{figure}[h!]
\begin{minipage}{.48\textwidth}
\centering
\includegraphics{mppics/pic-325}
\end{minipage}\hfill
\begin{minipage}{.48\textwidth}
\centering
\includegraphics{mppics/pic-326}
\end{minipage}
\end{figure}

Another useful example is a sigmoid --- nondecreasing function that vanishes for $t\le -\epsilon$ and takes value $1$ for any $t\ge \epsilon$.
For example the following function \label{page:sigma-function}
\[\sigma_\epsilon(t)
=
\int_{-\infty}^t b_\epsilon(x)\cdot dx.\]

%???Borel sets


\section{Multivariable calculus}

The following material is discussed in any course of multivariable calculus, the classical book of Walter Rudin \cite{rudin} is one of our favorites.

\subsection*{Regular value}

A map $\bm{f}\:\mathbb{R}^m\to\mathbb{R}^n$ can be thaut of as an array of functions 
\[f_1,\dots,f_n\:\mathbb{R}^m\to \mathbb{R}.\]
The map $\bm{f}$ is called \index{smooth!map}\emph{smooth} if each function $f_i$ is smooth;
that is, all partial derivatives of $f_i$ are defined in the domain of definition of $\bm{f}$.

The Jacobian matrix of $\bm{f}$ at $\bm{x}\in\mathbb{R}^m$ is defined as
\[\Jac_{\bm{x}}\bm{f}=
\begin{pmatrix}
\dfrac{\partial f_1}{\partial x_1} & \cdots & \dfrac{\partial f_1}{\partial x_m}\\
\vdots & \ddots & \vdots\\
\dfrac{\partial f_n}{\partial x_1} & \cdots & \dfrac{\partial f_n}{\partial x_m} \end{pmatrix};\]\index{10j@$\Jac$}
we assume that the right hand side is evaluated at $\bm{x}=(x_1,\dots,x_m)$.

If the Jacobian matrix defines a surjective linear map $\mathbb{R}^m\to\mathbb{R}^n$ (that is, if $\rank(\Jac_{\bm{x}}\bm{f})=n$) then we say that 
$\bm{x}$ is a \index{regular!point}\emph{regular point} of~$\bm{f}$.

If for some $\bm{y}\in \mathbb{R}^n$ each point $\bm{x}$ such that $\bm{f}(\bm{x})=\bm{y}$ is regular,
then we say that $\bm{y}$ is a \index{regular!value}\emph{regular value} of $\bm{f}$.
The following lemma states that {}\emph{most} values of a smooth map are regular.

\begin{thm}{Sard's lemma}\label{lem:sard}
Given a smooth map $\bm{f}\colon \Omega \z\to \mathbb{R}^n$ defined on an open set $\Omega\subset \mathbb{R}^m$, almost all values in $\mathbb{R}^n$ are regular.
\end{thm}

The words \index{almost all}\emph{almost all} mean all values, with the possible exceptions belonging to a set with vanishing {}\emph{Lebesgue measure}.
In particular if one chooses a random value equidistributed in an arbitrarily small ball $B\z\subset \mathbb{R}^n$, then it is a regular value of $\bm{f}$ with probability 1.

Note that if $m<n$, then any point $\bm{y}=\bm{f}(\bm{x})$ is not a regular value of $\bm{f}$.
Therefore the only regular value of $\bm{f}$ are the points in the complement of the image $\Im \bm{f}$.
In this case, the theorem states that almost all points in $\mathbb{R}^n$, do {}\emph{not} belong to $\Im \bm{f}$.


\subsection*{Inverse function theorem}

The \index{inverse function theorem}\emph{inverse function theorem} gives a sufficient condition for a smooth map $\bm{f}$ to be invertible in a neighborhood of a given point $\bm{x}$.
The condition is formulated in terms of the Jacobian matrix of $\bm{f}$ at $\bm{x}$.

The \index{implicit function theorem}\emph{implicit function theorem} is a close relative to the inverse function theorem;
in fact it can be obtained as its corollary.
It is used when we need to pass from parametric to implicit description of curves and surfaces.

Both theorems reduce the existence of a map satisfying a certain equation to a question in linear algebra.
We use these two theorems only for $n\le 3$.

\begin{thm}{Inverse function theorem}\label{thm:inverse}
Let $\bm{f}=(f_1,\dots,f_n)\:\Omega\to\mathbb{R}^n$ be a smooth map
defined on an open set $\Omega\subset \mathbb{R}^n$.
Assume that the Jacobian matrix
$\Jac_{\bm{x}}\bm{f}$
is invertible at some point $\bm{x}\in \Omega$.
Then there is a smooth map $\bm{h}\:\Phi\to\mathbb{R}^n$ defined in an open neighborhood $\Phi$ of ${\bm{y}}\z=\bm{f}(\bm{x})$ that is a {}\emph{local inverse of $\bm{f}$ at $\bm{x}$};
that is, there is a neighborhood $\Psi\ni \bm{x}$ such that
$\bm{f}$ defines a bijection $\Psi\leftrightarrow \Phi$ and
$\bm{h} \circ \bm{f}$ is an identity map on $\Psi$.

Moreover if an $\Omega$ contains an $\epsilon$-neighborhood of $\bm{x}$, and the first and second partial derivatives $\tfrac{\partial f_i}{\partial x_j}$, $\tfrac{\partial^2 f_i}{\partial x_j\partial x_k}$ are bounded by a constant $C$ for all $i$, $j$, and $k$, then we can assume that $\Phi$ is a $\delta$-neighborhood of $\bm{y}$, for some $\delta>0$ that depends only on $\epsilon$ and $C$. 
\end{thm}

\begin{thm}{Implicit function theorem}\label{thm:imlicit}
Let $\bm{f}=(f_1,\dots,f_n)\:\Omega\to\mathbb{R}^n$ be a smooth map, defined on a open subset $\Omega\subset\mathbb{R}^{n+m}$, where
$m,n\z\ge 1$.
Let us consider $\mathbb{R}^{n+m}$ as a product space $\mathbb{R}^n\times \mathbb{R}^m$ with coordinates 
$x_1,\dots,x_n,y_1,\dots,y_m$.
Consider the following matrix 
\[
M=\begin{pmatrix}
\dfrac{\partial f_1}{\partial x_1} & \cdots & \dfrac{\partial f_1}{\partial x_n}\\
\vdots & \ddots & \vdots\\
\dfrac{\partial f_n}{\partial x_1} & \cdots & \dfrac{\partial f_n}{\partial x_n} \end{pmatrix}\]
formed by the first $n$ columns of the Jacobian matrix.
Assume $M$ is invertible at some point $\bm{x}=(x_1,\dots,x_n,y_1,\dots y_m)$ in the domain of definition of $\bm{f}$ and $\bm{f}(\bm{x})=0$.
Then there is a neighborhood $\Psi\ni \bm{x}$
and a smooth function $\bm{h}\:\mathbb{R}^m\to\mathbb{R}^n$ defined in a neighborhood $\Phi\ni 0$ such that
for any $(x_1,\dots,x_n,y_1,\dots y_m)\in \Omega$, the equality
\[\bm{f}(x_1,\dots,x_n,y_1,\dots y_m)=0\]
holds if and only if 
\[(x_1,\dots x_n)=\bm{h}(y_1,\dots y_m).\]

\end{thm}

\subsection*{Multiple integral}

Let $\bm{f}\:\mathbb{R}^n\to\mathbb{R}^n$ is a smooth map (maybe partially defined).

\[\jac_{\bm{x}}\bm{f}\df|\det[\Jac_{\bm{x}}\bm{f}]|;\index{10j@$\jac$}\]
that is, $\jac_{\bm{x}}\bm{f}$ is the absolute value of the determinant of the Jacobian matrix of $\bm{f}$ at $\bm{x}$.

The following theorem plays the role of a substitution rule for multiple variables.

\index{Borel subsets}\emph{Borel subsets} are defined as the class of subsets that are generated from open sets by applying the following operations recursively: countable union, countable intersection, and complement.
Since complement of a closed set is open and the other way around these sets can be also generated from all closed sets.
This class of sets includes virtually all sets that naturally appear in geometry but does not include pathological examples that create problems with integration.



\begin{thm}{Theorem}\label{thm:mult-substitution} %???do we need measure here???
Let $h\:K\to\mathbb{R}$ be a continuous function on a Borel subset $K\subset \mathbb{R}^n$.
Assume $\bm{f}\:K\to \mathbb{R}^n$ is an injective smooth map.
Then 
\[\idotsint_{\bm{x}\in K} h(\bm{x})\cdot \jac_{\bm{x}}\bm{f}
=
\idotsint_{\bm{y}\in \bm{f}(K)} h\circ \bm{f}^{-1}(\bm{y}).\]

\end{thm}

\subsection*{Convex functions}

The following statements will be used only for $n\le 3$.

Let $f\:\mathbb{R}^n\to \mathbb{R}$ be a smooth function (maybe partially defined).
Choose a vector $\vec w\in \mathbb{R}^n$.
Given a point $p\in\mathbb{R}^n$ consider the function $\phi(t)=f(p+t\cdot \vec w)$.
Then the \index{directional derivative}\emph{directional derivative} $(D_{\vec w}f)(p)$ of $f$ at $p$ with respect to vector $\vec w$ is defined by
\[(D_{\vec w}f)(p)=\phi'(0).\]

Recall that a function $f$ is called \index{convex!function}\emph{convex} if 
its epigraph $z\ge f(\bm{x})$ is a convex set in $\mathbb{R}^n\times \mathbb{R}$.

\begin{thm}{Theorem}
A smooth function $f\:K\to \mathbb{R}$ defined on a convex subset $K\subset\mathbb{R}^n$ is convex if and only if one of the following equivalent condition holds:

\begin{subthm}{}
The second directional derivative of $f$ at any point in the direction of any vector is nonnegative; that is,
\[(D_{\vec w}^2f)(p)\ge 0\]
for any $p\in K$ and $\vec w\in\mathbb{R}^n$.
\end{subthm}

\begin{subthm}{}
The so-called \index{Jensen's inequality}\emph{Jensen's inequality}
\[f \left ((1-t)\cdot x_0 + t\cdot x_1 \right ) \le (1-t)\cdot f(x_0)+ t\cdot f(x_1)\]
holds for any $x_0,x_1\in K$ and $t\in[0,1]$.

\end{subthm}

\begin{subthm}{}
For any $x_0,x_1\in K$, we have 
\[f \left (\frac{x_0 + x_1}2 \right ) \le \frac{f(x_0) + f(x_1)}2.\]
\end{subthm}

\end{thm}




\section{Ordinary differential equations}

The following material is discussed at the very beginning of any course of ordinary differential equations; the classical book of Vladimir Arnold \cite{arnold} is one of our favorites.


\subsection*{Systems of first order}

The following theorem guarantees existence and uniqueness of solutions of an initial value problem
for a system of ordinary first order differential equations
\[
\begin{cases}
x_1'&=f_1(x_1,\dots,x_n,t),
\\
&\,\,\vdots
\\
x_n'&=f_n(x_1,\dots,x_n,t),
\end{cases}
\]
where each $t\mapsto x_i=x_i(t)$ is a real valued function defined on a real interval $\mathbb{I}$
and each $f_i$ is a smooth function defined on an open subset $\Omega\subset \mathbb{R}^n\times \mathbb{R}$.

The array of functions $(f_1,\dots,f_n)$ can be packed into one vector-valued function 
$\bm{f}\:\Omega\to \mathbb{R}^n$;
the same way the array $(x_1,\dots,x_n)$ can be packed into a vector  $\bm{x}\in\mathbb{R}^n$.
Therefore the system can be rewritten as one vector equation 
\[\bm{x}'=\bm{f}(\bm{x}, t).\] 

\begin{thm}{Theorem}\label{thm:ODE}
Suppose $\mathbb{I}$ is a real interval and $\bm{f}\:\Omega\to \mathbb{R}^n$ is a smooth function defined on an open subset $\Omega\subset \mathbb{R}^n\times \mathbb{R}$.
Then for any initial data $\bm{x}(t_0)=\bm{u}$ such that $(\bm{u},t)\in\Omega$ the differential equation 
\[\bm{x}'=\bm{f}(\bm{x},t)\]
has a unique solution $t\mapsto \bm{x}(t)$ defined at a maximal interval $\mathbb{J}$ that contains $t_0$.
Moreover
\begin{enumerate}[(a)]
\item  if $\mathbb{J}\ne \mathbb{R}$ (that is, if an end $a$ of $\mathbb{J}$ is finite) then $\bm{x}(t)$ does not have a limit point in $\Omega$ as $t\to a$;
\item  the function $(\bm{u},t_0,t)\mapsto \bm{x}(t)$ has open domain of definition in $\Omega\times \mathbb{R}$ and it is smooth in this domain.
\end{enumerate}

\end{thm}

\subsection*{Higher order}

Suppose we have an ordinary differential equation of order $k$
\[\bm{x}^{(k)}=\bm{f}(\bm{x},\dots,\bm{x}^{(k-1)},t),\]
where $\bm{x}=\bm{x}(t)$ is a function from a real interval to $\mathbb{R}^n$.

This equation can be rewritten as $k$ first order equations as follows with $k-1$ new variables $\bm{y}_i=\bm{x}^{(i)}$:
\[
\begin{cases}
\bm{x}'&=\bm{y}_1
\\
\bm{y}_1'&=\bm{y}_2
\\
&\,\,\vdots
\\
\bm{y}_{k-1}'(t)&=\bm{f}(\bm{x},\bm{y}_{1},\dots,\bm{y}_{k-1},t),
\end{cases}
\]

Using this trick one can reduce a higher order ordinary differential equation to a first order equation. 
In particular we get local existence and uniqueness for solutions of higher order equations as in Theorem \ref{thm:ODE}.



\section{Topology}\label{sec:topology}

The following material is covered in any introductory text to topology; 
one of our favorites is a textbook of Czes Kosniowski \cite{kosniowski}.

\subsection*{Compact sets}

A subset $K$ of a metric space is called \index{compact}\emph{compact} if any sequence of points $(x_n)$ in $K$ has a subsequence that converges to a point $x_\infty$ in $K$.

The following properties follow directly from the definition:

\begin{itemize}
\item A closed subset of a compact space is compact.
\item A continuous image of a compact space is compact.
\end{itemize}

\begin{thm}{Heine--Borel theorem}\label{thm:Heine--Borel}
A subset of Euclidean space is compact if and only if it is closed and bounded.
\end{thm}


\subsection*{Homeomorphisms and embedding}

A bijection $f\:\spc{X}\to\spc{Y}$ between metric spaces is called \index{homeomorphism}\emph{homeomorphism} if $f$ and its inverse $f^{-1}$ are continuous.
If $f\:\spc{X}\to\spc{Y}$ is a homeomorphism to its image $f(\spc{X})\subset \spc{Y}$, then it is called an \index{embedding}\emph{embedding}.

The following theorem characterizes homeomorphisms between compact spaces.

\begin{thm}{Theorem}\label{thm:Hausdorff-compact}
A continuous bijection $f$ between compact metric spaces has a continuous inverse.
In particular, we have the following:

\begin{subthm}{}
Any continuous bijection between compact metric spaces
is a homeomorphism.
\end{subthm}

\begin{subthm}{}
Any continuous injection from compact metric spaces to another metric space
is an embedding.
\end{subthm}


\end{thm}

\subsection*{Jordan's theorem}
\index{Jordan's theorem}

The first part of the following theorem was proved by Camille Jordan, the second part is due to Arthur Schoenflies:

\begin{thm}{Theorem}\label{thm:jordan}
The complement of any closed simple plane curve $\gamma$ has exactly two connected components. 

Moreover, there is a homeomorphism $h\:\mathbb{R}^2\to \mathbb{R}^2$ that maps the unit circle to $\gamma$.
In particular $\gamma$ bounds a topological disc.
\end{thm}

This theorem is known for its simple formulation and quite hard proof.
By now many proofs of this theorem are known.
For the first statement, a very short proof based on a somewhat developed technique is given by Patrick Doyle \cite{doyle},
among elementary proofs, one of our favorites is the proof given by Aleksei Filippov \cite{filippov}.

We use mostly the smooth case of this theorem which is much simpler.
An amusing proof of this case was given by Gregory Chambers and Yevgeny Liokumovich \cite{chambers-liokumovich}.

\subsection*{Connectedness}

Recall that a continuous map $\alpha$ from the unit interval $[0,1]$ to a Euclidean space is called a \index{path}\emph{path}.
If $p=\alpha (0)$ and $q = \alpha (1)$, then we say that $\alpha$ connects $p$ to $q$.


A set $X$ in the Euclidean space is called \index{path-connected set}\emph{path-connected} if any two points $x,y\in X$ can be connected by a path lying in $X$.

A set $X$ in the Euclidean space is called \index{connected set}\emph{connected} if one cannot cover $X$ with two disjoint open sets $V$ and $W$ such that both intersections $X\cap V$ and $X\cap W$ are nonempty.

\begin{thm}{Proposition}
Any path-connected set is connected.

Moreover, any open connected set in the Euclidean space or plane is path-connected.
\end{thm}

Given a point $x\in X$, the maximal connected subset of $X$ containing $x$ is called the \index{connected component}\emph{connected component} of $x$ in $X$.
