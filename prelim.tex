\chapter{Preliminaries}

In this chapter we state and discuss results from different branches of mathematics which were used further in the book.
The reader is not expected to know proofs of these statements, but it is better to check that his intuition agrees with each.  

\section*{Multivariable calculus}

A map $\bm{f}\:\RR^n\to\RR^k$ can be thought as array of functions 
\[f_1,\dots,f_k\:\RR^n\to \RR.\]
The map $\bm{f}$ is called \emph{smooth} if each function $f_i$ is smooth;
that is, all partial derivatives of $f_i$ are defined in the domain of definition of $\bm{f}$.

Inverse function theorem gives a sufficient condition for a smooth function to be invertible in a neighborhood of a given point $p$ in its domain.
The condition is formulated in terms of partial derivative of $f_i$ at $p$.

Implicit function theorem is a close relative to inverse function theorem;
in fact it can be obtained as its corollary.
It is used for instance when we need to pass from parametric to implicit description of curves and surface.

Both theorems reduce the existence of a map satisfying certain equation to a question in linear algebra. 
We use these two theorems only for $n\le 3$.

These two theorems are discussed in any course of multivariable calculus, the classical book of Walter Rudin \cite{rudin} is one of my favorites.

\begin{thm}{Inverse function theorem}\label{thm:inverse}
Let $\bm{f}=(f_1,\dots,f_n)\:\RR^n\to\RR^n$ be a smooth map.
Assume that the Jacobian matrix
\[
\begin{pmatrix}
\dfrac{\partial f_1}{\partial x_1} & \cdots & \dfrac{\partial f_1}{\partial x_n}\\
\vdots & \ddots & \vdots\\
\dfrac{\partial f_n}{\partial x_1} & \cdots & \dfrac{\partial f_n}{\partial x_n} \end{pmatrix}\]
is invertible at some point $p$ in the domain of definition of $\bm{f}$.
Then there is a smooth function $\bm{h}\:\RR^m\to\RR^n$ defined is a neighborhood $\Omega_q$ of $q=\bm{f}(p)$ that is \emph{local inverse of $\bm{f}$ at $p$};
that is, there are neighborhoods $\Omega_p\ni p$ such that
$\bm{f}$ defines a bijection $\Omega_p\to \Omega_q$ and
$\bm{f}(x)=y$ if and only if $x=\bm{h}(y)$ for any $x\in \Omega_p$ and any $y\in \Omega_q$.
\end{thm}

\begin{thm}{Implicit function theorem}\label{thm:imlicit}
Let $\bm{f}=(f_1,\dots,f_n)\:\RR^{n+m}\to\RR^n$ be a smooth map,
$m,n\ge 1$.
Let us consider $\RR^{n+m}$ as a product space $\RR^n\times \RR^m$ with coodinates 
$x_1,\dots,x_n,y_1,\dots,y_m$.
Consider the following matrix 
\[
M=\begin{pmatrix}
\dfrac{\partial f_1}{\partial x_1} & \cdots & \dfrac{\partial f_1}{\partial x_n}\\
\vdots & \ddots & \vdots\\
\dfrac{\partial f_n}{\partial x_1} & \cdots & \dfrac{\partial f_n}{\partial x_n} \end{pmatrix}\]
formed by first $n$ columns of the Jacobian matrix.
Assume $M$ is invertible at some point $p$ in the domain of definition of $\bm{f}$ and $\bm{f}(p)=0$.
Then there is a neighborhood $\Omega_p\ni p$
and smooth function $\bm{h}\:\RR^m\to\RR^n$ defined is a neighborhood $\Omega_0\ni 0$ that
for any $(x_1,\dots,x_n,y_1,\dots y_m)\in \Omega_p$ the equality
\[\bm{f}(x_1,\dots,x_n,y_1,\dots y_m)=0\]
holds if and only if 
\[(x_1,\dots x_n)=\bm{h}(y_1,\dots y_m).\]

\end{thm}

If the assumption in the theorem holds for any point $p$ such that $\bm{f}(p)=0$,
then we say that $0$ is a regular value of $\bm{f}$.
\emph{Sard's theorem} states that most of the values of smooth map are regular; in particular generic smooth function satisfies the assumption of the theorem.

\section*{Initial value problem}

The following theorem guarantees existence and uniqueness of a solution of an initial value problem
for a system of ordinary differential equations
\[
\begin{cases}
x_1'(t)&=f_1(x_1,\dots,x_n,t),
\\
&\dots
\\
x_n'(t)&=f_n(x_1,\dots,x_n,t),
\end{cases}
\]
where each $x_i=x_i(t)$ is a real valued function defined on a real interval $\mathbb{I}$
and each $f_i$ is a smooth function defined on $\RR^n\times \mathbb{I}$.

The array functions $(f_1,\dots,f_n)$ can be considered as one vector-valued function 
$\bm{f}\:\RR^n\times \mathbb{I}\to \RR^n$ and the array $(x_1,\dots,x_n)$ can be considered as a vector  $\bm{x}\in\RR^n$.
Therefore the system can be rewritten as one vector equation 
\[\bm{x}'(t)=\bm{f}(\bm{x}, t).\] 

\begin{thm}{Theorem}\label{thm:ODE}
Suppose $\mathbb{I}$ is a real interval and $\bm{f}\:\RR^n\times\mathbb{I}\to \RR^n$ is a smooth function.
Then for any initial data $\bm{x}(t_0)=\bm{u}$ the differential equation 
\[\bm{x}'(t)=\bm{f}(\bm{x},t)\]
has a unique solution $\bm{x}(t)$ defined at a maximal subinterval $\mathbb{J}$ of $\mathbb{I}$ that contains $t_0$.
Moreover
\begin{enumerate}[(a)]
\item  if $\mathbb{J}\ne \mathbb{I}$, that is, if an end $a$ of $\mathbb{J}$ lies in the interior of $\mathbb{I}$, then $\bm{x}(t)$ diverges for $t\to a$;
\item  the function $(\bm{u},t_0,t)\mapsto \bm{x}(t)$ is smooth.
\end{enumerate}


\end{thm}

\section*{Real analysis}

Recall that a function $f$ is called Lipschitz if there is a constant $L$ such that 
\[|f(x)-f(y)|\le L\cdot|x-y|\]
for values $x$ and $y$ in the domain of definition of $f$.
This definition works for maps between metric spaces, but we will use it for real-to-real functions only.

\begin{thm}{Rademacher's theorem}\label{thm:rademacher}
Let $f\:[a,b]\to\RR$ be a Lipschitz function.
then derivative $f'(x)$ is defined for alomst all $x\in [a,b]$.
Moreover the derivative $f'$ is a bounded measurable function defined almost everywhere in $[a,b]$ and it satisfies the fundamental theorem of calculus; that is, the following identity 
\[f(b)-f(a)=\int_a^b f'(x)\cdot dx,\]
holds if the integral understood in the sense of Lebesgue.
\end{thm}

It is often helps to work with measurable functions; 
it makes possible to extend many statements about continuous function to measurable functions.

\begin{thm}{Lusin's theorem}\label{thm:lusin}
Let $\phi\:[a,b]\to \RR$ be a measurable function.
Then for any $\eps>0$, there is a continuous function $\psi_\eps\:[a,b]\to \RR$ that coincides with $\phi$ outside of a set of measure at most $\eps$.
Moreover, $\phi$ is bounded above and/or below by some constants then we can assume that so is $\psi_\eps$.  
\end{thm}

\section*{Topology}

The first part of the following theorem is proved by Camille Jordan, the second part is due to Arthur Schoenflies.

\begin{thm}{Theorem}
The complement of any closed simple $\gamma$ plane curve has exactly two connected components. 

Moreover the there is a homeomorphism $h\:\RR^2\to \RR^2$ that maps the unit circle to $\gamma$.
In particular one of the components is a topological disc.
\end{thm}

This theorem is known for simple formulation and quite hard proof.
By now many proofs of this theorem are known.
For the first statement, a very short proof based on somewhat developed technique is given by Patrick Doyle \cite{doyle},
among elementary proofs, one of my favorites is the proof given by Aleksei Filippov \cite{filippov}.

A cute proof of general statement for smooth regular curves can be found in \cite{chambers-liokumovich}; we use this theorem mostly in this (much simpler) case.


