\chapter*{Afterword}
\addcontentsline{toc}{chapter}{Afterword}



For further study, you need tensor calculus; the book of Richard Bishop and Samuel Goldberg \cite{bishop-goldberg} is one of our favorites.
Once it is done, you are ready to do Riemannian geometry.

Further you may go with  ``Comparison geometry'' \cite{cheeger-ebin} --- the good old classics from Jeff Cheeger and David Ebin, in this case you might be surprised to see that almost all of it is already known altho it is written in a different language.

A safer option would be another classical book ``Riemannsche Geometrie im Großen'' \cite{gromoll-klingenberg-meyer} by 
Detlef Gromoll,
Wilhelm Klingenberg, 
and  Werner Meyer (it is available in German and Russian only).

Mikhael Gromov's ``Sign and geometric meaning of curvature'' \cite{gromov-1991} is more challenging, but worth trying.

Good luck.

\begin{flushright}
Anton Petrunin and\\
Sergio Zamora Barrera.
\end{flushright}
