\chapter{Parallel transport}

\section{Parallel fields}

Let $\Sigma$ be a smooth regular surface in the Euclidean space and $\alpha\:[a,b]\z\to \Sigma$ be a smooth curve.
Suppose $t\mapsto v(t)$ is a smooth vector valued function (a field on $\alpha$) such that
that at each $t$, the vector $v(t)$ lies in the tangent plane $\T_{\alpha(t)}\Sigma$.
The family of vectors $v(t)$ is called \emph{parallel} along $\alpha$ if $v'(t)\perp\T_{\alpha(t)}$ for any $t$.

In general the family of tangent planes $\T_{\alpha(t)}\Sigma$ is not parallel.
Therefore one can not expect to have a truly parallel family $v(t)$ with $v'\equiv 0$.
The condition $v'(t)\perp\T_{\alpha(t)}$ means that this family as parallel as possible, it rotates the same way as the tangent plane, but does not rotate inside of the plane.

Note that according to Claim~\ref{clm:gamma''}, for any geodesic $\gamma$, the velocity field $v(t)=\gamma'(t)$ is parallel along $\gamma$.


\begin{thm}{Exercise}\label{ex:parallel}
Let $\Sigma$ be a smooth regular surface in the Euclidean space and $\alpha\:[a,b]\to \Sigma$ be a smooth curve and $v(t)$, $w(t)$ a parallel vector fields along $\alpha$.
\begin{enumerate}[(a)]
 \item Show that $|v(t)|$ is constant.
 \item Show that the angle $\theta(t)$ between $v(t)$ and $w(t)$ is constant.
\end{enumerate}
\end{thm}

\section{Parallel transport}

Assume $p=\gamma(a)$ and $q=\gamma(b)$.
Given a tangent vector $v\in\T_p$ there is unique parallel field $v(t)$ along $\alpha$ such that $v(a)=v$.
The latter follows from Picard's theorem (a fundamental theorem of ordinary differential equations); uniqueness also follows from Exercise~\ref{ex:parallel}.

The vector $v(b)\in\T_q$ is called parallel transport of $v$ along $\alpha$ and denoted as $\iota_\alpha(v)$.
As it follow from the Exercise~\ref{ex:parallel}, parallel transport $\iota_\alpha\:\T_p\to\T_q$ is an an isometry;
it depends on the choice of $\alpha$ --- for another curve $\beta$ connecting $p$ to $q$ the parallel transport $\iota_\beta\:\T_p\to\T_q$ might be different.

To interpret the parallel transport physically, 
think of walking along $\alpha$ and carrying a perfectly balanced bike wheel;
you touch only the axis keeping it normal to $\Sigma$.
It should be physically evident that if the wheel is non-spinning at the starting point $p$, then it will not be spinning after stopping at $q$.
Indeed, by pushing axis one can not produce torque to spin the wheel. 
The map that sends the initial position of the wheel to the final position is  the parallel transportation~$\iota_\alpha$.

This physical interpretation will be used further.

On a more formal level, one can choose a partition $a=t_0<\dots<t_n=b$ of $[a,b]$
and consider the sequence of orthogonal projections $\phi_i\:\T_{\alpha(t_{i-1})}\to\T_{\alpha(t_i)}$.
For a fine partition, the composition 
\[\phi_n\circ\dots\circ\phi_1\:T_p\z\to\T_q\]
gives an approximation of $\iota_\alpha$.\footnote{Each $\phi_i$ does not increase the magnitude of a vector and so does the composition.
It is straightforward to see that if if the partition is sufficiently fine, then it almost preserves the composition almost preserves magnitudes.}

\section{Geodesic curvature}

Suppose $\Sigma$ is a smooth regular surface.
Assume $\Sigma$ is oriented,
In this case terms ``left'' and ``right'' can be used the same sense as in the plane.

\parbf{Broken geodesics.}
For a concatenation of two geodesics in $\Sigma$, let us define signed external angle at their common point;
the sign is positive if we turn left and negative if we turn right. 

A concatenation of minimizing geodesics in $\Sigma$ will be called \emph{broken geodesic}.

The sum of the signed external angles for a broken geodesic $\gamma$ in $\Sigma$ will be called \emph{total geodesic curvature} of $\gamma$; it will be denoted as $\tgc\gamma$.
Note that if we change the orientation of the curve, then the total geodesic curvature changes sign.

\parbf{Smooth regular curves.}
The total geodesic curvature can be also defined for a smooth unit-speed curve $\gamma$ on $\Sigma$.
Since $\Sigma$ is oriented, we can fix a Gauss map $\nu\:\Sigma\to \SS^2$.
Then for any $t$ the vectors $\nu(t)=\nu(\gamma(t))$ and the velocity vector $\tau(t)=\gamma'(t)$ are unit vectors that are normal to each other.
Denote by $\mu(t)$ the unit vector that is normal to both $\nu(t)$ and $\tau(t)$ that point to the left from $\gamma$.

Since $\gamma$ is unit-speed, the acceleration $\gamma''(t)$ is perpendicular to $\tau(t)$;
therefore at any parameter value $t$, we have
\[\gamma''(t)=k_n(t)\cdot \nu(t)+k_g(t)\cdot \mu(t),\]
for some real numbers $k_n(t)$ and $k_g(t)$.
The numbers $k_n(t)$ and $k_g(t)$ are called \emph{normal} and \emph{geodesic curvature} of $\gamma$ at $t$ correspondingly.

Note that the geodesic curvature vanishes if $\gamma$ is a geodesic; 
it measures how much a given curve diverges from a geodesic.

The total geodesic curvature of $\gamma$ can be defined as the integral of its geodesic curvature
\[\tgc\gamma=\int_\gamma k_g.\]

If $\gamma$ is a regular curve then one has to parameterize it by arc length and then apply the definition above.

The given two definitions for regular curve and broken geodesic agree in the sense that total geodesic curvature of smooth curve is a limit of the total geodesic curvatures of inscribed broken geodesics in $\gamma$ for finer and finer partitions.

\parbf{Piecewise smooth curves.}
One could also combine both definitions to define total geodesic curvature for a concatenation of smooth curves in $\Sigma$ --- we need to add the total geodesic curvature of all the edges and the signed external angle at each vertex. 


\begin{thm}{Proposition}
Assume $\gamma$ is a closed broken geodesic in a smooth oriented surface $\Sigma$ that starts and ends at point $p$.
Then the parallel transport $\iota_\gamma\:T_p\to\T_p$ is a rotation of the the plane $\T_p$ clockwise by angle $\tgc\gamma$.

Moreover, the same statement holds for smooth closed curves and piecewise smooth curves.
\end{thm}

\parit{Proof.}
Assume $\gamma$ is a concatination of geodesics $\gamma_1,\dots,\gamma_n$.
Fix a tangent vector $v$ at $p$ and extend it to a parallel vector field along $\gamma$.
Since $w_i=\gamma_i'$ is parallel along $\gamma_i$, the angle $\phi_i$ between $v$ and $w_i$ stays constant on each $\gamma_i$.

If $\theta_i$ denotes the external angle at this vertex of switch from $\gamma_{i}$ to $\gamma_{i+1}$, we have that 
\[\phi_{i+1}=\phi_i-\theta_i \pmod{2\cdot\pi}.\]
Therefore after going around we get the 
\[\phi_{n+1}-\phi_1=-\theta_1-\dots-\theta_n=-\tgc\gamma.\]
Hence the result.

For the smooth unit-speed curve $\gamma\:[a,b]\to\Sigma$, the proof is analogous.
If $\phi(t)$ denotes the angle between $v(t)$ and $\gamma'(t)$ then $\phi'(t)+k_g=0$.
Hence after going around it will rotate clockwise by angle
\[\tgc\gamma=\int_a^b k_g(t)\cdot dt.\]

The case of piecewise regular smooth curve is a straightforward combination of the above two cases. 
\qeds







\section{Area of spherical triangle}

\begin{thm}{Lemma}
Let $\Delta$ be a spherical triangle;
that is, $\Delta$ is the intersection of three closed half-spheres in the unit sphere $\SS^2$.
Then 
\[\area\Delta=\alpha+\beta+\gamma-\pi,\eqlbl{eq:area(Delta)}\]
where $\alpha$, $\beta$ and $\gamma$ are the angles of $\Delta$.
\end{thm}

The value $\alpha+\beta+\gamma-\pi$ is called \emph{excess} of the triangle $\Delta$.

\parit{Proof.}
Recall that 
\[\area\SS^2=4\cdot\pi.\eqlbl{eq:area(S2)}\]
The slice $S_\alpha$ of the sphere between two meridians under angle $\alpha$ is proportional to $\alpha$. Since for $S_\pi$ is a half-sphere we get that the coefficient is 2; that is,
\[\area S_\alpha=2\cdot \alpha.\eqlbl{eq:area(Sa)}\]

Extending the sides of $\Delta$ we get 6 slices: two $S_\alpha$, two $S_\beta$ and two $S_\gamma$ which cover whole sphere at once,
but $\Delta$ and its centrally symmetric copy $\Delta'$ is covered 3 times.
It follows that
\[2\cdot \area S_\alpha+2\cdot \area S_\beta+2\cdot \area S_\gamma
=\area\SS^2+4\cdot\area\Delta.\]
Substituting \ref{eq:area(S2)} and \ref{eq:area(Sa)} and simplifying, we get \ref{eq:area(Delta)}.
\qeds

If the contour $\partial\Delta$ of spherical triangle with angles $\alpha$, $\beta$ and $\gamma$ is oriented such that the triangle lies on the left, then its external angles are  $\pi-\alpha$, $\pi-\beta$ and $\pi-\gamma$.
Therefore \ref{eq:area(Delta)} can be rewritten as 
\[\tgc{\partial\Delta}+\area\Delta=2\cdot \pi.
\eqlbl{eq:sphere-gauss-bonnet}\]


