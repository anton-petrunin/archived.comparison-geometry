\chapter{Parallel transport}

\section{Parallel fields}

Let $\Sigma$ be a smooth regular surface in the Euclidean space and $\alpha\:[a,b]\z\to \Sigma$ be a smooth curve.
Suppose $t\mapsto v(t)$ is a smooth vector valued function (a field on $\alpha$) such that
that at each $t$, the vector $v(t)$ lies in the tangent plane $\T_{\alpha(t)}\Sigma$.
The family of vectors $v(t)$ is called \emph{parallel} along $\alpha$ if $v'(t)\perp\T_{\alpha(t)}$ for any $t$.

In general the family of tangent planes $\T_{\alpha(t)}\Sigma$ is not parallel.
Therefore one can not expect to have a truly parallel family $v(t)$ with $v'\equiv 0$.
The condition $v'(t)\perp\T_{\alpha(t)}$ means that this family as parallel as possible, it rotates the same way as the tangent plane, but does not rotate inside of the plane.

Note that according to Claim~\ref{clm:gamma''}, for any geodesic $\gamma$, the velocity field $v(t)=\gamma'(t)$ is parallel along $\gamma$.


\begin{thm}{Exercise}\label{ex:parallel}
Let $\Sigma$ be a smooth regular surface in the Euclidean space and $\alpha\:[a,b]\to \Sigma$ be a smooth curve and $v(t)$, $w(t)$ a parallel vector fields along $\alpha$.
\begin{enumerate}[(a)]
 \item Show that $|v(t)|$ is constant.
 \item Show that the angle $\theta(t)$ between $v(t)$ and $w(t)$ is constant.
\end{enumerate}
\end{thm}

\section{Parallel transport}

Assume $p=\gamma(a)$ and $q=\gamma(b)$.
Given a tangent vector $v\in\T_p$ there is unique parallel field $v(t)$ along $\alpha$ such that $v(a)=v$.
The latter follows from Picard's theorem (the  fundamental theorem of ordinary differential equations); the uniqueness also follows from Exercise~\ref{ex:parallel}.

The vector $v(b)\in\T_q$ is called \emph{parallel transport} of $v$ along $\alpha$ and denoted as $\iota_\alpha(v)$.
As it follow from the Exercise~\ref{ex:parallel}, parallel transport $\iota_\alpha\:\T_p\to\T_q$ is an an isometry;
it depends on the choice of $\alpha$ --- for another curve $\beta$ connecting $p$ to $q$ in $\Sigma$, the parallel transport $\iota_\beta\:\T_p\to\T_q$ might be different.

To interpret the parallel transport physically, 
think of walking along $\alpha$ and carrying a perfectly balanced bike wheel in such a way that you touch only its axis keeping it normal to $\Sigma$.
It should be physically evident that if the wheel is non-spinning at the starting point $p$, then it will not be spinning after stopping at $q$.\footnote{Indeed, by pushing axis one can not produce torque to spin the wheel.}
The map that sends the initial position of the wheel to the final position is  the parallel transportation~$\iota_\alpha$.

This physical interpretation was suggested by Mark Levi \cite{levi}; it will be used further.

On a more formal level, one can choose a partition $a=t_0<\dots\z<t_n=b$ of $[a,b]$
and consider the sequence of orthogonal projections $\phi_i\:\T_{\alpha(t_{i-1})}\to\T_{\alpha(t_i)}$.
For a fine partition, the composition 
\[\phi_n\circ\dots\circ\phi_1\:T_p\z\to\T_q\]
gives an approximation of $\iota_\alpha$.\footnote{Each $\phi_i$ does not increase the magnitude of a vector and so does the composition.
It is straightforward to see that if if the partition is sufficiently fine, then it almost preserves the composition almost preserves magnitudes.}

\section{Geodesic curvature}

Suppose $\Sigma$ is a smooth regular surface.
Assume $\Sigma$ is oriented;
in this case terms ``left'' and ``right'' can be used the same sense as in the plane.

\parbf{Broken geodesics.}
For a concatenation of two geodesics in $\Sigma$, let us define signed external angle at their common point;
the sign is positive if it turns left and negative if it turns right. 

A concatenation of minimizing geodesics in $\Sigma$ will be called \emph{broken geodesic}.

The sum of the signed external angles for a broken geodesic $\gamma$ in $\Sigma$ will be called \emph{total geodesic curvature} of $\gamma$; it will be denoted as~$\tgc\gamma$.

Note that if we change the orientation of the curve, then the total geodesic curvature changes sign.

\parbf{Smooth regular curves.}
The total geodesic curvature can be also defined for a smooth unit-speed curve $\gamma\:[a,b]\to\Sigma$.

Let $\nu\:\Sigma\to \SS^2$ be the Gauss map that defines the orientation on $\Sigma$.
Then for any $t$ the vectors $\nu(t)=\nu(\gamma(t))$ and the velocity vector $\tau(t)=\gamma'(t)$ are unit vectors that are normal to each other.
Denote by $\mu(t)$ the unit vector that is normal to both $\nu(t)$ and $\tau(t)$ that point to the left from $\gamma$.
Note that the triple $\tau(t),\mu(t),\nu(t)$ is a orthogonal basis for any $t$

Since $\gamma$ is unit-speed, the acceleration $\gamma''(t)$ is perpendicular to $\tau(t)$;
therefore at any parameter value $t$, we have
\[\gamma''(t)=k_n(t)\cdot \nu(t)+k_g(t)\cdot \mu(t),\]
for some real numbers $k_n(t)$ and $k_g(t)$.
The numbers $k_n(t)$ and $k_g(t)$ are called \emph{normal} and \emph{geodesic curvature} of $\gamma$ at $t$ correspondingly.

Note that the geodesic curvature vanishes if $\gamma$ is a geodesic. 
It measures how much a given curve diverges from a geodesic;
it is positive if $\gamma$ turns left and negative if it turns right.

The total geodesic curvature of $\gamma$ can be defined as the integral of its geodesic curvature
\[\tgc\gamma=\int_a^b k_g(t)\cdot dt.\]

If $\gamma$ is a regular curve then one has to parameterize it by arc length and then apply the definition above.

The given two definitions for regular curve and broken geodesic agree in the following sense: that total geodesic curvature of smooth curve is a limit of the total geodesic curvatures of inscribed broken geodesics in $\gamma$ for finer and finer partitions.

\parbf{Piecewise smooth curves.}
One could also combine both definitions to define total geodesic curvature for a concatenation of smooth curves in $\Sigma$ --- we need to add the total geodesic curvature of all the edges and the signed external angle at each vertex. 


\begin{thm}{Proposition}
Assume $\gamma$ is a closed broken geodesic in a smooth oriented surface $\Sigma$ that starts and ends at point $p$.
Then the parallel transport $\iota_\gamma\:T_p\to\T_p$ is a rotation of the the plane $\T_p$ clockwise by angle $\tgc\gamma$.

Moreover, the same statement holds for smooth closed curves and piecewise smooth curves.
\end{thm}

\parit{Proof.}
Assume $\gamma$ is a concatination of geodesics $\gamma_1,\dots,\gamma_n$.
Fix a tangent vector $v$ at $p$ and extend it to a parallel vector field along $\gamma$.
Since $w_i=\gamma_i'$ is parallel along $\gamma_i$, the angle $\phi_i$ between $v$ and $w_i$ stays constant on each $\gamma_i$.

If $\theta_i$ denotes the external angle at this vertex of switch from $\gamma_{i}$ to $\gamma_{i+1}$, we have that 
\[\phi_{i+1}=\phi_i-\theta_i \pmod{2\cdot\pi}.\]
Therefore after going around we get the 
\[\phi_{n+1}-\phi_1=-\theta_1-\dots-\theta_n=-\tgc\gamma.\]
Hence the result.

For the smooth unit-speed curve $\gamma\:[a,b]\to\Sigma$, the proof is analogous.
If $\phi(t)$ denotes the angle between $v(t)$ and $\gamma'(t)$ then 
\[\phi'(t)+k_g=0.\]
Hence after going around it will rotate clockwise by angle
\[\tgc\gamma=\int_a^b k_g(t)\cdot dt.\]

The case of piecewise regular smooth curve is a straightforward combination of the above two cases. 
\qeds







\section{Signed area on the sphere}

\begin{thm}{Lemma}
Let $\Delta$ be a spherical triangle;
that is, $\Delta$ is the intersection of three closed half-spheres in the unit sphere $\SS^2$.
Then 
\[\area\Delta=\alpha+\beta+\gamma-\pi,\eqlbl{eq:area(Delta)}\]
where $\alpha$, $\beta$ and $\gamma$ are the angles of $\Delta$.
\end{thm}

The value $\alpha+\beta+\gamma-\pi$ is called \emph{excess} of the triangle $\Delta$.

\begin{wrapfigure}{o}{22 mm}
\vskip-0mm
\centering
\includegraphics{mppics/pic-43}
\vskip-0mm
\end{wrapfigure}

\parit{Proof.}
Recall that 
\[\area\SS^2=4\cdot\pi.\eqlbl{eq:area(S2)}\]

Note that the area of a spherical slice $S_\alpha$ between two meridians meeting at angle $\alpha$ is proportional to $\alpha$.
Since for $S_\pi$ is a half-sphere, from \ref{eq:area(S2)}, we get $\area S_\pi\z=2\cdot\pi$.
Therefore the coefficient is 2; that is,
\[\area S_\alpha=2\cdot \alpha.\eqlbl{eq:area(Sa)}\]

Extending the sides of $\Delta$ we get 6 slices: two $S_\alpha$, two $S_\beta$ and two $S_\gamma$ which cover most of the sphere once,
but the triangle $\Delta$ and its centrally symmetric copy $\Delta'$ are covered 3 times.
It follows that
\[2\cdot \area S_\alpha+2\cdot \area S_\beta+2\cdot \area S_\gamma
=\area\SS^2+4\cdot\area\Delta.\]
Substituting \ref{eq:area(S2)} and \ref{eq:area(Sa)} and simplifying, we get \ref{eq:area(Delta)}.
\qeds

If the contour $\partial\Delta$ of spherical triangle with angles $\alpha$, $\beta$ and $\gamma$ is oriented such that the triangle lies on the left, then its external angles are  $\pi-\alpha$, $\pi-\beta$ and $\pi-\gamma$.
Therefore the total geodesic curvature of $\partial\Delta$ is $\tgc{\partial\Delta}=3\cdot\pi-\alpha-\beta-\gamma$.
The identity \ref{eq:area(Delta)} can be rewritten as 
\[\tgc{\partial\Delta}+\area\Delta=2\cdot \pi.
\eqlbl{eq:sphere-gauss-bonnet}\]

The formula \ref{eq:sphere-gauss-bonnet} holds for arbitrary spherical polygon bounded by a simple broken geodesic;
that is, intersection of finitely many closed half-spheres.
The latter can be proved by triangulating the poygon, applying the formula for each triangle in the triangulation and summing up the results.

\begin{wrapfigure}{o}{42 mm}
\vskip-0mm
\centering
\includegraphics{mppics/pic-45}
\vskip-0mm
\end{wrapfigure}

If a spherical polygon is $P$ divided in two polygons $Q$ and $R$ by a diagonal $vw$
then 
\[\tgc{\partial P}+2\cdot\pi =\tgc{\partial Q}+\tgc{\partial R}.\]
Indeed, for the external angles $\theta_P$, $\theta_Q$ and $\theta_R$ at $v$, we have 
\[\theta_P+\pi=\theta_Q+\theta_R.\]
the same holds for the external angles at $w$ and the rest of the external angles of $P$ appear once on $Q$ or $R$.)
Therefore if the formula \ref{eq:sphere-gauss-bonnet} holds for $\Delta=Q$ and $R$,
then it holds for $P$.


\parbf{Signed area.}
The same formula modulo $2\cdot \pi$ holds for any closed broken geodesic, if one use \emph{signed area} surrounded by curve instead of usual area;
that is, we count area of the regions taking into account how many times the curve goes around the region.

Namely, we have to choose a \emph{south pole} and state that its region has zero multiplicity.
When you cross the curve the mulitplicity has to changes by $\pm1$; it increases by 1 if the curve cross the way from left to right and it decreases by 1 otherwise.
The signed area surrounded by a closed curve is the sum of area of all the regions counted with the multiplicities.

\begin{wrapfigure}{o}{32 mm}
\vskip-0mm
\centering
\includegraphics{mppics/pic-44}
\vskip-0mm
\end{wrapfigure}

Here is an example of broken line with multiplicities assuming that the big region has a pole inside.

This signed-area formula can be proved in a similar way:
Apply the formula for each triangle with vertex at the north pole and base at each edge of the broken geodesic.
Sum the resulting identities taking each with a sign: plus if the triangle lies on the left from the edge and minus if the triangle lies on the right from edge.

Choosing a different pole will change all the coefficients by the same number.
So the resulting formula holds only modulo the area of $\SS^2$, which is $4\cdot \pi$ --- this will not destroy identity modulo $2\cdot\pi$.

Furthermore, by approximation, the signed-area formula holds for any reasonable curves, say piecewise smooth regular curves on the sphere.
Summarizing, we hope the discussion above convinced the reader that the following statement hold:

\begin{thm}{Proposition}
For any closed piecewise smooth regular curves on the sphere $\alpha$, 
we have that 
\[\tgc\alpha+\area\alpha=0 \pmod{2\cdot\pi},\]
where $\area\alpha$ denotes the signed area surrounded by $\alpha$ and $\tgc\alpha$ the total geodesic curvature of $\alpha$.

\end{thm}

