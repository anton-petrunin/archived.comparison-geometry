
\chapter{tmp}





\section*{Signed area}
The formula \ref{eq:sphere-gauss-bonnet} holds modulo $2\cdot \pi$ for any closed broken geodesic, if one use \emph{signed area} surrounded by curve instead of usual area;
that is, we count area of the regions taking into account how many times the curve goes around the region.

Namely, we have to choose a \emph{south pole} and state that its region has zero multiplicity.
When you cross the curve the mulitplicity changes by $\pm1$; we add 1 if the curve crosses your path from left to right and we subtract 1 otherwise.
The signed area surrounded by a closed curve is the sum of area of all the regions counted with multiplicities.

\begin{wrapfigure}{o}{32 mm}
\vskip-0mm
\centering
\includegraphics{mppics/pic-44}
\vskip-0mm
\end{wrapfigure}

Here is an example of a broken line with multiplicities assuming that the big region has the south pole inside.

This signed-area formula can be proved in a similar way:
Apply the formula for each triangle with vertex at the north pole and base at each edge of the broken geodesic.
Sum the resulting identities taking each with a sign: plus if the triangle lies on the left from the edge and minus if the triangle lies on the right from edge.

Choosing a different pole will change all the coefficients by the same number.
So the resulting formula holds only modulo the area of $\SS^2$, which is $4\cdot \pi$ --- this will not destroy identity modulo $2\cdot\pi$.

Furthermore, by approximation, the signed-area formula holds for any reasonable curve, say piecewise smooth regular curves on the sphere.
Summarizing, we hope the discussion above convinced the reader that the following statement holds.

A domain $\Delta$ in a surface is called a \emph{disc} (or more precisely \emph{topological disc}) if it is bounded by a closed simple curve and can be parameterized by a unit plane disc 
\[\DD=\set{(x,y)\in\RR^2}{x^2+y^2\le1}.\]
That is, there is a continuous bijection $\DD\to\Delta$.

\begin{thm}{Proposition}\label{prop:spherical-gb}
For any closed piecewise smooth regular curve $\alpha$ on the sphere, 
we have that 
\[\tgc\alpha+\area\alpha=0 \pmod{2\cdot\pi},\]
where $\area\alpha$ denotes the signed area surrounded by $\alpha$ and $\tgc\alpha$ the total geodesic curvature of $\alpha$.

Moreover, if $\alpha$ is a simple curve that bounds a disc $\Delta$ on the left from it, then we have 
\[\tgc\alpha+\area\Delta=2\cdot\pi.\]

\end{thm}





\section*{Gauss--Bonnet formula}


\begin{thm}{Theorem}\label{thm:gb}
Let $\Delta$ be a disc in a smooth oriented surface $\Sigma$ bounded by a simple piecewise smooth and regular curve $\partial \Delta$ that is oriented in such a way that $\Delta$ lies on its left.
Then 
\[\tgc{\partial\Delta}+\iint_\Delta G=2\cdot \pi,\eqlbl{eq:g-b}\]
where $G$ denotes the Gauss curvature of $\Sigma$.
\end{thm}

For geodesic triangles this theorem was proved by Carl Friedrich Gauss \cite{gauss};
Pierre Bonnet and Jacques Binet independently generalized the statement for arbitrary curves. 
The modern formulation described below was given by Wilhelm Blaschke. 


\parit{Remarks; (1).}
For a general compact domain $\Delta$ (not necessary a disc) we have that
\[\tgc{\partial\Delta}+\iint_{\Delta} G=2\cdot  \pi\cdot\chi(\Delta),\eqlbl{eq:g-b-euler}\]
where $\chi(\Delta)$ is the so called \emph{Euler's characteristic} of $\Delta$.
The Euler's characteristic is \emph{topological invariant}, in particular preserved in a continuous deformation.

If a surface $\Sigma$ (possibly with boundary) can be divided into $f$ discs by drawing $e$ edges connecting $v$ vertexes, then 
\[\chi(\Sigma)=v-e+f.\]
For example the disc $\DD$ has Euler's characteristic $1$; 
\begin{figure}[h!]
\vskip-0mm
\centering
\includegraphics{mppics/pic-49}
\vskip-0mm
\end{figure}
it can be divided into discs many ways, 
but each time we have $v-e=f=1$.
The latter agrees with \ref{eq:g-b} and \ref{eq:g-b-euler}.
It is useful to know that $\chi(\SS^2)=2$; $\chi(\TT^2)=0$ where $\TT^2$ denotes torus; 
$\chi( S_g)=2-2\cdot g$, where $S_g$ is a surface of genus $g$; that is, sphere with $g$ handles.

\parit{(2).} Note that if $\Sigma$ is a plane, then a geodesic in $\Sigma$ are formed by line segments.
In this case the statement of theorem follows from Exercise~\ref{ex:pm2pi}.

\parit{(3).} If $\Sigma$ is the unit sphere, then $G\equiv1$ and therefore formula \ref{eq:g-b} can be rewritten as 
\[\tgc{\partial\Delta}+\area\Delta=2\cdot \pi,\]
which follows from Proposition~\ref{prop:spherical-gb}.

\medskip

We will give an informal proof of \ref{thm:gb} based on the bike wheel interpretation described above.
We suppose that it is intuitively clear that moving the axis of the wheel without changing its direction does not change the direction of the wheel's spikes.

More precisely, assume we keep the axis of a non-spinning bike wheel and perform the following two experiments:
\begin{enumerate}[(i)]
\item We move it around and bring the axis back to the original position. 
As a result the wheel might rotate by some angle; let us measure this angle.

\item
We move the direction of the axis the same way as before without moving the center of the wheel.
After that we measure the angle of rotation.
\end{enumerate}
Then the resulting angle in these two experiments is the same. 

Consider a oriented smooth surface $\Sigma$ with the spherical; map $\Norm\:\Sigma\to \SS^2$.
Note that for any point $p$ on $\Sigma$, the tangent plane $\T_p\Sigma$ is parallel to the tangent plane $\T_{\Norm(p)}\SS^2$; so we can identify these tangent spaces.
From the experiments above, we get the following:

\begin{thm}{Lemma}\label{lem:spherical-image}
Suppose $\alpha$ is a piecewise smooth regular curve in a smooth regular surface $\Sigma$ which has a Gauss map $\Norm\:\Sigma\to \SS^2$.
Then the parallel transport along $\alpha$ in $\Sigma$ coincides with the parallel transport along the curve $\beta=\Norm\circ\alpha$ in $\SS^2$.
\end{thm}

\begin{thm}{Exercise}
Let $\Sigma$ be a smooth closed surface with positive Gauss curvature.
Given a line $\ell$ denote by $\omega_\ell$ the closed curve formed by points with tangent planes parallel to $\ell$.\footnote{Equivalently the normal vector at any point of $\omega_\ell$ is perpendicular to $\ell$. If the light falls on $\Sigma$ from one side parallel to $\ell$, then $\omega_\ell$ divides the bright and dark sides of~$\Sigma$.}
Show that parallel transport around $\omega_\ell$ is the identity map.
\end{thm}

Now we are ready to prove the theorem.

\parit{Proof of \ref{thm:gb}.}
Let $\alpha$ be the boundary $\partial\Delta$ parameterized in such a way that $\Delta$ lies on the left from it.
Assume $p$ is the point where $\alpha$ starts and ends.

Set $\beta=\Norm\circ\gamma$ and $q=\Norm(p)$, so the spherical curve $\beta$ starts and ends at $q$.

By Lemma \ref{lem:spherical-image} the parallel transport along $\alpha$ in $\Sigma$ coincides with the parallel transport along the curve $\beta$ in $\SS^2$.
By Proposition~\ref{prop:pt+tgc}, it follows that 
\[\tgc{\alpha,\Sigma}=\tgc{\beta,\SS^2} \pmod{2\cdot \pi}.\]

By Proposition~\ref{prop:spherical-gb},
\[\tgc{\beta,\SS^2}+\area\beta=0\pmod{2\cdot \pi}.\]
Therefore 
\[\tgc{\alpha,\Sigma}+\area\beta=0\pmod{2\cdot \pi}.\]

Recall that the shape  operator $s_p\: T_p\Sigma \to \T_{\Norm(p)}\SS^2=T_p\Sigma$ is the Jacobian of the Gauss map $\Norm\:\Sigma\to \SS^2$ at the point $p$.
In appropriately chosen coordinates in $T_p$, the shape operator can be presented by a diagonal matrix 
$\left(\begin{smallmatrix}
k_1&0
\\
0&k_2
\end{smallmatrix}\right)$, where $k_1$ and $k_2$ are the principle curvatures at $p$.
Therefore, the determinant of $s_p$ is the Gauss curvature at~$p$.

If $\Sigma$ is a closed surface with positive Gauss curvature, then the Gauss map $\Norm\:\Sigma\to\SS^2$ is a smooth bijection.
Therefore 
\[\iint_\Delta G=\area[\Norm(\Delta)].\]

In the general case we have to count the area $\Norm(\Delta)$ taking orientation and multiplicity of the Gauss map into account.
In this case 
\[\iint_\Delta G=\area\beta,\]
where $\area\beta$ is the signed area surrounded by $\beta$; it is defined above.
Therefore 
\[\tgc{\alpha,\Sigma}+\iint_\Delta G=0\pmod{2\cdot \pi}.\eqlbl{eq:gb(mod2pi)}\]

If $\Delta$ is a disc in the plane, then Gauss curvature vanishes and by Exercise~\ref{ex:pm2pi}, we have 
\[\tgc{\partial\Delta}+\iint_\Delta G=2\cdot \pi.\]
Assunme that $\Sigma_t$ is a smooth one parameter family of surfaces with 
a one parameter family of discs $\Delta_t\subset \Sigma_t$ and $\alpha_t$ is the boundary $\partial\Delta_t$ parameterized in such a way that $\Delta_t$ lies on the left from it.
The value 
\[f(t)=\tgc{\alpha_t}+\iint_\Delta G\]
is continuous in $t$ and by \ref{eq:gb(mod2pi)} it has to be constant.

If $\Sigma_0$ is a plane, then 
\[\tgc{\partial\Delta_0}+\iint_{\Delta_0} G=2\cdot \pi.\]
Intuitively it is clear that any disc can be obtained as a resultof continuous deformation of plane disc.
Therefore 
\[\tgc{\partial\Delta_1}+\iint_{\Delta_1} G=2\cdot \pi\]
for arbitrary disc $\Delta_1$; whence \ref{eq:g-b} follows.
\qeds











