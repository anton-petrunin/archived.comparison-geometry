\chapter{Parallel transport}

\section{Parallel fields}

Let $\Sigma$ be a smooth regular surface in the Euclidean space and $\alpha\:[a,b]\z\to \Sigma$ be a smooth curve.
A smooth vector valued function $t\mapsto v(t)$ is called a \emph{tangent field} on $\alpha$ if
the vector $v(t)$ lies in the tangent plane $\T_{\alpha(t)}\Sigma$ for each $t$.
The family of vectors $v(t)$ is called \emph{parallel} along $\alpha$ if $v'(t)\perp\T_{\alpha(t)}$ for any $t$.

In general the family of tangent planes $\T_{\alpha(t)}\Sigma$ is not parallel.
Therefore one can not expect to have a truly parallel family $v(t)$ with $v'\equiv 0$.
The condition $v'(t)\perp\T_{\alpha(t)}$ means that this family as parallel as possible, it rotates the same way as the tangent plane, but does not rotate inside of the plane.

Note that according to Claim~\ref{clm:gamma''}, for any geodesic $\gamma$, the velocity field $v(t)=\gamma'(t)$ is parallel along $\gamma$.


\begin{thm}{Exercise}\label{ex:parallel}
Let $\Sigma$ be a smooth regular surface in the Euclidean space and $\alpha\:[a,b]\to \Sigma$ be a smooth curve and $v(t)$, $w(t)$ a parallel vector fields along $\alpha$.
\begin{enumerate}[(a)]
 \item Show that $|v(t)|$ is constant.
 \item Show that the angle $\theta(t)$ between $v(t)$ and $w(t)$ is constant.
\end{enumerate}
\end{thm}

\section{Parallel transport}

Assume $p=\gamma(a)$ and $q=\gamma(b)$.
Given a tangent vector $v\in\T_p$ there is unique parallel field $v(t)$ along $\alpha$ such that $v(a)=v$.
The latter follows from Picard's theorem; the uniqueness also follows from Exercise~\ref{ex:parallel}.

The vector $v(b)\in\T_q$ is called \emph{parallel transport} of $v$ along $\alpha$ and denoted as $\iota_\alpha(v)$.
As it follow from the Exercise~\ref{ex:parallel}, parallel transport $\iota_\alpha\:\T_p\to\T_q$ is an an isometry;
it depends on the choice of $\alpha$ --- for another curve $\beta$ connecting $p$ to $q$ in $\Sigma$, the parallel transport $\iota_\beta\:\T_p\to\T_q$ might be different.

To interpret the parallel transport physically, 
think of walking along $\alpha$ and carrying a perfectly balanced bike wheel in such a way that you touch only its axis keeping it normal to $\Sigma$.
It should be physically evident that if the wheel is non-spinning at the starting point $p$, then it will not be spinning after stopping at $q$.\footnote{Indeed, by pushing axis one can not produce torque to spin the wheel.}
The map that sends the initial position of the wheel to the final position is  the parallel transportation~$\iota_\alpha$.

This physical interpretation was suggested by Mark Levi \cite{levi}; it will be used further.

On a more formal level, one can choose a partition $a=t_0<\dots\z<t_n=b$ of $[a,b]$
and consider the sequence of orthogonal projections $\phi_i\:\T_{\alpha(t_{i-1})}\to\T_{\alpha(t_i)}$.
For a fine partition, the composition 
\[\phi_n\circ\dots\circ\phi_1\:T_p\z\to\T_q\]
gives an approximation of $\iota_\alpha$.\footnote{Each $\phi_i$ does not increase the magnitude of a vector and so does the composition.
It is straightforward to see that if if the partition is sufficiently fine, then it almost preserves the composition almost preserves magnitudes.}

\begin{wrapfigure}{r}{39 mm}
\begin{lpic}[t(-0 mm),b(-4 mm),r(0 mm),l(0 mm)]{pics/unbend(1)}
\lbl[t]{11.5,29;$p$}
\lbl[r]{10.5,37;$p_t$}
\lbl[t]{32.5,26;$q$}
\lbl[t]{26,30;$\gamma(t)$}
\lbl{20,13;{\Large $\Sigma$}}
\end{lpic}
\end{wrapfigure}

\begin{thm}{Advanced exercise}
Let $\Sigma$ be a smooth closed strictly convex surface 
in $\RR^3$ 
and $\gamma\:[0,\ell]\z\to \Sigma$ be a unit-speed minimizing geodesic.
Set $p\z=\gamma(0)$, $q=\gamma(\ell)$ and 
$$p_t=\gamma(t)-t\cdot\gamma'(t),$$ 
where $\gamma'(t)$ denotes the velocity vector of $\gamma$ at $t$.

Show that for any $t\in (0,\ell)$,
one {}\emph{cannot see}  $q$ from $p_t$;
that is, the line segment $[p_tq]$ intersects $\Sigma$ at a point distinct from $q$.%
\footnote{Hint: Show that the concatenation of the line segment $[p_t\gamma(t)]$ and the arc $\gamma|_{[t,\ell]}$ is a minimizing geodesic in the closed set $W$ outside of $\Sigma$.}
\end{thm}

\section{Geodesic curvature}

Suppose $\Sigma$ is a smooth regular surface.
Assume $\Sigma$ is oriented;
in this case terms ``left'' and ``right'' can be used the same sense as in the plane.

\parbf{Broken geodesics.}
For a concatenation of two geodesics in $\Sigma$, let us define signed external angle at their common point;
the sign is positive if it turns left and negative if it turns right. 

A concatenation of minimizing geodesics in $\Sigma$ will be called \emph{broken geodesic}.

The sum of the signed external angles for a broken geodesic $\gamma$ in $\Sigma$ will be called \emph{total geodesic curvature} of $\gamma$; it will be denoted as~$\tgc\gamma$ or $\tgc{\gamma,\Sigma}$ if we need to emphasize that $\gamma$ is a curve in $\Sigma$.

Note that if we change the orientation of the curve, then the total geodesic curvature changes sign.

\parbf{Smooth regular curves.}
The total geodesic curvature can be also defined for a smooth unit-speed curve $\gamma\:[a,b]\to\Sigma$.

Let $\nu\:\Sigma\to \SS^2$ be the Gauss map that defines the orientation on $\Sigma$.
Then for any $t$ the vectors $\nu(t)=\nu(\gamma(t))$ and the velocity vector $\tau(t)=\gamma'(t)$ are unit vectors that are normal to each other.
Denote by $\mu(t)$ the unit vector that is normal to both $\nu(t)$ and $\tau(t)$ that point to the left from $\gamma$.
Note that the triple $\tau(t),\mu(t),\nu(t)$ is a orthogonal basis for any $t$.

Since $\gamma$ is unit-speed, the acceleration $\gamma''(t)$ is perpendicular to $\tau(t)$;
therefore at any parameter value $t$, we have
\[\gamma''(t)=k_g(t)\cdot \mu(t)-k_n(t)\cdot \nu(t),\]
for some real numbers $k_n(t)$ and $k_g(t)$.
The numbers $k_n(t)$ and $k_g(t)$ are called \emph{normal} and \emph{geodesic curvature} of $\gamma$ at $t$ correspondingly.

Note that the geodesic curvature vanishes if $\gamma$ is a geodesic. 
It measures how much a given curve diverges from a geodesic;
it is positive if $\gamma$ turns left and negative if it turns right.

The total geodesic curvature of $\gamma$ can be defined as the integral of its geodesic curvature
\[\tgc\gamma=\int_a^b k_g(t)\cdot dt.\]

If $\gamma$ is a regular curve then one has to parameterize it by arc length and then apply the definition above.

The given two definitions for regular curve and broken geodesic agree in the following sense.
If $\beta_n$ is a sequence of inscribed broken geodesics in $\gamma$ for finer and finer partitions, then 
\[\tgc{\beta_n}\to\tgc\gamma,\]
where the left and right hand sides are defined using the first and the second definitions above.
The proof is straightforward, it can be done the same way as the case in the plane. %??? is it???

\begin{thm}{Exercise}
Let $\gamma$ be a smooth unit-speed curve in a  smooth regular surface $\Sigma$ with a Gauss map $\nu$.
Set $\nu(t)=\nu\circ\gamma(t)$.
Show that 
\[k_n(t)=\langle\gamma'(t),\nu'(t)\rangle.\]

\end{thm}

\begin{thm}{Exercise}
Let $\gamma$ be a piecewise smooth regular curve in an oriented smooth regular surface $\Sigma$.
Show that 
\[|\tgc\gamma|\le \tc\gamma\]
and in case of equality $\gamma$ lies in a plane.
\end{thm}

\parbf{Piecewise smooth curves.}
One could also combine both definitions to define total geodesic curvature for 
piecewise smooth curve $\gamma$ in $\Sigma$; that is
a concatenation of smooth regular curves.
We need to add the total geodesic curvature of all the edges of $\gamma$ and the signed external angle at each vertex. 



\begin{thm}{Proposition}\label{prop:pt+tgc}
Assume $\gamma$ is a closed broken geodesic in a smooth oriented surface $\Sigma$ that starts and ends at point $p$.
Then the parallel transport $\iota_\gamma\:T_p\to\T_p$ is a rotation of the the plane $\T_p$ clockwise by angle $\tgc\gamma$.

Moreover, the same statement holds for smooth closed curves and piecewise smooth curves.
\end{thm}

\begin{wrapfigure}{o}{22 mm}
\vskip-0mm
\centering
\includegraphics{mppics/pic-48}
\vskip-0mm
\end{wrapfigure}

\parit{Proof.}
Assume $\gamma$ is a cyclic concatenation of geodesics $\gamma_1,\dots,\gamma_n$.
Fix a tangent vector $v$ at $p$ and extend it to a parallel vector field along $\gamma$.
Since $w_i(t)=\gamma_i'(t)$ is parallel along $\gamma_i$, the angle $\phi_i$ between $v$ and $w_i$ stays constant on each $\gamma_i$.

If $\theta_i$ denotes the external angle at this vertex of switch from $\gamma_{i}$ to $\gamma_{i+1}$, we have that 
\[\phi_{i+1}=\phi_i-\theta_i \pmod{2\cdot\pi}.\]
Therefore after going around we get that 
\[\phi_{n+1}-\phi_1=-\theta_1-\dots-\theta_n=-\tgc\gamma.\]
Hence the the first statement follows.

For the smooth unit-speed curve $\gamma\:[a,b]\to\Sigma$, the proof is analogous.
If $\phi(t)$ denotes the angle between $v(t)$ and $w(t)=\gamma'(t)$, then 
\[\phi'(t)+k_g(t)\equiv0\]
Whence the angle of rotation 
\begin{align*}
\phi(b)-\phi(a)&=\int_a^b \phi'(t)\cdot dt=
\\
&=-\int_a^b k_g\cdot dt=
\\
&=-\tgc\gamma
\end{align*}

The case of piecewise regular smooth curve is a straightforward combination of the above two cases. 
\qeds







\section{Signed area on the sphere}

\begin{thm}{Lemma}
Let $\Delta$ be a spherical triangle;
that is, $\Delta$ is the intersection of three closed half-spheres in the unit sphere $\SS^2$.
Then 
\[\area\Delta=\alpha+\beta+\gamma-\pi,\eqlbl{eq:area(Delta)}\]
where $\alpha$, $\beta$ and $\gamma$ are the angles of $\Delta$.
\end{thm}

The value $\alpha+\beta+\gamma-\pi$ is called \emph{excess} of the triangle $\Delta$.

\begin{wrapfigure}{o}{22 mm}
\vskip-0mm
\centering
\includegraphics{mppics/pic-43}
\vskip-0mm
\end{wrapfigure}

\parit{Proof.}
Recall that 
\[\area\SS^2=4\cdot\pi.\eqlbl{eq:area(S2)}\]

Note that the area of a spherical slice $S_\alpha$ between two meridians meeting at angle $\alpha$ is proportional to $\alpha$.
Since for $S_\pi$ is a half-sphere, from \ref{eq:area(S2)}, we get $\area S_\pi\z=2\cdot\pi$.
Therefore the coefficient is 2; that is,
\[\area S_\alpha=2\cdot \alpha.\eqlbl{eq:area(Sa)}\]

Extending the sides of $\Delta$ we get 6 slices: two $S_\alpha$, two $S_\beta$ and two $S_\gamma$ which cover most of the sphere once,
but the triangle $\Delta$ and its centrally symmetric copy $\Delta'$ are covered 3 times.
It follows that
\[2\cdot \area S_\alpha+2\cdot \area S_\beta+2\cdot \area S_\gamma
=\area\SS^2+4\cdot\area\Delta.\]
Substituting \ref{eq:area(S2)} and \ref{eq:area(Sa)} and simplifying, we get \ref{eq:area(Delta)}.
\qeds



If the contour $\partial\Delta$ of spherical triangle with angles $\alpha$, $\beta$ and $\gamma$ is oriented such that the triangle lies on the left, then its external angles are  $\pi-\alpha$, $\pi-\beta$ and $\pi-\gamma$.
Therefore the total geodesic curvature of $\partial\Delta$ is $\tgc{\partial\Delta}=3\cdot\pi-\alpha-\beta-\gamma$.
The identity \ref{eq:area(Delta)} can be rewritten as 
\[\tgc{\partial\Delta}+\area\Delta=2\cdot \pi.
\eqlbl{eq:sphere-gauss-bonnet}\]

The formula \ref{eq:sphere-gauss-bonnet} holds for arbitrary spherical polygon bounded by a simple broken geodesic;
that is, intersection of finitely many closed half-spheres.
The latter can be proved by triangulating the poygon, applying the formula for each triangle in the triangulation and summing up the results.

\begin{wrapfigure}{o}{42 mm}
\vskip-0mm
\centering
\includegraphics{mppics/pic-45}
\vskip-0mm
\end{wrapfigure}

If a spherical polygon is $P$ divided in two polygons $Q$ and $R$ by a diagonal $vw$
then 
\[\tgc{\partial P}+2\cdot\pi =\tgc{\partial Q}+\tgc{\partial R}.\]
Indeed, for the internal angles $Q$ and $R$ at $v$ are $\alpha$ and $\beta$,
then their external angles are $\pi-\alpha$ and $\pi-\beta$ respectfully.
The internal angle of $P$ in this case is $\alpha+\beta$ and its external angle is $\pi-\alpha-\beta$
Clearly we have that 
\[(\pi-\alpha)+(\pi-\beta)=(\pi-\alpha-\beta)+\pi;\]
that is, the sum of external angles of $Q$ and $R$ at $v$ is $\pi$ plus the external angle of $P$ at $v$. 
The same holds for the external angles at $w$ and the rest of the external angles of $P$ appear once on $Q$ or $R$.
Therefore if the formula \ref{eq:sphere-gauss-bonnet} holds for $Q$ and $R$,
then it holds for~$P$.

\begin{thm}{Exercise}
Assume $\gamma$ is a simple broken geodesic on $\SS^2$ that divides its area into two equal parts.
Show that $\tgc\gamma=0$.
\end{thm}



\parbf{Signed area.}
The formula \ref{eq:sphere-gauss-bonnet} holds modulo $2\cdot \pi$ for any closed broken geodesic, if one use \emph{signed area} surrounded by curve instead of usual area;
that is, we count area of the regions taking into account how many times the curve goes around the region.

Namely, we have to choose a \emph{south pole} and state that its region has zero multiplicity.
When you cross the curve the mulitplicity has to changes by $\pm1$; we add 1 if the curve cross the way from left to right and we subtract 1 otherwise.
The signed area surrounded by a closed curve is the sum of area of all the regions counted with the multiplicities.

\begin{wrapfigure}{o}{32 mm}
\vskip-0mm
\centering
\includegraphics{mppics/pic-44}
\vskip-0mm
\end{wrapfigure}

Here is an example of broken line with multiplicities assuming that the big region has the pole inside.

This signed-area formula can be proved in a similar way:
Apply the formula for each triangle with vertex at the north pole and base at each edge of the broken geodesic.
Sum the resulting identities taking each with a sign: plus if the triangle lies on the left from the edge and minus if the triangle lies on the right from edge.

Choosing a different pole will change all the coefficients by the same number.
So the resulting formula holds only modulo the area of $\SS^2$, which is $4\cdot \pi$ --- this will not destroy identity modulo $2\cdot\pi$.

Furthermore, by approximation, the signed-area formula holds for any reasonable curves, say piecewise smooth regular curves on the sphere.
Summarizing, we hope the discussion above convinced the reader that the following statement hold.

A domain $\Delta$ in a surface is called a \emph{disc} (or more precisely \emph{topological disc}) if it is bounded by a closed simple curve and can be parameterized by a unit plane disc 
\[\DD=\set{(x,y)\in\RR^2}{x^2+y^2\le1}.\]
That is there is a continuous bijection $\DD\to\Delta$.

\begin{thm}{Proposition}\label{prop:spherical-gb}
For any closed piecewise smooth regular curve $\alpha$ on the sphere, 
we have that 
\[\tgc\alpha+\area\alpha=0 \pmod{2\cdot\pi},\]
where $\area\alpha$ denotes the signed area surrounded by $\alpha$ and $\tgc\alpha$ the total geodesic curvature of $\alpha$.

Moreover, if $\alpha$ is a simple curve that bounds a disc $\Delta$ on the left from it, then we have 
\[\tgc\alpha+\area\Delta=2\cdot\pi.\]

\end{thm}





\section{Gauss--Bonnet formula}


\begin{thm}{Theorem}\label{thm:gb}
Let $\Delta$ be a disc in smooth oriented surface $\Sigma$ bounded by a simple piecewise smooth and regular curve $\partial \Delta$ that is oriented in such a way that $\Delta$ lies on its left.
Then 
\[\tgc{\partial\Delta}+\iint_\Delta G=2\cdot \pi,\eqlbl{eq:g-b}\]
where $G$ denotes the Gauss curvature of $\Sigma$.
\end{thm}

\parit{Remarks; (1).}
For a general compact domain $\Delta$ (not necessary a disc) we have that
\[\tgc{\partial\Delta}+\iint_{\Delta} G=2\cdot  \pi\cdot\chi(\Delta),\eqlbl{eq:g-b-euler}\]
where $\chi(\Delta)$ is the so called \emph{Euler's characteristic} of $\Delta$.
The Euler's characteristic is \emph{topological invariant}, in particular preserved in a continuous deformation.

If a surface $\Sigma$ (possibly with boundary) can be divided into $f$ discs by drawing $e$ edges connecting $v$ vertexes, then 
\[\chi(\Sigma)=v-e+f.\]
For example the disc $\DD$ has Euler's characteristic $1$; 
\begin{figure}[h!]
\vskip-0mm
\centering
\includegraphics{mppics/pic-49}
\vskip-0mm
\end{figure}
it can be divided into discs many ways, 
but each time we have $v-e=f=1$.
The latter agrees with \ref{eq:g-b} and \ref{eq:g-b-euler}.
It is useful to know that $\chi(\SS^2)=2$; $\chi(\TT^2)=0$ where $\TT^2$ denotes torus; 
$\chi( S_g)=2-2\cdot g$, where $S_g$ is a surface of genus $g$; that is, sphere with $g$ handles.

\parit{(2).} Note that if $\Sigma$ is a plane then geodesic in $\Sigma$ are formed by line segments.
In this case the statement of theorem follows from Exercise~\ref{ex:pm2pi}.

\parit{(3).} If $\Sigma$ is the unit sphere then $G\equiv1$ and therefore the formula \ref{eq:g-b} can rewritten as 
\[\tgc{\partial\Delta}+\area\Delta=2\cdot \pi,\]
which follow from Proposition~\ref{prop:spherical-gb}.

\medskip

We will give an informal proof of \ref{thm:gb} based on the bike wheel interpretation described above.
We suppose that it is intuitively clear that moving the axis of the wheel without changing its direction does not change the direction of the wheel's spokes.

More precisely, assume we keep the axis of a non-spinning bike wheel and perform the following two experiments:
\begin{enumerate}[(i)]
\item We moved it around and bring the axis back to the original position. 
As a result the wheel might rotate by some angle; let us measure this angle.

\item
We move the direction of the axis the same way as before without moving the center of the wheel.
After that we measure the angle of rotation.
\end{enumerate}
Then the resulting angle in these two experiments is the same. 

Consider a surface $\Sigma$ with a Gauss map $\nu\:\Sigma\to \SS^2$.
Note that for any point $p$ on $\Sigma$, the tangent plane $\T_p\Sigma$ is parallel to the tangent plane $\T_{\nu(p)}\SS^2$; so we can identify these tangent spaces.
From the experiments above, we get the following:

\begin{thm}{Lemma}\label{lem:spherical-image}
Suppose $\alpha$ is a piecewise smooth regular curve in a smooth regular surface $\Sigma$ which has a Gauss map $\nu\:\Sigma\to \SS^2$.
Then the parallel transport along $\alpha$ in $\Sigma$ coincides with the parallel transport along the curve $\beta=\nu\circ\alpha$ in $\SS^2$.
\end{thm}

\begin{thm}{Exercise}
Let $\Sigma$ be a smooth closed surface with positive Gauss curvature.
Given a line $\ell$ denote by $\omega_\ell$ the closed curve formed by points with tangent planes parallel to $\ell$.\footnote{Equivalently the normal vector at any point of $\omega_\ell$ is perpendicular to $\ell$. If the light falls on $\Sigma$ from one side parallel to $\ell$, then $\omega_\ell$ divides bright and dark side of~$\Sigma$.}
Show that parallel transport around $\omega_\ell$ is an identity map.
\end{thm}

Now we are ready to prove the theorem.

\parit{Proof of \ref{thm:gb}.}
Let $\alpha$ be the boundary $\partial\Delta$ parameterized in such a way that $\Delta$ lies on the left from it.
Assume $p$ is the point where $\alpha$ starts and ends.

Set $\beta=\nu\circ\gamma$ and $q=\nu(p)$, so the spherical curve $\beta$ starts and ends at $q$.

By Lemma \ref{lem:spherical-image} the parallel transport along $\alpha$ in $\Sigma$ coincides with the parallel transport along the curve $\beta$ in $\SS^2$.
By Proposition~\ref{prop:pt+tgc}, it follows that 
\[\tgc{\alpha,\Sigma}=\tgc{\beta,\SS^2} \pmod{2\cdot \pi}.\]

By Proposition~\ref{prop:spherical-gb},
\[\tgc{\beta,\SS^2}+\area\beta=0\pmod{2\cdot \pi}.\]
Therefore 
\[\tgc{\alpha,\Sigma}+\area\beta=0\pmod{2\cdot \pi}.\]

Recall that the shape  operator $s_p\: T_p\Sigma \to \T_{\nu(p)}\SS^2=T_p\Sigma$ is the Jacobian of the Gauss map $\nu\:\Sigma\to \SS^2$ at the point $p$.
In a appropriately chosen coordinates in $T_p$, the shape operator can be presented by a diagonal matrix 
$\left(\begin{smallmatrix}
k_1&0
\\
0&k_2
\end{smallmatrix}\right)$, where $k_1$ and $k_2$ are the principle curvatures at $p$.
Therefore, the determinant of $s_p$ is the Gauss curvature at~$p$.

If $\Sigma$ is a closed surface with positive Gauss curvature, then the Gauss map $\nu\:\Sigma\to\SS^2$ is a smooth bijection.
Therefore 
\[\iint_\Delta G=\area[\nu(\Delta)].\]

In general case we have to count area $\nu(\Delta)$ taking orientation and multiplicity of the Gauss map into account.
In this case 
\[\iint_\Delta G=\area\beta,\]
where $\area\beta$ is the signed area surrounded by $\beta$; it is defined above.
Therefore 
\[\tgc{\alpha,\Sigma}+\iint_\Delta G=0\pmod{2\cdot \pi}.\eqlbl{eq:gb(mod2pi)}\]

If $\Delta$ is a disc in the plane then Gauss curvature vanish and by Exercise~\ref{ex:pm2pi}, we have 
\[\tgc{\partial\Delta}+\iint_\Delta G=2\cdot \pi.\]
Assunme that $\Sigma_t$ is a smooth one parameter family of surfaces with 
a one parameter family of discs $\Delta_t\subset \Sigma_t$ and $\alpha_t$ is the boundary $\partial\Delta_t$ parameterized in such a way that $\Delta_t$ lies on the left from it.
The value 
\[f(t)=\tgc{\alpha_t}+\iint_\Delta G\]
is continuous in $t$ and by \ref{eq:gb(mod2pi)} it has to be constant.

If $\Sigma_0$ is a plane, then 
\[\tgc{\partial\Delta_0}+\iint_{\Delta_0} G=2\cdot \pi.\]
Intuitively it is clear that any disc can be obtained as a resultof continuous deformation of plane disc.
Therefore 
\[\tgc{\partial\Delta_1}+\iint_{\Delta_1} G=2\cdot \pi\]
for arbitrary disc $\Delta_1$; whence \ref{eq:g-b} follows.
\qeds





\begin{thm}{Exercise}
 Assume $\gamma$ is a closed simple curve with constant geodesic curvature $1$ in a smooth convex closed surface $\Sigma$.
 Show that 
 \[\length\gamma\le 2\cdot\pi;\]
that is the length of $\gamma$ can not exceed the length of the unit circle in the plane.  
\end{thm}


\begin{thm}{Exercise}
Let $\gamma$ be a closed simple geodesic on a smooth convex closed surface $\Sigma$.
Assume $\nu\:\Sigma\to\SS^2$ is a Gasuss map.
Show that the curve $\nu\circ\gamma$ divides the sphere into regions of equal area.
\end{thm}


\begin{wrapfigure}{o}{32 mm}
\vskip-0mm
\centering
\includegraphics{mppics/pic-46}
\vskip-0mm
\end{wrapfigure}

\begin{thm}{Exercise}
Let $\Sigma$ be a smooth closed surface with a closed geodesic $\gamma$.
Assume $\gamma$ has exactly 4 self-intersection at the points $a$, $b$, $c$ and $d$ that appear on $\gamma$ in the order $a,a,b,b,c,c,d,d$.
Show that $\Sigma$ can not have positive Gauss curvature.\footnote{Hint: estimate integral of Gauss curvature bounded by a simple geodesic loop.}
\end{thm}

\begin{thm}{Advanced exercise}
Let $\Sigma$ be a smooth closed surface with positive Gauss curvature and $p\in\Sigma$. 
Suppose $\gamma$ be a closed geodesic that does not pass thru $p$.
Assume $\Sigma\backslash\{p\}$ parametrized by the plane.
Can it happen that in this parametrization,  $\gamma$ look like one of the curves on the diagram?
\begin{figure}[h!]
\vskip-0mm
\centering
\includegraphics{mppics/pic-47}
\vskip-0mm
\end{figure}
Say as much as possible about possible/impossible diagrams of that type.
\end{thm}


