\chapter{Gauss--Bonnet formula}




\section{Signed area on the sphere}

\begin{thm}{Lemma}
Let $\Delta$ be a spherical triangle;
that is, $\Delta$ is the intersection of three closed half-spheres in the unit sphere $\SS^2$.
Then 
\[\area\Delta=\alpha+\beta+\gamma-\pi,\eqlbl{eq:area(Delta)}\]
where $\alpha$, $\beta$ and $\gamma$ are the angles of $\Delta$.
\end{thm}

The value $\alpha+\beta+\gamma-\pi$ is called \emph{excess} of the triangle $\Delta$.

\begin{wrapfigure}{o}{22 mm}
\vskip-0mm
\centering
\includegraphics{mppics/pic-43}
\vskip-0mm
\end{wrapfigure}

\parit{Proof.}
Recall that 
\[\area\SS^2=4\cdot\pi.\eqlbl{eq:area(S2)}\]

Note that the area of a spherical slice $S_\alpha$ between two meridians meeting at angle $\alpha$ is proportional to $\alpha$.
Since for $S_\pi$ is a half-sphere, from \ref{eq:area(S2)}, we get $\area S_\pi\z=2\cdot\pi$.
Therefore the coefficient is 2; that is,
\[\area S_\alpha=2\cdot \alpha.\eqlbl{eq:area(Sa)}\]

Extending the sides of $\Delta$ we get 6 slices: two $S_\alpha$, two $S_\beta$ and two $S_\gamma$ which cover most of the sphere once,
but the triangle $\Delta$ and its centrally symmetric copy $\Delta'$ are covered 3 times.
It follows that
\[2\cdot \area S_\alpha+2\cdot \area S_\beta+2\cdot \area S_\gamma
=\area\SS^2+4\cdot\area\Delta.\]
Substituting \ref{eq:area(S2)} and \ref{eq:area(Sa)} and simplifying, we get \ref{eq:area(Delta)}.
\qeds



If the contour $\partial\Delta$ of spherical triangle with angles $\alpha$, $\beta$ and $\gamma$ is oriented such that the triangle lies on the left, then its external angles are  $\pi-\alpha$, $\pi-\beta$ and $\pi-\gamma$.
Therefore the total geodesic curvature of $\partial\Delta$ is $\tgc{\partial\Delta}=3\cdot\pi-\alpha-\beta-\gamma$.
The identity \ref{eq:area(Delta)} can be rewritten as 
\[\tgc{\partial\Delta}+\area\Delta=2\cdot \pi.
\eqlbl{eq:sphere-gauss-bonnet}\]

The formula \ref{eq:sphere-gauss-bonnet} holds for arbitrary spherical polygon bounded by a simple broken geodesic;
that is, intersection of finitely many closed half-spheres.
The latter can be proved by triangulating the poygon, applying the formula for each triangle in the triangulation and summing up the results.

\begin{wrapfigure}{o}{42 mm}
\vskip-0mm
\centering
\includegraphics{mppics/pic-45}
\vskip-0mm
\end{wrapfigure}

If a spherical polygon is $P$ divided in two polygons $Q$ and $R$ by a diagonal $vw$
then 
\[\tgc{\partial P}+2\cdot\pi =\tgc{\partial Q}+\tgc{\partial R}.\]
Indeed, for the internal angles $Q$ and $R$ at $v$ are $\alpha$ and $\beta$,
then their external angles are $\pi-\alpha$ and $\pi-\beta$ respectfully.
The internal angle of $P$ in this case is $\alpha+\beta$ and its external angle is $\pi-\alpha-\beta$
Clearly we have that 
\[(\pi-\alpha)+(\pi-\beta)=(\pi-\alpha-\beta)+\pi;\]
that is, the sum of external angles of $Q$ and $R$ at $v$ is $\pi$ plus the external angle of $P$ at $v$. 
The same holds for the external angles at $w$ and the rest of the external angles of $P$ appear once on $Q$ or $R$.
Therefore if the formula \ref{eq:sphere-gauss-bonnet} holds for $Q$ and $R$,
then it holds for~$P$.

\begin{thm}{Exercise}
Assume $\gamma$ is a simple broken geodesic on $\SS^2$ that divides its area into two equal parts.
Show that $\tgc\gamma=0$.
\end{thm}



\parbf{Signed area.}
The formula \ref{eq:sphere-gauss-bonnet} holds modulo $2\cdot \pi$ for any closed broken geodesic, if one use \emph{signed area} surrounded by curve instead of usual area;
that is, we count area of the regions taking into account how many times the curve goes around the region.

Namely, we have to choose a \emph{south pole} and state that its region has zero multiplicity.
When you cross the curve the mulitplicity has to changes by $\pm1$; we add 1 if the curve cross the way from left to right and we subtract 1 otherwise.
The signed area surrounded by a closed curve is the sum of area of all the regions counted with the multiplicities.

\begin{wrapfigure}{o}{32 mm}
\vskip-0mm
\centering
\includegraphics{mppics/pic-44}
\vskip-0mm
\end{wrapfigure}

Here is an example of broken line with multiplicities assuming that the big region has the pole inside.

This signed-area formula can be proved in a similar way:
Apply the formula for each triangle with vertex at the north pole and base at each edge of the broken geodesic.
Sum the resulting identities taking each with a sign: plus if the triangle lies on the left from the edge and minus if the triangle lies on the right from edge.

Choosing a different pole will change all the coefficients by the same number.
So the resulting formula holds only modulo the area of $\SS^2$, which is $4\cdot \pi$ --- this will not destroy identity modulo $2\cdot\pi$.

Furthermore, by approximation, the signed-area formula holds for any reasonable curves, say piecewise smooth regular curves on the sphere.
Summarizing, we hope the discussion above convinced the reader that the following statement hold.

A domain $\Delta$ in a surface is called a \emph{disc} (or more precisely \emph{topological disc}) if it is bounded by a closed simple curve and can be parameterized by a unit plane disc 
\[\DD=\set{(x,y)\in\RR^2}{x^2+y^2\le1}.\]
That is there is a continuous bijection $\DD\to\Delta$.

\begin{thm}{Proposition}\label{prop:spherical-gb}
For any closed piecewise smooth regular curve $\alpha$ on the sphere, 
we have that 
\[\tgc\alpha+\area\alpha=0 \pmod{2\cdot\pi},\]
where $\area\alpha$ denotes the signed area surrounded by $\alpha$ and $\tgc\alpha$ the total geodesic curvature of $\alpha$.

Moreover, if $\alpha$ is a simple curve that bounds a disc $\Delta$ on the left from it, then we have 
\[\tgc\alpha+\area\Delta=2\cdot\pi.\]

\end{thm}





\section{Gauss--Bonnet formula}


\begin{thm}{Theorem}\label{thm:gb}
Let $\Delta$ be a disc in smooth oriented surface $\Sigma$ bounded by a simple piecewise smooth and regular curve $\partial \Delta$ that is oriented in such a way that $\Delta$ lies on its left.
Then 
\[\tgc{\partial\Delta}+\iint_\Delta G=2\cdot \pi,\eqlbl{eq:g-b}\]
where $G$ denotes the Gauss curvature of $\Sigma$.
\end{thm}

For geodesic triangles this theorem was proved by Carl Friedrich Gauss \cite{gauss};
Pierre Bonnet and Jacques Binet independently generalized the statement for arbitrary curves. 
The modern formulation described below was given by Wilhelm Blaschke. 


\parit{Remarks; (1).}
For a general compact domain $\Delta$ (not necessary a disc) we have that
\[\tgc{\partial\Delta}+\iint_{\Delta} G=2\cdot  \pi\cdot\chi(\Delta),\eqlbl{eq:g-b-euler}\]
where $\chi(\Delta)$ is the so called \emph{Euler's characteristic} of $\Delta$.
The Euler's characteristic is \emph{topological invariant}, in particular preserved in a continuous deformation.

If a surface $\Sigma$ (possibly with boundary) can be divided into $f$ discs by drawing $e$ edges connecting $v$ vertexes, then 
\[\chi(\Sigma)=v-e+f.\]
For example the disc $\DD$ has Euler's characteristic $1$; 
\begin{figure}[h!]
\vskip-0mm
\centering
\includegraphics{mppics/pic-49}
\vskip-0mm
\end{figure}
it can be divided into discs many ways, 
but each time we have $v-e=f=1$.
The latter agrees with \ref{eq:g-b} and \ref{eq:g-b-euler}.
It is useful to know that $\chi(\SS^2)=2$; $\chi(\TT^2)=0$ where $\TT^2$ denotes torus; 
$\chi( S_g)=2-2\cdot g$, where $S_g$ is a surface of genus $g$; that is, sphere with $g$ handles.

\parit{(2).} Note that if $\Sigma$ is a plane then geodesic in $\Sigma$ are formed by line segments.
In this case the statement of theorem follows from Exercise~\ref{ex:pm2pi}.

\parit{(3).} If $\Sigma$ is the unit sphere then $G\equiv1$ and therefore the formula \ref{eq:g-b} can rewritten as 
\[\tgc{\partial\Delta}+\area\Delta=2\cdot \pi,\]
which follow from Proposition~\ref{prop:spherical-gb}.

\medskip

We will give an informal proof of \ref{thm:gb} based on the bike wheel interpretation described above.
We suppose that it is intuitively clear that moving the axis of the wheel without changing its direction does not change the direction of the wheel's spokes.

More precisely, assume we keep the axis of a non-spinning bike wheel and perform the following two experiments:
\begin{enumerate}[(i)]
\item We moved it around and bring the axis back to the original position. 
As a result the wheel might rotate by some angle; let us measure this angle.

\item
We move the direction of the axis the same way as before without moving the center of the wheel.
After that we measure the angle of rotation.
\end{enumerate}
Then the resulting angle in these two experiments is the same. 

Consider a surface $\Sigma$ with a Gauss map $\nu\:\Sigma\to \SS^2$.
Note that for any point $p$ on $\Sigma$, the tangent plane $\T_p\Sigma$ is parallel to the tangent plane $\T_{\nu(p)}\SS^2$; so we can identify these tangent spaces.
From the experiments above, we get the following:

\begin{thm}{Lemma}\label{lem:spherical-image}
Suppose $\alpha$ is a piecewise smooth regular curve in a smooth regular surface $\Sigma$ which has a Gauss map $\nu\:\Sigma\to \SS^2$.
Then the parallel transport along $\alpha$ in $\Sigma$ coincides with the parallel transport along the curve $\beta=\nu\circ\alpha$ in $\SS^2$.
\end{thm}

\begin{thm}{Exercise}
Let $\Sigma$ be a smooth closed surface with positive Gauss curvature.
Given a line $\ell$ denote by $\omega_\ell$ the closed curve formed by points with tangent planes parallel to $\ell$.\footnote{Equivalently the normal vector at any point of $\omega_\ell$ is perpendicular to $\ell$. If the light falls on $\Sigma$ from one side parallel to $\ell$, then $\omega_\ell$ divides bright and dark side of~$\Sigma$.}
Show that parallel transport around $\omega_\ell$ is an identity map.
\end{thm}

Now we are ready to prove the theorem.

\parit{Proof of \ref{thm:gb}.}
Let $\alpha$ be the boundary $\partial\Delta$ parameterized in such a way that $\Delta$ lies on the left from it.
Assume $p$ is the point where $\alpha$ starts and ends.

Set $\beta=\nu\circ\gamma$ and $q=\nu(p)$, so the spherical curve $\beta$ starts and ends at $q$.

By Lemma \ref{lem:spherical-image} the parallel transport along $\alpha$ in $\Sigma$ coincides with the parallel transport along the curve $\beta$ in $\SS^2$.
By Proposition~\ref{prop:pt+tgc}, it follows that 
\[\tgc{\alpha,\Sigma}=\tgc{\beta,\SS^2} \pmod{2\cdot \pi}.\]

By Proposition~\ref{prop:spherical-gb},
\[\tgc{\beta,\SS^2}+\area\beta=0\pmod{2\cdot \pi}.\]
Therefore 
\[\tgc{\alpha,\Sigma}+\area\beta=0\pmod{2\cdot \pi}.\]

Recall that the shape  operator $s_p\: T_p\Sigma \to \T_{\nu(p)}\SS^2=T_p\Sigma$ is the Jacobian of the Gauss map $\nu\:\Sigma\to \SS^2$ at the point $p$.
In a appropriately chosen coordinates in $T_p$, the shape operator can be presented by a diagonal matrix 
$\left(\begin{smallmatrix}
k_1&0
\\
0&k_2
\end{smallmatrix}\right)$, where $k_1$ and $k_2$ are the principle curvatures at $p$.
Therefore, the determinant of $s_p$ is the Gauss curvature at~$p$.

If $\Sigma$ is a closed surface with positive Gauss curvature, then the Gauss map $\nu\:\Sigma\to\SS^2$ is a smooth bijection.
Therefore 
\[\iint_\Delta G=\area[\nu(\Delta)].\]

In general case we have to count area $\nu(\Delta)$ taking orientation and multiplicity of the Gauss map into account.
In this case 
\[\iint_\Delta G=\area\beta,\]
where $\area\beta$ is the signed area surrounded by $\beta$; it is defined above.
Therefore 
\[\tgc{\alpha,\Sigma}+\iint_\Delta G=0\pmod{2\cdot \pi}.\eqlbl{eq:gb(mod2pi)}\]

If $\Delta$ is a disc in the plane then Gauss curvature vanish and by Exercise~\ref{ex:pm2pi}, we have 
\[\tgc{\partial\Delta}+\iint_\Delta G=2\cdot \pi.\]
Assunme that $\Sigma_t$ is a smooth one parameter family of surfaces with 
a one parameter family of discs $\Delta_t\subset \Sigma_t$ and $\alpha_t$ is the boundary $\partial\Delta_t$ parameterized in such a way that $\Delta_t$ lies on the left from it.
The value 
\[f(t)=\tgc{\alpha_t}+\iint_\Delta G\]
is continuous in $t$ and by \ref{eq:gb(mod2pi)} it has to be constant.

If $\Sigma_0$ is a plane, then 
\[\tgc{\partial\Delta_0}+\iint_{\Delta_0} G=2\cdot \pi.\]
Intuitively it is clear that any disc can be obtained as a resultof continuous deformation of plane disc.
Therefore 
\[\tgc{\partial\Delta_1}+\iint_{\Delta_1} G=2\cdot \pi\]
for arbitrary disc $\Delta_1$; whence \ref{eq:g-b} follows.
\qeds





\begin{thm}{Exercise}
 Assume $\gamma$ is a closed simple curve with constant geodesic curvature $1$ in a smooth convex closed surface $\Sigma$.
 Show that 
 \[\length\gamma\le 2\cdot\pi;\]
that is, the length of $\gamma$ can not exceed the length of the unit circle in the plane.  
\end{thm}


\begin{thm}{Exercise}
Let $\gamma$ be a closed simple geodesic on a smooth convex closed surface $\Sigma$.
Assume $\nu\:\Sigma\to\SS^2$ is a Gasuss map.
Show that the curve $\nu\circ\gamma$ divides the sphere into regions of equal area.
\end{thm}

{

\begin{wrapfigure}{o}{32 mm}
\vskip-0mm
\centering
\includegraphics{mppics/pic-46}
\vskip-0mm
\end{wrapfigure}

\begin{thm}{Exercise}
Let $\Sigma$ be a smooth closed surface with a closed geodesic $\gamma$.
Assume $\gamma$ has exactly 4 self-intersection at the points $a$, $b$, $c$ and $d$ that appear on $\gamma$ in the order $a,a,b,b,c,c,d,d$.
Show that $\Sigma$ can not have positive Gauss curvature.\footnote{Hint: estimate integral of Gauss curvature bounded by a simple geodesic loop.}
\end{thm}

\begin{thm}{Advanced exercise}
Let $\Sigma$ be a smooth regular sphere with positive Gauss curvature and $p\in\Sigma$. 
Suppose $\gamma$ be a closed geodesic that does not pass thru $p$.
Assume $\Sigma\backslash\{p\}$ parametrized by the plane.
Can it happen that in this parametrization,  $\gamma$ look like one of the curves on the diagram?
\begin{figure}[h!]
\vskip-0mm
\centering
\includegraphics{mppics/pic-47}
\vskip-0mm
\end{figure}
Say as much as possible about possible/impossible diagrams of that type.
\end{thm}

}

\section{The remarkable theorem}

Let $\Sigma_1$ and $\Sigma_2$ be two smooth regular surfaces in the Euclidean space.
A map $f\:\Sigma_1\to \Sigma_2$ is called  length-preserving if for any curve $\gamma_1$ in $\Sigma_1$ the curve $\gamma_2=f\circ\gamma_1$ in $\Sigma_2$ has the same length. %???it is sufficient to consider smooth only curves???
If in addition $f$ is smooth and bijective then it is called \emph{intrinsic isometry}. 

A simple example of intrinsic isometry can obtained by warping a plane into a cylinder.
The following exercise produce slightly more interesting example.

\begin{thm}{Exercise}
Suppose $\gamma(t)=(x(t),y(t))$ is a smooth unit-curve in the plane such that $y(t)=a\cdot \cos t$.
Let $\Sigma_\gamma$ be the surface of revolution of $\gamma$ around $x$-axis.
Show that a small open domain in $\Sigma_\gamma$ admits a smooth length-pereserving map to the unit sphere.

Conclude that any round disc $\Delta$ in $\SS^2$ of intrinsic radius smaller than $\tfrac\pi2$ admits a smooth length preserving deformation; that is there is one parameter family of surfaces with boundary $\Delta_t$, such that $\Delta_0=\Delta$ and $\Delta_t$ is not congruent to $\Delta_0$ for any $t\ne0$.\footnote{In fact any disc in $\SS^2$ of intrinsic radius smaller than $\pi$ admits a smooth length preserving deformation. %???REF
}
\end{thm}


\begin{thm}{Theorem}\label{thm:remarkable}
Suppose $f\:\Sigma_1\to \Sigma_2$ is an intrinsic isometry between two smooth regular surfaces in  the Euclidean space; $p_1\in \Sigma_1$ and $p_2=f(p_1)\in \Sigma_1$.
Then 
\[G(p_1)_{\Sigma_1}=G(p_2)_{\Sigma_2};\]
that is, the Gauss curvature of $\Sigma_1$ at $p_1$ is the same as the Gauss curvature of $\Sigma_2$ at $p_2$.
\end{thm}

This theorem was proved by Carl Friedrich Gauss \cite{gauss} who called it \emph{Remarkable theorem} (Theorema Egregium).
The theorem is indeed remarkable because the Gauss curvature is defined as a product of principle curvatures which might be different at these points; however, according to the theorem, their product can not change.

In fact Gauss curvature of the surface at the given point can be found \emph{intrinsically},
by measuring the lengths of curves in the surface.
For example, Gauss curvature $G(p)$ in the following formula for the circumference $c(r)$ of a geodesic circle centered at $p$ in a surface: 
\[c(r)=2\cdot\pi\cdot r-\tfrac\pi3\cdot G(p)\cdot r^3+o(r^3).\]

Note that the theorem implies there is no smooth length-preserving map that sends an open region in the unit sphere to the plane.%
\footnote{There are plenty of non-smooth length-preserving maps from the sphere to the plane; see \cite{petrunin-yashinski} and the references there in.}
It follows since the Gauss curvature of the plane is zero and the unit sphere has Gauss curvature 1. 
In other words, there is no map of a region on Earth without distortion.

\parit{Proof.}
Set $g_1=G(p_1)_{\Sigma_1}$ and $g_2=G(p_2)_{\Sigma_2}$;
we need to show that 
\[g_1=g_2.\eqlbl{eq:g=g}\]

Suppose $\Delta_1$ is a small geodesic triangle in $\Sigma_1$ that contains $p_1$.
Set $\Delta_2=f(\Delta_1)$.
We may assume that the Gauss curvature is almost constant in $\Delta_1$ and $\Delta_2$;
that is, given $\eps>0$, we can assume that 
\[
\begin{aligned}
|G(x_1)_{\Sigma_1}-g_1|&<\eps,
\\
|G(x_2)_{\Sigma_2}-g_2|&<\eps
\end{aligned}
\eqlbl{eq:almost=}\]
for any $x_1\in \Delta_1$ and $x_2\in \Delta_2$.

Since $f$ is length-preserving the triangles $\Delta_2$ is geodesic and
\[\area\Delta_1=\area\Delta_2.\eqlbl{eq:area=}\]
Moreover, triangles $\Delta_1$ and $\Delta_2$ have the same corresponding angles; denote them by $\alpha$, $\beta$ and $\gamma$.

By Gauss--Bonnet formula, we get that 
\[\iint_{\Delta_1}G_{\Sigma_1}=\alpha+\beta+\gamma-\pi=\iint_{\Delta_2}G_{\Sigma_2}.\eqlbl{eq:gauss-int=}\]

By \ref{eq:almost=}, 
\begin{align*}
\left|g_1-\frac1{\area\Delta_1}\cdot\iint_{\Delta_1}G_{\Sigma_1}\right|&<\eps,
\\
\left|g_2-\frac1{\area\Delta_2}\cdot\iint_{\Delta_2}G_{\Sigma_2}\right|&<\eps.
\end{align*}
By \ref{eq:area=} and \ref{eq:gauss-int=},
\[\frac1{\area\Delta_1}\cdot\iint_{\Delta_1}G_{\Sigma_1}
=
\frac1{\area\Delta_2}\cdot\iint_{\Delta_2}G_{\Sigma_2},\]
therefore
\[|g_1-g_2|<2\cdot\eps.\]
Since $\eps>0$ is arbitrary, \ref{eq:g=g} follows.
\qeds


\section{Simple geodesic}

The following theorem provides an interesting application of Gauss--Bonnet formula;
it is proved by Stephan Cohn-Vossen \cite[Satz 9 in][]{convossen}.


\begin{thm}{Theroem}\label{thm:cohn-vossen}
Any open smooth regular surface with positive Gauss curvature has a simple two-sided infinite geodesic.
\end{thm}

\begin{thm}{Lemma}\label{lem:graph}
Suppose $\Sigma$ is an open surface in with positive Gauss curvature in the Euclidean space.
Then there is a convex function $f$ defined on a convex open region of $(x,y)$-plane 
such that $\Sigma$ can be presented as a graph $z=f(x,y)$ in some $(x,y,z)$-coordinate system of the Euclidean space.

Moreover 
\[\iint_\Sigma G\le 2\cdot\pi.\eqlbl{eq:int=<2pi}\]

\end{thm}

\parit{Proof.}
The surface $\Sigma$ is a boundary of an unbounded closed convex set $K$.

Fix $p\in \Sigma$ and consider a sequence of points $x_n$ such that $|x_n-p|\z\to \infty$ as $n\to \infty$.
Set $u_n=\tfrac{x_n-p}{|x_n-p|}$; it is the unit vector in the direction from $p$ to $x_n$.
Since the unit sphere is compact, we can pass to a subsequence of $(x_n)$ such that $u_n$ converges to a unit vector $u$.

Note that for any $q\in \Sigma$, the directions $v_n=\tfrac{x_n-q}{|x_n-q|}$ converge to $u$ as well.
The half-line from $q$ in the direction of $u$ lies in $K$.
Indeed any point on the half-line is a limit of points on the line segments $[q,x_n]$;
since $K$ is closed any of these poins lie in $K$.


Let us choose the $z$-axis in the direction of $u$.
Note that line segments can not lie in $\Sigma$, otherwise its Gauss curvature would vanish.
It follows that any vertical line can intersect $\Sigma$ at most at one point.
That is, $\Sigma$ is a graph of function $z=f(x,y)$.
Since $K$ is convex, the function $f$ is convex and it is defined in a region $\Omega$ which is convex.
The domain $\Omega$ is the projection of $\Sigma$ to the $(x,y)$-plane.
This projection is injective and by inverse function theorem, it maps open sets in $\Sigma$ to open sets in the plane;
hence $\Omega$ is open.

It follows that the outer normal vectors to $\Sigma$ at any point, points to the south hemisphere $\SS^2_-=\set{(x,y,z)\in\SS^2}{z< 0}$.
Therefore the area of the spherical image of $\Sigma$ is at most $\area\SS^2_-= 2\cdot\pi$.
The area of this image is the integral of the Gauss curvature along $\Sigma$.
That is,
\begin{align*}
\iint_{\Sigma}G&=\area[\nu(\Sigma)]\le 
\\
&\le \area\SS^2_-=
\\
&=2\cdot\pi,
\end{align*}
where $\nu(p)$ denotes the outer unit normal vector at $p$.
Hence \ref{eq:int=<2pi} follows.
\qeds

\parit{Proof of \ref{thm:cohn-vossen}.}
Let $\Sigma$ be an open surface in with positive Gauss curvature and $\gamma$ a two-sided infinite geodesic in $\Sigma$.
The following is the key statement in the proof.

\begin{thm}{Claim}
The geodesic $\gamma$ contains at most one simple loop.
\end{thm}

Assume $\gamma$ has a simple loop $\ell$.
By Lemma \ref{lem:graph}, $\Sigma$ is parameterized by a open convex region $\Omega$ in the plane;
therefore $\ell$ bounds a disc in $\Sigma$; denote it by $\Delta$.
If $\phi$ is the angle at the base of the loop, then by Gauss--Bonnet,
\[\iint_\Delta G=\pi+\phi.\] 
By Lemma \ref{lem:graph}, $\phi<\pi$; that is $\gamma$ has no concave simple loops 

Assume $\gamma$ has two simple loops, say $\ell_1$ and $\ell_2$ that bound discs $\Delta_1$ and $\Delta_2$.
Then the disks $\Delta_1$ and $\Delta_2$ have to overlap,
otherwise the curvature of $\Sigma$ would exceed $2\cdot\pi$.

We may assume that $\Delta_1\not\subset \Delta_2$; the loop $\ell_2$ appears after $\ell_1$ on $\gamma$ and there are no other simple loops between them.
In this case, after going around $\ell_1$ and before closing $\ell_2$, the curve $\gamma$ must enter $\Delta_1$ creating a concave loop.
The latter contradicts the above observation.

If a geodesic $\gamma$ has a self-intersection,
then it contains a simple loop.
From above, there is only one such loop;
it cuts a disk from $\Sigma$ 
and goes around it either clockwise or counterclockwise.
This way we divide all the self-intersecting geodesics 
into two sets which we will call {}\emph{clockwise} and {}\emph{counterclockwise}.

Note that the geodesic $t\mapsto \gamma(t)$ is clockwise 
if and only if the same geodesic traveled backwards
$t\mapsto \gamma(-t)$
is counterclockwise.
By shooting unit-speed geodesics in all directions at a given point $p=\gamma(0)$,
we get a one parameter family of geodesics $\gamma_s$ for $s\in[0,\pi]$ connecting the geodesic $t\mapsto \gamma(t)$ with
the $t\mapsto \gamma(-t)$; that is, $\gamma_0(t)\z=\gamma(t)$ and $\gamma_\pi(t)=\gamma(-t)$. 
It follows that there are geodesics 
which aren't clockwise nor counterclockwise.
Those geodesics have no self-intersections.\qeds

\begin{thm}{Exercise}
Suppose that $f\:\RR^2\to\RR$ is a $\tfrac{\sqrt{3}}2$-Lipshitz smooth convex function.
Show that any geodesic in the surface defined by the graph $z=f(x,y)$ has no self-intersections.
\end{thm}




