\chapter{Surfaces}

\section*{General definition}

Few times we will need the following general definition.

A path connected subset $\Sigma$ in a metric space is called \emph{surface} (more precisely \emph{embedded surface without boundary}) 
if any point of $p\in \Sigma$ admits a neighborhood $W$ in $\Sigma$ which is \emph{homeomorphic} to an open subset in the Euclidean plane;
that is, if there is an injective continuous map $U\to W$ from an open set $U\subset \RR^2$ such that its invese $W\to U$ is also continuous.

However, as well as in the case of curves we will be mostly interested in smooth surfaces in the Euclidean space describe in the following section.

\section*{Smooth surfaces}

Recall that a function $f$ of two variables $x$ and $y$ is called \emph{smooth} if all its partial derivatives $\frac{\partial^{m+n}}{\partial x^m\partial y^n}f$ are defined and are continuous in the domain of definition of $f$. 

A path connected set $\Sigma \subset \mathbb{R}^3$ is called a \emph{smooth surface} (or more precisely \emph{smooth regular embedded surface}) if it can be described locally as a graph of a smooth function in an appropriate coordinate system.\label{page:def-smooth-surface}

More precisely, for any point $p\in \Sigma$ one can choose a coordinate system $(x,y,z)$ and a neighborhood $U\ni p$ such that
the intersection $W=U\cap \Sigma$ is formed by a graph $z=f(x,y)$ of a smooth function $f$ defined in an open domain of the $(x,y)$-plane.

\parbf{Examples.}
The simplest example of a surface is the $(x,y)$-plane 
\[\Pi=\set{(x,y,z)\in\RR^3}{z=0}.\]
The plane $\Pi$ is a surface since
it can be described as the graph of the function $f(x,y)=0$.

All other planes are surfaces as well since one can choose a coordinate system so that it becomes the $(x,y)$-plane.
We can also present a plane as a graph of a linear function 
$f(x,y)=a\cdot x+b\cdot y+c$ for some constants $a$, $b$ and $c$
(assuming the plane is not perpendicular to the $(x,y)$-plane).

A more interesting example is the unit sphere 
\[\SS^2=\set{(x,y,z)\in\RR^3}{x^2+y^2+z^2=1}.\]
This set is not the graph of any function,
but $\SS^2$ is locally a graph;
in fact it can be covered by 6 graphs:
\begin{align*}
z&=f_\pm(x,y)=\pm \sqrt{1-x^2-y^2},
\\
y&=g_\pm(x,z)=\pm \sqrt{1-x^2-z^2},
\\
x&=h_\pm(y,z)=\pm \sqrt{1-y^2-z^2};
\end{align*}
each function $f_\pm,g_\pm,h_\pm$ is defined in an open unit disc.
That is, $\SS^2$ is a smooth surface.

\parbf{More conventions.}
If the surface $\Sigma$ is formed by a closed set, then it is called \emph{complete}.
For example, paraboloids 
\[z=x^2+y^2,\quad\quad z=x^2-y^2\]
or sphere 
\[x^2+y^2+z^2=1\]
are complete surfaces, while the
open disc in a plane 
\[\set{(x,y,z)\in\RR^3}{x^2+y^2<1, z=0}\]
is a surface which is not complete.

If moreover $\Sigma$ is a compact set, then it is called \emph{closed surface} (the term \emph{closed set} is not directly relevant).

If a complete surface $\Sigma$ is noncompact, then it is called  \emph{open surface} (again the term \emph{open set} is not relevant).

For example, paraboloids 
are open surfaces, 
and sphere is closed.

A closed subset in a surface that is bounded by one or more smooth %???smooth???
curves is called \emph{surface with boundary}; in this case the collection of curves is called the \emph{boundary line} of the surface.
When we say \emph{surface} we usually mean a surface without boundary;
we may use the term \emph{surface with possibly nonempty boundary} if we need to talk about surfaces with and without boundary.

\section*{Local parametrizations}

Let $U$ be an open domain in $\RR^2$ and $f\:U\to \RR^3$ be a smooth map.
We say that $f$ is regular if its Jacobian has maximal rank;
that is, if the vectors $\tfrac{\partial f}{\partial u}$ and $\tfrac{\partial f}{\partial v}$ are linearly independent at any $(u,v)\in U$;
equivalently $\tfrac{\partial f}{\partial u}\times\tfrac{\partial f}{\partial v}\ne 0$, where $\times$ denotes the vector product.

\begin{thm}{Proposition}\label{prop:graph-chart}
If $f\:U\to \RR^3$ is a smooth regular embedding of an open connected set $U\subset \RR^2$, then it image $\Sigma=f(U)$ is a smooth surface.
\end{thm}

\parit{Proof.}
Let $f(u,v)=(f_1(u,v),f_2(u,v),f_3(u,v))$.
Since $f$ is regular the Jacobian matrix
\[
\renewcommand\arraystretch{1.3}
\begin{pmatrix}
\tfrac{\partial f_1}{\partial u}&\tfrac{\partial f_1}{\partial v}\\
\tfrac{\partial f_2}{\partial u}&\tfrac{\partial f_2}{\partial v}\\
\tfrac{\partial f_3}{\partial u}&\tfrac{\partial f_3}{\partial v}
\end{pmatrix}
\]
has rank two.

Fix a point $p\in \Sigma$; by shifting the coordinate system we may assume that $p$ is the origin.
Permuting the coordinates $x,y,z$ if necessary, we may assume that 
the matrix 
\[
\renewcommand\arraystretch{1.3}
\begin{pmatrix}
\tfrac{\partial f_1}{\partial u}&\tfrac{\partial f_1}{\partial v}\\
\tfrac{\partial f_2}{\partial u}&\tfrac{\partial f_2}{\partial v}
\end{pmatrix}
\]
is invertible.
Let $\bar f\:U\to\RR^2$ be the projection of $f$ to the $(x,y)$-coordinate plane;
that is, $\bar f(u,v)=(f_1(u,v),f_2(u,v))$.
Note that the $2\times2$-matrix above is the Jacobian matrix of $\bar f$.

The inverse function theorem implies that there is a smooth regular function $h$ defined on an open set $W\ni 0$ in the $(x,y)$-plane
such that $h(0,0)=(0,0)$ and $\bar f\circ h$ is the identity map.

The graph $\Gamma$ described by $z=f_3\circ h(x,y)$ is a subset of $\Sigma$.
Indeed, if $(u,v)\z=h(x,y)$, then $x=f_1(u,v)$ and $y=f_2(u,v)$.
Therefore the identity $z=f_3\circ h(x,y)$ can be rewritten as $(x,y,z)=f(u,v)$.

Clearly $\Gamma$ is an open subset in $\Sigma$;
that is, $\Gamma$ a neighborhood of $p$ in $\Sigma$ that can be described as a graph of a smooth function $f_3\circ h\:W\to\RR$.
Since $p$ is arbitrary, we get that $\Sigma$ is a surface.
\qeds

If $f$ and $\Sigma$ as in the proposition, then we say that $f$ is a \emph{parametrization} of the surface $\Sigma$. 

Not all the smooth surfaces can be described by such a parametrization;
for example the sphere $\SS^2$ does not.
But any smooth surface $\Sigma$ admits a local parametrization; that is, any point $p\in\Sigma$ admits an open neighborhood $W\subset \Sigma$ with a smooth regular parametrization~$f$.
In this case any point in $W$ can be described by two parameters, usually denoted by $u$ and $v$, 
which are called \emph{local coordinates} at $p$.
The map $f$ is called a \emph{chart} of $\Sigma$.

If $W$ is a graph $z=h(x,y)$ then the map $f\:(u,v)\mapsto (u,v,h(u,v))$ is a chart.
Indeed, $f$ has an inverse $(u,v,h(u,v))\mapsto (u,v)$ which is continuous;
that is, $f$ is an embedding.
Further,
$\tfrac{\partial f}{\partial u}=(1,0,\tfrac{\partial h}{\partial u})$ and $\tfrac{\partial f}{\partial v}=(0,1,\tfrac{\partial h}{\partial v})$, whence $\tfrac{\partial f}{\partial u}$ and $\tfrac{\partial f}{\partial v}$ are linearly independent.

Note that from \ref{prop:graph-chart}, we obtain the following corollary.

\begin{thm}{Corollary}
A path connected set $\Sigma\subset \RR^3$ is a smooth regular surface if at any point $p\in \Sigma$ it has a local parametrization by a smooth regular map.
\end{thm}


\begin{thm}{Exercise}\label{ex:inversion}
Consider the following map 
\[f(u,v)=(\tfrac{2\cdot u}{1+u^2+v^2},\tfrac{2\cdot v}{1+u^2+v^2},\tfrac{2}{1+u^2+v^2}).\]
Show that $f$ is a chart of the unit sphere centered at $(0,0,1)$; describe the image of $f$.
\end{thm}

The map 
\[(u,v,1)\mapsto (\tfrac{2\cdot u}{1+u^2+v^2},\tfrac{2\cdot v}{1+u^2+v^2},\tfrac{2}{1+u^2+v^2})\]
is called \emph{stereographic projection}. 
Note that the point $(u,v,1)$ and its image $(\tfrac{2\cdot u}{1+u^2+v^2},\tfrac{2\cdot v}{1+u^2+v^2},\tfrac{2}{1+u^2+v^2})$ lie on one half-line starting at the origin.

Let $\gamma(t)=(x(t),y(t)$ be a plane curve.
Recall that the image of the map 
\[(t,\theta)\mapsto (x(t), y(t)\cdot\cos\theta,y(t)\cdot\sin\theta)\] 
is called \emph{surface of revolution} of the curve $\gamma$ around $x$-axis.

\begin{thm}{Exercise}\label{ex:revolution}
Assume $\gamma$ is a closed simple smooth regular plane curve that does not intersect $x$-axis.
Show that surface of revolution of $\gamma$ around $x$-axis is a smooth regular surface.
\end{thm}


\section*{Golbal parametrizations} 
A surface can be described by an embedding from a known surface to the space.
For example the ellipsoid
\[\Sigma_{a,b,c}=\set{(x,y,z)\in\RR^3}{\tfrac{x^2}{a^2}+\tfrac{y^2}{b^2}+\tfrac{z^2}{c^2}=1}\]
for some positive numbers $a,b,c$ can be defined as the image of the map $s\:\SS^2\to\RR^3$, defined as the restriction of the map $(x,y,z)\z\mapsto (a\cdot x, b\cdot y,c\cdot z)$ to the unit sphere $\SS^2$.

For a surface $\Sigma$, a map $s: \Sigma \to \RR ^3$ is called a 
\emph{smooth parametrized surface} if for for any chart $f\:U\to \Sigma$ 
the composition $s\circ f$ is smooth and regular;
that is all partial derivatives $\frac{\partial^{m+n}}{\partial u^m\partial v^n}(s\circ \tilde{f})$ exist and are continuous in the domain of definition and the following two vectors 
$\frac{\partial}{\partial u}(s\circ \tilde{f})$ $\frac{\partial}{\partial v}(s\circ \tilde{f})$ are linearly independent.

Evidently the parametric definition includes the embedded surfaces defined previously --- as the domain of parameters we can take the surface itself and the identity map as $s$,
but parametrized surfaces are more general, in particular they  might  have self-intersections.

If $\Sigma$ is a known surface for example a sphere or a plane, the paramtrized surface $s\:\Sigma\to\RR^3$ might be called by the same name.
For example any embedding $s\:\SS^2\to\RR^3$ might be called topological sphere
and if $s$ is smooth it might be called smooth sphere.
Similarly an embedding $s\:\RR^2\to\RR^3$ might be called topological plane
and if $s$ is smooth it might be called smooth plane.

\section*{Implicitly defined surfaces}

\begin{thm}{Proposition}
Let $f\:\RR^3\to \RR$ be a smooth function; 
that is, all its partial derivatives defined in its domain of definition.
Suppose that $0$ is a regular value of $f$;
that is $\nabla_p f\ne 0$ if $f(p)=0$.
Then any path connected component $\Sigma$ of the set of solutions of the equation $f(x,y,z)=0$ is a surface.
\end{thm}

\parit{Proof.}
Fix $p\in\Sigma$.
Since $\nabla_p f\ne 0$ we have 
\[\tfrac{\partial f}{\partial x}(p)\ne 0,\quad \tfrac{\partial f}{\partial y}(p)\ne 0,\quad \text{or}\quad\tfrac{\partial f}{\partial z}(p)\ne 0.\]
We may assume $\tfrac{\partial f}{\partial z}(p)\ne 0$;
otherwise permute the coordinates $x,y,z$.

The implicit function theorem (\ref{thm:imlicit}) implies that a neighborhood of $p$ in $\Sigma$ is the graph $z=h(x,y)$ of a smooth function $h$ defined on an open domain in $\RR^2$.
It remains to apply the definition of smooth surface (page \pageref{page:def-smooth-surface}).
\qeds

\begin{thm}{Exercise}\label{ex:hyperboloinds}
Describe the set of real numbers $a$
such that the equation
\begin{align*}
x^2+y^2-z^2&=a
\end{align*}
describes a smooth regular surface.
\end{thm}

\section*{Tangent plane}

Let $z=f(x,y)$ be a local graph realization of a surface. 
Assume $p=(x_p,y_p,z_p)$ lies on this graph, so $z_p=f(x_p,y_p)$.
The plane spanned by the vectors $(1,0,(\tfrac{\partial}{\partial x}f)(x_p,y_p))$ and  $(0,1,(\tfrac{\partial}{\partial y}f)(x_p,y_p))$ is called the \emph{tangent plane} of $\Sigma$ at $p$.
The tangent plane to $\Sigma$ at $p$ is usually denoted by $\T_p$ or $\T_p\Sigma$.
Vectors in $\T_p$ are called \emph{tangent vectors} of $\Sigma$ at $p$.

Tangent plane $\T_p$ might be considered as a linear subspace of $\RR^3$ or as an plane passing thru $p$.
In the latter case it can be interpreted as the best approximation of the surface $\Sigma$ by a plane at~$p$.

\begin{thm}{Proposition}
Let $\Sigma$ be a smooth surface.
A vector $w$ is a tangent vector of $\Sigma$ at $p$ if and only if there is a curve $\gamma$ that runs in $\Sigma$ and has $w$ as a velocity vector at $p$.  
\end{thm}

Note the tangent plane to a surface at a given point $p$ does not depend on the local graph representation of the surface;
indeed according to the proposition the tangent plane can be defined as the set of all velocity vectors of smooth parameterized curves that run in $\Sigma$.

\parit{Proof; ``only if'' part.}
We can assume that $\Sigma$ is a graph $z=f(x,y)$; 
otherwise pass to a local presentation of $\Sigma$ around $p$.

Suppose that $(x(t),y(t))$ denotes the projection of $\gamma(t)$ to the $(x,y)$-plane.
Since $\gamma$ runs in $\Sigma$, we have that
\[\gamma(t)=\bigl(x(t),y(t),f(x(t),y(t))\bigr).\]
Therefore 
\begin{align*}
\gamma'&=(x',y',\tfrac{\partial f}{\partial x}(x,y)\cdot x'+\tfrac{\partial f}{\partial y}(x,y)\cdot y')=
\\
&=x'\cdot (1,0,(\tfrac{\partial}{\partial x}f)(x,y))+y'\cdot (0,1,(\tfrac{\partial}{\partial y}f)(x,y));
\end{align*}
That is, $\gamma'(t)\in\T_{\gamma(t)}$ for any $t$.

\parit{``If'' part.}
Without loss of generality we can assume that $p$ is the origin.
Fix a tangent vector 
\[w=a\cdot (1,0,(\tfrac{\partial}{\partial x}f)(x_p,y_p))+b\cdot(0,1,(\tfrac{\partial}{\partial y}f)(x_p,y_p))\] 
and consider the curve $\gamma(t)=(a\cdot t, b\cdot t, f(a\cdot t,b\cdot t)$.
By construction $\gamma$ runs in $\Sigma$ and the direct calculations show that $\gamma'(0)=w$.
\qeds

\begin{thm}{Exercise}\label{ex:tangent-chart}
Assume $f\:U\to\RR^3$ is a smooth regular chart of a surface $\Sigma$ and $p=f(u_0,v_0)$.
Show that the tangent plane of $\Sigma$ at $p$ is spanned by vectors $\tfrac{\partial f}{\partial u}(u_0,v_0)$ and $\tfrac{\partial f}{\partial v}(u_0,v_0)$.
\end{thm}

\begin{thm}{Exercise}\label{ex:tangent-normal}
Let $f:\RR^3\to\RR$ be a smooth function with a regular value $0$ and $\Sigma$ is a surface described as a connected component of the set of solutions $f(x,y,z)=0$.
Show that the tangent plane $\T_p\Sigma$ is perpendicular to the gradient $\nabla_pf$ at any point $p\in\Sigma$.
\end{thm}


\section*{Normal vector and orientation}
A unit vector that is normal to $\T_p$ is usually denoted by $\nu_p$;
it is uniquely defined up to sign.

A surface $\Sigma$ is called \emph{oriented} if it is equipped with a unit normal vector field $\nu$;
that is, a continuous map $p\mapsto \nu_p$ such that $\nu_p\perp\T_p$ and $|\nu_p|=1$ for any $p$;
the choice of the field $\nu$ is called \emph{orientation} on $\Sigma$.
A surface $\Sigma$ is called \emph{orientable} if it can be oriented.
Note that each orientable surface admits two orientations $\nu$ and $-\nu$.

\begin{wrapfigure}{o}{40 mm}
\vskip-0mm
\centering
\includegraphics{mppics/pic-71}
\vskip0mm
\end{wrapfigure}

M\"obius strip shown on the diagram gives an example of nonorientable surface --- there is no choice of normal vector field that is continuous along the middle of the strip, 
when you go around it changes the sign.

Note that each surface is locally orientable.
In fact each chart $f(u,v)$ admits an orientation 
\[\nu=
\frac{\tfrac{\partial f}{\partial u}\times \tfrac{\partial f}{\partial v}}
{\left|\tfrac{\partial f}{\partial u}\times \tfrac{\partial f}{\partial v}\right|}.\]
Indeed the vectors $\tfrac{\partial f}{\partial u}$ and $\tfrac{\partial f}{\partial v}$ are tangent vectors at $p$; 
since they are linearly independent, their vector product is perpendicular to the tangent plane.
Therefore $\nu(u,v)$ is a unit normal vector at $f(u,v)$;
evidently $(u,v)\mapsto \nu(u,v)$ is a continuous map. 

\begin{thm}{Exercise}
Let $f:\RR^3\to\RR$ be a smooth function with a regular value $0$ and $\Sigma$ is a surface described as a connected component of the set of solutions $f(x,y,z)=0$.
Show that $\Sigma$ is orientable.
\end{thm}

\parit{Hint.} Show that $\nu=\tfrac{\nabla f}{|\nabla f|}$ defines a unit normal field on $\Sigma$.


In fact any complete smooth surface cuts the space into two connected components.
Therefore one could choose an orientation on any complete surface by taking normal vector at each point that points into one of these components. %???

\section*{Tangent-normal coordinates} 

Fix a point $p$ in a smooth surface $\Sigma$.
Consider a coordinate system $(x,y,z)$ with origin at $p$ such that the $(x,y)$-plane coincides with $\T_p$.
The same argument as in \ref{prop:graph-chart} shows that
we can present $\Sigma$ locally around $p$ as a graph of a function $f$. %???
Note that $f$ satisfies the following additional properties:
\begin{align*}
f(0,0)&=0,
&
(\tfrac{\partial}{\partial x}f)(0,0)&=0,
&
(\tfrac{\partial}{\partial y}f)(0,0)&=0.
\end{align*}
The first equality holds since $p=(0,0,0)$ lies on the graph and the last two equalities mean that the tangent plane at $p$ is horizontal.
 
Consider the Hessian matrix 
\[M_p=\begin{pmatrix}
   \ell
   &m
   \\
   m
   &n
  \end{pmatrix},
\]
where 
\begin{align*}
\ell&=(\tfrac{\partial^2}{\partial x^2}f)(0,0),
\\
m&=(\tfrac{\partial^2}{\partial x\partial y}f)(0,0)=(\tfrac{\partial^2}{\partial y\partial x}f)(0,0),
\\
n&=(\tfrac{\partial^2}{\partial y^2}f)(0,0).
\end{align*}

The components of the matrix describe the surface at up to the second order.
In fact the paraboloid
\[z=\tfrac12(\ell\cdot x^2+2\cdot m\cdot x\cdot y+n\cdot y^2)\]
gives the best approximation of the surface at $p$.

The second directional derivative $(D_wD_vf)(0,0)$ for two vectors $v,w$ in the $(x,y)$-plane is called \emph{second fundamental form}.
It is usually denoted by $\II_p(w,v)$; it takes two tangent vector at the point and spits a real number.
The second fundamental form can be written in terms of the Hessian matrix.
Indeed if $w=(a,b)$ and $v=(c,d)$, then $D_w=a\cdot \tfrac\partial{\partial x}+b\cdot \tfrac\partial{\partial y}$ and $D_v=c\cdot \tfrac\partial{\partial x}+d\cdot \tfrac\partial{\partial y}$.
Therefore 
\begin{align*}
(D_wD_vf)(0,0)&=a\cdot c\cdot \ell+(a\cdot d+ b\cdot c)\cdot m +b\cdot d\cdot n=
\\
&=v\cdot M_p\cdot w^\top.
\end{align*}

\section*{Principle curvatures}

Note that tangent-normal coordinates give an almost canonical coordinate system in a neighborhood of $p$;
it is unique up to a rotation of  the $(x,y)$-plane and switching the sign of the $z$-coordinate.
Rotating the $(x,y)$-plane is equivalent too changing its basis which results in the rewriting   
the Hessian matrix in the new basis.

Since Hessian matrix is symmetric, it is diagonalizable by orthogonal matrices. %???
That is, we can assume that $m=0$. %???+lin algebra material
Then the diagonal elements are called \emph{principle curvatures} of $\Sigma$ at $p$;
they are uniquely defined up to sign;
they are denoted as $k_1(p)$ and $k_2(p)$ or $k_1(p)_\Sigma$ and $k_2(p)_\Sigma$ if we need to emphasize that these are the curvatures of the surface $\Sigma$.
We will always assume that $k_1\le k_2$.

The principle curvatures can be also defined as the eigenvalues of $M_p$;
the eigendirections of $M_p$ are called \emph{principle directions} of $\Sigma$ at~$p$.

Assume we choose the coordinates in the $(x,y)$-plane so that the Hessian matrix is diagonalized, we can assume that
\[M_p=\begin{pmatrix}
   k_1(p)
   &0
   \\
   0
   &k_2(p)
  \end{pmatrix}.
\]
In this case the second directional derivative $(D^2_wf)(0,0)$
for a vector $w=(a,b)$ in the $(x,y)$-plane can be written as 
\[
(D^2_wf)(0,0)=a^2\cdot k_1(p) +b^2\cdot k_2(p).
\]
If $w$ is unit, then the second directional derivative $D^2_wf(0,0)$ can be intepreted as the signed curvature of the curve formed by the intersection of $\Sigma$ with the plane thru $p$ spanned by $\nu_p$ and $w$.
By that reason $D^2_wf(0,0)$ is called \emph{normal curvature} in the direction $w$;
it is denoted by $k_w(p)$ or $k_w(p)_\Sigma$ if we need to emphasize that .
Since $|w|=1$, we have $a^2+b^2=1$.
Therefore we get the following observation.

\begin{thm}{Observation}\label{obs:k1-k2}
For any point $p$ on an oriented smooth surface $\Sigma$,
the principle curvatures $k_1(p)$ and $k_2(p)$ are correspondingly minimum and maximum of the normal curvatures at $p$.
\end{thm}

A smooth regular curve on a surface $\Sigma$ that always runs in the principle directions is called \emph{line of curvature} of $\Sigma$.  

\begin{thm}{Exercise}\label{ex:line-of-curvature}
Assume that a smooth surface $\Sigma$ is mirror symmetric with respect to  a plane $\Pi$.
Suppose that $\Sigma$ and $\Pi$ intersect along a curve $\gamma$.
Show that $\gamma$ is a line of curvature of $\Sigma$.
\end{thm}


\section*{Gauss and mean curvatures}

Fix an oriented smooth surface $\Sigma$ and a point $p\in\Sigma$.
The product 
\[K(p)=k_1(p)\cdot k_2(p)\]
is called Gauss curvature at $p$.
We may denote it by $K(p)_\Sigma$ if we need to emphasize that this is curvature of $\Sigma$.
The Gauss curvature can be also interpreted as determinant of the Hessian matrix $M_p$.

The sum 
\[H(p)=k_1(p)+ k_2(p)\]
is called mean curvature at $p$.
We may denote it by $H(p)_\Sigma$ if we need to emphasize that this is curvature of $\Sigma$.
The mean curvature can be also interpreted as trace of the Hessian matrix $M_p$.

Note that the Gauss curvature depends only on $\Sigma$ and $p$,
and not on the choice of the coordinate system.
The same is true up to sign for the principle curvatures and the mean curvature. 

\begin{thm}{Exercise}\label{ex:gauss+orientable}
Show that any surface with positive Gauss curvature is orientable. 
\end{thm}


\section*{Supporting surfaces}

Assume two oriented surfaces $\Sigma_1$ and $\Sigma_2$ have a common point $p$.
If there is a neighborhood $U$ of $p$ such that $\Sigma_1\cap U$ lies on one side from $\Sigma_2$ in $U$, then we say that $\Sigma_2$ \emph{locally supports} $\Sigma_1$ at $p$.

Note that in this case $\T_p\Sigma_1=\T_p\Sigma_2$; that is, the tangent planes of $\Sigma_1$ and $\Sigma_2$ at $p$ coincide.
Therefore we can assume that $\Sigma_1$ and $\Sigma_2$ are cooriented at $p$;
that is, they have common unit normal vector at $p$.
If not we can revert the orientation of one of the surfaces.

If $\Sigma_1$ and $\Sigma_2$ are cooriented at $p$,
then we can say that $\Sigma_1$ supports $\Sigma_2$ from \emph{inside} or from \emph{outside},
assuming that the normal vector points \emph{inside} the domain bounded by surface $\Sigma_2$ in $U$.

More precisely, we can use for $\Sigma_1$ and $\Sigma_2$ one tangent-normal coordinate system at $p$,
assuming that the axis $z$ points in the direction of the unit normal vector $\nu_p$ to both surfaces.
This way we write $\Sigma_1$ and $\Sigma_2$ locally as graphs: $z=f_1(x,y)$  and $z=f_2(x,y)$ correspondingly.
Then $\Sigma_1$ supports $\Sigma_2$ from inside (from outside)  if $f_1(x,y)\ge f_2(x,y)$ (correspondingly $f_1(x,y)\le f_2(x,y)$) for $(x,y)$ in a sufficiently small neighborhood of the origin.

\begin{thm}{Proposition}\label{prop:surf-support}
Let $\Sigma_1$ and $\Sigma_2$ be oriented surfaces.
Assume $\Sigma_1$ supports $\Sigma_2$ from inside at the point $p$.
Then $k_1(p)_{\Sigma_1}\ge k_1(p)_{\Sigma_2}$ and $k_2(p)_{\Sigma_1}\ge k_2(p)_{\Sigma_2}$.
\end{thm}

\parit{Proof.} We can assume that $\Sigma_1$ and $\Sigma_2$ are graphs $z=f_1(x,y)$  and $z=f_2(x,y)$ in a common tangent-normal coordinates at $p$, so we have $f_1\ge f_2$.

Fix a unit vector $w\in \T_p\Sigma_1=T_p\Sigma_2$.
Consider the plane $\Pi$ passing thru $p$ and spanned by the normal vector $\nu_p$ and $w$.
Let $\gamma_1$ and $\gamma_2$ be the curves of intersection of $\Sigma_1$ and $\Sigma_2$ with $\Pi$.
Note that the curve $\gamma_1$ supports the curve $\gamma_2$ and therefore we have the following inequality for the normal curvatures of $\Sigma_1$ and $\Sigma_2$ at $p$ in the direction of $w$:
\[k_w(p)_{\Sigma_1}\ge k_w(p)_{\Sigma_2}.\eqlbl{kw>=kw}\]

According to \ref{obs:k1-k2},
\[k_1(p)_{\Sigma_i}=\min\set{k_w(p)_{\Sigma_i}}{w\in\T_p, |w|=1}\]
for $i=1,2$.
Choose $w$ so that $k_1(p)_{\Sigma_1}=k_w(p)_{\Sigma_1}$.
Then 
\begin{align*}
k_1(p)_{\Sigma_1}&=k_w(p)_{\Sigma_1}\ge
\\
&\ge k_w(p)_{\Sigma_2}\ge
\\
&\ge\min\set{k_w(p)_{\Sigma_2}}{}=
\\
&=k_1(p)_{\Sigma_2};
\end{align*}
that is, $k_1(p)_{\Sigma_1}\ge k_1(p)_{\Sigma_2}$.

Similarly, by \ref{obs:k1-k2}, we have that
\[k_2(p)_{\Sigma_i}=\max\set{k_w(p)_{\Sigma_i}}{}.\]
Fix $w$ so that $k_2(p)_{\Sigma_2}=k_w(p)_{\Sigma_2}$.
Then 
\begin{align*}
k_2(p)_{\Sigma_2}&=k_w(p)_{\Sigma_2}\le
\\
&\le k_w(p)_{\Sigma_1}\le
\\
&\le\max\set{k_w(p)_{\Sigma_1}}{}=
\\
&=k_2(p)_{\Sigma_1};
\end{align*}
that is, $k_2(p)_{\Sigma_1}\ge k_2(p)_{\Sigma_2}$.
\qeds

\begin{thm}{Corollary}\label{cor:surf-support}
Let $\Sigma_1$ and $\Sigma_2$ be oriented surfaces.
Assume $\Sigma_1$ supports $\Sigma_2$ from inside at the point $p$.
Then
\begin{enumerate}[(a)]
\item\label{cor:surf-support:mean} $H(p)_{\Sigma_1}\ge H(p)_{\Sigma_2}$;
\item\label{cor:surf-support:gauss} If $k_1(p)_{\Sigma_2}\ge 0$, then $K(p)_{\Sigma_1}\ge K(p)_{\Sigma_2}$.
\end{enumerate}
 
\end{thm}

\parit{Proof.} Part \ref{cor:surf-support:mean} follow from \ref{prop:surf-support} since 
$H(p)_{\Sigma_i}=k_1(p)_{\Sigma_i}+k_2(p)_{\Sigma_i}$.

By \ref{prop:surf-support}, we get that $k_1(p)_{\Sigma_1}\ge k_1(p)_{\Sigma_2}$ and $k_2(p)_{\Sigma_2}\ge k_2(p)_{\Sigma_2}$.
Since $k_2\ge k_1$, we get that all the principle curvatures 
$k_1(p)_{\Sigma_1}$, $k_1(p)_{\Sigma_1}$ and $k_2(p)_{\Sigma_1}$ and $k_2(p)_{\Sigma_2}$ are nonnegative.
Whence
\begin{align*}
K(p)_{\Sigma_1}&=k_1(p)_{\Sigma_1}\cdot k_2(p)_{\Sigma_1}\ge 
\\
&\ge k_1(p)_{\Sigma_2}\cdot k_2(p)_{\Sigma_2}=
\\
&=K(p)_{\Sigma_2}.
\end{align*}
\qedsf




\begin{thm}{Exercise}\label{ex:positive-gauss}
Show that any closed surface has a point with positive Gauss curvature.
\end{thm}

\begin{thm}{Exercise}
Show that there is no closed surface with vanishing mean curvature.
\end{thm}

\parit{Hint.}
Consider the minimal sphere that encloses the surface.

\begin{thm}{Exercise}
Assume a closed surface $\Sigma$ surrounds a unit disc.
Show that Gauss curvature of $\Sigma$ is at most 1 at some point. 

Try to prove the same assuming that $\Sigma$ surrounds a unit circle.
\end{thm}

\parit{Hint.}
Look for a supporting spherical dome with the unit circle as the boundary.

