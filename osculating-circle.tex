\chapter{Plane curves}


\section{Unit-speed curves}

Any regular smooth curve can be parametrized by its length.
The obtained curve $\alpha$ (that is the constructed reparametrization of the given curve) has unit speed; 
that is, $|\alpha'(t)|=1$ for any $t$.
A curve with such parametrization is called \emph{unit-speed} curve
or a curve with a \emph{natural parametrization}. 

It is straightforward to show any smooth regular curve remains smooth (and regular) if equipped with a natural parametrization; 
here smooth means that all derivatives $\alpha^{(n)}(t)$ are defined for all values of $t$ in the domain of definition and any $n$.

\begin{thm}{Proposition}\label{prop:a'-pertp-a''}
Assume $\alpha\:[a,b]\to\RR^2$ be a smooth unit-speed curve.
Then 
\[\alpha'(t)\perp \alpha''(t)\]
for any $t$.
\end{thm}

The scalar product (also known as dot product) of two vectors $v$ and $w$ will be denoted by $\langle v,w\rangle$.
Recall that for derivative of scalar product the product rule holds;
that is if $v=v(t)$ and $w=w(t)$ are smooth vector-valued functions of real argument $t$, then
\[\langle v,w\rangle'=\langle v',w\rangle+\langle v,w'\rangle.\]

\parit{Proof.}
Since $|\alpha'(t)|=1$, we have
\[\langle\alpha'(t),\alpha'(t)\rangle=1.\]
Taking derivative of both sides we get
\[2\cdot\langle\alpha''(t),\alpha'(t)\rangle=0,\]
hence the result.
\qeds

\section{Signed curvature}

Given a vector $v\in \RR^2$ denote by $i\cdot v$ the vector obtained from $v$ by the counterclockwise rotation by $\tfrac\pi2$.
(The ``multiplication'' by $i$ agrees with the miultiplication by imaginary unit if one use  complex coordinates on the plane $z=x+i\cdot y$.)

Assume $\alpha\:[a,b]\to\RR^2$ be a smooth unit-speed curve.
Recall that curvature of $\alpha$ at $t$ can be defined as $|\alpha''(t)|$.

The \emph{signed curvature} $\kappa(t)$ is uniquely defined by
the identity 
\[\alpha''(t)=\kappa(t)\cdot i\cdot \alpha'(t).\]
Note that by Proposition~\ref{prop:a'-pertp-a''} this equation has a solution.
Since $|\alpha'(t)|=1$ we have that $|\kappa(t)|=|\alpha''(t)|$ for any $t$.

The signed curvature measures how fast the direction $\tau(t)=\alpha'(t)$ rotates;
if is positive if it turns left and negative if it turns right;
if the curve goes straight then its curvature vanish.

\section{Osculating circline}

It is straightforward to prove the following statement.

\begin{thm}{Proposition}\label{prop:circline}
Given a point $p$,
a unit vector $u$ 
and a real number $\kappa$ there is unique smooth unit-speed curve $\gamma\:\RR\to\RR^2$ 
that starts at $p$ in the direction of $u$ and has constant signed curvature $\kappa$.

Moreover if $\kappa=0$, then $\gamma$ is a line $\gamma=p+t\cdot u$
and if $\kappa\ne 0$, then $\gamma$ runs around a circle of radius $\tfrac1{|\kappa|}$ with center at $p+\tfrac i\kappa\cdot u$. 
\end{thm}

Further we will use term \emph{circline} for \emph{circle or line}.

{

\begin{wrapfigure}{r}{30 mm}
\vskip-4mm
\centering
\includegraphics{mppics/pic-21}
\vskip0mm
\end{wrapfigure}

\begin{thm}{Definition}
Let $\alpha$ be a smooth unit-speed plane curve;
denote by $\kappa(t)$ the signed curvature of $\alpha$ at $t$.

A unit-speed plane curve $\gamma$ of constant signed curvature $\kappa(t_0)$ that starts at $\alpha(t_0)$ and runs in the direction $\alpha'(t_0)$ is called \emph{osculating circline} of $\alpha$ at $t_0$.
\end{thm}

}

The center and radius of the osculating circle at a given point are called \emph{center of curvature} and \emph{radius of curvature} of the curve at that point.

\section
{Spiral}
\label{spiral}

The following problem states that 
if you drive on the plane and turn the steering wheel to the right all the time,
then you will not be able to come back to the same place.
This theorem was proved by Peter Tait \cite[see][]{tait}
and later rediscovered by Adolf Kneser \cite[see][]{kneser}.

\begin{thm}{Theorem}
Assume $\gamma$ is a smooth regular plane curve with strictly monotonic curvature. 
Then $\gamma$ has no self-intersections.
\end{thm}%???rewrite as via nested osculating circles

The proof is based on observation that the osculating circles of $\gamma$ are nested.
The same statement holds for signed curvature; the proof requires only minor modifications.

\begin{wrapfigure}{o}{25 mm}
\begin{lpic}[t(-4 mm),b(-2 mm),r(0 mm),l(0 mm)]{pics/kneser-log(1)}
\end{lpic}
\end{wrapfigure}



\parit{Proof.}
Without loss of generality we may assume that the curve $\gamma$ is parametrized by its length and its
curvature $\kappa(t)$ decreases and stays positive.

Let $z(t)$ be curvature center
and $r(t)=\tfrac1{\kappa(t)}$ is radius of curvature of $\gamma$ at $t$.
Note that 
\begin{align*}
z(t)&=\gamma(t)+r(t)\cdot i\cdot \gamma'(t).
\intertext{Therefore}
z'(t)&=\gamma'(t)+r'(t)\cdot i\cdot \gamma'(t)+r(t)\cdot i\cdot \gamma''(t)=
\\
&=\gamma'(t)+r'(t)\cdot i\cdot \gamma'(t)+r(t)\cdot i\cdot \kappa(t)\cdot i\gamma'(t)=
\\
&=\gamma'(t)+r'(t)\cdot i\cdot \gamma'(t)-\gamma'(t)=
\\
&=r'(t)\cdot i\cdot \gamma'(t).
\end{align*}
In particular $|z'(t)|= r'(t)$ and $z'(t)\perp\gamma'(t)$.

Note that the curve $z(t)$ does not have straight arcs;
therefore
\[
\begin{aligned}
|z(t_1)-z(t_0)|&<\int_{t_0}^{t_1}|z'(t)|\cdot dt=
\\
&=\int_{t_0}^{t_1}r'(t)\cdot dt=
\\
&=r(t_1)-r(t_0).
\end{aligned}
\leqno({*})
\]

By $({*})$, the osculating circle at $t_0$ lies inside of the osculating circle at $t_1$ without touching it.
In particular, $\gamma(t_1)\ne \gamma(t_0)$ if $t_1>t_0$.\qeds

The osculating circles of the curve give a peculiar decomposition of an annulus into circles; it has the following property: if a smooth function is constant on each osculating circle it must be constant in the annulus \cite[see][Lecture 10]{fuchs-tabachnikov}.

\begin{thm}{Exercise}
Show that a 3-dimensional analog of the theorem does not hold.
That is, there are self-intersecting smooth regular space curves with strictly monotonic curvature.
\end{thm}

\begin{thm}{Exercise}\label{ex:double-tangent}
Assume that $\gamma$ is a smooth regular plane curve with strictly monotonic curvature.
\begin{enumerate}[(a)]
\item\label{ex:double-tangent:a}Show that no line can be tangent to $\gamma$ at two distinct points.
\item Show that no circle can be tangent to $\gamma$ at three distinct points. 
\end{enumerate}
\end{thm} %???monotone or monotonic???

\begin{wrapfigure}{r}{35 mm}
\vskip-0mm
\centering
\includegraphics{mppics/pic-25}
\vskip0mm
\end{wrapfigure}

Note that part (\ref{ex:double-tangent:a}) does not hold for smooth regular plane curve with strictly monotonic \emph{signed} curvature; an example is shown on the diagram.

Note that if the curve $\gamma(t)$ is defined for $t\in[0,\infty)$ and its curvature converges to $\infty$ as $t\to \infty$, 
then the problem implies the convergence of $\gamma(t)$ as $t\to\infty$.
The latter could be considered as a continuous analog of the Leibniz's test for alternating series.



\section{Supporting circlines}

Suppose $\alpha\:[a,b]\to\RR^2$ be a smooth unit-speed plane curve and $t_0\z\in(a,b)$.

A unit-speed circline $\gamma$ supports $\alpha$ at $t_0$ if $\alpha(t_0)$ lies on $\gamma$
and the points $\alpha(t)$ lie on one closed side of $\gamma$ for all values $t$ sufficiently close to $t_0$.

The following claim resembles the first derivative test.

\begin{thm}{Claim}
Suppose a unit-speed circline $\gamma$ supports a smooth unit-speed plane curve $\alpha$ at $t_0$ from right (correspondingly left).
Without loss of generality we can assume that $\gamma(0)=\alpha(t_0)$.

Denote by $\kappa$ the signed curvature of $\gamma$ and by $\kappa(t_0)$ the signed curvature if $\alpha$ at $t_0$. 
Then $\gamma'(0)=\pm\alpha'(t_0)$.
\end{thm}

Otherwise the curve $\alpha$ would cross $\gamma$ transversely and therefore could not stay at the same side for values close to $t_0$.

Reverting the parametrization of $\gamma$ if necessary we may (and further will) assume that 
\[\gamma'(0)=\alpha'(t_0)\]
holds for any supporting circline $\gamma$ to $\alpha$ at $t_0$.
In this case we say that $\gamma$ supports $\alpha$ from rigth (correspondingly left) if $\alpha$ lies on left (correspondingly right) side of $\gamma$.

The following proposition resembles the second derivative test. 

\begin{thm}{Proposition}\label{prop:supporting-circline}
Assume $\gamma$ is a circle that that supports $\alpha$ at $t_0$ from rigth (correspondingly left).  
Then 
\[\kappa(t_0)\ge \kappa
\quad(\text{correspondingly}\quad\kappa(t_0)\le \kappa).
\] 
where $\kappa$ is the signed curvature of $\gamma$ 
and $\kappa(t_0)$ is the signed curvature of $\alpha$ at $t_0$.

A partial converse also holds.
Namely suppose a unit-speed circline $\gamma$ with signed curvature $\kappa$ starts at $\alpha(t_0)$ in the direction $\alpha'(t_0)$.
Then $\gamma$ supports $\alpha$ at $t_0$ from the right (correspondingly left) if 
\[\kappa(t_0)> \kappa
\quad(\text{correspondingly}\quad\kappa(t_0)< \kappa).
\]

\end{thm}

\parit{Proof.}
We prove only case $\kappa>0$.
The 2 remaining cases $\kappa=0$ and $\kappa<0$ can be done essentially same way.

Since $\kappa\ne0$, the curve $\gamma$ is a circle (it can not be a line).
According to Proposition~\ref{prop:circline},
$\gamma$ has radius $\tfrac1\kappa$ and it is centered at 
\[z=\alpha(t_0)+\tfrac i\kappa\cdot \alpha'(t_0).\]
Consider the function 
\[f(t)=|z-\alpha(t)|^2-\tfrac1{\kappa^2}.\]

Note that $f(t)\le0$ (correspondingly $f(t)\ge0$) 
if an only if $\alpha(t)$ lies on the closed left (correspondingly right) side from $\gamma$.
It follow that 
\begin{itemize}
\item if $\gamma$ supports $\alpha$ at $t_0$ from right, 
then
\[f'(t_0)=0\quad\text{and}\quad f''(t_0)\le 0;\]

\item if $\gamma$ supports $\alpha$ at $t_0$ from  left, 
then 
\[f'(t_0)=0\quad\text{and}\quad f''(t_0)\ge 0;\]

\item if 
\[f'(t_0)=0\quad\text{and}\quad f''(t_0)< 0,\]
then $\gamma$ supports $\alpha$ at $t_0$ from  right;

\item if 
\[f'(t_0)=0\quad\text{and}\quad f''(t_0)> 0,\] then $\gamma$ supports $\alpha$ at $t_0$ from  left;
\end{itemize}

Direct calculations show that
\begin{align*}
f(t_0)&=0;
\\
f'(t_0)&=\left.\langle z-\alpha(t),z-\alpha(t) \rangle'\right|_{t=t_0}=
\\
&=-2\cdot \langle \alpha'(t_0),z-\alpha(t_0) \rangle=
\\&=-2\cdot \langle \alpha'(t_0),\tfrac i\kappa \cdot\alpha'(t_0) \rangle=
\\
&=0;
\\
f''(t_0)&=\langle z-\alpha(t),z-\alpha(t) \rangle''|_{t=t_0}=
\\
&=2\cdot\left( \langle \alpha'(t_0),\alpha'(t) \rangle-\langle \alpha''(t_0),z-\alpha(t) \rangle \right)=
\\
&=2\cdot\left(1-\kappa\cdot \frac1{\kappa(t_0)}\right)
\end{align*}
Hence the result.\qeds


\begin{thm}{Exercise}
Assume $\alpha$ is a closed smooth unit-speed plane curve that runs in a unit disk.
Show that there is a point on $\alpha$ with curvature at least $1$.

Give two proofs, one based on DNA inequality \ref{thm:DNA} and the other based on Proposition~\ref{prop:supporting-circline}.
\end{thm}

\begin{thm}{Lemma}\label{lem:lens}
Let $\alpha$ be a smooth regular simple curve that runs from $p$ to $q$.
Assume that $\alpha$ runs on right side (correspondingly left side) of the oriented line $pq$ and only its end points $p$ and $q$ lie on the line.
Then $\alpha$ has a point with positive  (correspondingly negative) curvature.
\end{thm}

\begin{wrapfigure}{r}{35 mm}
\vskip-4mm
\centering
\includegraphics{mppics/pic-22}
\vskip0mm
\end{wrapfigure}

Note that the lemma fails for curves with self-intersections;
the curve $\alpha$ on the diagram has negative curvature everywhere and it lies on the right side of the line $pq$. 

\parit{Proof.}
Choose points $p'$ and $q'$ on $\ell$
so that the points $p', p, q, q'$ appear in the same order on $\ell$.

Consider the smallest disc segment with chord $[p'q']$ that contains $\alpha$.
Note that its arc $\gamma$ supports at a point $s=\alpha(t_0)$.

\begin{wrapfigure}{r}{50 mm}
\centering
\includegraphics{mppics/pic-23}
\bigskip
\includegraphics{mppics/pic-24}
\vskip0mm
\end{wrapfigure}
Note that the $\alpha'(t_0)$ is tangent to $\gamma$ at $s$.
Moreover $\alpha'(t_0)$ points in the direction of $q'$;
that is, if we go along $\gamma$ in the direction  of $\alpha'(t_0)$ then we have to start at $p'$ and end at $q'$.
If the direction is opposite, then the arc of $\alpha$ from $s$ to $q$ would be trapped in the curvelinear triangle $psp'$ bounded by arcs of $\gamma$, $\alpha$ and the line segment $[p'p]$.
But this is impossible since $q$ does not belong to this triangle.

It follows that $\gamma$ supports $\alpha$ at $t_0$ from right.
By Proposition~\ref{prop:supporting-circline}, 
\[\kappa(t_0)\ge \kappa,\]
where $\kappa(t_0)$ is signed curvature of $\alpha$ at $t_0$ and $\kappa$ is the curvature of $\gamma$.
Evidently $\kappa>0$, hence the result.
\qeds

\begin{thm}{Exercise}
Assume that a bounded domain $D$ in the plane is bounded by a closed regular smooth simple plane curve $\alpha$.
Show that $D$ is convex if and only if the signed curvature of $\alpha$ does not change the sign.
\end{thm}









