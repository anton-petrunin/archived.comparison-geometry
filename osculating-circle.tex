\chapter{Osculating circlines}


\section{Acceleration of unit-speed curve}

Any regular smooth curve can be parametrized by its length.
The obtained curve $\alpha$ has unit speed; 
that is, $|\alpha'(t)|=1$ for all $t$.
This is called the \emph{natural parametrization}. 

It is straightforward to show any smooth regular curve remains smooth (and surely regular) if equipped with a natural parametrization; 
here smooth means that all derivatives $\alpha^{(n)}(t)$ are defined for any $n$ and all values of $t$ in the domain of definition of $\alpha$.

The following proposition essentailly states that the acceleration vector is perpendicular to the velocity vector if the speed remains constant.

\begin{thm}{Proposition}\label{prop:a'-pertp-a''}
Assume $\alpha\:[a,b]\to\RR^2$ be a smooth unit-speed curve.
Then 
\[\alpha'(t)\perp \alpha''(t)\]
for any $t$.
\end{thm}

The scalar product (also known as dot product) of two vectors $v$ and $w$ will be denoted by $\langle v,w\rangle$.
Recall that the derivative of a scalar product satisfies the product rule;
that is if $v=v(t)$ and $w=w(t)$ are smooth vector-valued functions of a real parameter $t$, then
\[\langle v,w\rangle'=\langle v',w\rangle+\langle v,w'\rangle.\]

\parit{Proof.}
Since $|\alpha'(t)|=1$, we have
\[\langle\alpha'(t),\alpha'(t)\rangle=1.\]
Differentiating both sides we get
\[2\cdot\langle\alpha''(t),\alpha'(t)\rangle=0,\]
hence the result.
\qeds

\section{Signed curvature}

Given a vector $v\in \RR^2$ denote by $i\cdot v$ the vector obtained from $v$ by the counterclockwise rotation by $\tfrac\pi2$.
(The ``multiplication'' by $i$ agrees with the miultiplication by the imaginary unit if one uses  complex coordinates on the plane $z=x+i\cdot y$.)

Suppose $\alpha\:[a,b]\to\RR^2$ is a smooth unit-speed curve.
Recall that curvature of $\alpha$ at $t$ can be defined as $|\alpha''(t)|$.

The \emph{signed curvature} $\kappa_\alpha(t)$ is uniquely defined by
the identity 
\[\alpha''(t)=\kappa_\alpha(t)\cdot i\cdot \alpha'(t).\]
Note that by Proposition~\ref{prop:a'-pertp-a''} this equation has a solution.
Since $|\alpha'(t)|=1$ we have $|\kappa_\alpha(t)|=|\alpha''(t)|$ for any $t$.

The signed curvature measures how fast the direction $\tau(t)=\alpha'(t)$ rotates;
the signed curvature is positive if $\tau $ turns left and negative if $\tau$ turns right;
if the curve goes straight then its curvature vanishes.

\section{Osculating circline}

It is straightforward to prove the following statement.

\begin{thm}{Proposition}\label{prop:circline}
Given a point $p$,
a unit vector $u$ 
and a real number $\kappa$ there is unique smooth unit-speed curve $\gamma\:\RR\to\RR^2$ 
that starts at $p$ in the direction of $u$ and has constant signed curvature $\kappa$.

Moreover, if $\kappa=0$, then $\gamma$ runs along the line $\gamma=p+t\cdot u$
and if $\kappa\ne 0$, then $\gamma$ runs around the circle of radius $\tfrac1{|\kappa|}$ and center $p+\tfrac i\kappa\cdot u$. 
\end{thm}

Further we will use the term \emph{circline} for \emph{a circle or a line}.

{

\begin{wrapfigure}{r}{30 mm}
\vskip-4mm
\centering
\includegraphics{mppics/pic-21}
\vskip0mm
\end{wrapfigure}

\begin{thm}{Definition}
Let $\alpha$ be a smooth unit-speed plane curve;
denote by $\kappa_\alpha(t)$ the signed curvature of $\alpha$ at $t$.

For $t_0 \in [a,b]$, the unit-speed curve $\gamma$ of constant signed curvature $\kappa_\alpha(t_0)$ that starts at $\alpha(t_0)$ in the direction $\alpha'(t_0)$ is called the \emph{osculating circline} of $\alpha$ at $t_0$.
\end{thm}

}

The center and radius of the osculating circle at a given point are called \emph{center of curvature} and \emph{radius of curvature} of the curve at that point.

\section
{Spiral theorem}
\label{spiral}

The following theorem states that 
if you drive on the plane and turn the steering wheel to the right all the time,
then you will not be able to come back to the same place.
This theorem was proved by Peter Tait \cite[see][]{tait}
and later rediscovered by Adolf Kneser \cite[see][]{kneser}.

\begin{thm}{Theorem}\label{thm:spiral}
Assume $\alpha$ is a smooth regular plane curve with strictly monotonic curvature. 
Then $\alpha$ is simple.
\end{thm}

The same statement also holds for signed curvature; the proof requires only minor modifications.

\begin{thm}{Exercise}
Show that a 3-dimensional analog of the theorem does not hold.
That is, there are self-intersecting smooth regular space curves with strictly monotonic curvature.
\end{thm}


The proof of theorem is based on the following lemma.

\begin{thm}{Lemma}
Assume that $\alpha$ is a smooth regular plane curve with strictly decreasing positive signed curvature. Then the osculating circles of $\alpha$ are nested; that is, if $\gamma_t$ denoted the osculating circle of $\alpha$ at $t$,
then $\gamma_{t_0}$ lies in the open disc bounded by $\gamma_{t_1}$ for any $t_0<t_1$. 
\end{thm}

The osculating circles of the curve $\alpha$ give a peculiar foliation of an annulus by circles; it has the following property: if a smooth function is constant on each osculating circle it must be constant in the annulus \cite[see][Lecture 10]{fuchs-tabachnikov}.
Also note that the curve $\alpha$ is tangent to a circle of the foliation at each of its points. However, it does not run along a circle.



\begin{wrapfigure}{o}{25 mm}
\begin{lpic}[t(-4 mm),b(-2 mm),r(0 mm),l(0 mm)]{pics/kneser-log(1)}
\end{lpic}
\end{wrapfigure}



\parit{Proof.}
Let $z(t)$ be the curvature center
and 
\[r(t)=\tfrac1{\kappa_\alpha(t)}\]
the radius of curvature of $\alpha$ at $t$.
Note that 
\[z(t)=\alpha(t)+r(t)\cdot i\cdot \alpha'(t).\]
Therefore
\begin{align*}
z'(t)&=\alpha'(t)+r'(t)\cdot i\cdot \alpha'(t)+r(t)\cdot i\cdot \alpha''(t)=
\\
&=\alpha'(t)+r'(t)\cdot i\cdot \alpha'(t)+r(t)\cdot i\cdot \kappa_\alpha(t)\cdot i\cdot\alpha'(t)=
\\
&=\alpha'(t)+r'(t)\cdot i\cdot \alpha'(t)-\alpha'(t)=
\\
&=r'(t)\cdot i\cdot \alpha'(t).
\end{align*}
Since $\kappa_\alpha(t)$ is decreasing, $r(t)$ is increasing;
therefore $r'\ge 0$.
It follows that $|z'(t)|= r'(t)$ and $z'(t)\perp\alpha'(t)$.

Note that the curve $z(t)$ does not have straight arcs;
therefore
\[
\begin{aligned}
|z(t_1)-z(t_0)|&<\int_{t_0}^{t_1}|z'(t)|\cdot dt=
\\
&=\int_{t_0}^{t_1}r'(t)\cdot dt=
\\
&=r(t_1)-r(t_0).
\end{aligned}
\leqno({*})
\]

By $({*})$, the osculating circle at $t_0$ lies inside the osculating circle at $t_1$ without touching it.
\qeds

\parit{Proof of \ref{thm:spiral}.}
Note that $\alpha(t)\in \gamma_t$ for any $t$.
Applying the lemma we get
$\alpha(t_1)\ne \alpha(t_0)$ if $t_1\ne t_0$.
Hence the result.\qeds

The lemma can be used to solve the following exercise.

\begin{thm}{Exercise}\label{ex:double-tangent}
Assume that $\alpha$ is a smooth regular plane curve with strictly monotonic curvature.
\begin{enumerate}[(a)]
\item\label{ex:double-tangent:a}Show that no line can be tangent to $\alpha$ at two distinct points.
\item Show that no circle can be tangent to $\alpha$ at three distinct points. 
\end{enumerate}
\end{thm} %???monotone or monotonic???

\begin{wrapfigure}{r}{35 mm}
\vskip-0mm
\centering
\includegraphics{mppics/pic-25}
\vskip0mm
\end{wrapfigure}

Note that part (\ref{ex:double-tangent:a}) does not hold for smooth regular plane curve with strictly monotonic \emph{signed} curvature; an example is shown on the diagram.




