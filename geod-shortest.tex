\chapter{Geodesics vs shortest paths}

Let $\Sigma$ be a smooth surface.
A smooth curve $\gamma$ on $\Sigma$ is called geodesic if its osculation is perpendicular to $\Sigma$ at each point;
that is, if $\gamma''(t)\perp\T_{\gamma(t)}$ at any time $t$.

\begin{thm}{Proposition}
Given an initial point $p\in \Sigma$ and a tangent vector $v\in\T_p\Sigma$ there is unique geodesic $\gamma$ defined on a maximal open interval containing zero that starts at $p$ with velocity $v$; that is,
such that $\gamma(0)=p$ and $\gamma'(0)=v$.
\end{thm}

\parit{Informal sketch.}
Consider a local description of $\Sigma$ as a graph $z=f(x,y)$ in the tangent-normal coordinates at $p$.
Any curve $\gamma$ in $\Sigma$ that starts at $p$ can be written locally as 
$\gamma(t)=(x(t),y(t),f(x(t),y(t)))$ in a neighborhood of $0$.

Let us show that the condition $\gamma''(t)\perp\T_{\gamma(t)}$ can be written as an odinary differential equation.
Indeed, the tangent plane is spanned by two vectors $(1,0,\tfrac{\partial f}{\partial x})$ and $(0,1,\tfrac{\partial f}{\partial y})$.
The acceleration of $\gamma$ can be written as $\gamma''(t)=(x'',y'',\tfrac{\partial^2 f}{\partial x^2}\cdot (x')^2+\tfrac{\partial^2 f}{\partial y^2}\cdot (y')^2+\tfrac{\partial f}{\partial x}\cdot x''+\tfrac{\partial f}{\partial y}\cdot y'')$.
Therefore we get $\gamma''(t)\perp\T_{\gamma(t)}$ is equivalent two two equations:




The smoothness should be intuitively obvious; at least the curve should be twice differentiable otherwise it can be shortened.

Let us give an informal physical explanation why $\gamma''(t)\perp\T_{\gamma(t)}\Sigma$.
One may think about the geodesic $\gamma$ as of stable position of a stretched elastic thread that is forced to lie on a frictionless surface.
Since it is frictionless, the force density $N(t)$ that keeps the geodesic $\gamma$ in the surface must be therefore proportional to the normal vector to the surface at $\gamma(t)$.
The tension in the thread has to be the same at all points (otherwise the thread would move back or forth and it would not be stable).
The tension at the ends of small arc is roughly proportional to the angle between the tangent lines at the ends of the arc. 
Passing to the limit as the length of the arc goes to zero, we get that the density of this force $F(t)$ is proportional to $\gamma''(t)$.
According to the second Newton's law, we have $F(t)+N(t)=0$;
which implies that  $\gamma''(t)$ is perpendicular to $\T_{\gamma(t)}\Sigma$.%
\footnote{In fact $\gamma''(t)+\nu\cdot \langle s(\gamma'(t)),\gamma'(t)\rangle=0$, where $s$ is the shape operator of $\Sigma$ at $\gamma(t)$ or equivalently,
$\gamma''(t)+\nu\cdot  \II(\gamma'(t)),\gamma'(t))=0$, where $\II$ is the second fundamental form of $\Sigma$ at $\gamma(t)$.}

The third statement can be also understood using physical intuition --- $\gamma(t)$ is the trajectory of a particle that slides on $\Sigma$ without friction and with initial velocity $v$.
Formally, existence and uniqueness follows from Picard's theorem (the  fundamental theorem of ordinary differential equations).
\qeds
