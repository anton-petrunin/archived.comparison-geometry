\chapter{Local comparison}

\section{First variation formula}

\begin{thm}{Proposition}\label{prop:first-var}
Assume $(s,t)\mapsto w(s,t)$ be a local parametrization of an oriented smooth regular surface $\Sigma$ such that 
$\tfrac{\partial}{\partial s}w\perp \tfrac{\partial}{\partial t}w$, $|\tfrac{\partial}{\partial s}w|=1$ and the vector $\tfrac{\partial}{\partial s}w$ points to the right from $\tfrac{\partial}{\partial t}w$ at any parameter value $(s,t)$.

Fix an closed real interval $[a,b]$ and consider one parameter family of curves $\sigma_s\:[a,b]\to \Sigma$  defined as the coordinate lines $\sigma_s(t)=w(s,t)$.
Set $\ell(s)=\length \sigma_s$.
Then
\[\ell'(s)=\Theta_{\sigma_s}\]
for any $s$.
\end{thm}

The proof is done by direct calculations.

\parit{Proof.}
Since $\tfrac{\partial}{\partial s}w\perp \tfrac{\partial}{\partial t}w$, we have that
\[\langle\tfrac{\partial}{\partial s}w, \tfrac{\partial}{\partial t}w\rangle=0\]
and therefore
\[\langle\tfrac{\partial^2}{\partial s\partial t}w, \tfrac{\partial}{\partial t}w\rangle
+\langle\tfrac{\partial}{\partial s}w, \tfrac{\partial^2}{\partial t^2}w\rangle=\tfrac{\partial}{\partial t}\langle\tfrac{\partial}{\partial s}w, \tfrac{\partial}{\partial t}w\rangle=0.\]

Note that $|\gamma'_s(t)|=|\tfrac{\partial}{\partial t}w(s,t)|$ and therefore
\begin{align*}
\tfrac \partial {\partial s}|\gamma'_s(t)|&=\tfrac \partial {\partial s}\sqrt{\langle \tfrac{\partial}{\partial t}w(s,t),\tfrac{\partial}{\partial t}w(s,t)\rangle}=
\\
&=\frac{\langle \tfrac{\partial^2}{\partial s\partial t}w(s,t),\tfrac{\partial}{\partial t}w(s,t)\rangle}{\sqrt{\langle \tfrac{\partial}{\partial t}w(s,t),\tfrac{\partial}{\partial t}w(s,t)\rangle}}=
\\
&=-\frac{\langle\tfrac{\partial}{\partial s}w, \tfrac{\partial^2}{\partial t^2}w\rangle}{|\gamma'_s(t)|}=
\\
&=-\frac{\langle\tfrac{\partial}{\partial s}w, \gamma''_s(t)\rangle}{|\gamma'_s(t)|}.
\end{align*}

The values $\ell(s)$ do not change if we reparametrize $\gamma_s$,
so we can assume that for a fixed value $s$ the curve $\sigma_s$ is unit-speed.
Since $|\tfrac{\partial}{\partial s}w|=1$ and $\tfrac{\partial}{\partial s}w$ points to the right from $\tfrac{\partial}{\partial t}w=\gamma'_s(t)$, the last expression equals to $k_g(s,t)$,
where $k_g(s,t)$ denotes the geodesic curvature of $\sigma_s$ at $t$. 
Therefore, for this particular $s$ we have
\begin{align*}
\ell'(s)&= \int_a^b \tfrac \partial {\partial s} |\gamma'_s(t)|\cdot dt =
\\
&= \int_a^b k_g(s,t)\cdot dt=
\\
&=\Theta_{\sigma_s}.
\end{align*}
Since the left hand side and the right hands side of this formula do not depend on the parametrization of $\sigma_s$, this formula holds for all $s$.\footnote{One may avid passing the a unit speed parametrization by using the following formula for geodesic curvature which holds for any regular parametrization: 
\[k_g(t,s)=\langle \nu(\sigma_s(t)),[\sigma'_s(t),\sigma''_s(t)]\rangle/|\sigma_s'(t)|^3;\]
it will save your thinking by making calculations longer.}
\qeds

The parametrization of a surface satisfying the conditions in the proposition are called \emph{semigeodesic coordinates}.
The following exercise explains the reason for this name.

\begin{thm}{Exercise}
Assume $(s,t)\mapsto w(s,t)$ be a local parametrization of an oriented smooth regular surface $\Sigma$ as in the proposition above.
Show that for any fixed $t$ the curve $\gamma_t(s)= w(s,t)$ is a geodesic.\footnote{Hint: note that in order to show that $\gamma''_t(s)\perp\T_{\gamma_t(s)}$, it is sufficient to show that $\langle\tfrac{\partial^2}{\partial s^2}w,\tfrac{\partial}{\partial t}w\rangle=0$.}
\end{thm}


\section{Exponential map}

Let $\Sigma$ be smooth regular surface and $p\in \Sigma$.
Given a tangent vector $v\in \T_p$ consider a geodesic $\gamma_v$ in $\Sigma$ that runs from $p$ with the initial velocity $v$;  
that is, $\gamma(0)=p$ and $\gamma'(0)=v$.

The point $q=\gamma_v(1)$ is called \emph{exponential map} of $v$, or briefly $q=\exp_pv$.
The map $\exp_p\:\T_p\to \Sigma$ is defined in a neighborhood of zero.
We assume that it is intuitively obvious that the map $\exp_p$ is smooth;
formally it follows since the solution of the initial value problem for the equation $\gamma_v''(t)\perp\T_{\gamma_v(t)}$ which describes the geodesic $\gamma_v$ smoothly depend on the initial data $v$.
Note that the Jacobian of $\exp_p$ at zero is the identity matrix.
Therefore frm the inverse function theorem we get the following statement:

\begin{thm}{Proposition}\label{prop:exp}
Let $\Sigma$ be smooth regular surface and $p\in \Sigma$.
Then the exponential map $\exp_p\:\T_p\to \Sigma$ is a smooth regular parametrization of a neighborhood of $p$ in $\Sigma$ by a neighborhood of $0$ in the tangent plane~$\T_p$.

Moreover for any $p\in \Sigma$ there is $\eps>0$ such that for any $x\in \Sigma$ such that $|x-p|_\Sigma<\eps$ the map 
$\exp_x\:\T_x\to \Sigma$ is a smooth regular parametrization of the $\eps$-neighborhood of $x$ in $\Sigma$ by the $\eps$-neighborhood of zero in the tangent plane~$\T_x$.
\end{thm}

Note that if there are two minimizing geodesics between two points $x$ and $y$ in a surface,
then there are two distinct vectors $v,v'\in \T_x$ such that $y=\exp_xv=\exp_xv'$.
Therefore by proposition above we get the following:

\begin{thm}{Corollary}
Let $\Sigma$ be a smooth regular surface.
Then for any point $p\in\Sigma$ there is $\eps>0$ such that any two points $x$ and $y$ in the $\eps$-neightbohood of $p$ in $\Sigma$ can be connected by a unique minimizing geodesic $[xy]_\Sigma$.
\end{thm}


\section{Polar coordinates}

Proposition~\ref{prop:exp} implies existence of polar coordinates in a neighborhood of any point in $p$ in $\Sigma$.
That is, any point $x$ in $\Sigma$ sufficiently close to $p$ 
can be uniquely described by the distance $|x-p|_\Sigma$ and the direction from $p$ to $x$.

Assume $(\theta,r)$ are the described polar coordinates at $p$.
Namely, assume $\~w(\theta,r)$ denotes the tangent vector at $p$ with polar coordinates $(\theta,r)$ and $w( \theta,r)=\exp_p[\~w(\theta,r)]$.
By the definition of exponential map, for a fixed $\theta$, the curve $\gamma_\theta(t)=w(\theta,t)$ is a unit-speed geodesic that starts at $p$;
in particular $|\tfrac{\partial}{\partial r}w|=|\gamma_\theta'(r)|=1$ and $\gamma''_\theta(r)\perp\T_{\gamma_\theta(r)}$.

The curve $\sigma_r(t)=w(t,r)$ is a parametrization of the circle of radius $r$ and center at $p$ in $\Sigma$; that is, if $q=\sigma_r(t)$, then $|q-p|_\Sigma=r$.
If the latter is not the case, then a minimizing geodesic $[pq]_\Sigma$ would be shorter than $r$ and therefore $q$ would not be described uniquely in the polar coordinates. 

Note that $\tfrac{\partial}{\partial r}w\perp \tfrac{\partial}{\partial \theta}w$ if $r>0$;
otherwise for small $\eps>0$ the intrinsic distance from $p$ to $w(\theta\pm \eps,r)$ would be shorter than $r$, which contradicts the previous statement.

\begin{thm}{Proposition}\label{prop:loc-comp-l}
Let $w(\theta,r)$ and $\~w(\theta,r)$ be the polar coordinates of a surface $\Sigma$ at $p$ and its tangent plane $\T_p$ at zero, so $w(\theta,r)\z=\exp_p[\~w(\theta,r)]$.
Given a real interval $[a,b]$ consider the one parameter families of circular arcs $\sigma_r\:[a,b]\to \Sigma$ and $\~\sigma_r\:[a,b]\to \T_p$
$\sigma_r(t)\z=w(t,r)$ and $\~\sigma_r(t)=\~w(t,r)$.
Set $\ell(r)=\length \sigma_r$ and $\~\ell(r)\z=\length \~\sigma_r$.%
\footnote{Note that angular measure of $\~\sigma_r$ is $b-a$; therefore $\~\ell(r)=r\cdot(b-a)$.}

\begin{enumerate}[(i)]
 \item If the Gauss curvature of $\Sigma$ is nonnegative, then 
 \[\ell(r)\le \~\ell(r)\]
 for all small $r>0$.
 \item If the Gauss curvature of $\Sigma$ is nonpositive, then 
 \[\ell(r)\ge \~\ell(r)\]
 for all small $r>0$.
\end{enumerate}

\end{thm}

Taking a limit as $b\to a$, we obtain the following corollary.

\begin{thm}{Corollary}\label{cor:w<w}
Let $w(\theta,r)$ and $\~w(\theta,r)$ be the polar coordinates of a surface $\Sigma$ at $p$ and its tangent plane $\T_p$ at zero, so $w(\theta,r)\z=\exp_p[\~w(\theta,r)]$.
\begin{enumerate}[(i)]
 \item If the Gauss curvature of $\Sigma$ is nonnegative, then 
 \[|\tfrac{\partial}{\partial \theta} w|\le |\tfrac{\partial}{\partial \theta} \~w|\]
 for all small $r>0$.
 \item If the Gauss curvature of $\Sigma$ is nonpositive, then 
\[|\tfrac{\partial}{\partial \theta} w|\ge |\tfrac{\partial}{\partial \theta} \~w|\]
 for all small $r>0$.
\end{enumerate}
\end{thm}


\parit{Proof.}
From the above discussion, the polar coordinates $w(\theta,r)$ are semigeodesic;
that is $w(\theta,r)$ satisfies the conditions in the first variation formula (\ref{prop:first-var}).
In particular if $\ell(r)=\length \sigma_r$, then
\[\ell'(r)=\Theta_{\sigma_r}\]
for any $r>0$.

By Gauss--Bonnet formula, the last identity can be rewritten as
\[\ell'(r)=2\cdot (b-a) -\iint_{\Delta_r}G,\eqlbl{eq:ell'}\]
where $\Delta_r$ is the sector in $\Sigma$ in the polar coordinates at $p$
\[\set{w(t,s)}{a\le t\le b,\ 0\le s\le r};\]
which is bounded by two geodesics from $p$ with angle $b-a$ 
and a circular arc that meets these geodesics at right angle.

Since the plane has vanishing Gauss curvature, we have
\[\~\ell'(r)=2\cdot (b-a),\eqlbl{eq:tilde-ell'}\]
which agrees with the formula for the length of the arc $\~\ell(r)\z=2\cdot\pi\cdot r$.

If the Gauss curvature of $\Sigma$ is nonnegative,
the equations \ref{eq:ell'} and \ref{eq:tilde-ell'} imply that
\[\ell'(r)\le \~\ell'(r)\]
for any small $r$.

If the Gauss curvature of $\Sigma$ is nonnegative,
the same equations imply that
\[\ell'(r)\ge \~\ell'(r)\]
for any small $r$.

Since $\ell(0)=\~\ell(0)$, integrating the inequalities proves both statements.\qeds

The following exercise provides a stronger statement.
It almost follow from the proof above, but one has to make an extra observation.


\begin{thm}{Exercise}
Assume $\Sigma$ is a smooth regular surface and $p\in\Sigma$,
denote by $\ell(r)$ the circumference of the circle with the center at $p$ and radius $r$ in $\Sigma$
and let $\~\ell(r)=2\cdot\pi\cdot r$ the circumference of the plane circle of radius $r$.

\begin{enumerate}[(i)]
 \item Show that if Gauss curvature of $\Sigma$ is nonnegative, then the function $r\mapsto \ell(r)$ is concave for small $r>0$. Conclude that the function $r\mapsto \frac{\ell(r)}{\~\ell(r)}$ is nonincreasing for small $r>0$.
\item Show that if Gauss curvature of $\Sigma$ is nonpositive, then the function $r\mapsto \ell(r)$ is convex for small $r>0$. Conclude that the function $r\mapsto \frac{\ell(r)}{\~\ell(r)}$ is nondecresing for small $r>0$.
\end{enumerate}

\end{thm}

\section{Local comparison}

The following proposition is a special case of the so a comparison theorem, proved by Harry Rauch \cite{rauch}.

\begin{thm}{Theorem}\label{thm:rauch}
Let $\Sigma$ be a smooth regular surface and $p\in\Sigma$.
Assume $\~\gamma\:[a,b]$ is a curve the tangent plane $\T_p\Sigma$ that runs in a sufficiently small neighborhood of the origin; 
consider the curve 
\[\gamma=\exp_p\circ\gamma\] in $\Sigma$.

\begin{enumerate}[(i)]
 \item If Gauss curvature of $\Sigma$ is nonnegative, then 
 \[\length \gamma\le \length \~\gamma\]
\item If Gauss curvature of $\Sigma$ is nonpositive, then 
 \[\length \gamma\ge \length \~\gamma.\]
\end{enumerate}
\end{thm}

The proof is a direct application of Corollary \ref{cor:w<w}.

\parit{Proof.}
Let us denote $\~w(\theta,r)$ and $w(\theta,r)$ the polar coordinates of $\T_p$ and $\Sigma$ at $p$
Recall that 
\begin{align*}
\tfrac{\partial \~w}{\partial\theta}&\perp \tfrac{\partial \~w}{\partial r};
&
|\tfrac{\partial \~w}{\partial r}|&=1;
\\
\tfrac{\partial w}{\partial\theta}&\perp \tfrac{\partial w}{\partial r};
&
|\tfrac{\partial w}{\partial r}|&=1;
\intertext{By Corollary \ref{cor:w<w}, we also have}
|\tfrac{\partial \~w}{\partial \theta}|&\ge |\tfrac{\partial \~w}{\partial \theta}|;
&
|\tfrac{\partial \~w}{\partial \theta}|&\le |\tfrac{\partial \~w}{\partial \theta}|;
\end{align*}
if Gauss curvature is nonnegative or nonpositive correspondingly.

It is sufficient to show that
\[|\gamma'(t)|\le |\~\gamma'(t)|\quad\text{or, correspondingly }\quad|\gamma'(t)|\ge |\~\gamma'(t)|\eqlbl{gamma<gamma}\]
for any $t$.

Note that both curves $\gamma(t)$ and $\~\gamma(t)$ described the same way in the polar coordinates;
denote these coordinates by $(\theta(t),r(t))$.
Then 
\begin{align*}
\gamma'(t)|^2&=|\tfrac{\partial w}{\partial\theta}\cdot\theta'(t)+\tfrac{\partial w}{\partial r}\cdot r'(t)|^2=
\\
&=|\tfrac{\partial w}{\partial\theta}|^2\cdot|\theta'(t)|^2+|r'(t)|^2
\intertext{The same way} 
\~\gamma'(t)|^2&=|\tfrac{\partial \~w}{\partial\theta}|^2\cdot|\theta'(t)|^2+|r'(t)|^2;
\end{align*}
hence \ref{gamma<gamma} follows.
\qeds
