\chapter{Local comparison}

\section{First variation formula}

\begin{thm}{Proposition}\label{prop:first-var}
Assume $(s,t)\mapsto w(s,t)$ be a local parametrization of an oriented smooth regular surface $\Sigma$ such that 
$\tfrac{\partial}{\partial s}w\perp \tfrac{\partial}{\partial t}w$, $|\tfrac{\partial}{\partial s}w|=1$ and the vector $\tfrac{\partial}{\partial s}w$ points to the right from $\tfrac{\partial}{\partial t}w$ at any parameter value $(s,t)$.

Fix a closed real interval $[a,b]$ and consider a one parameter family of curves $\sigma_s\:[a,b]\to \Sigma$  defined as the coordinate lines $\sigma_s(t)=w(s,t)$.
Set $\ell(s)=\length \sigma_s$.
Then
\[\ell'(s)=\Theta_{\sigma_s}\]
for any $s$.
\end{thm}

The proof is done by direct calculations.

\parit{Proof.}
Since $\tfrac{\partial}{\partial s}w\perp \tfrac{\partial}{\partial t}w$, we have that
\[\langle\tfrac{\partial}{\partial s}w, \tfrac{\partial}{\partial t}w\rangle=0\]
and therefore
\[\langle\tfrac{\partial^2}{\partial s\partial t}w, \tfrac{\partial}{\partial t}w\rangle
+\langle\tfrac{\partial}{\partial s}w, \tfrac{\partial^2}{\partial t^2}w\rangle=\tfrac{\partial}{\partial t}\langle\tfrac{\partial}{\partial s}w, \tfrac{\partial}{\partial t}w\rangle=0.\]

Note that $|\gamma'_s(t)|=|\tfrac{\partial}{\partial t}w(s,t)|$ and therefore
\begin{align*}
\tfrac \partial {\partial s}|\gamma'_s(t)|&=\tfrac \partial {\partial s}\sqrt{\langle \tfrac{\partial}{\partial t}w(s,t),\tfrac{\partial}{\partial t}w(s,t)\rangle}=
\\
&=\frac{\langle \tfrac{\partial^2}{\partial s\partial t}w(s,t),\tfrac{\partial}{\partial t}w(s,t)\rangle}{\sqrt{\langle \tfrac{\partial}{\partial t}w(s,t),\tfrac{\partial}{\partial t}w(s,t)\rangle}}=
\\
&=-\frac{\langle\tfrac{\partial}{\partial s}w, \tfrac{\partial^2}{\partial t^2}w\rangle}{|\gamma'_s(t)|}=
\\
&=-\frac{\langle\tfrac{\partial}{\partial s}w, \gamma''_s(t)\rangle}{|\gamma'_s(t)|}.
\end{align*}

The values $\ell(s)$ do not change if we reparametrize $\gamma_s$,
so we can assume that for a fixed value $s$ the curve $\sigma_s$ is unit-speed.
Since $|\tfrac{\partial}{\partial s}w|=1$ and $\tfrac{\partial}{\partial s}w$ points to the right from $\tfrac{\partial}{\partial t}w=\gamma'_s(t)$, the last expression equals to $k_g(s,t)$,
where $k_g(s,t)$ denotes the geodesic curvature of $\sigma_s$ at $t$. 
Therefore, for this particular $s$ we have
\begin{align*}
\ell'(s)&= \int_a^b \tfrac \partial {\partial s} |\gamma'_s(t)|\cdot dt =
\\
&= \int_a^b k_g(s,t)\cdot dt=
\\
&=\Theta_{\sigma_s}.
\end{align*}
Since the left hand side and the right hands side of this formula do not depend on the parametrization of $\sigma_s$, this formula holds for all $s$.\footnote{One may avoid passing the a unit-speed parametrization by using the following formula for geodesic curvature which holds for any regular parametrization: 
\[k_g(t,s)=\langle \Norm(\sigma_s(t)),[\sigma'_s(t),\sigma''_s(t)]\rangle/|\sigma_s'(t)|^3;\]
it saves thinking but makes the calculations longer.}
\qeds

The parametrization of a surface satisfying the conditions in the proposition are called \emph{semigeodesic coordinates}.
The following exercise explains the reason for this name.

\begin{thm}{Exercise}\label{ex:geod-semigeod}
Assume $(s,t)\mapsto w(s,t)$ be a local parametrization of an oriented smooth regular surface $\Sigma$ as in the proposition above.
Show that for any fixed $t$ the curve $\gamma_t(s)= w(s,t)$ is a geodesic.
\end{thm}







\section{Local comparison}

The following proposition is a special case of the so a comparison theorem, proved by Harry Rauch \cite{rauch}.

\begin{thm}{Theorem}\label{thm:rauch}
Let $\Sigma$ be a smooth regular surface and $p\in\Sigma$.
Assume $\~\gamma\:[a,b]$ is a curve the tangent plane $\T_p\Sigma$ that runs in a sufficiently small neighborhood of the origin; 
consider the curve 
\[\gamma=\exp_p\circ\gamma\] in $\Sigma$.

\begin{enumerate}[(i)]
 \item If Gauss curvature of $\Sigma$ is nonnegative, then 
 \[\length \gamma\le \length \~\gamma\]
\item If Gauss curvature of $\Sigma$ is nonpositive, then 
 \[\length \gamma\ge \length \~\gamma.\]
\end{enumerate}
\end{thm}

The proof is a direct application of Corollary \ref{cor:w<w}.

\parit{Proof.}
Let us denote $\~w(\theta,r)$ and $w(\theta,r)$ the polar coordinates of $\T_p$ and $\Sigma$ at $p$
Recall that 
\begin{align*}
\tfrac{\partial \~w}{\partial\theta}&\perp \tfrac{\partial \~w}{\partial r};
&
|\tfrac{\partial \~w}{\partial r}|&=1;
\\
\tfrac{\partial w}{\partial\theta}&\perp \tfrac{\partial w}{\partial r};
&
|\tfrac{\partial w}{\partial r}|&=1;
\intertext{By Corollary \ref{cor:w<w}, we also have}
|\tfrac{\partial \~w}{\partial \theta}|&\ge |\tfrac{\partial \~w}{\partial \theta}|;
&
|\tfrac{\partial \~w}{\partial \theta}|&\le |\tfrac{\partial \~w}{\partial \theta}|;
\end{align*}
if Gauss curvature is nonnegative or nonpositive correspondingly.

It is sufficient to show that
\[|\gamma'(t)|\le |\~\gamma'(t)|\quad\text{or, correspondingly }\quad|\gamma'(t)|\ge |\~\gamma'(t)|\eqlbl{gamma<gamma}\]
for any $t$.

Note that both curves $\gamma(t)$ and $\~\gamma(t)$ described the same way in the polar coordinates;
denote these coordinates by $(\theta(t),r(t))$.
Then 
\begin{align*}
\gamma'(t)|^2&=|\tfrac{\partial w}{\partial\theta}\cdot\theta'(t)+\tfrac{\partial w}{\partial r}\cdot r'(t)|^2=
\\
&=|\tfrac{\partial w}{\partial\theta}|^2\cdot|\theta'(t)|^2+|r'(t)|^2
\intertext{The same way} 
\~\gamma'(t)|^2&=|\tfrac{\partial \~w}{\partial\theta}|^2\cdot|\theta'(t)|^2+|r'(t)|^2;
\end{align*}
hence \ref{gamma<gamma} follows.
\qeds

