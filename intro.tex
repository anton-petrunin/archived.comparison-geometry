
\chapter*{Preface}

These notes are designed for those who either plan to work in differential geometry,
or at least want to have a good reason \emph{not} to do~it.
It should be more than sufficient for a semester-long course. 

Differential geometry exploits several branches of mathematics including 
real analysis, 
measure theory,
calculus of variations,
differential equations,
elementary and convex geometry,
topology, and more.
This subject is wide even at the beginning. 
For that reason, it is fun and painful both to teach and to study.

In this book, we discuss smooth curves and surfaces --- the main gate to differential geometry.
This subject provides a collection of examples and ideas critical for further study.
It is wise to become a master in this subject before making further steps --- there is no need to rush.

We give a general overview of the subject, keeping it
problem-centered,
elementary, 
visual, 
and virtually rigorous; we allow gaps that belong to other branches of mathematics (mostly to the subjects discussed briefly in the preliminaries).

We focus on the techniques that absolutely essential for further study.
By that reason we omit number of topics that traditionally included in the introductory texts;
for example, we do not touch %Frenet calculus, 
minimal surfaces and Peterson--Codazzi formulas.
At the same, we get to advanced applications
 that are not in the scope of typical introductory texts.
 
The first example is the theorem of Vladimir Ionin and Herman Pestov about \emph{the Moon in a puddle} (\ref{thm:moon-orginal}).
This theorem might be the simplest meaningful example of the so-called {}\emph{local to global theorems} which lies in the heart of differential geometry;
by that reason it is a good answer to the main question of this book --- ``What is differential geometry?''.

Other examples include the theorem of Sergei Bernstein's on saddle graphs, and two theorems of Stephan Cohn-Vossen on existence of simple two-sided infinite geodesic on an open convex surface and the splitting theorem.

If we could find an \emph{excellent} textbook on the subject, then we would not have written this one.
We extensively used two textbooks which are \emph{among the best}: by Aleksei Chernavskii \cite{chernavsky} and by Victor Toponogov~\cite{toponogov-book}.
Both of these books are based on extensive teaching experience.

For further study of differential geometry one should read some text on tensor calculus; the book of Richard Bishop and Samuel Goldberg \cite{bishop-goldberg} is one our favorites.
Once it is done, the most straightforward direction would be Riemannian geometry.
You may go with  ``Comparison geometry'' \cite{cheeger-ebin} --- the good old classics from Jeff Cheeger and David Ebin, in this case you might be surprised to see that almost all of it already is already known.
Another classical book is ``Riemannsche Geometrie im Großen'' \cite{gromoll-klingenberg-meyer} by 
Detlef Gromoll,
Wilhelm Klingenberg, 
and  Werner Meyer (it is available in German and Russian only).
Mikhael Gromov's ``Sign and geometric meaning of curvature'' \cite{gromov-1991} will be more challenging, but worth trying.
But if you choose another branch of differential geometry, the covered material will be still helpful.

These notes are based on the lectures given at the MASS program (Mathematics Advanced Study Semesters at Pennsylvania State University) Fall semester 2018.
We want to thank the students in our class, Yurii Burago, and Nina Lebedeva for help.
A large number of these topics were presented by Yurii Burago in his lectures teaching the first author at the Leningrad University.

\begin{flushright}
Anton Petrunin and\\
Sergio Zamora Barrera.
\end{flushright}



\newpage
\tableofcontents
