
\chapter*{Preface}

The notes are designed for those who plan to do differential geometry in the future,
or at least who want to have a solid ground to decide not to do it.

Differential geometry does geometry on top of several branches of mathematics including real analysis, differential equations, topology and other branches of geometry, including elementary and convex geometry.
This subject is wide even on the introductory level. 
By that reason it is fun and pain to teach and to study.

We discuss differential geometry of curves and surfaces.
This subject is the main gate to differential geometry;
it provides a collection of examples critical for further study.
In fact one has to become a master in curves and surfaces before making further steps in differential geometry.

We try to give an idea about subject, keeping it elementary and virtually rigorous, meaning that we allow gaps that belong to other branches of mathematics (these subjects discussed briefly in the appendixes).
Sections marked with * are not used in the sequel.

We try to minimize the distance between definitions and meaningful results.
We have three highest points:
the theorem of Vladimir Ionin and Herman Pestov about closed curves with bounded curvature,
the theorem of Sergei Bernstein's on saddle graphs,
and the theorem of Stephan Cohn-Vossen on existence of simple two-sided infinite geodesic on an open convex surface.
These two theorems give the first nontrivial examples of the so called \emph{local to global theorems} which are in the hart of differential geometry. 

These notes are based on lectures at MASS program (Mathematics Advanced Study Semesters at Pennsylvania State University) Fall semester 2018.
Number of these topics were presented on the lectures of Yurii Burago who was teaching the first author at the Leningrad University.
For further study, we recommend the textbook of Victor Toponogov~\cite{toponogov}.
