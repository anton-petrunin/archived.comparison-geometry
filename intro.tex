\section*{Preface}

The course is oriented to the student who plan to do differential geometry in the future,
or at least those who want to have a solid ground to decide not to do it.
The differential geometry of curves and surfaces is a classical subject that is introductory to differential geometry.
It does not make sense to do differential geometry until one is a master in curves and surfaces.

Differential geometry is doing geometry on top of several branches of mathematics including real analysis, differential equations, topology and few other branches of geometry, including elementary and convex geometry.
By that reason this subject is hard to teach and hard to study.

The subject of differential geometry is huge,
it is easy to imagine two professional differential geometers who can not find a single subject in the field which they are both slightly interested in.
This is another reason why it is hard to teach --- courses sharped for students with major physics, biology, architecture, economy, and philosophy would have almost no common topic.

The choice of subjects is taken mostly from the lectures of Yurii Burago who was teaching author at the university.
Among the other textbooks on the subject, I liked the book of Victor Toponogov \cite{toponogov};
despite big number of small mistakes this book gives.
I believe that this choice of subject makes the easiest way to transit to further study of differential geometry, but if the goal is different, then one should consider to change the book.
If you major is physics I would recommend 
