
\chapter*{Preface}

These notes are designed for those who either plan to work in differential geometry,
or at least want to have a good reason \emph{not} to do~it.

Differential geometry exploits several branches of mathematics including 
real analysis, 
measure theory,
calculus of variations,
differential equations,
elementary and convex geometry,
topology, and more.
This subject is wide even at the beginning. 
For that reason, it is fun and painful both to teach and to study.

In this book, we discuss smooth curves and surfaces --- the main gate to differential geometry.
This subject provides a collection of examples and ideas critical for further study.
It is wise to become a master in this subject before making further steps --- there is no need to rush.

We give a general overview of the subject, keeping it elementary, visual, and virtually rigorous; we allow gaps that belong to other branches of mathematics (mostly to the subjects discussed briefly in the preliminaries).

The highest points in these notes are
the theorem of Vladimir Ionin and Herman Pestov about closed curves with bounded curvature,
the theorem of Sergei Bernstein's on saddle graphs,
and two theorems of Stephan Cohn-Vossen on existence of simple two-sided infinite geodesic on an open convex surface and the splitting theorem.
Three of these results give the first nontrivial examples of the so-called {}\emph{local to global theorems} --- the heart of differential geometry.

These notes are based on the lectures given at the MASS program (Mathematics Advanced Study Semesters at Pennsylvania State University) Fall semester 2018.
A large number of these topics were presented by Yurii Burago in his lectures teaching the first author at the Leningrad University.

If we could find an \emph{excellent} textbook on the subject, then we would not have written this one.
We extensively used two textbooks which are \emph{among the best}: by Aleksei Chernavskii \cite{chernavsky} and by Victor Toponogov~\cite{toponogov-book}.

(Chapters or sections titled in parenthesis might be skipped without worrying to lose the thread in the sequel.)

\begin{flushright}
Anton Petrunin and\\
Sergio Zamora Barrera.
\end{flushright}



\newpage
\tableofcontents
