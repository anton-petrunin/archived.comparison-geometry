\chapter{Length}

The material of this and the following chapters overlaps largely with \cite[Chapter 5]{fuchs-tabachnikov}.

\section{Length of curve}


\begin{thm}{Definition}\label{def:curve}
Consider a \emph{plane curve}\index{plane curve} $\alpha\:[a,b]\z\to \RR^2$; a continuous mapping from the real interval $[a,b]$ to the Euclidean plane $\RR^2$. 


If $\alpha(a)=p$ and $\alpha(b)=q$,
we say that $\alpha$ is a \emph{curve from $p$ to $q$}\index{curve from $p$ to $q$}.

A curve $\alpha\:[a,b]\to \RR^2$ is called \emph{closed} if $\alpha(a)=\alpha(b)$.

A curve $\alpha$ is called \emph{simple} if it is described by an injective map;
that is $\alpha(t)=\alpha(t')$ if and only if $t=t'$.
However, a closed curve $\alpha\:[a,b]\to \RR^2$ is called simple if it is injective 
everywhere except at the ends; that is, if
$\alpha(t)=\alpha(t')$ for $t<t'$ then $t=a$ and $t'=b$.

A closed curve is called \emph{convex} if it bounds a convex region.
\end{thm}

\begin{thm}{Advanced exercise}
Let $\alpha\:[0,1]\to\RR^2$  from $p$ to $q$.
Assume $p\ne q$
Show that there is a simple curve $\beta\:[0,1]\to\RR^2$  from $p$ to $q$
that runs in the image of $\alpha$;
that is for any $t\in [0,1]$ there is $t'\in [0,1]$ such that $\beta(t)=\alpha(t')$.
\end{thm}

 
Recall that a sequence 
\[a=t_0 < t_1 < \cdots < t_k=b.\]
is called a \emph{partition} of the interval $[a,b]$.

\begin{thm}{Definition}\label{def:length}
Let $\alpha\:[a,b]\to \RR^2$ be a curve.
The \emph{length}\index{length of curve} of $\alpha$ is defined as

\begin{align*}
\length \alpha
= 
\sup \{|\alpha(t_0)-\alpha(t_1)|&+|\alpha(t_1)-\alpha(t_2)|+\dots
\\
&\dots+|\alpha(t_{k-1})-\alpha(t_k)|\}. 
\end{align*}

where the exact upper bound is taken over all partitions
\[a=t_0 < t_1 < \cdots < t_k=b.\]

Note that $\length\alpha\in[0,\infty]$;
the curve $\alpha$ is called \emph{rectifiable}\index{rectifiable curve} if its length is finite.
\end{thm}

Informally, one could say that the length of a curve is the exact upper bound of the lengths of polygonal lines \emph{inscribed} in the curve.

\begin{thm}{Exercise}
Assume $\alpha\:[a,b]\to\RR^2$ is a smooth curve, in particular the velocity vector $\alpha'(t)$ is defined and depends continuously on $t$.
Show that
\[\length \alpha=\int_a^b|\alpha'(t)|\cdot dt.\]
\end{thm}

\begin{thm}{Exercise}\label{ex:nonrectifiable-curve}
Construct a nonrectifiable curve $\alpha\:[0,1]\to\RR^2$.
\end{thm}

A closed simple plane curve is called \emph{convex} if it bounds a convex figure.

\begin{thm}{Proposition}\label{prop:convex-curve}
Assume a convex figure $A$ bounded by a curve $\alpha$ lies inside a figure $B$ bounded by a curve $\beta$.
Then
\[\length\alpha\le \length\beta.\]
\end{thm}

Note that it is sufficient to show that for any polygon  $P$ inscribed in $\alpha$ there is a polygon $Q$ inscribed in $\beta$ with 
$\perim P\le \perim Q$, where $\perim P$ denotes the perimeter of $P$.

Therefore it is sufficient to prove the following lemma.


\begin{thm}{Lemma}\label{lem:perimeter}
Let $P$ and $Q$ be polygons.
Assume $P$ is convex and $Q\supset P$.
Then $\perim P\le \perim Q$.
\end{thm}


\begin{wrapfigure}{r}{24 mm}
\vskip-4mm
\centering
\includegraphics{mppics/pic-7}
%\caption*{}
\end{wrapfigure}

\parit{Proof.}
Note that by the triangle inequality,
the inequality
\[\perim P\le \perim Q\]

holds
if $P$ can be obtained from $Q$ by cutting it along a chord;
that is, a line segment with ends on the boundary of $Q$ that lies in $Q$.


Note that there is an increasing sequence of polygons 
$$P=P_0\subset P_1\subset\dots\subset P_n=Q$$
such that $P_{i-1}$ obtained from $P_{i}$ by cutting along a chord.
Therefore 
\begin{align*}
\perim P=\perim P_0&\le\perim P_1\le\dots
\\
\dots&\le\perim P_n=\perim Q
\end{align*}
and the lemma follows.
\qeds

\begin{thm}{Corollary}
Any convex closed curve is rectifiable.  
\end{thm}

\parit{Proof.}
Any closed curve is bounded; that is, it lies in a sufficiently large square.


By Proposition~\ref{prop:convex-curve}, the length of the curve can not exceed the perimeter of the square, hence the result.
\qeds



\section{Semicontinuity of length}

Recall that the lower limit 
of a sequence of real numbers $(x_n)$ is denoted by
\[\liminf_{n\to\infty} x_n.\] 
It is defined as the lowest partial limit; that is, the lowest possible limit of a subsequence of $(x_n)$.
The lower limit is defined for any sequence of real numbers and it lies in the exteded real line $[-\infty,\infty]$


\begin{thm}{Theorem}\label{thm:length-semicont}
Length is a lower semi-continuous with respect to pointwise convergence of curves. 

More precisely, assume that a sequence
of curves $\alpha_n\:[a,b]\to \RR^2$ converges pointwise 
to a curve $\alpha_\infty\:[a,b]\to \RR^2$;
that is, $\alpha_n(t)\z\to\alpha_\infty(t)$ for any fixed $t\in[a,b]$ as $n\to\infty$. 
Then 
$$\liminf_{n\to\infty} \length\alpha_n \ge \length\alpha_\infty.\eqlbl{eq:semicont-length}$$
\end{thm}



\begin{wrapfigure}{r}{20 mm}
\vskip-0mm
\centering
\includegraphics{mppics/pic-6}
\end{wrapfigure}


Note that the inequality \ref{eq:semicont-length} might be strict.
For example the diagonal $\alpha_\infty$ of the unit square 

can be  approximated by a sequence of stairs-like
polygonal curves $\alpha_n$
with sides parallel to the sides of the square ($\alpha_6$ is on the picture).
In this case
\[\length\alpha_\infty=\sqrt{2}\quad
\text{and}\quad \length\alpha_n=2\]
for any $n$.

\parit{Proof.}
Fix $\eps > 0$ and choose a partition $a=t_0<t_1<\dots<t_k=b$
such that 
\begin{align*}
\length\alpha_\infty<
|\alpha_\infty(t_0)-\alpha_\infty(t_1)|+\dots+|\alpha_\infty(t_{k-1})-\alpha_\infty(t_k)|+\eps.
\end{align*}

Set 
\begin{align*}\Sigma_n
&\df
|\alpha_n(t_0)-\alpha_n(t_1)|+\dots+|\alpha_n(t_{k-1})-\alpha_n(t_k)|.
\\
\Sigma_\infty
&\df
|\alpha_\infty(t_0)-\alpha_\infty(t_1)|+\dots+|\alpha_\infty(t_{k-1})-\alpha_\infty(t_k)|.
\end{align*}

Note that $\Sigma_n\to \Sigma_\infty$ as $n\to\infty$
and $\Sigma_n\le\length\alpha_n$ for each $n$.
Hence
$$\liminf_{n\to\infty} \length\alpha_n \ge \length\alpha_\infty-\eps.$$
Since $\eps>0$ is arbitrary, we get \ref{eq:semicont-length}.\qeds

\section{Axioms of length}

\parbf{Concatenation.}
Assume $\alpha\:[a,b]\to \RR^2$ and $\beta\:[b,c]\z\to \RR^2$ are two curves such that $\alpha(b)=\beta(b)$.
Then one can combine these two curves into one $\gamma\:[a,c]\z\to \RR^2$ by the rule $\gamma(t)=\alpha(t)$ for $t\le b$ and $\gamma(t) = \beta(t)$ for $t\ge b$.
The obtained curve $\gamma$ is called the 
\emph{concatenation} of $\alpha$ and $\beta$ and is denoted as $\gamma=\alpha*\beta$.

\begin{wrapfigure}{r}{30 mm}
\vskip-0mm
\centering
\includegraphics{mppics/pic-10}
\end{wrapfigure}

Note that
\[\length(\alpha*\beta)=\length\alpha+\length\beta\]
for any two curves $\alpha$ and $\beta$ such that the concatenation $\alpha*\beta$ is defined.

\parbf{Reparametrization.}
Assume $\alpha\:[a,b]\to \RR^2$ is a curve and $\tau\:[c,d]\to [a,b]$ is a continuous strictly monotonic onto map.
Consider the curve $\alpha'\:[c,d]\to \RR^2$ defined by $\alpha'=\alpha\circ\tau$.
The curve $\alpha'$ is called a \emph{reparametrization} of $\alpha$.

Note that 
\[\length\alpha'=\length\alpha\]
whenever $\alpha'$ is a reparametrization of $\alpha$.





\begin{thm}{Proposition}\label{prop:length-axioms}
Let $\ell$ be a functional that returns a value in $[0,\infty]$ for any curve $\alpha\:[a,b]\to\RR$.

Assume it satisfies the following properties:
\begin{enumerate}[(i)]
\item\label{Normalization} (Normalization) If $\alpha\:[a,b]\to \RR^2$ is a linear curve,%
\footnote{That is $\alpha=w+v\cdot t$ for some vectors $w$ and $v$.} then
\[\ell(\alpha)=|\alpha(a)-\alpha(b)|.\]
\item\label{Additivity} (Additivity) If the concatenation $\alpha*\beta$ is defined, then
\[\ell(\alpha*\beta)=\ell(\alpha)+\ell(\beta).\]
\item\label{Motion invariance} (Motion invariance) The functional $\ell$ is invariant with respect to the motions of the plane; that is, if $m$ is an isometry of the plane, then 
\[\ell(m\circ\alpha)=\ell(\alpha)\]
for any curve $\alpha$.
\item\label{Reparametrization invariance} (Reparametrization invariance) If $\alpha'$ is a reparametrization of a curve $\alpha$ then
\[\ell(\alpha')=\ell(\alpha).\]
(In fact linear reparametrizations will be sufficient.)
\item\label{Semi-continuity} (Semi-continuity) If a sequence of curves $\alpha_n\:[a,b]\to \RR^2$ converges pointwise to a curve to a curve $\alpha_\infty\:[a,b]\to \RR^2$, then 
\[\liminf_{n\to\infty} \ell(\alpha_n) \ge \ell(\alpha_\infty).\]
\end{enumerate}
Then 
\[\ell(\alpha)=\length \alpha\eqlbl{eq:l=length}\] 
for any plane curve $\alpha$.

\end{thm}

\parit{Proof.}
Note that from normalization and additivity, the identity 
\[\ell(\beta)=\length \beta\eqlbl{eq:=poly}\]
holds for any polygonal line $\beta$ that is linear on each edge.

Note that the following two inequalities 
\[\ell(\alpha)\le\length \alpha\eqlbl{eq:l<length}\]
\[\ell(\alpha)\ge\length \alpha\eqlbl{eq:l>length}\]
imply \ref{eq:l=length}; we will prove them separately. 

Fix a curve $\alpha\:[a,b]\to \RR^2$ and a partition $a=t_0\z<t_1\z<\z\dots\z<t_k=b$. 
Consider the curve $\beta\:[a,b]\to \RR^2$ defined as the linear segment from $\alpha(t_i)$ to $\alpha(t_{i+1})$  on each interval $t\in[t_i,t_j]$.
By the definition of length, 
\[\length\beta\le \length\alpha.\]

Since the map  $\alpha\:[a,b]\to \RR^2$ is continuous,
one can find a sequence of partitions of $[a,b]$ such that the corresponding curves $\beta_n$ converge to $\alpha$ pointwise.
Applying the semi-continuity of $\ell$, \ref{eq:=poly} and the definition of length, we get that 
\begin{align*}
\ell(\alpha)&\le \liminf_{n\to\infty}\ell(\beta_n)=
\\
&=\liminf_{n\to\infty}\length\beta_n\le
\\
&\le\length \alpha.
\end{align*}
Hence \ref{eq:l<length} follows.

\begin{wrapfigure}{r}{36 mm}
\vskip-4mm
\centering
\includegraphics{mppics/pic-8}
\end{wrapfigure}

Note that a curve $\alpha\:[a,b]\to \RR^2$ with a partition $a=t_0\z<t_1\z<\z\dots\z<t_k=b$ can be considered as a concatenation
\[\alpha=\alpha_1*\alpha_2*\dots*\alpha_k\]
where $\alpha_i$ is the restriction of $\alpha$ to $[t_{i-1},t_i]$.

Observe that there is a sequence of motions $m_i$ of the plane so that 
\[m_i\circ\alpha(t_i)=m_{i+1}\circ\alpha(t_i)\] 
for any $i$ and 
the points 
\[m_1\circ\alpha(t_0), m_1\circ\alpha(t_1),\dots m_k\circ\alpha(t_k)\] 
lie in that order on a single line.
For the concatenation 
\[\gamma=(m_1\circ\alpha_1)*(m_2\circ\alpha_2)*\dots*(m_k\circ\alpha_k)\]
we have
\[\ell(\gamma)=\ell(\alpha).\]

Assume $\alpha$ is rectifiable.
In this case we can find a sequence of partitions of $[a,b]$ such that reparametrizations of  $\gamma_n$ converge to a linear segment $\gamma_\infty'$;
denote these reparametrizations by $\gamma'_n$.
Also, $\length\gamma'_\infty\z=\length\alpha$;
indeed, since $\gamma'_\infty$ is linear,
\begin{align*}
\length \gamma'_\infty&=|\gamma'_\infty (a)-\gamma'_\infty (b)|=
\\
&=\lim_{n\to\infty}\Sigma_n=
\\
&=\length\alpha.
\end{align*}
where $\Sigma_n$ is the sum in the definition of length for the $n$-th partition.
Hence it is sufficient to choose a sequence of partitions such that $\Sigma_n\z\to\length\alpha$.

Applying additivity, invariance of $\ell$ with respact to motions and reparametizations, we get that
\begin{align*}
\ell(\alpha)&=\lim_{n\to\infty}\ell(\gamma_n)=
\\
&=\lim_{n\to\infty}\ell(\gamma_n')\ge
\\
&\ge \ell(\gamma_\infty')=
\\
&=\length\alpha.
\end{align*}
Hence \ref{eq:l>length} follows.

If $\alpha$ is not rectifiable, a similar constriction produce an approximation of an arbitrary long line segment. (We need to run zig-zag to reduce the distance  $|\gamma'_\infty (a)-\gamma'_\infty (b)|$.)
It follows that 
\[\ell(\alpha)\ge |\gamma'_\infty (a)-\gamma'_\infty (b)|.\]
Since $|\gamma'_\infty (a)-\gamma'_\infty (b)|$ can take arbitrary large value,
we get $\ell(\alpha)=\infty$.
\qeds

\begin{thm}{Exercise} 
Construct a functional $\ell$ that satisfies all the conditions in 
 Proposition~\ref{prop:length-axioms} except the semi-continuity.
\end{thm}



\section{Crofton formula}

Let $\alpha$ be a plane curve and $u$ a unit vector.
Denote by $\alpha_u$ the orthogonal projection of $\alpha$ to a line $\ell$ in the direction of $u$;
that is, $\alpha_u(t)\in\ell$ and $\alpha(t)-\alpha_u(t)\perp \ell$ for any $t$.


\begin{thm}{Crofton formula}
The length of any plane curve $\alpha$ is proportional to the average of the lengths of its projections $\alpha_u$ for all unit vectors $u$.
Moreover for any plane curve $\alpha$ we have
\[\length\alpha=\tfrac\pi2\cdot\overline{\length \alpha_u},\]
where $\overline{\length \alpha_u}$ denotes the average value of $\length \alpha_u$.
\end{thm}

\parit{Proof.}
First let us show that the formula 
\[\length\alpha=k\cdot\overline{\length \alpha_u},\eqlbl{eq:crofton-k}\]
holds for some fixed coefficient $k$.
It will follow once we show that both sides of formula satisfy the length axioms in \ref{prop:length-axioms}.

The normalization can be achieved by adjusting~$k$.

The semi-continuity of the right hand side follows since $\length\alpha_u$ is semi-continuous and therefore the average has to be semi-continuous.

It is straightforward to check the remaining properties.


It remains to find $k$.
Let us apply the formula \ref{eq:crofton-k} to the unit circle.
The circle has length $2\cdot\pi$ and its projection to any line has length 4 --- it is a segment of length 2 traveled back and forth.
Evidently the average value is also $4$,
so \[2\cdot \pi=k\cdot 4 , \]
hence $k=\tfrac\pi2$.
\qeds

\parbf{Reformulation via number of intersections.}
Given a unit vector $u$ and a real number $\rho$,
consider the line of vectors $w$ on the plane satisfying the equation
\[\langle u, w\rangle =\rho,\]
where $\langle u, w\rangle$ denotes the scalar product. 
Any line on the plane admits exactly two such presentations with pairs $(u,\rho)$ and $(-u,-\rho)$.
A pair $(u,\rho)$ describes uniquely an \emph{oriented} line --- that is a line with a chosen unit normal vector.

Fix a unit vector $u_0$ and denote by $u(\phi)$ the result of rotating $u_0$ counterclockwise by the angle $\phi $. %pic here should be nice
Denote by $\ell(\phi,\rho)$ the oriented line asociated to the pair $(u(\phi),\rho)$.
To describe any line, we need a pair $(\phi,\rho)\in (-\pi,\pi]\times \RR$.

For a curve $\alpha$, set $n_\alpha(\phi,\rho)$ to be the number of parameter values $t$ such that $\alpha(t)$ lies on the line $\ell(\phi,\rho)$. 
The value $n_\alpha(\phi,\rho)$ is a nonnegative integer or $\infty$.
Note that if $\alpha$ is a simple curve, then $n_\alpha(\ell)$ is the number of intersections of $\alpha$ with $\ell$.


\begin{thm}{Another Crofton formula}
For any curve $\alpha$,
\[\length\alpha=\tfrac14\cdot\iint_{(-\pi,\pi]\times \RR} n_\alpha(\rho,\phi)\cdot d\rho\cdot d\phi.\]
the integral is to be understood in the sense of Lebesgue.
\end{thm}

By definition of average value,
\[\overline{\length\alpha_u}
=
\frac1{2\cdot\pi}\cdot\int_{-\pi}^\pi \length\alpha_{u(\phi)}\cdot d\phi.\]
Therefore the proof of this reformulation of the Crofton follows from the following observation.

\begin{thm}{Observation}
If $u=u(\phi)$, then 
\[\length\alpha_u=\int_\RR n_\alpha(\rho,\phi)\cdot d\rho;\]

\end{thm}

The proof is straightforward for those who understand Lebesgue integral.

\parbf{Variations.}
The same argument can be used to derive other formulas of the same type.
For example.

Recall that a big circle in a sphere is the intersection of the sphere with a plane passing thru its center.
For example, the equator as well as the meridians are big circles.

\begin{thm}{Spherical Crofton formula}\label{thm:crofton-sphere}
The length of any curve $\alpha$ in the unit sphere is $\pi$ times the average number of its crossings with big circles.

More presciently, given a unit vector $u$, denote by $n_\alpha(u)$ the number of crossings of $\alpha$ and the equator with pole at $u$.
Then 
\[\length\alpha=\pi\cdot\overline{n_\alpha(u)}.\]

Equivalently,
\[\length\alpha=\overline{\length\alpha_u},\]
where $\alpha_u$ denotes the curve obtianed by closest point projection of $\alpha$ to the equator with pole at $u$.
\end{thm}


\begin{thm}{Exercise}\label{ex:3d-crofton}
Come up with Crofton formulas for curves in the Euclidean space via projections to lines and to planes.
Find the coefficients in those formulas.
\end{thm}

\section{Applications}

\parit{Alternative proof of Proposition~\ref{prop:convex-curve}.}
Note that 
\[\length \beta_u\ge \length \alpha_u\]
for any unit vector $u$.
Indeed $\alpha_u$ runs back and forth along a line segment and $\beta_u$ has to run at least that much.

It follows that 
\[\overline{\length \beta_u}\ge \overline{\length \alpha_u}.\]
It remains to apply the Crofton formula.
\qeds


Recall that the diameter of a plane figure $F$ is defined as the least upper bound on the distances between pairs of its points;
that is,
\[\diam F=\sup\set{|x-y|}{x,y\in F}.\]

The equilateral triangle with side 1 gives an example of a convex figure of diameter 1 that cannot be covered by a round disc of diameter~1.

\begin{thm}{Exercise} 
Assume $F$ is a convex figure of diameter 1 and $D$ is the round disc of diameter 1.
Show that
\[\perim F\le \perim D.\]
\end{thm}

\begin{wrapfigure}{r}{33 mm}
\vskip-5mm
\centering
\includegraphics{mppics/pic-9}
\end{wrapfigure}

A convex figure $F$ has constant width $a$ if the orthogonal projection of $F$ to any line has length $a$.
There are many non-circular shapes of constant width. 
A nontrivial example is the Reuleaux triangle shown on the picture;
it is the intersection of three round disks of the same radius, each having its center on the boundary of the other two.
The following exercise is the so called Barbier's theorem.

\begin{thm}{Exercise} 
Show that figures with constant width $a$ have the same perimeter (which equals $\pi\cdot a$ --- the perimeter of the round disc of diameter $a$).
\end{thm}

\begin{thm}{Exercise}\label{ex:2pi-sphere}
Let $\gamma$ be a closed curve in the unit sphere of length shorter than $2\cdot\pi$.
Show that $\gamma$ lies in a hemisphere.
\end{thm}

\begin{thm}{Exercise} 
Let $\alpha$ be a closed curve of length $\pi$.
Show that it lies between a pair of parallel lines at distance $1$ from each other.
\end{thm}

\begin{thm}{Exercise}
A spaceship flies around a nonrotating planet of unit radius and comes back to the original position;
it was able to take a picture of every point on the surface of the planet.

Try to use the Crofton formulas to get a lower bound on the length of its trajectory (does not need to be exact, but should be larger than $2\cdot\pi$).

What do you think could be the shortest trajectory?
\end{thm}

The Hausdorff distance $d_H(F,G)$ between two closed bounded sets $F$ and $G$ in the plane is defined as the exact lower bound on $\eps>0$ such that the $\eps$-neightborhood of $F$ contains $G$ and the $\eps$-neightborhood of $G$ contains~$F$.

\begin{thm}{Exercise}\label{ex:perim-hausdorff}
Assume $F$ and $G$ are two closed convex figures on the plane such that $d_H(F,G)<\eps$.
Show that 
\[|\perim F-\perim G|<2\cdot\pi\cdot\eps.\]

\end{thm}

\begin{wrapfigure}{r}{33 mm}
\vskip-0mm
\centering
\includegraphics{mppics/pic-11}
\vskip0mm
\end{wrapfigure}

Given two sets $A$ and $B$ on the plane, the set $C$ is called their \emph{Minkowski sum}  (briefly $C=A+B$) if $C$ is formed by adding each vector in $A$ to each vector in $B$;
that is, 
\[C = \set{a+b}{a\in A,\ b\in B}.\]

Note that if $A$ and $B$ are convex then so is $C=A+B$. % I think this is trivial and should not be on the notes.

Indeed, $A$ is convex if and only if for any pair of points $a_0,a_1\in A$ and any $t\in[0,1]$,
the point $a_t=(1-t)\cdot a_0+t\cdot a_1$ belongs to $A$.
Similarly, $B$ is convex if and only if for any pair of points $b_0,b_1\in B$ and any $t\in[0,1]$,
the point $b_t=(1-t)\cdot b_0+t\cdot b_1$ belongs to $A$.

Fix a pair of points $c_0,c_1\in C$;
by the definition of Minkowski sum, there are two pairs of points $a_0,a_1\in A$ and $b_0,b_1\in B$ such that $c_0=a_0+b_0$ and $c_1=a_1+b_1$.
Then 
\begin{align*}
c_t&=(1-t)\cdot c_0+t\cdot c_1=
\\
&=(1-t)\cdot (a_0+b_0)+t\cdot(a_1+b_1)=
\\
&=[(1-t)\cdot a_0+t\cdot a_1]+[(1-t)\cdot b_0+t\cdot b_1]=
\\
&=a_t+b_t.
\end{align*}
That is, $c_t\in C$ for any $t\in [0,1]$, hence the result.


\begin{thm}{Exercise}\label{ex:perim+mink}
Show that 
\[\perim(A+B)=\perim A+\perim B\]
for any pair of convex figures in the plane.
\end{thm}

\begin{thm}{Exercise}
Use Exercise~\ref{ex:perim+mink} and Lemma~\ref{lem:perimeter} to give another solution of Exercise~\ref{ex:perim-hausdorff}.
\end{thm}

\begin{thm}{Exercise}
Let $\gamma$ be a curve that lies in a convex figure $F$ in the plane.

Let $\gamma$ be a curve that lies inside a convex figure $F$ on the plane.
Assume that
\[2\cdot \length \gamma\ge n\cdot \perim F\]
for some integer $n$.
Show that there is a line $\ell$ that intersects $\gamma$ in at least $n$ distinct points.
\end{thm}

