\chapter{Length}

The material of this and the following chapters overlaps largely with \cite[Chapter 5]{fuchs-tabachnikov}.

\section{Length of curve}


\begin{thm}{Definition}\label{def:curve}
Consider a \emph{plane curve}\index{plane curve} $\alpha\:[a,b]\z\to \RR^2$;
that is $\alpha$ is a continuous mapping from the real interval $[a,b]$ to the Euclidean plane $\RR^2$. 


If $\alpha(a)=p$ and $\alpha(b)=q$,
we say that $\alpha$ is a \emph{curve from $p$ to $q$}\index{curve from $p$ to $q$}.

A curve $\alpha\:[a,b]\to \RR^2$ is called \emph{closed} if $\alpha(a)=\alpha(b)$.

A curve $\alpha$ called \emph{simple} if it is described by an injective map;
that is $\alpha(t)=\alpha(t')$ if and only if $t=t'$.
However, a closed curve $\alpha\:[a,b]\to \RR^2$ is called simple if it is injective 
everywhere except the ends; that is, if
$\alpha(t)=\alpha(t')$ for $t<t'$ then $t=a$ and $t'=b$.
\end{thm}
 
Recall that a sequence 
\[a=t_0 < t_1 < \cdots < t_k=b.\]
is called \emph{partition} of interval $[a,b]$.

\begin{thm}{Definition}\label{def:length}
Let $\alpha\:[a,b]\to \RR^2$ be a curve.
The \emph{length}\index{length of curve} of $\alpha$ is defined as

\begin{align*}
\length \alpha
= 
\sup \{|\alpha(t_0)-\alpha(t_1)|&+|\alpha(t_1)-\alpha(t_2)|+\dots
\\
&\dots+|\alpha(t_{k-1})-\alpha(t_k)|\}. 
\end{align*}

where the exact upper bound is taken over all partitions
\[a=t_0 < t_1 < \cdots < t_k=b.\]

Note that $\length\alpha\in[0,\infty]$;
the curve $\alpha$ is called \emph{rectifiable}\index{rectifiable curve} if its length is finite.
\end{thm}

Informally, one could say that the length of curve is the exact upper bound of lengths of polygonal lines \emph{inscribed} in the curve.

\begin{thm}{Exercise}
Assume $\alpha\:[a,b]\to\RR^2$ is smooth curve, in particular the velocity vector $\alpha'(t)$ is defined and depends continuously on $t$.
Show that
\[\length \alpha=\int_a^b|\alpha'(t)|\cdot dt.\]
\end{thm}

\begin{thm}{Exercise}\label{ex:nonrectifiable-curve}
Construct a nonrectifiable curve $\alpha\:[0,1]\to\RR^2$.
\end{thm}

A closed simple plane curve is called \emph{convex} if it bounds a convex figure.

\begin{thm}{Proposition}\label{prop:convex-curve}
Assume a convex figure $A$ bounded by a curve $\alpha$ lies in a figure $B$ bounded by a curve $\beta$.
Then
\[\length\alpha\le \length\beta.\]
\end{thm}

Note that it is sufficient to show that for any polygon  $P$ inscribed in $\alpha$ there is a polygon $Q$ inscribed in $\beta$ such that 
$\perim P\le \perim Q$, where $\perim P$ denotes the perimeter of $P$.

Therefore it is sufficient to prove the following lemma.


\begin{thm}{Lemma}\label{lem:perimeter}
Let $P$ and $Q$ be polygons.
Assume $P$ is convex and $Q\supset P$.
Then $\perim P\le \perim Q$.
\end{thm}


\begin{wrapfigure}{r}{24 mm}
\vskip-4mm
\centering
\includegraphics{mppics/pic-7}
%\caption*{}
\end{wrapfigure}

\parit{Proof.}
Note that by triangle inequality,
the inequality
\[\perim P\le \perim Q\]

holds
if $P$ can be obtained from $Q$ by cutting it along a chord;
that is, a line segment with ends on the boundary of $Q$ that lies in $Q$.


Note that there is an increasing sequence of polygons 
$$P=P_0\subset P_1\subset\dots\subset P_n=Q$$
such that $P_{i-1}$ obtained from $P_{i}$ by cutting along a chord.
Therefore 
\begin{align*}
\perim P=\perim P_0&\le\perim P_1\le\dots
\\
\dots&\le\perim P_n=\perim Q
\end{align*}
and the lemma follows.
\qeds

\begin{thm}{Corollary}
Any convex closed curve is rectifiable.  
\end{thm}

\parit{Proof.}
Fix a curve $\alpha\:[a,b]\to\RR^2$.
Note that $\alpha$ is bounded; indeed  

Any closed curve is bounded; that is, it lies in a sufficiently large square.


By Proposition~\ref{prop:convex-curve}, the length of the curve can not exceed the perimeter of the square, hence the result.
\qeds



\warning 
\section{Semicontinuity of length}

Recall that lower limit 
of a sequence of real numbers $(x_n)$ is denoted by
\[\liminf_{n\to\infty} x_n.\] 
It is defined as the lowest partial limit; that is a the lowest possible limit of a subsequence of $(x_n)$.
The lower limit is defined for any sequence of real numbers and it takes value in the exteded real line $[-\infty,\infty]$


\begin{thm}{Theorem}\label{thm:length-semicont}
Length is a lower semi-continuous with respect to pointwise convergence of curves. 

More precisely, assume that a sequence
of curves $\alpha_n\:[a,b]\to \RR^2$ converges pointwise 
to a curve $\alpha_\infty\:[a,b]\to \RR^2$;
that is, $\alpha_n(t)\z\to\alpha_\infty(t)$ for any fixed $t\in[a,b]$ and $n\to\infty$. 
Then 
$$\liminf_{n\to\infty} \length\alpha_n \ge \length\alpha_\infty.\eqlbl{eq:semicont-length}$$
\end{thm}



\begin{wrapfigure}{r}{20 mm}
\vskip-0mm
\centering
\includegraphics{mppics/pic-6}
\end{wrapfigure}


Note that the inequality \ref{eq:semicont-length} might be strict.
For example the diagonal of unit square $\alpha_\infty$ 

can be  approximated by a sequence of stairs-like
polygonal curves $\alpha_n$
with sides parallel to the sides of the square ($\alpha_6$ is on the picture).
In this case
\[\length\alpha_\infty=\sqrt{2}\quad
\text{and}\quad \length\alpha_n=2\]
for any $n$.

\parit{Proof.}
Fix $\eps > 0$ and choose a sequence $a=t_0<t_1<\dots<t_k=b$
such that 
\begin{align*}
\length\alpha_\infty<
|\alpha_\infty(t_0)-\alpha_\infty(t_1)|+\dots+|\alpha_\infty(t_{k-1})-\alpha_\infty(t_k)|+\eps.
\end{align*}

Set 
\begin{align*}\Sigma_n
&\df
|\alpha_n(t_0)-\alpha_n(t_1)|+\dots+|\alpha_n(t_{k-1})-\alpha_n(t_k)|.
\\
\Sigma_\infty
&\df
|\alpha_\infty(t_0)-\alpha_\infty(t_1)|+\dots+|\alpha_\infty(t_{k-1})-\alpha_\infty(t_k)|.
\end{align*}

Note that $\Sigma_n\to \Sigma_\infty$ as $n\to\infty$
and $\Sigma_n\le\length\alpha_n$ for each $n$.
Hence
$$\liminf_{n\to\infty} \length\alpha_n \ge \length\alpha_\infty-\eps.$$
Since $\eps>0$ is arbitrary, we get \ref{eq:semicont-length}.\qeds

\section{Axioms of length}

\parbf{Concatenation.}
Assume $\alpha\:[a,b]\to \RR^2$ and $\beta\:[b,c]\z\to \RR^2$ are two curves such that $\alpha(b)=\beta(b)$.
Then one can combine these two curves in one $\gamma\:[a,c]\z\to \RR^2$ by assuming that $\gamma(t)=\alpha(t)$ for $t\le b$ and $\gamma(t)\ge\beta(t)$ for $t\ge b$.
The obtained curve $\gamma$ is called the 
\emph{concatenation} of $\alpha$ and $\beta$ which can be written as $\gamma=\alpha*\beta$.

\begin{wrapfigure}{r}{30 mm}
\vskip-0mm
\centering
\includegraphics{mppics/pic-10}
\end{wrapfigure}

Note that
\[\length(\alpha*\beta)=\length\alpha+\length\beta\]
for any two curves $\alpha$ and $\beta$ such that the concatenation $\alpha*\beta$ is defined.

\parbf{Reparametrization.}
Assume $\alpha\:[a,b]\to \RR^2$ is a curve and $\tau\:[c,d]\to [a,b]$ is a continuous strictly monotonic onto map.
Consider the curve $\alpha'\:[c,d]\to \RR^2$ defined by $\alpha'=\alpha\circ\tau$.
The curves $\alpha'$ is called \emph{reparametrization} of $\alpha$.

Note that 
\[\length\alpha'=\length\alpha\]
if $\alpha'$ is a reparametrization of $\alpha$.





\begin{thm}{Proposition}\label{prop:length-axioms}
Let $\ell$ be a functional that returns a value in $[0,\infty]$ for any curve $\alpha\:[a,b]\to\RR$.

Assume it satisfies the following properties:
\begin{enumerate}[(i)]
\item\label{Normalization} (Normalization) If $\alpha\:[a,b]\to \RR^2$ is a linear curve,%
\footnote{That is $\alpha=w+v\cdot t$ for some vectors $w$ and $v$.} then
\[\ell(\alpha)=|\alpha(a)-\alpha(b)|.\]
\item\label{Additivity} (Additivity) If the concatenation $\alpha*\beta$ is defined, then
\[\ell(\alpha*\beta)=\ell(\alpha)+\ell(\beta).\]
\item\label{Motion invariance} (Motion invariance) The functional $\ell$ is invariant with respect to the motions of the plane; that is, if $m$ is a motion then 
\[\ell(m\circ\alpha)=\ell(\alpha)\]
for any curve $\alpha$.
\item\label{Reparametrization invariance} (Reparametrization invariance) If $\alpha'$ is a reparametrization of a curve $\alpha$ then
\[\ell(\alpha')=\ell(\alpha).\]
(In fact linear reparametrizations will be sufficient.)
\item\label{Semi-continuity} (Semi-continuity) If a sequence of curves $\alpha_n\:[a,b]\to \RR^2$ converges to a curve pointwise to a curve $\alpha_\infty\:[a,b]\to \RR^2$, then 
\[\liminf_{n\to\infty} \ell(\alpha_n) \ge \ell(\alpha_\infty).\]
\end{enumerate}
Then 
\[\ell(\alpha)=\length \alpha\eqlbl{eq:l=length}\] 
for any plane curve $\alpha$.

\end{thm}

\parit{Proof.}
Note that from normalization and additivity, the identity 
\[\ell(\beta)=\length \beta\eqlbl{eq:=poly}\]
holds for any polyhonal line $\beta$ that is linear on each edge.

Note that the following two inequalities 
\[\ell(\alpha)\le\length \alpha\eqlbl{eq:l<length}\]
\[\ell(\alpha)\ge\length \alpha\eqlbl{eq:l>length}\]
imply \ref{eq:l=length}; we will prove them separately. 

Fix a curve $\alpha\:[a,b]\to \RR^2$ and a partition $a=t_0\z<t_1\z<\z\dots\z<t_k=b$. 
Consider the curve $\beta\:[a,b]\to \RR^2$ defined as a linear curve from $\alpha(t_i)$ to $\alpha(t_{i+1})$  on each segment $t\in[t_i,t_j]$.
By the definition of length, 
\[\length\beta\le \length\alpha.\]

Since the map  $\alpha\:[a,b]\to \RR^2$ is continuous,
one can find a sequence of partitions of $[a,b]$ such that the corresponding curves $\beta_n$ converge to $\alpha$ pointwise.
Applying semi-continuity of $\ell$, \ref{eq:=poly} and the definition of length, we get that 
\begin{align*}
\ell(\alpha)&\le \liminf_{n\to\infty}\ell(\beta_n)=
\\
&=\liminf_{n\to\infty}\length\beta_n\le
\\
&\le\length \alpha.
\end{align*}
Hence \ref{eq:l<length} follows.

\begin{wrapfigure}{r}{36 mm}
\vskip-4mm
\centering
\includegraphics{mppics/pic-8}
\end{wrapfigure}

Note that a curve $\alpha\:[a,b]\to \RR^2$ with a partition $a=t_0\z<t_1\z<\z\dots\z<t_k=b$ can be considered as a concatenation
\[\alpha=\alpha_1*\alpha_2*\dots*\alpha_k\]
where $\alpha_i$ is the restriction of $\alpha$ to $[t_{i-1},t_i]$.

Note that there is a sequence of motions $m_i$ of the plane so that 
\[m_i\circ\alpha(t_i)=m_{i+1}\circ\alpha(t_i)\] 
for any $i$ and 
the points 
\[m_1\circ\alpha(t_0), m_1\circ\alpha(t_1),\dots m_k\circ\alpha(t_k)\] 
appear on a line in the same order.
For the concatenation 
\[\gamma=(m_1\circ\alpha_1)*(m_2\circ\alpha_2)*\dots*(m_k\circ\alpha_k)\]
we have
\[\ell(\gamma)=\ell(\alpha).\]

Note that one can find a sequence of partitions of $[a,b]$ such that reparametrizationsof  $\gamma_n$ converge to linear curve $\gamma_\infty'$;
denote these reparametrizations by $\gamma'_n$.
We can assume in addition that $\length\gamma'_\infty\z=\length\alpha$;
indeed since $\gamma'_\infty$ is linear,
\begin{align*}
\length \gamma'_\infty&=|\gamma'(a)-\gamma'(b)|=
\\
&=\lim_{n\to\infty}\Sigma_n=
\\
&=\length\alpha.
\end{align*}
where $\Sigma_n$ is the sum in the definition of length for the $n$-th partition.
Hence it is sufficient to choose a sequence of partitions such that $\Sigma_n\z\to\length\alpha$ --- by the definition of length this is possible.

Applying additivity, invariance of $\ell$ with respact to motions and reparametizations, we get that
\begin{align*}
\ell(\alpha)&=\lim_{n\to\infty}\ell(\gamma_n)=
\\
&=\lim_{n\to\infty}\ell(\gamma_n')\ge
\\
&\ge \ell(\gamma_\infty')=
\\
&=\length\alpha.
\end{align*}
Hence \ref{eq:l>length} follows.
\qeds

\begin{thm}{Exercise} 
Construct a functional $\ell$ that is different from length and satisfies all the conditions in 
 Proposition~\ref{prop:length-axioms} except the semi-continuity.
\end{thm}



\section{Crofton formula}

Let $\alpha$ be a plane curve and $u$ is a unit vector.
Denote by $\alpha_u$ the orthogonal projection of $\alpha$ to a line $\ell$ in the direction of $u$;
that is, $\alpha_u(t)\in\ell$ and $\alpha(t)-\alpha_u(t)\perp \ell$ for any $t$.

\begin{thm}{Crofton formula}
The length of any plane curve $\alpha$ is proportional to the average of lengths of its projection $\alpha_u$ for all unit vectors $u$.
Moreover for any plane curve $\alpha$ we have
\[\length\alpha=\tfrac\pi2\cdot\overline{\length \alpha_u},\]
where $\overline{\length \alpha_u}$ denotes the average value of $\length \alpha_u$.
\end{thm}

\parit{Proof.}
First let us show that the formula 
\[\length\alpha=k\cdot\overline{\length \alpha_u},\eqlbl{eq:crofton-k}\]
holds for some fixed coefficient $k$.
It will follow once we show that both sides of formula satisfies the length axioms in \ref{prop:length-axioms}.

The normalization can be acheaved by adjusting~$k$.

The semi-continuity of the right hand side follows since $\length\alpha_u$ is semi-continuous and therefore the average has to be semi-continuous.

It is straightforward to check the remaining properties.


It remains to find $k$.
Let us apply the formula \ref{eq:crofton-k} to the unit circle.
The circle has length $2\cdot\pi$ and its projection to any line has length 4 --- it is a segment of length 2 traveled back and forth.
Evidently the average value is also $4$. 
Therefore \[2\cdot \pi=k\cdot 4\]
and therefore $k=\tfrac\pi2$.
\qeds

\parbf{Reformulation via number of intersections.}
Given a pair a unit vector $u$ and a real number $\rho$,
consider the line of vectors $w$ on the plane described by the equation
\[\langle u, w\rangle =\rho,\]
where $\langle u, w\rangle$ denotes the scalar product. 
Any line on the plane admits exactly two such presentations with pair $(u,\rho)$ and $(-u,-\rho)$.
A pair $(u,\rho)$ describes uniquely an \emph{oriented} line --- that is a line with chosen unit normal vector.

Fix a unit vector $u_0$ and denote by $u(\phi)$ its rotation by angle $\phi$.
Denote by $\ell(\phi,\rho)$ the oriented line for the pair $(u(\phi),\rho)$.
To describe any line, we need a pair $(\phi,\rho)\in (-\pi,\pi]\times \RR$.

For a curve $\alpha$  set $n_\alpha(\phi,\rho)$ to be the number of parameter values $t$ such that $\alpha(t)$ lies on the line $\ell(\phi,\rho)$. 
The value $n_\alpha(\phi,\rho)$ has to be nonnegative integer or $\infty$.
Note that if $\alpha$ is simple curve then $n_\alpha(\ell)$ is the number of intersections of $\alpha$ with $\ell$.


\begin{thm}{An other Crofton formula}
For any curve $\alpha$,
\[\length\alpha=\tfrac14\cdot\iint_{(-\pi,\pi]\times \RR} n_\alpha(\rho,\phi)\cdot d\rho\cdot d\phi.\]
the integral is to be understood in the sense of Lebesgue.
\end{thm}

By definition of average value,
\[\overline{\length\alpha_u}
=
\frac1{2\cdot\pi}\cdot\int_{-\pi}^\pi \length\alpha_{u(\phi)}\cdot d\phi.\]
Therefore  proof of this reformulation of the Crofton follows from the following observation.

\begin{thm}{Observation}
If $u=u(\phi)$, then 
\[\length\alpha_u=\int_\RR n_\alpha(\rho,\phi)\cdot d\rho;\]

\end{thm}

The proof is straightforward for those who understand Lebesgue integral.

\parbf{Variations.}
The same argument can be used to derive other formulas of the same type.
For example.

Recall that big circle in a sphere is the intersection of sphere with a plane passing thru its center.
For example equator as well as any meridian are a big circles.

\begin{thm}{Spherical Crofton formula}
The length of any curve $\alpha$ in the unit sphere is $\pi$ times the average number of its crossings with big circles.

More presciently, given a unit vector $u$, denote by $n_\alpha(u)$ the number of crossings of $\alpha$ and the equator with pole at $u$.
Then 
\[\length\alpha=\overline{n_\alpha(u)}.\]

Equivalently,
\[\length\alpha=\overline{\length\alpha_u},\]
where $\alpha_u$ denotes the curve obtianed by cloasest point projection of $\alpha$ on the equator with pole at $u$.
\end{thm}


\begin{thm}{Exercise}
Come up with a Crofton formulas for curves in the Euclidean space via projections to lines and to planes.
Find the coefficients in these formulas.
\end{thm}

\section{Applications}

\parit{Alternative proof of Proposition~\ref{prop:convex-curve}.}
Note that 
\[\length \beta_u\ge \length \alpha_u\]
for any unit vector $u$.
Indeed $\alpha_u$ runs back and forth along a line segment and $\beta_u$ has to run at least as much.

It follows that 
\[\overline{\length \beta_u}\ge \overline{\length \alpha_u}.\]
It remains to apply Crofton formula.
\qeds


Recall that diameter of a plane figure $F$ is defined as the least upper bound on the distances between pairs of its points;
that is,
\[\diam F=\sup\set{|x-y|}{x,y\in F}.\]

The equilateral triangle with side 1 gives an example of convex figure of diameter 1 that can not be covered by a round disc of diameter~1.

\begin{thm}{Exercise} 
Assume $F$ is a convex figure of diameter 1 and $D$ is the round disc of diameter 1.
Show that
\[\perim F\le \perim D.\]
\end{thm}

\begin{wrapfigure}{r}{33 mm}
\vskip-5mm
\centering
\includegraphics{mppics/pic-9}
\end{wrapfigure}

A convex figure $F$ has constant width $a$ if the orthogonal projection of $F$ to any line has length $a$.
There are many non-circular shapes of constant width. 
A nontrivial example is the Reuleaux triangle shown on the picture;
it is the intersection of three round disks, each having its center on the boundary of the other two.
The following exercise is the so called Barbier's theorem.

\begin{thm}{Exercise} 
Show that figures has constant width $a$ have the same perimeter (which is equal to $\pi\cdot a$ --- the perimeter of the round disc of diameter $a$).
\end{thm}

\begin{thm}{Exercise} 
Let $\gamma$ be a closed curve in the unit sphere of length smaller than $2\cdot\pi$.
Show that $\gamma$ lies in a hemisphere.
\end{thm}

\begin{thm}{Exercise} 
Let $\alpha$ be a closed curve of length $\pi$.
Show that it lies between a pair of parallel lines on distance $1$ from each other.
\end{thm}

\begin{thm}{Exercise}
A spaceship flies around nonrotating planet or unit radius and came back to the original position;
it was able to make a picture of every point on the surface of the planet.

Try to use the Crofton formulas to get a lower bound on the length of its trajectory (does not need to be exact, but should be bigger than $2\cdot\pi$).

What do you think could be the shortest trajectory?
\end{thm}

Hausdorff distance $d_H(F,G)$ between closed bounded sets $F$ and $G$ in the plane is defined as the exact lower bound on $\eps>0$ such that $\eps$-neightborhood of $F$ contains $G$ and $\eps$-neightborhood of $G$ contains~$F$.

\begin{thm}{Exercise}
Assume $F$ and $G$ be two closed convex figures on the plane such that $d_H(F,G)<\eps$.
Show that 
\[|\perim F-\perim G|<2\cdot\pi\cdot\eps.\]

\end{thm}

The Minkowski sum of two sets $A$ and $B$ in the plane $\RR^2$ is the set denoted by $A+B$ that is formed by adding each vector in $A$ to each vector in $B$;
that is, 
\[A + B = \set{a+b}{a\in A,\ b\in B}.\]

\begin{thm}{Exercise}
Show that 
\[\perim(A+B)=\perim A+\perim B\]
for any pair of convex figures in the plane.
\end{thm}

\begin{thm}{Exercise}
Let $\gamma$ be a curve that lies in a convex figure $F$ in the plane.
Assume that
\[2\cdot \length \gamma\ge n\cdot \perim F\]
for some integer $n$.
Show that there is a line $\ell$ that has $\gamma$  at least $n$ distinct points of intersection with $\gamma$.
\end{thm}

