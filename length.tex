

\chapter{Length of curves}



\begin{thm}{Definition}\label{def:curve}
A \emph{plane curve}\index{plane curve} is a continuous mapping $\alpha\:[a,b]\z\to \RR^2$,
where $\RR^2$ denotes the Euclidean plane. 
If $\alpha(a)=p$ and $\alpha(b)=q$,
we say that $\alpha$ is a \emph{curve from $p$ to $q$}\index{curve from $p$ to $q$}.



A curve $\alpha\:[a,b]\to \RR^2$ is called \emph{closed} if $\alpha(a)=\alpha(b)$.

A curve $\alpha$ called \emph{simple} if it is an injective map;
that is $\alpha(t)=\alpha(t')$ if and only if $t=t'$.

However, a closed curve $\alpha\:[a,b]\to \RR^2$ is called simple if it is injective 
everywhere except the ends; that is, if
$\alpha(t)=\alpha(t')$ for $t<t'$ then $t=a$ and $t'=b$.

\end{thm}

 





\begin{thm}{Definition}\label{def:length}
Let $\alpha\:[a,b]\to \RR^2$ be a curve.
The \emph{length}\index{length of curve} of $\alpha$ is defined as

\begin{align*}
\length \alpha
&= 
\sup \{|\alpha(t_0)-\alpha(t_1)|+|\alpha(t_1)-\alpha(t_2)|+\dots
\\
&\ \ \ \ \ \ \ \ \ \ \ \ \ \ \ \ \ \ \ \ \ \ \ \ \ \ \ \ \ \ \ \ \dots+|\alpha(t_{k-1})-\alpha(t_k)|\}. 
\end{align*}

where the exact upper bound is taken over all $k$ and all sequences

\[a=t_0 < t_1 < \cdots < t_k=b.\]

(Such sequences called \emph{partitions} of $[a,b]$.)



A curve is called \emph{rectifiable}\index{rectifiable curve} if its length is finite.

\end{thm}



In other words, the length of a curve is the exact upper bound of the lengths of polygonal lines \emph{inscribed} in the curve.



\begin{thm}{Exercise}
Assume $\alpha\:[a,b]\to\RR^2$ is a smooth curve, in particular, the velocity vector $\alpha'(t)$ is defined and depends continuously on $t$.

Show that

\[\length \alpha=\int_a^b|\alpha'(t)|\cdot dt.\]

\end{thm}



\begin{thm}{Exercise}\label{ex:nonrectifiable-curve}
Construct a nonrectifiable curve $\alpha\:[0,1]\to\RR^2$.

\end{thm}



A closed simple plane curve is called \emph{convex} if it bounds a convex figure.



\begin{thm}{Proposition}\label{prop:convex-curve}
Assume a convex figure $A$ bounded by a curve $\alpha$ lies inside a figure $B$ bounded by a curve $\beta$.
Then

\[\length\alpha\le \length\beta.\]

\end{thm}



Note that it is sufficient to show that for any polygon  $P$ inscribed in $\alpha$ there is a polygon $Q$ inscribed in $\beta$ with  
$\perim P\le \perim Q$, where $\perim P$ denotes the perimeter of $P$.
Therefore it is sufficient to prove the following lemma.



\begin{thm}{Lemma}\label{lem:perimeter}
Let $P$ and $Q$ be polygons.
Assume $P$ is convex and $Q\supset P$.
Then $\perim P\le \perim Q$.
\end{thm}



\begin{wrapfigure}{r}{24 mm}
\vskip-4mm
\centering

\includegraphics{mppics/pic-7}

%\caption*{}

\end{wrapfigure}



\parit{Proof.}
Note that by the triangle inequality,
the inequality

\[\perim P\le \perim Q\]

holds
if $P$ can be obtained from $Q$ by cutting it along a chord;
that is, the boundary curve of $P$ is formed by a part of boundary curve of $Q$ and one line segment lying in $Q$.



Note that there is an increasing sequence of polygons 

$$P=P_0\subset P_1\subset\dots\subset P_n=Q$$

such that $P_{i-1}$ is obtained from $P_{i}$ by cutting along a chord.
Therefore 

\begin{align*}
\perim P=\perim P_0&\le\perim P_1\le\dots
\\
\dots&\le\perim P_n=\perim Q
\end{align*}

and the lemma follows.

\qeds



\begin{thm}{Corollary}
Any convex closed curve is rectifiable.  
\end{thm}



\parit{Proof.}
Any closed curve is bounded; that is, it lies in a sufficiently large square.
By Proposition~\ref{prop:convex-curve}, the length of the curve can not exceed the perimeter of the square, hence the result.
\qeds


\warning 

\section{Semicontinuity of length}





\begin{thm}{Theorem}\label{thm:length-semicont}
Length is a lower semi-continuous with respect to point-wise convergence of curves. 
More precisely, assume that a sequence
of curves $\alpha_n\:[a,b]\to \RR^2$ converges point-wise 
to a curve $\alpha_\infty\:[a,b]\to \RR^2$;
that is, $\alpha_n(t)\z\to\alpha_\infty(t)$ for any fixed $t\in[a,b]$ as $n\to\infty$. 
Then 

$$\liminf_{n\to\infty} \length\alpha_n \ge \length\alpha_\infty,\eqlbl{eq:semicont-length}$$

where $\liminf_{n\to\infty}a_n$ denotes the lower limit; that is the lowest limit among all converging subsequences of $a_n$.
\end{thm}





\begin{wrapfigure}{r}{20 mm}
\vskip-0mm
\centering
\includegraphics{mppics/pic-6}
\end{wrapfigure}



Note that the inequality \ref{eq:semicont-length}%the numbering for the equations is preserved from the previous chapter and it does not coincide with what it should be
might be strict.
For example the diagonal of the unit square $\alpha_\infty$ 
can be  approximated by a sequence of stairs-like
polygonal curves $\alpha_n$
with sides parallel to the sides of the square,
$\alpha_6$ is on the picture.
In this case

\[\length\alpha_\infty=\sqrt{2}\quad
\text{and}\quad \length\alpha_n=2\]

for all $n$.



\parit{Proof.}
Fix $\eps > 0$ and choose a partition $a=t_0<t_1<\dots<t_k=b$
such that 

\begin{align*}
\length\alpha_\infty-
(|\alpha_\infty(t_0)-\alpha_\infty(t_1)|&+|\alpha_\infty(t_1)-\alpha_\infty(t_2)|+\dots
\\
&\dots+|\alpha_\infty(t_{k-1})-\alpha_\infty(t_k)|)<\eps
\end{align*}





Set 
\begin{align*}\Sigma_n
&\df
|\alpha_n(t_0)-\alpha_n(t_1)|+|\alpha_n(t_1)-\alpha_n(t_2)|+\dots
\\
&\ \ \ \ \ \ \ \ \ \ \ \ \ \ \ \ \ \ \ \ \ \ \ \ \ \ \ \ \ \ \ \ \dots+|\alpha_n(t_{k-1})-\alpha_n(t_k)|.
\\
\Sigma_\infty
&\df
|\alpha_\infty(t_0)-\alpha_\infty(t_1)|+|\alpha_\infty(t_1)-\alpha_\infty(t_2)|+\dots
\\
&\ \ \ \ \ \ \ \ \ \ \ \ \ \ \ \ \ \ \ \ \ \ \ \ \ \ \ \ \ \ \ \ \dots+|\alpha_\infty(t_{k-1})-\alpha_\infty(t_k)|.
\end{align*}

Note that $\Sigma_n\to \Sigma_\infty$ as $n\to\infty$
and $\Sigma_n\le\length\alpha_n$ for each $n$.
Hence

$$\liminf_{n\to\infty} \length\alpha_n \ge \length\alpha_\infty-\eps.$$

Since $\eps>0$ is arbitrary, we get \ref{eq:semicont-length}.\qeds%same here with the equations.



\section{Axioms of length}



Assume $\alpha\:[a,b]\to \RR^2$ and $\beta\:[b,c]\z\to \RR^2$ are two curves such that $\alpha(b)=\beta(b)$.
Then one can combine these two curves into one $\gamma\:[a,c]\z\to \RR^2$ by the rule $\gamma(t)=\alpha(t)$ for $t\le b$ and $\gamma(t)=\beta(t)$ for $t\ge b$.
The obtained curve $\gamma$ is called the 
\emph{concatenation} of $\alpha$ and $\beta$, and is denoted as $\gamma=\alpha*\beta$.

Note that if the concatenation $\alpha*\beta$ is defined, then

\[\length(\alpha*\beta)=\length\alpha+\length\beta.\]



Assume $\alpha\:[a,b]\to \RR^2$ is a curve and $\tau\:[c,d]\to [a,b]$ is a continuous strictly monotonic onto map.
Consider the curve $\alpha'\:[c,d]\to \RR^2$ defined by $\alpha'=\alpha\circ\tau$.
The curve $\alpha'$ is called a \emph{reparametrization} of $\alpha$.

Note that 

\[\length\alpha'=\length\alpha\]

whenever $\alpha'$ is a reparametrization of $\alpha$.









\begin{thm}{Proposition}\label{prop:length-axioms}
Let $\ell$ be a functional that returns a value in $[0,\infty]$ for any curve $\alpha\:[a,b]\to\RR$.
Assume it satisfies the following properties

\begin{enumerate}[(i)]

\item\label{Normalization} (Normalization) If $\alpha\:[a,b]\to \RR^2$ is a linear curve,%

\footnote{That is $\alpha=w+v\cdot t$ for some vectors $w$ and $v$.} then

\[\ell(\alpha)=|\alpha(a)-\alpha(b)|.\]

\item\label{Additivity} (Additivity) If the concatenation $\alpha*\beta$ is defined, then

\[\ell(\alpha*\beta)=\ell(\alpha)+\ell(\beta).\]

\item\label{Motion invariance} (Motion invariance) The functional $\ell$ is invariant with respect to the motions of the plane; that is, if $m$ is an isometry of the plane, then 

\[\ell(m\circ\alpha)=\ell(\alpha)\]

for any curve $\alpha$.

\item\label{Reparametrization invariance} (Reparametrization invariance) If $\alpha'$ is a reparametrization of a curve $\alpha$ then

\[\ell(\alpha')=\ell(\alpha).\]

(In fact linear reparametrizations will be sufficient.)

\item\label{Semi-continuity} (Semi-continuity) If a sequence of curves $\alpha_n\:[a,b]\to \RR^2$ converges pointwise to a curve to a curve $\alpha_\infty\:[a,b]\to \RR^2$, then 

\[\liminf_{n\to\infty} \ell(\alpha_n) \ge \ell(\alpha_\infty).\]

\end{enumerate}

Then $\ell(\alpha)=\length \alpha$ for any plane curve $\alpha$.



\end{thm}



\parit{Proof.}
Note that from normalization and additivity, the identity 

\[\ell(\beta)=\length \beta\eqlbl{eq:=poly}\]

holds for any polygonal line $\beta$ that is linear on each edge.



Fix a curve $\alpha\:[a,b]\to \RR^2$ and a partition $a=t_0\z<t_1\z<\z\dots\z<t_k=b$. 
Consider the curve $\beta\:[a,b]\to \RR^2$ defined as the linear segment from $\alpha(t_i)$ to $\alpha(t_{i+1})$  on each interval $t\in[t_i,t_j]$.

Note that 

\[\length\beta=|\alpha(t_0)-\alpha(t_1)|+\dots+|\alpha(t_{k-1})-\alpha(t_k)|.\]



Since the map  $\alpha\:[a,b]\to \RR^2$ is continuous,
one can find a sequence of partitions of $[a,b]$ such that the corresponding curves $\beta_n$ converge to $\alpha$ pointwise.
Applying the semi-continuity of $\ell$, \ref{eq:=poly}%equation numbering problem again
and the definition of length, we get that 

\begin{align*}
\ell(\alpha)&\le \liminf_{n\to\infty}\ell(\beta_n)=
\\
&=\liminf_{n\to\infty}\length\beta_n\le
\\
&\le\length \alpha.
\end{align*}



\begin{wrapfigure}{r}{36 mm}

\vskip-4mm

\centering

\includegraphics{mppics/pic-8}

\end{wrapfigure}



It remains to show the opposite inequality.

Note that a curve $\alpha\:[a,b]\to \RR^2$ with a partition $a=t_0\z<t_1\z<\z\dots\z<t_k=b$ can be considered as a concatenation

\[\alpha=\alpha_1*\alpha_2*\dots*\alpha_k\]

where $\alpha_i$ is the restriction of $\alpha$ to $[t_{i-1},t_i]$.



Observe that there is a sequence of motions $m_i$ of the plane so that $m_i\circ\alpha(t_i)=m_{i+1}\circ\alpha(t_i)$ for any $i$ and 
the points 

\[m_1\circ\alpha(t_0), m_1\circ\alpha(t_1),\dots m_k\circ\alpha(t_k)\] 

appear on a line in the same order.
Then for the concatenation 

\[\gamma=(m_1\circ\alpha_1)*(m_2\circ\alpha_2)*\dots*(m_k\circ\alpha_k)\]

we have

\[\ell(\gamma)=\ell(\alpha).\]



Note that one can find a sequence of partitions of $[a,b]$ such that reparametrizations $\gamma'_n$ of 
the corresponding curves $\gamma_n$ converge to a linear segment $\gamma_\infty'$. 
Also, $\length\gamma'_\infty=\length\alpha$;
indeed since $\gamma'_\infty$ is linear,

\begin{align*}
\length \gamma'_\infty&=|\gamma'(a)-\gamma'(b)|=
\\
&=\lim_{n\to\infty}\Sigma_n=
\\
&=\length\alpha.
\end{align*}

where $\Sigma_n$ is the sum in the definition of length for the $n$-th partition.



Applying the invariance of $\ell$ with respact to reparametizations, we get that

\begin{align*}
\ell(\alpha)&=\lim_{n\to\infty}\ell(\gamma_n)=
\\
&=\lim_{n\to\infty}\ell(\gamma_n')\ge
\\
&\ge \ell(\gamma_\infty')=
\\
&=\length\alpha.
\end{align*}

\qedsf

