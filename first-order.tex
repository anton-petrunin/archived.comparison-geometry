\chapter{First order structure}
\label{chap:first-order}
\section{Tangent plane}

\begin{thm}{Definition}\label{def:tangent-vector}
Let $\Sigma$ be a smooth surface.
A vector $\vec w$ is a \index{tangent vector}\emph{tangent vector} of $\Sigma$ at $p$ if and only if there is a curve $\gamma$ that runs in $\Sigma$ and has $\vec w$ as a velocity vector at $p$;
that is, $p=\gamma(t)$ and $\vec w=\gamma'(t)$ for some $t$.
\end{thm}

\begin{thm}{Proposition-Definition}\label{def:tangent-plane}
Let $\Sigma$ be a smooth surface and $p\in \Sigma$.
Then the set of tangent vectors of $\Sigma$ at $p$ forms a plane;
this plane is called the \index{tangent plane}\emph{tangent plane} of $\Sigma$ at $p$.

Moreover if $s\:U\to \Sigma$ is a local chart and $p=s(u_p,v_p)$, then 
the tangent plane of $\Sigma$ at $p$ is spanned by vectors $s_u(u_p,v_p)$ and $s_v(u_p,v_p)$.
\end{thm}

The tangent plane to $\Sigma$ at $p$ is usually denoted by $\T_p$ or $\T_p\Sigma$.
This plane $\T_p$ might be considered as a linear subspace of $\RR^3$ or as a parallel plane passing thru $p$;
the latter is sometimes called the \index{affine tangent plane}\emph{affine tangent plane}.
The affine tangent plane can be interpreted as the best approximation at~$p$ of the surface $\Sigma$ by a plane.
More precisely, 
it has \index{order of contact}\emph{first order of contact} with $\Sigma$ at $p$;
that is, $\rho(q)\z=o(|p-q|)$, where $q\in \Sigma$ and $\rho(q)$ denotes the distance from $q$ to $\T_p$.

\parit{Proof.}
Fix a chart $s$ at $p$.
Assume $\gamma$ is a smooth curve that starts at~$p$.
Without loss of generality, we can assume that $\gamma$ is covered by the chart;
in particular, there are smooth functions $u(t)$ and $v(t)$ such that 
\[\gamma(t)=s(u(t),v(t)).\]
Applying chain rule, we get
\[\gamma'=s_u\cdot u'+ s_v\cdot v';\]
that is, $\gamma'$ is a linear combination of $s_u$ and $s_v$.

Since the smooth functions $u(t)$ and $v(t)$ can be chosen arbitrarily, any linear combination of $s_u$ and $s_v$ is a tangent vector at $p$. 
\qeds


\begin{thm}{Exercise}\label{ex:tangent-normal}
Let $f:\RR^3\to\RR$ be a smooth function with $0$ as a regular value and $\Sigma$ be a surface described as a connected component of the set of solutions $f(x,y,z)=0$.
Show that the tangent plane $\T_p\Sigma$ is perpendicular to the gradient $\nabla_pf$ at any point $p\in\Sigma$.
\end{thm}

\begin{thm}{Exercise}\label{ex:vertical-tangent}
Let $\Sigma$ be a smooth surface and $p\in\Sigma$.
Choose $(x,y,z)$-coordinates.
Show that a neighborhood of $p$ in $\Sigma$ is a graph $z=f(x,y)$ of a smooth function $f$ defined on an open subset in the $(x,y)$-plane if and only if the tangent plane $\T_p$ is not {}\emph{vertical}; that is, if $\T_p$ is not perpendicular to the $(x,y)$-plane.
\end{thm}

\begin{thm}{Exercise}\label{ex:tangent-single-point}
Show that if a smooth surface $\Sigma$ meets a plane $\Pi$ at a single point $p$, then $\Pi$ is tangent to $\Sigma$ at $p$.
\end{thm}


\section{Directional derivative}

In this section we extend the definition of directional derivative to smooth functions defined on smooth surfaces.

First let us recall the standard definition of directional derivative.

Suppose $f$ is a function defined at a point $p$ in the space, and $\vec w$ a vector.
Consider the function
\[h(t)=f(p+t\cdot\vec w).\]
Then the directional derivative of $f$ at $p$ along $\vec w$ is defined as 
\[D_{\vec w}f(p)\df h'(0).\]
\index{10d@$D_{\vec{w}}f$}

\begin{thm}{Proposition-Definition}\label{def:directional-derivative}
Let $\Sigma$ be a smooth regular surface and $f$ a smooth function defined on $\Sigma$. 
Suppose $\gamma$ is a smooth curve in $\Sigma$ that starts at $p$ with velocity vector $\vec{w}\in \T_p$;
that is, $\gamma(0)=p$ and $\gamma'(0)=\vec{w}$.
Then the derivative $(f\circ\gamma)'(0)$
depends only on $f$, $p$ and $\vec{w}$;
it is called the \index{directional derivative}\emph{directional derivative of $f$ along $\vec{w}$ at $p$}
and denoted by
\[D_{\vec{w}}f,\quad D_{\vec{w}}f(p), \quad\text{or}\quad D_{\vec{w}}f(p)_\Sigma\] 
--- we may omit $p$ and $\Sigma$ if it is clear from the context.

Moreover, if $(u,v)\mapsto s(u,v)$ is a local chart at $p$, then 
\[D_{\vec{w}}f=a\cdot f_u+b\cdot f_v,\]
where $\vec{w}=a\cdot s_u +b\cdot s_v$. 
\end{thm}

Note that our definition agrees with the standard definition of directional derivative if $\Sigma$ is a plane.
Indeed, in this case $\gamma(t)=p+\vec w\cdot t$ is a curve in $\Sigma$ that starts at $p$ with velocity vector $\vec{w}$.
For a general surface the point $p+\vec w\cdot t$ might not lie on the surface; therefore the function $f$ might be undefined at this point; therefore the standard definition does not work.

\parit{Proof.}
Without loss of generality, we may assume that $\gamma$ is covered by the chart $s$;
if not we can chop $\gamma$.
In this case 
\[\gamma(t)=s(u(t),v(t))\]
for some smooth functions $u,v$ defined in a neighborhood of $0$ such that 
$u(0)=u_p$ and $v(0)=v_p$.

Applying the chain rule, we get that
\begin{align*}
\gamma'(0)&=u'(0)\cdot s_u+v'(0)\cdot s_v
\end{align*}
at $(u_p,v_p)$.
Since $\vec{w}=\gamma'(0)$ and the vectors $s_u$, $s_v$ are linearly independent, we get that $a=u'(0)$ and $b=v'(0)$.

Applying the chain rule again, we get that
\[
(f\circ\gamma)'(0)=a\cdot f_u+b\cdot f_v.
\eqlbl{eq:f-gamma}
\]
at $(u_p,v_p)$.

Notice that the left hand side in \ref{eq:f-gamma} does not depend on the choice of the chart $s$ and the right hand side depends only on $p$, $\vec w$, $f$, and $s$. 
It follows that $(f\circ\gamma)'(0)$ depends only on $p$, $\vec w$ and $f$.

The last statement follows from \ref{eq:f-gamma}.
\qeds

\section{Tangent vectors as functionals}

In this section we introduce a more conceptual way to define tangent vectors.
We will not use this approach in the sequel, but it is better to know about it.

A tangent vector $\vec w\in \T_p$ to a smooth surface $\Sigma$ 
defines a linear functional%
\footnote{Term \index{functional}\emph{functional} is used for functions that take a function as an argument and return a number.} $D_{\vec w}$
that swallows a smooth function $\phi$ defined in a neighborhood of $p$ in $\Sigma$ and spits its directional derivative $D_{\vec w}\phi$.
It is straightforward to check that the functional $D$ obeys the product rule:
\[D_{\vec w}(\phi\cdot\psi)=(D_{\vec w}\phi)\cdot \psi(p)+\phi(p)\cdot(D_{\vec w}\psi).
\eqlbl{eq:tangent-functional}\]

It is not hard to show that the tangent vector $\vec w$ is completely determined by the functional $D_{\vec w}$.
Moreover tangent vectors at $p$ can be \emph{defined} as linear functionals on the space of smooth functions
that satisfy the product rule \ref{eq:tangent-functional}.

This definition grabs the key algebraic property of tangent vectors.
It might be a less intuitive way to think about tangent vectors, but it is often convenient to use in the proofs. 
For example \ref{def:directional-derivative} becomes a tautology.

\section{Differential of map}

Any smooth map $s$ from a surface $\Sigma$ to $\RR^3$ can be described by its coordinate functions 
$ s(p)=(x(p),y(p),z(p))$. 
To take a directional derivative of the map we should take the  directional derivative of each of its coordinate functions.
\[D_{\vec{w}} s\df(D_{\vec{w}}x,D_{\vec{w}}y,D_{\vec{w}}z).\]

Assume $ s$ is a smooth map from one smooth surface $\Sigma_0$ to another $\Sigma_1$ and $p\in \Sigma_0$.
Note that $D_{\vec w} s(p)\in \T_{s(p)}\Sigma_1$ for any $\vec w\in \T_p$.
Indeed, choose a curve $\gamma_0$ in $\Sigma_0$ such that $\gamma_0(0)=p$ and $\gamma_0'(0)=\vec w$.
Observe that $\gamma_1= s\circ \gamma_0$ is a smooth curve in $\Sigma_1$ and 
by the definition of directional derivative, we have $D_{\vec w} s(p)=\gamma_1'(0)$.
It remains to note that $\gamma_1(0)\z= s(p)$ and therefore its velocity $\gamma_1'(0)$ is in $\T_{ s(p)}\Sigma_1$.

Recall that \ref{def:directional-derivative} implies that 
$d_p s\:\vec w \mapsto D_{\vec w} s$ defines a linear map $d_p s\:\T_p\Sigma_0\to \T_{ s(p)}\Sigma_1$;
that is,
\[D_{c\cdot \vec w} s=c\cdot D_{\vec w} s(p)
\quad\text{and}\quad D_{\vec v+ \vec w} s=D_{\vec v} s(p)+ D_{\vec w} s(p)\]
for any $c\in\RR$ and $\vec v, \vec w\in\T_p$.
The map $d_p s$ is called the \index{differential of map}\emph{differential} of $s$ at $p$.

The differential $d_p s$ can be described by a $2{\times}2$-matrix $M$ in orthonormal bases of $\T_p$ and $\T_{ s(p)}\Sigma_1$.
Set $\jac_p s=|\det M|$; this value  
does not depend on the choice of orthonormal bases in $\T_p$ and $\T_{ s(p)}\Sigma_1$. \label{page:|L|}\index{10d@$d_p s$}

Let $ s_1\:\Sigma_1\to\Sigma_2$ be another smooth map between smooth surfaces $\Sigma_1$ and $\Sigma_2$.
Suppose that ${p_1}= s(p)\in\Sigma_1$;
observe that 
\[d_p( s_1\circ s)=d_{p_1} s_1 \circ d_p s.\]
It follows that
\[\jac_p( s_1\circ s)
=
\jac_{p_1} s_1\cdot\jac_p s.\eqlbl{eq:jac-composition}\]


If $\Sigma_0$ is a domain in the $(u,v)$-plane, then the value $\jac_p s$ can be found using the following formulas 
\begin{align*}
\jac s
&=|s_v\times s_u|=
\\
&=\sqrt{\langle s_u, s_u\rangle\cdot\langle s_v, s_v\rangle -\langle s_u, s_v\rangle^2}=
\\
&=\sqrt{\det[\Jac^\top s\cdot \Jac s]}.
\end{align*}
where $\Jac s$ denotes the Jacobian matrix of $s$; it is a 2$\times$3 matrix with column vectors $s_u$ and $ s_v$.

The value $\jac_p s$ has the following geometric meaning:
if $P_0$ is a region in $\T_p$ and $P_1=(d_p s)(P_0)$, then
\[\area P_1=\jac_p s\cdot \area P_0.\]
This identity will become important in the definition of surface area.



\section{Surface integral and area}

Let $\Sigma$ be a smooth surface and $h\:\Sigma\to\RR$ be a smooth function.
Let us define the integral $\iint_R h$ of the function $h$ along a region $R\subset \Sigma$.
The definition will be used mostly along surfaces with boundary, but the definition can be applied to any Borel set $R\subset \Sigma$.

Recall that $\jac_ps$ is defined in the previous section.
Assume that there is a chart $(u,v)\mapsto s(u,v)$ of $\Sigma$ defined on an open set $U\subset\RR^2$ such that $R\subset s(U)$.
In this case set
\[\iint_R h\df \iint_{s^{-1}(R)} h\circ s(u,v)\cdot \jac_{(u,v)}s  \cdot du\cdot dv.\eqlbl{eq:area-def}\]


By the substitution rule (\ref{thm:mult-substitution}), the right hand side in \ref{eq:area-def} does not depend on the choice of $s$.
That is, if $s_1\:U_1\to \Sigma$ is another chart such that $s_1(U_1)\supset R$, then 
\[\iint_{s^{-1}(R)} h\circ s(u,v)\cdot \jac_{(u,v)}s  \cdot du\cdot dv=\iint_{s_1^{-1}(R)} h\circ s_1(u,v)\cdot \jac_{(u,v)}s_1  \cdot du\cdot dv.\]
In other words, the defining identity \ref{eq:area-def} makes sense.

A general region $R$ can be subdivided into regions $R_1,R_2\dots$ such that each $R_i$ lies in the image of some chart.
After that one could define the integral along $R$ as the sum
\[\iint_Rh
\df
\iint_{R_1}h+\iint_{R_2}h+\dots\]
It is straightforward to check that the value $\iint_Rh$ does not depend on the choice of such subdivision.

The area of a region $R$ in a smooth surface $\Sigma$ is defined as the surface integral 
\[\area R=\iint_R 1.\]

The following proposition provides a substitution rule for surface integral.

\begin{thm}{Area formula}\label{prop:surface-integral}
Suppose $ s\:\Sigma_0\to \Sigma_1$ is a smooth parametrization of a smooth surface $\Sigma_1$ by  a smooth surface $\Sigma_0$.
Then for any region $R\subset \Sigma_0$ and any smooth function $f\:\Sigma_1\to\RR$ we have
\[\iint_R (f\circ s)\cdot \jac  s=\int_{ s(R)}f.\]
In particular, if $f\equiv 1$, we have
\[\iint_R \jac  s=\area [ s(R)].\]

\end{thm}

\parit{Proof.}
Follows from \ref{eq:jac-composition} and the definition of surface integral.
\qeds

\parbf{Remark.}
The notion of area of a surface is closely related to the length of a curve.
However, to define length we use a different idea --- it was defined as the least upper bound on the lengths of inscribed polygonal lines.
It turns out that an analogous definition does not work even for very simple surfaces.
The latter is shown by a classical example --- the so-called \emph{Schwarz's boot}.
This example and different approaches to the notion of area are discussed in a popular article of Vladimir Dubrovsky \cite{dubrovsky}.

\section{Normal vector and orientation}
A unit vector that is normal to $\T_p$ is usually denoted by $\Norm(p)$;
it is uniquely defined up to sign.\index{10nu@$\Norm$}

A surface $\Sigma$ is called \index{oriented surface}\emph{oriented} if it is equipped with a unit normal vector field $\Norm$;
that is, a continuous map $p\mapsto \Norm(p)$ such that $\Norm(p)\perp\T_p$ and $|\Norm(p)|=1$ for any $p$.
The choice of the field $\Norm$ is called the {}\emph{orientation} of $\Sigma$.
A surface $\Sigma$ is called {}\emph{orientable} if it can be oriented.
Note that each orientable surface admits two orientations: $\Norm$ and $-\Norm$.

Let $\Sigma$ be a smooth oriented surface with unit normal field $\Norm$.
The map $\Norm\:\Sigma\to \mathbb{S}^2$ defined by $p\mapsto \Norm(p)$ is called the \index{spherical map}\emph{spherical map} or \index{Gauss map}\emph{Gauss map}.

For surfaces, the spherical map plays essentially the same role as the tangent indicatrix for curves.

The M\"obius strip shown on the diagram gives an example of a nonorientable surface --- there is no choice of normal vector field that is continuous along the middle of the strip (it changes the sign if you try to go around).

\begin{wrapfigure}{o}{42 mm}
\vskip-0mm
\centering
\includegraphics{asy/moebius}
\vskip0mm
\end{wrapfigure}

Note that each surface is locally orientable.
In fact each chart $s(u,v)$ admits an orientation 
\[\Norm=
\frac{s_u\times s_v}
{\left|s_u\times s_v\right|}.\]
Indeed, the vectors $s_u$ and $s_v$ are tangent vectors at $p$; 
since they are linearly independent, their vector product does not vanish and it is perpendicular to the tangent plane.
Evidently $(u,v)\mapsto \Norm(u,v)$ is a continuous map.
Therefore $\Norm$ is a unit normal field. 

\begin{thm}{Exercise}\label{ex:implicit-orientable}
Let $h:\RR^3\to\RR$ be a smooth function with $0$ as a regular value and $\Sigma$ a surface described as a connected component of the set of solutions $h(x,y,z)=0$.
Show that $\Sigma$ is orientable.
\end{thm}

Recall that any proper surface without boundary in the Euclidean space divides it into two connected components (\ref{clm:proper-divides}).
Therefore we can choose the unit normal field on any smooth proper surface that points into one of the components of the complement.
Therefore we obtain the following observation. 

\begin{thm}{Observation}
Any smooth proper surface in the Euclidean space is oriented.
\end{thm}

In particular it follows that the M\"obius strip cannot be extended to a proper smooth surface without boundary.

\section{Sections}

\begin{thm}{Advanced exercise}\label{ex:plane-section}
Let $\Pi$ be the $(x,y)$-plane and $A \subset \Pi$ be any closed subset. Show that there is an open smooth regular surface $\Sigma$ with $\Sigma \cap \Pi = A$.
\end{thm}

The exercise above says that plane sections of a smooth regular surface might look complicated.
The following lemma makes it possible to perturb the plane so that the section becomes nice.

\begin{thm}{Lemma}\label{lem:reg-section}
Let $\Sigma$ be a smooth regular surface.
Suppose $f\:\RR^3\z\to\RR$ is a smooth function.
Then for any constant $r_0$ there is an arbitrarily close value $r$ such that 
each connected component of the intersection of the level set $L_{r}=f^{-1}\{r\}$ with
$\Sigma$ is a smooth regular curve.
\end{thm}

\parit{Proof.}
The surface $\Sigma$ can be covered by a countable set of charts $s_i\:U_i\z\to \Sigma$.
Note that the composition $f\circ s_i$ is a smooth function for any $i$.
By Sard's lemma (\ref{lem:sard}), almost all real numbers $r$ are regular values for each $f\circ s_i$.

Fix such a value $r$ sufficiently close to $r_0$ and consider the level set $L_r$ described by the equation $f(x,y,z)=r$.
Any point in the intersection $\Sigma\cap L_r$ lies in the image of one of the charts.
From above it admits a neighborhood which is a regular smooth curve;
hence the result.\qeds

\begin{thm}{Corollary}
Let $\Sigma$ be a smooth surface.
Then for any plane $\Pi$ there is a parallel plane $\Pi^{*}$ that lies arbitrary close to $\Pi$ and such that the intersection $\Sigma\cap\Pi^{*}$ is a union of disjoint smooth curves.
\end{thm}



