\chapter{Isoperimetric inequality}

\warning

For any plane figure $F$ with perimeter $\ell$, its area $a$ satisfies the following inequality:
\[4\pi\cdot  a \le \ell^2.\eqlbl{eq:plane-isop}\]
Moreover the equality holds if $F$ is congruent to a round disc.

This is so-called \emph{isoperimetric inequality} on the plane.
Let us reformulate it without formulas, using the comparison language.

\begin{thm}{Isoperimetric inequality}
The area of plane figure bounded by a closed curve of length $\ell$ can not exceed the area of round disc with the same circumference $\ell$.
Moreover the equality holds only if the figure is congruent to the disc.
\end{thm}

The comparison reformulation has some advantages --- it is more intuitive and it is also easier to generalize. 

\begin{thm}{Exercise} Come up with a formulation of isoperimetric inequality on the units sphere. 
Try to reformulate it as an algebraic inequality similar to \ref{eq:plane-isop}.
\end{thm}

\begin{wrapfigure}{r}{58 mm}
\vskip-4mm
\centering
\includegraphics{mppics/pic-1}
%\caption*{}
\end{wrapfigure}

Recall that a plane figure $F$ is called \index{convex figure}\emph{convex} if for every pair of points $x,y\in F$, the line segment $[x,y]$ that joins the pair of points lies also in~$F$.


The following exercise reduces the isoperimetric inequality to the case of convex figures:

\begin{thm}{Exercise}
Assume $F$ is a plane figure bounded by a closed curve of length $\ell$.
Show that there is a convex figure $F'\supset F$ bounded by a closed curve of length at most $\ell$.
\end{thm}

The following problem is named after Dido, the legendary founder and first queen of Carthage.

\begin{thm}{Dido's problem}
A figure of the maximal area bounded by a straight line and a curve of given length with endpoints on that line is a half-disc.
\end{thm}

\begin{thm}{Exercise}\label{ex:dido-isop}
Show that Dido's problem follows from the isoperimetric inequality and the other way around.
\end{thm}


\begin{thm}{Exercise}
Use the isoperimetric inequality in the plane to show that 
inscribed convex polygons have maximal area among all the polygons with the given sides.
\end{thm}

\begin{thm}{Exercise}
Find the minimal length of curve that divides the unit square in the given ratio $\alpha$.  
\end{thm}

\section{Lawlor's proof}

Here we present a sketch of the proof of Dido's problem based of the idea of Gary Lawlor in \cite{lawlor}.
Before getting into the proof, try to solve the following exercise.

{

\begin{wrapfigure}{r}{34 mm}
\vskip-4mm
\centering
\includegraphics{mppics/pic-2}
%\caption*{}
\end{wrapfigure}

\begin{thm}{Exercise}
An old man walks along a trail around a convex meadows and pulls a brick on a rope of unit length (the rope is always strained).
After walking around he noticed that the brick is at the same position as at the beginning.
Show that the area between the trail and the path of the brick equals to the area of the unit disc. 

\end{thm}

}

\parit{Sketch of the proof.}
Let $F$ be a convex figure bounded by a line and a curve $\gamma(t)$ of length $\ell$;
we can assume that $\gamma$ is a unit speed curve so the set of parameters is $[0,\ell]$.

Imagine that we are walking along the curve with a stick of length $r$ so that the other end of the stick drags on the flow.
Assume that at initially $t=0$ the stick points in the direction of $\gamma(\ell)$ --- the other end of $\gamma$.

\begin{wrapfigure}{r}{34 mm}
\vskip-0mm
\centering
\includegraphics{mppics/pic-3}
%\caption*{}
\bigskip
\includegraphics{mppics/pic-4}
%\caption*{}
\bigskip
\includegraphics{mppics/pic-5}
\end{wrapfigure}

Note that is $r$ is small then most of the time we drag the stick behind. Therefore at the end of walk the stick will make more than half turn and will point to the same side of the figure.

Let $R$ be the radius of half-circle $\tilde\gamma(t)$ of length $\ell$.
Assume we walk along $\tilde\gamma$  with a stick as or length $R$ the same way as described above.
Note that the other end does not move (it always lies in the center) and the direction of stick changes with rate $\tfrac1R$.
Note further for $\gamma$ this rate would be at most $\tfrac1R$.
Therefore after walking along $\gamma$,
the stick of length $R$ will rotate at most as much as if we would walk along $\tilde\gamma$.

It follows that there is a positive value $r\z\le R$ such that after walking along $\gamma$, a stick of length $r$ will rotate exactly half turn, so it will point in the direction of $\gamma(0)$.

Imagine that the stick of length $r$ is covered with red paint and it paints the area below it.
If we move with velocity $v$ then angular velocity (radians per second) of the stick is at most $\tfrac vr$ the equality hold if we move perpendicularly to the stick.
Therefore we paint at the rate at most most $\tfrac12 v\cdot r$.

If we do the same for the half-disc of radius $R$ and stick of length $R$ with blue paint,
then we paint area of disc without overlap with the rate $\tfrac12 v\cdot R$.
Since $r\le R$, the total red-painted area can not exceed the blue-painted area, that is, $D$. 

It remains to show that all $F$ is red-painted.
Fix a point $p\in F$.
Notice that at the beginning the point $p$ lies on the left from the stick and at the end it lies on the right form it.
Therefore it will be a moment of time $t_0$ when the sides change from left to right.
At this time the point must be on the line containing the stick. 
Moreover if it lies on the extension then the sides change from right to left. Therefore $p$ have to lie under the stick; that is, $p$ is painted.
\qeds

\begin{thm}{Exercise}
Find the places with cheating in the proof above and try to fix them.
\end{thm}

\begin{thm}{Exercise} Read and understand the original proof of Gary Lawlor in \cite{lawlor}.
\end{thm}



\chapter{Length of curves}

\begin{thm}{Definition}\label{def:curve}
A \emph{plane curve}\index{plane curve} is a continuous mapping $\alpha\:[a,b]\z\to \RR^2$,
where $\RR^2$ denotes the Euclidean plane. 

If $\alpha(a)=p$ and $\alpha(b)=q$,
we say that $\alpha$ is a \emph{curve from $p$ to $q$}\index{curve from $p$ to $q$}.

A curve $\alpha\:[a,b]\to \RR^2$ is called \emph{closed} if $\alpha(a)=\alpha(b)$.

A curve $\alpha$ called \emph{simple} if it is described by an injective map;
that is $\alpha(t)=\alpha(t')$ if and only if $t=t'$.
However, a closed curve $\alpha\:[a,b]\to \RR^2$ is called simple if it is injective 
everywhere except the ends; that is, if
$\alpha(t)=\alpha(t')$ for $t<t'$ then $t=a$ and $t'=b$.
\end{thm}
 
Recall that a sequence 
\[a=t_0 < t_1 < \cdots < t_k=b.\]
is called \emph{partition} of interval $[a,b]$.

\begin{thm}{Definition}\label{def:length}
Let $\alpha\:[a,b]\to \RR^2$ be a curve.
The \emph{length}\index{length of curve} of $\alpha$ is defined as
\begin{align*}
\length \alpha
&= 
\sup \{|\alpha(t_0)-\alpha(t_1)|+|\alpha(t_1)-\alpha(t_2)|+\dots
\\
&\ \ \ \ \ \ \ \ \ \ \ \ \ \ \ \ \ \ \ \ \ \ \ \ \ \ \ \ \ \ \ \ \dots+|\alpha(t_{k-1})-\alpha(t_k)|\}. 
\end{align*}
where the exact upper bound is taken over all partitions
\[a=t_0 < t_1 < \cdots < t_k=b.\]

A curve is called \emph{rectifiable}\index{rectifiable curve} if its length is finite.
\end{thm}

Informally, one could say that the length of curve is the exact upper bound of lengths of polygonal lines \emph{inscribed} in the curve.

\begin{thm}{Exercise}
Assume $\alpha\:[a,b]\to\RR^2$ is smooth curve, in particular the velocity vector $\alpha'(t)$ is defined and depends continuously on $t$.
Show that
\[\length \alpha=\int_a^b|\alpha'(t)|\cdot dt.\]
\end{thm}

\begin{thm}{Exercise}\label{ex:nonrectifiable-curve}
Construct a nonrectifiable curve $\alpha\:[0,1]\to\RR^2$.
\end{thm}

A closed simple plane curve is called \emph{convex} if it bounds a convex figure.

\begin{thm}{Proposition}\label{prop:convex-curve}
Assume a convex figure $A$ bounded by a curve $\alpha$ lies in a figure $B$ bounded by a curve $\beta$.
Then
\[\length\alpha\le \length\beta.\]
\end{thm}

Note that it is sufficient to show that for any polygon  $P$ inscribed in $\alpha$ there is a polygon $Q$ inscribed in $\beta$ such that 
$\perim P\le \perim Q$, where $\perim P$ denotes the perimeter of $P$.

Therefore it is sufficient to prove the following lemma.

\begin{thm}{Lemma}\label{lem:perimeter}
Let $P$ and $Q$ be polygons.
Assume $P$ is convex and $Q\supset P$.
Then $\perim P\le \perim Q$.
\end{thm}

\begin{wrapfigure}{r}{24 mm}
\vskip-4mm
\centering
\includegraphics{mppics/pic-7}
%\caption*{}
\end{wrapfigure}

\parit{Proof.}
Note that by triangle inequality,
the inequality
\[\perim P\le \perim Q\]
holds
if $P$ can be obtained from $Q$ by cutting it along a chord;
that is, the boundary curve of $P$ is formed by a part of boundary curve of $Q$ and one line segment lying in $Q$.

Note that there is an increasing sequence of polygons 
$$P=P_0\subset P_1\subset\dots\subset P_n=Q$$
such that $P_{i-1}$ obtained from $P_{i}$ by cutting along a chord.
Therefore 
\begin{align*}
\perim P=\perim P_0&\le\perim P_1\le\dots
\\
\dots&\le\perim P_n=\perim Q
\end{align*}
and the lemma follows.
\qeds

\begin{thm}{Corollary}
Any convex closed curve is rectifiable.  
\end{thm}

\parit{Proof.}
Any closed curve is bounded; that is, it lies in a sufficiently large square.
By Proposition~\ref{prop:convex-curve}, the length of the curve can not exceed the perimeter of the square, hence the result.
\qeds

\section{Semicontinuity of length}


\begin{thm}{Theorem}\label{thm:length-semicont}
Length is a lower semi-continuous with respect to point-wise convergence of curves. 

More precisely, assume that a sequence
of curves $\alpha_n\:[a,b]\to \RR^2$ converges point-wise 
to a curve $\alpha_\infty\:[a,b]\to \RR^2$;
that is, $\alpha_n(t)\z\to\alpha_\infty(t)$ for any fixed $t\in[a,b]$ and $n\to\infty$. 
Then 
$$\liminf_{n\to\infty} \length\alpha_n \ge \length\alpha_\infty,\eqlbl{eq:semicont-length}$$
where $\liminf_{n\to\infty}a_n$ denotes the lower limit; that is the lowest limit of subsequences of $a_n$.
\end{thm}


\begin{wrapfigure}{r}{20 mm}
\vskip-0mm
\centering
\includegraphics{mppics/pic-6}
\end{wrapfigure}

Note that the inequality \ref{eq:semicont-length} might be strict.
For example the diagonal of unit square $\alpha_\infty$ 
can be  approximated by a sequence of stairs-like
polygonal curves $\alpha_n$
with sides parallel to the sides of the square,
$\alpha_6$ is on the picture.
In this case
\[\length\alpha_\infty=\sqrt{2}\quad
\text{and}\quad \length\alpha_n=2\]
for all $n$.

\parit{Proof.}
Fix $\eps > 0$ and choose a sequence $a=t_0<t_1<\dots<t_k=b$
such that 
\begin{align*}
\length\alpha_\infty-
(|\alpha_\infty(t_0)-\alpha_\infty(t_1)|&+|\alpha_\infty(t_1)-\alpha_\infty(t_2)|+\dots
\\
&\dots+|\alpha_\infty(t_{k-1})-\alpha_\infty(t_k)|)<\eps
\end{align*}


Set 
\begin{align*}\Sigma_n
&\df
|\alpha_n(t_0)-\alpha_n(t_1)|+|\alpha_n(t_1)-\alpha_n(t_2)|+\dots
\\
&\ \ \ \ \ \ \ \ \ \ \ \ \ \ \ \ \ \ \ \ \ \ \ \ \ \ \ \ \ \ \ \ \dots+|\alpha_n(t_{k-1})-\alpha_n(t_k)|.
\\
\Sigma_\infty
&\df
|\alpha_\infty(t_0)-\alpha_\infty(t_1)|+|\alpha_\infty(t_1)-\alpha_\infty(t_2)|+\dots
\\
&\ \ \ \ \ \ \ \ \ \ \ \ \ \ \ \ \ \ \ \ \ \ \ \ \ \ \ \ \ \ \ \ \dots+|\alpha_\infty(t_{k-1})-\alpha_\infty(t_k)|.
\end{align*}
Note that $\Sigma_n\to \Sigma_\infty$ as $n\to\infty$
and $\Sigma_n\le\length\alpha_n$ for each $n$.
Hence
$$\liminf_{n\to\infty} \length\alpha_n \ge \length\alpha_\infty-\eps.$$
Since $\eps>0$ is arbitrary, we get \ref{eq:semicont-length}.\qeds

\section{Axioms of length}

Assume $\alpha\:[a,b]\to \RR^2$ and $\beta\:[b,c]\z\to \RR^2$ are two curves such that $\alpha(b)=\beta(b)$.
Then one can combine these two curves in one $\gamma\:[a,c]\z\to \RR^2$ the assuming that $\gamma(t)=\alpha(t)$ for $t\le b$ and $\gamma(t)\ge\beta(t)$ for $t\ge b$.
The obtained curve $\gamma$ is called the 
\emph{concatenation} of $\alpha$ and $\beta$ which can be written as $\gamma=\alpha*\beta$.
Note that if the concatenation $\alpha*\beta$ is defined, then
\[\length(\alpha*\beta)=\length\alpha+\length\beta.\]

Assume $\alpha\:[a,b]\to \RR^2$ is a curve and $\tau\:[c,d]\to [a,b]$ is a continuous strictly monotonic onto map.
Consider the curve $\alpha'\:[c,d]\to \RR^2$ defined by $\alpha'=\alpha\circ\tau$.
The curves $\alpha'$ is called \emph{reparametrization} of $\alpha$.
Note that 
\[\length\alpha'=\length\alpha\]
if $\alpha'$ is a reparametrization of $\alpha$.




\begin{thm}{Proposition}\label{prop:length-axioms}
Let $\ell$ be a functional that returns a value in $[0,\infty]$ for any curve $\alpha\:[a,b]\to\RR$.
Assume it satisfies the following properties
\begin{enumerate}[(i)]
\item\label{Normalization} (Normalization) If $\alpha\:[a,b]\to \RR^2$ is a linear curve,%
\footnote{That is $\alpha=w+v\cdot t$ for some vectors $w$ and $v$.} then
\[\ell(\alpha)=|\alpha(a)-\alpha(b)|.\]
\item\label{Additivity} (Additivity) If the concatenation $\alpha*\beta$ is defined, then
\[\ell(\alpha*\beta)=\ell(\alpha)+\ell(\beta).\]
\item\label{Motion invariance} (Motion invariance) The functional $\ell$ is invariant with respect to the motions of the plane; that is, if $m$ is a motion then 
\[\ell(m\circ\alpha)=\ell(\alpha)\]
for any curve $\alpha$.
\item\label{Reparametrization invariance} (Reparametrization invariance) If $\alpha'$ is a reparametrization of a curve $\alpha$ then
\[\ell(\alpha')=\ell(\alpha).\]
(In fact linear reparametrizations will be sufficient.)
\item\label{Semi-continuity} (Semi-continuity) If a sequence of curves $\alpha_n\:[a,b]\to \RR^2$ converges to a curve pointwise to a curve $\alpha_\infty\:[a,b]\to \RR^2$, then 
\[\liminf_{n\to\infty} \ell(\alpha_n) \ge \ell(\alpha_\infty).\]
\end{enumerate}
Then $\ell(\alpha)=\length \alpha$ for any plane curve $\alpha$.

\end{thm}

\parit{Proof.}
Note that from normalization and additivity, the identity 
\[\ell(\beta)=\length \beta\eqlbl{eq:=poly}\]
holds for any polyhonal line $\beta$ that is linear on each edge.

Fix a curve $\alpha\:[a,b]\to \RR^2$ and a partition $a=t_0\z<t_1\z<\z\dots\z<t_k=b$. 
Consider the curve $\beta\:[a,b]\to \RR^2$ defined as a linear curve from $\alpha(t_i)$ to $\alpha(t_{i+1})$  on each segment $t\in[t_i,t_j]$.
Note that 
\[\length\beta=|\alpha(t_0)-\alpha(t_1)|+\dots+|\alpha(t_{k-1})-\alpha(t_k)|.\]

Since the map  $\alpha\:[a,b]\to \RR^2$ is continuous,
one can find a sequence of partitions of $[a,b]$ such that the corresponding curves $\beta_n$ converge to $\alpha$ pointwise.
Applying semi-continuity of $\ell$, \ref{eq:=poly} and the definition of length, we get that 
\begin{align*}
\ell(\alpha)&\le \liminf_{n\to\infty}\ell(\beta_n)=
\\
&=\liminf_{n\to\infty}\length\beta_n\le
\\
&\le\length \alpha.
\end{align*}

\begin{wrapfigure}{r}{36 mm}
\vskip-4mm
\centering
\includegraphics{mppics/pic-8}
\end{wrapfigure}

It remains to prove the opposite inequality.
Note that a curve $\alpha\:[a,b]\to \RR^2$ with a partition $a=t_0\z<t_1\z<\z\dots\z<t_k=b$ can be considered as a concatenation
\[\alpha=\alpha_1*\alpha_2*\dots*\alpha_k\]
where $\alpha_i$ is the restriction of $\alpha$ to $[t_{i-1},t_i]$.

Note that there is a sequence of motions $m_i$ of the plane so that $m_i\circ\alpha(t_i)=m_{i+1}\circ\alpha(t_i)$ for any $i$ and 
the points 
\[m_1\circ\alpha(t_0), m_1\circ\alpha(t_1),\dots m_k\circ\alpha(t_k)\] 
appear on a line in the same order.
Note that for the concatenation 
\[\gamma=(m_1\circ\alpha_1)*(m_2\circ\alpha_2)*\dots*(m_k\circ\alpha_k)\]
we have
\[\ell(\gamma)=\ell(\alpha).\]

Note that one can find a sequence of partitions of $[a,b]$ such that reparametrizations $\gamma'_n$ of 
the corresponding curves $\gamma_n$ converge to linear curve $\gamma_\infty'$.
We can assume in addition that $\length\gamma'_\infty=\length\alpha$;
indeed since $\gamma'_\infty$ is linear,
\begin{align*}
\length \gamma'_\infty&=|\gamma'(a)-\gamma'(b)|=
\\
&=\lim_{n\to\infty}\Sigma_n=
\\
&=\length\alpha.
\end{align*}
where $\Sigma_n$ is the sum in the definition of length for the $n$-th partition.

Applying invariance of $\ell$ with respact to reparametizations, we get that
\begin{align*}
\ell(\alpha)&=\lim_{n\to\infty}\ell(\gamma_n)=
\\
&=\lim_{n\to\infty}\ell(\gamma_n')\ge
\\
&\ge \ell(\gamma_\infty')=
\\
&=\length\alpha.
\end{align*}
\qedsf




\section{Crofton formula}

Let $\alpha$ be a plane curve and $u$ is a unit vector.
Denote by $\alpha_u$ the orthogonal projection of $\alpha$ to a line $\ell$ in the direction of $u$;
that is, $\alpha_u(t)\in\ell$ and $\alpha(t)-\alpha_u(t)\perp \ell$ for any $t$.

\begin{thm}{Crofton formula}
The length of any plane curve $\alpha$ is proportional to the average of lengths of its projection $\alpha_u$ for all unit vectors $u$.
Moreover for any plane curve $\alpha$ we have
\[\length\alpha=\tfrac\pi2\cdot\overline{\length \alpha_u},\]
where $\overline{\length \alpha_u}$ denotes the average value of $\length \alpha_u$.
\end{thm}

\parit{Proof.}
First let us show that the formula 
\[\length\alpha=k\cdot\overline{\length \alpha_u},\eqlbl{eq:crofton-k}\]
holds for some fixed coefficient $k$.
It will follow once we show that both sides of formula satisfies the length axioms in \ref{prop:length-axioms}.

The semi-continuity of the right hand side follows since $\length\alpha_u$ is semi-continuous and therefore the average has to be semi-continuous.
It is straightforward to check the remaining properties exept the normalization; the normalization needs a coefficient which we denote by~$k$.

It remains to find $k$.
Let us apply \ref{eq:crofton-k} to the unit circle.
The circle has length $2\cdot\pi$ and its projection to any line has length 4 --- it is a segment of length 2 traveled back and forth.
Therefore \[2\cdot \pi=k\cdot 4;\]
hence $k=\tfrac\pi2$.
\qeds

Let us use Crofton formula to give an other proof of Proposition~\ref{prop:convex-curve}.

\parit{Alternative proof of Proposition~\ref{prop:convex-curve}.}
Note that 
\[\length \beta_u\ge \length \alpha_u\]
for any unit vector $u$.
Indeed $\alpha_u$ runs back and forth along a line segment and $\beta_u$ has to run at least as much.

It follows that 
\[\overline{\length \beta_u}\ge \overline{\length \alpha_u}.\]
It remains to apply Crofton formula.
\qeds

An oriented line can be characterized by the angle $w$ to the chosen direction  and the distance $\rho$ from the origin taken with sign --- it is positive if $w$ points counterclockwise, negative if clockwise.
The vector $w$ can be described by angle to a fixed direction.
Hence any oriented line can be described by a pair $(\phi,\rho)\in (-\pi,\pi]\times \RR$; denote this line by $\ell(\phi,\rho)$.

For a curve $\alpha$  set $n_\alpha(\phi,\rho)$ to be the number of parameter values $t$ such that $\alpha(t)$ lies on the line $\ell(\phi,\rho)$. 
The value $n_\alpha(\phi,\rho)$ has to be nonnegative integer or $\infty$.
Note that if $\alpha$ is simple curve then $n_\alpha(\ell)$ is the number of intersections of $\alpha$ with $\ell$.


\begin{thm}{Reformulation of Crofton formula}
For any curve $\alpha$,
\[\length\alpha=\int_{(-\pi,\pi]\times \RR} n_\alpha(\rho,\phi)\cdot d\rho\cdot d\phi.\]
\end{thm}

The proof of this reformulation of the Crofton formula relies on the following observation.

\begin{thm}{Observation}
If $u\perp \ell(\phi,\rho)$, then 
\[\length\alpha_u=\int_\RR n_\alpha(\rho,\phi)\cdot d\rho.\]
\end{thm}
   

\begin{thm}{Exercise}
Come up with a Crofton formula for curves in the unit sphere.
\end{thm}

Recall that diameter of a plane figure $F$ is defined as the least upper bound on the distances between pairs of its points;
that is,
\[\diam F=\sup\set{|x-y|}{x,y\in F}.\]

The equilateral triangle with side 1 gives an example of convex figure of diameter 1 that can not be covered by a round disc of diameter~1.

\begin{thm}{Exercise} 
Assume $F$ is a convex figure of diameter 1 and $D$ is the round disc of diameter 1.
Show that
\[\perim F\le \perim D.\]
\end{thm}

A convex figure $F$ has constant width $a$ if the orthogonal projection of $F$ to any line has length $a$.
There are many non-circular shapes of constant width. 
A nontrivial example is the Reuleaux triangle shown on the picture.

\begin{thm}{Exercise} 
Show that figures has constant width $a$ have the same perimeter, which is equal to $\pi\cdot a$.
\end{thm}

\begin{thm}{Exercise} 
Let $\gamma$ be a closed curve in the unit sphere of length smalle than $2\cdot\pi$.
Show that $\gamma$ lies in a hemisphere.
\end{thm}

\begin{thm}{Exercise} 
Let $\alpha$ be a closed curve of length $\pi$.
Show that it lies between a pair of parallel lines on distance $1$ from each other.
\end{thm}







