\chapter{Isoperimetric inequality}

\warning

For any plane figure $F$ with perimeter $\ell$, its area $a$ satisfies the following inequality:
\[4\pi\cdot  a \le \ell^2.\eqlbl{eq:plane-isop}\]
Moreover the equality holds if $F$ is congruent to a round disc.

This is so-called \emph{isoperimetric inequality} on the plane.
Let us reformulate it without formulas, using the comparison language.

\begin{thm}{Isoperimetric inequality}
The area of plane figure bounded by a closed curve of length $\ell$ can not exceed the area of round disc with the same circumference $\ell$.
Moreover the equality holds only if the figure is congruent to the disc.
\end{thm}

The comparison reformulation has some advantages --- it is more intuitive and it is also easier to generalize. 

\begin{thm}{Exercise} Come up with a formulation of isoperimetric inequality on the units sphere. 
Try to reformulate it as an algebraic inequality similar to \ref{eq:plane-isop}.
\end{thm}

\begin{wrapfigure}{r}{58 mm}
\vskip-4mm
\centering
\includegraphics{mppics/pic-1}
%\caption*{}
\end{wrapfigure}

Recall that a plane figure $F$ is called \index{convex figure}\emph{convex} if for every pair of points $x,y\in F$, the line segment $[x,y]$ that joins the pair of points lies also in~$F$.


The following exercise reduces the isoperimetric inequality to the case of convex figures:

\begin{thm}{Exercise}
Assume $F$ is a plane figure bounded by a closed curve of length $\ell$.
Show that there is a convex figure $F'\supset F$ bounded by a closed curve of length at most $\ell$.
\end{thm}

The following problem is named after Dido, the legendary founder and first queen of Carthage.

\begin{thm}{Dido's problem}
A figure of the maximal area bounded by a straight line and a curve of given length with endpoints on that line is a half-disc.
\end{thm}

\begin{thm}{Exercise}\label{ex:dido-isop}
Show that Dido's problem follows from the isoperimetric inequality and the other way around.
\end{thm}


\begin{thm}{Exercise}
Use the isoperimetric inequality in the plane to show that 
inscribed convex polygons have maximal area among all the polygons with the given sides.
\end{thm}

\begin{thm}{Exercise}
Find the minimal length of curve that divides the unit square in the given ratio $\alpha$.  
\end{thm}

\section{Lawlor's proof}

Here we present a sketch of the proof of Dido's problem based of the idea of Gary Lawlor in \cite{lawlor}.
Before getting into the proof, try to solve the following exercise.

{

\begin{wrapfigure}{r}{34 mm}
\vskip-4mm
\centering
\includegraphics{mppics/pic-2}
%\caption*{}
\end{wrapfigure}

\begin{thm}{Exercise}
An old man walks along a trail around a convex meadows and pulls a brick on a rope of unit length (the rope is always strained).
After walking around he noticed that the brick is at the same position as at the beginning.
Show that the area between the trail and the path of the brick equals to the area of the unit disc. 

\end{thm}

}

\parit{Sketch of the proof.}
Let $F$ be a convex figure bounded by a line and a curve $\gamma(t)$ of length $\ell$;
we can assume that $\gamma$ is a unit speed curve so the set of parameters is $[0,\ell]$.

Imagine that we are walking along the curve with a stick of length $r$ so that the other end of the stick drags on the flow.
Assume that at initially $t=0$ the stick points in the direction of $\gamma(\ell)$ --- the other end of $\gamma$.

\begin{wrapfigure}{r}{34 mm}
\vskip-0mm
\centering
\includegraphics{mppics/pic-3}
%\caption*{}
\bigskip
\includegraphics{mppics/pic-4}
%\caption*{}
\end{wrapfigure}

Note that is $r$ is small then most of the time we drag the stick behind. Therefore at the end of walk the stick will make more than half turn and will point to the same side of the figure.

Let $R$ be the radius of half-circle $\tilde\gamma(t)$ of length $\ell$.
Assume we walk along $\tilde\gamma$  with a stick as or length $R$ the same way as described above.
Note that the other end does not move (it always lies in the center) and the direction of stick changes with rate $\tfrac1R$.
Note further for $\gamma$ this rate would be at most $\tfrac1R$.
Therefore after walking along $\gamma$,
the stick of length $R$ will rotate at most as much as if we would walk along $\tilde\gamma$.

It follows that there is a positive value $r\le R$ such that after walking along $\gamma$, a stick of length $r$ will rotate exactly half turn, so it will point in the direction of $\gamma(0)$.

Let us show that the area of $F$ can not exceed the area of half-disc $D$ or radius $r$;
since $r\le R$, the latter implies Dido's problem.

Imagine that the stick is covered with paint and it paints the area below it.
Notice that to color maxmimal area one has to move perpendicularly to the stick.
Therefore the total area colored after walking along $\gamma$ can not exeed the area of $D$.

It remains to show that all $F$ is painted.
Fix a point $p\in F$.
Notice that at the beginning the point $p$ lies on the left from the stick and at the end it lies on the right form it.
Therefore it will be a moment of time $t_0$ when the sides change from left to right.
At this time the point must be on the line containing the stick. 
Moreover if it lies on the extension then the sides change from right to left. Therefore $p$ have to lie under the stick; that is, $p$ is painted.
\qeds

\begin{thm}{Exercise}
Find the places with cheating in the proof above and try to fix them.
\end{thm}

\begin{thm}{Exercise} Read and understand the original proof of Gary Lawlor in \cite{lawlor}.
\end{thm}


