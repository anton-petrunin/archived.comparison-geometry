\chapter{Geodesics}

The following exercise might look like a hard problem in calculus, but actually it is an easy problem in geometry.

%???angle of inclination

\begin{thm}{Exercise}
There is a mountain of frictionless ice in the shape of a perfect cone with a circular base.
A cowboy is at the bottom and he wants to climb the mountain.
So, he throws up his lasso which slips neatly over the top of the cone, he pulls it tight and starts to climb.
If the mountain is very steep, with a narrow angle $\theta$ at the top, there is no problem; the lasso grips tight and up he goes.
On the other hand if the mountain is very flat, with a very shallow angle $\theta$ at the top, the lasso slips off as soon as the cowboy pulls on it.

What is the critical angle $\theta_0$ at which the cowboy can no longer climb the ice-mountain?
\end{thm}

\section{Closest point projection}

\begin{thm}{Lemma}
Let $K$ be a closed convex set in $\RR^3$.
Then for every point $p\in\RR^3$ there is unique point $\bar p\in K$ that minimize the distance $|p-x|$ for all points $x\in K$.

Moreover the map $p\mapsto \bar p$ is short;
that is,
\[|p-q|\ge|\bar p-\bar q| \eqlbl{eq:short-cpp}\]
for any pair of points $p,q\in \RR^3$.
\end{thm}

The map $p\mapsto \bar p$ is called \emph{closest point projection};
it maps Euclidean space to $K$.
Note that if $p\in K$, then $\bar p=p$.

\parit{Proof.}
Fix a point $p$ and set 
\[\ell=\inf_{x\in K}\{|p-x|\}.\]
Choose a sequence $x_n\in K$ such that $|p-x_n|\to \ell$ as $n\to\infty$.

Without loss of generality, we can assume that all the points $x_n$ lie in a ball or radius $\ell+1$ centered at $p$.
Therefore we can pass to a partial limit $\bar p$ of $x_n$; that is, $\bar p$ is a limit of a subsequence of $x_n$.
Since $K$ is closed $\bar p\in K$.
By construction 
\[|p-\bar p|=\ell=\lim_{n\to\infty}|p-x_n|.\]
Hence the existence follows.

\begin{wrapfigure}{i}{22 mm}
\vskip-0mm
\centering
\includegraphics{mppics/pic-40}
\vskip-0mm
\end{wrapfigure}

Assume there are two distinct points $\bar p, \bar p'\in K$ that minimize the distance to $p$.
Since $K$ is convex, their midpoint $m=\tfrac12\cdot (\bar p+\bar p')$ lies in~$K$.
Note that $|p-\bar p|\z=|p-\bar p'|=\ell$; that is $\triangle p\bar p\bar p'$ is isosceles and therefore $\triangle p\bar p m$ is right with the right angle at $m$.
Since leg of right triangle is shorter than its hypotenuse, we have $|p-m|<\ell$ --- a contradiction. 

It remains to prove inequality \ref{eq:short-cpp}.

We can assume that $\bar p\ne\bar q$, otherwise there is nothing to prove.
Note that if $p\ne \bar p$ (that is, if $p\notin K$), 
then $\angle p \bar p \bar q$ is right or obtuse.
Otherwise there would be a point $x$ on the line segment $[\bar q,\bar p]$ that is closer to $p$ than $\bar p$.
Since $K$ is convex, the line segment $[\bar q,\bar p]$ and therefore $x$ lie in $K$.
Hence $\bar p$ is not closest to $p$ --- a contradiction.

\begin{wrapfigure}{o}{37 mm}
\vskip-4mm
\centering
\includegraphics{mppics/pic-41}
\vskip-0mm
\end{wrapfigure}

The same way we can show that  if $q\z\ne \bar q$, then $\angle q \bar q \bar p$ is right or obtuse.
In all cases it implies that the orthogonal projection of the line segment $[p,q]$ to the line $\bar p\bar q$ contains the line segment $[\bar p,\bar q]$.
In particular \[|p-q|\ge |\bar p-\bar q|.\]
\qedsf

\section{Geodesics}

\begin{wrapfigure}{r}{25 mm}
\begin{lpic}[t(-0 mm),b(-4 mm),r(0 mm),l(0 mm)]{pics/convex-hat(1)}
\lbl{15,8.6;$\Sigma$}
\lbl[w]{5,18.6;$\Delta$}
\end{lpic}
\end{wrapfigure}

Let $\Sigma$ be a surface. 
Assume a curve $\gamma$ in $\Sigma$ connects two points $p,q\in \Sigma$ and
minimizes the lenght among all such curves.
Then $\gamma$ is called a \emph{minimizing geodesic} from $p$ to $q$.


\begin{thm}{Exercise}
Suppose $\Sigma$ is a smooth closed surface that bounds a convex body $K$ 
in $\RR^3$ 
and $\Pi$ is a plane that cuts from $\Sigma$ a disk $\Delta$.
Assume that the reflection of $\Delta$ with respect to $\Pi$ lies inside of $\Sigma$.
Show that $\Delta$ is \index{convex set}\emph{convex} with respect to the intrinsic metric  of $\Sigma$;
that is, 
if both ends of a minimizing geodesic in $\Sigma$ 
lie in $\Delta$,
then the whole geodesic lies in $\Delta$.
\end{thm}

Recall that \emph{diameter} of a set $K$ in the Euclidean space is defined as the exact uppper bound on the distances between pairs of points in $K$.
Let us define \emph{intrinsic diameter} of a closed surface $\Sigma$ as the exact upper bound on the lengths of minimizing geodesic in the surface.

\begin{thm}{Exercise}\label{ex:intrinsic-diameter}
Assume that a closed smooth surface $\Sigma$ bounds a convex body $K$ of diameter $D$.
\begin{itemize}
\item Show that intrinsic diameter of can exceed $\pi\cdot D$.
\item Show that the area of $\Sigma$ can not exceed the area of sphere of radius $D$.
\end{itemize}


\end{thm}

\begin{wrapfigure}{r}{39 mm}
\begin{lpic}[t(-0 mm),b(-4 mm),r(0 mm),l(0 mm)]{pics/unbend(1)}
\lbl[t]{11.5,29;$p$}
\lbl[r]{10.5,37;$p_t$}
\lbl[t]{32.5,26;$q$}
\lbl[t]{26,30;$\gamma(t)$}
\lbl{20,13;{\Large $\Sigma$}}
\end{lpic}
\end{wrapfigure}


\begin{thm}{Exercise}
Let $\Sigma$ be a smooth closed strictly convex surface 
in $\RR^3$ 
and $\gamma\:[0,\ell]\z\to \Sigma$ be a unit-speed minimizing geodesic.
Set $p\z=\gamma(0)$, $q=\gamma(\ell)$ and 
$$p_t=\gamma(t)-t\cdot\gamma'(t),$$ 
where $\gamma'(t)$ denotes the velocity vector of $\gamma$ at $t$.

Show that for any $t\in (0,\ell)$,
one {}\emph{cannot see}  $q$ from $p_t$;
that is, the line segment $[p_tq]$ intersects $\Sigma$ at a point distinct from $q$.%
\footnote{Hint: Show that the concatenation of the line segment $[p_t\gamma(t)]$ and the arc $\gamma|_{[t,\ell]}$ is a minimizing geodesic in the closed set $W$ outside of $\Sigma$.}
\end{thm}

\section{Liberman's lemma}

A curve $\gamma\:[a,b]\to \Sigma$ is called \emph{geodesic}  if for some partition $a\z=t_0<t_1<\dots <t_n=b$ of the interval the each arc $\gamma|_{[t_{i-1},t_{i+1}]}$ is a minimizing geodesic.

The following lemma provides was proved by Joseph Liberman \cite{liberman}.

\begin{thm}{Liberman's lemma}
Assume $\gamma$ is a geodesic on the graph $z=f(x,y)$ of a concave function $f$ defined on an open subset of the plane.
Consider a reparametrization $(x(t),y(t),z(t)$ of $\gamma$ such that the curve  $t\mapsto (x(t),y(t))$ is a unit-speed curve.
Then $z(t)$ is a concave function.
\end{thm}

If we draw a line parallel to the $z$-axis thru each point of $\gamma$, we get a surface which can be developed on the plane --- that is, it can be parametrized by a strip in the plane between parallel lines so that the length of all curves in the strip survive after the mapping.
If we assume that the strip is oriented vertically on the plane  then the curve becomes a graph of a function and the theorem states that this function is concave.

\parit{Proof.}
Denote the graph by $\Sigma$.
Choose a partition such that $\gamma|_{[t_{i-1},t_{i+1}]}$ is minimizing.
If the function $z$ is convex on each interval $[t_{i-1},t_{i+1}]$, then it is convex on whole interval.
Therefore it is sufficient to prove the case if $\gamma\:[a,b]\to $ is a minimizing geodesic.

Further, passing to a finer partition, we can assume that the projection of $\gamma$ to the $(x,y)$-plane lies completely in a closed disc $\Delta$ in the domain of definition of $f$;
moreover the distance from  the projection of $\gamma$ to the boundary of disc is much larger than length of $\gamma$.
In this case the curve lies in the boundary of closed convex set 
\[K=\set{(x,y,z)\in \RR^3}{(x,y)\in\Delta,\ z\le f(x,y)};\]
so we can apply the lemma on closest point projection.

\begin{figure}[h!]
\centering
 \begin{lpic}[t(-2 mm),b(-0 mm),r(0 mm),l(0 mm)]{pics/liberman(1)}
\lbl[t]{25,8;$t$}
\lbl[r]{1,27;$z$}
\lbl[r]{36,27;$z$}
\lbl[bl]{55.5,3.5;$x$}
\lbl[rb]{53,16;$y$}
\lbl[br]{62,22;$\gamma$}
\end{lpic}
\end{figure}

If the function $z$ is not concave, then there is an other function $\check z\ge z$ with shorter graph such that $\check z(a)=z(a)$, $\check z(a)=z(a)$.
Consider the curve $\check\gamma(t)=(x(t),y(t),\check z(t)$;
$\check\gamma$ lies higher than $\gamma$ and therefore can on the boundary or outside of $K$.
The closest point projection of $\check \gamma$ to $K$ gives a curve connecting endpoints of $\gamma$, by construction it runs in $\Sigma$ and by the lemma on closest point projection it is a shorter than $\gamma$ --- a contradiction. 
\qeds

\begin{wrapfigure}{o}{30 mm}
\vskip-4mm
\centering
\includegraphics{mppics/pic-42}
\vskip-0mm
\end{wrapfigure}

\begin{thm}{Exercise}
Assume $\gamma$ is a minimizing geodesic on a smooth closed convex surface $\Sigma$ and $p$ in the interior of convex set bounded by $\Sigma$.
Consider the cone $C$ with the tip at $p$ with ruling half-lines passing thru the points of $\gamma$.
Let us develop $C$ on the plane; the curve $\gamma$ becomes a plane curve $\tilde\gamma$ and the tip of the cone is mapped to a point~$\tilde p$.

Show that $\tilde\gamma$ is convex toward to $\tilde p$; that is for any sufficiently small arc $\tilde x\tilde y$ of $\tilde\gamma$, the curvelinear triangle $\tilde p\tilde x\tilde y$ is convex. 
\end{thm}



\section{Bound on total curvature}




\begin{thm}{Theorem}\label{thm:usov}
Assume $\Sigma$ is a graph $z=f(x,y)$ of a convex $\ell$-Lipschitz function $f$ defined on an open set in the $(x,y)$-plane.
Then the total curvature of any geodesic in $\Sigma$ is at most $2\cdot \ell$.
\end{thm}

The following theorem was proved by Vladimir Usov \cite{usov},
later David Berg \cite{berg} pointed out that the same proof works for the geodesics in closed epigraphs of $\ell$-Lipschitz functions which are not necessary concave; that is, the set of the type 
\[W=\set{(x,y,z)\in\RR^3}{z\ge f(x,y)}\]


Recall that by definition has constant speed.
In the proof we will use the following clam without proof.

\begin{thm}{Claim}\label{clm:gamma''}
Let $\gamma$ be a geodesic in a smooth regular surface $\Sigma$.
Then $\gamma$ is a smooth regular curve or it is constant.
Moreover, for any $t$ the second derivative $\gamma''(t)$ is normal to the tangent plane $\T_{\gamma(t)}\Sigma$.
\end{thm}

The first statement in the proof should be intuitively obvious.
Let us give an informal physical explanation of the second statement about $\gamma''(t)$.

One may think about geodesic $\gamma$ as of a stable position of a stretched elastic thread that is forced to lie on a frictionless surface.
Since it is frictionless, the force density $N(t)$ that keeps geodesic $\gamma$ in the surface must be therefore proportional to the normal vector to the surface at $\gamma(t)$.
The tension in the thread has to be the same at all points (otherwise the thread would move back or forth and would not be stable).
The tension at the ends of small arc is roughly proportional to the angle between the tangent lines at the ends of the arc. 
Passing to the limit as the length of arc goes to zero, we get that density of this force $F(t)$ is proportional to $\gamma''(t)$.
According to the second Newton's law, we have $F(t)+N(t)=0$;
which implies that  $\gamma''(t)$ is perpendicular to $\T_{\gamma(t)}\Sigma$.%
\footnote{In fact $\gamma''(t)+\nu\cdot \langle s(\gamma'(t)),\gamma'(t)\rangle=0$, where $s$ is the shape operator of $\Sigma$ at $\gamma(t)$ or equivalently,
$\gamma''(t)+\nu\cdot  \II(\gamma'(t)),\gamma'(t))=0$, where $\II$ is the second fundamental form of $\Sigma$ at $\gamma(t)$.}

\begin{thm}{Lemma}\label{lem:unit-speed} Assume $f\:[a,b]\to\RR$ is a smooth  function. 
Let $(x(t),y(t))$ be a unit-speed parametrization of graph $y=f(x)$.
Then $f$ is concave if and only if the function $t\mapsto y(t)$ is concave.
\end{thm}

\parit{Proof.} 
We can assume that the function $t\mapsto x(t)$ is increasing.

Note that 
\[y'(t)=\frac{f'(x(t))}{\sqrt{1+(f'(x(t)))^2}}.\]
It follows that  $y'(t)$ is nonincreasing if and only if $f'(x)$ is nonincreasing, hence the result.
\qeds

\parit{Proof of \ref{thm:usov}.}
Let $\gamma(t)=(x(t),y(t),z(t))$ be a geodesic on $\Sigma$ with a unit-speed parametrization.
According to Liberman's lemma and Lemma~\ref{lem:unit-speed},
$z(t)$ is concave.

Since the slope of $f$ is at most $\ell$, we have
\[|z'(t)|\le \tfrac{\ell}{\sqrt{1+\ell^2}}.\]
If $\gamma$ is defined on the interval $[a,b]$, then
\begin{align*}
\int_a^b |z''(t)|&=z'(a)-z'(b)\le 
\\
&\le 2\cdot \tfrac{\ell}{\sqrt{1+\ell^2}}.
\end{align*}

Further note that $z''$ is the projection of $\gamma''$ to the $z$-axis.
By 
\qeds

\begin{thm}{Exercise} Suppose a smooth surface $\Sigma$ bounds a convex set $K$ in the Euclidean spece.
Assume $K$ contains a unit ball and has diameter $D$.
Find an upper bound on the total curvatures of minimizing geodesics in $\Sigma$.\footnote{Hint: Use Exercise~\ref{ex:intrinsic-diameter}.}
\end{thm}

In fact there is a fixed bound on the total curvature of any minimizing geodesic on any closed convex surface \cite{lebedeva-petrunin}.

