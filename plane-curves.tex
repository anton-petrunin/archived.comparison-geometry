\chapter{Plane curves}

\section*{Signed curvature}

Suppose $\gamma$ is a smooth unit-speed plane curve,
so $\tau(s)=\gamma'(s)$ is its unit tangent vector.

Let us rotate $\tau(s)$ by angle $\tfrac\pi 2$ counterclockwise; 
denote the obtained vector by $\nu(s)$.
The pair $\tau(s),\nu(s)$ is an oriented orthonormal frame in the plane which is analogous to the Frenet frame for space curves; we will keep the name \emph{Frenet frame} for it.

Recall that $\gamma''(s)\perp \gamma'(s)$ (\ref{prop:a'-pertp-a''}).
Therefore 
\[\tau'(s)=\skur(s)\cdot \nu(s).\eqlbl{eq:tau'}\]
for some real number $\skur(s)$;
the value $\skur(s)$ is called \emph{signed curvature} of $\gamma$ at $s$.
Note that up to sign it equals to the curvature of $\gamma$ at $s$ as it defined o page \pageref{page:curvature};
the sign tells which direction $\gamma$ turns --- if it turns left, then it is positive.

Note that if we reverse the parametrization of $\gamma$ or change the orientation of the plane, then
the signed curvature changes its sign.

Since $\tau(s),\nu(s)$ is an orthonormal frame, we get that 
\begin{align*}
\langle\tau,\tau\rangle&=1,
&
\langle\nu,\nu\rangle&=1,
&
\langle\tau,\nu\rangle&=0,
\end{align*}
Differentiating these idenitites we get that 
\begin{align*}
\langle\tau',\tau\rangle&=0,
&
\langle\nu',\nu\rangle&=0,
&
\langle\tau',\nu\rangle+\langle\tau,\nu'\rangle&=0,
\end{align*}
By \ref{eq:tau'}, $\langle\tau',\nu\rangle=\skur$ and therefore $\langle\tau,\nu'\rangle=-\skur$.
Whence we get 
\[\nu'(s)=-\skur(s)\cdot \tau(s).\eqlbl{eq:nu'}\]
The equations \ref{eq:tau'} and \ref{eq:nu'} are plane analogs of the Frenet formulas. 
They could be also written in a matrix form:
\[
\begin{pmatrix}
\tau'
\\
\nu'
\end{pmatrix}
=
\begin{pmatrix}
0&\kur
\\
-\kur&0
\end{pmatrix}
\cdot
\begin{pmatrix}
\tau
\\
\nu
\end{pmatrix}.
\]


The following theorem is analogous to the fundamental theorem of curves (\ref{thm:fund-curves}) and it can be proved along the same lines.

\begin{thm}{Theorem}\label{thm:fund-curves-2D}
Let $\skur(s)$ be two smooth real valued functions defined on a real interval $\mathbb{I}$.
Then there is a smooth unit-speed curve $\gamma\:\mathbb{I}\to\RR^2$ with signed curvature $\kur(s)$ at every $s$.
Moreover $\gamma$ is uniquely defined up to a rigid motion of the plane.
\end{thm}

\section*{Osculating circline}

As a direct corollary of Theorem~\ref{thm:fund-curves-2D}, we get the following:

\begin{thm}{Proposition}\label{prop:circline}
Given a point $p$,
a unit vector $\tau$ 
and a real number $\skur$, there is a unique smooth unit-speed curve $\sigma\:\RR\to\RR^2$ 
that starts at $p$ in the direction of $\tau$ and has constant signed curvature $\skur$.

Moreover, if $\skur=0$, then $\sigma$ runs along the line $\sigma=p+s\cdot \tau$
and if $\skur\ne 0$, then $\sigma$ runs around the circle of radius $\tfrac1{|\skur|}$ and center $p+\tfrac1\skur\cdot \nu$, where $\tau,\nu$ is an oriented orthonoral frame.
\end{thm}

Further we will use the term \emph{circline} for \emph{a circle or a line}.

{

\begin{wrapfigure}{r}{30 mm}
\vskip-0mm
\centering
\includegraphics{mppics/pic-21}
\vskip0mm
\end{wrapfigure}

\begin{thm}{Definition}
Let $\gamma$ be a smooth unit-speed plane curve;
denote by $\skur(s)$ the signed curvature of $\gamma$ at $s$.

For $s_0 \in [a,b]$, the unit-speed curve $\sigma$ of constant signed curvature $\skur(s_0)$ that starts at $\gamma(s_0)$ in the direction $\gamma'(s_0)$ is called the \emph{osculating circline} of $\gamma$ at $s_0$.
\end{thm}

}

The center and radius of the osculating circle at a given point are called \emph{center of curvature} and \emph{radius of curvature} of the curve at that point.

This is a unique circline that has \emph{second order of contact} with $\gamma$ at $s$;
that is, $\rho(\ell)=o(\ell^2)$, where $\rho(\ell)$ denotes the distance from $\gamma(s+\ell)$ to the osculating circline at $s$.

\section*{Spiral lemma}
\label{spiral}

The following lemma was proved by Peter Tait \cite{tait}
and later rediscovered by Adolf Kneser \cite{kneser}.

\begin{thm}{Lemma}
Assume that $\gamma$ is a smooth regular plane curve with strictly decreasing positive signed curvature. Then the osculating circles of $\gamma$ are nested; that is, if $\sigma_s$ denoted the osculating circle of $\gamma$ at $s$,
then $\sigma_{s_0}$ lies in the open disc bounded by $\sigma_{s_1}$ for any $s_0<s_1$. 
\end{thm}

\begin{wrapfigure}{o}{25 mm}
\begin{lpic}[t(-0 mm),b(-2 mm),r(0 mm),l(0 mm)]{pics/kneser-log(1)}
\end{lpic}
\end{wrapfigure} %???change pic

It turnes out that osculating circles of the curve $\gamma$ give a peculiar foliation of an annulus by circles; it has the following property: if a smooth function is constant on each osculating circle it must be constant in the annulus \cite[see][Lecture 10]{fuchs-tabachnikov}.
Also note that the curve $\gamma$ is tangent to a circle of the foliation at each of its points. However, it does not run along a circle.

\parit{Proof.}
Let $\tau(s),\nu(s)$ be the Frenet frame,
$z(s)$ the curvature center
and $r(s)$
the radius of curvature of $\gamma$ at $s$.
Recall that $r(s)\cdot\skur(s)=1$.
By \ref{prop:circline},
\[z(s)=\gamma(s)+r(s)\cdot \nu(s).\]

Applying Frenet formula \ref{eq:nu'}, we get that
\begin{align*}
z'(s)&=\gamma'(s)+r'(s)\cdot \nu(s)+r(s)\cdot \nu'(s)=
\\
&=\tau(s)+r'(s)\cdot \nu(s)-r(s)\cdot \skur(s)\cdot \tau(s)=
\\
&=r'(s)\cdot \nu(s).
\end{align*}
Since $\skur(s)$ is decreasing, $r(s)$ is increasing;
therefore $r'\ge 0$.
It follows that $|z'(s)|= r'(s)$ and $z'(s)$ points in the direction of $\nu(s)$.

Since $\nu'(s)=-\skur(s)\cdot\tau(s)$, the directiion of $z'(s)$ always rotates;
that is, the curve $s\mapsto z(s)$ does not have straight arcs.
It follows that 
\[
\begin{aligned}
|z(s_1)-z(s_0)|&<\length(z|_{[s_0,s_1]})=
\\
&=\int_{s_0}^{s_1}|z'(s)|\cdot ds=
\\
&=\int_{s_0}^{s_1}r'(s)\cdot ds=
\\
&=r(s_1)-r(s_0).
\end{aligned}
\leqno({*})
\]
In other words, the distance between the centers of $\siga_{s_1}$ and $\sigma_{s_0}$
is less than the difference between their raduses.
Therefore the osculating circle at $s_0$ lies inside the osculating circle at $s_1$ without touching it.
\qeds

The following theorem states that 
if you drive on the plane and turn the steering wheel to the right all the time,
then you will not be able to come back to the same place.

\begin{thm}{Theorem}\label{thm:spiral}
Assume $\gamma$ is a smooth regular plane curve with strictly monotonic curvature. 
Then $\gamma$ is simple.
\end{thm}

\parit{Proof of \ref{thm:spiral}.}
Note that $\gamma(s)\in \sigma_s$ for any $s$.
Applying the lemma we get
$\gamma(s_1)\ne \gamma(s_0)$ if $s_1\ne s_0$.
Hence the result.\qeds

The same statement also holds for signed curvature; the proof requires only minor modifications.

\begin{thm}{Exercise}
Show that a 3-dimensional analog of the theorem does not hold.
That is, there are self-intersecting smooth regular space curves with strictly monotonic curvature.
\end{thm}

\begin{thm}{Exercise}\label{ex:double-tangent}
Assume that $\gamma$ is a smooth regular plane curve with positive strictly monotonic signed curvature.
\begin{enumerate}[(a)]
\item\label{ex:double-tangent:a}Show that no line can be tangent to $\gamma$ at two distinct points.
\item Show that no circle can be tangent to $\gamma$ at three distinct points. 
\end{enumerate}
\end{thm} %???monotone or monotonic???

\begin{wrapfigure}{r}{35 mm}
\vskip-0mm
\centering
\includegraphics{mppics/pic-25}
\vskip0mm
\end{wrapfigure}

Note that part (\ref{ex:double-tangent:a}) does not hold if we alow the curvature to be negative; an example is shown on the diagram.
