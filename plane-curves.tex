\chapter{Plane curves}

\section*{Signed curvature}

Suppose $\gamma$ is a smooth unit-speed plane curve,
so $\tau(s)=\gamma'(s)$ is its unit tangent vector.

Let us rotate $\tau(s)$ by angle $\tfrac\pi 2$ counterclockwise; 
denote the obtained vector by $\nu(s)$.
The pair $\tau(s),\nu(s)$ is an oriented orthonormal frame in the plane which is analogous to the Frenet frame for space curves; we will keep the name \emph{Frenet frame} for it.

Recall that $\gamma''(s)\perp \gamma'(s)$ (\ref{prop:a'-pertp-a''}).
Therefore 
\[\tau'(s)=\skur(s)\cdot \nu(s).\eqlbl{eq:tau'}\]
for some real number $\skur(s)$;
the value $\skur(s)$ is called \emph{signed curvature} of $\gamma$ at $s$.
Note that up to sign it equals to the curvature of $\gamma$ at $s$ as it defined o page \pageref{page:curvature};
the sign tells which direction $\gamma$ turns --- if it turns left, then it is positive.

Note that if we reverse the parametrization of $\gamma$ or change the orientation of the plane, then
the signed curvature changes its sign.

Since $\tau(s),\nu(s)$ is an orthonormal frame, we get that 
\begin{align*}
\langle\tau,\tau\rangle&=1,
&
\langle\nu,\nu\rangle&=1,
&
\langle\tau,\nu\rangle&=0,
\end{align*}
Differentiating these idenitites we get that 
\begin{align*}
\langle\tau',\tau\rangle&=0,
&
\langle\nu',\nu\rangle&=0,
&
\langle\tau',\nu\rangle+\langle\tau,\nu'\rangle&=0,
\end{align*}
By \ref{eq:tau'}, $\langle\tau',\nu\rangle=\skur$ and therefore $\langle\tau,\nu'\rangle=-\skur$.
Whence we get 
\[\nu'(s)=-\skur(s)\cdot \tau(s).\eqlbl{eq:nu'}\]
The equations \ref{eq:tau'} and \ref{eq:nu'} are Frenet formulas for plane curves. 
They could be also written in a matrix form:
\[
\begin{pmatrix}
\tau'
\\
\nu'
\end{pmatrix}
=
\begin{pmatrix}
0&\kur
\\
-\kur&0
\end{pmatrix}
\cdot
\begin{pmatrix}
\tau
\\
\nu
\end{pmatrix}.
\]


The following theorem is the fundamental theorem of plane curves; it is direct analog of \ref{thm:fund-curves} and it can be proved along the same lines.

\begin{thm}{Theorem}\label{thm:fund-curves-2D}
Let $\skur(s)$ be two smooth real valued function defined on a real interval $\mathbb{I}$.
Then there is a smooth unit-speed curve $\gamma\:\mathbb{I}\to\RR^2$ with signed curvature $\kur(s)$ at every $s$.
Moreover $\gamma$ is uniquely defined up to a rigid motion of the plane.
\end{thm}

\section*{Total signed curvature}

\begin{thm}{Lemma}\label{lem:2D-angle}
Suppose $\gamma\:[a,b]\to\RR^2$ is a smooth unit-speed  curve.
Then there is a smooth function $\theta\:[a,b]\to\RR$ such that 
\[\gamma'(s)=(\cos[\theta(s)],\sin[\theta(s)])\quad\text{and}\quad \theta'(s)=\skur(s)\]
for any $s$,
where $\skur(s)$ denotes the signed curvature of $\gamma$.
\end{thm}

\parit{Proof.}
Since $\gamma$ is unit-speed, $\gamma'(a)=(\cos\theta_0,\sin\theta_0)$ for some $\theta_0$.
Set 
\[\theta(s)=\theta_0+\int_a^s\skur(t)\cdot dt;\]
by the fundamental theorem of calculus, we have $\theta'(s)=\skur(s)$ for any~$s$.

Set $\tau(s)=(\cos[\theta(s)],\sin[\theta(s)])$ and let $\nu(s)$ be its counterclockwise rotation by angle $\tfrac\pi2$; so $\tau(s)=(-\sin[\theta(s)],\cos[\theta(s)])$.
Note that
\begin{align*}
\tau'(s)&=(\cos[\theta(s)]',\sin[\theta(s)]')=
\\
&=\theta'(s)\cdot (-\sin[\theta(s)],\cos[\theta(s)])=
\\
&=\skur(s)\cdot \nu(s)
\\
\nu'(s)&=(-\sin[\theta(s)]',\cos[\theta(s)]')=
\\
&=\theta'(s)\cdot (-\cos[\theta(s)],\sin[\theta(s)])=
\\
&=-\skur(s)\cdot \nu(s)
\end{align*}
That is, $\tau$ and $\nu$ satisfy the Frenet formulas \ref{eq:tau'} and \ref{eq:nu'} for $\gamma$.
By construction $\tau(a),\nu(a)$ is the Frenet frame at $a$; therefore $\tau(s),\nu(s)$ is the Frenet frame at any $s$.
In particular, 
\[\gamma'(s)=\tau(s)=(\cos[\theta(s)],\sin[\theta(s)])\]
for any $s$.
\qeds


Let $\gamma$ be a smooth unit-speed plane curve.
The integral of its signed curvature is called \emph{total signed curvature} and it denoted by $\Psi(\gamma)$;
so if $\theta$ and $\gamma$ is as in \ref{lem:2D-angle}, then 
\[\Psi(\gamma)= \int_a^bk(s)\cdot ds=\theta(b)-\theta(a).\eqlbl{eq:tsc-theta}\]
Since $\left|\int k(s)\cdot ds\right|\le \int|k(s)|\cdot ds$, we have that
\[|\Psi(\gamma)|\le \tc\gamma\eqlbl{eq:tsc-tc}\] 
for any smooth regular plane curve $\gamma$. 


\begin{thm}{Proposition}\label{prop:total-signed-curvature}
The total signed curvature of any closed simple smooth plane curve $\gamma$ is $\pm2\cdot\pi$; it is $+2\cdot\pi$
if the region bounded by $\gamma$ lies on the left from it and  $-2\cdot\pi$ otherwise.
\end{thm}

This proposition is a differential-geometric analog of the theorem about sum of the internal angles of a polygon (\ref{thm:sum=(n-2)pi}) which we use in the proof.
A more conceptual proof was given by Heinz Hopf \cite[p. 42]{hopf}, \cite[p. 42]{hopf-book}.

\parit{Proof.}
Without loss of generality we may assume that $\gamma$ is oriented in such a way that the region bounded by $\gamma$ lies on the left from it.
We can also assume that it parametrized by arc length.

Consider a closed polygonal line $p_1\dots p_n$ inscribed in $\gamma$.
We can assume that the arcs between the vertexes are sufficiently small;
in this case the polygonal line is simple and each arc $\gamma_i$ from $p_i$ to $p_{i+1}$ have small total curvature, say  $\tc{\gamma_i}<\pi$ for each $i$.
(As usual we use indexes modulo $n$, in particular $p_{n+1}\z=p_1$.)

\begin{wrapfigure}{o}{40 mm}
\vskip-0mm
\centering
\includegraphics{mppics/pic-59}
\vskip0mm
\end{wrapfigure}

Assume $p_i=\gamma(t_i)$.
Set 
\begin{align*}
w_i&=p_{i+1}-p_i,& v_i&=\gamma'(t_i),
\\
\alpha_i&=\measuredangle (v_i,w_i),&\beta_i&=\measuredangle (w_{i-1},v_i),
\end{align*}
where $\alpha_i,\beta_i\in(-\pi,\pi]$ are oriented angles.

By \ref{eq:tsc-theta}, the value
\[\Psi(\gamma_i)-\alpha_i-\beta_{i+1}\]
is a multiple of $2\cdot\pi$.
Since $\tc{\gamma_i}<\pi$, by chord lemma (\ref{lem:chord}), we have that $|\alpha_i|\z+|\beta_i|<\pi$.
By \ref{eq:tsc-tc}, we have that $|\Psi(\gamma_i)|\le\tc{\gamma_i}$;
therefore
\[\Psi(\gamma_i)=\alpha_i+\beta_{i+1}.\]
for each $i$.

Note that $\delta_i=\pi-\alpha_i-\beta_i$ is the internal angle of $p_1\dots p_n$ at $p_i$;
$\delta_i\in (0,2\cdot\pi$ for each $i$.
Recall that the sum of internal angles of an $n$-gon is $(n-2)\cdot \pi$ (see \ref{thm:sum=(n-2)pi}); that is,
\[\delta_1+\dots+\delta_n=(n-2)\cdot \pi.\]
Therefore 
\begin{align*}
\Psi(\gamma)&=\Psi(\gamma_1)+\dots+\Psi(\gamma_n)=
\\
&=(\alpha_1+\beta_2)+\dots+(\alpha_n+\beta_1)=
\\
&=(\beta_1+\alpha_1)+\dots+(\beta_n+\alpha_n)=
\\
&=(\pi-\delta_1)+\dots+(\pi-\delta_n)=
\\
&=n\cdot\pi-(n-2)\cdot \pi=
\\
&=2\cdot\pi.
\end{align*}
\qedsf

\begin{thm}{Exercise}\label{ex:zero-tsc}
Draw a smooth regular closed plane curve with vanishing total signed curvature.
\end{thm}

\section*{Osculating circline}

As a direct corollary of Theorem~\ref{thm:fund-curves-2D}, we get the following:

\begin{thm}{Proposition}\label{prop:circline}
Given a point $p$,
a unit vector $\tau$ 
and a real number $\skur$, there is a unique smooth unit-speed curve $\sigma\:\RR\to\RR^2$ 
that starts at $p$ in the direction of $\tau$ and has constant signed curvature $\skur$.

Moreover, if $\skur=0$, then $\sigma=p+s\cdot \tau$ which runs along the line;
if $\skur\ne 0$, then $\sigma$ runs around the circle of radius $\tfrac1{|\skur|}$ and center $p+\tfrac1\skur\cdot \nu$, where $\tau,\nu$ is an oriented orthonoral frame.
\end{thm}

Further we will use the term \emph{circline} for \emph{a circle or a line}.

{

\begin{wrapfigure}{r}{30 mm}
\vskip-0mm
\centering
\includegraphics{mppics/pic-21}
\vskip0mm
\end{wrapfigure}

\begin{thm}{Definition}
Let $\gamma$ be a smooth unit-speed plane curve;
denote by $\skur(s)$ the signed curvature of $\gamma$ at $s$.

For $s_0 \in [a,b]$, the unit-speed curve $\sigma$ of constant signed curvature $\skur(s_0)$ that starts at $\gamma(s_0)$ in the direction $\gamma'(s_0)$ is called the \emph{osculating circline} of $\gamma$ at $s_0$.
\end{thm}

}

The center and radius of the osculating circle at a given point are called \emph{center of curvature} and \emph{radius of curvature} of the curve at that point.

This is a unique circline that has \emph{second order of contact} with $\gamma$ at $s$;
that is, $\rho(\ell)=o(\ell^2)$, where $\rho(\ell)$ denotes the distance from $\gamma(s+\ell)$ to the osculating circline at $s$.

\section*{Spiral lemma}
\label{spiral}

The following lemma was proved by Peter Tait \cite{tait}
and later rediscovered by Adolf Kneser \cite{kneser}.

\begin{thm}{Lemma}
Assume that $\gamma$ is a smooth regular plane curve with strictly decreasing positive signed curvature. Then the osculating circles of $\gamma$ are nested; that is, if $\sigma_s$ denoted the osculating circle of $\gamma$ at $s$,
then $\sigma_{s_0}$ lies in the open disc bounded by $\sigma_{s_1}$ for any $s_0<s_1$. 
\end{thm}

\begin{wrapfigure}{o}{25 mm}
\begin{lpic}[t(-0 mm),b(-2 mm),r(0 mm),l(0 mm)]{pics/kneser-log(1)}
\end{lpic}
\end{wrapfigure} %???change pic

It turnes out that osculating circles of the curve $\gamma$ give a peculiar foliation of an annulus by circles; it has the following property: if a smooth function is constant on each osculating circle it must be constant in the annulus \cite[see][Lecture 10]{fuchs-tabachnikov}.
Also note that the curve $\gamma$ is tangent to a circle of the foliation at each of its points. However, it does not run along a circle.

\parit{Proof.}
Let $\tau(s),\nu(s)$ be the Frenet frame,
$z(s)$ the curvature center
and $r(s)$
the radius of curvature of $\gamma$ at $s$.
Recall that $r(s)\cdot\skur(s)=1$.
By \ref{prop:circline},
\[z(s)=\gamma(s)+r(s)\cdot \nu(s).\]

Applying Frenet formula \ref{eq:nu'}, we get that
\begin{align*}
z'(s)&=\gamma'(s)+r'(s)\cdot \nu(s)+r(s)\cdot \nu'(s)=
\\
&=\tau(s)+r'(s)\cdot \nu(s)-r(s)\cdot \skur(s)\cdot \tau(s)=
\\
&=r'(s)\cdot \nu(s).
\end{align*}
Since $\skur(s)$ is decreasing, $r(s)$ is increasing;
therefore $r'\ge 0$.
It follows that $|z'(s)|= r'(s)$ and $z'(s)$ points in the direction of $\nu(s)$.

Since $\nu'(s)=-\skur(s)\cdot\tau(s)$, the directiion of $z'(s)$ always rotates;
that is, the curve $s\mapsto z(s)$ contains no line segments.
It follows that 
\[
\begin{aligned}
|z(s_1)-z(s_0)|&<\length(z|_{[s_0,s_1]})=
\\
&=\int_{s_0}^{s_1}|z'(s)|\cdot ds=
\\
&=\int_{s_0}^{s_1}r'(s)\cdot ds=
\\
&=r(s_1)-r(s_0).
\end{aligned}
\leqno({*})
\]
In other words, the distance between the centers of $\sigma_{s_1}$ and $\sigma_{s_0}$
is strictly less than the difference between their radiuses.
Therefore the osculating circle at $s_0$ lies inside the osculating circle at $s_1$ without touching it.
\qeds

The following theorem has the following intuitive formulation: 
\emph{if you drive on the plane and turn the steering wheel to the right all the time,
then you will not be able to come back to the same place.}

\begin{thm}{Theorem}\label{thm:spiral}
Assume $\gamma$ is a smooth regular plane curve with strictly monotonic curvature. 
Then $\gamma$ is simple.
\end{thm}

\parit{Proof of \ref{thm:spiral}.}
Note that $\gamma(s)\in \sigma_s$ for any $s$.
Applying the lemma we get
$\gamma(s_1)\ne \gamma(s_0)$ if $s_1\ne s_0$.
Hence the result.\qeds

The same statement also holds for signed curvature; the proof requires only minor modifications.

\begin{thm}{Exercise}
Show that a 3-dimensional analog of the theorem does not hold.
That is, there are self-intersecting smooth regular space curves with strictly monotonic curvature.
\end{thm}

\begin{thm}{Exercise}\label{ex:double-tangent}
Assume that $\gamma$ is a smooth regular plane curve with positive strictly monotonic signed curvature.
\begin{enumerate}[(a)]
\item\label{ex:double-tangent:a}Show that no line can be tangent to $\gamma$ at two distinct points.
\item Show that no circle can be tangent to $\gamma$ at three distinct points. 
\end{enumerate}
\end{thm} %???monotone or monotonic???

\begin{wrapfigure}{r}{35 mm}
\vskip-0mm
\centering
\includegraphics{mppics/pic-25}
\vskip0mm
\end{wrapfigure}

Note that part (\ref{ex:double-tangent:a}) does not hold if we alow the curvature to be negative; an example is shown on the diagram.

We say that a smooth regular plane curve $\gamma$ has a \emph{vertex} at $s$
if the signed curvature function has extremal at $s$;
that is, if $\skur'_\gamma(s)=0$.
If $\gamma$ is simple we could say that the point $p=\gamma(s)$ is a vertex of $\gamma$ without ambiguity.

\begin{thm}{Exercise} 
Assume that a smooth regular plane curve $\gamma$ runs on one side of its osculating circle at $s$.
Show that $\gamma$ has a vertex at~$s$.
\end{thm}

