\chapter{Signed curvature}

Suppose $\gamma$ is a smooth unit-speed plane curve,
so $\tan(s)=\gamma'(s)$ is its unit tangent vector for any $s$.

Let us rotate $\tan(s)$ by angle $\tfrac\pi 2$ counterclockwise; 
denote the obtained vector by $\norm(s)$.
The pair $\tan(s),\norm(s)$ is an oriented orthonormal frame in the plane which is analogous to the Frenet frame defined on page \pageref{page:frenet-frame}; we will keep the name \emph{Frenet frame} for it.

Recall that $\gamma''(s)\perp \gamma'(s)$ (see \ref{prop:a'-pertp-a''}).
Therefore 
\[\tan'(s)=\skur(s)\cdot \norm(s).\eqlbl{eq:tau'}\]
for some real number $\skur(s)$;
the value $\skur(s)$ is called \emph{signed curvature} of $\gamma$ at $s$.
We may use notation $\skur(s)_\gamma$ if we need to specify the curve $\gamma$.

Note that 
\[\kur(s)=|\skur(s)|;\]
that is, up to sign, the signed curvature $\skur(s)$ equals to the curvature $\kur(s)$  of $\gamma$ at $s$  defined on page \pageref{page:curvature};
the sign tells which direction it turns --- if $\gamma$ turns left, then $\skur>0$.
If we want to emphasise that we work with \emph{nonsigned} curvature of the curve, 
we call it \emph{absolute curvature}.

Note that if we reverse the parametrization of $\gamma$ or change the orientation of the plane, then
the signed curvature changes its sign.

Since $\tan(s),\norm(s)$ is an orthonormal frame, we have that 
\begin{align*}
\langle\tan,\tan\rangle&=1,
&
\langle\norm,\norm\rangle&=1,
&
\langle\tan,\norm\rangle&=0,
\end{align*}
Differentiating these idenitites we get that 
\begin{align*}
\langle\tan',\tan\rangle&=0,
&
\langle\norm',\norm\rangle&=0,
&
\langle\tan',\norm\rangle+\langle\tan,\norm'\rangle&=0,
\end{align*}
By \ref{eq:tau'}, $\langle\tan',\norm\rangle=\skur$ and therefore $\langle\tan,\norm'\rangle=-\skur$.
Whence we get 
\[\norm'(s)=-\skur(s)\cdot \tan(s).\eqlbl{eq:nu'}\]
The equations \ref{eq:tau'} and \ref{eq:nu'} are Frenet formulas for plane curves. 
They could be also written in a matrix form:
\[
\begin{pmatrix}
\tan'
\\
\norm'
\end{pmatrix}
=
\begin{pmatrix}
0&\skur
\\
-\skur&0
\end{pmatrix}
\cdot
\begin{pmatrix}
\tan
\\
\norm
\end{pmatrix}.
\]

\begin{thm}{Exercise}\label{ex:bike}
Let $\gamma_0\:[a,b]\to\RR^2$ be a smooth regular curve and $\tan$ its tangent indicatrix.
Consider another curve $\gamma_1\:[a,b]\to\RR^2$ defined by $\gamma_1(t)=\gamma_0(t)+\tan(t)$.
Show that
\[\length\gamma_0\le \length\gamma_1.\]

\end{thm}

The curves $\gamma_0$ and $\gamma_1$ in the exercise above describe tracks of idealized bicycle with the distance 1 from rear to front wheel.
Thus by the exercise, the front wheel have to have the longer track.
For more on geometry of bicycle tracks see \cite{FLT} and the references there in.

\section*{Fundamental theorem of plane curves}

\begin{thm}{Theorem}\label{thm:fund-curves-2D}
Let $\skur(s)$ be a smooth real valued function defined on a real interval $\mathbb{I}$.
Then there is a smooth unit-speed curve $\gamma\:\mathbb{I}\to\RR^2$ with signed curvature $\skur(s)$ at every $s$.
Moreover $\gamma$ is uniquely defined up to a rigid motion of the plane.
\end{thm}

This is the fundamental theorem of plane curves; it is direct analog of \ref{thm:fund-curves} and it can be proved along the same lines.
We give a slightly simpler proof.

\parit{Proof.} 
Fix $s_0\in\mathbb{I}$.
Consider the function
\[\theta(s)=\int_{s_0}^s\skur(t)\cdot dt.\]
Note that by the fundamental theorem of calculus, we have $\theta'(s)\z=\skur(s)$ for any~$s$.

Set 
\[\tan(s)=(\cos[\theta(s)],\sin[\theta(s)])\] and let $\norm(s)$ be its counterclockwise rotation by angle $\tfrac\pi2$; so 
\[\norm(s)=(-\sin[\theta(s)],\cos[\theta(s)]).\]
Consider the curve 
\[\gamma(s)=\int_{s_0}^s\tan(s)\cdot ds.\]
Since $|\gamma'|=|\tan|=1$, the curve $\gamma$ is unit-speed and $\tan,\norm$ is its Frenet frame. 

Note that
\begin{align*}
\gamma''(s)&=\tan'(s)=
\\
&=(\cos[\theta(s)]',\sin[\theta(s)]')=
\\
&=\theta'(s)\cdot (-\sin[\theta(s)],\cos[\theta(s)])=
\\
&=\skur(s)\cdot \norm(s).
\end{align*}
That is, $\skur(s)$ is the signed curvature of $\gamma$ at $s$.

The existence is proved; it remains to prove uniqueness.

Assume $\gamma_1$ and $\gamma_2$ are two curves that satisfy the assumptions of the theorem.
Applying a rigid motion, we can assume that $\gamma_1(s_0)=\gamma_2(s_0)=0$ and the Frenet frame of both curves at $s_0$ is formed by the coordinate frame $(1,0),(0,1)$.
Let us denote by $\tan_1,\norm_1$ and $\tan_2,\norm_2$ the Frenet frames of $\gamma_1$ and $\gamma_2$ correspondingly.
The triples $\gamma_i,\tan_i,\norm_i$ satisfy the same system system of ordinary differential equations 
\[
\begin{cases}
\gamma_i'=\tan_i,
\\
\tan_i'=\skur\cdot\norm_i,
\\
\norm_i'=-\skur\cdot\tan_i.
\end{cases}
\]
Motreover, they have the same the initial values at $s_0$.
Therefore $\gamma_1=\gamma_2$.
\qeds

Note that from the proof of theorem we obtain the following corollary:



\begin{thm}{Corollary}\label{cor:2D-angle}
Suppose $\gamma\:\mathbb{I}\to\RR^2$ is a smooth unit-speed curve and $s_0\in \mathbb{I}$.
Denote by $\skur$ the signed curvature of $\gamma$.
Assume an oriented $(x,y)$-coordinate system on is chosen in such a way that $\gamma(s_0)$ is the origin and $\gamma'(s_0)$ points in the direction of $x$-axis.
Then 
\[\gamma'(s)=(\cos[\theta(s)],\sin[\theta(s)])\]
where 
\[\theta(s)=\int_{s_0}^s\skur(t)\cdot dt.\]
\end{thm}


\section*{Total signed curvature}

Let $\gamma\:\mathbb{I}\to\RR^2$ be a smooth unit-speed plane curve.
The integral of its signed curvature is called \emph{total signed curvature} and it denoted by $\Psi(\gamma)$;
so 
\[\Psi(\gamma)= \int_\mathbb{I} \skur(s)\cdot ds,\eqlbl{eq:tsc-theta}\]
where $\skur$ denotes signed curvature of $\gamma$.

If $\mathbb{I}=[a,b]$, then 
\[\Psi(\gamma)=\theta(b)-\theta(a),\]
where $\theta$ is as in \ref{cor:2D-angle}.

If $\gamma$ is a piecewise smooth and regular plane curve, then we define its total signed curvature as the sum of total signed curvatures of its arcs plus the sum of signed external angles at the joints;
it is positive if $\gamma$ turns left, negative if  $\gamma$ turns right, 0 if it goes straight and undefined if it turns backward.
That is, if $\gamma$ is a concatenation of smooth and regular arcs $\gamma_1,\dots,\gamma_n$ then 
\[\Psi(\gamma)=\Psi(\gamma_1)+\dots+\Psi(\gamma_n)+\theta_1+\dots+\theta_{n-1}\]
where $\theta_i$ is the signed external angle at the joint between $\gamma_i$ and $\gamma_{i+1}$.
If $\gamma$ is closed, then the concatenation is cyclic and
\[\Psi(\gamma=\Psi(\gamma_1)+\dots+\Psi(\gamma_n)+\theta_1+\dots+\theta_{n},\]
where $\theta_n$ is the signed external angle at the joint between $\gamma_n$ and $\gamma_1$.

Since $\left|\int \skur(s)\cdot ds\right|\le \int|\skur(s)|\cdot ds$, we have that
\[|\Psi(\gamma)|\le \tc\gamma\eqlbl{eq:tsc-tc}\] 
for any smooth regular plane curve $\gamma$;
that is, total signed curvature can not exceed total curvature by absolute value.

%???+ add corners


\begin{thm}{Proposition}\label{prop:total-signed-curvature}
The total signed curvature of any closed simple smooth regular plane curve $\gamma$ is $\pm2\cdot\pi$; it is $+2\cdot\pi$
if the region bounded by $\gamma$ lies on the left from it and  $-2\cdot\pi$ otherwise.

Moreover the same statement holds for any closed piecewise simple smooth regular plane curve $\gamma$ if its total signed curvature is defined.
\end{thm}

This proposition is called sometimes \emph{Umlaufsatz}; it is a differential-geometric analog of the theorem about sum of the internal angles of a polygon (\ref{thm:sum=(n-2)pi}) which we use in the proof.
A more conceptual proof was given by Heinz Hopf \cite{hopf}, \cite[p. 42]{hopf-book}.

\parit{Proof.}
Without loss of generality we may assume that $\gamma$ is oriented in such a way that the region bounded by $\gamma$ lies on the left from it.
We can also assume that it parametrized by arc length.

Consider a closed polygonal line $p_1\dots p_n$ inscribed in $\gamma$.
We can assume that the arcs between the vertexes are sufficiently small;
in this case the polygonal line is simple and each arc $\gamma_i$ from $p_i$ to $p_{i+1}$ have small total absolute curvature, say  $\tc{\gamma_i}<\pi$ for each $i$.

\begin{wrapfigure}{o}{40 mm}
\vskip-0mm
\centering
\includegraphics{mppics/pic-59}
\vskip0mm
\end{wrapfigure}

As usual we use indexes modulo $n$, in particular $p_{n+1}\z=p_1$.
Assume $p_i=\gamma(t_i)$.
Set 
\begin{align*}
\vec w_i&=p_{i+1}-p_i,& \vec v_i&=\gamma'(t_i),
\\
\alpha_i&=\measuredangle (\vec v_i,\vec w_i),&\beta_i&=\measuredangle (\vec w_{i-1},\vec v_i),
\end{align*}
where $\alpha_i,\beta_i\in(-\pi,\pi]$ are signed angles --- $\alpha_i$ is positive if $\vec w_i$ points to the left from $\vec v_i$.

By \ref{eq:tsc-theta}, the value
\[\Psi(\gamma_i)-\alpha_i-\beta_{i+1}\eqlbl{eq:Psi-alpha-beta}\]
is a multiple of $2\cdot\pi$.
Since $\tc{\gamma_i}<\pi$, by chord lemma (\ref{lem:chord}), we also have that $|\alpha_i|\z+|\beta_i|<\pi$.
By \ref{eq:tsc-tc}, we have that $|\Psi(\gamma_i)|\le\tc{\gamma_i}$;
therefore the value in \ref{eq:Psi-alpha-beta} vanishes, or equivalently
\[\Psi(\gamma_i)=\alpha_i+\beta_{i+1}\]
for each $i$.

Note that 
\[\delta_i=\pi-\alpha_i-\beta_i\eqlbl{eq:delta=pi-alpha-beta}\] 
is the internal angle of $p_1\dots p_n$ at $p_i$;
$\delta_i\in (0,2\cdot\pi)$ for each $i$.
Recall that the sum of internal angles of an $n$-gon is $(n-2)\cdot \pi$ (see \ref{thm:sum=(n-2)pi}); that is,
\[\delta_1+\dots+\delta_n=(n-2)\cdot \pi.\]
Therefore 
\[
\begin{aligned}
\Psi(\gamma)&=\Psi(\gamma_1)+\dots+\Psi(\gamma_n)=
\\
&=(\alpha_1+\beta_2)+\dots+(\alpha_n+\beta_1)=
\\
&=(\beta_1+\alpha_1)+\dots+(\beta_n+\alpha_n)=
\\
&=(\pi-\delta_1)+\dots+(\pi-\delta_n)=
\\
&=n\cdot\pi-(n-2)\cdot \pi=
\\
&=2\cdot\pi.
\end{aligned}\eqlbl{eq:delta=pi-alpha-beta-sum}\]

The piecewise smooth and regular curve is done the same way;
we need to subdivide the arcs in the cyclic concatenation further to meet the requirement above and instead of equation \ref{eq:delta=pi-alpha-beta} we have 
\[\delta_i=\pi-\alpha_i-\beta_i-\theta_i,\]
where $\theta_i$ is the signed external angle at $p_i$; it vanishes if the curve $\gamma$ is smooth at $p_i$.
Therefore instead of equation \ref{eq:delta=pi-alpha-beta-sum}, we have
\begin{align*}
\Psi(\gamma)&=\Psi(\gamma_1)+\dots+\Psi(\gamma_n)+\theta_1+\dots+\theta_n=
\\
&=(\alpha_1+\beta_2)+\dots+(\alpha_n+\beta_1)=
\\
&=(\beta_1+\alpha_1+\theta_1)+\dots+(\beta_n+\alpha_n+\theta_n)=
\\
&=(\pi-\delta_1)+\dots+(\pi-\delta_n)=
\\
&=n\cdot\pi-(n-2)\cdot \pi=
\\
&=2\cdot\pi.
\end{align*}
\qedsf

\begin{thm}{Exercise}\label{ex:zero-tsc}
Draw a smooth regular closed plane curve with zero total signed curvature.
\end{thm}

\begin{thm}{Exercise}\label{ex:length'}
Let $\gamma\:[a,b]\to\RR$ be a smooth regular plane curve with Frenet frame $\tan,\norm$.
Given a real parameter $\ell$, consider
the curve $\gamma_\ell(t)=\gamma(t)+\ell\cdot\norm(t)$; it is called a \emph{parallel curve of $\gamma$ on the signed distance $\ell$}.

\begin{enumerate}[(a)]
\item Show that $\gamma_\ell$ is a regular curve if $\ell\cdot \skur(t)\ne 1$ for any $t$, where $\skur(t)$ denotes the signed curvature of $\gamma$. 
\item Set $L(\ell)=\length\gamma_\ell$.
Show that 
\[L(\ell)=L(0)-\ell\cdot\Psi(\gamma)\]
for all $\ell$ sufficiently close to $0$. 
Describe an example showing that this formula does not hold for all $\ell$. 
\end{enumerate}

\end{thm}


\section*{Osculating circline}

\begin{thm}{Proposition}\label{prop:circline}
Given a point $p$,
a unit vector $\tan$ 
and a real number $\skur$, there is a unique smooth unit-speed curve $\sigma\:\RR\to\RR^2$ 
that starts at $p$ in the direction of $\tan$ and has constant signed curvature $\skur$.

Moreover, if $\skur=0$, then $\sigma(s)=p+s\cdot \tan$ which runs along the line;
if $\skur\ne 0$, then $\sigma$ runs around the circle of radius $\tfrac1{|\skur|}$ and center $p+\tfrac1\skur\cdot \norm$, where $\tan,\norm$ is an oriented orthonoral frame.
\end{thm}

Further we will use the term \emph{circline} for \emph{a circle or a line};
these are the only plane curves with constant signed curvature.

\parit{Proof.}
The proof is done by calculation based on \ref{thm:fund-curves-2D} and \ref{cor:2D-angle}.

Suppose $s_0=0$, choose coordinate system such that $p$ is its origin, $\tan$ points in the direction of $x$-axis and therefore $\norm$ points in the direction of $y$-axis.
Then
\begin{align*}\theta(s)&=\int_{0}^s\skur\cdot dt=
\\
&=\skur\cdot s.
\end{align*}
Therefore
\[\sigma'(s)=(\cos[\skur\cdot s],\sin[\skur\cdot s]).\]
It remains to integrate the last identity.
If $\skur=0$, we get 
\[\sigma(s)=(s,0)\]
which describes the line $\sigma(s)=p+s\cdot \tan$.

If $\skur\ne 0$, we get
\[\sigma(s)=(\tfrac1\skur\cdot\sin[\skur\cdot s],
\tfrac1\skur\cdot(1-\cos[\skur\cdot s])).\]
which is the circle of radius $r=\tfrac1{|\skur|}$ centered at $(0,\tfrac1\skur)=p+\tfrac1\skur\cdot\norm$.
\qeds

\begin{thm}{Definition}
Let $\gamma$ be a smooth unit-speed plane curve;
denote by $\skur(s)$ the signed curvature of $\gamma$ at $s$.

The unit-speed curve $\sigma$ of constant signed curvature $\skur(s)$ that starts at $\gamma(s)$ in the direction $\gamma'(s)$ is called the \emph{osculating circline} of $\gamma$ at~$s$.
\end{thm}


\begin{wrapfigure}{o}{31 mm}
\vskip-0mm
\centering
\includegraphics{mppics/pic-21}
\vskip0mm
\end{wrapfigure}

The center and radius of the osculating circle at a given point are called \emph{center of curvature} and \emph{radius of curvature} of the curve at that point.

The osculating circle $\sigma_s$ can be also defined as the (necessarily unique) circline that has \emph{second order of contact} with $\gamma$ at $s$;
that is, $\rho(\ell)\z=o(\ell^2)$, where $\rho(\ell)$ denotes the distance from $\gamma(s+\ell)$ to $\sigma_s$.

\section*{Spiral lemma}
\label{spiral}

The following lemma was proved by Peter Tait \cite{tait}
and later rediscovered by Adolf Kneser \cite{kneser}.

\begin{thm}{Lemma}\label{lem:spiral}
Assume that $\gamma$ is a smooth regular plane curve with strictly decreasing positive signed curvature. Then the osculating circles of $\gamma$ are nested; that is, if $\sigma_s$ denoted the osculating circle of $\gamma$ at $s$,
then $\sigma_{s_0}$ lies in the open disc bounded by $\sigma_{s_1}$ for any $s_0<s_1$. 
\end{thm}

\begin{wrapfigure}{o}{31 mm}
\vskip-4mm
\begin{lpic}[t(-0 mm),b(-2 mm),r(0 mm),l(0 mm)]{mppics/pic-61}
\end{lpic}
\end{wrapfigure}

It turns out that osculating circles of the curve $\gamma$ give a peculiar foliation of an annulus by circles; it has the following property: if a smooth function is constant on each osculating circle it must be constant in the annulus \cite[see][Lecture 10]{fuchs-tabachnikov}.
Also note that the curve $\gamma$ is tangent to a circle of the foliation at each of its points.
However, it does not run along a circle.

\parit{Proof.}
Let $\tan(s),\norm(s)$ be the Frenet frame,
$\omega(s)$ the curvature center
and $r(s)$
the radius of curvature of $\gamma$ at $s$.
By \ref{prop:circline},
\[\omega(s)=\gamma(s)+r(s)\cdot \norm(s).\]

Since $\skur>0$, we have that $r(s)\cdot\skur(s)=1$.
Therefore applying Frenet formula \ref{eq:nu'}, we get that
\begin{align*}
\omega'(s)&=\gamma'(s)+r'(s)\cdot \norm(s)+r(s)\cdot \norm'(s)=
\\
&=\tan(s)+r'(s)\cdot \norm(s)-r(s)\cdot \skur(s)\cdot \tan(s)=
\\
&=r'(s)\cdot \norm(s).
\end{align*}
Since $\skur(s)$ is decreasing, $r(s)$ is increasing;
therefore $r'\ge 0$.
It follows that $|\omega'(s)|= r'(s)$ and $\omega'(s)$ points in the direction of $\norm(s)$.

{

\begin{wrapfigure}{o}{41 mm}
\vskip-0mm
\begin{lpic}[t(-0 mm),b(-2 mm),r(0 mm),l(0 mm)]{mppics/pic-84}
\end{lpic}
\end{wrapfigure}
%???change pic

Since $\norm'(s)=-\skur(s)\cdot\tan(s)$, the direction of $\omega'(s)$ cannot have constant direction on a nontrivial interval;
that is, the curve $s\mapsto \omega(s)$ contains no line segments.
It follows that 
\begin{align*}
|\omega(s_1)-\omega(s_0)|&<\length(\omega|_{[s_0,s_1]})=
\\
&=\int_{s_0}^{s_1}|\omega'(s)|\cdot ds=
\\
&=\int_{s_0}^{s_1}r'(s)\cdot ds=
\\
&=r(s_1)-r(s_0).
\end{align*}

}

In other words, the distance between the centers of $\sigma_{s_1}$ and $\sigma_{s_0}$
is strictly less than the difference between their radiuses.
Therefore the osculating circle at $s_0$ lies inside the osculating circle at $s_1$ without touching it.
\qeds

The curve $s\mapsto \omega(s)$ is called \emph{evolute} of $\gamma$; 
it traces the centers of curvature of the curve. 
The evolute of $\gamma$ can be written as 
\[\omega(t)=\gamma(t)+\tfrac1{\skur(t)}\cdot \norm(t)\] and  
in the proof we showed that $(\tfrac1{\skur})'\cdot\norm$ is its velocity vector.

\begin{thm}{Exercise}\label{ex:evolute-of-ellipse}
Show that the stretched astroid 
\[\alpha(t)=(\tfrac{a^2-b^2}{a}\cdot \cos^3 t,  \tfrac{b^2-a^2}{b}\cdot\sin^3 t)\]
is an evolute of the ellipse $\gamma(t)= (a\cdot \cos t, b\cdot\sin t)$.
\end{thm}



The following theorem states formally that 
\emph{if you drive on the plane and turn the steering wheel to the right all the time,
then you will not be able to come back to the same place.}

\begin{thm}{Theorem}\label{thm:spiral}
Assume $\gamma$ is a smooth regular plane curve with positive and strictly monotonic signed curvature. 
Then $\gamma$ is simple.
\end{thm}

The same statement holds without assuming positivity of curvature; the proof requires only minor modifications.

\parit{Proof of \ref{thm:spiral}.}
Note that $\gamma(s)$ lies on the osculating circle $\sigma_s$ of $\gamma$ at $s$.
If $s_1\ne s_0$, then by lemma $\sigma_{s_0}$ does not intersect $\sigma_{s_1}$.
Therefore $\gamma(s_1)\ne \gamma(s_0)$,
hence the result.\qeds

\begin{thm}{Exercise}\label{ex:3D-spiral}
Show that a 3-dimensional analog of the theorem does not hold.
That is, there are self-intersecting smooth regular space curves with strictly monotonic curvature.
\end{thm}

\begin{thm}{Exercise}\label{ex:double-tangent}
Assume that $\gamma$ is a smooth regular plane curve with positive strictly monotonic signed curvature.
\begin{enumerate}[(a)]
\item\label{ex:double-tangent:a}Show that no line can be tangent to $\gamma$ at two distinct points.
\item Show that no circle can be tangent to $\gamma$ at three distinct points. 
\end{enumerate}
\end{thm} %???monotone or monotonic???

{

\begin{wrapfigure}{o}{35 mm}
\vskip-4mm
\centering
\includegraphics{mppics/pic-25}
\vskip0mm
\end{wrapfigure}

Note that part (\ref{ex:double-tangent:a}) does not hold if we allow the curvature to be negative; an example is shown on the diagram.

}

\chapter{Supporting curves}

\section*{Cooriented tangent curves}

Suppose $\gamma_1$ and $\gamma_2$ are smooth regular plane curves.
Recall that the curves $\gamma_1$ and $\gamma_2$ are tangent at the  time parameters $t_1$ and $t_2$
if $\gamma_1(t_1)=\gamma_2(t_2)$
and they share the tangent line at these time parameters;
that is, the tangent lines of $\gamma_1$ at $t_1$ coincides with the tangent line $\gamma_2$ at $t_2$.

In this case the point $p=\gamma_1(t_1)=\gamma_2(t_2)$ is called a \emph{point of tangency} of the curves.
If one of the curves is simple, then we may say that $\gamma_1$ and $\gamma_2$ are tangent at the point $p$ 
without ambiguity.

Note that if $\gamma_1$  $\gamma_2$ are tangent to at the time parameters $t_1$ and $t_2$, 
then the velocity vectors $\gamma_1'(t_1)$ and $\gamma_2'(t_2)$ are parallel.
\begin{figure}[h!]
\vskip-0mm
\centering
\includegraphics{mppics/pic-85}
\vskip0mm
\end{figure}
If $\gamma_1'(t_1)$ and $\gamma_2'(t_2)$ point in the same direction we say that the curves a \emph{cooriented},
if these directions are opposite, we say that the curves are \emph{controriented} at the  time parameters $t_1$ and $t_2$.

Note that reverting parametrization of one of the curves, cooriented curves become counteroriented and the other way around; so we can always assume that the curves are cooriented at the given point of tangency.

\section*{Supporting curves}

Let $\gamma_1$ and $\gamma_2$ be two smooth regular plane curves that share a point \[p=\gamma_1(t_1)=\gamma_2(t_2)\] which is not an endpoint of any of the curves.
Suppose that there is $\eps>0$ such that the arc $\gamma_2|_{[t_2-\eps, t_2+\eps]}$ lies in a closed plane region $R$ with the arc $\gamma_1|_{[t_1-\eps, t_1+\eps]}$ in its boundary,
then we say that $\gamma_1$ supports $\gamma_2$ at the time parameters $t_1$ and $t_2$.
If both curves are simple, then we also could say that $\gamma_1$ \emph{locally supports} $\gamma_2$ at the point $p$ without ambiguity.

\begin{wrapfigure}{o}{43 mm}
\vskip-3mm
\centering
\includegraphics{mppics/pic-86}
\vskip0mm
\end{wrapfigure}

If $\gamma_1$ is simple proper curve, so it divides the plane into two closed region that lie on left and right from $\gamma_1$, then we say that $\gamma_1$ \emph{globally supports} $\gamma_2$ at $t_2$ 
if $\gamma_2$ runs in one of these closed regions and 
$\gamma_2(t_2)$ lies on $\gamma$.

Note that if $\gamma_1$ and $\gamma_2$ share a point $p=\gamma_1(t_1)=\gamma_2(t_2)$ and not tangent at $t_1$ and $t_2$, then  $\gamma_2$ crosses $\gamma_1$ at $t_2$ moving from one of its sides to the other.
It follows that $\gamma_1$ can not locally support $\gamma_2$ at the time parameters $t_1$ and $t_2$.
Whence we get the following:

\begin{thm}{Definition-Observation}
Let $\gamma_1$ and $\gamma_2$ be two smooth regular plane curves.
Suppose $\gamma_1$ locally supports $\gamma_2$ at time parameters $t_1$ and $t_2$.
Then $\gamma_1$ is tangent to $\gamma_2$ at $t_1$ and $t_2$.

In particular, we could say if $\gamma_1$ and $\gamma_2$ are coorineted or controriented at at the time parameters $t_1$ and $t_2$.
If the curves are coorienated and the region $R$ in the definition of supporting curves lie on the right (left) from the arc of $\gamma_1$ then we say that 
$\gamma_1$ supports $\gamma_2$ from the left (correspondingly right).
\end{thm}

If the curves on the diagram oriented according the the arrows then $\gamma_1$ supports $\gamma_2$ from the right at $p$ (as well as $\gamma_2$ supports $\gamma_1$ from the left at $p$).

We say that a smooth regular plane curve $\gamma$ has a \emph{vertex} at $s$
if the signed curvature function is critical at $s$;
that is, if $\skur'(s)_\gamma=0$.
If $\gamma$ is simple we could say that the point $p=\gamma(s)$ is a vertex of $\gamma$ without ambiguity.

\begin{thm}{Exercise}\label{ex:vertex-support}
Assume that osculating circle $\sigma_s$ of a smooth regular simple plane curve $\gamma$ locally supports $\gamma$ at $p=\gamma(s)$.
Show that $p$ is a vertex of $\gamma$.
\end{thm}

\section*{Supporting test}

The following proposition resembles the second derivative test. 

\begin{thm}{Proposition}\label{prop:supporting-circline}
Let $\gamma_1$ and $\gamma_2$ be two smooth regular plane curves.

Suppose $\gamma_1$ locally supports $\gamma_2$ from the left (right) at the time parameters $t_1$ and $t_2$.
Then 
\[\skur_1(t_1)\le \skur_2(t_2)\quad(\text{correspondingly}\quad \skur_1(t_1)\ge \skur_2(t_2)).\]
where $\skur_1$ and $\skur_2$ denote the signed curvature of $\gamma_1$ and $\gamma_2$ correspondingly.

A partial converse also holds.
Namely, if $\gamma_1$ and $\gamma_2$ tangent and cooriented at the time parameters $t_1$ and $t_2$
then $\gamma_1$ locally supports $\gamma_2$ from the left (right) at the time parameters $t_1$ and $t_2$
if 
\[\skur_1(t_1)< \skur_2(t_2)\quad(\text{correspondingly}\quad \skur_1(t_1)> \skur_2(t_2)).\]

\end{thm}

\parit{Proof.} Without loss of generality, we can assume that $t_1=t_2=0$, the shared point $\gamma_1(0)=\gamma_2(0)$ is the origin and the velocity vectors $\gamma'_1(0)$, $\gamma'_2(0)$ point in the direction of $x$-axis.

Note that small arcs of $\gamma_1|_{[-\eps,+\eps]}$ and  $\gamma_2|_{[-\eps,+\eps]}$ can be described as a graph 
$y=f_1(x)$ and $y=f_2(x)$ for smooth functions $f_1$ and $f_2$ such that $f_i(0)=0$ and $f_i'(0)=0$.
Note that $f_1''(0)=\skur_1(0)$ and $f_1''(0)=\skur_1(0)$ (see \ref{ex:curvature-graph})

Clearly, $\gamma_1$ supports $\gamma_2$ from the left (right) if 
\[f_1(x)\le f_2(x)\quad(\text{correspondingly}\quad f_1(x)\ge f_2(x))\]
for all sufficiently small values $x$.
Applying the second derivative test, we get the result.
\qeds


\begin{thm}{Advanced exercise}\label{ex:support}
Let $\gamma_1$ and $\gamma_2$ be two smooth unit-speed simple plane curves that are tangent and cooriented at the point $p=\gamma_1(0)=\gamma_2(0)$.
Assume $\skur_1(s)\ge\skur_2(s)$ for any $s$.
Show that $\gamma_1$ locally supports $\gamma_2$ from the left at $p$.

Give an example of two proper curves $\gamma_1$ and $\gamma_2$ satisfying the above condition such that $\gamma_1$ does not globally support $\gamma_2$ at $p$.
\end{thm}

Note that according to the DNA inequality (\ref{thm:DNA}) for any closed smooth regular curve that runs in a unit disc, the average of its absolute curvature at lest 1; in particular it has a point with absolute curvature at lest 1.
The following exercise says that the last statement holds for loops.

\begin{thm}{Exercise}\label{ex:between-parallels-1}
Assume a closed smooth regular plane loop $\gamma$ runs in a unit disc.
Show that there is a point on $\gamma$ with absolute curvature at least 1.
\end{thm}


\begin{thm}{Exercise}\label{ex:between-parallels-1}
Assume a closed smooth regular plane curve $\gamma$ runs between parallel lines on distance 2 from each other.
Show that there is a point on $\gamma$ with absolute curvature at least 1.

Try to prove the same for a smooth regular plane loop.
\end{thm}

\begin{thm}{Exercise}\label{ex:in-triangle}
Assume a closed smooth regular plane curve $\gamma$ runs inside of a triangle $\triangle$ with inradius 1; that is, the inscribed circle of $\triangle$ has radius 1.
Show that there is a point on $\gamma$ with absolute curvature at least~$1$.
\end{thm}

The last exercise above is a baby case of a \ref{ex:moon-rad}.

{

\begin{wrapfigure}{o}{35 mm}
\vskip-4mm
\centering
\includegraphics{mppics/pic-70}
\vskip0mm
\end{wrapfigure}

\begin{thm}{Exercise}
Let $F$ be a plane figure bounded by two circle arcs $\sigma_1$ and $\sigma_2$ of signed curvature 1 that run from $x$ to $y$.
Suppose $\sigma_1$ is a shorter than $\sigma_2$.
Assume a simple arc $\gamma$ runs in $F$ and has the end points on $\sigma_1$.
Show that the absolute curvature of $\gamma$ is at least 1 at some parameter value.

\end{thm}

}

\section*{Convex curves}

Recall that a plane curve is convex if it bounds a convex region.

\begin{thm}{Proposition}\label{prop:convex}
Suppose that a closed simple curve $\gamma$ bounds a figure $F$.
Then $F$ is convex if and only if the signed curvature of $\gamma$ does not change sign.
\end{thm}


\begin{thm}{Lens lemma}\label{lem:lens}
Let $\gamma$ be a smooth regular simple curve that runs from $x$ to $y$.
Assume that $\gamma$ runs on the right side (correspondingly left side) of the oriented line $xy$ and only its end points $x$ and $y$ lie on the line.
Then $\gamma$ has a point with positive (correspondingly negative) signed curvature.
\end{thm}

\begin{wrapfigure}{o}{35 mm}
\vskip-4mm
\centering
\includegraphics{mppics/pic-22}
\vskip0mm
\end{wrapfigure}

Note that the lemma fails for curves with self-intersections;
the curve $\gamma$ on the diagram always turns right, 
so it has negative curvature everywhere, but it lies on the right side of the line $xy$.

\parit{Proof.}
Choose points $p$ and $q$ on the line $xy$
so that the points $p, x, y, q$ appear in the same order.
We can assume that $p$ and $q$ lie sufficiently far from $x$ and $y$, so the half-disc with diameter $pq$ contains $\gamma$.

Consider the smallest disc segment with chord $[pq]$ that contains $\gamma$.
Note that its arc $\sigma$ supports $\gamma$ at some point $w=\gamma(t_0)$.

\begin{wrapfigure}{o}{50 mm}
\centering
\includegraphics{mppics/pic-24}
\bigskip
\includegraphics{mppics/pic-23}
\vskip0mm
\end{wrapfigure}

Note that the $\gamma$ is tangent to $\sigma$ at $w$.
Let us parameterise $\sigma$ from $p$ to $q$.
Then $\gamma$ and $\sigma$ are cooriented as $w$.
If not, then the arc of $\gamma$ from $w$ to $y$ would be trapped in the curvelinear triangle $xwp$ bounded by arcs of $\sigma$, $\gamma$ and the line segment $[px]$.
But this is impossible since $y$ does not belong to this triangle.

It follows that $\sigma$ supports $\gamma$ at $t_0$ from the right.
By \ref{prop:supporting-circline}, 
\[\skur(t_0)_\gamma\ge \skur_\sigma,\]
Evidently $\skur_\sigma>0$, hence the result.
\qeds

\parit{Remark.}
Instead of taking minimal disc segment, one can take a point $w$ on $\gamma$ that maximize the distance to the line $xy$.
The same argument shows that curvature at $w$ is nonnegative, which is slightly weaker than the required positive curvature.

\begin{wrapfigure}{o}{35 mm}
\centering
\includegraphics{mppics/pic-68}
\vskip0mm
\end{wrapfigure}

\parit{Proof of \ref{prop:convex}.}
If $F$ is convex, then every tangent line of $\gamma$ supports $\gamma$.
If a point moves along $\gamma$, the figure $F$ has to stay on one side from its tangent line;
that is, we can assume that each tangent line supports $\gamma$ on one side, say on the right.
Since line has vanishing curvature, the supporting test (\ref{prop:supporting-circline}) implies that $\skur\ge 0$ at each point.

Now assume $F$ is not convex.
Then there is a line that supports $\gamma$ at two points, say $x$ and $y$ that divide $\gamma$ in two arcs $\gamma_1$ and $\gamma_2$, both distinct from the line segment $[x,y]$.
Note the one of the arcs is parametrized from $x$ to $y$ and the other from $y$ to $x$.
Passing to a smaller arc if necessary we can ensure that only its endpoints lie on the line; 
so we can apply the lens lemma from which we get that the arcs $\gamma_1$ and $\gamma_2$ contain points with curvatures of opposite signs.

That is, if $F$ is not convex, then curvature of $\gamma$ changes sign.
Equivalently: if curvature of $\gamma$ does not change sign then $F$ is convex.
\qeds

\begin{thm}{Exercise}\label{ex:convex small}
Suppose $\gamma$ is a smooth regular simple closed convex plane curve of diameter bigger than 2.
Show that $\gamma$ has a point with absolute curvature less than 1.
\end{thm}

\begin{thm}{Exercise}\label{ex:line-curve-intersections}
Suppose $\gamma$ is a simple smooth regular curve in the plane with positive curvature.
Assume $\gamma$ crosses a line $\ell$ at the points $p_1,p_2,\dots p_n$ and these points appear on $\gamma$ in the same order.
\begin{enumerate}[(a)]

\item\label{ex:line-curve-intersections:a} Show that $p_2$ cannot lie between $p_1$ and $p_3$ on $\ell$.

\item\label{ex:line-curve-intersections:b} Show that if $p_3$ lies between $p_1$ and $p_2$ on $\ell$ then the points appear on $\ell$ in the following order:  
\[p_1,p_3,\dots,p_4 ,p_2.\]

\begin{figure}[h!]
\vskip-0mm
\centering
\includegraphics{mppics/pic-29}
\vskip0mm
\end{figure}

\item Try to describe all possible orders when $p_1$ lies between $p_2$ and $p_3$ (see the diagram).

\end{enumerate}
\end{thm}

\section*{Moon in a puddle}

\begin{thm}{Theorem}\label{thm:moon}
Assume $\gamma$ is a simple closed smooth regular plane curve.
Then at least two of its osculating circles support $\gamma$ from the left and  at least two from the right.
\end{thm}

\begin{wrapfigure}{o}{33 mm}
\vskip4mm
\centering
\includegraphics{mppics/pic-63}
\vskip0mm
\end{wrapfigure}

The diagram shows for supporting osculating circles, two from inside and two outside the curve for the given curve.

The above theorem is a slight generalization of the following theorem proved by Vladimir Ionin and German Pestov in \cite{pestov-ionin}:

\begin{thm}{Theorem}\label{thm:moon-orginal}
Assume $\gamma$ is a simple closed smooth regular plane curve of absolute curvature bounded by 1.
Then it surrounds a unit disc.
\end{thm}

This theorem is a direct corollary of \ref{thm:moon};
indeed, since absolute curvature is bounded by 1, every osculating circle has radius at least 1 and by \ref{thm:moon} two pf these circles are surrounded by $\gamma$.

This theorem gives a simple but nontrivial example of the so-called \emph{local to global theorems} --- based on some local data (in this case the curvature of a curve) we conclude a global property (in this case existence of a large disc surrounded by the curve).
For convex curves, this result was known earlier \cite[\S 24]{blaschke}.

\begin{wrapfigure}{o}{32 mm}
\vskip-0mm
\centering
\includegraphics{mppics/pic-62}
\vskip0mm
\end{wrapfigure}

A straightforward approach to the latter theorem would be to start with some disc in the region bounded by the curve and blow it up to maximize its radius.
However, as one may see from the diagram it does not always lead to a solution a closed plane curve of absolute curvature bounded by 1 may surround a disc of radius smaller than 1 that cannot be enlarged continuously.  

Recall that a vertex of a smooth regular curve is defined as a critical point of its signed curvature;
in particular, any local minimum (or maximum) of the signed curvature is a vertex.

According to \ref{ex:vertex-support}, if an osculating circle supports the curve at the same point $p$, then $p$ is a vertex.
Therefore \ref{thm:moon} implies existence of 4 vertexes of $\gamma$.
That is, we proved the following theorem:

{

\begin{wrapfigure}{o}{20 mm}
\vskip-0mm
\centering
\includegraphics{mppics/pic-26}
\vskip0mm
\end{wrapfigure}

\begin{thm}{Four-vertex theorem}\label{thm:4-vert}
Any smooth regular simple plane curve has at least four
vertices.
\end{thm}

Evidently any closed curve has at least two vertexes --- where the minimum and the maximum of the curvature are attained.
On the diagram the vertexes are marked;
the first curve has one self-intersection and exactly two vertexes;
the second curve has exactly four vertexes and no self-intersections.

}

The four-vertex theorem was first proved by Syamadas Mukhopadhyaya \cite{mukhopadhyaya} for convex curves.
By now it has a large number of different proofs and generalizations.
One of my favorite proofs was given by Robert Osserman \cite{osserman};
the proof of Vladimir Ionin and German Pestov given below is even better.

\parit{Proof of \ref{thm:moon}.}
Denote by $F$ the closed region surrounded by $\gamma$;
as usual we parametrize $\gamma$ so that $F$ lies on the left from it.

First let us show that one osculating circle is supporting $\gamma$ from the left; that is, it lies completely in $F$ --- this is the main part of the proof.

\begin{figure}[h!]%{r}{38 mm}
\vskip-0mm
\centering
\includegraphics{mppics/pic-32}
\vskip0mm
\end{figure}

Assume contrary; that is, the osculating circle at each point $p\in \gamma$ does not lie in $F$.
For each point $p\in\gamma$ let us consider the maximal circle $\sigma$ that lies completely in $F$ and tangent to $\gamma$ at $p$;
in other words, $\sigma$ has minimal signed curvature among these circles.
Note that $\sigma$ has to touch $\gamma$ at another point;
otherwise we could increase its radius slightly while keeping the circle in $F$.

Fix a point $p_1$ and let $\sigma_1$ be the corresponding circle.
Denote by $\gamma_1$ an arc of $\gamma$ from $p_1$ to a first point $q_1$ on $\sigma_1$.
Denote by $\hat\sigma_1$ and $\check\sigma_1$ two arcs of $\sigma_1$ from $p_1$ to $q_1$ such that the cyclic concatenation of $\hat\sigma_1$ and $\gamma_1$ surrounds $\check\sigma_1$.

Let $p_2$ be the midpoint of $\gamma_1$ and $\sigma_2$ be the corresponding circle. 

Note that $\sigma_2$ cannot intersect $\hat\sigma_1$.
Otherwise, if $\sigma_2$ intersects $\hat\sigma_1$ at some point $s$, then $\sigma_2$ has two more common points with $\check\sigma_1$ --- $x$ and $y$, one for each arc of $\sigma_2$ from $p_2$ to $s$.
That is, $\sigma_1$ and $\sigma_2$ have common point $s$, $x$ and $y$.
Therefore $\sigma_1\z=\sigma_2$ as two circles with three common points. 
On the other hand, by construction $p_2\in \sigma_2$ and $p_2\notin \sigma_1$ --- a contradiction.

\begin{wrapfigure}{o}{32 mm}
\vskip-4mm
\centering
\includegraphics{mppics/pic-64}
\caption*{Two ovals on the diagram pretend to be circles.}
\vskip0mm
\end{wrapfigure}

Recall that $\sigma_2$ has to touch $\gamma$ at another point.
From above it follows that it can only touch $\gamma_1$ and therefore we can choose an arc $\gamma_2\subset \gamma_1$ that runs from $p_2$ to a first point $q_2$ on $\sigma_2$.
Note that by construction we have that
\[\length \gamma_2< \tfrac12\cdot\length\gamma_1.\eqlbl{eq:length<length/2}\]

Let us repeat this construction recessively.
We get an infinite sequence of arcs $\gamma_1\supset \gamma_2\supset\dots$.
By \ref{eq:length<length/2}, we also get that 
\[\length\gamma_n\to0\quad\text{as}\quad n\to\infty.\] 
Therefore the intersection 
\[\bigcap_n\gamma_n\]
contains a single point; denote it by $p_\infty$.

Let $\sigma_\infty$ be the corresponding circle at $p_\infty$; it has to touch $\gamma$ at another point $q_\infty$.
The same argument as above shows that $q_\infty\in\gamma_n$ for any $n$.
It follows that $q_\infty =p_\infty$ --- a contradiction.

Now suppose that there is only one point $q\in\gamma$ at which osculating circle supports $\gamma$ from the left.
In this case for any point $p\ne q$ on $\gamma$ the corresponding circle touches $\gamma$ at another point.

Chose a point $p_1\ne q$ on $\gamma$, take its corresponding circle $\sigma_1$ and note that there are two choices for arc $\gamma_1$ one of which does not contain~$q$.
Repeating the same construction starting from $\gamma_1$ we also arrive to a contradiction.

It remains to show that existence of a pair of osculating circlines that support $\gamma$ form the right.
This is done the same way, one only has to change the definition of corresponding circline --- given $p\in\gamma$, it has to be the circline of maximal signed curvature that supports $\gamma$ from the right at $p$.
We leave it as an exercise:

\begin{thm}{Exercise}\label{ex:finish-moon}
List the necessary changes in the proof above for the existence of circlines that support $\gamma$ form the right.\qeds
\end{thm}

{

\begin{wrapfigure}{o}{12 mm}
\vskip-0mm
\centering
\includegraphics{mppics/pic-67}
\vskip0mm
\end{wrapfigure}

Theorem \ref{thm:moon-orginal} admits the following generalization:

\begin{thm}{Theorem}\label{thm:moon-gen}
Let $\gamma$ be a smooth regular simple plane loop.
Suppose that absolute curvature of $\gamma$ does not exceed~1.
Then $\gamma$ surrounds a unit circle.
\end{thm}

}


\begin{thm}{Exercise}\label{ex:moon-loop}
Describe the modifications in the proof of \ref{thm:moon} which are necessary to prove \ref{thm:moon-gen}.
\end{thm}

\begin{thm}{Exercise}\label{ex:moon-rad}
Assume that a closed smooth regular curve $\gamma$ lies in a figure $F$ bounded by a closed simple plane curve.
Suppose that $R$ is the maximal radius of discs that lies in $F$.
Show that absolute curvature of $\gamma$ is at least $\tfrac1R$ at some parameter value.
\end{thm}

\begin{wrapfigure}{o}{30 mm}
\vskip-8mm
\centering
\includegraphics{mppics/pic-65}
\vskip0mm
\end{wrapfigure}

\begin{thm}{Advanced exercise}\label{ex:curve-crosses-circle}
Suppose $\gamma$ is a closed simple smooth regular plane curve and $\sigma$ is a circle.
Assume $\gamma$ crosses $\sigma$ at the points $p_1,\dots,p_{2{\cdot} n}$ and these points appear in the same cycle order on $\gamma$ and on $\sigma$.
Show that osculating circles at $n$ distinct points of $\gamma$ lie inside $\gamma$ and that osculating circles at other $n$ distinct points of $\gamma$ lie outside of~$\gamma$.
In particular the curve $\gamma$ has at least $2\cdot n$ vertexes.

Construct an example of a closed simple smooth regular plane curve $\gamma$ with only 4 vertexes that crosses a given circle at arbitrarily many points. 
\end{thm}

Recall that the \emph{inverse} of a point $x$ with respect to the unit circle centered at the origin is the point $\hat x=\tfrac{x}{|x|^2}$.


\begin{thm}{Advanced exercise}\label{ex:inverse}
Suppose $\gamma$ is a smooth regular plane curve that does not pass thru the origin.
Let $\hat \gamma$ be the inversion of $\gamma$ in the unit circle centered at the origin.
Show that osculating circline of $\hat\gamma$ at $s$ is the inversion of osculating circline of $\gamma$ at $s$.
Conclude that every vertex of $\hat\gamma$ is the inversion of a vertex of $\gamma$.
\end{thm}


Note that the exercise provides an alternative way to finish the proof of \ref{thm:moon} --- once we proved the existence of two osculating circles that support $\gamma$ from the left,
we can apply to $\gamma$ inversion with the center surrounded by $\gamma$.
In this case the curve $\gamma$ is mapped to a curve $\hat \gamma$,
the domain inside $\gamma$ is mapped to the domain outside $\hat\gamma$ and the other way around.
It follows that if an osculating circle supports the obtained curve $\hat\gamma$ on the right 
then its inversion supports  $\gamma$ from the left and the other way around.
That is, from the existance of two supporting circles on the left we also get the existance of two supporting circles on the right.


