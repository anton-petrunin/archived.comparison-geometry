



Choose two points $x,y$ in one level set $L_a$.
Suppose $z=\gamma(t)$.
Note that as $t\to \infty$ we have $\measuredangle\hinge xyz\to\tfrac\pi2$.
Indeed, let $\alpha$ be the extension of $[xy]$ to both sides.
If there is $\eps>0$ such that $\measuredangle\hinge xyz<\tfrac\pi2-\eps$ for a sequence $t_n\to\infty$, then by comparison there is a point $\bar y=\alpha(


Choose a point $x\in\Sigma$ and $t_1,t_2>0$;
set $p=\gamma(0)$, $y=\gamma(-t_1)$, and $z=\gamma(t_2)$.
Consider the triangle $[xyz]$.
Note that 
\[t_1+t_2\le|x-y|_\Sigma+|x-z|_\Sigma\le t_1+t_2+2\cdot|p-x|_\Sigma.\]
It follows that $\angk xyz\to \pi$ as $t_1,t_2\to \infty$.


Choose a level set $L_a$ and a point $x\in L_a$.
Consider geodesic $\gamma_x$ that runs from $x$ in the direction perpendicular to $L_a$.
Let us show that $\gamma_x$ is a line;
moreover it can be parametrized so that $|\gamma'_x|\equiv1$ and 

Choose two level sets $L_a$ and $L_b$.
Suppose $x\in L_a$ and $y\in L_b$.
Since the functions $\bus_\pm$ are 1-Lipschitz,
we get that 
\begin{align*}
|x-y|_\Sigma
\ge
|\bus_\pm(x)-\bus_\pm(y)|=
\end{align*}

In other words, any two points of $L_{c_1}$ and $L_{c_2}$ lie on distance at least $|c_1-c_2|$.

Let us show that for any


















\section{Inheritance lemma}

The following lemma will play key role in the proof of \ref{thm:comp:cat}.

\begin{wrapfigure}{r}{25mm}
\vskip-0mm
\centering
\includegraphics{mppics/pic-2320}
\end{wrapfigure}

\begin{thm}{Inheritance Lemma}
\label{lem:inherit-angle} 
Assume that a triangle $\trig p x y$ in a surface $\Sigma$ \index{decomposed triangle}\emph{decomposes} 
into two triangles $\trig p x z$ and $\trig p y z$;
that is, $\trig p x z$ and $\trig p y z$ have common side $[p z]$, and the sides $[x z]$ and $[z y]$ together form the side $[x y]$ of $\trig p x y$.

If  both triangles $\trig p x z$ and $\trig p y z$ are thin, 
then so is  $\trig p x y$.
\end{thm} 

We shall need the following lemma in plane geometry.


\begin{thm}{Lemma}\label{lem:quadrangle}
Let $\solidtriangle{\tilde p}{\tilde x}{\tilde y}$ be a solid plane triangle; that is, $\solidtriangle{\tilde p}{\tilde x}{\tilde y}\z=\Conv\{{\tilde p},{\tilde x},{\tilde y}\}$.
Given  $\tilde z\in[\tilde x\tilde y]$,
consider  points $\dot p, \dot x, \dot z, \dot y$ in the plane such that 
\begin{align*}
\dist{\dot p}{\dot x}{}&=\dist{\tilde p}{\tilde x}{},
&
\dist{\dot p}{\dot y}{}&=\dist{\tilde p}{\tilde y}{},
&
\dist{\dot p}{\dot z}{}&\le \dist{\tilde p}{\tilde z}{},
\\
\dist{\dot x}{\dot z}{}&=\dist{\tilde x}{\tilde z}{},
&
\dist{\dot y}{\dot z}{}&=\dist{\tilde y}{\tilde z}{},
\end{align*}
where points $\dot x$ and $\dot y$ lie on either side of $[\dot p\dot z]$.
Then there is a short map 
\[F\:\solidtriangle{\tilde p}{\tilde x}{\tilde y}\to \solidtriangle{\dot p}{\dot x}{\dot z}\cup \solidtriangle{\dot p} {\dot y} {\dot z}\]
that maps $\tilde p$, $\tilde x$, $\tilde y$ and $\tilde z$ to $\dot p$, $\dot x$, $\dot y$ and $\dot z$ respectively.
\end{thm}

\begin{figure}[h!]
\vskip-0mm
\centering
\includegraphics{mppics/pic-2324}
\end{figure}

\begin{wrapfigure}{r}{45 mm}
\vskip-4mm
\centering
\includegraphics{mppics/pic-2325}
\end{wrapfigure}

\parit{Proof.} 
Note that 
\begin{align*}
|\dot x-\dot y|&\le |\dot x-\dot z|+|\dot z-\dot y|=
\\
&=|\tilde  x-\tilde  z|+|\tilde  z-\tilde  y|
\\
&=|\tilde  x-\tilde  y|.
\end{align*}
Applying angle monotonicity (\ref{lem:angle-monotonicity}), we get that
\[\measuredangle\hinge{\dot p}{\dot x}{\dot y}\le \measuredangle\hinge{\tilde  p}{\tilde  x}{\tilde  y}.\]
It follows that there are nonoverlapping triangles 
$\trig{\tilde p}{\tilde x}{\tilde w}\cong\trig {\dot p}{\dot x}{\dot z}$ 
and 
$\trig{\tilde p}{\tilde y}{\tilde v}\cong\trig {\dot p}{\dot y}{\dot z}$
 inside of the triangle $\trig{\tilde p}{\tilde x}{\tilde y}$.


Connect points in each pair
$(\tilde z,\tilde v)$, 
$(\tilde v,\tilde w)$ 
and $(\tilde w,\tilde z)$ 
with arcs of circles centered at 
$\tilde y$, $\tilde p$, and $\tilde x$ respectively. 
Define $F$ as follows.
\begin{itemize}
\item Map  $\solidtriangle{\tilde p}{\tilde x}{\tilde w}$ isometrically onto  $\solidtriangle {\dot p}{\dot x}{\dot y}$;
similarly map $\solidtriangle{\tilde p}{\tilde y}{\tilde v}$ onto $\solidtriangle {\dot p}{\dot y}{\dot z}$.
\item If a point $w$ lies in one of the three circular sectors, say at distance $r$ from center of the circle, let $F(w)$ be the point on the corresponding segment 
$[\dot p \dot z]$, 
$[\dot x \dot z]$ 
or $[\dot y \dot z]$ whose distance from the lefthand endpoint of the segment is $r$.
\item Finally, if $w$ lies in the remaining curved triangle $\tilde z \tilde v \tilde w$, 
set $F(w) =\dot z$. 
\end{itemize}
By construction, $F$ satisfies the remaining conditions of the lemma. 
\qeds


\parit{Proof of the inheritance lemma (\ref{lem:inherit-angle}).}
Construct model triangles $\trig{\dot p}{\dot x}{\dot z}\z=\modtrig(p x z)$ 
and $\trig {\dot p} {\dot y} {\dot z}=\modtrig(p y z)$ so that $\dot x$ and $\dot y$ lie on opposite sides of $[\dot p\dot z]$.

\begin{wrapfigure}{r}{30 mm}
\vskip-0mm
\centering
\includegraphics{mppics/pic-2330}
\end{wrapfigure}

Suppose
\[\angk{z}{p}{x}+\angk{z}{p}{y}
<
\pi.\]
Then for some point $\dot w\in[\dot p\dot z]$, we have \[\dist{\dot x}{\dot w}{}+\dist{\dot w}{\dot y}{}
<
\dist{\dot x}{\dot z}{}+\dist{\dot z}{\dot y}{}=\dist{x}{y}{}.\]
Let $w\in[p z]$ correspond to $\dot w$; that is, $\dist{z}{w}{}=\dist{\dot z}{\dot w}{}$. 
Since $\trig p x z$ and $\trig p y z$ are thin, we have 
\[\dist{x}{w}{}+\dist{w}{y}{}<\dist{x}{y}{},\]
contradicting the triangle inequality. 

Thus 
\[\angk{z}{p}{x}+\angk{z}{p}{y}
\ge
\pi.\]
By Alexandrov's lemma (\ref{lem:alex-reformulation}), this is equivalent to 
\[\angk x p z\le\angk x p y.
\eqlbl{eq:for|pz|}\]

Let $\trig{\tilde  p}{\tilde  x}{\tilde  y}=\modtrig (p x y)$ 
and $\tilde  z\in[\tilde  x\tilde  y]$ correspond to $z$; that is, $\dist{x}{z}{}=\dist{\tilde  x}{\tilde  z}{}$.
Inequality~\ref{eq:for|pz|} is equivalent to $\dist{ p}{ z}{}\le \dist{\tilde  p}{\tilde  z}{}$.
Hence  Lemma~\ref{lem:quadrangle} applies;
let $F\:\solidtriangle{\tilde  p}{\tilde  x}{\tilde  y}\to\solidtriangle {\dot p}{\dot x}{\dot z}\cup \solidtriangle {\dot p} {\dot y} {\dot z}$ be the provided short map.

Fix $v,w$ on the sides of $\trig p x y$;
let $\tilde v,\tilde w$ be the corresponding points on the sides of the model triangle $\trig {\tilde p}{\tilde x}{\tilde y}=\modtrig p x y$
and $\dot v,\dot w$ be the corresponding points on the sides of the model triangles $\trig {\dot p} {\dot x}{\dot z}=\modtrig p x z$ and $\trig {\dot p} {\dot y}{\dot z}=\modtrig p y z$.
Denote by $\ell$ the length of shortest curve from $\dot v$ to $\dot w$ in $\solidtriangle {\dot p}{\dot x}{\dot z}\cup \solidtriangle {\dot p} {\dot y} {\dot z}$.
Since $F$ is short, $|\tilde v-\tilde w|_{\RR^2}\ge \ell$.
Since both triangles $\trig p x z$ and $\trig p y z$ are thin, $\ell\ge|v-w|_{\Sigma}$.

It follows that $|\tilde v-\tilde w|_{\RR^2}\ge |v-w|_{\Sigma}$ for any $v$ and $w$;
that is, the triangle $\trig p x y$ is thin.
\qeds

\section{Nonpositive curvature}

\begin{thm}{Lemma}
Let $\Sigma$ be an open smooth regular surface with nonpositive Gauss curvature.
Then for any two points $x$ and $y$ in $\Sigma$ are joined by unique geodesic $[xy]$.

Moreover the geodesic $[xy]$ depends continuously on its endpoints $x$ and $y$. 
That is, if $\gamma_{[xy]}\:[0,1]\to \Sigma$ is the constant speed parametrization of $[xy]$ from $x$ to $y$,
then the map $(x,y,t)\mapsto \gamma_{[xy]}(t)$ is continuous in three arguments.
\end{thm}

\parit{Proof.}
The first part of lemma follows from \ref{ex:unique-geod}.

For the second part, assume contrary, that is, $x_n\to x$, $y_n\to y$ and $t_n\to t$ as $n\to \infty$ and 
$\gamma_{[x_ny_n]}(t_n)$ does not converge to $\gamma_{[xy]}(t)$. 
Then we can pass to a subsequence such that $\gamma_{[x_ny_n]}(t_n)$ converges to a point distinct from $w\ne \gamma_{[xy]}(t)$.
Note that $w\notin [xy]$. 
Therefore there will be two distinct geodesics from $x$ to $y$;
one is $[xy]$ which does not contain $w$ and the other is the limit of $[x_ny_n]$ which contains $w$ --- a contradiction.
\qeds

\parit{Proof of \ref{thm:comp:cat}.}
Fix a triangle $[p x y]$; 
by Proposition~\ref{prop:comp-reformulations}, it is sufficient to show that the triangle $\trig p x y$ is thin.   

Divide $[xy]$ by points $x=x^{0,N},\dots,x^{N,N}=y$ into $N$ equal parts.
Further divide each geodesic $[p\,x^{i,N}]$ into $N$ equal parts by points $p=x^{i,0},\dots,x^{i,N}$.
Assume $N$ is large.
Since the geodesic depends continuously on its end points, we can assume that each triangle 
 $\trig{x^{i,j}}{\,x^{i,j+1}}{\,x^{i+1,j+1}}$ and $\trig{x^{i,j}}{\,x^{i+1,j}}{\,x^{i+1,j+1}}$ is small;
 in particular, by local comparison (\ref{thm:loc-comp}), each of these triangles is thin. 
 
\begin{figure}[h!]
\vskip0mm
\centering
\includegraphics{mppics/pic-2335}
\end{figure}

Now we show that the thin property propagates to $\trig p x y$ by repeated application of the inheritance lemma (\ref{lem:inherit-angle}):
\begin{itemize}
\item 
First, for fixed $i$, 
sequentially applying the lemma shows  that the triangles 
$\trig{x}{\,x^{i,1}}{\,x^{i+1,2}}$, 
$\trig{x}{\,x^{i,2}}{\,x^{i+1,2}}$, 
$\trig{x}{\,x^{i,2}}{\,x^{i+1,3}}$,
and so on are thin. 
\end{itemize}
In particular, for each $i$, the long triangle $\trig{x}{\,x^{i,N}}{\,x^{i+1,N}}$ is thin.
\begin{itemize} 
\item 
Applying the lemma again shows that the  triangles $\trig{x}{\,x^{0,N}}{\,x^{2,N}}$, $\trig{x}{\,x^{0,N}}{\,x^{3,N}}$, and so on are thin. 
\end{itemize}
In particular, $\trig p x y=\trig{p}{\,x^{0,N}}{\,x^{N,N}}$ is thin.
\qeds




\begin{thm}{Exercise}\label{ex:convex-dist+}
Suppose $\delta$ is a geodesic in an open smooth regular simply connected surface $\Sigma$ with nonpositive Gauss curvature.
Given a point $p\in\Delta$, denote by $f(p)$ the intrinsic distance from $p$ to $\delta$;
that is,
\[f(p)=\inf\set{|p-x|_\Sigma}{x\in\delta}.\]
Use \ref{ex:convex-dist} to show that the function $f$ is convex;
that is, for any geodesic $\gamma$ in $\Sigma$ the function $t\mapsto f\circ\gamma(t)$ is concave.
\end{thm}
































\subsection*{Morse lemma}

\begin{thm}{Lemma}\label{lem:morse}
Let $f(\bm{x})$ be a smooth function defined for $\bm{x}=(x_1,\dots,x_n$ in a neighborhood of the origin in $\RR^n$.
Suppose that $f(\bm{0})=0$, $\tfrac{\partial f}{\partial x_i}(\bm{0})=0$ and the Hessian matrix
\[M=\begin{pmatrix}
\dfrac{\partial f}{\partial x_1\partial x_1} & \cdots & \dfrac{\partial f}{\partial x_1\partial x_n}\\
\vdots & \ddots & \vdots\\
\dfrac{\partial f}{\partial x_n\partial x_1} & \cdots & \dfrac{\partial f}{\partial x_n\partial x_n} \end{pmatrix}\]
is invertable at zero.

Then there is a 
\end{thm}
















\begin{thm}{Advanced exercise}\label{ex:arm-lemma}
Let $\tilde x_1\dots\tilde x_n$ be a convex plane polygon and
$x_1\dots x_n$ be a broken geodesic in an open simply connected surface $\Sigma$ with nonpositive curvature.
Assume that
$|x_i-x_{i-1}|_\Sigma=|\tilde x_i-\tilde x_{i-1}|_{\RR^2}$ and
$\measuredangle\hinge{x_{i-1}}{x_i}{x_{i+1}}\ge \measuredangle\hinge{x_{i-1}}{x_i}{x_{i+1}}$
 for each $i$.
Show that \[|x_1-x_n|_\Sigma\ge |\tilde x_1\z-\tilde x_n|_{\RR^2}.\]
\end{thm}

For $\Sigma=\RR^2$, the exercise above is the so called \index{arm lemma}\emph{arm lemma} \cite{sabitov}; 
you can use it without proof.

\parbf{\ref{ex:arm-lemma}.} 
This problem is very hard, we put it as a joke.

One way is to use the so called \emph{Reshetnyak majorization theorem} and standard arm lemma; it is done this way in \cite{alexander-kapovitch-petrunin2027}.






















\begin{thm}{Lemma}\label{lem:graph}
Suppose $\Sigma$ is an open surface in with positive Gauss curvature in the Euclidean space.
Then there is a convex function $f$ defined on a convex open region of $(x,y)$-plane 
such that $\Sigma$ can be presented as a graph $z=f(x,y)$ in some $(x,y,z)$-coordinate system of the Euclidean space.

Moreover 
\[\iint_\Sigma K\le 2\cdot\pi.\eqlbl{eq:int=<2pi}\]

\end{thm}

\parit{Proof.}
The surface $\Sigma$ is a boundary of an unbounded closed convex set $K$.

Fix $p\in \Sigma$ and consider a sequence of points $x_n$ such that $|x_n-p|\z\to \infty$ as $n\to \infty$.
Set $u_n=\tfrac{x_n-p}{|x_n-p|}$; the unit vector in the direction from $p$ to $x_n$.
Since the unit sphere is compact, we can pass to a subsequence of $(x_n)$ such that $u_n$ converges to a unit vector $u$.

Note that for any $q\in \Sigma$, the directions $v_n=\tfrac{x_n-q}{|x_n-q|}$ converge to $u$ as well.
The half-line from $q$ in the direction of $u$ lies in $K$.
Indeed any point on the half-line is a limit of points on the line segments $[qx_n]$;
since $K$ is closed, all of these poins lie in $K$.


Let us choose the $z$-axis in the direction of $u$.
Note that line segments can not lie in $\Sigma$, otherwise its Gauss curvature would vanish.
It follows that any vertical line can intersect $\Sigma$ at most at one point.
That is, $\Sigma$ is a graph of a function $z=f(x,y)$.
Since $K$ is convex, the function $f$ is convex and it is defined in a region $\Omega$ which is convex.
The domain $\Omega$ is the projection of $\Sigma$ to the $(x,y)$-plane.
This projection is injective and by the inverse function theorem, it maps open sets in $\Sigma$ to open sets in the plane;
hence $\Omega$ is open.

It follows that the outer normal vectors to $\Sigma$ at any point, points to the south hemisphere $\mathbb{S}^2_-=\set{(x,y,z)\in\mathbb{S}^2}{z< 0}$.
Therefore the area of the spherical image of $\Sigma$ is at most $\area\mathbb{S}^2_-= 2\cdot\pi$.
The area of this image is the integral of the Gauss curvature along $\Sigma$.
That is,
\begin{align*}
\iint_{\Sigma}K&=\area[\Norm(\Sigma)]\le 
\\
&\le \area\mathbb{S}^2_-=
\\
&=2\cdot\pi,
\end{align*}
where $\Norm(p)$ denotes the outer unit normal vector at $p$.
Hence \ref{eq:int=<2pi} follows.
\qeds

























Since $\Sigma_0$ is a plane domain, we have $\area[\Norm_0(\Delta_0)]=0$.
Therefore by \ref{prop:total-signed-curvature} we gave 
\[\tgc{\partial\Delta_0}_{\Sigma_0}+\area[\Norm_0(\Delta_0)]=2\cdot \pi.\]

Note that 
\[\tgc{\partial\Delta_t}_{\Sigma_t}+\area[\Norm_t(\Delta_t)]\]
depends continuously on $t$.
According to \ref{eq:sum=2pin}, its value is a multiple of $2\cdot\pi$;
therefore it has to be constant.
Whence the Gauss--Bonnet formula follows.

If $\Delta$ does not lie in one graph, then one could divide it into smaller discs, apply the formula for each and sum up the result.
The proof is done along the same lines as \ref{prop:area-of-spher-polygon}.

By \ref{prop:K(semigeodesic)},
it is suffucient to prove the formula in case when $\Delta$ is covered by a semigeodesic chart.
The same argument shows that it is suffucient to prove the formula in case when $\Delta$ lies in a graph of a smooth function.


Now applying
\ref{prop:pt+tgc},
\ref{lem:rotation-parallel},
and \ref{lem:rotation-semigeoesic} to $\Delta$, we
get
\[\rot_\gamma\vec+\iint_{\Delta}=2\cdot n\cdot \pi
\eqlbl{eq:rot-intK}\]
for an integer $n$.
Since $\Delta$ is a disc, we can \index{homotopy}\emph{homotop} (deform) $\gamma$ continuousely to a point.
Namely we can choose a continuous map $(t,\tau)\mapsto \gamma_\tau(t)$ such that $\gamma_1(t)$ is constant,
$\gamma(t)=\gamma_0(t)$ for any $t$,
and $t\to \gamma_\tau(t)$ is a simple loop for any fixed $\tau\in [0,1]$.

Denote by $\Delta_\tau$ the disc bounded by $\gamma_\tau$;
it degenerates to a point for $\tau=1$.
Therefore we have 
\[\rot_{\gamma_1}\vec+\iint_{\Delta_1}K=0.\]
Now note that value $\rot_{\gamma_\tau}\vec+\iint_{\Delta_\tau}K$ depend continusely on $\tau$ and it must be an integer multiple of $2\cdot\pi$.
Therefore this value does not depend on $\tau$;
whence $n=0$ in \ref{eq:rot-intK}. 































\begin{thm}{Exercise}\label{ex:cohn-vossen}
Suppose that $\Sigma$ is an open smooth regular surface with positive Gauss curvature.
Show that any two both side infinite geodesics on $\Sigma$ have a common point.
\end{thm}

\parbf{\ref{ex:cohn-vossen}.}
Suppose that $\gamma_1$ and $\gamma_2$ are two both-side infinite geodesics on $\Sigma$.

Consider the following cases:
\begin{itemize}
 \item both $\gamma_1$ and $\gamma_2$;
\end{itemize}




















\chapter{Positive Gauss curvature}




\section*{Closed surfaces}

\begin{thm}{Lemma}\label{lem:gauss=sphere}
Assume $\Sigma$ is a closed smooth surface with positive Gauss curvature.
Then $\Sigma$ is a smooth sphere; that is, $\Sigma$ admits a smooth regular parametrization by $\mathbb{S}^2$.
\end{thm}

\begin{wrapfigure}{O}{33 mm}
\vskip-0mm
\centering
\includegraphics{mppics/pic-78}
\vskip-0mm
\end{wrapfigure}

\parit{Proof.}
Without loss of generality we can assume that the origin lies in the interior of the convex region $R$ bounded by $\Sigma$.

By convexity of $R$, any half-line starting at the origin intersects $\Sigma$ at a single point;
that is, there is a positive function $\rho\:\mathbb{S}^2\to\RR$ such that $\Sigma$ is formed by points $q=\rho(\xi)\cdot \xi$ for $\xi\in \mathbb{S}^2$.

Let us show that $\rho$ is a smooth function.
Fix a point $p=(x_p,y_p,z_p)$ on $\Sigma$.
Consider a local implicit description of $\Sigma$ at $p$ as a solution of equation $h(x,y,z)=0$ with nonvanishing gradient;
so $h(p)=0$.
Note that for any point $q$ in a neighborhood of $p$, we have that 
\[h(q)=0 \iff q\in \Sigma \iff q=\rho(\xi)\cdot\xi\] for some $\xi\in \mathbb{S}^2$.
In other words $h$ defines implicitly $\rho$ as in the implicit function theorem.



Recall that $\nabla_p h\perp \T_p$.
Since the origin lies in the interior of $R$, it cannot lie on $\T_p$;
that is, $\langle\nabla_p f,p \rangle\ne0$;
or equivalently $D_\eta h(p)\ne 0$, where $\eta=\tfrac{p}{|p|}$ is the unit vector in the direction of $p$ and $D$ denotes the directional derivative.

Fix a $(u,v)$ chart $s$ on $\mathbb{S}^2$ in a neighborhood of $\eta$;
note that the map 
\[S\:(u,v,w)\mapsto w\cdot s(u,v)\] 
is smooth and regular for $w>0$; that is, the vectors 
\[\tfrac{\partial S}{\partial u},\quad \tfrac{\partial S}{\partial v},\quad  \tfrac{\partial S}{\partial w}=w\] are linearly independent.
Note that the function $h\circ S$ is smooth and
$\tfrac{\partial h\circ S}{\partial \rho}(p)=D_\eta h(p)\ne 0$.
Applying implicit function theorem, we get that $\rho$ is smooth in a neighborhood of $\eta$;
since $\eta$ is arbitrary, $\rho$ is smooth on whole $\mathbb{S}^2$.
\qeds


If one only needs to show that $\Sigma$ is a topological sphere, then one only needs to show that $\rho$ is continuous.
The latter is a consequence of another classical result in topology --- the so-called \index{closed graph theorem}\emph{closed graph theorem}.





\section*{Open surfaces}

\begin{thm}{Lemma}\label{lem:graph}
Suppose $\Sigma$ is an open surface with positive Gauss curvature.
Then there is a coordinate system such that 
$\Sigma$ is a graph $z=f(x,y)$ of a convex function $f$ defined on a convex open region of the $(x,y)$-plane.
\end{thm}

\parit{Proof.}
\qeds

 




 


























\section*{Immersed surfaces}

It seems that first formulation and proof of the following theorem was given by James Stoker \cite{stoker} who attributed it to Jacques Hadamard, who proved a closely relevant statement in \cite[item 23]{hadamard}.


\begin{thm}{Theorem}\label{thm:convex-immersed}
Any closed connected immersed surface with positive Gauss curvature is embedded.
\end{thm}

\begin{wrapfigure}{o}{20 mm}
\vskip-0mm
\centering
\includegraphics{mppics/pic-35}
\vskip-0mm
\end{wrapfigure}

In other words such surface can not have self-intersections.
Note that an analogous statement does not hold in the plane;
on the diagram you can see a closed curve with a self-intersection and positive curvature at all points.
Exercise~\ref{ex:convex+2pi} gives a condition that guarantees simplicity of a locally convex plane curve;
it will be used in the following proof.



Before going into the proof, note that theorems \ref{thm:convex-immersed} and \ref{thm:convex-embedded}
imply the following:

\begin{thm}{Corollary}
Any closed connected immersed surface with positive Gauss curvature is an embedded sphere that bounds a convex region.
\end{thm}

In the following sections we will give one complete proof and sketch an alternative proof.

The first proof uses a \index{Morse-type argument}\emph{Morse-type argument} for the height function;
that is, we study how the part of the surface that lies below a plane changes when we move the plane upward.
Little more careful analysis of this changes would imply the corollary above directly, without using Theorem~\ref{thm:convex-embedded}.

The sketch use equidistants surfaces and the Gauss map.
We will not proove a topological statement relying on intuition.

In the proof we abuse notation slightly;
we say a {}\emph{point of the immersed surface} instead of a {}\emph{point in the parameter domain of the immersed surface}.
So each point of self-intersection is considered as two or more ``distinct'' points of the surface.

\section*{Morse-type proof}

Let $\Sigma$ be a closed surface with positive Gauss curvature, possibly with self-intersections. 

Fix a horizontal plane $\Pi_h$ defined by the equation $z=h$ in an $(x,y,z)$-coordinate system.
Note that the intersection $W_h=\Sigma\cap\Pi_h$ is formed by a finite collection of closed curves and isolated points.
(These curves and isolated points might intersect in the Euclidean space, but they are disjoint in the domain of parameters of $\Sigma$.)

Indeed, if $\T_p=\Pi_h$, then, since the principle curvatures are positive, $p$ is a local minimum or local maximum of the height function.
In both cases, $p$ is an isolated point of $W_h$ in $\Sigma$.
If the tangent plane $\T_p$ is not $\Pi_h$, then it is not perpendicular to $(x,z)$-plane or $(y,z)$-plane.
Therefore by Proposition~\ref{prop:perp}, the surface can be written locally as a graph $x=f(y,z)$ or $y=f(x,z)$;
in both cases $p$ lies on the curve $x=f(y,h)$ or $y=f(x,h)$ respectively.

Summarizing, the closed set $W_h\subset \Sigma$ locally looks like a curve or an isolated point.
Since $\Sigma$ is compact, so is $W$.
Therefore $W$ is a finite disjoint collection of isolated points and closed simple curves in $\Sigma$.

Assume $\alpha_{h_0}$ is a closed curve in $W_{h_0}$.
Note that its neighborhood is swept by curves $\alpha_h$ in $W_{h}$ for $h\approx h_0$.
Indeed a neighborhood of $\alpha_{h_0}$ in $\Sigma$ can be covered by a finite number of graphs of the type $x=f(y,z)$ (or $y=f(x,z)$) and the curves $\alpha_h$ can be described locally as curves of the form $t\mapsto (f(t,h),t,h)$ (or respectively $t\z\mapsto(t,f(t,h),h)$) for $h\approx h_0$.

As $\alpha_h$ is the intersection of a locally convex surface with a plane,
the curvature of $\alpha_h$ has fixed sign;
so if we choose an orientation for the curves properly, we can assume that they all have positive curvature.

The family $\alpha_h$ depends smoothly on $h$ and the same holds for its tangent indicatrix.
Therefore the total signed curvature $K_h$ of $\alpha_h$ depends continuously on $h$.
If $K_h=2\cdot\pi$ for some $h$, then $K_h=2\cdot\pi$ for every $h$.
It follows since, the function $h\mapsto K_h$ is continuous and its value is a multiple of $2\cdot\pi$.
In this case, by Exercise~\ref{ex:convex+2pi}, all curves $\alpha_h$ are simple and each one bounds a convex region in the plane $\Pi_h$.

Summarizing, if one of the curves in the constructed family $\alpha_{h}$ is simple,
then each curve in the family is simple and each $\alpha_{h}$ bounds a convex region in the plane $\Pi_h$. 

Choose a point $p\in \Sigma$ that minimizes the height function $z$.
Without loss of generality we may assume that $p$ is the origin and therefore the surface lies in the upper half-space.

Fix $h>0$.
The intersection of the set $z\le h$ with the surface may contain several connected components;
one of them contains $p$, denote this component by $\Sigma_h$.\footnote{These components might intersect in the space, but they are disjoint in the domain of parameters. Note also that from the corollary, it follows that there is only one component $\Sigma_h$, but we can not use it before the theorem is proved.}


From above, $\Sigma_h$ is a surface with possibly nonempty boundary.
Indeed it might be bounded only by few closed curves in $W_h$;
any isolated point of $W_h$ either lies in $\Sigma_h$ together with its neighborhood or does not lie in $\Sigma_h$.


Note that for small values of $h$, the surface $\Sigma_h$ is an embedded disc.
Indeed, if $z=f(x,y)$ is a graph represetation of $\Sigma$ around $p$,
then $\Sigma_h$ is a graph of $f$ over 
\[\Delta=\set{(x,y)\in\RR^2}{f(x,y)\le h}.\]
Since the Gauss curvature is positive, the function $f$ is convex and therefore $\Delta$ is convex and bounded by a smooth curve;
any such set can be parameterized by a disc.

\begin{wrapfigure}{o}{30 mm}
\vskip-0mm
\centering
\includegraphics{mppics/pic-36}
\vskip-0mm
\end{wrapfigure}

Let $H>0$ be the maximal value such that $\Sigma_h$ has no self-intersections for any $h<H$.
For a sequence $h_n\to H^-$, choose a point $q_n$ on the boundary of $\Sigma_h$ and pass to a partial limit $q$ of $q_n$ in $\Sigma$;
that is, $q$ is a limit of a subsequence of~$(q_n)$.

If the tangent plane at $q$ is {}\emph{not} horizontal, 
then there is a closed curve $\alpha_H$ in $\Sigma$ that passes thru $q$ and lies on the plane $z=H$.
From the above discussion, the curve $\alpha_H$, as well as all $\alpha_h$ with $h\approx H$ are closed embedded convex curves.
Hence $\Sigma_h$ has no self-intersections for some $h>H$ --- a contradiction.

If the tangent plane at $q$ is horizontal,
then the surface $\Sigma_H$ has no boundary.
Since $\Sigma$ is connected, $\Sigma_H=\Sigma$.
Since $\Sigma_h$ has no self-intersections for $h<H$, we get that $\Sigma$ is an embedded surface.
\qeds

\begin{thm}{Exercise}
Modify the proof of the theorem to show that any open immersed surface with positive Gauss curvature is embedded.
\end{thm}




























\section*{DNA inequality revisited*}

In this section we will give an alternative proof of \ref{thm:DNA} that works for arbitrary, not necessarily smooth, curves.
In the proof we use \ref{def:total-curv-poly} to define for the total curvature;
according to \ref{thm:total-curvature=}, it is more general than the smooth definition given on page \pageref{page:total curvature of:smooth-def}.

\parit{Alternative proof of \ref{thm:DNA}.}
We will show that 
\[\tc\gamma> \length\gamma.\]
for any closed polygonal line $\gamma=p_1\dots p_{n}$ in a unit ball.
It implies the theorem since in any nontrivial closed curve we can inscribe a closed polygonal line with arbitrary close total curvature and length.

The indexes are taken modulo $n$, in particular $p_{n}=p_0$, $p_{n+1}=p_1$ and so on.
Denote by $\theta_i$ the external angle of $\gamma$ at $p_i$;
that is,
\[\theta_i=\pi-\measuredangle p_{i-1}p_ip_{i+1}.\]

Denote by $o$ the center of the ball.
Consider a sequence of $n+1$ plane triangles
\begin{align*}
[q_0s_0q_1]
&\cong 
[p_0op_1],
\\
[q_1s_1q_2]
&\cong 
[p_1op_2],
\\
&\dots
\\
[q_{n}s_nq_{n+1}]
&\cong 
[p_nop_{n+1}],
\end{align*}
such that the points $q_0,q_1\dots$ lie on one line in that order and all the points $s_0,\dots,s_n$ lie on one side from this line.

\begin{figure}[h!]
\vskip-0mm
\centering
\includegraphics{mppics/pic-16}
\vskip0mm
\end{figure}

Since $p_0=p_n$ and $p_1=p_{n+1}$, we have that congruence of the following triangles:
\[[q_{n}s_nq_{n+1}]\cong 
[p_nop_{n+1}]=[p_0op_1]\cong[q_{0}s_0q_1],\]
so $s_0q_0q_ns_n$ is a parallelogram.
Therefore
\begin{align*}
|s_0-s_1|+\dots+|s_{n-1}-s_n|
&\ge|s_n-s_0|=
\\
&=|q_0-q_n|=
\\
&=|p_0-p_1|+\dots+|p_{n-1}-p_n|
\\
&=\length \gamma.
\end{align*}

Note that 
\begin{align*}
\theta_i&=\pi-\measuredangle p_{i-1}p_ip_{i+1}\ge
\\
&\ge\pi-\measuredangle p_{i-1}p_io-\measuredangle op_ip_{i+1}=
\\
&=\pi-\measuredangle q_{i-1}q_is_{i-1}-\measuredangle s_iq_iq_{i+1}=
\\
&=\measuredangle s_{i-1}q_is_i>
\\
&>|s_{i-1}-s_i|;
\end{align*}
the last inequality follows since $|q_i-s_{i-1}|=|q_i-s_i|=|p_i-o|\le 1$.
That is, 
\[\theta_i>|s_{i-1}-s_i|\]
for each $i$.

It follows that
\begin{align*}
\tc \gamma
&=\theta_1+\dots+\theta_n>
\\
&> |s_{0}-s_1|+\dots |s_{n-1}-s_n|\ge 
\\
&\ge\length \gamma.
\end{align*}
Hence the result.
\qeds


Let us mention the following closely related statement:

\begin{thm}{Theorem}
Suppose a closed regular smooth curve $\gamma$ lies in a convex figure with the perimeter $2\cdot \pi$.
Then 
\[\tc\gamma\ge \length\gamma.\]

\end{thm}

It was  conjectured by Serge Tabachnikov~\cite{tabachnikov} and proved by Jeffrey Lagarias and Thomas Richardson \cite{lagarias-richardso}; latter a simpler proof was given by Alexander Nazarov and Fedor Petrov~\cite{nazarov-petrov}.
Despite the simplicity of the formulation, the proof is annoyingly difficult.




























\chapter{Spherical map}





\section*{Spherical image}


\begin{thm}{Theorem}\label{thm:spherical-image}
Let $\Sigma$ be an oriented proper surface without boundary and with positive Gauss curvature.
Then the spherical map $\Norm\:\Sigma\to\mathbb{S}^2$ is injective and
\[\int_R K=\area[\Norm(R)]\]
for any region $R$ in $\Sigma$. %??? measurable
\end{thm}

\parit{Proof.} 
Fix two distinct points $p,q\in\Sigma$.
Recall that $\Sigma$ bounds a strictly convex region.
Therefore the $\Norm_p$ makes an obtuse angle with the line segment $[p,q]$. %??? why
The same way we can show that $\Norm_q$ makes an obtuse angle with the line segment $[q,p]$.
In other words the projections of $\Norm_p$ and $\Norm_q$ on the line $pq$ point in the opposite directions.
In particular $\Norm_p\ne \Norm_q$; that is, the spherical map is injective.

It remains to prove the identity.
Note that we can assume that the region $R$ is covered by one chart $(u,v)\mapsto s(u,v)$; if not cut $R$ into smaller regions and sum up the results.

Applying the definition of integral, we have the following expression for the left hand side
\[\int_R K \df \iint_{s^{-1}(R)} K[s(u,v)]\cdot |\tfrac{\partial s}{\partial v}(u,v)\times\tfrac{\partial s}{\partial u}(u,v)|  \cdot du\cdot dv.\]
Applying the definition of area, we have the following expression for the right hand side
\[\area[\Norm(R)] \df \iint_{s^{-1}(R)}  |\tfrac{\partial \Norm\circ s}{\partial v}(u,v)\times\tfrac{\partial \Norm\circ s}{\partial u}(u,v)|  \cdot du\cdot dv.\]
Therefore it is sufficient to show that 
\[\tfrac{\partial \Norm\circ s}{\partial v}(u,v)\times\tfrac{\partial \Norm\circ s}{\partial u}(u,v)=K[s(u,v)]\cdot \tfrac{\partial s}{\partial v}(u,v)\times\tfrac{\partial s}{\partial u}(u,v)\eqlbl{eq:gauss-curv}\]
for any $(u,v)$ in the domain of definition. 

Fix a point $p=s(u,v)$.
Recall that 
\[\tfrac{\partial \Norm\circ s}{\partial u}=S_p(\tfrac{\partial  s}{\partial u})\quad\text{and}\quad\tfrac{\partial \Norm\circ s}{\partial v}=S_p(\tfrac{\partial  s}{\partial v}).\]
Therefore 
\[\tfrac{\partial \Norm\circ s}{\partial v}\times\tfrac{\partial \Norm\circ s}{\partial u}=\det S_p\cdot \tfrac{\partial s}{\partial v}\times\tfrac{\partial s}{\partial u}.\]
Since $K(p)=\det S_p$, \ref{eq:gauss-curv} follows.
\qeds

\begin{thm}{Exercise}\label{ex:int-gauss=4pi}
Let $\Sigma$ be a closed surface with positive Gauss curvature.
Show that 
\[\int_\Sigma K=4\cdot\pi.\]

\end{thm}

\begin{thm}{Exercise}\label{ex:gauss-integral-open}
Let $\Sigma$ be an open surface with positive Gauss curvature.
Show that 
\[\int_\Sigma K\le 2\cdot\pi.\]

\end{thm}































\chapter{Geodesics vs shortest paths}

Let $\Sigma$ be a smooth surface.
A smooth curve $\gamma$ on $\Sigma$ is called geodesic if its osculation is perpendicular to $\Sigma$ at each point;
that is, if $\gamma''(t)\perp\T_{\gamma(t)}$ at any time $t$.

\begin{thm}{Proposition}
Given an initial point $p\in \Sigma$ and a tangent vector $v\in\T_p\Sigma$ there is unique geodesic $\gamma$ defined on a maximal open interval containing zero that starts at $p$ with velocity $v$; that is,
such that $\gamma(0)=p$ and $\gamma'(0)=v$.
\end{thm}

\parit{Informal sketch.}
Consider a local description of $\Sigma$ as a graph $z=f(x,y)$ in the tangent-normal coordinates at $p$.
Any curve $\gamma$ in $\Sigma$ that starts at $p$ can be written locally as 
$\gamma(t)=(x(t),y(t),f(x(t),y(t)))$ in a neighborhood of $0$.

Let us show that the condition $\gamma''(t)\perp\T_{\gamma(t)}$ can be written as an odinary differential equation.
Indeed, the tangent plane is spanned by two vectors $(1,0,\tfrac{\partial f}{\partial x})$ and $(0,1,\tfrac{\partial f}{\partial y})$.
The acceleration of $\gamma$ can be written as $\gamma''(t)=(x'',y'',\tfrac{\partial^2 f}{\partial x^2}\cdot (x')^2+\tfrac{\partial^2 f}{\partial y^2}\cdot (y')^2+\tfrac{\partial f}{\partial x}\cdot x''+\tfrac{\partial f}{\partial y}\cdot y'')$.
Therefore we get $\gamma''(t)\perp\T_{\gamma(t)}$ is equivalent two equations:




The smoothness should be intuitively obvious; at least the curve should be twice differentiable otherwise it can be shortened.

Let us give an informal physical explanation why $\gamma''(t)\perp\T_{\gamma(t)}\Sigma$.
One may think about the geodesic $\gamma$ as of stable position of a stretched elastic thread that is forced to lie on a frictionless surface.
Since it is frictionless, the force density $N(t)$ that keeps the geodesic $\gamma$ in the surface must be therefore proportional to the normal vector to the surface at $\gamma(t)$.
The tension in the thread has to be the same at all points (otherwise the thread would move back or forth and it would not be stable).
The tension at the ends of small arc is roughly proportional to the angle between the tangent lines at the ends of the arc. 
Passing to the limit as the length of the arc goes to zero, we get that the density of this force $F(t)$ is proportional to $\gamma''(t)$.
According to the second Newton's law, we have $F(t)+N(t)=0$;
which implies that  $\gamma''(t)$ is perpendicular to $\T_{\gamma(t)}\Sigma$.%
\footnote{In fact $\gamma''(t)+\nu\cdot \langle s(\gamma'(t)),\gamma'(t)\rangle=0$, where $s$ is the shape operator of $\Sigma$ at $\gamma(t)$ or equivalently,
$\gamma''(t)+\nu\cdot  \langle S(\gamma'(t)),\gamma'(t)\rangle=0$, where $S$ is the shape operator of $\Sigma$ at $\gamma(t)$.}

The third statement can be also understood using physical intuition --- $\gamma(t)$ is the trajectory of a particle that slides on $\Sigma$ without friction and with initial velocity $v$.
Formally, existence and uniqueness follows from Picard's theorem (the  fundamental theorem of ordinary differential equations).
\qeds




























\chapter{Local comparison}

\section{First variation formula}

\begin{thm}{Proposition}\label{prop:first-var}
Assume $(s,t)\mapsto w(s,t)$ be a local parametrization of an oriented smooth regular surface $\Sigma$ such that 
$\tfrac{\partial}{\partial s}w\perp \tfrac{\partial}{\partial t}w$, $|\tfrac{\partial}{\partial s}w|=1$ and the vector $\tfrac{\partial}{\partial s}w$ points to the right from $\tfrac{\partial}{\partial t}w$ at any parameter value $(s,t)$.

Fix a closed real interval $[a,b]$ and consider a one parameter family of curves $\sigma_s\:[a,b]\to \Sigma$  defined as the coordinate lines $\sigma_s(t)=w(s,t)$.
Set $\ell(s)=\length \sigma_s$.
Then
\[\ell'(s)=\Theta_{\sigma_s}\]
for any $s$.
\end{thm}

The proof is done by direct calculations.

\parit{Proof.}
Since $\tfrac{\partial}{\partial s}w\perp \tfrac{\partial}{\partial t}w$, we have that
\[\langle\tfrac{\partial}{\partial s}w, \tfrac{\partial}{\partial t}w\rangle=0\]
and therefore
\[\langle\tfrac{\partial^2}{\partial s\partial t}w, \tfrac{\partial}{\partial t}w\rangle
+\langle\tfrac{\partial}{\partial s}w, \tfrac{\partial^2}{\partial t^2}w\rangle=\tfrac{\partial}{\partial t}\langle\tfrac{\partial}{\partial s}w, \tfrac{\partial}{\partial t}w\rangle=0.\]

Note that $|\gamma'_s(t)|=|\tfrac{\partial}{\partial t}w(s,t)|$ and therefore
\begin{align*}
\tfrac \partial {\partial s}|\gamma'_s(t)|&=\tfrac \partial {\partial s}\sqrt{\langle \tfrac{\partial}{\partial t}w(s,t),\tfrac{\partial}{\partial t}w(s,t)\rangle}=
\\
&=\frac{\langle \tfrac{\partial^2}{\partial s\partial t}w(s,t),\tfrac{\partial}{\partial t}w(s,t)\rangle}{\sqrt{\langle \tfrac{\partial}{\partial t}w(s,t),\tfrac{\partial}{\partial t}w(s,t)\rangle}}=
\\
&=-\frac{\langle\tfrac{\partial}{\partial s}w, \tfrac{\partial^2}{\partial t^2}w\rangle}{|\gamma'_s(t)|}=
\\
&=-\frac{\langle\tfrac{\partial}{\partial s}w, \gamma''_s(t)\rangle}{|\gamma'_s(t)|}.
\end{align*}

The values $\ell(s)$ do not change if we reparametrize $\gamma_s$,
so we can assume that for a fixed value $s$ the curve $\sigma_s$ is unit-speed.
Since $|\tfrac{\partial}{\partial s}w|=1$ and $\tfrac{\partial}{\partial s}w$ points to the right from $\tfrac{\partial}{\partial t}w=\gamma'_s(t)$, the last expression equals to $k_g(s,t)$,
where $k_g(s,t)$ denotes the geodesic curvature of $\sigma_s$ at $t$. 
Therefore, for this particular $s$ we have
\begin{align*}
\ell'(s)&= \int_a^b \tfrac \partial {\partial s} |\gamma'_s(t)|\cdot dt =
\\
&= \int_a^b k_g(s,t)\cdot dt=
\\
&=\Theta_{\sigma_s}.
\end{align*}
Since the left hand side and the right hands side of this formula do not depend on the parametrization of $\sigma_s$, this formula holds for all $s$.\footnote{One may avoid passing the a unit-speed parametrization by using the following formula for geodesic curvature which holds for any regular parametrization: 
\[k_g(t,s)=\langle \Norm(\sigma_s(t)),[\sigma'_s(t),\sigma''_s(t)]\rangle/|\sigma_s'(t)|^3;\]
it saves thinking but makes the calculations longer.}
\qeds

The parametrization of a surface satisfying the conditions in the proposition are called \index{semigeodesic coordinates}\emph{semigeodesic coordinates}.
The following exercise explains the reason for this name.

\begin{thm}{Exercise}\label{ex:geod-semigeod}
Assume $(s,t)\mapsto w(s,t)$ be a local parametrization of an oriented smooth regular surface $\Sigma$ as in the proposition above.
Show that for any fixed $t$ the curve $\gamma_t(s)= w(s,t)$ is a geodesic.
\end{thm}







\section{Local comparison}

The following proposition is a special case of the so a comparison theorem, proved by Harry Rauch \cite{rauch}.

\begin{thm}{Theorem}\label{thm:rauch}
Let $\Sigma$ be a smooth regular surface and $p\in\Sigma$.
Assume $\~\gamma\:[a,b]$ is a curve the tangent plane $\T_p\Sigma$ that runs in a sufficiently small neighborhood of the origin; 
consider the curve 
\[\gamma=\exp_p\circ\gamma\] in $\Sigma$.

\begin{enumerate}[(i)]
 \item If Gauss curvature of $\Sigma$ is nonnegative, then 
 \[\length \gamma\le \length \~\gamma\]
\item If Gauss curvature of $\Sigma$ is nonpositive, then 
 \[\length \gamma\ge \length \~\gamma.\]
\end{enumerate}
\end{thm}

The proof is a direct application of Corollary \ref{cor:w<w}.

\parit{Proof.}
Let us denote $\~w(\theta,r)$ and $w(\theta,r)$ the polar coordinates of $\T_p$ and $\Sigma$ at $p$
Recall that 
\begin{align*}
\tfrac{\partial \~w}{\partial\theta}&\perp \tfrac{\partial \~w}{\partial r};
&
|\tfrac{\partial \~w}{\partial r}|&=1;
\\
\tfrac{\partial w}{\partial\theta}&\perp \tfrac{\partial w}{\partial r};
&
|\tfrac{\partial w}{\partial r}|&=1;
\intertext{By Corollary \ref{cor:w<w}, we also have}
|\tfrac{\partial \~w}{\partial \theta}|&\ge |\tfrac{\partial \~w}{\partial \theta}|;
&
|\tfrac{\partial \~w}{\partial \theta}|&\le |\tfrac{\partial \~w}{\partial \theta}|;
\end{align*}
if Gauss curvature is nonnegative or nonpositive respectively.

It is sufficient to show that
\[|\gamma'(t)|\le |\~\gamma'(t)|\quad\text{or, correspondingly}\quad|\gamma'(t)|\ge |\~\gamma'(t)|\eqlbl{gamma<gamma}\]
for any $t$.

Note that both curves $\gamma(t)$ and $\~\gamma(t)$ described the same way in the polar coordinates;
denote these coordinates by $(\theta(t),r(t))$.
Then 
\begin{align*}
\gamma'(t)|^2&=|\tfrac{\partial w}{\partial\theta}\cdot\theta'(t)+\tfrac{\partial w}{\partial r}\cdot r'(t)|^2=
\\
&=|\tfrac{\partial w}{\partial\theta}|^2\cdot|\theta'(t)|^2+|r'(t)|^2
\intertext{The same way} 
\~\gamma'(t)|^2&=|\tfrac{\partial \~w}{\partial\theta}|^2\cdot|\theta'(t)|^2+|r'(t)|^2;
\end{align*}
hence \ref{gamma<gamma} follows.
\qeds



















\chapter{Semieodesic coordinates}

\section*{Semigeodesic map}

Let $(u,v)\mapsto s(u,v)$ be a smooth map from a rectangle $(a,b)\times(c,d)$ to a smooth surface $\Sigma$.
If the first coordinate lines $u\mapsto s(u,v)$ are unit-speed geodesics in $\Sigma$, then the map $s$ is called semigeodesic.
One might think that $v$ is the parameter in a family of unit-speed geodesics $\gamma_v:u\mapsto s(u,v)$.

\begin{thm}{Proposition}\label{prop:semigeodesic}
Let $(u,v)\mapsto s(u,v)$ be a semigedesic map to a smooth surface.
Then the value $\langle\tfrac{\partial s}{\partial u},\tfrac{\partial s}{\partial v}\rangle$ does not depend on $u$.

In particular,
if for some fixed $u_0$ the point $s(u_0,v)$ does not depend on $v$, 
then $\tfrac{\partial s}{\partial u}\perp \tfrac{\partial s}{\partial v}$ for all $u$ and $v$.

\end{thm}

\parit{Proof.}
Fix $v$; we need to show that 
\[\tfrac{\partial}{\partial u}\langle\tfrac{\partial s}{\partial u},\tfrac{\partial s}{\partial v}\rangle=0.\]

Since $u\to s(u,v)$ is a unit-speed curve we have $|\tfrac{\partial s}{\partial u}|=1$,
or equivalently,
\[\langle\tfrac{\partial s}{\partial u},\tfrac{\partial s}{\partial u}\rangle=1.\]
Therefore
\begin{align*}
0
&=\tfrac{\partial}{\partial v}\langle\tfrac{\partial s}{\partial u},\tfrac{\partial s}{\partial u}\rangle=
\\
&=2\cdot\langle\tfrac{\partial s}{\partial u},\tfrac{\partial^2 s}{\partial u\partial v}\rangle.
\end{align*}
In particular,
\[\langle\tfrac{\partial s}{\partial u},\tfrac{\partial^2 s}{\partial u\partial v}\rangle=0.\]

Further the equation for geodesic for $u\to s(u,v)$ implies that $\tfrac{\partial^2 s}{\partial u^2}(u,v)$ is perpendicular to the tangent plane $T_{s(u,v)}$.
Since $\tfrac{\partial s}{\partial u}(u,v)$ and $\tfrac{\partial s}{\partial v}(u,v)$ are tangent vectors at $s(u,v)$, we have the following two identities:
\begin{align*}
\langle\tfrac{\partial^2 s}{\partial u^2},\tfrac{\partial s}{\partial u}\rangle&=0,
\\
\langle\tfrac{\partial^2 s}{\partial u^2},\tfrac{\partial s}{\partial v}\rangle&=0.
\end{align*}

Therefore
\begin{align*}
\tfrac{\partial}{\partial u}\langle\tfrac{\partial s}{\partial u},\tfrac{\partial s}{\partial v}\rangle
&=\langle\tfrac{\partial^2 s}{\partial u^2},\tfrac{\partial s}{\partial v}\rangle
+
\langle\tfrac{\partial s}{\partial u},\tfrac{\partial^2 s}{\partial u\partial v}\rangle=
\\
&=0;
\end{align*}
that is, for fixed $v$, the value $\langle\tfrac{\partial s}{\partial u},\tfrac{\partial s}{\partial v}\rangle$ is constant.

If $v\mapsto s(u_0,v)$ is constant, then $\tfrac{\partial s}{\partial u}(u_0,v)=0$.
In particular $\langle\tfrac{\partial s}{\partial u},\tfrac{\partial s}{\partial v}\rangle(u_0,v)=0$ for any $v$.
Since $\langle\tfrac{\partial s}{\partial u},\tfrac{\partial s}{\partial v}\rangle$ does not depend on $u$, we have that $\langle\tfrac{\partial s}{\partial u},\tfrac{\partial s}{\partial v}\rangle(u,v)=0$, or equivalently
$\tfrac{\partial s}{\partial u}\perp \tfrac{\partial s}{\partial v}$ for any $u$ and~$v$.
\qeds

\section*{Polar coordinates}

Fix a point $p$ in a smooth oriented surface $\Sigma$ and a unit tangent vector $u\in\T_p$.
Let us construct an analog of polar coordinates in a neighborhood of $p$.

Fix a pair or real numbers $(\theta,\rho)$.
Denote by $w$ the counterclockwise rotation of $u$ by angle $\theta$ and let $\gamma_\theta$ be the unit-speed geodesic that runs from $p$ in the direction $w$.
Further set $s(\theta,\rho)=\gamma_\theta(\rho)$.

Note that the function $s\:\RR\times \RR\to \Sigma$ could be also defined as 
\[s(\theta,\rho)=\exp_p\tilde s(\theta,\rho),\]
where $\tilde s(\theta,\rho)$ denotes the vector in $\T_p$ with polar coordinates $(\theta,\rho)$, if one takes the axis in the direction of $u$.
By ???, it follows that there is $\eps>$ such that $s(\theta,\rho)$ is defined if $|\rho|<\eps$;
moreover, if $|\rho_i|<\eps$, then $s(\theta_1,\rho_1)=s(\theta_2,\rho_2)$ if and only if $\rho_1=\rho_2=0$ or $\theta_1=\theta_2+\pi\cdot n$ and $\rho_1=(-1)^n\cdot \rho_2$ for some integer $n$;
that is, in the strip described by the inequality $|\rho|<\eps$ the polar coordinates on $\Sigma$ behave as usual polar coordinates.

Consider the polar coordinates $(\theta,\rho)$ in the tangent plane $\T_p$.
Recall that the polar coordinates of a point are not uniquely defined;
first of all the origin has can be described by coordinates $(\theta,0)$ for any $\theta$ and also 
the coordinates $(\theta,r)$, $(\theta\pm\pi,-r)$, $(\theta\pm2\cdot \pi,r),\dots$ describe the same point for all real pairs $(\theta,r)$.

Let us use the exponential map to transfer the polar coordinates to the surface.
That is, a point has polar coordinates $(r,\theta)$ is 

\begin{thm}{Proposition}
For any point $p$ in a smooth surface $\Sigma$ there is a positive function $\eps$
such that any geodesic of length less than $\eps$ that starts at $p$ is a shortest path.  
\end{thm}

\parit{Proof.} Let $\eps$ be as in ???.
Consider a curve $\alpha$ with arc-length parametrization in $\Sigma$ that starts at $p$.
Assume $\length\alpha<\eps$.

Consider the polar coordinates $(\theta,\rho)\mapsto s(\theta,\rho)$ at $p$.
Note that the polar coordinates at $p$ satisfy the conditions in \ref{prop:semigeodesic}.
In particular
\[\tfrac{\partial s}{\partial\theta}\perp\tfrac{\partial s}{\partial\rho}.\eqlbl{eq:perp-partial}\]

Let us reqrite $\alpha$ in the polar coordinates 
\[\alpha(t)=s(\theta(t),\rho(t)).\]
Applying the chain rule, we get 
\[\alpha'=\tfrac{\partial s}{\partial\theta}\cdot\theta'+\tfrac{\partial s}{\partial\rho}\cdot\rho'.\]
By \ref{eq:perp-partial}, 
\[|\alpha'|\ge \tfrac{\partial s}{\partial\rho}\cdot\rho'\]
and the equality holds if and only if $\theta'=0$ and $\rho'\ge 0$.
Taking the integral, we get 
\[\length(\alpha|_{[0,t]}\ge\rho(t)\]
for any $t>0$ and the equality holds only if $\alpha$ always runs in a radial direction in the polar coordinates.

It follows that 
\[|p-s(\theta,\rho)|_\Sigma\ge |\rho|\]
for any point $s(\theta,\rho)$ in the ???.
On the other hand the geodesic $t\mapsto s(\theta,t)$ for $t\in[0,\rho]$ has length $\rho$ and therefore 
\[|p-s(\theta,\rho)|_\Sigma\le |\rho|\]
It follows that 































\chapter{tmp}





\section*{Signed area}
The formula \ref{eq:sphere-gauss-bonnet} holds modulo $2\cdot \pi$ for any closed broken geodesic, if one use \index{signed area}\emph{signed area} surrounded by curve instead of usual area;
that is, we count area of the regions taking into account how many times the curve goes around the region.

Namely, we have to choose a {}\emph{south pole} and state that its region has zero multiplicity.
When you cross the curve the mulitplicity changes by $\pm1$; we add 1 if the curve crosses your path from left to right and we subtract 1 otherwise.
The signed area surrounded by a closed curve is the sum of area of all the regions counted with multiplicities.

\begin{wrapfigure}{o}{32 mm}
\vskip-0mm
\centering
\includegraphics{mppics/pic-44}
\vskip-0mm
\end{wrapfigure}

Here is an example of a broken line with multiplicities assuming that the big region has the south pole inside.

This signed-area formula can be proved in a similar way:
Apply the formula for each triangle with vertex at the north pole and base at each edge of the broken geodesic.
Sum the resulting identities taking each with a sign: plus if the triangle lies on the left from the edge and minus if the triangle lies on the right from edge.

Choosing a different pole will change all the coefficients by the same number.
So the resulting formula holds only modulo the area of $\mathbb{S}^2$, which is $4\cdot \pi$ --- this will not destroy identity modulo $2\cdot\pi$.

Furthermore, by approximation, the signed-area formula holds for any reasonable curve, say piecewise smooth regular curves on the sphere.
Summarizing, we hope the discussion above convinced the reader that the following statement holds.

A domain $\Delta$ in a surface is called a \index{disc}\emph{disc} (or more precisely \index{topological disc}\emph{topological disc}) if it is bounded by a closed simple curve and can be parameterized by a unit plane disc 
\[\DD=\set{(x,y)\in\RR^2}{x^2+y^2\le1}.\]
That is, there is a continuous bijection $\DD\to\Delta$.

\begin{thm}{Proposition}\label{prop:spherical-gb}
For any closed piecewise smooth regular curve $\alpha$ on the sphere, 
we have that 
\[\tgc\alpha+\area\alpha=0 \pmod{2\cdot\pi},\]
where $\area\alpha$ denotes the signed area surrounded by $\alpha$ and $\tgc\alpha$ the total geodesic curvature of $\alpha$.

Moreover, if $\alpha$ is a simple curve that bounds a disc $\Delta$ on the left from it, then we have 
\[\tgc\alpha+\area\Delta=2\cdot\pi.\]

\end{thm}





\section*{Gauss--Bonnet formula}


\begin{thm}{Theorem}\label{thm:gb}
Let $\Delta$ be a disc in a smooth oriented surface $\Sigma$ bounded by a simple piecewise smooth and regular curve $\partial \Delta$ that is oriented in such a way that $\Delta$ lies on its left.
Then 
\[\tgc{\partial\Delta}+\iint_\Delta G=2\cdot \pi,\eqlbl{eq:g-b}\]
where $G$ denotes the Gauss curvature of $\Sigma$.
\end{thm}

For geodesic triangles this theorem was proved by Carl Friedrich Gauss \cite{gauss};
Pierre Bonnet and Jacques Binet independently generalized the statement for arbitrary curves. 
The modern formulation described below was given by Wilhelm Blaschke. 


\parit{Remarks; (1).}
For a general compact domain $\Delta$ (not necessary a disc) we have that
\[\tgc{\partial\Delta}+\iint_{\Delta} G=2\cdot  \pi\cdot\chi(\Delta),\eqlbl{eq:g-b-euler}\]
where $\chi(\Delta)$ is the so called \index{Euler's characteristic}\emph{Euler's characteristic} of $\Delta$.
The Euler's characteristic is \index{topological invariant}\emph{topological invariant}, in particular preserved in a continuous deformation.

If a surface $\Sigma$ (possibly with boundary) can be divided into $f$ discs by drawing $e$ edges connecting $v$ vertexes, then 
\[\chi(\Sigma)=v-e+f.\]
For example the disc $\DD$ has Euler's characteristic $1$; 
\begin{figure}[h!]
\vskip-0mm
\centering
\includegraphics{mppics/pic-49}
\vskip-0mm
\end{figure}
it can be divided into discs many ways, 
but each time we have $v-e=f=1$.
The latter agrees with \ref{eq:g-b} and \ref{eq:g-b-euler}.
It is useful to know that $\chi(\mathbb{S}^2)=2$; $\chi(\TT^2)=0$ where $\TT^2$ denotes torus; 
$\chi(S_g)=2-2\cdot g$, where $S_g$ is a surface of genus $g$;
that is, sphere with $g$ handles.

\parit{(2).} Note that if $\Sigma$ is a plane, then a geodesic in $\Sigma$ are formed by line segments.
In this case the statement of theorem follows from Exercise~\ref{ex:pm2pi}.

\parit{(3).} If $\Sigma$ is the unit sphere, then $G\equiv1$ and therefore formula \ref{eq:g-b} can be rewritten as 
\[\tgc{\partial\Delta}+\area\Delta=2\cdot \pi,\]
which follows from Proposition~\ref{prop:spherical-gb}.

\medskip

We will give an informal proof of \ref{thm:gb} based on the bike wheel interpretation described above.
We suppose that it is intuitively clear that moving the axis of the wheel without changing its direction does not change the direction of the wheel's spikes.

More precisely, assume we keep the axis of a non-spinning bike wheel and perform the following two experiments:
\begin{enumerate}[(i)]
\item We move it around and bring the axis back to the original position. 
As a result the wheel might rotate by some angle; let us measure this angle.

\item
We move the direction of the axis the same way as before without moving the center of the wheel.
After that we measure the angle of rotation.
\end{enumerate}
Then the resulting angle in these two experiments is the same. 

Consider a oriented smooth surface $\Sigma$ with the spherical; map $\Norm\:\Sigma\to \mathbb{S}^2$.
Note that for any point $p$ on $\Sigma$, the tangent plane $\T_p\Sigma$ is parallel to the tangent plane $\T_{\Norm(p)}\mathbb{S}^2$; so we can identify these tangent spaces.
From the experiments above, we get the following:

\begin{thm}{Lemma}\label{lem:spherical-image}
Suppose $\alpha$ is a piecewise smooth regular curve in a smooth regular surface $\Sigma$ which has a Gauss map $\Norm\:\Sigma\to \mathbb{S}^2$.
Then the parallel transport along $\alpha$ in $\Sigma$ coincides with the parallel transport along the curve $\beta=\Norm\circ\alpha$ in $\mathbb{S}^2$.
\end{thm}

\begin{thm}{Exercise}
Let $\Sigma$ be a smooth closed surface with positive Gauss curvature.
Given a line $\ell$ denote by $\omega_\ell$ the closed curve formed by points with tangent planes parallel to $\ell$.\footnote{Equivalently the normal vector at any point of $\omega_\ell$ is perpendicular to $\ell$. If the light falls on $\Sigma$ from one side parallel to $\ell$, then $\omega_\ell$ divides the bright and dark sides of~$\Sigma$.}
Show that parallel transport around $\omega_\ell$ is the identity map.
\end{thm}

Now we are ready to prove the theorem.

\parit{Proof of \ref{thm:gb}.}
Let $\alpha$ be the boundary $\partial\Delta$ parameterized in such a way that $\Delta$ lies on the left from it.
Assume $p$ is the point where $\alpha$ starts and ends.

Set $\beta=\Norm\circ\gamma$ and $q=\Norm(p)$, so the spherical curve $\beta$ starts and ends at $q$.

By Lemma \ref{lem:spherical-image} the parallel transport along $\alpha$ in $\Sigma$ coincides with the parallel transport along the curve $\beta$ in $\mathbb{S}^2$.
By Proposition~\ref{prop:pt+tgc}, it follows that 
\[\tgc{\alpha,\Sigma}=\tgc{\beta,\mathbb{S}^2} \pmod{2\cdot \pi}.\]

By Proposition~\ref{prop:spherical-gb},
\[\tgc{\beta,\mathbb{S}^2}+\area\beta=0\pmod{2\cdot \pi}.\]
Therefore 
\[\tgc{\alpha,\Sigma}+\area\beta=0\pmod{2\cdot \pi}.\]

Recall that the shape  operator $s_p\: T_p\Sigma \to \T_{\Norm(p)}\mathbb{S}^2=T_p\Sigma$ is the Jacobian of the Gauss map $\Norm\:\Sigma\to \mathbb{S}^2$ at the point $p$.
In appropriately chosen coordinates in $T_p$, the shape operator can be presented by a diagonal matrix 
$\left(\begin{smallmatrix}
k_1&0
\\
0&k_2
\end{smallmatrix}\right)$, where $k_1$ and $k_2$ are the principle curvatures at $p$.
Therefore, the determinant of $s_p$ is the Gauss curvature at~$p$.

If $\Sigma$ is a closed surface with positive Gauss curvature, then the Gauss map $\Norm\:\Sigma\to\mathbb{S}^2$ is a smooth bijection.
Therefore 
\[\iint_\Delta G=\area[\Norm(\Delta)].\]

In the general case we have to count the area $\Norm(\Delta)$ taking orientation and multiplicity of the Gauss map into account.
In this case 
\[\iint_\Delta G=\area\beta,\]
where $\area\beta$ is the signed area surrounded by $\beta$; it is defined above.
Therefore 
\[\tgc{\alpha,\Sigma}+\iint_\Delta G=0\pmod{2\cdot \pi}.\eqlbl{eq:gb(mod2pi)}\]

If $\Delta$ is a disc in the plane, then Gauss curvature vanishes and by Exercise~\ref{ex:pm2pi}, we have 
\[\tgc{\partial\Delta}+\iint_\Delta G=2\cdot \pi.\]
Assunme that $\Sigma_t$ is a smooth one parameter family of surfaces with 
a one parameter family of discs $\Delta_t\subset \Sigma_t$ and $\alpha_t$ is the boundary $\partial\Delta_t$ parameterized in such a way that $\Delta_t$ lies on the left from it.
The value 
\[f(t)=\tgc{\alpha_t}+\iint_\Delta G\]
is continuous in $t$ and by \ref{eq:gb(mod2pi)} it has to be constant.

If $\Sigma_0$ is a plane, then 
\[\tgc{\partial\Delta_0}+\iint_{\Delta_0} G=2\cdot \pi.\]
Intuitively it is clear that any disc can be obtained as a resultof continuous deformation of plane disc.
Therefore 
\[\tgc{\partial\Delta_1}+\iint_{\Delta_1} G=2\cdot \pi\]
for arbitrary disc $\Delta_1$; whence \ref{eq:g-b} follows.
\qeds






































\chapter{Gauss--Bonnet formula}







\parbf{Signed area.}
The formula \ref{eq:sphere-gauss-bonnet} holds modulo $2\cdot \pi$ for any closed broken geodesic, if one use \index{signed area}\emph{signed area} surrounded by curve instead of usual area;
that is, we count area of the regions taking into account how many times the curve goes around the region.

Namely, we have to choose a {}\emph{south pole} and state that its region has zero multiplicity.
When you cross the curve the mulitplicity changes by $\pm1$; we add 1 if the curve crosses your path from left to right and we subtract 1 otherwise.
The signed area surrounded by a closed curve is the sum of area of all the regions counted with multiplicities.

\begin{wrapfigure}{o}{32 mm}
\vskip-0mm
\centering
\includegraphics{mppics/pic-44}
\vskip-0mm
\end{wrapfigure}

Here is an example of a broken line with multiplicities assuming that the big region has the south pole inside.

This signed-area formula can be proved in a similar way:
Apply the formula for each triangle with vertex at the north pole and base at each edge of the broken geodesic.
Sum the resulting identities taking each with a sign: plus if the triangle lies on the left from the edge and minus if the triangle lies on the right from edge.

Choosing a different pole will change all the coefficients by the same number.
So the resulting formula holds only modulo the area of $\mathbb{S}^2$, which is $4\cdot \pi$ --- this will not destroy identity modulo $2\cdot\pi$.

Furthermore, by approximation, the signed-area formula holds for any reasonable curve, say piecewise smooth regular curves on the sphere.
Summarizing, we hope the discussion above convinced the reader that the following statement holds.

A domain $\Delta$ in a surface is called a \index{disc}\emph{disc} (or more precisely \index{topological disc}\emph{topological disc}) if it is bounded by a closed simple curve and can be parameterized by a unit plane disc 
\[\DD=\set{(x,y)\in\RR^2}{x^2+y^2\le1}.\]
That is there is a continuous bijection $\DD\to\Delta$.

\begin{thm}{Proposition}\label{prop:spherical-gb}
For any closed piecewise smooth regular curve $\alpha$ on the sphere, 
we have that 
\[\tgc\alpha+\area\alpha=0 \pmod{2\cdot\pi},\]
where $\area\alpha$ denotes the signed area surrounded by $\alpha$ and $\tgc\alpha$ the total geodesic curvature of $\alpha$.

Moreover, if $\alpha$ is a simple curve that bounds a disc $\Delta$ on the left from it, then we have 
\[\tgc\alpha+\area\Delta=2\cdot\pi.\]

\end{thm}





\section{Gauss--Bonnet formula}


\begin{thm}{Theorem}\label{thm:gb}
Let $\Delta$ be a topological disc in a smooth oriented surface $\Sigma$ bounded by a simple piecewise smooth and regular curve $\partial \Delta$ that is oriented in such a way that $\Delta$ lies on its left.
Then 
\[\tgc{\partial\Delta}+\iint_\Delta K=2\cdot \pi,\eqlbl{eq:g-b}\]
where $G$ denotes the Gauss curvature of $\Sigma$.
\end{thm}

For geodesic triangles this theorem was proved by Carl Friedrich Gauss \cite{gauss};
Pierre Bonnet and Jacques Binet independently generalized the statement for arbitrary curves. 
The modern formulation described below was given by Wilhelm Blaschke. 


\parit{Remarks; (1).}
For a general compact domain $\Delta$ (not necessary a disc) we have that
\[\tgc{\partial\Delta}+\iint_{\Delta} K=2\cdot  \pi\cdot\chi(\Delta),\eqlbl{eq:g-b-euler}\]
where $\chi(\Delta)$ is the so called \index{Euler's characteristic}\emph{Euler's characteristic} of $\Delta$.
The Euler's characteristic is \index{topological invariant}\emph{topological invariant}, in particular preserved in a continuous deformation.

If a surface $\Sigma$ (possibly with boundary) can be divided into $f$ discs by drawing $e$ edges connecting $v$ vertexes, then 
\[\chi(\Sigma)=v-e+f.\]
For example the disc $\DD$ has Euler's characteristic $1$; 
\begin{figure}[h!]
\vskip-0mm
\centering
\includegraphics{mppics/pic-49}
\vskip-0mm
\end{figure}
it can be divided into discs many ways, 
but each time we have $v-e=f=1$.
The latter agrees with \ref{eq:g-b} and \ref{eq:g-b-euler}.
It is useful to know that $\chi(\mathbb{S}^2)=2$; $\chi(\TT^2)=0$ where $\TT^2$ denotes torus; 
$\chi( S_g)=2-2\cdot g$, where $S_g$ is a surface of genus $g$; that is, sphere with $g$ handles.

\parit{(2).} Note that if $\Sigma$ is a plane, then a geodesic in $\Sigma$ are formed by line segments.
In this case the statement of theorem follows from Exercise~\ref{ex:pm2pi}.

\parit{(3).} If $\Sigma$ is the unit sphere, then $K\equiv1$. Therefore the formula \ref{eq:g-b} can be rewritten as 
\[\tgc{\partial\Delta}+\area\Delta=2\cdot \pi,\]
which follows from Proposition~\ref{prop:spherical-gb}.

\medskip

We will give an informal proof of \ref{thm:gb} based on the bike wheel interpretation described above.
We suppose that it is intuitively clear that moving the axis of the wheel without changing its direction does not change the direction of the wheel's spikes.

More precisely, assume we keep the axis of a non-spinning bike wheel and perform the following two experiments:
\begin{enumerate}[(i)]
\item We move it around and bring the axis back to the original position. 
As a result the wheel might rotate by some angle; let us measure this angle.

\item
We move the direction of the axis the same way as before without moving the center of the wheel.
After that we measure the angle of rotation.
\end{enumerate}
Then the resulting angle in these two experiments is the same. 

Consider a surface $\Sigma$ with a Gauss map $\nu\:\Sigma\to \mathbb{S}^2$.
Note that for any point $p$ on $\Sigma$, the tangent plane $\T_p\Sigma$ is parallel to the tangent plane $\T_{\nu(p)}\mathbb{S}^2$; so we can identify these tangent spaces.
From the experiments above, we get the following:

\begin{thm}{Lemma}\label{lem:spherical-image}
Suppose $\alpha$ is a piecewise smooth regular curve in a smooth regular surface $\Sigma$ which has a Gauss map $\nu\:\Sigma\to \mathbb{S}^2$.
Then the parallel transport along $\alpha$ in $\Sigma$ coincides with the parallel transport along the curve $\beta=\nu\circ\alpha$ in $\mathbb{S}^2$.
\end{thm}

\begin{thm}{Exercise}
Let $\Sigma$ be a smooth closed surface with positive Gauss curvature.
Given a line $\ell$ denote by $\omega_\ell$ the closed curve formed by points with tangent planes parallel to $\ell$.\footnote{Equivalently the normal vector at any point of $\omega_\ell$ is perpendicular to $\ell$. If the light falls on $\Sigma$ from one side parallel to $\ell$, then $\omega_\ell$ divides the bright and dark sides of~$\Sigma$.}
Show that parallel transport around $\omega_\ell$ is the identity map.
\end{thm}

Now we are ready to prove the theorem.

\parit{Proof of \ref{thm:gb}.}
Let $\alpha$ be the boundary $\partial\Delta$ parameterized in such a way that $\Delta$ lies on the left from it.
Assume $p$ is the point where $\alpha$ starts and ends.

Set $\beta=\nu\circ\gamma$ and $q=\nu(p)$, so the spherical curve $\beta$ starts and ends at $q$.

By Lemma \ref{lem:spherical-image} the parallel transport along $\alpha$ in $\Sigma$ coincides with the parallel transport along the curve $\beta$ in $\mathbb{S}^2$.
By Proposition~\ref{prop:pt+tgc}, it follows that 
\[\tgc{\alpha,\Sigma}=\tgc{\beta,\mathbb{S}^2} \pmod{2\cdot \pi}.\]

By Proposition~\ref{prop:spherical-gb},
\[\tgc{\beta,\mathbb{S}^2}+\area\beta=0\pmod{2\cdot \pi}.\]
Therefore 
\[\tgc{\alpha,\Sigma}+\area\beta=0\pmod{2\cdot \pi}.\]

Recall that the shape  operator $s_p\: T_p\Sigma \to \T_{\nu(p)}\mathbb{S}^2=T_p\Sigma$ is the Jacobian of the Gauss map $\nu\:\Sigma\to \mathbb{S}^2$ at the point $p$.
In appropriately chosen coordinates in $T_p$, the shape operator can be presented by a diagonal matrix 
$\left(\begin{smallmatrix}
k_1&0
\\
0&k_2
\end{smallmatrix}\right)$, where $k_1$ and $k_2$ are the principle curvatures at $p$.
Therefore, the determinant of $s_p$ is the Gauss curvature at~$p$.

If $\Sigma$ is a closed surface with positive Gauss curvature, then the Gauss map $\nu\:\Sigma\to\mathbb{S}^2$ is a smooth bijection.
Therefore 
\[\iint_\Delta K=\area[\nu(\Delta)].\]

In the general case we have to count the area $\nu(\Delta)$ taking orientation and multiplicity of the Gauss map into account.
In this case 
\[\iint_\Delta K=\area\beta,\]
where $\area\beta$ is the signed area surrounded by $\beta$; it is defined above.
Therefore 
\[\tgc{\alpha,\Sigma}+\iint_\Delta K=0\pmod{2\cdot \pi}.\eqlbl{eq:gb(mod2pi)}\]

If $\Delta$ is a disc in the plane, then Gauss curvature vanishes and by Exercise~\ref{ex:pm2pi}, we have 
\[\tgc{\partial\Delta}+\iint_\Delta K=2\cdot \pi.\]
Assunme that $\Sigma_t$ is a smooth one parameter family of surfaces with 
a one parameter family of discs $\Delta_t\subset \Sigma_t$ and $\alpha_t$ is the boundary $\partial\Delta_t$ parameterized in such a way that $\Delta_t$ lies on the left from it.
The value 
\[f(t)=\tgc{\alpha_t}+\iint_\Delta K\]
is continuous in $t$ and by \ref{eq:gb(mod2pi)} it has to be constant.

If $\Sigma_0$ is a plane, then 
\[\tgc{\partial\Delta_0}+\iint_{\Delta_0} K=2\cdot \pi.\]
Intuitively it is clear that any disc can be obtained as a resultof continuous deformation of plane disc.
Therefore 
\[\tgc{\partial\Delta_1}+\iint_{\Delta_1} K=2\cdot \pi\]
for arbitrary disc $\Delta_1$; whence \ref{eq:g-b} follows.
\qeds





\begin{thm}{Exercise}
 Assume $\gamma$ is a closed simple curve with constant geodesic curvature $1$ in a smooth convex closed surface $\Sigma$.
 Show that 
 \[\length\gamma\le 2\cdot\pi;\]
that is, the length of $\gamma$ can not exceed the length of the unit circle in the plane.  
\end{thm}


\begin{thm}{Exercise}
Let $\gamma$ be a closed simple geodesic on a smooth convex closed surface $\Sigma$.
Assume $\nu\:\Sigma\to\mathbb{S}^2$ is a Gauss map.
Show that the curve $\alpha=\nu\circ\gamma$ divides the sphere into regions of equal area.

Conclude that
\[\length \alpha\ge 2\cdot\pi.\]
\end{thm}

{

\begin{wrapfigure}{o}{32 mm}
\vskip-0mm
\centering
\includegraphics{mppics/pic-46}
\vskip-0mm
\end{wrapfigure}

\begin{thm}{Exercise}
Let $\Sigma$ be a smooth closed surface with a closed geodesic $\gamma$.
Assume $\gamma$ has exactly 4 self-intersection at the points $a$, $b$, $c$ and $d$ that appear on $\gamma$ in the order $a,a,b,b,c,c,d,d$.
Show that $\Sigma$ can not have positive Gauss curvature.\footnote{Hint: estimate integral of Gauss curvature bounded by a simple geodesic loop.}
\end{thm}

\begin{thm}{Advanced exercise}
Let $\Sigma$ be a smooth regular sphere with positive Gauss curvature and $p\in\Sigma$. 
Suppose $\gamma$ be a closed geodesic that does not pass thru $p$.
Assume $\Sigma\backslash\{p\}$ parametrized by the plane.
Can it happen that in this parametrization,  $\gamma$ looks like one of the curves on the diagram?
\begin{figure}[h!]
\vskip-0mm
\centering
\includegraphics{mppics/pic-47}
\vskip-0mm
\end{figure}
Say as much as possible about possible/impossible diagrams of that type.
\end{thm}

}

\section{The remarkable theorem}

Let $\Sigma_1$ and $\Sigma_2$ be two smooth regular surfaces in the Euclidean space.
A map $f\:\Sigma_1\to \Sigma_2$ is called  length-preserving if for any curve $\gamma_1$ in $\Sigma_1$ the curve $\gamma_2=f\circ\gamma_1$ in $\Sigma_2$ has the same length. %???it is sufficient to consider smooth only curves???
If in addition $f$ is smooth and bijective, then it is called \index{intrinsic isometry}\emph{intrinsic isometry}. 

A simple example of intrinsic isometry can obtained by warping a plane into a cylinder.
The following exercise produce slightly more interesting example.

\begin{thm}{Exercise}
Suppose $\gamma(t)=(x(t),y(t))$ is a smooth unit-curve in the plane such that $y(t)=a\cdot \cos t$.
Let $\Sigma_\gamma$ be the surface of revolution of $\gamma$ around the $x$-axis.
Show that a small open domain in $\Sigma_\gamma$ admits a smooth length-pereserving map to the unit sphere.

Conclude that any round disc $\Delta$ in $\mathbb{S}^2$ of intrinsic radius smaller than $\tfrac\pi2$ admits a smooth length-preserving deformation; that is, there is one parameter family of surfaces with boundary $\Delta_t$, such that $\Delta_0=\Delta$ and $\Delta_t$ is not congruent to $\Delta_0$ for any $t\ne0$.\footnote{In fact any disc in $\mathbb{S}^2$ of intrinsic radius smaller than $\pi$ admits a smooth length preserving deformation. %???REF
}
\end{thm}


\begin{thm}{Theorem}\label{thm:remarkable}
Suppose $f\:\Sigma_1\to \Sigma_2$ is an intrinsic isometry between two smooth regular surfaces in  the Euclidean space; $p_1\in \Sigma_1$ and $p_2=f(p_1)\in \Sigma_1$.
Then 
\[K(p_1)_{\Sigma_1}=K(p_2)_{\Sigma_2};\]
that is, the Gauss curvature of $\Sigma_1$ at $p_1$ is the same as the Gauss curvature of $\Sigma_2$ at $p_2$.
\end{thm}

This theorem was proved by Carl Friedrich Gauss \cite{gauss} who called it \index{Remarkable theorem}\emph{Remarkable theorem} (\index{Theorema Egregium}\emph{Theorema Egregium}).
The theorem is indeed remarkable because the Gauss curvature is defined as a product of principle curvatures which might be different at these points; however, according to the theorem, their product can not change.

In fact Gauss curvature of the surface at the given point can be found {}\emph{intrinsically},
by measuring the lengths of curves in the surface.
For example, Gauss curvature $K(p)$ in the following formula for the circumference $c(r)$ of a geodesic circle centered at $p$ in a surface: 
\[c(r)=2\cdot\pi\cdot r-\tfrac\pi3\cdot K(p)\cdot r^3+o(r^3).\]

Note that the theorem implies there is no smooth length-preserving map that sends an open region in the unit sphere to the plane.%
\footnote{There are plenty of non-smooth length-preserving maps from the sphere to the plane; see \cite{petrunin-yashinski} and the references there in.}
It follows since the Gauss curvature of the plane is zero and the unit sphere has Gauss curvature 1. 
In other words, there is no map of a region on Earth without distortion.

\parit{Proof.}
Set $g_1=K(p_1)_{\Sigma_1}$ and $g_2=K(p_2)_{\Sigma_2}$;
we need to show that 
\[g_1=g_2.\eqlbl{eq:g=g}\]

Suppose $\Delta_1$ is a small geodesic triangle in $\Sigma_1$ that contains $p_1$.
Set $\Delta_2=f(\Delta_1)$.
We may assume that the Gauss curvature is almost constant in $\Delta_1$ and $\Delta_2$;
that is, given $\eps>0$, we can assume that 
\[
\begin{aligned}
|K(x_1)_{\Sigma_1}-g_1|&<\eps,
\\
|K(x_2)_{\Sigma_2}-g_2|&<\eps
\end{aligned}
\eqlbl{eq:almost=}\]
for any $x_1\in \Delta_1$ and $x_2\in \Delta_2$.

Since $f$ is length-preserving the triangles $\Delta_2$ is geodesic and
\[\area\Delta_1=\area\Delta_2.\eqlbl{eq:area=}\]
Moreover, triangles $\Delta_1$ and $\Delta_2$ have the same corresponding angles; denote them by $\alpha$, $\beta$ and $\gamma$.

By Gauss--Bonnet formula, we get that 
\[\iint_{\Delta_1}K_{\Sigma_1}=\alpha+\beta+\gamma-\pi=\iint_{\Delta_2}K_{\Sigma_2}.
\eqlbl{eq:gauss-int=}\]

By \ref{eq:almost=}, 
\begin{align*}
\left|g_1-\frac1{\area\Delta_1}\cdot\iint_{\Delta_1}K_{\Sigma_1}\right|&<\eps,
\\
\left|g_2-\frac1{\area\Delta_2}\cdot\iint_{\Delta_2}K_{\Sigma_2}\right|&<\eps.
\end{align*}
By \ref{eq:area=} and \ref{eq:gauss-int=},
\[\frac1{\area\Delta_1}\cdot\iint_{\Delta_1}K_{\Sigma_1}
=
\frac1{\area\Delta_2}\cdot\iint_{\Delta_2}K_{\Sigma_2},\]
therefore
\[|g_1-g_2|<2\cdot\eps.\]
Since $\eps>0$ is arbitrary, \ref{eq:g=g} follows.
\qeds


\section{Simple geodesic}

The following theorem provides an interesting application of Gauss--Bonnet formula;
it is proved by Stephan Cohn-Vossen \cite[Satz 9 in][]{convossen}.


\begin{thm}{Theroem}\label{thm:cohn-vossen}
Any open smooth regular surface with positive Gauss curvature has a simple two-sided infinite geodesic.
\end{thm}

\begin{thm}{Lemma}\label{lem:graph}
Suppose $\Sigma$ is an open surface in with positive Gauss curvature in the Euclidean space.
Then there is a convex function $f$ defined on a convex open region of $(x,y)$-plane 
such that $\Sigma$ can be presented as a graph $z=f(x,y)$ in some $(x,y,z)$-coordinate system of the Euclidean space.

Moreover 
\[\iint_\Sigma K\le 2\cdot\pi.\eqlbl{eq:int=<2pi}\]

\end{thm}

\parit{Proof.}
The surface $\Sigma$ is a boundary of an unbounded closed convex set $K$.

Fix $p\in \Sigma$ and consider a sequence of points $x_n$ such that $|x_n-p|\z\to \infty$ as $n\to \infty$.
Set $u_n=\tfrac{x_n-p}{|x_n-p|}$; the unit vector in the direction from $p$ to $x_n$.
Since the unit sphere is compact, we can pass to a subsequence of $(x_n)$ such that $u_n$ converges to a unit vector $u$.

Note that for any $q\in \Sigma$, the directions $v_n=\tfrac{x_n-q}{|x_n-q|}$ converge to $u$ as well.
The half-line from $q$ in the direction of $u$ lies in $K$.
Indeed any point on the half-line is a limit of points on the line segments $[q,x_n]$;
since $K$ is closed, all of these poins lie in $K$.


Let us choose the $z$-axis in the direction of $u$.
Note that line segments can not lie in $\Sigma$, otherwise its Gauss curvature would vanish.
It follows that any vertical line can intersect $\Sigma$ at most at one point.
That is, $\Sigma$ is a graph of a function $z=f(x,y)$.
Since $K$ is convex, the function $f$ is convex and it is defined in a region $\Omega$ which is convex.
The domain $\Omega$ is the projection of $\Sigma$ to the $(x,y)$-plane.
This projection is injective and by the inverse function theorem, it maps open sets in $\Sigma$ to open sets in the plane;
hence $\Omega$ is open.

It follows that the outer normal vectors to $\Sigma$ at any point, points to the south hemisphere $\mathbb{S}^2_-=\set{(x,y,z)\in\mathbb{S}^2}{z< 0}$.
Therefore the area of the spherical image of $\Sigma$ is at most $\area\mathbb{S}^2_-= 2\cdot\pi$.
The area of this image is the integral of the Gauss curvature along $\Sigma$.
That is,
\begin{align*}
\iint_{\Sigma}K&=\area[\nu(\Sigma)]\le 
\\
&\le \area\mathbb{S}^2_-=
\\
&=2\cdot\pi,
\end{align*}
where $\nu(p)$ denotes the outer unit normal vector at $p$.
Hence \ref{eq:int=<2pi} follows.
\qeds

\parit{Proof of \ref{thm:cohn-vossen}.}
Let $\Sigma$ be an open surface in with positive Gauss curvature and $\gamma$ a two-sided infinite geodesic in $\Sigma$.
The following is the key statement in the proof.

\begin{thm}{Claim}
The geodesic $\gamma$ contains at most one simple loop.
\end{thm}

Assume $\gamma$ has a simple loop $\ell$.
By Lemma \ref{lem:graph}, $\Sigma$ is parameterized by a open convex region $\Omega$ in the plane;
therefore $\ell$ bounds a disc in $\Sigma$; denote it by $\Delta$.
If $\phi$ is the angle at the base of the loop, then by Gauss--Bonnet,
\[\iint_\Delta K=\pi+\phi.\] 
By Lemma \ref{lem:graph}, $\phi<\pi$; that is $\gamma$ has no concave simple loops 

Assume $\gamma$ has two simple loops, say $\ell_1$ and $\ell_2$ that bound discs $\Delta_1$ and $\Delta_2$.
Then the disks $\Delta_1$ and $\Delta_2$ have to overlap,
otherwise the curvature of $\Sigma$ would exceed $2\cdot\pi$.

We may assume that $\Delta_1\not\subset \Delta_2$; the loop $\ell_2$ appears after $\ell_1$ on $\gamma$ and there are no other simple loops between them.
In this case, after going around $\ell_1$ and before closing $\ell_2$, the curve $\gamma$ must enter $\Delta_1$ creating a concave loop.
The latter contradicts the above observation.

If a geodesic $\gamma$ has a self-intersection,
then it contains a simple loop.
From above, there is only one such loop;
it cuts a disk from $\Sigma$ 
and goes around it either clockwise or counterclockwise.
This way we divide all the self-intersecting geodesics 
into two sets which we will call {}\emph{clockwise} and {}\emph{counterclockwise}.

Note that the geodesic $t\mapsto \gamma(t)$ is clockwise 
if and only if the same geodesic traveled backwards
$t\mapsto \gamma(-t)$
is counterclockwise.
By shooting unit-speed geodesics in all directions at a given point $p=\gamma(0)$,
we get a one parameter family of geodesics $\gamma_s$ for $s\in[0,\pi]$ connecting the geodesic $t\mapsto \gamma(t)$ with
the $t\mapsto \gamma(-t)$; that is, $\gamma_0(t)\z=\gamma(t)$ and $\gamma_\pi(t)=\gamma(-t)$. 
It follows that there are geodesics 
which aren't clockwise nor counterclockwise.
Those geodesics have no self-intersections.\qeds

\begin{thm}{Exercise}
Suppose that $f\:\RR^2\to\RR$ is a $\sqrt{3}$-Lipshitz smooth convex function.
Show that any geodesic in the surface defined by the graph $z=f(x,y)$ has no self-intersections.
\end{thm}








































\section*{The remarkable theorem}

Let $\Sigma_1$ and $\Sigma_2$ be two smooth regular surfaces in the Euclidean space.
A map $f\:\Sigma_1\to \Sigma_2$ is called  length-preserving if for any curve $\gamma_1$ in $\Sigma_1$ the curve $\gamma_2=f\circ\gamma_1$ in $\Sigma_2$ has the same length. %???it is sufficient to consider smooth only curves???
If in addition $f$ is smooth and bijective, then it is called \emph{intrinsic isometry}. 

A simple example of intrinsic isometry can obtained by warping a plane into a cylinder.
The following exercise produce slightly more interesting example.

\begin{thm}{Exercise}\label{ex:deformation}
Suppose $\gamma(t)=(x(t),y(t))$ is a smooth unit-speed curve in the plane such that $y(t)=a\cdot \cos t$.
Let $\Sigma_\gamma$ be the surface of revolution of $\gamma$ around the $x$-axis.
Show that a small open domain in $\Sigma_\gamma$ admits a smooth length-pereserving map to the unit sphere.

Conclude that any round disc $\Delta$ in $\mathbb{S}^2$ of intrinsic radius smaller than $\tfrac\pi2$ admits a smooth length-preserving deformation; that is, there is one parameter family of surfaces with boundary $\Delta_t$, such that $\Delta_0=\Delta$ and $\Delta_t$ is not congruent to $\Delta_0$ for any $t\ne0$.\footnote{In fact any disc in $\mathbb{S}^2$ of intrinsic radius smaller than $\pi$ admits a smooth length preserving deformation. %???REF
}
\end{thm}


\begin{thm}{Theorem}\label{thm:remarkable}
Suppose $f\:\Sigma_1\to \Sigma_2$ is an intrinsic isometry between two smooth regular surfaces in  the Euclidean space; $p_1\in \Sigma_1$ and $p_2=f(p_1)\in \Sigma_1$.
Then 
\[G(p_1)_{\Sigma_1}=G(p_2)_{\Sigma_2};\]
that is, the Gauss curvature of $\Sigma_1$ at $p_1$ is the same as the Gauss curvature of $\Sigma_2$ at $p_2$.
\end{thm}

This theorem was proved by Carl Friedrich Gauss \cite{gauss} who called it \emph{Remarkable theorem} (Theorema Egregium).
The theorem is indeed remarkable because the Gauss curvature is defined as a product of principle curvatures which might be different at these points; however, according to the theorem, their product can not change.

In fact Gauss curvature of the surface at the given point can be found \emph{intrinsically},
by measuring the lengths of curves in the surface.
For example, Gauss curvature $G(p)$ in the following formula for the circumference $c(r)$ of a geodesic circle centered at $p$ in a surface: 
\[c(r)=2\cdot\pi\cdot r-\tfrac\pi3\cdot G(p)\cdot r^3+o(r^3).\]

Note that the theorem implies there is no smooth length-preserving map that sends an open region in the unit sphere to the plane.%
\footnote{There are plenty of non-smooth length-preserving maps from the sphere to the plane; see \cite{petrunin-yashinski} and the references there in.}
It follows since the Gauss curvature of the plane is zero and the unit sphere has Gauss curvature 1. 
In other words, there is no map of a region on Earth without distortion.

\parit{Proof.}
Set $g_1=G(p_1)_{\Sigma_1}$ and $g_2=G(p_2)_{\Sigma_2}$;
we need to show that 
\[g_1=g_2.\eqlbl{eq:g=g}\]

Suppose $\Delta_1$ is a small geodesic triangle in $\Sigma_1$ that contains $p_1$.
Set $\Delta_2=f(\Delta_1)$.
We may assume that the Gauss curvature is almost constant in $\Delta_1$ and $\Delta_2$;
that is, given $\eps>0$, we can assume that 
\[
\begin{aligned}
|G(x_1)_{\Sigma_1}-g_1|&<\eps,
\\
|G(x_2)_{\Sigma_2}-g_2|&<\eps
\end{aligned}
\eqlbl{eq:almost=}\]
for any $x_1\in \Delta_1$ and $x_2\in \Delta_2$.

Since $f$ is length-preserving the triangles $\Delta_2$ is geodesic and
\[\area\Delta_1=\area\Delta_2.\eqlbl{eq:area=}\]
Moreover, triangles $\Delta_1$ and $\Delta_2$ have the same corresponding angles; denote them by $\alpha$, $\beta$ and $\gamma$.

By Gauss--Bonnet formula, we get that 
\[\iint_{\Delta_1}G_{\Sigma_1}=\alpha+\beta+\gamma-\pi=\iint_{\Delta_2}G_{\Sigma_2}.\eqlbl{eq:gauss-int=}\]

By \ref{eq:almost=}, 
\begin{align*}
\left|g_1-\frac1{\area\Delta_1}\cdot\iint_{\Delta_1}G_{\Sigma_1}\right|&<\eps,
\\
\left|g_2-\frac1{\area\Delta_2}\cdot\iint_{\Delta_2}G_{\Sigma_2}\right|&<\eps.
\end{align*}
By \ref{eq:area=} and \ref{eq:gauss-int=},
\[\frac1{\area\Delta_1}\cdot\iint_{\Delta_1}G_{\Sigma_1}
=
\frac1{\area\Delta_2}\cdot\iint_{\Delta_2}G_{\Sigma_2},\]
therefore
\[|g_1-g_2|<2\cdot\eps.\]
Since $\eps>0$ is arbitrary, \ref{eq:g=g} follows.
\qeds















Note that there is a smooth bijection $\iota$ between the cylinder $z=x^2$ and the plane $z=0$ that preserves the lengths of all curves; in other words the cylinder can be \emph{unfolded} on the plane.
In particular $\iota$ maps shortest paths in the cylinder to shortest paths in the plane.
Therefore, according to the theorem, such a bijection sends geodesics in the cylinder to geodesics on the plane and the other way around. 
However a geodesic on the cylinder might have nonvanishing second derivative while geodesics on the plane are straight lines with vanishing second derivative.
















The proof of the theorem is built on the following lemmas which are interesting on their own.

\begin{thm}{Lemma}
Let $D$ be a closed domain in the $(x,y)$-plane bounded by a closed piecewise smooth simple curve $\gamma$,
and $f$ a smooth function defined on $D$.
Let $\Delta$ be the graph $z=f(x,y)$ and $\Delta'$ be another piecewise smooth surface that has the same boundary as $\Delta$ and lies in the cylinder $D\times \RR$.
Then
\[\area\Delta'\ge\area \Delta.\]
Moreover in case of equality, we have $\Delta'=\Delta$.
\end{thm}

\parit{Proof.}
Note that $\Delta'$ is oriented.
Indeed, by ???, the complement of $\Delta'$ in $D\times \RR$ has two components (lower and upper), so we can equip $\Delta'$ with a unit normal vector field $\Norm'$ that points in upper component. 

Let us denote by $\Norm(x,y)$ the unit normal vector at the point $(x,y,f(x,y))\in\Delta$ that points up;
that is, its $z$-coordinate is positive at each point.
Consider the vector field $\vec u$ on the domain $\RR\times D$ defined by 
$\vec u(x,y,z)\z=\Norm (x,y)$.
By construction $|\vec u|\equiv 1$ and according to Lemma~\ref{lem:div+H}, $\div \vec u\equiv -H\equiv 0$.
Therefore $\vec u$ is a calibrating $\Delta$ in $\RR\times D$.
In particular
\[\begin{aligned}
\area \Delta&=\flux_{\vec u}\Delta,
\\
\area \Delta'&\ge\flux_{\vec u}\Delta'.
\end{aligned}
\eqlbl{area=>=flux}\]

Observe that $\RR\times D$ is contactable???.
By ??? it follows that there is a vector field $\vec v$ such that $\curl \vec v=\vec u$.
Since $\Delta$ and $\Delta'$ share the boundary, the curl theorem (???), implies that 
\[\flux_{\vec u}\Delta=\flux_{\vec u}\Delta'.\]
Finally observe that the last equality and \ref{area=>=flux} imply the main statement in the lemma.

Recall that
\[\flux_{\vec u}\Delta'\df\iint_{\Delta'}\langle\vec u,\Norm'\rangle.\]
Therefore if the equality the second line of \ref{area=>=flux}, then normal $\vec u=\Norm'$ everywhere on $\Delta'$.
In particular $\Delta'$ is a graph of 
\qeds

If the both surfaces $\Delta$ and $\Delta'$ would be oriented,
then the statement can be proved the same way as previous lemma.
However, it might be not the case.
While reading the proof it would be useful to keep the following example in mind.

\begin{thm}{Exercise}
Construct a smooth embedded M\"obius strip with unit circle as a boundary. 
\end{thm}

\begin{thm}{Lemma}
Let $f$ be a function defined on a open domain $\Omega$???
Suppose that $\Delta$ is a surface in the graph $z=f(x,y)$ bounded by a smooth closed curve $\gamma$ and $\Delta'$ be another surface whtih the same boundary.
Then
\[\area\Delta'\ge\area \Delta.\]
Moreover in case of equality, we have $\Delta'=\Delta$.
\end{thm}















\section*{Monotonicity}

\begin{thm}{Proposition}
Let $\Sigma$ be a area-minimizing surface with boundary curve $\gamma$.
Suppose $p\in \Sigma$ and $R$ is the distance from $p$ to $\gamma$.
Given 
\[f\:r\mapsto \frac{\area \Sigma_r}{r^2}\]
is a nondecreasing function for $r\in (0,R]$;
here $\Sigma_r=\Sigma\cap \cap B(p,r)$.
\end{thm}

\parit{Proof.}
Without loss of generality we may assume that $p$ is the origin.

According to ???, for almost all $r$, the intersection of the $r$-sphere centered at $p$ with $\Sigma$ is a collection of smooth curves.
Denote by $\ell(r)$ the total length of this collection.

Fix small $\eps>0$.
Let us modify the surface $\Sigma$ in the ball $B(p,r)$ the following way.
Consider the rescaled copy of the intersection $\Sigma_r$ with coefficient $1-\eps$ and a conic collar 
formed by the line segments from $(1-\eps)\cdot x$ to $x$ for all points $x\in\gamma$.
Note that the obtained piecewise smooth surface $\Sigma'$ has boundary $\gamma$, the same as the boundary of $\Sigma$.

Denote by $a(r)$ the area of $\Sigma_r$.
Note the rescaled copy of $\Sigma_r$ has area $(1-\eps)^2\cdot a(r)$
and the area of collar is $(1-\tfrac\eps2)^2\cdot\ell\cdot r$.

Since $\Sigma$ is area-minimizing, we have
\[\area\Sigma'\ge \area \Sigma.\]
The latter is equivalent to 
\[(1-\eps)^2\cdot a+(1-\tfrac\eps2)^2\cdot\ell\cdot r\ge a.\]
Applying this inequality for $\eps\to 0+$, we get that
\[\ell(r)\cdot r\ge 2\cdot a(r).\]

According to the coarea formula, 
\[a'(r)\ge \ell(r)\]
if the left and right hand sides are defined at $r$.
Whence 
\[a'(r)\cdot r\ge 2\cdot a(r).\]




















Let $f$ be a function on $\RR^3$ and $c$ be a real constant.
The set $L_c=\set{(x,y,z)\in\RR^3}{f(x,y,z)=c}$ is called \emph{level set} of $f$.
Recall that if the gradient of $f$ is does not vanish on $L_c$, then $L_c$ is a smooth surface.

Let $\vec v$ be a vector field defined on a set $A\subset \RR^3$.
A vector field $\vec w$ defined on a bigger set $B\supset A$
is called \emph{extension} of $\vec v$ if $\vec w=\vec v$ at any point in $A$.
If in addition $\vec v$ is a smooth vector field, than it is called \emph{smooth extension}. 


\begin{thm}{Lemma}
Suppose a smooth surface $\Sigma$ bounds a convex region $K$ in $\RR^3$.
Denote by $W$ the closed outer region of $\Sigma$; that is $W$ contains $\Sigma$ and all points in the complement of $K$.  
Then there is a unit vector field $\vec u$ defined on $W$ such that 
$\div \vec u\ge 0$.
\end{thm}

\parit{Proof.}
Given a point 
\qeds






















It remains to show that $\tfrac{\partial}{\partial r}\langle\tfrac{\partial w_p}{\partial\theta},\tfrac{\partial w_p}{\partial r}\rangle=0$.

The curve $\sigma_r(t)=w(t,r)$ is a parametrization of the circle of radius $r$ and center at $p$ in $\Sigma$; that is, if $q=\sigma_r(t)$, then $|q-p|_\Sigma=r$.
If the latter is not the case, then a minimizing geodesic $[pq]_\Sigma$ would be shorter than $r$ and therefore $q$ would not be described uniquely in the polar coordinates. 

Note that $\tfrac{\partial}{\partial r}w\perp \tfrac{\partial}{\partial \theta}w$ if $r>0$;
otherwise for small $\eps>0$ the intrinsic distance from $p$ to $w(\theta\pm \eps,r)$ would be shorter than $r$, which contradicts the previous statement.

\begin{thm}{Proposition}\label{prop:loc-comp-l}
Let $w(\theta,r)$ and $\~w(\theta,r)$ be the polar coordinates of a surface $\Sigma$ at $p$ and its tangent plane $\T_p$ at zero, so $w(\theta,r)\z=\exp_p[\~w(\theta,r)]$.
Given a real interval $[a,b]$ consider the one parameter families of circular arcs $\sigma_r\:[a,b]\to \Sigma$ and $\~\sigma_r\:[a,b]\to \T_p$
$\sigma_r(t)\z=w(t,r)$ and $\~\sigma_r(t)=\~w(t,r)$.
Set $\ell(r)=\length \sigma_r$ and $\~\ell(r)\z=\length \~\sigma_r$.%
\footnote{Note that angular measure of $\~\sigma_r$ is $b-a$; therefore $\~\ell(r)=r\cdot(b-a)$.}

\begin{enumerate}[(i)]
 \item If the Gauss curvature of $\Sigma$ is nonnegative, then 
 \[\ell(r)\le \~\ell(r)\]
 for all small $r>0$.
 \item If the Gauss curvature of $\Sigma$ is nonpositive, then 
 \[\ell(r)\ge \~\ell(r)\]
 for all small $r>0$.
\end{enumerate}

\end{thm}

Taking a limit as $b\to a$, we obtain the following corollary.

\begin{thm}{Corollary}\label{cor:w<w}
Let $w(\theta,r)$ and $\~w(\theta,r)$ be the polar coordinates of a surface $\Sigma$ at $p$ and its tangent plane $\T_p$ at zero, so $w(\theta,r)\z=\exp_p[\~w(\theta,r)]$.
\begin{enumerate}[(i)]
 \item If the Gauss curvature of $\Sigma$ is nonnegative, then 
 \[|\tfrac{\partial}{\partial \theta} w|\le |\tfrac{\partial}{\partial \theta} \~w|\]
 for all small $r>0$.
 \item If the Gauss curvature of $\Sigma$ is nonpositive, then 
\[|\tfrac{\partial}{\partial \theta} w|\ge |\tfrac{\partial}{\partial \theta} \~w|\]
 for all small $r>0$.
\end{enumerate}
\end{thm}


\parit{Proof.}
From the above discussion, the polar coordinates $w(\theta,r)$ are semigeodesic;
that is, $w(\theta,r)$ satisfies the conditions in the first variation formula (\ref{prop:first-var}).
In particular if $\ell(r)=\length \sigma_r$, then
\[\ell'(r)=\Theta_{\sigma_r}\]
for any $r>0$.

By Gauss--Bonnet formula, the last identity can be rewritten as
\[\ell'(r)=2\cdot (b-a) -\iint_{\Delta_r}G,\eqlbl{eq:ell'}\]
where $\Delta_r$ is the sector in $\Sigma$ in the polar coordinates at $p$
\[\set{w(t,s)}{a\le t\le b,\ 0\le s\le r};\]
which is bounded by two geodesics from $p$ with angle $b-a$ 
and a circular arc that meets these geodesics at right angle.

Since the plane has vanishing Gauss curvature, we have
\[\~\ell'(r)=2\cdot (b-a),\eqlbl{eq:tilde-ell'}\]
which agrees with the formula for the length of the arc $\~\ell(r)\z=2\cdot\pi\cdot r$.

If the Gauss curvature of $\Sigma$ is nonnegative,
the equations \ref{eq:ell'} and \ref{eq:tilde-ell'} imply that
\[\ell'(r)\le \~\ell'(r)\]
for any small $r$.

If the Gauss curvature of $\Sigma$ is nonnegative,
the same equations imply that
\[\ell'(r)\ge \~\ell'(r)\]
for any small $r$.

Since $\ell(0)=\~\ell(0)$, integrating the inequalities proves both statements.\qeds

The following exercise provides a stronger statement.
It almost follow from the proof above, but one has to make an extra observation.


\begin{thm}{Exercise}
Assume $\Sigma$ is a smooth regular surface and $p\in\Sigma$,
denote by $\ell(r)$ the circumference of the circle with the center at $p$ and radius $r$ in $\Sigma$
and let $\~\ell(r)=2\cdot\pi\cdot r$ the circumference of the plane circle of radius $r$.

\begin{enumerate}[(i)]
 \item Show that if Gauss curvature of $\Sigma$ is nonnegative, then the function $r\mapsto \ell(r)$ is concave for small $r>0$. Conclude that the function $r\mapsto \frac{\ell(r)}{\~\ell(r)}$ is nonincreasing for small $r>0$.
\item Show that if Gauss curvature of $\Sigma$ is nonpositive, then the function $r\mapsto \ell(r)$ is convex for small $r>0$. Conclude that the function $r\mapsto \frac{\ell(r)}{\~\ell(r)}$ is nondecresing for small $r>0$.
\end{enumerate}

\end{thm}






















\section*{Proof via equidistant surfaces}

Recall that a surface $\Sigma$ is called \emph{orientable} if one can choose at each point $p$ of the surface
a unit normal vector $\Norm_p$  in such a way that the function $p\mapsto \Norm_p$ is continuous in every chart of $\Sigma$.
For immersed surfaces we should say that $\Norm$ is a continuous function defined on the parameter domain of the surface.
The map $\Norm$ is called a \emph{Gauss map} of the surface.

\begin{thm}{Claim}
Assume $\Sigma$ is a closed immersed surface with positive Gauss curvature, then it is orientable.
\end{thm}


\parit{Proof.} Indeed we can choose the unit normal vector $\Norm_p$ in such a way that both principle curvatures are positive. 
In this case the surface lies locally on the side of tangent plane $\T_p$ which is opposite from $\Norm_p$.

Evidently this choice is  continuous.
\qeds

The unit normal described in the proof of the claim will be called \emph{outer normal}.

\begin{thm}{Lemma}\label{lem:gauss-inverse}
Assume $\Sigma$ is a closed connected immersed surface with positive Gauss curvature.
Then the Gauss map $\Norm\:\Sigma \to \mathbb{S}^2$ has a smooth regular inverse;
in particular, $\Sigma$ is a sphere.
\end{thm}

This lemma follows from two facts:
(1) if Gauss curvature does not vanish, then the  Gauss map is regular, in particular this map has a local inverse at each point
and
(2) the sphere $\mathbb{S}^2$ is \emph{simply connected};
that is, $\mathbb{S}^2$ is connected any closed curve in $\mathbb{S}^2$ can be deformed continuously into a trivial curve that stays at one point.
The proof is standard in topology, we hope that the statement is intuitively obvious.
The reader might be able to reinvent the theory by trying to prove that if the map $\phi\:\mathbb{S}^2\to\mathbb{S}^2$ is smooth and regular, then it has an inverse.

\parbf{Equidistant surfaces.}
Assume $\Norm\:\Sigma\to \mathbb{S}^2$ is a Gauss map of a smooth surface $\Sigma$.
Fix a real number $R$ and consider the map $h_R\:\Sigma\z\to\RR^3$ defined by $h_R\:p\mapsto p+R\cdot\Norm_p$.
The map $h_R$ describe the so called \emph{equidistant surface};
it is smooth by definition, but in general it does not have to be regular.

\begin{thm}{Lemma}\label{lem:curc<1/R}
Suppose $\Norm\:\Sigma\to \mathbb{S}^2$ is a Gauss map of a surface $\Sigma$.
Assume the corresponding principle curvatures are nonnegative at all points. 
Then the equidistant surface $\Sigma_R$ is regular and its principle curvatures are positive and strictly smaller than $\tfrac1R$.
\end{thm}

\parit{Proof.}
To prove regularity, let us use the special representation of $\Sigma$ as a graph $z=f(x,y)$ with the $x$ and $y$ axis in the principle directions of $\Sigma$ at $p$.\footnote{If we assume that $\Norm_p$ points in the direction of the $z$-axis, then $\Sigma$ is given in parametric form 
$h_0\:(x,y)\mapsto (x,y,f(x,y))$,
where $f\z=-\tfrac{k_1}2\cdot x^2-\tfrac{k_2}2\cdot y^2+o(x^2+y^2)$.}

Due to the choice of directions of $x$ and $y$ axis,
for the Gauss map $g(x,y)$, we have 
\begin{align*}
\tfrac{\partial}{\partial x}g(0,0)&=(k_1,0,0),
\\
\tfrac{\partial}{\partial y}g(0,0)&=(0,k_2,0).
\end{align*}
Then $h_R=h_0+R\cdot g$; therefore
\begin{align*}
\tfrac{\partial}{\partial x} h_R&=(1+R\cdot k_1,0,0),
\\
\tfrac{\partial}{\partial y} h_R&=(0,1+R\cdot k_2,0)
\end{align*}
which are linearly independent if $R\ge0$ and $k_1,k_2\ge 0$. 

If $\Sigma$ bounds a convex closed set $K$.
Then $\Sigma_R$ bounds $K_R$ --- the closed $R$-neighborhood of $K$;
that is, $K_R$ is the set of all points at distance at most $R$ from $K$.

Since $\Sigma$ is smooth it is supported at each point $p$ from inside by a small ball $B_\eps(o)$.
Then the ball $B_{R+\eps}(o)$ lies in $K_R$ and touches its boundary at the point corresponding to $p$.
Hence the principle curvatures at $p$ are at least $\tfrac1{R+\eps}$.

In the general case, a local chart of $\Sigma$ can be modified so that it has a piece of the original surface around $p$  and bounds a convex set.
Here is one way to do this:

\begin{wrapfigure}{o}{30 mm}
\vskip-0mm
\centering
\includegraphics{mppics/pic-38}
\vskip-0mm
\end{wrapfigure}

Choose a smooth function $\phi(x)$ that is convex increasing and such that for sufficiently small $\eps>0$ we have $\phi(x)=x$ if $x<\eps$ and $\phi(x)\to\infty$ as $x\to 2\cdot\eps$.
(Such functions do exist; moreover an explicit formula can be written, but we leave it without a proof.)

Assume $z=f(x,y)$ is a special representation of $\Sigma$ around $p$ by some convex function $f$.
Direct computations show that $h=\phi\circ f(x,y)$ is still convex.
The surface $\Sigma'$ described as the graph $z=h(x,y)$ bounds a convex closed set $K$ and the part of $\Sigma'$ described by 
the parameters $\set{(x,y)}{f(x,y)<\eps}$ coincide with a neighborhood of $p$ in $\Sigma$.
Hence the general case follows.\qeds

\parit{Proof assembling.}
Let $s\:\mathbb{S}^2\z\to\RR^3$ be the parametization of $\Sigma$ provided by Lemma~\ref{lem:gauss-inverse}.
Then the equidistant surface $\Sigma_R$ can be parametrized by $s_R(u)= s(u)+R\cdot u$ for $u\in\mathbb{S}^2$.
Rescaling $s_R$ by a  factor of $\tfrac1r$ we get the map $u\mapsto \tfrac1R\cdot s(u)+u$ which converges smoothly to the indentity map on the sphere $\mathbb{S}^2$.
Therefore $\Sigma_R$ is embedded for sufficiently large $R$.

\begin{wrapfigure}{o}{47 mm}
\vskip-0mm
\centering
\includegraphics{mppics/pic-39}
\vskip-0mm
\end{wrapfigure}

Applying Theorem~\ref{thm:convex-embedded}, we get that $\Sigma_R$ bounds a convex set.

By Lemma~\ref{lem:curc<1/R}, the principle curvatures of $\Sigma_R$ are smaller than $\tfrac1R$.
Therefore the same idea as in Exercise~\ref{ex:convex-lagunov} shows that any point $p$ of $\Sigma_R$ can be supported by a ball of radius $R$ from inside.
Note that the center $p'$ of such ball has to lie on $\Sigma$;
indeed it lies at distance $R$ in the normal direction.
In other words, the map $s_0(u)=s_R(u)-R\cdot u$ is injective, or equivalently $\Sigma$  has no self-intersection.
\qeds





















\section*{Curve in a surface}


\begin{thm}{Proposition}\label{prop:gamma''=II}
Suppose $\gamma$ is a smooth curve in a smooth oriented surface $\Sigma$ with a unit normal field $\Norm$.
Then, the following identity holds for any time parameter $t$:
\[\langle \gamma''(t),\Norm\circ\gamma(t)\rangle=-\langle \gamma'(t),(\Norm\circ\gamma)'(t)\rangle=\langle S_{\gamma(t)}(\gamma'(t)),\gamma'(t)\rangle.\]

\end{thm}

\parit{Proof.} 
Fix a parameter value $t_0$; set $p=\gamma(t_0)$, $v=\gamma'(t_0)$ and $a\z=\gamma''(t_0)$;
so we need to show that
\[\langle a,\Norm_{p}\rangle=\II_p(v,v).\eqlbl{a-nu-II}\]
Let $z=f(x,y)$ be the local representaion of $\Sigma$ in the tangent-normal coordinates at $p$;
we assume that $\Norm$ points in the direction of~$\Norm_p$.

Without loss of generality may assume that $\gamma$ runs in the graph $z=f(x,y)$;
so 
\[\gamma(t)=\left(x(t),y(t),f(x(t),y(t))\right).\]
Then
\begin{align*}
\gamma'&=(x',y',\tfrac{\partial f}{\partial x}\cdot x'+\tfrac{\partial f}{\partial y}\cdot y');
\\
\gamma''
&=
{\small(x'',
y'',
 \tfrac{\partial^2 f}{\partial x^2}\cdot (x')^2
+
2\cdot \tfrac{\partial^2 f}{\partial x\partial y}\cdot x'\cdot y'
+
\tfrac{\partial^2 f}{\partial y^2}\cdot (y')^2
+
\tfrac{\partial f}{\partial x}\cdot x''
+
\tfrac{\partial f}{\partial y}\cdot y'')}.
\end{align*}

Recall that $p=\gamma(t_0)=(0,0,0)$ and
\begin{align*}
f(0,0)&=0,
&
\tfrac{\partial f}{\partial x}(0,0)&=0,
&
\tfrac{\partial f}{\partial y}(0,0)&=0.
\end{align*}

Therefore 
\begin{align*}
v&=\left(x',y',0\right)(t_0);
\\
a&=\left(x'',y'',
\tfrac{\partial^2 f}{\partial x^2}\cdot (x')^2
+
2\cdot \tfrac{\partial^2 f}{\partial x\partial y}\cdot x'\cdot y'
+
\tfrac{\partial^2 f}{\partial y^2}\cdot (y')^2\right)(t_0).
\end{align*}

Note that 
\[\II_p(v,v)=\left(\tfrac{\partial^2 f}{\partial x^2}\cdot (x')^2
+
2\cdot \tfrac{\partial^2 f}{\partial x\partial y}\cdot x'\cdot y'
+
\tfrac{\partial^2 f}{\partial y^2}\cdot (y')^2\right)(t_0);\]
that is, the $z$-coordinate of the acceleration $a$ equals $\II_p(v,v)$ which is equivalent to~\ref{a-nu-II}.
\qeds



















\chapter{tmp}

\section{Shape}

The linear operator $S\:\T_p\to \T_p$ defined by the matrix multiplication
\[S\:(\begin{smallmatrix}
x\\y
\end{smallmatrix})
\mapsto
M_p\cdot(\begin{smallmatrix}
x\\y
\end{smallmatrix})\] is called \emph{shape operator} of $\Sigma$ at $p$.

For a vector $\vec w\in\T_p$, we use notation $S(\vec w)$ when it is clear from the context which base point $p$ and which surface we are working with;
otherwise we may use notations 
\[S_p(\vec w)\quad\text{or even}\quad S_p(\vec w)_\Sigma.\]













\section*{Shape operator}

Let $z=f(x,y)$ be a local representation of a smooth surface $\Sigma$ in the tangent-normal coordinates at a point $p$.
Let $M_p$ be the Hessian matrix.
Recall that $(x,y)$-plane coincides with the tangent plane $\T_p$.

The operator $S_p\:\T_p\to \T_p$ defined by the matrix multiplication
\[S_p\:(\begin{smallmatrix}
x\\y
\end{smallmatrix})
\mapsto
M_p\cdot(\begin{smallmatrix}
x\\y
\end{smallmatrix})\] is called \emph{shape operator} of $\Sigma$ at $p$.
The following proposition implies that $S_p$ does not depend on the choice of $(x,y)$-coordinates on $\T_p$.





Given two vectors $\vec v=(\begin{smallmatrix}a\\b
\end{smallmatrix})$ and $\vec w=(\begin{smallmatrix}c\\d
\end{smallmatrix})$ in the $(x,y)$-plane, consider the value 
\[\II_p(\vec v,\vec w)\df(D_{\vec w}D_{\vec v}f)(0,0),\]
where $D$ denotes the directional derivative.
The function $(v,w)\z\mapsto \II_p(v,w)$ is called the \emph{second fundamental form} at $p$;\label{page:second fundamental form}
it takes two tangent vectors $v$ and $w$ at $p$ and spits out the real number $\II_p(v,w)$.

The second fundamental form can be written in terms of the Hessian matrix.
Indeed if 
$w=(\begin{smallmatrix}a\\b
\end{smallmatrix})$ 
and 
$v=(\begin{smallmatrix}c\\d
\end{smallmatrix})$, then 
\[D_w=a\cdot \tfrac\partial{\partial x}+b\cdot \tfrac\partial{\partial y}\quad\text{and}\quad D_v=c\cdot \tfrac\partial{\partial x}+d\cdot \tfrac\partial{\partial y}.\]
Therefore 
\[\begin{aligned}
\II_p(w,v)&\df(D_wD_vf)(0,0)=
\\
&=a\cdot c\cdot \ell+(a\cdot d+ b\cdot c)\cdot m +b\cdot d\cdot n=
\\
&=\langle M_p\cdot w,v\rangle=
\\
&=\langle M_p\cdot v,w\rangle.
\end{aligned}
\eqlbl{eq:DwDv}\]
Note that from \ref{eq:DwDv} it follows that $\II_p$ is symmetric; that is,
\[\II_p(v,w)=\II_p(w,v)\]
for any two tangent vectors $v,w\in \T_p$.





\parit{Proof.}
Assume an oriented surface $\Sigma$ is written locally as a graph $z\z=f(x,y)$ in the tangent-normal coordinates at $p\in\Sigma$.
As usual we assume that the normal vector $\Norm_p$ points in the direction of the $z$-axis,
in this case the normal vector at any point of the graph points up; that is, its $z$-coordinate  is positive.

Consider the corresponding chart  of $\Sigma$:
\[s(x,y)\z=(x,y,f(x,y)).\]
Denote by $\Norm(x,y)$ the unit normal vector at $s(x,y)$; it is a shortcut notation for $\Norm_{s(x,y)}$.

Note that $\tfrac{\partial s}{\partial x}(0,0)=(1,0,0)$ and $\tfrac{\partial s}{\partial y}(0,0)=(0,1,0)$.
For a tangent vector 
\[v=(a\cdot\tfrac{\partial s}{\partial x}+b\cdot\tfrac{\partial s}{\partial y})(0,0) =(a,b,0)\in \T_p\]
we have that
\[
\begin{aligned}
S_p(v)&=-D_v\Norm(0,0)=
\\
&=-(a\cdot \tfrac{\partial \Norm}{\partial x}+b\cdot \tfrac{\partial \Norm}{\partial y})(0,0),
\end{aligned}
\eqlbl{eq:S=D}
\]
where $D_v$ denotes the directional derivative along a vector $v$ in the $(x,y)$-plane which is $\T_p$.

Indeed, the first equality follows from \ref{eq:shape} and the definition of differential \ref{eq:differenital} applied for the curve $\gamma(t)=(a\cdot t,b\cdot t, f(a\cdot t,b\cdot t))$ at $t=0$ and the second follow since
$D_v=a\cdot \tfrac{\partial }{\partial x}+b\cdot \tfrac{\partial }{\partial y}$.

Taking partial derivatives of $\langle\Norm,\Norm\rangle=1$, we get that 
\begin{align*}
0&=\tfrac{\partial}{\partial x} \langle\Norm,\Norm\rangle=
\\
&=2\cdot\langle\tfrac{\partial\Norm}{\partial x},\Norm\rangle.
\end{align*}
That is, $\langle\tfrac{\partial\Norm}{\partial x},\Norm\rangle=0$ and the same way we get $\langle\tfrac{\partial\Norm}{\partial y},\Norm\rangle=0$.
By \ref{eq:S=D} it follows that $S_p(v)\perp \Norm_p$, or equivalently $S_p(v)\in\T_p$ for any $v\in \T_p$.

Further, since $\tfrac{\partial s}{\partial x}, \tfrac{\partial s}{\partial y}\in\T_{s(x,y)}\Sigma$
and $\Norm(x,y)\perp\T_{s(x,y)}\Sigma$,
we have that
\[\langle \Norm,\tfrac{\partial s}{\partial x}\rangle\equiv 0
\quad\text{and}\quad
\langle \Norm,\tfrac{\partial s}{\partial y}\rangle\equiv 0.\]
Taking a derivative of these identities, we get that
\[\begin{aligned}
\langle \tfrac{\partial\Norm}{\partial x},\tfrac{\partial s}{\partial x}\rangle+\langle \Norm,\tfrac{\partial^2 s}{\partial x^2}\rangle&\equiv 0,
\\
\langle \tfrac{\partial\Norm}{\partial y},\tfrac{\partial s}{\partial x}\rangle+\langle \Norm,\tfrac{\partial^2 s}{\partial y\partial x}\rangle&\equiv 0,
\\
\langle \tfrac{\partial\Norm}{\partial x},\tfrac{\partial s}{\partial y}\rangle+\langle \Norm,\tfrac{\partial^2 s}{\partial x\partial y}\rangle&\equiv 0,
\\
\langle \tfrac{\partial\Norm}{\partial y},\tfrac{\partial s}{\partial y}\rangle+\langle \Norm,\tfrac{\partial^2 s}{\partial y^2}\rangle&\equiv 0,
\end{aligned}
\]

Fix two vectors $v=(a,b,0)$ and $w=(c,d,0)$ in $\T_p$ (which is the $(x,y)$-plane).
Since $\Norm(0,0)=(0,0,1)$ we get 
\[f(x,y)\equiv\langle\Norm(0,0),s(x,y)\rangle.\]
Therefore by \ref{eq:S=D}, \ref{eq:shape=second} and \ref{eq:DwDv} we get that 
\begin{align*}
\langle S_p(v),w\rangle 
&=-\langle  a\cdot \tfrac{\partial \Norm}{\partial x}+b\cdot \tfrac{\partial \Norm}{\partial y},c\cdot \tfrac{\partial s}{\partial x}+d\cdot \tfrac{\partial s}{\partial y}\rangle(0,0)=
\\
&=-\bigl(a\cdot c\cdot\langle \tfrac{\partial \Norm}{\partial x},\tfrac{\partial s}{\partial x}\rangle 
+a\cdot d\cdot\langle \tfrac{\partial \Norm}{\partial x},\tfrac{\partial s}{\partial y}\rangle+
\\&\quad
+b\cdot c\cdot\langle \tfrac{\partial \Norm}{\partial y},\tfrac{\partial s}{\partial x}\rangle
+b\cdot d\cdot\langle \tfrac{\partial \Norm}{\partial y},\tfrac{\partial s}{\partial y}\rangle\bigr)(0,0)=
\\
&=\bigl(a\cdot c\cdot\langle \Norm,\tfrac{\partial^2 s}{\partial x^2}\rangle 
+a\cdot d\cdot\langle \Norm,\tfrac{\partial^2 s}{\partial x\partial y}\rangle+
\\&\quad
+b\cdot c\cdot\langle \Norm,\tfrac{\partial^2 s}{\partial y\partial x}\rangle
+b\cdot d\cdot\langle \Norm,\tfrac{\partial^2 s}{\partial y^2}\rangle\bigr)(0,0)=
\\
&=\bigl(a\cdot c\cdot\tfrac{\partial^2 f}{\partial x^2} 
+a\cdot d\cdot\tfrac{\partial^2 f}{\partial x\partial y}+
\\&\quad
+b\cdot c\cdot\tfrac{\partial^2 f}{\partial y\partial x}
+b\cdot d\cdot\tfrac{\partial^2 f}{\partial y^2}\bigr)(0,0)=
\\
&=\II_p(v,w).
\end{align*}
It remains to apply \ref{eq:II=II} on page \pageref{eq:II=II}.
\qeds
















\section*{???}

Denote by $\nu(x,y)$ the normal vector at the point $(x,y,f(x,y))\in \Sigma$, so we may think that $\nu$ is defined on the tangent plane.

\begin{thm}{Claim}
Let $\Sigma$ be a smooth surface with unit normal field $\Norm$.
Suppose $p\in \Sigma$ and $S\:\T_p\to\T_p$ is the shape operator at $p$.
Then 
\[S(\vec w)=-D_{\vec w}\nu\]
for any $\vec w\in \T_p$.
\end{thm}

















Then 
\begin{align*}
\langle\tfrac{\partial^2 s}{\partial^2 u},\Norm\circ s\rangle
&=
-\langle\tfrac{\partial s}{\partial u},\tfrac{\partial \Norm\circ s}{\partial u}\rangle,
\\
\langle\tfrac{\partial^2 s}{\partial u\partial v},\Norm\circ s\rangle
&=
-\langle\tfrac{\partial s}{\partial v},\tfrac{\partial \Norm\circ s}{\partial u}\rangle=
\\
=\langle\tfrac{\partial^2 s}{\partial v\partial u},\Norm\circ s\rangle
&=
-\langle\tfrac{\partial s}{\partial u},\tfrac{\partial \Norm\circ s}{\partial v}\rangle,
\\
\langle\tfrac{\partial^2 s}{\partial^2 v},\Norm\circ s\rangle
&=
-\langle\tfrac{\partial s}{\partial v},\tfrac{\partial \Norm\circ s}{\partial v}\rangle.
\end{align*}




%???product rule???

The proof is based on three identities $\langle\Norm,\Norm\rangle=1$ and $\langle\Norm,\vec w\rangle=0$ for any tangent vector $\vec w$ and the unit normal vector $\Norm$ with the same base point.
It is sufficient to take directional derivatives of these two identity and understand the result.
In the proof we use the following product rule:
\[D_{\vec w}\langle\vec u,\vec v\rangle
=
\langle D_{\vec w}\vec u,\vec v\rangle+\langle\vec u,D_{\vec w}\vec v\rangle.\]
which easely follows from the standard product rule and the definition of directional derivative.

\parit{Proof.}
Since $\Norm$ is a unit normal field, we have the identity
\[1=\langle\Norm,\Norm\rangle.\]
Taking the directional derivative along $\vec w$, we get
\begin{align*}0&=D_{\vec w}\langle\Norm,\Norm\rangle=
\\
&=2\cdot \langle D_{\vec w}\Norm,\Norm\rangle.
\end{align*}
That is, $D_{\vec w}\Norm$ lies in the orthogonal complement of $\Norm_p$;
or, equivalently, $D_{\vec w}\Norm\in\T_p$.

Note that both sides of the identity \ref{eq:shape=-dNorm} are linear in $\vec w$.
Therefore it is sufficient to check the identity for two basis vectors $\vec u$ and $\vec v$ of $\T_p$.
That is, it is sufficient to establish the following two identities:
\begin{align*}
S(\vec u)&= -D_{\vec u}\Norm,
&
S(\vec v)&= -D_{\vec v}\Norm.
\end{align*}
Further, since $D_{\vec u}\Norm,D_{\vec v}\Norm\in\T_p$, to prove the two vector indentities, it sufficient to prove the following four real identities:
\[\begin{aligned}
\langle S(\vec u),\vec u\rangle &= -\langle D_{\vec u}\Norm,\vec u\rangle,
&
\langle S(\vec v),\vec u\rangle &= -\langle D_{\vec v}\Norm,\vec u\rangle,
\\
\langle S(\vec u),\vec v\rangle &= -\langle D_{\vec u}\Norm,\vec v\rangle,
&
\langle S(\vec v),\vec v\rangle &= -\langle D_{\vec v}\Norm,\vec v\rangle.
\end{aligned}
\eqlbl{eq:<S(u),v>=<-Du,v>}
\]

Let $z=f(x,y)$ be a local description of $\Sigma$ in the tangent-normal coordinates at $p$.
Note that 
\[s(u,v)=(u,v,f(u,v))\]
describes a chart of $\Sigma$ at $p$.
Let us prove the indentities \ref{eq:<S(u),v>=<-Du,v>} for the basis $\vec u=\tfrac{\partial s}{\partial u}$ and $\vec v=\tfrac{\partial s}{\partial v}$.
Note that 
\[f=\langle s, \vec k\rangle=\langle s, \nu(p)\rangle.\]???


Note that the vector fields $\vec u=\tfrac{\partial s}{\partial u}$ and $\vec v=\tfrac{\partial s}{\partial v}$ are tangent to $\Sigma$  and therefore they are orthogonal to $\Norm$;
that is, we have two identities:
\[\begin{aligned}
0&=\langle \Norm,\vec u\rangle=\langle \Norm,\tfrac{\partial s}{\partial u}\rangle,
\\
0&=\langle \Norm,\vec v\rangle=\langle \Norm,\tfrac{\partial s}{\partial v}\rangle.
\end{aligned}
\eqlbl{eq:<Norm,uv>}\]
Applying the product rule for the directional derivative of the first identity in \ref{eq:<Norm,uv>} along $\vec u$, we get that
\begin{align*}
0&=D_{\vec u}\langle \Norm,\vec u\rangle=
\\
&=\langle D_{\vec v}\Norm,\vec u\rangle
+
\langle \Norm,D_{\vec v}\vec u\rangle
\end{align*}
Note that $\langle\nu(p),s(u,v)\rangle=f(u,v)$.
Therefore evaluating the last expression at $p$, we get
\[\langle D_{\vec v}\Norm,\tfrac{\partial s}{\partial u}\rangle+\tfrac{\partial^2 f}{\partial v\partial u}=0.\]



Since $\tfrac{\partial^2 s}{\partial u\partial v}=\tfrac{\partial^2 s}{\partial v\partial u}$, we get
\[\langle S({\vec u}),\vec v\rangle=\langle {\vec u},S(\vec v)\rangle;\]
that is, the identity \ref{eq:S-self-adjoint} for two vectors $\vec u=\tfrac{\partial s}{\partial u}$ and $\vec v=\tfrac{\partial s}{\partial v}$.
Evidently the identity \ref{eq:S-self-adjoint} holds if $\vec u=\vec v$.

It follows that the identity \ref{eq:S-self-adjoint} holds for any pair of vectors in a basis of $\T_p$; that is we have the following four identities:
\begin{align*}
 \langle S({\vec u}),\vec u\rangle&=\langle {\vec u},S(\vec u)\rangle,
 &\langle S({\vec u}),\vec v\rangle&=\langle {\vec u},S(\vec v)\rangle
 \\
 \langle S({\vec v}),\vec u\rangle&=\langle {\vec v},S(\vec u)\rangle,
 &\langle S({\vec v}),\vec v\rangle&=\langle {\vec v},S(\vec v)\rangle.
\end{align*}

Since both sides of this identiy \ref{eq:S-self-adjoint} are linear in $\vec u$ and $\vec v$,
the latter is sufficient to conclude that \ref{eq:S-self-adjoint} for any pair of tangent vectors at $p$.
Let us present the formal calculations for two vectors $a\cdot \vec u+b\cdot\vec v$ and $c\cdot \vec u+d\cdot\vec v$:
\begin{align*} 
\langle S(a\cdot \vec u+b\cdot\vec v),c\cdot \vec u+d\cdot\vec v\rangle
&=a\cdot c\cdot\langle S({\vec u}),\vec u\rangle + a\cdot d\cdot \langle S({\vec u}),\vec v\rangle+
\\
&+b\cdot c\cdot\langle S({\vec v}),\vec u\rangle + b\cdot d\cdot \langle S({\vec v}),\vec v\rangle=
\\
&=a\cdot c\cdot\langle {\vec u},S(\vec u)\rangle + a\cdot d\cdot \langle {\vec u},S(\vec v)\rangle+
\\
&+b\cdot c\cdot\langle {\vec v},S(\vec u)\rangle + b\cdot d\cdot \langle {\vec v},S(\vec v)\rangle=
\\
&=\langle a\cdot \vec u+b\cdot\vec v,S(c\cdot \vec u+d\cdot\vec v)\rangle.
\end{align*}

The vectors $\vec u$ and $\vec v$ are tangent to $\Sigma$, we have



\qeds



Let $\Sigma$ be an smooth oriented surface in the Euclidean space;
denote by $\Norm$ its unit normal field.

Given a point $p$ on $\Sigma$ and a tangent vector $\vec w$ at $p$,
the shape operator of $\vec w$ at $p$ is defined as
\[S(\vec w)=-D_{\vec w}\Norm,\]
where $D_{\vec w}$ denotes the directional derivative defined in \ref{def:directional-derivative}.




\begin{thm}{Proposition}
Let $\Sigma$ be an smooth oriented surface and $p\in \Sigma$.
Then the shape operator $S_p$ is a linear self-adjoint operator defined on the tangent plane $\T_p$;
that is, $\vec v\mapsto S_p(\vec v)$ defines a linear map $\T_p\to \T_p$ and
\[\langle S_p(\vec u), \vec v\rangle =\langle \vec u, S_p(\vec v)\rangle\eqlbl{eq:S-self-adjoint}\]
for any two vectors $\vec u, \vec v \in \T_p$.
\end{thm}

\parit{Proof.}
The linearity follow from \ref{ex:linerity}.

Since $\Norm$ is a unit normal field, we have the identity
\[1=\langle\Norm,\Norm\rangle.\]
Taking the directional derivative along $\vec v$, we get
\begin{align*}0&=D_{\vec v}\langle\Norm,\Norm\rangle=
\\
&=2\cdot \langle D_{\vec v}\Norm,\Norm\rangle=
\\
&=-2\cdot \langle S({\vec v}),\Norm\rangle.
\end{align*}
That is, $S_p({\vec v})$ lies in the orthogonal complement of $\Norm_p$;
or, equivalently, $S_p({\vec v})\in\T_p$.

Now let us fix a chart $(u,v)\mapsto s(u,v)$ of $\Sigma$ at $p$.
Note that the vectors $\vec u=\tfrac{\partial s}{\partial u}$ and $\vec v=\tfrac{\partial s}{\partial v}$ are tangent to $\Sigma$ at their base points and therefore they are orthogonal to $\Norm$;
that is we have two identities:
\[\begin{aligned}
0&=\langle \Norm,\tfrac{\partial s}{\partial u}\rangle,
\\
0&=\langle \Norm,\tfrac{\partial s}{\partial v}\rangle.
\end{aligned}
\eqlbl{eq:<Norm,uv>}\]
Taking a directional derivative of the first identity in \ref{eq:<Norm,uv>} along $\vec v$ , we get that
\begin{align*}
0&=D_{\vec v}\langle \Norm,\tfrac{\partial s}{\partial u}\rangle=
\\
&=\langle D_{\vec v}\Norm,\tfrac{\partial s}{\partial u}\rangle
+
\langle \Norm,D_{\vec v}\tfrac{\partial s}{\partial u}\rangle=
\\
&=-\langle S({\vec v}),\vec u\rangle+\langle \Norm,\tfrac{\partial^2 s}{\partial v\partial u}\rangle.
\intertext{Analogously, taking a directional derivative of the second identity in \ref{eq:<Norm,uv>} along $\vec u$, we get that}
0&=D_{\vec u}\langle \Norm,\tfrac{\partial s}{\partial v}\rangle=
\\
&=\langle D_{\vec u}\Norm,\tfrac{\partial s}{\partial v}\rangle
+
\langle \Norm,D_{\vec u}\tfrac{\partial s}{\partial v}\rangle=
\\
&=-\langle S({\vec u}),\vec v\rangle+\langle \Norm,\tfrac{\partial^2 s}{\partial u\partial v}\rangle.
\end{align*}

Since $\tfrac{\partial^2 s}{\partial u\partial v}=\tfrac{\partial^2 s}{\partial v\partial u}$, we get
\[\langle S({\vec u}),\vec v\rangle=\langle {\vec u},S(\vec v)\rangle;\]
that is, the identity \ref{eq:S-self-adjoint} for two vectors $\vec u=\tfrac{\partial s}{\partial u}$ and $\vec v=\tfrac{\partial s}{\partial v}$.
Evidently the identity \ref{eq:S-self-adjoint} holds if $\vec u=\vec v$.

It follows that the identity \ref{eq:S-self-adjoint} holds for any pair of vectors in a basis of $\T_p$; that is we have the following four identities:
\begin{align*}
 \langle S({\vec u}),\vec u\rangle&=\langle {\vec u},S(\vec u)\rangle,
 &\langle S({\vec u}),\vec v\rangle&=\langle {\vec u},S(\vec v)\rangle
 \\
 \langle S({\vec v}),\vec u\rangle&=\langle {\vec v},S(\vec u)\rangle,
 &\langle S({\vec v}),\vec v\rangle&=\langle {\vec v},S(\vec v)\rangle.
\end{align*}

Since both sides of this identiy \ref{eq:S-self-adjoint} are linear in $\vec u$ and $\vec v$,
the latter is sufficient to conclude that \ref{eq:S-self-adjoint} for any pair of tangent vectors at $p$.
Let us present the formal calculations for two vectors $a\cdot \vec u+b\cdot\vec v$ and $c\cdot \vec u+d\cdot\vec v$:
\begin{align*} 
\langle S(a\cdot \vec u+b\cdot\vec v),c\cdot \vec u+d\cdot\vec v\rangle
&=a\cdot c\cdot\langle S({\vec u}),\vec u\rangle + a\cdot d\cdot \langle S({\vec u}),\vec v\rangle+
\\
&+b\cdot c\cdot\langle S({\vec v}),\vec u\rangle + b\cdot d\cdot \langle S({\vec v}),\vec v\rangle=
\\
&=a\cdot c\cdot\langle {\vec u},S(\vec u)\rangle + a\cdot d\cdot \langle {\vec u},S(\vec v)\rangle+
\\
&+b\cdot c\cdot\langle {\vec v},S(\vec u)\rangle + b\cdot d\cdot \langle {\vec v},S(\vec v)\rangle=
\\
&=\langle a\cdot \vec u+b\cdot\vec v,S(c\cdot \vec u+d\cdot\vec v)\rangle.
\end{align*}
\qedsf






























 On page 58, at the end of the first paragraph, you have "... its invese W → U is also continuous." Notice, "inverse" has a spelling error in the sentence. 
 
 In the second paragraph of this page, you have "However, as well as in the case of curves we will be mostly interested in smooth surfaces in the Euclidean space describe in the following section." This is missing a comma which should be placed after "curves," 
 
 and "describe" should be "described." 
 
 On page 60, in the last sentence, you have "If s and Σ as in the proposition..." which should be either "If s and Σ (are defined)/(exist) as in the proposition..." or "If we have s and Σ as in the proposition..." 
 
 In 6.4 on page 61, you have "Assume γ is a closed simple smooth regular plane curve that does not intersect x-axis. Show that surface of revolution of γ around x-axis is a smooth regular surface." This should include a "the" before both of the "x-axis" and before "surface of revolution." 
 
 On page 62, in the paragraph above Implicitly Defined Curves, you have "For example, any embedding s: S 2 → R 3 might be called topological sphere and if s is smooth and regular, then it might be called smooth sphere. (A smooth regular map s: S 2 → R 3 which is not necessary an embedding is called smooth regular immersion, so we can say that s describes a smooth immersed sphere.) Similarly an embedding s: R 2 → R 3 might be called topological plane and if s is smooth it might be called smooth plane. " This should include an "a" before both uses of "topological," as well as before "smooth regular immersion" and "smooth plane." 
 
 Also, the last sentence should have two additional commas; one after "topological plane," 
 
 and one after "if s is smooth." 
 
 In 6.10 on page 64, you have "Show that a neighborhood of p in Σ is a graph z = f(x, y) of a smooth function f defined on an open subset in (x, y)- plane if and only if the tangent plane Tp is not a vertical plane; that is if the projection of Tp to (x, y)-plane does not degenerates to a line. " 
 This should have a "the" before both uses of "(x, y)-plane," 
 
 and "degenerates" should be "degenerate." 
 
 Another on page 64, in the third paragraph of the Normal vector and orientation section, you have "The choice of the field ν is called orientation on Σ." This should include a "the" before "orientation." 
 
 Just below this in the next paragraph, "Mobius strip shown on the diagram gives an example of nonorientable surface." This should either have an "a" before "nonorientable," or should have "surface" changed to "surfaces." 
 
 In the hint of 6.11 on page 65, "Its proof is not at all trivial; a standard proof use the so called Alexander’s duality which is a classical technique in algebraic topology," in which "use" should be changed to "uses," 
 
 and "so called" should be hyphenated. 
 
 In the sentence between 6.12 and 6.13 on page 65, "filed" needs to be corrected to "field." 
 
 In the short paragraph between 6.14 and 6.15 on this page, "section" is supposed to be "intersection," and there should be an "it" between the words "makes possible." 
 
 Below, in 6.15, there should be an "an" before "arbitrarily." 
 
 On page 66, in the sentence "Note that Π' || Π and arbitrary close to it," you should add an "is" after "and," and "arbitrary" should be changed to "arbitrarily." At the end of page 67, "so called" should be hyphenated. 
 
 On page 68, at the top, "The function (v, w) → → IIp(v, w) is called second fundamental form at p; it takes two tangent vector v and w at p and spits the real number IIp(v, w)." I don't know if you meant to have two arrows following (v,w) since there is a line break, but I would assume not. Also, there should be a "the" added after "called," 
 
 "out" added after "spits," 
 
 and "vector" should be "vectors." 
 
 On page 68, in the first paragraph beneath Principle curvatures, "... it is unique up to a rotation of the (x, y)-plane and switching the sign of the z-coordinate," should have an "a" added before "switching the," 
 
 and an "of" added between "switching the." 
 
 In the next paragraph, you have "... they are uniquely defined up to sign; they are denoted as k1(p) and k2(p) or k1(p)Σ and k2(p)Σ if we need to emphasize that these are the curvatures of the surface Σ;" which should include "a change in the" between "to" and "sign," as well as a comma after k2(p). 
 
 On page 69, above 7.2, you have "The last identity is the so called Euler’s formula. A smooth regular curve on a surface Σ that always runs in the principle directions is called line of curvature of Σ." "So called" should be hyphenated and there should be an "a" added between "called" and "line." 
 
 On page 70, in the first two paragraphs, you use the phrase "... if we need to emphasize that this is curvature of Σ." I believe writing "... if we need to emphasize that this is a curvature value of Σ." would be more clear given the context.  In the third paragraph, "up to sign" should be "up to a change in the sign." On page 73, just below equation (5), "derection" should be "direction." 
 
 Just above 7.12 on this page, "acelaration a equals to IIp(v, v)" has acceleration misspelled and 
 
 should either be "acceleration a is equal to IIp(v, v)" or just "acceleration a equals IIp(v, v)." 
 
 On page 75, in the second paragraph, "Note that in this case tangent plane is does not support the surface even locally," there should be a "the" before tangent 
 
 and "is" should be removed. 
 
 Later in this paragraph, "moving along the surface in the principle directions at a given point, one gets above and below the tangent plane at this point," should have "gets" changed to "goes." 
 
 On page 76, in 8.3, there should be an "a" between "is" and "complete." 
 
 In the paragraph following, "A smooth regular curve that always run in an asymptotic direction is called asymptotic line," should be changed to "A smooth regular curve that always runs in an asymptotic direction is called an asymptotic line." 
 
 On page 78, in 8.11, "a invertible" should be "an invertible." 
 
 On the same page, in the paragraph above 8.13, " applying 8.11 we get that saddle graph cannot lie between parallel planes, not necessarily horizontal. The following exercise shows that the theorem does not hold for saddle surface which are not graphs," in which the "graph" in the first sentence should be "graphs," 
 
 and the "surface" in the second sentence should be "surfaces." 
 
 On page 79, in the hint of 8.14, "Look at two section of the graph by planes parallel to (x, y)- plane and to (x, z)-plane and apply Meusnier’s theorem (7.12)," which would be more proper and clear by changing "section" to "sections" and writing instead "Look at two sections of the graph by considering planes parallel to the (x, y)- plane and to the (x, z)-plane, then apply Meusnier’s theorem (7.12)." 
 
 At the bottom of this page, "Since the plane Π is a convex, this statement contradicts 8.5," should be changed to "... plane Π is convex..." 
 
 At the beginning of page 81, "Moving the plane Tp little up, we can cut from Σ is a complete surface with boundary line lying in this plane (see 6.15)" which I think would be much more clear written as"Moving the plane Tp slightly upward, we can cut a complete surface  Π  from Σ such that the boundary line of Π lies in this slightly modified tangent plane (see 6.15)." 
 
 Further in this page, in the first paragraph under remarks, either the "a" should be removed from "... be found among infinite cylinders over a smooth regular curves," or "curves" should instead by "curve." 
 
 In the next sentence, I'm not sure what you are trying to say, should there be a "the" between "are" and "only?"
 
 In the last paragrah on this page, you have "This definition can be used for arbitrary surfaces not necessarily smooth. Some results, for example Bernshtein’s characterization of saddle graphs can be extended to generalized saddle surface, but this class of surfaces is far from being understood. Some nontrivial properties were proved by in Samuil Shefel..." There should be a comma in the first sentence before "not,"
 
 the second sentence should have "saddle surfaces," not "saddle surface," 
 
 and the last sentence should have "in" removed. 
 
 On page 83, in the first sentence, "...  so that the both principle curvatures..." should instead be "... so that both of the principle curvatures..." 
 
 On page 83, in the proof of 9.2, "sets" should be "set" in the sentence "Therefore the interior of R is a convex sets." 
 
 On page 84, right above the Open Surfaces section, "... classical result it topology — the so called closed graph theorem," should have "so called" hyphenated and "it" should be "in." 
 
 Further down this page, in the 9.4 Lemma, "in" should be removed from the sentence "Suppose Σ is an open surface in with positive Gauss..." 
 
 On page 85, about one-third of the way down the page, "poins" should be "points" in the sentence "... all of these poins lie in R." 
 
 In the third sentence on page 88, the word "minimize" should instead be "minimizes." 
 
 Further down the page, in the Nonuniqueness sentence, you have "There plenty..." which should have an "are" added between "there" and "plenty." 
 
 In the lemma for 10.3 on page 89, you have "... there is unique point..." This should have an "a" added between "is" and "unique." 
 
 Further in the page, in the proof of this lemma, you have "... lie in a ball or radius l + 1 centered at p." I believe the "or" here should be an "of." 
 
 On page 93, in the second paragraph of the Exponential Map section, you have "(There is a reason to call this map exponential, but it will takes us to far from the subject.)" "Takes" should be changed to "take" 
 
 and "to should be changed to "too." 
 
 In the following sentence on the page, the word "compete" should be corrected to "complete." 
 
 Just a little further in the page, in the paragraph just above 10.12, the word "has" should be removed from the sentence "It follows that the Jacobian matrix of the projection of expp to the (x, y)-plane has is the identity matrix." 
 
 On page 94, half way down the page, you have "For instance shortest path γ is an object of intrinsic geometry of the surface Σ, while definition of geodesic is not intrinsic — it requires the second derivative γ'' which needs the ambient space." This sentence should have a comma after instance, 
 
 another comma after γ'', 
 
 and "the" added between "while" and "definition." 
 
 In the following sentence on page 94, there should be an "a" added before the word "cylinder." 
 
 On page 95, at the end of 10.16, the word "is" should be removed from the sentence "Show that the statement is does not hold if γ fails to be minimizing." 
 
 On page 96, in the paragraph above 10.18, the word "plan" should be "plane" in the sentence "... we have that tangent plan supports the graph..." 
 
 On page 98, in both paragraphs of 10.21, the phrase "a convex function n-Lipschitz function" is used, clearly the "function" following "convex" can be removed from both paragraphs. 
 
 In the hint below 10.21, "guarantee" should be changed to "guarantees," 
 
 and "the" should be added in before the word "cone." 
 
 On page 99, in the paragraph directly under equation 1,  the word "produce" should be changed to "produces." 
 
 On page 101, in the paragraph above 11.3, you have "... sames plane (otherwise we could not call it operator )." In this sentence, "sames" should be changed to "same" 
 
 and the word "an" should be added in before "operator." 
 
 On page 104, in the hint of 11.6, "unite" should be changed to "unit." 
 
 On page 105, in the first paragraph of the proof of 11.8, the phrase "a obtuse" is used twice, 
 
 and "a" should be changed to "an" in both of these occurences. 
 
 On page 108, in 12.2, "is" should be changed to "be" in the first sentence, 
 
 and the "a" before "along" in the last sentence should be removed. 
 
 On page 115, in 12.12, "Lipshitz" is mispelled, missing the 'c.'






















According to \ref{ex:differential-range}, the shape operator takes a tangent vector at $p$ and spits a tangent vector to $\mathbb{S}^2$ at $\nu_p$.
Note that $\nu_p$ is the normal vector to $\mathbb{S}^2$ at $\nu_p$.
In other words the tangent plane $\T_p\Sigma$ is parallel to the tangent plane $\T_{\nu_p}\mathbb{S}^2$,
or $\T_p\Sigma=\T_{\nu_p}\mathbb{S}^2$, if we consider the tangent planes as linear spaces. 












Let us write $\Sigma$ in the spherical $(\rho,u)$-coordinates where $\rho\ge 0$ and $u\in\mathbb{S}^2$.



Consider the map $f\:\Sigma\to\mathbb{S}^2$ defined by $f(x)=\tfrac1{|x|}\cdot x$.
By construction $f$ is a smooth map.
Since $R$ is convex, and $\Sigma$ is bounded, on any ray from the origin there is exactly one point of $\Sigma$;
that is, $f$ gives a continuous bijection between $\Sigma$ and $\mathbb{S}^2$.
Since $\mathbb{S}^2$ is compact, the inverse map $f^{-1}\:\mathbb{S}^2\to \Sigma$ is continuous; 
this is, $f$ is a homeomorphism.
The latter statement is a standard result in topology \ref{thm:Hausdorff-compact}.

Summarizing, we proved that $\Sigma$ is a topological sphere.

Let us show that $f$ is regular.
In other words, for any cart $(u,v)\mapsto s(u,v)$ of $\Sigma$
the vectors $\tfrac{\partial f\circ s}{\partial u}$ and $\tfrac{\partial f\circ s}{\partial v}$ are linearly independent.

Note that
\begin{align*}
\frac{\partial (f\circ s)}{\partial u}&=\frac{\partial}{\partial u}\frac{s}{|s|}=
\\
&=\frac{1}{|s|}\cdot \frac{\partial s}{\partial u}+\left(\frac{\partial}{\partial u}\tfrac{1}{\sqrt{\langle s,s\rangle}}\right)\cdot s=
\\
&=\frac{1}{|s|}\cdot \frac{\partial s}{\partial u}
-
\left(\frac{\langle s,\frac{\partial s}{\partial u}\rangle}{|s|^3}\right)\cdot s=
\\
&=\frac{1}{|s|}\cdot \left(\frac{\partial s}{\partial u}\right)^\perp_s,
\intertext{where $v^\perp_s$ denotes the orthogonal projection of the vector $v$ to the line in the direction of $s$. 
The same way we get}
\frac{\partial (f\circ s)}{\partial v}&=\frac{1}{|s|}\cdot \left(\frac{\partial s}{\partial v}\right)^\perp_s.
\end{align*}

Since $R$ is convex and the origin in its interior, 
$s(u,v)$ can not point in the direction of the tangent plane $\T_{s(u,v)}$;
that is, three vectors $s(u,v)$, $\frac{\partial s}{\partial u}(u,v)$ and $\frac{\partial s}{\partial v}(u,v)$ are linearly independent.
Whence the projections $\left(\frac{\partial s}{\partial u}\right)^\perp_s$ and $ \left(\frac{\partial s}{\partial v}\right)^\perp_s$ are linearly independent
and therefore $f$ is regular.
By the inverse function theorem the inverse $f^{-1}\:\mathbb{S}^2\to \Sigma$ is also regular;
that is, $\Sigma$ is a smooth sphere.
\qeds



















\begin{thm}{Theorem}
Let $\Sigma$ be a complete oriented surface with 
nonnegative principal curvatures at all points.
Then $\Sigma$ bounds a convex region $R$.
\end{thm}

\parit{Proof.}
Let $\nu$ be the unit normal field on $\Sigma$.
Consider the open region $R$ bounded by $\Sigma$ that lies on the side of $\nu$;
that is $\nu$ points inside of $R$ at any point of $\Sigma$.

Fix a plane $\Pi$ intersecting $R$.
Consider a conn

Since $\Sigma$ is connected, so is $R$;
moreover any two points in the interior of $R$ can be connected by a polygonal line in the interior of~$R$.

Assume the interior of $R$ is not convex; that is, there are points $x,y\in R$ and a point $z$ between $x$ and $y$ that does not lie in the interior of $R$.
Consider a polygonal  line $\beta$ from $x$ to $y$ in the interior of $R$.
Let $y_0$ be the first point on $\beta$ such that the chord $[x,y_0]$ touches $\Sigma$ at some point, say~$z_0$.

\qeds












\parit{Proof.}
Denote by $\Gamma$ the graph $z=f(x,y)$.
Assume contrary; that is, $\Gamma$ lies between two planes $z=\pm C$.

Note $f$ can not be constant.
It follows that there is a line $\ell$ in $\RR^3$ with three points $a,b,c$ in the same order such that 
$a$ and $c$ are above $\Gamma$ and $b$ is below $\Gamma$.
Without loss of generality we can assume that the $z$-coordinate of $c$ is bigger than $C$.

Consider one parameter family of planes $\Pi_t$ containing $\ell$ that starts and ends at the vertical plane $\Pi_0=\Pi_1$ and rotates by angle $\pi$ while $t$ goes from $0$ to $1$.
For each $t$, consider the connected component $\Omega_t$ of $a$ in the subset of $\Pi_t$ that lies above $\Gamma$.

Let us show that $\Omega_t\not\ni c$ for some $t$.

If  $\Omega_t\ni c$, then $a$ can be connected by a path to $c$ in $\Omega_t$.
Since $\Gamma$ is saddle, the point $b$ cannot be surrounded by a closed curve in $\Omega_t$;
otherwise we would get a contradiction to ???.
Whence a any path from $a$ to $c$ in $\Omega_t$ might pass $b$ on the right side or on the left side,
but there is no pair of such paths that pass $b$ on both sides.
In particular we can divide subdivide the parameter values $[0,1]$ into three sets $t\in R$ if the path  in $\Omega_t$  goes on the right side from $b$,  $L$ if the path in $\Omega_t$ goes on the right side from $b$ and $N=[0,1]\backslash(R\cup L)$ if there is no path in $\Omega_t$ from $a$ to $b$.

Notice that $R$ and $L$ are open; indeed, if a path $\gamma$ lies in $\Omega_{t_0}$, then, since the epigraph $z>f(x,y)$ is open, all close paths lie in the epigraph.
In particular there will be a path from $a$ to $b$ in $\Omega_t$ for all $t$ sufficiently close to $t_0$,
evidently it goes on the same side from $b$.

Notice that $R\ni 0$ and $L\ni 1$; that is $R$ and $L$ are nonempty.
Since $[0,1]$ is connected, it can not be decomposed into two open nonempty sets.
Whence $N$ is not empty.
(In fact one can show that $s=\sup R$ lies in $N$.)


Fix $t\in N$; note that $\Omega_t$ lies between the planes $z=\pm C$;
otherwise $\Omega_t\ni c$.
Denote by $\Gamma_t$ the part of the graph $\Gamma$ that lies above $\Omega_t$.
By Sard's lemma we can assume that $\Gamma_t$ is a smooth surface with boundary on $\Pi_t$.

Denote by $N$ the plane perpendicular to $\ell$.
Let $n$ be the projection of $\Pi_t$ to $N$; 
it is a line in $N$.
Since $\Gamma_t$ and $\Omega_t$ lies between the planes $z=\pm C$, the projection of $\Gamma_t$ to $L$ lies on a bounded distance from $n$.

Applying \ref{ex:saddle-projection}, we get that the complement of the projection of $\Gamma_t$ in the $L^+$ convex. 
In order to be bounded it has to be a 


In particular the projection of $\Gamma_t$ the 

consider the projection $\proj\Omega_t$ of $\Omega_t$ to the plane $L$ perpendicular to $\ell$.
According to \ref{ex:saddle-projection},
each connected component of the complement of $\proj\Omega_t$  in the halfplane ??? is convex.
On the other hand by construction $\proj\Omega_t$ lies on the finite distance to ???.
It follows that $\proj\Omega_t$ is a strip of fixed width to ...

It follows that a plane parallel to $\Pi_t$ is supporting $\Gamma$ along a line.

\qeds

\begin{wrapfigure}{o}{40 mm}
\vskip-0mm
\centering
\includegraphics{mppics/pic-72}
\vskip0mm
\end{wrapfigure}

\begin{thm}{Exercise}\label{ex:saddle-projection}
Let $\Omega$ is an open convex set in the $(x,y)$-plane and 
$\Sigma$ be a compact saddle surface in $\RR^3$ with boundary line $\gamma$.
Denote by $\hat \Sigma$ and $\hat\gamma$ the orthogonal projections of $\Sigma$ and $\gamma$ to the $(x,y)$-plane in $\RR^3$.
Assume that $\hat\gamma$ lies outside of $\Omega$.
Show that each connected component of $\Omega\backslash \hat\Sigma$ is convex. 
\end{thm}


\parit{Hint:} 
Assume contrary; that is, for some connected component $\Theta$ there is a line $\ell$ that goes in $\Theta$, out and in again.
Observe that the same argument as in the lens lemma (\ref{lem:lens}) shows that the boundary of $\Theta$ has a locally supporting circle from inside.
Take the cylinder $C$ over this circle, observe that $C$ is supports $\Sigma$ and apply \ref{cor:surf-support}.

\begin{thm}{Lemma}
There is no complete strictly saddle surface in the cylinder $y^2+z^2\le R^2, z\ge 0$ that has boundary line on the $(x,y)$-plane.
\end{thm}

\parit{Proof.}
Assume contrary; let $\Sigma$ be such a surface.
Consider the projection $\hat\Sigma$ of $\Sigma$ to $(x,z)$-plane;
it lies on distance at most $R$ from $x$-axis.

Note that by \ref{ex:saddle-projection} the complement of $\hat\Sigma$ in the upper half-plane is convex.
It follows that $\hat\Sigma$ is a strip described by $0\le z\le r$ for some $r>0$.
Whence the plane $z=r$ supports $\Sigma$ at some point $p$. 
By \ref{cor:surf-support}, $K(p)_\Sigma\ge0$ --- a contradiction.
\qeds




























\begin{thm}{Exercise}\label{ex:signed-distance}
Let $\gamma$ be a simple smooth regular closed plane curve.
Assume it is parametrized so that the bounded region lies on the left from $\gamma$.
Given a point $p\in\RR^2$ denote by $f(p)$ the \emph{signed distance} to $\gamma$;
that is, $|f(p)|$ is the distance to $\gamma$ and $f(p)<0$ if $p$ is surrounded by $\gamma$; otherwise $f(p)\ge0$.
Suppose $\tau,\nu$ denotes the Frenet frame of $\gamma$.

\begin{enumerate}[(a)]
\item\label{ex:signed-distance:a} Show that there is $\eps>0$ such that if $|x|<\eps$, then $f(\gamma(t)+x\cdot \nu(t))=x$.
\item Use part (\ref{ex:signed-distance:a}) to show that $f$ is smooth in the $\eps$-neighborhood of $\gamma$. 
\item Show that for any $p=\gamma(t)$, the Laplacian $\Delta_p f$ equals to the signed curvature of $\gamma$ at $p$.
\end{enumerate}

\end{thm}















\section*{DNA inequality*}

Recall that curvature of a spherical curve is at least $1$
(Exercise~\ref{ex:curvature-of-spherical-curve}).
In particular the length of spherical curve can not exceed its total curvature.
The following theorem shows that the same inequality holds for \emph{closed} curves in a unit ball.

\begin{thm}{Theorem}\label{thm:DNA}
Let $\gamma$ be a nontrivial closed curve that lies in a unit ball.
Then 
\[\tc\gamma\ge \length\gamma.\]

\end{thm}

In the proof we use \ref{def:total-curv-poly} to define for the total curvature;
according to \ref{thm:total-curvature=}, it is more general than the smooth definition on page \pageref{page:total curvature of:smooth-def}.

\parit{Proof.}
We will show that 
\[\tc\gamma> \length\gamma.\]
for any closed polygonal line $\gamma=p_1\dots p_{n}$ in a unit ball.
It implies the theorem since in any nontrivial closed curve we can inscribe a closed polygonal line with arbitrary close total curvature and length.

The indexes are taken modulo $n$, in particular $p_{n}=p_0$, $p_{n+1}=p_1$ and so on.
Denote by $\theta_i$ the external angle of $\gamma$ at $p_i$;
that is,
\[\theta_i=\pi-\measuredangle p_{i-1}p_ip_{i+1}.\]

Denote by $o$ the center of the ball.
Consider a sequence of $n+1$ plane triangles
\begin{align*}
\triangle q_0s_0q_1
&\cong 
\triangle p_0op_1,
\\
\triangle q_1s_1q_2
&\cong 
\triangle p_1op_2,
\\
&\dots
\\
\triangle q_{n}s_nq_{n+1}
&\cong 
\triangle p_nop_{n+1},
\end{align*}
such that the points $q_0,q_1\dots$ lie on one line in that order and all the points $s_0,\dots,s_n$ lie on one side from this line.

\begin{figure}[h!]
\vskip-0mm
\centering
\includegraphics{mppics/pic-16}
\vskip0mm
\end{figure}

Since $p_0=p_n$ and $p_1=p_{n+1}$, we have that
\[[q_{n}s_nq_{n+1}]\cong 
[p_nop_{n+1}]=[p_0op_1]\cong[q_{0}s_0q_1],\]
so $s_0q_0q_ns_n$ is a parallelogram.
Therefore
\begin{align*}
|s_0-s_1|+\dots+|s_{n-1}-s_n|
&\ge|s_n-s_0|=
\\
&=|q_0-q_n|=
\\
&=|p_0-p_1|+\dots+|p_{n-1}-p_n|
\\
&=\length \gamma.
\end{align*}

Note that 
\begin{align*}
\theta_i&=\pi-\measuredangle p_{i-1}p_ip_{i+1}\ge
\\
&\ge\pi-\measuredangle p_{i-1}p_io-\measuredangle op_ip_{i+1}=
\\
&=\pi-\measuredangle q_{i-1}q_is_{i-1}-\measuredangle s_iq_iq_{i+1}=
\\
&=\measuredangle s_{i-1}q_is_i>
\\
&>|s_{i-1}-s_i|;
\end{align*}
the last inequality follows since $|q_i-s_{i-1}|=|q_i-s_i|=|p_i-o|\le 1$.
That is, 
\[\theta_i>|s_{i-1}-s_i|\]
for each $i$.

It follows that
\begin{align*}
\tc \gamma
&=\theta_1+\dots+\theta_n>
\\
&> |s_{0}-s_1|+\dots |s_{n-1}-s_n|\ge 
\\
&\ge\length \gamma.
\end{align*}
Hence the result.
\qeds

This theorem was proved by Don Chakerian \cite{chakerian};
for plane curves it was prved earlier by Istv\'{a}n F\'{a}ry \cite{fary-DNA}.
Few proofs of this theorem are discussed by Serge Tabachnikov~\cite{tabachnikov}.
He also conjectured the following closely related statement:

\begin{thm}{Theorem}
Suppose a closed regular smooth curve $\gamma$ lies in a convex figure with the perimeter $2\cdot \pi$.
Then 
\[\tc\gamma\ge \length\gamma.\]

\end{thm}

It was proved by Jeffrey Lagarias and Thomas Richardson \cite{lagarias-richardso}; latter a simpler proof was given by Alexander Nazarov and Fedor Petrov~\cite{nazarov-petrov}.
The proof is annoyingly difficult; we do not present it here.












\section*{Complex coordinates}

It is often convenient to use complex coordinate on the plane,
it packs two coordinates $(x,y)$ in one complex number $z=x+i\cdot y$.

{

\begin{wrapfigure}{r}{20 mm}
\vskip-0mm
\centering
\includegraphics{mppics/pic-58}
\vskip0mm
\end{wrapfigure}

In particular any vector $w$ in $\RR^2$ can be regarded as a complex number.
Note that $i\cdot w$ is the vector obtained from $w$ by the counterclockwise rotation by $\tfrac\pi2$.

}































The same formula holds for any closed curve, if one use \emoh{signed area} surrounded by curve;
that is we count area surrounded by curve with multiplicity --- for each region cut by the curve from the sphere we count how many times the curve goes around it counterclockwise and multiply the area of the region by this number.
In order to be defined correctly we need to specify a pole on the spherelying and stating that its region has zero multiplicity. 
When you cross the curve the mulitplicity changes by $\pm1$; it increases is the curve cross the ways from left to right and decreasing otherwise.

Note that for the Gauss curvature of unit sphere is identically 1
Therefore 
\[\area\Delta=\int_\Delta G.\]
That is, \ref{eq:sphere-gauss-bonnet} is Gauss--Bonnet formula for spherical triangle.
Applying this formula for each triangle in a  triangulation of polygon and summing up, we get Gauss--Bonnet formula for arbitrary spherical polygon.

Since any smooth simple closed curve can be approximated by a polygon, we get that the 





Let $\Sigma$ be a smooth regular oriented surface and $\nu\:\Sigma\to \mathbb{S}^2$ be its Gauss map.
Assume $\alpha$ is a smooth unit-speed curve in $\Sigma$. 
Then for any $t$ the vectors $\nu(t)=\nu(\gamma(t))$ and the velocity vector $\tau(t)\gamma'(t)$ are unit vectos that are normal to each other.
Denote by $\mu(t)$ the unit vector that is normal to both $\nu(t)$ and $\tau(t)$ such that for any $t$ the triple $(\tau(t),\mu(t),\nu(t)$ is an oriented basis (say $\mu(t)=[\nu(t),\tau(t)]$, where $[{*},{*}]$ denotes the vector product).

Since $\alpha$ is unit-speed, the acceleration $\alpha''(t)\perp\tau(t)$;
therefore at any parameter value $t$, we have
\[\alpha''(t)=k_n(t)\cdot \nu(t)+k_g(t)\cdot \mu(t),\]
for some real numbers $k_n$ and $k_g$.
The numbers $k_n(t)$ and $k_g(t)$ are called \emph{normal} and \emph{geodesic curvature} of $\alpha$ at $t$ respectively.
The geodesic curvature vanishes if and only if $\alpha$ is a geodesic; 
it measures how much a given curve diverges from a geodesic.

Locally $\alpha$ cuts $\Sigma$ into two parts, left and right.
The left part is the one that lies in the direction of $\mu(t)$ and the opposite is right.


\begin{thm}{Theorem}
Suppose $\Delta$ is a disc in a surfaces $\Sigma$ 
bounded by a simple closed regular curve $\alpha$.
Assume that $\alpha$ is oriented such that $\Delta$ lies on the left from $\alpha$.
Then
\[\int_\Delta G+\int_\alpha k_g=2\cdot \pi,\]
where $G$ denotes the Gauss curvature of $\Sigma$ and
$k_g$ denotes the geodesic curvature of $\alpha$.

\end{thm}

We will give an informal proof of this formula based on the bike wheel interpretation described in the previous section.
We suppose that it is intuitively clear that moving the axis of the wheel without changing its direction does not change the direction of the wheel's spokes.
More precisely, we need the following:

\begin{thm}{Claim}
Assume we keep the axis of a non-spinning bike wheel and perform the following two experiments:

\begin{enumerate}[(i)]
\item We moved it around and bring it back to the original position. 
As a result the wheel might rotate by some angle; let us measure this angle.

\item we move the direction of the axis the same as before without moving the center of the wheel and again measure the angle of rotation.
\end{enumerate}

Then the resulting angle in these two experiments is the same. 
\end{thm}

\section{Gauss--Bonnet formula}

\begin{thm}{Theorem}
Suppose $\gamma$ is a simple broken geodesic line that cuts a disc $\Delta$ from the surface $\Sigma$.
Assume that $\gamma$ is oriented so that $\Delta$ lies on the left from $\Sigma$.
Then 
\[\tgc\gamma+\iint_\Delta G=2\cdot \pi,\]
where $G$ denotes the Gauss curvature of $\Sigma$.
\end{thm}

Note that if $\Sigma$ is a plane, then geodesic in $\Sigma$ are formed by line segments.
In this case the statement of theorem follows from Exercise~\ref{ex:pm2pi}.
In the next section we will prove the formula for the unit sphere;
latter we will use it in the sketch of the proof of the general case.









\section{Gauss map}

Let $\Sigma$ be a surface in the Euclidean space.
Given a point $p\in\Sigma$ consider a unit vector $\nu(p)$ that is normal to the tangent plane $\T_p$. 
The unit normal vector $\nu(p)$ is defined uniquely up to sign at each point $p\in \Sigma$.
If the choice of the sign is made so that the map $\nu\:p\mapsto \nu(p)$ is continuous,
then the map $\nu$ is called \emph{Gauss map}.

The Gauss map sends the surface $\Sigma$ to the unit sphere $\mathbb{S}^2$.
It can be always defined locally (that is in a neighborhood of a point);
if it can be defined globally, then the surface $\Sigma$ is called \emph{oriented}.

M\"obius band gives an example of nonoriented surface with boundary.
Klein bottle is an example of closed immerersed surface which is not oriented.
Any closed embedded surface is oriented since one can choose the  direction pointing outside of the region bounded by the surface.

Fix a point $p\in \Sigma$ and a smooth curve $\alpha$ in $\Sigma$ that starts at $p$;
that is $\alpha(0)=p$.
Set  $v=\alpha'(0)$ and $w=(\nu\circ\alpha)'(0)$.
Note that $v$ lie in $T_p$ --- the tangent plane at $p$.

Further $w\in \T_p$ as well.
Indeed $\nu\circ\alpha$ is a curve that starts at $\nu(p)$;
so $w=(\nu\circ\alpha)'(0)$ lies in the tangent plane $\T_{\nu(p)}\mathbb{S}^2$.
But at $\nu(p)$, the sphere $\mathbb{S}^2$ has normal $\nu(p)$ and therefore its tangent plane $\T_{\nu(p)}\mathbb{S}^2$ is parallel $T_p$, so $\T_{\nu(p)}\mathbb{S}^2$ and $T_p$ are identical as vector spaces.

Further note that since Gauss map is smooth,
the vector $w$ depends only on $v$;
that is if we choose a different curve $\alpha$ such that $v=\alpha'(0)$ we will get the same value $w=(\nu\circ\alpha)'(0)$.
The map $s_p\:v\mapsto w$ is called shape operator;
it maps the tangent plane $\T_p$ to itself.

It is straightforwardto check that the eigenvalues of $s_p$ are the principle curvatures at $p$; it agrees with the definition given above if the normal vector $\nu$ is chosen to point down in the specital graph representation $z=f(x,y)$.







The Gauss map can be defined (globally) if and only if the surface is orientable, in which case its degree is half the Euler characteristic. The Gauss map can always be defined locally (i.e. on a small piece of the surface). The Jacobian determinant of the Gauss map is equal to Gaussian curvature, and the differential of the Gauss map is called the shape operator. 


















\section{Comments}


\section{Tennis ball theorem}

Suppose that a curve $\alpha$ runs on the unit sphere 
\[\mathbb{S}^2=\set{(x,y,z)\in\RR^3}{x^2+y^2+z^2=1}.\] 
Let us denote by $\alpha''(t)^\top$ the projection of vector $\alpha''(t)$ to the tanget plane of the sphere at $\alpha(t)$.

Assume $\alpha$ is a unit-speed curve.
By Proposition~\ref{prop:a'-pertp-a''}, $\alpha''(t)\perp\alpha'(t)$ and therefore $\alpha''(t)^\top\perp\alpha'(t)$.
If $w(t)$ denotes by the vector $\alpha'(t)$ rotated
counterclockwise by angle $\tfrac\pi2$ in the tangent plane to the sphere at $\alpha(t)$, then 
\[\alpha''(t)^\top=\kappa_\alpha(t)\cdot w(t)\]
for some real number $\kappa(t)$;
this number is called \emph{signed geodesic curvature of $\alpha$ at $t$}.

The geodesic curvature measures how $\alpha$ diverges from an equator --- the most straight curve on the curved $\mathbb{S}^2$.
TBC














The following diagram shows a simple curve with positive curvature that intersects a line at points $p_1,\dots p_11$; these points appear on the curve in the same order, and on the line  in the order $p_5,p_3,p_1,p_2,p_4,p_7,p_9,p_{11},p_{10},p_8,p_6$; 
The following Exercise is about this pattern.

\begin{figure}[h!]
\vskip-0mm
\centering
\includegraphics{mppics/pic-29}
\vskip0mm
\end{figure}

\begin{thm}{Exercise}
Suppose $\alpha$ is a simple smooth regular curve in the plane that crosses a line $\ell$ at the points $p_1,p_2,\dots p_n$.
Assume that the points $p_1,p_2,\dots p_n$ appear on $\alpha$ in the same order.
Then the order on $\ell$ can be obtained from the following sequence 
\[p_1,p_3,\dots,p_4 ,p_2\]
by shift last $k$ elements to the beginning, reverting the order in the tail starting from $(2\cdot k+1)$-th element and reverting the order of the obtained sequence if necessary.
\end{thm}





\parit{Proof; only-if part.} 
Without loss of generality we can assume that $D$ lies on that $D$ lies on the left side of $\alpha$.
If so,
the only-if part will follow if the curvature of $\alpha$ is nonnegative.

Assume contrary, that is $\kappa(t_0)<0$ for some $t_0$.
Then the tangent line $\ell$ strictly supports $\alpha$ at $t_0$ from left;
that is, $\ell$ pass thru $\\alpha(t_0)$ and runs in the interior of $D$ shortly before and after.
A line $\ell'\parallel \ell$ which lies slightly to the right from $\ell$ has a line segment with the ends in the interior of $D$ that crosses $\alpha$.
That is $D$ is not convex.

\parit{If part.} assume $D$ is not convex; that is there are two points $p$ and $q$ in $D$ such that the line segment $[pq]$ does not lie completely in $D$.
Without loss of generality we may assume that $p$ and $q$ lie in the interior of $D$.

Let $\ell$ be the line thru $p$ and $q$.
Note that $\alpha$ changes the side of $\ell$ at least 4 times;
moreover one can choose crossing points $a$, $b$, $c$ and $d$ that appear in the same order on the line and $\alpha$.













The total signed curvature of a smooth unit-speed curve $\alpha\:[a,b]\z\to\RR^2$ can be defined as the integral
\[\tsc\alpha=\int_{[a,b]}\kappa(t)\cdot dt.\]

It is straightforward to show that for sufficienlty fine partition of closed unit-speed curve $\alpha$ the inscribed polygonal line has the same total signed curvature.
Therefore by exercises~\ref{ex:2kpi} and \ref{ex:pm2pi},
we get that the total signed curvature of any closed simple curve is mutiple of $2\cdot\pi$ and if the curve is simple, it has to be $\pm2\cdot\pi$.

\begin{thm}{Exercise}\label{ex:curvature-of-circle}
Find a unit-speed parametrization of the circle $\gamma$ or radius $R$ centered at the point $p$;
it is defined as the set 
\[\gamma=\set{v\in\RR^2}{|p-v|=R}.\]
Show that the signed curvature of this circle is $\pm \tfrac1R$.
\end{thm}
