





\section*{Curve in a surface}


\begin{thm}{Proposition}\label{prop:gamma''=II}
Suppose $\gamma$ is a smooth curve in a smooth oriented surface $\Sigma$ with a unit normal field $\Norm$.
Then, the following identity holds for any time parameter $t$:
\[\langle \gamma''(t),\Norm\circ\gamma(t)\rangle=-\langle \gamma'(t),(\Norm\circ\gamma)'(t)\rangle=\langle S_{\gamma(t)}(\gamma'(t)),\gamma'(t)\rangle.\]

\end{thm}

\parit{Proof.} 
Fix a parameter value $t_0$; set $p=\gamma(t_0)$, $v=\gamma'(t_0)$ and $a\z=\gamma''(t_0)$;
so we need to show that
\[\langle a,\Norm_{p}\rangle=\II_p(v,v).\eqlbl{a-nu-II}\]
Let $z=f(x,y)$ be the local representaion of $\Sigma$ in the tangent-normal coordinates at $p$;
we assume that $\Norm$ points in the direction of~$\Norm_p$.

Without loss of generality may assume that $\gamma$ runs in the graph $z=f(x,y)$;
so 
\[\gamma(t)=\left(x(t),y(t),f(x(t),y(t))\right).\]
Then
\begin{align*}
\gamma'&=(x',y',\tfrac{\partial f}{\partial x}\cdot x'+\tfrac{\partial f}{\partial y}\cdot y');
\\
\gamma''
&=
{\small(x'',
y'',
 \tfrac{\partial^2 f}{\partial x^2}\cdot (x')^2
+
2\cdot \tfrac{\partial^2 f}{\partial x\partial y}\cdot x'\cdot y'
+
\tfrac{\partial^2 f}{\partial y^2}\cdot (y')^2
+
\tfrac{\partial f}{\partial x}\cdot x''
+
\tfrac{\partial f}{\partial y}\cdot y'')}.
\end{align*}

Recall that $p=\gamma(t_0)=(0,0,0)$ and
\begin{align*}
f(0,0)&=0,
&
\tfrac{\partial f}{\partial x}(0,0)&=0,
&
\tfrac{\partial f}{\partial y}(0,0)&=0.
\end{align*}

Therefore 
\begin{align*}
v&=\left(x',y',0\right)(t_0);
\\
a&=\left(x'',y'',
\tfrac{\partial^2 f}{\partial x^2}\cdot (x')^2
+
2\cdot \tfrac{\partial^2 f}{\partial x\partial y}\cdot x'\cdot y'
+
\tfrac{\partial^2 f}{\partial y^2}\cdot (y')^2\right)(t_0).
\end{align*}

Note that 
\[\II_p(v,v)=\left(\tfrac{\partial^2 f}{\partial x^2}\cdot (x')^2
+
2\cdot \tfrac{\partial^2 f}{\partial x\partial y}\cdot x'\cdot y'
+
\tfrac{\partial^2 f}{\partial y^2}\cdot (y')^2\right)(t_0);\]
that is, the $z$-coordinate of the acceleration $a$ equals $\II_p(v,v)$ which is equivalent to~\ref{a-nu-II}.
\qeds



















\chapter{tmp}

\section{Shape}

The linear operator $S\:\T_p\to \T_p$ defined by the matrix multiplication
\[S\:(\begin{smallmatrix}
x\\y
\end{smallmatrix})
\mapsto
M_p\cdot(\begin{smallmatrix}
x\\y
\end{smallmatrix})\] is called \emph{shape operator} of $\Sigma$ at $p$.

For a vector $\vec w\in\T_p$, we use notation $S(\vec w)$ when it is clear from the context which base point $p$ and which surface we are working with;
otherwise we may use notations 
\[S_p(\vec w)\quad\text{or even}\quad S_p(\vec w)_\Sigma.\]













\section*{Shape operator}

Let $z=f(x,y)$ be a local representation of a smooth surface $\Sigma$ in the tangent-normal coordinates at a point $p$.
Let $M_p$ be the Hessian matrix.
Recall that $(x,y)$-plane coincides with the tangent plane $\T_p$.

The operator $S_p\:\T_p\to \T_p$ defined by the matrix multiplication
\[S_p\:(\begin{smallmatrix}
x\\y
\end{smallmatrix})
\mapsto
M_p\cdot(\begin{smallmatrix}
x\\y
\end{smallmatrix})\] is called \emph{shape operator} of $\Sigma$ at $p$.
The following proposition implies that $S_p$ does not depend on the choice of $(x,y)$-coordinates on $\T_p$.





Given two vectors $\vec v=(\begin{smallmatrix}a\\b
\end{smallmatrix})$ and $\vec w=(\begin{smallmatrix}c\\d
\end{smallmatrix})$ in the $(x,y)$-plane, consider the value 
\[\II_p(\vec v,\vec w)\df(D_{\vec w}D_{\vec v}f)(0,0),\]
where $D$ denotes the directional derivative.
The function $(v,w)\z\mapsto \II_p(v,w)$ is called the \emph{second fundamental form} at $p$;\label{page:second fundamental form}
it takes two tangent vectors $v$ and $w$ at $p$ and spits out the real number $\II_p(v,w)$.

The second fundamental form can be written in terms of the Hessian matrix.
Indeed if 
$w=(\begin{smallmatrix}a\\b
\end{smallmatrix})$ 
and 
$v=(\begin{smallmatrix}c\\d
\end{smallmatrix})$, then 
\[D_w=a\cdot \tfrac\partial{\partial x}+b\cdot \tfrac\partial{\partial y}\quad\text{and}\quad D_v=c\cdot \tfrac\partial{\partial x}+d\cdot \tfrac\partial{\partial y}.\]
Therefore 
\[\begin{aligned}
\II_p(w,v)&\df(D_wD_vf)(0,0)=
\\
&=a\cdot c\cdot \ell+(a\cdot d+ b\cdot c)\cdot m +b\cdot d\cdot n=
\\
&=\langle M_p\cdot w,v\rangle=
\\
&=\langle M_p\cdot v,w\rangle.
\end{aligned}
\eqlbl{eq:DwDv}\]
Note that from \ref{eq:DwDv} it follows that $\II_p$ is symmetric; that is,
\[\II_p(v,w)=\II_p(w,v)\]
for any two tangent vectors $v,w\in \T_p$.





\parit{Proof.}
Assume an oriented surface $\Sigma$ is written locally as a graph $z\z=f(x,y)$ in the tangent-normal coordinates at $p\in\Sigma$.
As usual we assume that the normal vector $\Norm_p$ points in the direction of the $z$-axis,
in this case the normal vector at any point of the graph points up; that is, its $z$-coordinate  is positive.

Consider the corresponding chart  of $\Sigma$:
\[s(x,y)\z=(x,y,f(x,y)).\]
Denote by $\Norm(x,y)$ the unit normal vector at $s(x,y)$; it is a shortcut notation for $\Norm_{s(x,y)}$.

Note that $\tfrac{\partial s}{\partial x}(0,0)=(1,0,0)$ and $\tfrac{\partial s}{\partial y}(0,0)=(0,1,0)$.
For a tangent vector 
\[v=(a\cdot\tfrac{\partial s}{\partial x}+b\cdot\tfrac{\partial s}{\partial y})(0,0) =(a,b,0)\in \T_p\]
we have that
\[
\begin{aligned}
S_p(v)&=-D_v\Norm(0,0)=
\\
&=-(a\cdot \tfrac{\partial \Norm}{\partial x}+b\cdot \tfrac{\partial \Norm}{\partial y})(0,0),
\end{aligned}
\eqlbl{eq:S=D}
\]
where $D_v$ denotes the directional derivative along a vector $v$ in the $(x,y)$-plane which is $\T_p$.

Indeed, the first equality follows from \ref{eq:shape} and the definition of differential \ref{eq:differenital} applied for the curve $\gamma(t)=(a\cdot t,b\cdot t, f(a\cdot t,b\cdot t))$ at $t=0$ and the second follow since
$D_v=a\cdot \tfrac{\partial }{\partial x}+b\cdot \tfrac{\partial }{\partial y}$.

Taking partial derivatives of $\langle\Norm,\Norm\rangle=1$, we get that 
\begin{align*}
0&=\tfrac{\partial}{\partial x} \langle\Norm,\Norm\rangle=
\\
&=2\cdot\langle\tfrac{\partial\Norm}{\partial x},\Norm\rangle.
\end{align*}
That is, $\langle\tfrac{\partial\Norm}{\partial x},\Norm\rangle=0$ and the same way we get $\langle\tfrac{\partial\Norm}{\partial y},\Norm\rangle=0$.
By \ref{eq:S=D} it follows that $S_p(v)\perp \Norm_p$, or equivalently $S_p(v)\in\T_p$ for any $v\in \T_p$.

Further, since $\tfrac{\partial s}{\partial x}, \tfrac{\partial s}{\partial y}\in\T_{s(x,y)}\Sigma$
and $\Norm(x,y)\perp\T_{s(x,y)}\Sigma$,
we have that
\[\langle \Norm,\tfrac{\partial s}{\partial x}\rangle\equiv 0
\quad\text{and}\quad
\langle \Norm,\tfrac{\partial s}{\partial y}\rangle\equiv 0.\]
Taking a derivative of these identities, we get that
\[\begin{aligned}
\langle \tfrac{\partial\Norm}{\partial x},\tfrac{\partial s}{\partial x}\rangle+\langle \Norm,\tfrac{\partial^2 s}{\partial x^2}\rangle&\equiv 0,
\\
\langle \tfrac{\partial\Norm}{\partial y},\tfrac{\partial s}{\partial x}\rangle+\langle \Norm,\tfrac{\partial^2 s}{\partial y\partial x}\rangle&\equiv 0,
\\
\langle \tfrac{\partial\Norm}{\partial x},\tfrac{\partial s}{\partial y}\rangle+\langle \Norm,\tfrac{\partial^2 s}{\partial x\partial y}\rangle&\equiv 0,
\\
\langle \tfrac{\partial\Norm}{\partial y},\tfrac{\partial s}{\partial y}\rangle+\langle \Norm,\tfrac{\partial^2 s}{\partial y^2}\rangle&\equiv 0,
\end{aligned}
\]

Fix two vectors $v=(a,b,0)$ and $w=(c,d,0)$ in $\T_p$ (which is the $(x,y)$-plane).
Since $\Norm(0,0)=(0,0,1)$ we get 
\[f(x,y)\equiv\langle\Norm(0,0),s(x,y)\rangle.\]
Therefore by \ref{eq:S=D}, \ref{eq:shape=second} and \ref{eq:DwDv} we get that 
\begin{align*}
\langle S_p(v),w\rangle 
&=-\langle  a\cdot \tfrac{\partial \Norm}{\partial x}+b\cdot \tfrac{\partial \Norm}{\partial y},c\cdot \tfrac{\partial s}{\partial x}+d\cdot \tfrac{\partial s}{\partial y}\rangle(0,0)=
\\
&=-\bigl(a\cdot c\cdot\langle \tfrac{\partial \Norm}{\partial x},\tfrac{\partial s}{\partial x}\rangle 
+a\cdot d\cdot\langle \tfrac{\partial \Norm}{\partial x},\tfrac{\partial s}{\partial y}\rangle+
\\&\quad
+b\cdot c\cdot\langle \tfrac{\partial \Norm}{\partial y},\tfrac{\partial s}{\partial x}\rangle
+b\cdot d\cdot\langle \tfrac{\partial \Norm}{\partial y},\tfrac{\partial s}{\partial y}\rangle\bigr)(0,0)=
\\
&=\bigl(a\cdot c\cdot\langle \Norm,\tfrac{\partial^2 s}{\partial x^2}\rangle 
+a\cdot d\cdot\langle \Norm,\tfrac{\partial^2 s}{\partial x\partial y}\rangle+
\\&\quad
+b\cdot c\cdot\langle \Norm,\tfrac{\partial^2 s}{\partial y\partial x}\rangle
+b\cdot d\cdot\langle \Norm,\tfrac{\partial^2 s}{\partial y^2}\rangle\bigr)(0,0)=
\\
&=\bigl(a\cdot c\cdot\tfrac{\partial^2 f}{\partial x^2} 
+a\cdot d\cdot\tfrac{\partial^2 f}{\partial x\partial y}+
\\&\quad
+b\cdot c\cdot\tfrac{\partial^2 f}{\partial y\partial x}
+b\cdot d\cdot\tfrac{\partial^2 f}{\partial y^2}\bigr)(0,0)=
\\
&=\II_p(v,w).
\end{align*}
It remains to apply \ref{eq:II=II} on page \pageref{eq:II=II}.
\qeds
















\section*{???}

Denote by $\nu(x,y)$ the normal vector at the point $(x,y,f(x,y))\in \Sigma$, so we may think that $\nu$ is defined on the tangent plane.

\begin{thm}{Claim}
Let $\Sigma$ be a smooth surface with unit normal field $\Norm$.
Suppose $p\in \Sigma$ and $S\:\T_p\to\T_p$ is the shape operator at $p$.
Then 
\[S(\vec w)=-D_{\vec w}\nu\]
for any $\vec w\in \T_p$.
\end{thm}

















Then 
\begin{align*}
\langle\tfrac{\partial^2 s}{\partial^2 u},\Norm\circ s\rangle
&=
-\langle\tfrac{\partial s}{\partial u},\tfrac{\partial \Norm\circ s}{\partial u}\rangle,
\\
\langle\tfrac{\partial^2 s}{\partial u\partial v},\Norm\circ s\rangle
&=
-\langle\tfrac{\partial s}{\partial v},\tfrac{\partial \Norm\circ s}{\partial u}\rangle=
\\
=\langle\tfrac{\partial^2 s}{\partial v\partial u},\Norm\circ s\rangle
&=
-\langle\tfrac{\partial s}{\partial u},\tfrac{\partial \Norm\circ s}{\partial v}\rangle,
\\
\langle\tfrac{\partial^2 s}{\partial^2 v},\Norm\circ s\rangle
&=
-\langle\tfrac{\partial s}{\partial v},\tfrac{\partial \Norm\circ s}{\partial v}\rangle.
\end{align*}




%???product rule???

The proof is based on three identities $\langle\Norm,\Norm\rangle=1$ and $\langle\Norm,\vec w\rangle=0$ for any tangent vector $\vec w$ and the unit normal vector $\Norm$ with the same base point.
It is sufficient to take directional derivatives of these two identity and understand the result.
In the proof we use the following product rule:
\[D_{\vec w}\langle\vec u,\vec v\rangle
=
\langle D_{\vec w}\vec u,\vec v\rangle+\langle\vec u,D_{\vec w}\vec v\rangle.\]
which easely follows from the standard product rule and the definition of directional derivative.

\parit{Proof.}
Since $\Norm$ is a unit normal field, we have the identity
\[1=\langle\Norm,\Norm\rangle.\]
Taking the directional derivative along $\vec w$, we get
\begin{align*}0&=D_{\vec w}\langle\Norm,\Norm\rangle=
\\
&=2\cdot \langle D_{\vec w}\Norm,\Norm\rangle.
\end{align*}
That is, $D_{\vec w}\Norm$ lies in the orthogonal complement of $\Norm_p$;
or, equivalently, $D_{\vec w}\Norm\in\T_p$.

Note that both sides of the identity \ref{eq:shape=-dNorm} are linear in $\vec w$.
Therefore it is sufficient to check the identity for two basis vectors $\vec u$ and $\vec v$ of $\T_p$.
That is, it is sufficient to establish the following two identities:
\begin{align*}
S(\vec u)&= -D_{\vec u}\Norm,
&
S(\vec v)&= -D_{\vec v}\Norm.
\end{align*}
Further, since $D_{\vec u}\Norm,D_{\vec v}\Norm\in\T_p$, to prove the two vector indentities, it sufficient to prove the following four real identities:
\[\begin{aligned}
\langle S(\vec u),\vec u\rangle &= -\langle D_{\vec u}\Norm,\vec u\rangle,
&
\langle S(\vec v),\vec u\rangle &= -\langle D_{\vec v}\Norm,\vec u\rangle,
\\
\langle S(\vec u),\vec v\rangle &= -\langle D_{\vec u}\Norm,\vec v\rangle,
&
\langle S(\vec v),\vec v\rangle &= -\langle D_{\vec v}\Norm,\vec v\rangle.
\end{aligned}
\eqlbl{eq:<S(u),v>=<-Du,v>}
\]

Let $z=f(x,y)$ be a local description of $\Sigma$ in the tangent-normal coordinates at $p$.
Note that 
\[s(u,v)=(u,v,f(u,v))\]
describes a chart of $\Sigma$ at $p$.
Let us prove the indentities \ref{eq:<S(u),v>=<-Du,v>} for the basis $\vec u=\tfrac{\partial s}{\partial u}$ and $\vec v=\tfrac{\partial s}{\partial v}$.
Note that 
\[f=\langle s, \vec k\rangle=\langle s, \nu(p)\rangle.\]???


Note that the vector fields $\vec u=\tfrac{\partial s}{\partial u}$ and $\vec v=\tfrac{\partial s}{\partial v}$ are tangent to $\Sigma$  and therefore they are orthogonal to $\Norm$;
that is, we have two identities:
\[\begin{aligned}
0&=\langle \Norm,\vec u\rangle=\langle \Norm,\tfrac{\partial s}{\partial u}\rangle,
\\
0&=\langle \Norm,\vec v\rangle=\langle \Norm,\tfrac{\partial s}{\partial v}\rangle.
\end{aligned}
\eqlbl{eq:<Norm,uv>}\]
Applying the product rule for the directional derivative of the first identity in \ref{eq:<Norm,uv>} along $\vec u$, we get that
\begin{align*}
0&=D_{\vec u}\langle \Norm,\vec u\rangle=
\\
&=\langle D_{\vec v}\Norm,\vec u\rangle
+
\langle \Norm,D_{\vec v}\vec u\rangle
\end{align*}
Note that $\langle\nu(p),s(u,v)\rangle=f(u,v)$.
Therefore evaluating the last expression at $p$, we get
\[\langle D_{\vec v}\Norm,\tfrac{\partial s}{\partial u}\rangle+\tfrac{\partial^2 f}{\partial v\partial u}=0.\]



Since $\tfrac{\partial^2 s}{\partial u\partial v}=\tfrac{\partial^2 s}{\partial v\partial u}$, we get
\[\langle S({\vec u}),\vec v\rangle=\langle {\vec u},S(\vec v)\rangle;\]
that is, the identity \ref{eq:S-self-adjoint} for two vectors $\vec u=\tfrac{\partial s}{\partial u}$ and $\vec v=\tfrac{\partial s}{\partial v}$.
Evidently the identity \ref{eq:S-self-adjoint} holds if $\vec u=\vec v$.

It follows that the identity \ref{eq:S-self-adjoint} holds for any pair of vectors in a basis of $\T_p$; that is we have the following four identities:
\begin{align*}
 \langle S({\vec u}),\vec u\rangle&=\langle {\vec u},S(\vec u)\rangle,
 &\langle S({\vec u}),\vec v\rangle&=\langle {\vec u},S(\vec v)\rangle
 \\
 \langle S({\vec v}),\vec u\rangle&=\langle {\vec v},S(\vec u)\rangle,
 &\langle S({\vec v}),\vec v\rangle&=\langle {\vec v},S(\vec v)\rangle.
\end{align*}

Since both sides of this identiy \ref{eq:S-self-adjoint} are linear in $\vec u$ and $\vec v$,
the latter is sufficient to conclude that \ref{eq:S-self-adjoint} for any pair of tangent vectors at $p$.
Let us present the formal calculations for two vectors $a\cdot \vec u+b\cdot\vec v$ and $c\cdot \vec u+d\cdot\vec v$:
\begin{align*} 
\langle S(a\cdot \vec u+b\cdot\vec v),c\cdot \vec u+d\cdot\vec v\rangle
&=a\cdot c\cdot\langle S({\vec u}),\vec u\rangle + a\cdot d\cdot \langle S({\vec u}),\vec v\rangle+
\\
&+b\cdot c\cdot\langle S({\vec v}),\vec u\rangle + b\cdot d\cdot \langle S({\vec v}),\vec v\rangle=
\\
&=a\cdot c\cdot\langle {\vec u},S(\vec u)\rangle + a\cdot d\cdot \langle {\vec u},S(\vec v)\rangle+
\\
&+b\cdot c\cdot\langle {\vec v},S(\vec u)\rangle + b\cdot d\cdot \langle {\vec v},S(\vec v)\rangle=
\\
&=\langle a\cdot \vec u+b\cdot\vec v,S(c\cdot \vec u+d\cdot\vec v)\rangle.
\end{align*}

The vectors $\vec u$ and $\vec v$ are tangent to $\Sigma$, we have



\qeds



Let $\Sigma$ be an smooth oriented surface in the Euclidean space;
denote by $\Norm$ its unit normal field.

Given a point $p$ on $\Sigma$ and a tangent vector $\vec w$ at $p$,
the shape operator of $\vec w$ at $p$ is defined as
\[S(\vec w)=-D_{\vec w}\Norm,\]
where $D_{\vec w}$ denotes the directional derivative defined in \ref{def:directional-derivative}.




\begin{thm}{Proposition}
Let $\Sigma$ be an smooth oriented surface and $p\in \Sigma$.
Then the shape operator $S_p$ is a linear self-adjoint operator defined on the tangent plane $\T_p$;
that is, $\vec v\mapsto S_p(\vec v)$ defines a linear map $\T_p\to \T_p$ and
\[\langle S_p(\vec u), \vec v\rangle =\langle \vec u, S_p(\vec v)\rangle\eqlbl{eq:S-self-adjoint}\]
for any two vectors $\vec u, \vec v \in \T_p$.
\end{thm}

\parit{Proof.}
The linearity follow from \ref{ex:linerity}.

Since $\Norm$ is a unit normal field, we have the identity
\[1=\langle\Norm,\Norm\rangle.\]
Taking the directional derivative along $\vec v$, we get
\begin{align*}0&=D_{\vec v}\langle\Norm,\Norm\rangle=
\\
&=2\cdot \langle D_{\vec v}\Norm,\Norm\rangle=
\\
&=-2\cdot \langle S({\vec v}),\Norm\rangle.
\end{align*}
That is, $S_p({\vec v})$ lies in the orthogonal complement of $\Norm_p$;
or, equivalently, $S_p({\vec v})\in\T_p$.

Now let us fix a chart $(u,v)\mapsto s(u,v)$ of $\Sigma$ at $p$.
Note that the vectors $\vec u=\tfrac{\partial s}{\partial u}$ and $\vec v=\tfrac{\partial s}{\partial v}$ are tangent to $\Sigma$ at their base points and therefore they are orthogonal to $\Norm$;
that is we have two identities:
\[\begin{aligned}
0&=\langle \Norm,\tfrac{\partial s}{\partial u}\rangle,
\\
0&=\langle \Norm,\tfrac{\partial s}{\partial v}\rangle.
\end{aligned}
\eqlbl{eq:<Norm,uv>}\]
Taking a directional derivative of the first identity in \ref{eq:<Norm,uv>} along $\vec v$ , we get that
\begin{align*}
0&=D_{\vec v}\langle \Norm,\tfrac{\partial s}{\partial u}\rangle=
\\
&=\langle D_{\vec v}\Norm,\tfrac{\partial s}{\partial u}\rangle
+
\langle \Norm,D_{\vec v}\tfrac{\partial s}{\partial u}\rangle=
\\
&=-\langle S({\vec v}),\vec u\rangle+\langle \Norm,\tfrac{\partial^2 s}{\partial v\partial u}\rangle.
\intertext{Analogously, taking a directional derivative of the second identity in \ref{eq:<Norm,uv>} along $\vec u$, we get that}
0&=D_{\vec u}\langle \Norm,\tfrac{\partial s}{\partial v}\rangle=
\\
&=\langle D_{\vec u}\Norm,\tfrac{\partial s}{\partial v}\rangle
+
\langle \Norm,D_{\vec u}\tfrac{\partial s}{\partial v}\rangle=
\\
&=-\langle S({\vec u}),\vec v\rangle+\langle \Norm,\tfrac{\partial^2 s}{\partial u\partial v}\rangle.
\end{align*}

Since $\tfrac{\partial^2 s}{\partial u\partial v}=\tfrac{\partial^2 s}{\partial v\partial u}$, we get
\[\langle S({\vec u}),\vec v\rangle=\langle {\vec u},S(\vec v)\rangle;\]
that is, the identity \ref{eq:S-self-adjoint} for two vectors $\vec u=\tfrac{\partial s}{\partial u}$ and $\vec v=\tfrac{\partial s}{\partial v}$.
Evidently the identity \ref{eq:S-self-adjoint} holds if $\vec u=\vec v$.

It follows that the identity \ref{eq:S-self-adjoint} holds for any pair of vectors in a basis of $\T_p$; that is we have the following four identities:
\begin{align*}
 \langle S({\vec u}),\vec u\rangle&=\langle {\vec u},S(\vec u)\rangle,
 &\langle S({\vec u}),\vec v\rangle&=\langle {\vec u},S(\vec v)\rangle
 \\
 \langle S({\vec v}),\vec u\rangle&=\langle {\vec v},S(\vec u)\rangle,
 &\langle S({\vec v}),\vec v\rangle&=\langle {\vec v},S(\vec v)\rangle.
\end{align*}

Since both sides of this identiy \ref{eq:S-self-adjoint} are linear in $\vec u$ and $\vec v$,
the latter is sufficient to conclude that \ref{eq:S-self-adjoint} for any pair of tangent vectors at $p$.
Let us present the formal calculations for two vectors $a\cdot \vec u+b\cdot\vec v$ and $c\cdot \vec u+d\cdot\vec v$:
\begin{align*} 
\langle S(a\cdot \vec u+b\cdot\vec v),c\cdot \vec u+d\cdot\vec v\rangle
&=a\cdot c\cdot\langle S({\vec u}),\vec u\rangle + a\cdot d\cdot \langle S({\vec u}),\vec v\rangle+
\\
&+b\cdot c\cdot\langle S({\vec v}),\vec u\rangle + b\cdot d\cdot \langle S({\vec v}),\vec v\rangle=
\\
&=a\cdot c\cdot\langle {\vec u},S(\vec u)\rangle + a\cdot d\cdot \langle {\vec u},S(\vec v)\rangle+
\\
&+b\cdot c\cdot\langle {\vec v},S(\vec u)\rangle + b\cdot d\cdot \langle {\vec v},S(\vec v)\rangle=
\\
&=\langle a\cdot \vec u+b\cdot\vec v,S(c\cdot \vec u+d\cdot\vec v)\rangle.
\end{align*}
\qedsf






























 On page 58, at the end of the first paragraph, you have "... its invese W → U is also continuous." Notice, "inverse" has a spelling error in the sentence. 
 
 In the second paragraph of this page, you have "However, as well as in the case of curves we will be mostly interested in smooth surfaces in the Euclidean space describe in the following section." This is missing a comma which should be placed after "curves," 
 
 and "describe" should be "described." 
 
 On page 60, in the last sentence, you have "If s and Σ as in the proposition..." which should be either "If s and Σ (are defined)/(exist) as in the proposition..." or "If we have s and Σ as in the proposition..." 
 
 In 6.4 on page 61, you have "Assume γ is a closed simple smooth regular plane curve that does not intersect x-axis. Show that surface of revolution of γ around x-axis is a smooth regular surface." This should include a "the" before both of the "x-axis" and before "surface of revolution." 
 
 On page 62, in the paragraph above Implicitly Defined Curves, you have "For example, any embedding s: S 2 → R 3 might be called topological sphere and if s is smooth and regular, then it might be called smooth sphere. (A smooth regular map s: S 2 → R 3 which is not necessary an embedding is called smooth regular immersion, so we can say that s describes a smooth immersed sphere.) Similarly an embedding s: R 2 → R 3 might be called topological plane and if s is smooth it might be called smooth plane. " This should include an "a" before both uses of "topological," as well as before "smooth regular immersion" and "smooth plane." 
 
 Also, the last sentence should have two additional commas; one after "topological plane," 
 
 and one after after "if s is smooth." 
 
 In 6.10 on page 64, you have "Show that a neighborhood of p in Σ is a graph z = f(x, y) of a smooth function f defined on an open subset in (x, y)- plane if and only if the tangent plane Tp is not a vertical plane; that is if the projection of Tp to (x, y)-plane does not degenerates to a line. " 
 This should have a "the" before both uses of "(x, y)-plane," 
 
 and "degenerates" should be "degenerate." 
 
 Another on page 64, in the third paragraph of the Normal vector and orientation section, you have "The choice of the field ν is called orientation on Σ." This should include a "the" before "orientation." 
 
 Just below this in the next paragraph, "Mobius strip shown on the diagram gives an example of nonorientable surface." This should either have an "a" before "nonorientable," or should have "surface" changed to "surfaces." 
 
 In the hint of 6.11 on page 65, "Its proof is not at all trivial; a standard proof use the so called Alexander’s duality which is a classical technique in algebraic topology," in which "use" should be changed to "uses," 
 
 and "so called" should be hyphenated. 
 
 In the sentence between 6.12 and 6.13 on page 65, "filed" needs to be corrected to "field." 
 
 In the short paragraph between 6.14 and 6.15 on this page, "section" is supposed to be "intersection," and there should be an "it" between the words "makes possible." 
 
 Below, in 6.15, there should be an "an" before "arbitrarily." 
 
 On page 66, in the sentence "Note that Π' || Π and arbitrary close to it," you should add an "is" after "and," and "arbitrary" should be changed to "arbitrarily." At the end of page 67, "so called" should be hyphenated. 
 
 On page 68, at the top, "The function (v, w) → → IIp(v, w) is called second fundamental form at p; it takes two tangent vector v and w at p and spits the real number IIp(v, w)." I don't know if you meant to have two arrows following (v,w) since there is a line break, but I would assume not. Also, there should be a "the" added after "called," 
 
 "out" added after "spits," 
 
 and "vector" should be "vectors." 
 
 On page 68, in the first paragraph beneath Principle curvatures, "... it is unique up to a rotation of the (x, y)-plane and switching the sign of the z-coordinate," should have an "a" added before "switching the," 
 
 and an "of" added between "switching the." 
 
 In the next paragraph, you have "... they are uniquely defined up to sign; they are denoted as k1(p) and k2(p) or k1(p)Σ and k2(p)Σ if we need to emphasize that these are the curvatures of the surface Σ;" which should include "a change in the" between "to" and "sign," as well as a comma after k2(p). 
 
 On page 69, above 7.2, you have "The last identity is the so called Euler’s formula. A smooth regular curve on a surface Σ that always runs in the principle directions is called line of curvature of Σ." "So called" should be hyphenated and there should be an "a" added between "called" and "line." 
 
 On page 70, in the first two paragraphs, you use the phrase "... if we need to emphasize that this is curvature of Σ." I believe writing "... if we need to emphasize that this is a curvature value of Σ." would be more clear given the context.  In the third paragraph, "up to sign" should be "up to a change in the sign." On page 73, just below equation (5), "derection" should be "direction." 
 
 Just above 7.12 on this page, "acelaration a equals to IIp(v, v)" has acceleration misspelled and 
 
 should either be "acceleration a is equal to IIp(v, v)" or just "acceleration a equals IIp(v, v)." 
 
 On page 75, in the second paragraph, "Note that in this case tangent plane is does not support the surface even locally," there should be a "the" before tangent 
 
 and "is" should be removed. 
 
 Later in this paragraph, "moving along the surface in the principle directions at a given point, one gets above and below the tangent plane at this point," should have "gets" changed to "goes." 
 
 On page 76, in 8.3, there should be an "a" between "is" and "complete." 
 
 In the paragraph following, "A smooth regular curve that always run in an asymptotic direction is called asymptotic line," should be changed to "A smooth regular curve that always runs in an asymptotic direction is called an asymptotic line." 
 
 On page 78, in 8.11, "a invertible" should be "an invertible." 
 
 On the same page, in the paragraph above 8.13, " applying 8.11 we get that saddle graph cannot lie between parallel planes, not necessarily horizontal. The following exercise shows that the theorem does not hold for saddle surface which are not graphs," in which the "graph" in the first sentence should be "graphs," 
 
 and the "surface" in the second sentence should be "surfaces." 
 
 On page 79, in the hint of 8.14, "Look at two section of the graph by planes parallel to (x, y)- plane and to (x, z)-plane and apply Meusnier’s theorem (7.12)," which would be more proper and clear by changing "section" to "sections" and writing instead "Look at two sections of the graph by considering planes parallel to the (x, y)- plane and to the (x, z)-plane, then apply Meusnier’s theorem (7.12)." 
 
 At the bottom of this page, "Since the plane Π is a convex, this statement contradicts 8.5," should be changed to "... plane Π is convex..." 
 
 At the beginning of page 81, "Moving the plane Tp little up, we can cut from Σ is a complete surface with boundary line lying in this plane (see 6.15)" which I think would be much more clear written as"Moving the plane Tp slightly upward, we can cut a complete surface  Π  from Σ such that the boundary line of Π lies in this slightly modified tangent plane (see 6.15)." 
 
 Further in this page, in the first paragraph under remarks, either the "a" should be removed from "... be found among infinite cylinders over a smooth regular curves," or "curves" should instead by "curve." 
 
 In the next sentence, I'm not sure what you are trying to say, should there be a "the" between "are" and "only?"
 
 In the last paragrah on this page, you have "This definition can be used for arbitrary surfaces not necessarily smooth. Some results, for example Bernshtein’s characterization of saddle graphs can be extended to generalized saddle surface, but this class of surfaces is far from being understood. Some nontrivial properties were proved by in Samuil Shefel..." There should be a comma in the first sentence before "not,"
 
 the second sentence should have "saddle surfaces," not "saddle surface," 
 
 and the last sentence should have "in" removed. 
 
 On page 83, in the first sentence, "...  so that the both principle curvatures..." should instead be "... so that both of the principle curvatures..." 
 
 On page 83, in the proof of 9.2, "sets" should be "set" in the sentence "Therefore the interior of R is a convex sets." 
 
 On page 84, right above the Open Surfaces section, "... classical result it topology — the so called closed graph theorem," should have "so called" hyphenated and "it" should be "in." 
 
 Further down this page, in the 9.4 Lemma, "in" should be removed from the sentence "Suppose Σ is an open surface in with positive Gauss..." 
 
 On page 85, about one-third of the way down the page, "poins" should be "points" in the sentence "... all of these poins lie in R." 
 
 In the third sentence on page 88, the word "minimize" should instead be "minimizes." 
 
 Further down the page, in the Nonuniqueness sentence, you have "There plenty..." which should have an "are" added between "there" and "plenty." 
 
 In the lemma for 10.3 on page 89, you have "... there is unique point..." This should have an "a" added between "is" and "unique." 
 
 Further in the page, in the proof of this lemma, you have "... lie in a ball or radius l + 1 centered at p." I believe the "or" here should be an "of." 
 
 On page 93, in the second paragraph of the Exponential Map section, you have "(There is a reason to call this map exponential, but it will takes us to far from the subject.)" "Takes" should be changed to "take" 
 
 and "to should be changed to "too." 
 
 In the following sentence on the page, the word "compete" should be corrected to "complete." 
 
 Just a little further in the page, in the paragraph just above 10.12, the word "has" should be removed from the sentence "It follows that the Jacobian matrix of the projection of expp to the (x, y)-plane has is the identity matrix." 
 
 On page 94, half way down the page, you have "For instance shortest path γ is an object of intrinsic geometry of the surface Σ, while definition of geodesic is not intrinsic — it requires the second derivative γ'' which needs the ambient space." This sentence should have a comma after instance, 
 
 another comma after γ'', 
 
 and "the" added between "while" and "definition." 
 
 In the following sentence on page 94, there should be an "a" added before the word "cylinder." 
 
 On page 95, at the end of 10.16, the word "is" should be removed from the sentence "Show that the statement is does not hold if γ fails to be minimizing." 
 
 On page 96, in the paragraph above 10.18, the word "plan" should be "plane" in the sentence "... we have that tangent plan supports the graph..." 
 
 On page 98, in both paragraphs of 10.21, the phrase "a convex function n-Lipschitz function" is used, clearly the "function" following "convex" can be removed from both paragraphs. 
 
 In the hint below 10.21, "guarantee" should be changed to "guarantees," 
 
 and "the" should be added in before the word "cone." 
 
 On page 99, in the paragraph directly under equation 1,  the word "produce" should be changed to "produces." 
 
 On page 101, in the paragraph above 11.3, you have "... sames plane (otherwise we could not call it operator )." In this sentence, "sames" should be changed to "same" 
 
 and the word "an" should be added in before "operator." 
 
 On page 104, in the hint of 11.6, "unite" should be changed to "unit." 
 
 On page 105, in the first paragraph of the proof of 11.8, the phrase "a obtuse" is used twice, 
 
 and "a" should be changed to "an" in both of these occurences. 
 
 On page 108, in 12.2, "is" should be changed to "be" in the first sentence, 
 
 and the "a" before "along" in the last sentence should be removed. 
 
 On page 115, in 12.12, "Lipshitz" is mispelled, missing the 'c.'






















According to \ref{ex:differential-range}, the shape operator takes a tangent vector at $p$ and spits a tangent vector to $\SS^2$ at $\nu_p$.
Note that $\nu_p$ is the normal vector to $\SS^2$ at $\nu_p$.
In other words the tangent plane $\T_p\Sigma$ is parallel to the tangent plane $\T_{\nu_p}\SS^2$,
or $\T_p\Sigma=\T_{\nu_p}\SS^2$, if we consider the tangent planes as linear spaces. 












Let us write $\Sigma$ in the spherical $(\rho,u)$-coordinates where $\rho\ge 0$ and $u\in\SS^2$.



Consider the map $f\:\Sigma\to\SS^2$ defined by $f(x)=\tfrac1{|x|}\cdot x$.
By construction $f$ is a smooth map.
Since $R$ is convex, and $\Sigma$ is bounded, on any ray from the origin there is exactly one point of $\Sigma$;
that is, $f$ gives a continuous bijection between $\Sigma$ and $\SS^2$.
Since $\SS^2$ is compact, the inverse map $f^{-1}\:\SS^2\to \Sigma$ is continuous; 
this is, $f$ is a homeomorphism.
The latter statement is a standard result in topology \ref{thm:Hausdorff-compact}.

Summarizing, we proved that $\Sigma$ is a topological sphere.

Let us show that that $f$ is regular.
In other words, for any cart $(u,v)\mapsto s(u,v)$ of $\Sigma$
the vectors $\tfrac{\partial f\circ s}{\partial u}$ and $\tfrac{\partial f\circ s}{\partial v}$ are linearly independent.

Note that
\begin{align*}
\frac{\partial (f\circ s)}{\partial u}&=\frac{\partial}{\partial u}\frac{s}{|s|}=
\\
&=\frac{1}{|s|}\cdot \frac{\partial s}{\partial u}+\left(\frac{\partial}{\partial u}\tfrac{1}{\sqrt{\langle s,s\rangle}}\right)\cdot s=
\\
&=\frac{1}{|s|}\cdot \frac{\partial s}{\partial u}
-
\left(\frac{\langle s,\frac{\partial s}{\partial u}\rangle}{|s|^3}\right)\cdot s=
\\
&=\frac{1}{|s|}\cdot \left(\frac{\partial s}{\partial u}\right)^\perp_s,
\intertext{where $v^\perp_s$ denotes the orthogonal projection of the vector $v$ to the line in the direction of $s$. 
The same way we get}
\frac{\partial (f\circ s)}{\partial v}&=\frac{1}{|s|}\cdot \left(\frac{\partial s}{\partial v}\right)^\perp_s.
\end{align*}

Since $R$ is convex and the origin in its interior, 
$s(u,v)$ can not point in the direction of the tangent plane $\T_{s(u,v)}$;
that is, three vectors $s(u,v)$, $\frac{\partial s}{\partial u}(u,v)$ and $\frac{\partial s}{\partial v}(u,v)$ are linearly independent.
Whence the projections $\left(\frac{\partial s}{\partial u}\right)^\perp_s$ and $ \left(\frac{\partial s}{\partial v}\right)^\perp_s$ are linearly independent
and therefore $f$ is regular.
By the inverse function theorem the inverse $f^{-1}\:\SS^2\to \Sigma$ is also regular;
that is, $\Sigma$ is a smooth sphere.
\qeds



















\begin{thm}{Theorem}
Let $\Sigma$ be a complete oriented surface with 
nonnegative principal curvatures at all points.
Then $\Sigma$ bounds a convex region $R$.
\end{thm}

\parit{Proof.}
Let $\nu$ be the unit normal field on $\Sigma$.
Consider the open region $R$ bounded by $\Sigma$ that lies on the side of $\nu$;
that is $\nu$ points inside of $R$ at any point of $\Sigma$.

Fix a plane $\Pi$ intersecting $R$.
Consider a conn

Since $\Sigma$ is connected, so is $R$;
moreover any two points in the interior of $R$ can be connected by a polygonal line in in the interior of~$R$.

Assume the interior of $R$ is not convex; that is, there are points $x,y\in R$ and a point $z$ between $x$ and $y$ that does not lie in the interior of $R$.
Consider a polygonal  line $\beta$ from $x$ to $y$ in the interior of $R$.
Let $y_0$ be the first point on $\beta$ such that the chord $[x,y_0]$ touches $\Sigma$ at some point, say~$z_0$.

\qeds












\parit{Proof.}
Denote by $\Gamma$ the graph $z=f(x,y)$.
Assume contrary; that is, $\Gamma$ lies between two planes $z=\pm C$.

Note $f$ can not be constant.
It follows that there is a line $\ell$ in $\RR^3$ with three points $a,b,c$ in the same order such that 
$a$ and $c$ are above $\Gamma$ and $b$ is below $\Gamma$.
Without loss of generality we can assume that the $z$-coordinate of $c$ is bigger than $C$.

Consider one parameter family of planes $\Pi_t$ containing $\ell$ that starts and ends at the vertical plane $\Pi_0=\Pi_1$ and rotates by angle $\pi$ while $t$ goes from $0$ to $1$.
For each $t$, consider the connected component $\Omega_t$ of $a$ in the subset of $\Pi_t$ that lies above $\Gamma$.

Let us show that $\Omega_t\not\ni c$ for some $t$.

If  $\Omega_t\ni c$ then $a$ can be connected by a path to $c$ in $\Omega_t$.
Since $\Gamma$ is saddle, the point $b$ cannot be surrounded by a closed curve in $\Omega_t$;
otherwise we would get a contradiction to ???.
Whence a any path from $a$ to $c$ in $\Omega_t$ might pass $b$ on the right side or on the left side,
but there is no pair of such paths that pass $b$ on both sides.
In particular we can divide subdivide the parameter values $[0,1]$ into three sets $t\in R$ if the path  in $\Omega_t$  goes on the right side from $b$,  $L$ if the path in $\Omega_t$ goes on the right side from $b$ and $N=[0,1]\backslash(R\cup L)$ if there is no path in $\Omega_t$ from $a$ to $b$.

Notice that $R$ and $L$ are open; indeed, if a path $\gamma$ lies in $\Omega_{t_0}$ then, since the epigraph $z>f(x,y)$ is open, all close paths lie in the epigraph.
In particular there will be a path from $a$ to $b$ in $\Omega_t$ for all $t$ sufficiently close to $t_0$,
evidently it goes on the same side from $b$.

Notice that $R\ni 0$ and $L\ni 1$; that is $R$ and $L$ are nonempty.
Since $[0,1]$ is connected, it can not be decomposed into two open nonempty sets.
Whence $N$ is not empty.
(In fact one can show that $s=\sup R$ lies in $N$.)


Fix $t\in N$; note that $\Omega_t$ lies between the planes $z=\pm C$;
otherwise $\Omega_t\ni c$.
Denote by $\Gamma_t$ the part of the graph $\Gamma$ that lies above $\Omega_t$.
By Sard's lemma we can assume that $\Gamma_t$ is a smooth surface with boundary on $\Pi_t$.

Denote by $N$ the plane perpendicular to $\ell$.
Let $n$ be the projection of $\Pi_t$ to $N$; 
it is a line in $N$.
Since $\Gamma_t$ and $\Omega_t$ lies between the planes $z=\pm C$, the projection of $\Gamma_t$ to $L$ lies on a bounded distance from $n$.

Applying \ref{ex:saddle-projection}, we get that the complement of the projection of $\Gamma_t$ in the $L^+$ convex. 
In order to be bounded it has to be a 


In particular the projection of $\Gamma_t$ the 

consider the projection $\proj\Omega_t$ of $\Omega_t$ to the plane $L$ perpendicular to $\ell$.
According to \ref{ex:saddle-projection},
each connected component of the complement of $\proj\Omega_t$  in the halfplane ??? is convex.
On the other hand by construction $\proj\Omega_t$ lies on the finite distance to ???.
It follows that $\proj\Omega_t$ is a strip of fixed width to ...

It follows that a plane parallel to $\Pi_t$ is supporting $\Gamma$ along a line.

\qeds

\begin{wrapfigure}{o}{40 mm}
\vskip-0mm
\centering
\includegraphics{mppics/pic-72}
\vskip0mm
\end{wrapfigure}

\begin{thm}{Exercise}\label{ex:saddle-projection}
Let $\Omega$ is an open convex set in the $(x,y)$-plane and 
$\Sigma$ be a compact saddle surface in $\RR^3$ with boundary line $\gamma$.
Denote by $\hat \Sigma$ and $\hat\gamma$ the orthogonal projections of $\Sigma$ and $\gamma$ to the $(x,y)$-plane in $\RR^3$.
Assume that $\hat\gamma$ lies outside of $\Omega$.
Show that each connected component of $\Omega\backslash \hat\Sigma$ is convex. 
\end{thm}


\parit{Hint:} 
Assume contrary; that is, for some connected component $\Theta$ there is a line $\ell$ that goes in $\Theta$, out and in again.
Observe that the same argument as in the lens lemma (\ref{lem:lens}) shows that the boundary of $\Theta$ has a locally supporting circle from inside.
Take the cylinder $C$ over this circle, observe that $C$ is supports $\Sigma$ and apply \ref{cor:surf-support}.

\begin{thm}{Lemma}
There is no complete strictly saddle surface in the cylinder $y^2+z^2\le R^2, z\ge 0$ that has boundary line on the $(x,y)$-plane.
\end{thm}

\parit{Proof.}
Assume contrary; let $\Sigma$ be such a surface.
Consider the projection $\hat\Sigma$ of $\Sigma$ to $(x,z)$-plane;
it lies on distance at most $R$ from $x$-axis.

Note that by \ref{ex:saddle-projection} the complement of $\hat\Sigma$ in the upper half-plane is convex.
It follows that $\hat\Sigma$ is a strip described by $0\le z\le r$ for some $r>0$.
Whence the plane $z=r$ supports $\Sigma$ at some point $p$. 
By \ref{cor:surf-support}, $K(p)_\Sigma\ge0$ --- a contradiction.
\qeds




























\begin{thm}{Exercise}\label{ex:signed-distance}
Let $\gamma$ be a simple smooth regular closed plane curve.
Assume it is parametrized so that the bounded region lies on the left from $\gamma$.
Given a point $p\in\RR^2$ denote by $f(p)$ the \emph{signed distance} to $\gamma$;
that is, $|f(p)|$ is the distance to $\gamma$ and $f(p)<0$ if $p$ is surrounded by $\gamma$; otherwise $f(p)\ge0$.
Suppose $\tau,\nu$ denotes the Frenet frame of $\gamma$.

\begin{enumerate}[(a)]
\item\label{ex:signed-distance:a} Show that there is $\eps>0$ such that if $|x|<\eps$, then $f(\gamma(t)+x\cdot \nu(t))=x$.
\item Use part (\ref{ex:signed-distance:a}) to show that $f$ is smooth in the $\eps$-neighborhood of $\gamma$. 
\item Show that for any $p=\gamma(t)$, the Laplacian $\Delta_p f$ equals to the signed curvature of $\gamma$ at $p$.
\end{enumerate}

\end{thm}















\section*{DNA inequality*}

Recall that curvature of a spherical curve is at least $1$
(Exercise~\ref{ex:curvature-of-spherical-curve}).
In particular the length of spherical curve can not exceed its total curvature.
The following theorem shows that the same inequality holds for \emph{closed} curves in a unit ball.

\begin{thm}{Theorem}\label{thm:DNA}
Let $\gamma$ be a nontrivial closed curve that lies in a unit ball.
Then 
\[\tc\gamma\ge \length\gamma.\]

\end{thm}

In the proof we use \ref{def:total-curv-poly} to define for the total curvature;
according to \ref{thm:total-curvature=}, it is more general than the smooth definition on page \pageref{page:total curvature of:smooth-def}.

\parit{Proof.}
We will show that 
\[\tc\gamma> \length\gamma.\]
for any closed polygonal line $\gamma=p_1\dots p_{n}$ in a unit ball.
It implies the theorem since in any nontrivial closed curve we can inscribe a closed polygonal line with arbitrary close total curvature and length.

The indexes are taken modulo $n$, in particular $p_{n}=p_0$, $p_{n+1}=p_1$ and so on.
Denote by $\theta_i$ the external angle of $\gamma$ at $p_i$;
that is,
\[\theta_i=\pi-\measuredangle p_{i-1}p_ip_{i+1}.\]

Denote by $o$ the center of the ball.
Consider a sequence of $n+1$ plane triangles
\begin{align*}
\triangle q_0s_0q_1
&\cong 
\triangle p_0op_1,
\\
\triangle q_1s_1q_2
&\cong 
\triangle p_1op_2,
\\
&\dots
\\
\triangle q_{n}s_nq_{n+1}
&\cong 
\triangle p_nop_{n+1},
\end{align*}
such that the points $q_0,q_1\dots$ lie on one line in that order and all the points $s_0,\dots,s_n$ lie on one side from this line.

\begin{figure}[h!]
\vskip-0mm
\centering
\includegraphics{mppics/pic-16}
\vskip0mm
\end{figure}

Since $p_0=p_n$ and $p_1=p_{n+1}$, we have that
\[\triangle q_{n}s_nq_{n+1}\cong 
\triangle p_nop_{n+1}=\triangle p_0op_1\cong\triangle q_{0}s_0q_1,\]
so $s_0q_0q_ns_n$ is a parallelogram.
Therefore
\begin{align*}
|s_0-s_1|+\dots+|s_{n-1}-s_n|
&\ge|s_n-s_0|=
\\
&=|q_0-q_n|=
\\
&=|p_0-p_1|+\dots+|p_{n-1}-p_n|
\\
&=\length \gamma.
\end{align*}

Note that 
\begin{align*}
\theta_i&=\pi-\measuredangle p_{i-1}p_ip_{i+1}\ge
\\
&\ge\pi-\measuredangle p_{i-1}p_io-\measuredangle op_ip_{i+1}=
\\
&=\pi-\measuredangle q_{i-1}q_is_{i-1}-\measuredangle s_iq_iq_{i+1}=
\\
&=\measuredangle s_{i-1}q_is_i>
\\
&>|s_{i-1}-s_i|;
\end{align*}
the last inequality follows since $|q_i-s_{i-1}|=|q_i-s_i|=|p_i-o|\le 1$.
That is, 
\[\theta_i>|s_{i-1}-s_i|\]
for each $i$.

It follows that
\begin{align*}
\tc \gamma
&=\theta_1+\dots+\theta_n>
\\
&> |s_{0}-s_1|+\dots |s_{n-1}-s_n|\ge 
\\
&\ge\length \gamma.
\end{align*}
Hence the result.
\qeds

This theorem was proved by Don Chakerian \cite{chakerian};
for plane curves it was prved earlier by Istv\'{a}n F\'{a}ry \cite{fary-DNA}.
Few proofs of this theorem are discussed by Serge Tabachnikov~\cite{tabachnikov}.
He also conjectured the following closely related statement:

\begin{thm}{Theorem}
Suppose a closed regular smooth curve $\gamma$ lies in a convex figure with the perimeter $2\cdot \pi$.
Then 
\[\tc\gamma\ge \length\gamma.\]

\end{thm}

It was proved by Jeffrey Lagarias and Thomas Richardson \cite{lagarias-richardso}; latter a simpler proof was given by Alexander Nazarov and Fedor Petrov~\cite{nazarov-petrov}.
The proof is annoyingly difficult; we do not present it here.












\section*{Complex coordinates}

It is often convenient to use complex coordinate on the plane,
it packs two coordinates $(x,y)$ in one complex number $z=x+i\cdot y$.

{

\begin{wrapfigure}{r}{20 mm}
\vskip-0mm
\centering
\includegraphics{mppics/pic-58}
\vskip0mm
\end{wrapfigure}

In particular any vector $w$ in $\RR^2$ can be regarded as a complex number.
Note that $i\cdot w$ is the vector obtained from $w$ by the counterclockwise rotation by $\tfrac\pi2$.

}































The same formula holds for any closed curve, if one use \emoh{signed area} surrounded by curve;
that is we count area surrounded by curve with multiplicity --- for each region cut by the curve from the sphere we count how many times the curve goes around it counterclockwise and multiply the area of the region by this number.
In order to be defined correctly we need to specify a pole on the spherelying and stating that its region has zero multiplicity. 
When you cross the curve the mulitplicity changes by $\pm1$; it increases is the curve cross the ways from left to right and decreasing otherwise.

Note that for the Gauss curvature of unit sphere is identically 1
Therefore 
\[\area\Delta=\int_\Delta G.\]
That is, \ref{eq:sphere-gauss-bonnet} is Gauss--Bonnet formula for spherical triangle.
Applying this formula for each triangle in a  triangulation of polygon and summing up, we get Gauss--Bonnet formula for arbitrary spherical polygon.

Since any smooth simple closed curve can be approximated by a polygon, we get that the 





Let $\Sigma$ be a smooth regular oriented surface and $\nu\:\Sigma\to \SS^2$ be its Gauss map.
Assume $\alpha$ is a smooth unit-speed curve in $\Sigma$. 
Then for any $t$ the vectors $\nu(t)=\nu(\gamma(t))$ and the velocity vector $\tau(t)\gamma'(t)$ are unit vectos that are normal to each other.
Denote by $\mu(t)$ the unit vector that is normal to both $\nu(t)$ and $\tau(t)$ such that for any $t$ the triple $(\tau(t),\mu(t),\nu(t)$ is an oriented basis (say $\mu(t)=[\nu(t),\tau(t)]$, where $[{*},{*}]$ denotes the vector product).

Since $\alpha$ is unit-speed, the acceleration $\alpha''(t)\perp\tau(t)$;
therefore at any parameter value $t$, we have
\[\alpha''(t)=k_n(t)\cdot \nu(t)+k_g(t)\cdot \mu(t),\]
for some real numbers $k_n$ and $k_g$.
The numbers $k_n(t)$ and $k_g(t)$ are called \emph{normal} and \emph{geodesic curvature} of $\alpha$ at $t$ correspondingly.
The geodesic curvature vanishes if and only if $\alpha$ is a geodesic; 
it measures how much a given curve diverges from a geodesic.

Locally $\alpha$ cuts $\Sigma$ into two parts, left and right.
The left part is the one that lies in the direction of $\mu(t)$ and the opposite is right.


\begin{thm}{Theorem}
Suppose $\Delta$ is a disc in a surfaces $\Sigma$ 
bounded bounded by a simple closed regular curve $\alpha$.
Assume that $\alpha$ is oriented such that $\Delta$ lies on the left from $\alpha$.
Then
\[\int_\Delta G+\int_\alpha k_g=2\cdot \pi,\]
where $G$ denotes the Gauss curvature of $\Sigma$ and
$k_g$ denotes the geodesic curvature curvature of $\alpha$.

\end{thm}

We will give an informal proof of this formula based on the bike wheel interpretation described in the previous section.
We suppose that it is intuitively clear that moving the axis of the wheel without changing its direction does not change the direction of the wheel's spokes.
More precisely, we need the following:

\begin{thm}{Claim}
Assume we keep the axis of a non-spinning bike wheel and perform the following two experiments:

\begin{enumerate}[(i)]
\item We moved it around and bring it back to the original position. 
As a result the wheel might rotate by some angle; let us measure this angle.

\item we move the direction of the axis the same as before without moving the center of the wheel and again measure the angle of rotation.
\end{enumerate}

Then the resulting angle in these two experiments is the same. 
\end{thm}

\section{Gauss--Bonnet formula}

\begin{thm}{Theorem}
Suppose $\gamma$ is a simple broken geodesic line that cuts a disc $\Delta$ from the surface $\Sigma$.
Assume that $\gamma$ is oriented so that $\Delta$ lies on the left from $\Sigma$.
Then 
\[\tgc\gamma+\iint_\Delta G=2\cdot \pi,\]
where $G$ denotes the Gauss curvature of $\Sigma$.
\end{thm}

Note that if $\Sigma$ is a plane then geodesic in $\Sigma$ are formed by line segments.
In this case the statement of theorem follows from Exercise~\ref{ex:pm2pi}.
In the next section we will prove the formula for the unit sphere;
latter we will use it in the sketch of the proof of the general case.









\section{Gauss map}

Let $\Sigma$ be a surface in the Euclidean space.
Given a point $p\in\Sigma$ consider a unit vector $\nu(p)$ that is normal to the tangent plane $\T_p$. 
The unit normal vector $\nu(p)$ is defined uniquely up to sign at each point $p\in \Sigma$.
If the choice of the sign is made so that the map $\nu\:p\mapsto \nu(p)$ is continuous,
then the map $\nu$ is called \emph{Gauss map}.

The Gauss map sends the surface $\Sigma$ to the unit sphere $\SS^2$.
It can be always defined locally (that is in a neighborhood of a point);
if it can be defined globally then the surface $\Sigma$ is called \emph{oriented}.

M\"obius band gives an example of nonoriented surface with boundary.
Klein bottle is an example of closed immerersed surface which is not oriented.
Any closed embedded surface is oriented since one can choose the  direction pointing outside of the region bounded by the surface.

Fix a point $p\in \Sigma$ and a smooth curve $\alpha$ in $\Sigma$ that starts at $p$;
that is $\alpha(0)=p$.
Set  $v=\alpha'(0)$ and $w=(\nu\circ\alpha)'(0)$.
Note that $v$ lie in $T_p$ --- the tangent plane at $p$.

Further $w\in \T_p$ as well.
Indeed $\nu\circ\alpha$ is a curve that starts at $\nu(p)$;
so $w=(\nu\circ\alpha)'(0)$ lies in the tangent plane $\T_{\nu(p)}\SS^2$.
But at $\nu(p)$, the sphere $\SS^2$ has normal $\nu(p)$ and therefore its tangent plane $\T_{\nu(p)}\SS^2$ is parallel $T_p$, so $\T_{\nu(p)}\SS^2$ and $T_p$ are identical as vector spaces.

Further note that since Gauss map is smooth,
the vector $w$ depends only on $v$;
that is if we choose a different curve $\alpha$ such that $v=\alpha'(0)$ we will get the same value $w=(\nu\circ\alpha)'(0)$.
The map $s_p\:v\mapsto w$ is called shape operator;
it maps the tangent plane $\T_p$ to itself.

It is straightforwardto check that the eigenvalues of $s_p$ are the principle curvatures at $p$; it agrees with the definition given above if the normal vector $\nu$ is chosen to point down in the specital graph representation $z=f(x,y)$.







The Gauss map can be defined (globally) if and only if the surface is orientable, in which case its degree is half the Euler characteristic. The Gauss map can always be defined locally (i.e. on a small piece of the surface). The Jacobian determinant of the Gauss map is equal to Gaussian curvature, and the differential of the Gauss map is called the shape operator. 


















\section{Comments}


\section{Tennis ball theorem}

Suppose that a curve $\alpha$ runs on the unit sphere 
\[\SS^2=\set{(x,y,z)\in\RR^3}{x^2+y^2+z^2=1}.\] 
Let us denote by $\alpha''(t)^\top$ the projection of vector $\alpha''(t)$ to the tanget plane of the sphere at $\alpha(t)$.

Assume $\alpha$ is a unit-speed curve.
By Proposition~\ref{prop:a'-pertp-a''}, $\alpha''(t)\perp\alpha'(t)$ and therefore $\alpha''(t)^\top\perp\alpha'(t)$.
If $w(t)$ denotes by the vector $\alpha'(t)$ rotated
counterclockwise by angle $\tfrac\pi2$ in the tangent plane to the sphere at $\alpha(t)$, then 
\[\alpha''(t)^\top=\kappa_\alpha(t)\cdot w(t)\]
for some real number $\kappa(t)$;
this number is called \emph{signed geodesic curvature of $\alpha$ at $t$}.

The geodesic curvature measures how $\alpha$ diverges from an equator --- the most straight curve on the curved $\SS^2$.
TBC














The following diagram shows a simple curve with positive curvature that intersects a line at points $p_1,\dots p_11$; these points appear on the curve in the same order, and on the line  in the order $p_5,p_3,p_1,p_2,p_4,p_7,p_9,p_{11},p_{10},p_8,p_6$; 
The following Exercise is about this pattern.

\begin{figure}[h!]
\vskip-0mm
\centering
\includegraphics{mppics/pic-29}
\vskip0mm
\end{figure}

\begin{thm}{Exercise}
Suppose $\alpha$ is a simple smooth regular curve in the plane that crosses a line $\ell$ at the points $p_1,p_2,\dots p_n$.
Assume that the points $p_1,p_2,\dots p_n$ appear on $\alpha$ in the same order.
Then the order on $\ell$ can be obtained from the following sequence 
\[p_1,p_3,\dots,p_4 ,p_2\]
by shift last $k$ elements to the beginning, reverting the order in the tail starting from $(2\cdot k+1)$-th element and reverting the order of the obtained sequence if necessary.
\end{thm}





\parit{Proof; only-if part.} 
Without loss of generality we can assume that $D$ lies on that $D$ lies on the left side of $\alpha$.
If so,
the only-if part will follow if the curvature of $\alpha$ is nonnegative.

Assume contrary, that is $\kappa(t_0)<0$ for some $t_0$.
Then the tangent line $\ell$ strictly supports $\alpha$ at $t_0$ from left;
that is, $\ell$ pass thru $\\alpha(t_0)$ and runs in the interior of $D$ shortly before and after.
A line $\ell'\parallel \ell$ which lies slightly to the right from $\ell$ has a line segment with the ends in the interior of $D$ that crosses $\alpha$.
That is $D$ is not convex.

\parit{If part.} assume $D$ is not convex; that is there are two points $p$ and $q$ in $D$ such that the line segment $[pq]$ does not lie completely in $D$.
Without loss of generality we may assume that $p$ and $q$ lie in the interior of $D$.

Let $\ell$ be the line thru $p$ and $q$.
Note that $\alpha$ changes the side of $\ell$ at least 4 times;
moreover one can choose crossing points $a$, $b$, $c$ and $d$ that appear in the same order on the line and $\alpha$.













The total signed curvature of a smooth unit-speed curve $\alpha\:[a,b]\z\to\RR^2$ can be defined as the integral
\[\tsc\alpha=\int_{[a,b]}\kappa(t)\cdot dt.\]

It is straightforward to show that for sufficienlty fine partition of closed unit-speed curve $\alpha$ the inscribed polygonal line has the same total signed curvature.
Therefore by exercises~\ref{ex:2kpi} and \ref{ex:pm2pi},
we get that the total signed curvature of any closed simple curve is mutiple of $2\cdot\pi$ and if the curve is simple, it has to be $\pm2\cdot\pi$.

\begin{thm}{Exercise}\label{ex:curvature-of-circle}
Find a unit-speed parametrization of the circle $\gamma$ or radius $R$ centered at the point $p$;
it is defined as the set 
\[\gamma=\set{v\in\RR^2}{|p-v|=R}.\]
Show that the signed curvature of this circle is $\pm \tfrac1R$.
\end{thm}
