\chapter{Definitions}


\begin{figure}[h!]
\begin{minipage}{.48\textwidth}
\centering
\includegraphics{mppics/pic-110}
\end{minipage}\hfill
\begin{minipage}{.48\textwidth}
\centering
\includegraphics{mppics/pic-115}
\end{minipage}
\bigskip
\begin{minipage}{.48\textwidth}
\centering
\includegraphics{mppics/pic-120}
\end{minipage}\hfill
\begin{minipage}{.48\textwidth}
\centering
\includegraphics{mppics/pic-125}
\end{minipage}
\end{figure}

\section{Simple curves}

In the following definition we use the notion of {}\emph{metric space} which is discussed in the preliminaries; see \ref{sec:metric-spcaes}.
The Euclidean plane and space are the main examples of metric spaces that one should keep in mind. A {}\emph{real interval} is a connected subset of the real numbers.

Recall that a bijective continuous map $f\:X\to Y$ between subsets of some metric spaces is called a {}\emph{homeomorphism} if its inverse $f^{-1}\:Y\z\to X$ is continuous.  

\begin{thm}{Definition} 
A connected subset $\gamma$ in a metric space is called a \index{simple curve}\emph{simple curve} if it is {}\emph{locally} homeomorphic to a real interval.
\end{thm}

It turns out that any simple curve $\gamma$ can be \index{parametrization of curve}\emph{parametrized} by a real interval or a circle.
That is, there is a homeomorphism $G\to\gamma$ 
where $G$ is a real interval (open, closed or semi-open) or the circle
\[\mathbb{S}^1=\set{(x,y)\in\RR^2}{x^2+y^2=1}.\]
A proof of this statement would take us away from the subject, so we will omit it. We hope that the statement is intuitively clear, but a full proof can be found in . % Lafontaine, J. (2012). Introduction aux variétés différentielles. EDP sciences.

If $G$ is an open interval or a circle, we say that $\gamma$ is a \index{endpoints}\emph{curve without endpoints}, otherwise it is 
called a {}\emph{curve with endpoints}.
In the case when $G$ is a circle we say that the curve is \emph{closed}. 
When $G$ is a closed interval, $\gamma$ is called an \index{arc}\emph{arc}.


A parametrization describes a curve completely.
We will denote a curve and its parametrization by the same letter;
for example, we may say a plane curve $\gamma$ is given with a parametrization $\gamma\:(a,b)\z\to \RR^2$.
Note, however, that any simple curve admits many different parametrizations. 

\begin{thm}{Exercise}\label{ex:9}
Find a continuous injective map $\gamma\:(0,1)\to\RR^2$ such that its image is not a simple curve.
\end{thm}


\section{Parametrized curves}

A \index{parameterized curve}\emph{parameterized curve} is defined as a continuous map $\gamma$ from a circle or a real interval (open, closed or semi-open) to a metric space. 
For a parameterized curve we do not assume that $\gamma$ is injective; in other words a parameterized curve might have \index{self-intersections}\emph{self-intersections}.

\begin{wrapfigure}{o}{17 mm}
\vskip-3mm
\centering
\includegraphics{mppics/pic-130}
\end{wrapfigure}

If we say \index{curve}\emph{curve} it means we do not want to specify whether it is a parameterized curve or a simple curve.

Whenever the domain of a parameterized curve is the closed unit interval $[0,1]$, then it is also called a \index{path}\emph{path}.
If in addition $p=\gamma(0)=\gamma(1)$, then $\gamma$ is called a \index{loop}\emph{loop};
in this case the point $p$ is called the \index{base of loop}\emph{base} of the loop.

\begin{thm}{Advanced exercise}\label{aex:simple-curve}
Let $X$ be a subset of the plane.
Suppose that two distinct points $p,q\in X$ can be connected by a path in $X$.
Show that there is a simple arc in~$X$ connecting $p$ to $q$.
\end{thm}

\section{Smooth curves}

Curves in Euclidean space or plane are called \index{space curve}\emph{space} or \index{plane curve}\emph{plane curves}, respectively.

A parameterized space curve can be described by its coordinate functions 
\[\gamma(t)=(x(t),y(t),z(t)).\]
Plane curves can be considered as a partial case of space curves with $z(t)\equiv 0$.

Recall that a real-to-real function is called \index{smooth function}\emph{smooth} if its derivatives of all orders are defined everywhere in the domain of definition.  
If each of the coordinate functions $x(t), y(t)$ and $z(t)$ of a space curve $\gamma$ are smooth, then the parametrized curve is called \index{smooth curve}\emph{smooth}.

If the \index{velocity vector}\emph{velocity vector} 
\[\gamma'(t)=(x'(t),y'(t),z'(t))\] 
does not vanish at any point, then the parameterized curve $\gamma$ is called \index{regular curve}\emph{regular}.

A simple space curve is called \index{smooth curve}\emph{smooth} (resp. \index{regular curve}\emph{regular}) if it admits a smooth (resp. regular) parametrization.
Regular smooth curves are among the main objects in differential geometry;
colloquially, the term \index{smooth curve}\emph{smooth curve} is often used as a shortcut for {}\emph{smooth regular curve}. 

\begin{thm}{Exercise}\label{ex:L-shape}
Let 
\[f(t)=
\begin{cases}
0&\text{if}\ t\le 0,
\\
\frac{t}{e^{1\!/\!t}}&\text{if}\ t> 0.
\end{cases}
\]
Show that $\alpha(t)=(f(t),f(-t))$ gives a smooth parametrization of a simple curve formed by the union of two half-axis in the plane.

Show that any smooth parametrization of this curve has vanishing velocity vector at the origin.
Conclude that this curve is smooth, but not regular;
that is, it does admit a smooth parametrization, but doesn't admit a regular smooth parametrization.
\end{thm}


\begin{thm}{Exercise}\label{ex:cycloid}
Describe the set of real numbers $\ell $
such that the plane curve $\gamma_\ell (t)= (t+\ell \cdot \sin t,\ell \cdot \cos t)$, $t\in\RR$ is



\begin{subthm}{ex:cycloid:regular}
smooth; % This is trivial, but I think it's better to make sure they understand the basics. They just read the definition!
\end{subthm}

\begin{subthm}{ex:cycloid:regular}
regular;
\end{subthm}

\begin{subthm}{ex:cycloid:simple}
simple.
\end{subthm}

\end{thm}

\section{Periodic parametrization}
A natural way to describe a closed simple curve is as a \index{periodic parameterization}\emph{periodic} parameterized curve $\gamma\: \RR\to \mathcal{X}$; that is, a curve such that $\gamma(t+\ell)=\gamma(t)$ for a fixed period $\ell > 0$ and all $t$.
For example, the unit circle in the plane can be described by the $2{\cdot}\pi$-periodic parametrization $\gamma(t)=(\cos t,\sin t)$.

{

\begin{wrapfigure}{o}{17 mm}
\vskip-3mm
\centering
\includegraphics{mppics/pic-51}
\end{wrapfigure}

Any smooth regular closed curve can be described by a smooth regular loop.
But in general the closed curve described by a smooth regular loop might fail to be regular at its base; an example is shown on the diagram.

}

\section{Implicitly defined curves}

Suppose $f\:\RR^2\to \RR$ is a smooth function; 
that is, all its partial derivatives are defined in its domain of definition.
Let $\gamma\subset \RR^2$ be the set of solutions of the equation $f(x,y)=0$.

Assume $\gamma$ is connected.
According to the implicit function theorem (\ref{thm:imlicit}), the set $\gamma$ is a smooth regular simple curve if $0$ is a \index{regular value}\emph{regular value} of $f$.
This condition is equivalent to the gradient $\nabla f$ not vanishing at any point $p\in \gamma$.
In other words, if $f(p)=0$, then   
$f_x(p)\ne 0$ or $f_y(p)\ne 0$.%
\footnote{Here $f_x$ is a shortcut notation for the partial derivative
$\tfrac{\partial f}{\partial x}$.\index{$f_x$}}

Similarly, assume $f,h$ is a pair of smooth functions defined in $\RR^3$.
The system of equations
\[\begin{cases}
   f(x,y,z)=0,
   \\
   h(x,y,z)=0.
  \end{cases}
\]
defines a regular smooth space curve if the set $\gamma$ of solutions is connected and $0$ is a regular value of the map $F\:\RR^3\to\RR^2$ defined as
\[F\:(x,y,z)\mapsto (f(x,y,z),h(x,y,z)).\]
In this case it means that the gradients $\nabla f$ and $\nabla h$ are linearly independent at any point $p\in \gamma$.
In other words, the Jacobian matrix
\[
\Jac_pF=\begin{pmatrix}
f_x&f_y&f_z\\
h_x&h_y&h_z
\end{pmatrix}
\]
for the map $F\:\RR^3\to\RR^2$ has rank 2 at any point $p \in \gamma$.

The way described above to define a curve is called \index{implicitly defined curve}\emph{implicit};
if a curve is defined by its parametrization, we say that it is \index{explicitly defined curve}\emph{explicitly defined}.
While implicit function theorem guarantees the existence of regular smooth parametrizations,
when it comes to calculations, it is usualy easier to work directly with implicit representations. 

\begin{thm}{Exercise}\label{ex:y^2=x^3}
Consider the set in the plane described by the equation
\[y^2=x^3.\]
Is it a simple curve?
Is it a smooth regular curve?
\end{thm}

\begin{thm}{Exercise}\label{ex:viviani}
Describe the set of real numbers $\ell$
such that the system of equations
\[\begin{cases}
x^2+y^2+z^2&=1
\\
x^2+\ell\cdot x+y^2&=0
\end{cases}\]
describes a smooth regular curve.
\end{thm}

\section{Proper curves}

A parametrized curve $\gamma$ in a metric space $X$ is called \index{proper curve}\emph{proper} if for any compact set $K \subset X$, the inverse image $\gamma^{-1}(K)$ is compact.

For example, the curve $\gamma(t)=(e^t,0,0)$ defined on the real line is not proper.
Indeed, the half-line $(-\infty,0]$ is not compact, but it is the inverse image of the unit closed ball around the origin.  

Note that closed curves and arcs are automatically proper since the parameter set is compact.

A simple curve is called proper if it admits a proper parametrization.
It turns out that a simple curve is proper if and only if its image is a closed set.
In particular, any implicitly defined plane or space curve is proper.
We leave the proof of this statement as an easy exercise to the reader. %??? ref

\begin{thm}{Exercise}\label{ex:proper-curve}
Use the Jordan's theorem (\ref{thm:jordan}) to show that any proper simple plane curve without endpoints divides the plane in two connected components.  
\end{thm}



