\chapter{Curves}

\parbf{Paths.}
Let $\mathcal{X}$ be a metric space.
A continuous map $f\:[0,1]\to\mathcal{X}$ is called a \emph{path}.
If $p=f(0)$ and $q=f(1)$, then we say that \emph{$f$ connects $p$ to $q$}.

If any two points in $\mathcal{X}$ can be connected by a path then $\mathcal{X}$ is called \emph{path connected}.
Similarly, a subset $A\subset \mathcal{X}$ is called \emph{path connected} if any two points $p,q\in A$ can be connected by a path that runs in $A$;
equivalently, the subspace $A$ is path connected.

\parbf{Simple curves.}

\begin{thm}{Definition} 
A path connected subset $\gamma$ in a metric space is called a \emph{simple curve} if it is \emph{locally} homeomorphic to a real interval; that is, any point $p\in\gamma$ has a neighborhood $U\ni p$ such that the intersection
$U\cap \gamma$ is homeomorphic to a real interval.
\end{thm} %???change def

It turns out that any curve $\gamma$ admits a homeomorphism from a real interval or a circle;
that is, there is a continuous bijection $G\to \gamma$ with continuous inverse;
here (and further) $G$ denotes a circle or real interval.
We omit a proof of this statement, but it is not hard.

The homeomorphism $G\to \gamma$ as above is called \emph{parametrization} of~$\gamma$.
The parametrization completely defines the curve.
Often will use the same letter for curve and its parametrization, so we can say curve $\gamma$ has parametrization $\gamma\:G\to \mathcal{X}$.
Note however that any curve admits many different parametrization. 

\begin{thm}{Exercise}
Find a continuous injective map $\gamma\:[0,1)\to\RR^2$ such that its image is not a simple curve.
\end{thm}

\parit{Hint:} The image of $\gamma$ should have a shape of digit $9$.


If $G$ is a circle, then the curve $\gamma\:G\to \mathcal{X}$ is called \emph{closed}.
If $G$ is a real interval, then  we may say that $\gamma$ is an \emph{arc}.

\parbf{Parameterized curves.}
A \emph{parameterized curve} is defined as a continuous map $\gamma\: G\to \mathcal{X}$. 
For a parameterized curve we do not assume that $\gamma$ is injective; in other words the parameterized curve might have self-intersections.

\begin{thm}{Advanced exercise}
Let $\alpha\:[0,1]\to\mathcal{X}$ be a path from $p$ to $q$.
Assume $p\ne q$.
Show that there is a simple path connecting from $p$ to $q$ in~$\mathcal{X}$.
\end{thm}

\section*{Smooth curves}

A curve in the Euclidean space or plane, called \emph{space} or \emph{plane curve} correspondingly.

A space curve can be described by its coordinate functions 
\[\gamma(t)=(x(t),y(t),z(t)).\]
Plane curves can be considered as a partial case of space curves with $z(t)\equiv 0$.

If each of the coordinate functions $x(t),y(t),z(t)$ of the space curve $\gamma$ is a smooth (that is, it has derivatives of all orders everywhere in its domain) then the parameterized curve is called \emph{smooth}.

If the \emph{velocity vector} 
\[\gamma'(t)=(x'(t),y'(t),z'(t))\] 
does not vanish at all points, then the parameterized curve $\gamma$ is called \emph{regular}.

A simple space curve is called \emph{smooth and regular} if it admits a smooth and regular parametrization correspondingly.
Regular smooth curves are among the main objects in differential geometry;
the term \emph{smooth curve} often used for \emph{smooth regular curve}. 

\begin{thm}{Exercise}\label{ex:L-shape}
The function 
\[f(t)=
\begin{cases}
0&\text{if}\ t\le 0,
\\
\frac{t}{e^{1\!/\!t}}&\text{if}\ t> 0.
\end{cases}
\]
is smooth.\footnote{The existence of all derivatives $f^{(n)}(x)$ at $x\ne 0$ is evident and direct calculations show that $f^{(n)}(0)=0$ for any $n$.}

Show that $\gamma(t)=(f(t),f(-t))$ gives a smooth parametrization of the curve $S$ formed by the union of two half-axis in the plane.

Show that any smooth parametrization of $S$ has vanishing velocity vector at the origin.
Conclude that the curve $S$ is not regular and smooth.
\end{thm}


\begin{thm}{Exercise}\label{ex:cycloid}
Describe the set of real numbers $a$
such that the plane curve $\gamma_a(t)= (t+a\cdot \sin t,a\cdot \cos t)$, $t\in\RR$ is
\begin{enumerate}[(a)]
\item regular;
\item simple.
\end{enumerate}

\end{thm}

\parbf{Loops and periodic parametrization.}
A closed simple curve can be described as an image of a parameterized curve $\gamma\: [0,1]\to \mathcal{X}$ such that $p=\gamma(0)=\gamma(1)$;
such curves are called \emph{loops}; 
the point $p$ in this case is called \emph{base} of the loop.

However, it is more natural to present it as a \emph{periodic} parameterized cure $\gamma\: \RR\to \mathcal{X}$; that is, there is a constant $\ell$ such that $\gamma(t+\ell)=\gamma(t)$ for any $t$.
For example the unit circle in the plane can be described by $2{\cdot}\pi$-periodic parametrization $\gamma(t)=(\cos t,\sin t)$.

\begin{wrapfigure}{r}{17 mm}
\vskip-5mm
\centering
\includegraphics{mppics/pic-51}
\end{wrapfigure}

Any smooth regular closed curve can be described by a smooth regular loop.
But in general the closed curve that described by a smooth regular loop might fail to be smooth and regular --- it might fail to be smooth at its base; an example shown on the diagram.


\section*{Implicitly defined curves}

Let $f\:\RR^2\to \RR$ be a smooth function; 
that is, all its partial derivatives defined in its domain of definition.
Consider the set $S$ of solution of equation $f(x,y)=0$ in the plane.

Assume $S$ is path connected.
According to implicit function theorem (\ref{thm:imlicit}), the set $S$ is a smooth regular simple curve if $0$ is a \emph{regular value} of $f$.
In this case it means that the gradient $\nabla f$ does not vanish at any point $p\in S$.
In other words, if $f(p)=0$, then  $\tfrac{\partial f}{\partial x}(p)\ne 0$ or $\tfrac{\partial f}{\partial y}(p)\ne 0$.

Similarly, assume $f,h$ is a pair of smooth functions defined in $\RR^3$.
The system of equations
$f(x,y,z)=h(x,y,z)=0$
defines a regular smooth space curve if the set of solutions is path connected and $0$ is a regular value of the map $F\:(x,y,z)\mapsto (f(x,y,z),h(x,y,z))$.
In this case it means that the gradients $\nabla f$ and $\nabla h$ are linearly independent at any point $p\in S$.
In other words, if $f(p)=0$, then at the Jacobian matrix
\[
\begin{pmatrix}
\tfrac{\partial f}{\partial x}&\tfrac{\partial f}{\partial y}&\tfrac{\partial f}{\partial z}\\
\tfrac{\partial h}{\partial x}&\tfrac{\partial h}{\partial y}&\tfrac{\partial f}{\partial z}
\end{pmatrix}
\]
for the map $F\:\RR^3\to\RR^2$ has rank 2 at $p$.

The described way to define a curve is called \emph{implicit};
if a curve is defined by its parametrization, we say that it is \emph{explicitly defined}.
While implicit function theorem guarantees the existence of regular smooth parametrizations, do not expect it to be in a closed form. 
When it comes to calculations, usually it is easier to work directly with implicit presentation. 

\begin{thm}{Exercise}
Consider the set in the plane described by the equation
\[y^2=x^3.\]
Is it a simple curve? and if ``yes'', is it a smooth regular curve?
\end{thm}

\begin{thm}{Exercise}\label{ex:viviani}
Describe the set of real numbers $a$
such that the system of equations
\begin{align*}
x^2+y^2+z^2&=1
\\
x^2+a\cdot x+y^2&=0
\end{align*}
describes a smooth regular curve.
\end{thm}



