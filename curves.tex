\chapter{Definitions}


\section*{Simple curves}

In the following definition we use the notion of \emph{metric space} which is discussed in Appendix~\ref{app:metric-spcaes}.
The Euclidean plane and space are the main examples of metric spaces that one should keep in mind. 

Recall that that a bijective continuous map $f\:X\to Y$ between subsets of some metric spaces is called \emph{homeomorphism} if its inverse $f^{-1}\:Y\z\to X$ is continuous.  

\begin{thm}{Definition} 
A connected subset $\gamma$ in a metric space is called a \emph{simple curve} if it is \emph{locally} homeomorphic to a real interval; that is, any point $p\in\gamma$ has a neighborhood $U\ni p$ such that the intersection
$U\cap \gamma$ is homeomorphic to an open real interval.
\end{thm}

It turns out that any curve can be \emph{parameterized} by an open real interval or a circle.
That is, for any curve $\gamma$ there is a homeomorphism $(a,b)\to\gamma$ or $\SS^1\to\gamma$ 
where $\SS^1$ denotes the unit circle; that is 
\[\SS^1=\set{(x,y)\in\RR^2}{x^2+y^2=1}.\]
We omit a proof of this statement; it is not hard, but would take us away from the subject.
We hope however that this statement is intuitively obvious. % a formal proof can be found in???.

Curves that admit a parametrization by a circle are called \emph{closed}.
The subsets of curves bounded from one or two sides by points are called \emph{curves with endpoints}.
If it has two endpoints, then it is called \emph{arc}; note that any arc can be parameterized by a closed interval.
A curve as well as a curve with endpoint(s) can be regarded as a curve;
if we need to emphasize that we work with a genuine curve we may say a \emph{curve without endpoints}.


A parametrization describes a curve completely.
Often we will denote a curve and its parametrization by the same letter;
for example, we may say a plane curve $\gamma$ is given with a parametrization $\gamma\:(a,b)\to \RR^2$.
Note however that any curve admits many different parametrization. 

\begin{thm}{Exercise}\label{ex:9}
Find a continuous injective map $\gamma\:(0,1)\to\RR^2$ such that its image is not a simple curve.
\end{thm}


\section*{Parameterized curves}

A \emph{parameterized curve} is defined as a continuous map $\gamma$ from a circle or a real interval (open, closed or semi-open) to a metric space. 
For a parameterized curve we do not assume that $\gamma$ is injective; in other words a parameterized curve might have \emph{self-intersections}.

If we say \emph{curve} it means we do not want to specify whether it is a parameterized curve or a simple curve.

If the domain of a parameterized curve is the closed unit interval $[0,1]$, then it is also called a \emph{path}.
If in addition $p=\gamma(0)=\gamma(1)$, then $\gamma$ is called a loop;
the point $p$ in this case is called \emph{base} of the loop.

\begin{thm}{Advanced exercise}
Let $X$ be a subset of the plane.
Suppose that two distinct points $p,q\in X$ can be connected by a path in $X$.
Show that there is a simple arc in~$X$ connecting $p$ to $q$.
\end{thm}

\section*{Smooth curves}

A curve in the Euclidean space or plane, is called \emph{space} or \emph{plane curve} correspondingly.

A parameterized space curve can be described by its coordinate functions 
\[\gamma(t)=(x(t),y(t),z(t)).\]
Plane curves can be considered as a partial case of space curves with $z(t)\equiv 0$.

Recall that a real-to-real function is called \index{smooth function}\emph{smooth} if its derivatives of all orders are defined everywhere in the domain of definition.  
If each of the coordinate functions $t\mapsto x(t),t\mapsto y(t)$ and $t\mapsto z(t)$ of the space curve $\gamma$ is a smooth, then the parameterized curve is called \index{smooth curve}\emph{smooth}.

If the \emph{velocity vector} 
\[\gamma'(t)=(x'(t),y'(t),z'(t))\] 
does not vanish at all points, then the parameterized curve $\gamma$ is called \emph{regular}.

A simple space curve is called \emph{smooth and regular} if it admits a smooth and regular parametrization.
Regular smooth curves are among the main objects in differential geometry;
colloquially, the term \emph{smooth curve} often used as a shortcut for \emph{smooth regular curve}. 

\begin{thm}{Exercise}\label{ex:L-shape}
Note that the function 
\[f(t)=
\begin{cases}
0&\text{if}\ t\le 0,
\\
\frac{t}{e^{1\!/\!t}}&\text{if}\ t> 0.
\end{cases}
\]
is smooth. Indeed, the existence of all derivatives $f^{(n)}(x)$ at $x\ne 0$ is evident and direct calculations show that $f^{(n)}(0)=0$ for any $n$.

Show that $\gamma(t)=(f(t),f(-t))$ gives a smooth parametrization of a simple curve formed by the union of two half-axis in the plane.

Show that any smooth parametrization of this curve has vanishing velocity vector at the origin.
Conclude that this curve is not regular and smooth;
that is, it does not admit a regular smooth parametrization.
\end{thm}


\begin{thm}{Exercise}\label{ex:cycloid}
Describe the set of real numbers $\ell $
such that the plane curve $\gamma_\ell (t)= (t+\ell \cdot \sin t,\ell \cdot \cos t)$, $t\in\RR$ is
\begin{enumerate}[(a)]
\item regular;
\item simple.
\end{enumerate}

\end{thm}

\section*{Periodic parametrization}
Note that any closed simple curve can be described as an image of a loop.
However, it is more natural to present it as a \emph{periodic} parameterized curve $\gamma\: \RR\to \mathcal{X}$; that is, such that $\gamma(t+\ell)=\gamma(t)$ for a fixed period $\ell$ and any $t$.
For example the unit circle in the plane can be described by $2{\cdot}\pi$-periodic parametrization $\gamma(t)=(\cos t,\sin t)$.

{

\begin{wrapfigure}{o}{17 mm}
\vskip-3mm
\centering
\includegraphics{mppics/pic-51}
\end{wrapfigure}

Any smooth regular closed curve can be described by a smooth regular loop.
But in general the closed curve that described by a smooth regular loop might fail to be smooth at its base; an example is shown on the diagram.

}

\section*{Implicitly defined curves}

Suppose $f\:\RR^2\to \RR$ is a smooth function; 
that is, all its partial derivatives defined in its domain of definition.
Let $\gamma\subset \RR^2$ be the set of solutions of the equation $f(x,y)=0$.

Assume $\gamma$ is connected.
According to implicit function theorem (\ref{thm:imlicit}), the set $\gamma$ is a smooth regular simple curve if $0$ is a \emph{regular value} of $f$.
In this case it means that the gradient $\nabla f$ does not vanish at any point $p\in \gamma$.
In other words, if $f(p)=0$, then  $\tfrac{\partial f}{\partial x}(p)\ne 0$ or $\tfrac{\partial f}{\partial y}(p)\ne 0$.

Similarly, assume $f,h$ is a pair of smooth functions defined in $\RR^3$.
The system of equations
\[\begin{cases}
   f(x,y,z)=0,
   \\
   h(x,y,z)=0.
  \end{cases}
\]
defines a regular smooth space curve if the set of solutions is connected and $0$ is a regular value of the map $F\:\RR^3\to\RR^2$ defined as
\[F\:(x,y,z)\mapsto (f(x,y,z),h(x,y,z)).\]
In this case it means that the gradients $\nabla f$ and $\nabla h$ are linearly independent at any point $p\in \gamma$.
In other words, the Jacobian matrix
\[
\begin{pmatrix}
\tfrac{\partial f}{\partial x}&\tfrac{\partial f}{\partial y}&\tfrac{\partial f}{\partial z}\\
\tfrac{\partial h}{\partial x}&\tfrac{\partial h}{\partial y}&\tfrac{\partial f}{\partial z}
\end{pmatrix}
\]
for the map $F\:\RR^3\to\RR^2$ has rank 2 at any $p$ such that $f(p)=h(p)=0$.

The described way to define a curve is called \emph{implicit};
if a curve is defined by its parametrization, we say that it is \emph{explicitly defined}.
While implicit function theorem guarantees the existence of regular smooth parametrizations, do not expect it to be in a closed form. 
When it comes to calculations, usually it is easier to work directly with implicit representation. 

\begin{thm}{Exercise}
Consider the set in the plane described by the equation
\[y^2=x^3.\]
Is it a simple curve? and if ``yes'', is it a smooth regular curve?
\end{thm}

\begin{thm}{Exercise}\label{ex:viviani}
Describe the set of real numbers $\ell$
such that the system of equations
\[\begin{cases}
x^2+y^2+z^2&=1
\\
x^2+\ell\cdot x+y^2&=0
\end{cases}\]
describes a smooth regular curve.
\end{thm}

\section*{Proper curves}

A parametrized curve $\gamma$ in a metric space $\mathcal{X}$ is called \emph{proper} if for any compact set $K$ the inverse image $\gamma^{-1}(K)$ is compact.

For example curve $\gamma(t)=(e^t,0,0)$ defined on whole real line is not proper.
Indeed the half-line $(-\infty,0]$ is not compact and it is the inverse image of unit closed ball around the origin.  

Note that any closed curve as well arc are proper curves since its parameter set is compact.

A simple curve is called proper if it admits a proper parametrization.
It turns out that simple curve is proper if and only if its image is a closed set.
In particular any implicitly defined plane or space curve is proper.
We omit the proof of this statement, but it is not hard. %??? ref

\begin{thm}{Exercise}\label{ex:proper-curve}
Use the Jordan's theorem (\ref{thm:jordan}) to show that any proper plane curve divides the plane in two connected components.  
\end{thm}



