\chapter{Definitions}

\section{Topological surfaces}

We will be mostly interested in smooth regular surfaces defined in the following section.
However we will sometimes use the following general definition.

A connected subset $\Sigma$ in the Euclidean space $\RR^3$
is called a \index{topological surface}\emph{topological surface} (more precisely an {}\emph{embedded surface without boundary}) 
if any point of $p\in \Sigma$ admits a neighborhood $W$ in $\Sigma$ 
that can be parametrized by an open subset in the Euclidean plane; 
that is, there is an injective continuous map $U\to W$ from an open set $U\subset \RR^2$ such that its inverse $W\to U$ is also continuous.


\section{Smooth surfaces}\label{sec:def-smooth-surface}

Recall that a function $f$ of two variables $x$ and $y$ is called \index{smooth function}\emph{smooth} if all its partial derivatives $\frac{\partial^{m+n}}{\partial x^m\partial y^n}f$ are defined and are continuous in the domain of definition of $f$. 

A connected set $\Sigma \subset \mathbb{R}^3$ is called a \index{smooth surface}\emph{smooth surface} (we use it as a shortcut for the more precise term {}\emph{smooth regular embedded surface}) if it can be described locally as a graph of a smooth function in an appropriate coordinate system.

More precisely, for any point $p\in \Sigma$ one can choose a coordinate system $(x,y,z)$ and a neighborhood $U\ni p$ such that
the intersection $W=U\cap \Sigma$ is a graph $z=f(x,y)$ of a smooth function $f$ defined in an open domain of the $(x,y)$-plane.

\parbf{Examples.}
The simplest example of a smooth surface is the $(x,y)$-plane 
\[\Pi=\set{(x,y,z)\in\RR^3}{z=0}.\]
The plane $\Pi$ is a surface since
it can be described as the graph of the function $f(x,y)=0$.

All other planes are smooth surfaces as well since one can choose a coordinate system so that it becomes the $(x,y)$-plane.
We may also present a plane as a graph of a linear function 
$f(x,y)=a\cdot x+b\cdot y+c$ for some constants $a$, $b$ and $c$
(assuming the plane is not perpendicular to the $(x,y)$-plane, in which case a different coordinate system is required to write the plane as the graph of a function).

A more interesting example is the unit sphere 
\[\mathbb{S}^2=\set{(x,y,z)\in\RR^3}{x^2+y^2+z^2=1}.\]
This set is not the graph of any function,
but $\mathbb{S}^2$ is locally a graph;
it can be covered by the following 6 graphs:
\begin{align*}
z&=f_\pm(x,y)=\pm \sqrt{1-x^2-y^2},
\\
y&=g_\pm(x,z)=\pm \sqrt{1-x^2-z^2},
\\
x&=h_\pm(y,z)=\pm \sqrt{1-y^2-z^2},
\end{align*}
where each function $f_\pm,g_\pm,h_\pm$ is defined in an open unit disc.
Any point $p\in\mathbb{S}^2$ lies in one of these graphs therefore $\mathbb{S}^2$ is a surface.
Since each function is smooth, so is the surface $\mathbb{S}^2$.

\section{Surfaces with boundary}
A connected subset in a surface that is bounded by one or more 
curves is called a  \index{surface with boundary}\emph{surface with boundary}; such curves form the \index{boundary line}\emph{boundary line} of the surface.

When we say {}\emph{surface} we usually mean a {}\emph{smooth regular surface without boundary};
we may use the terms {}\emph{surface without boundary} if we need to emphasize it;
otherwise we may use the term {}\emph{surface with possibly nonempty boundary}.

\section{Proper, closed and open surfaces}
If the surface $\Sigma$ is formed by a closed set, then it is called \index{proper surface}\emph{proper}.
For example, for any smooth function $f$ defined on the whole plane, its graph $z=f(x,y)$ is a proper surface.
The sphere $\mathbb{S}^2$ gives another example of proper surface.

On the other hand, the open disc 
\[\set{(x,y,z)\in\RR^3}{x^2+y^2<1,\  z=0}\]
is not proper; this set is neither open nor closed.

A compact surface without boundary is called \index{closed surface}\emph{closed}
(this term is closely related to {}\emph{closed curve} but has nothing to do with {}\emph{closed set}).

A proper noncompact surface without boundary is called \index{open surfac}\emph{open} (again the term {}\emph{open set} is not relevant).

For example, the paraboloid $z=x^2+y^2$
is an open surface; the 
sphere $\mathbb{S}^2$ is a closed surface.

Note that any proper surface without boundary is either closed or open.

The following claim is a three-dimesional analog of the plane separation theorem (\ref{ex:proper-curve}).
Despite it might look obvious, its proof is not trivial at all; a standard proof uses the so-called {}\emph{Alexander's duality} which is a classical technique in algebraic topology \cite[see][]{hatcher}.
We omit its proof since it would take us far away from the main subject.

\begin{thm}{Claim}\label{clm:proper-divides}
The complement of any proper topological surface without boundary (or, equivalently any open or closed topological surface) has exactly two connected components. 
\end{thm}

\section{Implicitly defined surfaces}

\begin{thm}{Proposition}\label{prop:implicit-surface}
Let $f\:\RR^3\to \RR$ be a smooth function.
Suppose that $0$ is a regular value of $f$;
that is, $\nabla_p f\ne 0$ at any point $p$ such that $f(p)=0$.
Then any connected component $\Sigma$ of the set of solutions of the equation $f(x,y,z)=0$ is a smooth surface.
\end{thm}

\parit{Proof.}
Fix $p\in\Sigma$.
Since $\nabla_p f\ne 0$ we have 
\[f_x(p)\ne 0,\quad f_y(p)\ne 0,\quad \text{or}\quad f_z(p)\ne 0.\]
We may assume that $f_z(p)\ne 0$;
otherwise permute the coordinates $x,y,z$.

The implicit function theorem (\ref{thm:imlicit}) implies that a neighborhood of $p$ in $\Sigma$ is the graph $z=h(x,y)$ of a smooth function $h$ defined on an open domain in $\RR^2$.
It remains to apply the definition of smooth surface (Section~\ref{sec:def-smooth-surface}).
\qeds

\begin{thm}{Exercise}\label{ex:hyperboloinds}
For which constants $\ell$ 
does the following equation
\begin{align*}
x^2+y^2-z^2&=\ell
\end{align*}
describe a smooth regular surface.
\end{thm}

\section{Local parametrizations}

Let $U$ be an open domain in $\RR^2$ and $s\:U\to \RR^3$ be a smooth map.
We say that $s$ is regular if its Jacobian has maximal rank;
in this case it means that the vectors $s_u$ and $s_v$ are linearly independent at any $(u,v)\in U$;
equivalently $s_u\times s_v\ne 0$, where $\times$ denotes the vector product.

\begin{thm}{Proposition}\label{prop:graph-chart}
If $s\:U\to \RR^3$ is a smooth regular embedding of an open connected set $U\subset \RR^2$, then its image $\Sigma=s(U)$ is a smooth surface.
\end{thm}

\parit{Proof.}
Set 
\[s(u,v)=(x(u,v),y(u,v),z(u,v)).\]
Since $s$ is regular, its Jacobian matrix
\[\Jac s=
\renewcommand\arraystretch{1.3}
\begin{pmatrix}
x_u&x_v\\
y_u&y_v\\
z_u&z_v
\end{pmatrix}
\]
has rank two at any pint $(u,v)\in U$.

Choose a point $p\in \Sigma$; by shifting the $(x,y,z)$ and $(u,v)$ coordinate systems we may assume that $p$ is the origin and $p=s(0,0)$.
Permuting the coordinates $x,y,z$ if necessary, we may assume that 
the matrix 
\[
\renewcommand\arraystretch{1.3}
\begin{pmatrix}
x_u&x_v\\
y_u&y_v
\end{pmatrix},
\] 
is invertible at the origin.
Note that this is the Jacobian matix of the map
\[(u,v)\mapsto (x(u,v),y(u,v)).\]

The inverse function theorem (\ref{thm:inverse}) implies that there is a smooth regular map
$w\:(x,y)\mapsto (u,v)$ defined on an open set $W\ni 0$ in the $(x,y)$-plane
such that $w(0,0)=(0,0)$ and  $s\circ w(x,y)=(x,y,f(x,y))$ for some smooth function~$f$.
That is, the graph $z=f(x,y)$ for $(x,y)\z\in W$ is a subset in $\Sigma$.
By the inverse function theorem this graph is open in $\Sigma$.

Since $p$ is arbitrary, we get that $\Sigma$ is a surface.
\qeds

If we have $s$ and $\Sigma$ as in the proposition, then we say that $s$ is a \index{smooth parametrization}\emph{smooth parametrization} of the surface $\Sigma$. 

Not all the smooth surfaces can be described by such a parametrization;
for example the sphere $\mathbb{S}^2$ cannot.
However, any smooth surface $\Sigma$ admits a local parametrization; that is, any point $p\in\Sigma$ admits an open neighborhood $W\subset \Sigma$ with a smooth regular parametrization~$s$.
In this case any point in $W$ can be described by two parameters, usually denoted by $u$ and $v$, 
which are called \index{local coordinates}\emph{local coordinates} at $p$.
The map $s$ is called a \index{chart}\emph{chart} of $\Sigma$.

If $W$ is a graph $z=h(x,y)$ of a smooth function $h$, then the map 
\[s\:(u,v)\mapsto (u,v,h(u,v))\] is a chart.
Indeed, $s$ has an inverse $(u,v,h(u,v))\mapsto (u,v)$ which is continuous;
that is, $s$ is an embedding.
Further,
$s_u=(1,0,h_u)$ and $s_v=(0,1,h_v)$. 
Whence the partial derivatives $s_u$ and $s_v$ are linearly independent;
that is, $s$ is a regular map.

Note that from \ref{prop:graph-chart}, we obtain the following corollary.

\begin{thm}{Corollary}\label{cor:reg-parmeterization}
A connected set $\Sigma\subset \RR^3$ is a smooth regular surface if and only if a neighborhood of any point in $\Sigma$ can be covered by a chart.
\end{thm}


\begin{thm}{Exercise}\label{ex:inversion-chart}
Consider the following map 
\[s(u,v)=(\tfrac{2\cdot u}{1+u^2+v^2},\tfrac{2\cdot v}{1+u^2+v^2},\tfrac{2}{1+u^2+v^2}).\]
Show that $s$ is a chart of the unit sphere centered at $(0,0,1)$; describe the image of $s$.
\end{thm}

\begin{wrapfigure}{o}{31 mm}
\vskip-6mm
\centering
\includegraphics{mppics/pic-750}
\vskip0mm
\end{wrapfigure}

The map $s$ in the exercise can be visualized using the following map
\[(u,v,1)\mapsto (\tfrac{2\cdot u}{1+u^2+v^2},\tfrac{2\cdot v}{1+u^2+v^2},\tfrac{2}{1+u^2+v^2})\]
which is called \index{stereographic projection}\emph{stereographic projection} from the plane $z=1$ to the unit sphere with center at $(0,0,1)$.
Note that the point $(u,v,1)$ and its image lie on the same half-line emerging from the origin. 

\begin{wrapfigure}{o}{31 mm}
\vskip0mm
\centering
\includegraphics{asy/torus}
\vskip0mm
\end{wrapfigure}

Let $\gamma(t)=(x(t),y(t))$ be a plane curve.
Recall that the \index{surface of revolution}\emph{surface of revolution} of the curve $\gamma$ around the $x$-axis can be described as the 
image of the map 
\[(t, s)\mapsto (x(t), y(t)\cdot\cos s,y(t)\cdot\sin s).\]
For fixed $t$ or $s$ the obtained curves are called \index{meridian}\emph{meridians} or \index{parallel}\emph{parallels} of the surface, respectively; note that parallels are formed by circles in the plane perpendicular to the axis of rotation.
The curve $\gamma$ is called the \index{generatrix}\emph{generatrix} of the surface.

\begin{thm}{Exercise}\label{ex:revolution}
Assume $\gamma$ is a closed simple smooth regular plane curve that does not intersect the $x$-axis.
Show that surface of revolution around the $x$-axis with generatrix $\gamma$ is a smooth regular surface.
\end{thm}


\section{Global parametrizations} 
A surface can be described by an embedding from a known surface to the space.

For example, consider the ellipsoid
\[\Sigma_{a,b,c}=\set{(x,y,z)\in\RR^3}{\tfrac{x^2}{a^2}+\tfrac{y^2}{b^2}+\tfrac{z^2}{c^2}=1}\]
for some positive numbers $a$, $b$, and $c$.
Note that by \ref{prop:implicit-surface}, $\Sigma_{a,b,c}$ is a smooth regular surface.
Indeed, set $f(x,y,z)=\tfrac{x^2}{a^2}+\tfrac{y^2}{b^2}+\tfrac{z^2}{c^2}$,
then
\[\nabla f(x,y,z)=(\tfrac{2}{a^2}\cdot x,\tfrac{2}{b^2}\cdot y,\tfrac{2}{c^2}\cdot z).\]
Therefore $\nabla f\ne0$ if $f=1$; that is, $1$ is a regular value of $f$.

Note that $\Sigma_{a,b,c}$ can be defined as the image of the map $s\:\mathbb{S}^2\to\RR^3$, defined as the restriction of the following map to the unit sphere $\mathbb{S}^2$:
\[(x,y,z)\z\mapsto (a\cdot x, b\cdot y,c\cdot z).\]

For a surface $\Sigma$, a map $s\: \Sigma \to \RR ^3$ is called a 
\emph{smooth parametrized surface} if $s$ is injective and for any chart $f\:U\to \Sigma$ 
the composition $s\circ f$ is smooth and regular;
that is, all partial derivatives $\frac{\partial^{m+n}}{\partial u^m\partial v^n}(s\circ f)$ exist and are continuous in the domain of definition and the two vectors 
$\frac{\partial}{\partial u}(s\circ f)$ and $\frac{\partial}{\partial v}(s\circ f)$ are linearly independent.

Note that in this case the image $\Sigma^{*}=s(\Sigma)$ is a smooth surface.
The later follows since for any chart $f\:U\to \Sigma$ the composition $s\circ f\:U\to \Sigma^{*}$ is a chart of $\Sigma^{*}$. 
The map $s$ is called a \index{diffeomorphism}\emph{diffeomorphism} from $\Sigma$ to $\Sigma^{*}$; the surfaces $\Sigma$ and $\Sigma^{*}$ are said to be {}\emph{diffeomorphic} if there is a diffeomorphism $s\:\Sigma\to\Sigma^{*}$.


\begin{thm}{Advanced exercise}\label{ex:star-shaped-disc}
Show that the surfaces $\Sigma$ and $\Theta$ are diffeomorphic if

\begin{subthm}{ex:plane-n}
$\Sigma$ and $\Theta$ obtained from the plane by removing $n$ points.
\end{subthm}


\begin{subthm}{ex:star-shaped-disc:smooth}
$\Sigma$ and $\Theta$ are open convex subsets of a plane bounded by smooth curves.
\end{subthm}


\begin{subthm}{ex:star-shaped-disc:nonsmooth}
$\Sigma$ and $\Theta$ are open convex subsets of a plane.
\end{subthm}

\begin{subthm}{ex:star-shaped-disc:star-shaped}
$\Sigma$ and $\Theta$ are open star-shaped subsets of a plane.
\end{subthm}
\end{thm}

\chapter{First order structure}

\section{Tangent plane}

\begin{thm}{Definition}\label{def:tangent-vector}
Let $\Sigma$ be a smooth surface.
A vector $\vec w$ is a \index{tangent vector}\emph{tangent vector} of $\Sigma$ at $p$ if and only if there is a curve $\gamma$ that runs in $\Sigma$ and has $\vec w$ as a velocity vector at $p$;
that is, $p=\gamma(t)$ and $\vec w=\gamma'(t)$ for some $t$.
\end{thm}

\begin{thm}{Proposition-Definition}\label{def:tangent-plane}
Let $\Sigma$ be a smooth surface and $p\in \Sigma$.
Then the set of tangent vectors of $\Sigma$ at $p$ forms a plane;
this plane is called the \index{tangent plane}\emph{tangent plane} of $\Sigma$ at $p$.

Moreover if $s\:U\to \Sigma$ is a local chart and $p=s(u_p,v_p)$, then 
the tangent plane of $\Sigma$ at $p$ is spanned by vectors $s_u(u_p,v_p)$ and $s_v(u_p,v_p)$.
\end{thm}

The tangent plane to $\Sigma$ at $p$ is usually denoted by $\T_p$ or $\T_p\Sigma$.
This plane $\T_p$ might be considered as a linear subspace of $\RR^3$ or as a parallel plane passing thru $p$;
the latter is sometimes called the \index{affine tangent plane}\emph{affine tangent plane}.
The affine tangent plane can be interpreted as the best approximation at~$p$ of the surface $\Sigma$ by a plane.
More precisely, 
it has \index{order of contact}\emph{first order of contact} with $\Sigma$ at $p$;
that is, $\rho(q)\z=o(|p-q|)$, where $q\in \Sigma$ and $\rho(q)$ denotes the distance from $q$ to $\T_p$.

\parit{Proof.}
Fix a chart $s$ at $p$.
Assume $\gamma$ is a smooth curve that starts at~$p$.
Without loss of generality, we can assume that $\gamma$ is covered by the chart;
in particular, there are smooth functions $u(t)$ and $v(t)$ such that 
\[\gamma(t)=s(u(t),v(t)).\]
Applying chain rule, we get
\[\gamma'=s_u\cdot u'+ s_v\cdot v';\]
that is, $\gamma'$ is a linear combination of $s_u$ and $s_v$.

Since the smooth functions $u(t)$ and $v(t)$ can be chosen arbitrarily, any linear combination of $s_u$ and $s_v$ is a tangent vector at $p$. 
\qeds


\begin{thm}{Exercise}\label{ex:tangent-normal}
Let $f:\RR^3\to\RR$ be a smooth function with $0$ as a regular value and $\Sigma$ be a surface described as a connected component of the set of solutions $f(x,y,z)=0$.
Show that the tangent plane $\T_p\Sigma$ is perpendicular to the gradient $\nabla_pf$ at any point $p\in\Sigma$.
\end{thm}

\begin{thm}{Exercise}\label{ex:vertical-tangent}
Let $\Sigma$ be a smooth surface and $p\in\Sigma$.
Choose $(x,y,z)$-coordinates.
Show that a neighborhood of $p$ in $\Sigma$ is a graph $z=f(x,y)$ of a smooth function $f$ defined on an open subset in the $(x,y)$-plane if and only if the tangent plane $\T_p$ is not {}\emph{vertical}; that is, if $\T_p$ is not perpendicular to the $(x,y)$-plane.
\end{thm}

\begin{thm}{Exercise}\label{ex:tangent-single-point}
Show that if a smooth surface $\Sigma$ meets a plane $\Pi$ at a single point $p$, then $\Pi$ is tangent to $\Sigma$ at $p$.
\end{thm}


\section{Directional derivative}

In this section we extend the definition of directional derivative to smooth functions defined on smooth surfaces.

First let recall the standard definition of directional derivative.

Suppose $f$ is a function defined at a point $p$ in the space, and $\vec w$ a vector.
Consider the function
\[h(t)=f(p+t\cdot\vec w).\]
Then the directional derivative of $f$ at $p$ along $\vec w$ is defined as 
\[D_{\vec w}f(p)\df h'(0).\]
\index{10d@$D_{\vec{w}}f$}

\begin{thm}{Proposition-Definition}\label{def:directional-derivative}
Let $\Sigma$ be a smooth regular surface and $f$ a smooth function defined on $\Sigma$. 
Suppose $\gamma$ is a smooth curve in $\Sigma$ that starts at $p$ with velocity vector $\vec{w}\in \T_p$;
that is, $\gamma(0)=p$ and $\gamma'(0)=\vec{w}$.
Then the derivative $(f\circ\gamma)'(0)$
depends only on $f$, $p$ and $\vec{w}$;
it is called the \index{directional derivative}\emph{directional derivative of $f$ along $\vec{w}$ at $p$}
and denoted by
\[D_{\vec{w}}f,\quad D_{\vec{w}}f(p), \quad\text{or}\quad D_{\vec{w}}f(p)_\Sigma\] 
--- we may omit $p$ and $\Sigma$ if it is clear from the context.

Moreover, if $(u,v)\mapsto s(u,v)$ is a local chart at $p$, then 
\[D_{\vec{w}}f=a\cdot f_u+b\cdot f_v,\]
where $\vec{w}=a\cdot s_u +b\cdot s_v$. 
\end{thm}

Note that our definition agrees with the standard definition of directional derivative if $\Sigma$ is a plane.
Indeed, in this case $\gamma(t)=p+\vec w\cdot t$ is a curve in $\Sigma$ that starts at $p$ with velocity vector $\vec{w}$.
For a general surface the point $p+\vec w\cdot t$ might not lie on the surface; therefore the function $f$ might be undefined at this point; therefore the standard definition does not work.

\parit{Proof.}
Without loss of generality, we may assume that $\gamma$ is covered by the chart $s$;
if not we can chop $\gamma$.
In this case 
\[\gamma(t)=s(u(t),v(t))\]
for some smooth functions $u,v$ defined in a neighborhood of $0$ such that 
$u(0)=u_p$ and $v(0)=v_p$.

Applying the chain rule, we get that
\begin{align*}
\gamma'(0)&=u'(0)\cdot s_u+v'(0)\cdot s_v
\end{align*}
at $(u_p,v_p)$.
Since $\vec{w}=\gamma'(0)$ and the vectors $s_u$, $s_v$ are linearly independent, we get that $a=u'(0)$ and $b=v'(0)$.

Applying the chain rule again, we get that
\[
(f\circ\gamma)'(0)=a\cdot f_u+b\cdot f_v.
\eqlbl{eq:f-gamma}
\]
at $(u_p,v_p)$.

Notice that the left hand side in \ref{eq:f-gamma} does not depend on the choice of the chart $s$ and the right hand side depends only on $p$, $\vec w$, $f$, and $s$. 
It follows that $(f\circ\gamma)'(0)$ depends only on $p$, $\vec w$ and $f$.

The last statement follows from \ref{eq:f-gamma}.
\qeds

\section{Tangent vectors as functionals}

In this section we introduce a more conceptual way to define tangent vectors.
We will not use this approach in the sequel, but it is better to know about it.

A tangent vector $\vec w\in \T_p$ to a smooth surface $\Sigma$ 
defines a linear functional%
\footnote{Term \index{functional}\emph{functional} is used for functions that take a function as an argument and return a number.} $D_{\vec w}$
that swallows a smooth function $\phi$ defined in a neighborhood of $p$ in $\Sigma$ and spits its directional derivative $D_{\vec w}\phi$.
It is straightforward to check that the functional $D$ obeys the product rule:
\[D_{\vec w}(\phi\cdot\psi)=(D_{\vec w}\phi)\cdot \psi(p)+\phi(p)\cdot(D_{\vec w}\psi).
\eqlbl{eq:tangent-functional}\]

It is not hard to show that the tangent vector $\vec w$ is completely determined by the functional $D_{\vec w}$.
Moreover tangent vectors at $p$ can be \emph{defined} as linear functionals on the space of smooth functions
that satisfy the product rule \ref{eq:tangent-functional}.

This definition grabs the key algebraic property of tangent vectors.
It might be a less intuitive way to think about tangent vectors, but it is often convenient to use in the proofs. 
For example \ref{def:directional-derivative} becomes a tautology.

\section{Differential of map}

Any smooth map $s$ from a surface $\Sigma$ to $\RR^3$ can be described by its coordinate functions 
$ s(p)=(x(p),y(p),z(p))$. 
To take a directional derivative of the map we should take the  directional derivative of each of its coordinate functions.
\[D_{\vec{w}} s\df(D_{\vec{w}}x,D_{\vec{w}}y,D_{\vec{w}}z).\]

Assume $ s$ is a smooth map from one smooth surface $\Sigma_0$ to another $\Sigma_1$ and $p\in \Sigma_0$.
Note that $D_{\vec w} s(p)\in \T_{s(p)}\Sigma_1$ for any $\vec w\in \T_p$.
Indeed, choose a curve $\gamma_0$ in $\Sigma_0$ such that $\gamma_0(0)=p$ and $\gamma_0'(0)=\vec w$.
Observe that $\gamma_1= s\circ \gamma_0$ is a smooth curve in $\Sigma_1$ and 
by the definition of directional derivative, we have $D_{\vec w} s(p)=\gamma_1'(0)$.
It remains to note that $\gamma_1(0)\z= s(p)$ and therefore its velocity $\gamma_1'(0)$ is in $\T_{ s(p)}\Sigma_1$.

Recall that \ref{def:directional-derivative} implies that 
$d_p s\:\vec w \mapsto D_{\vec w} s$ defines a linear map $d_p s\:\T_p\Sigma_0\to \T_{ s(p)}\Sigma_1$;
that is,
\[D_{c\cdot \vec w} s=c\cdot D_{\vec w} s(p)
\quad\text{and}\quad D_{\vec v+ \vec w} s=D_{\vec v} s(p)+ D_{\vec w} s(p)\]
for any $c\in\RR$ and $\vec v, \vec w\in\T_p$.
The map $d_p s$ is called the \index{differential of map}\emph{differential} of $s$ at $p$.

The differential $d_p s$ can be described by a $2{\times}2$-matrix $M$ in orthonormal bases of $\T_p$ and $\T_{ s(p)}\Sigma_1$.
Set $\jac_p s=|\det M|$; this value  
does not depend on the choice of orthonormal bases in $\T_p$ and $\T_{ s(p)}\Sigma_1$. \label{page:|L|}\index{10d@$d_p s$}

Let $ s_1\:\Sigma_1\to\Sigma_2$ be another smooth map between smooth surfaces $\Sigma_1$ and $\Sigma_2$.
Suppose that ${p_1}= s(p)\in\Sigma_1$;
observe that 
\[d_p( s_1\circ s)=d_{p_1} s_1 \circ d_p s.\]
It follows that
\[\jac_p( s_1\circ s)
=
\jac_{p_1} s_1\cdot\jac_p s.\eqlbl{eq:jac-composition}\]


If $\Sigma_0$ is a domain in the $(u,v)$-plane, then the value $\jac_p s$ can be found using the following formulas 
\begin{align*}
\jac s
&=|s_v\times s_u|=
\\
&=\sqrt{\langle s_u, s_u\rangle\cdot\langle s_v, s_v\rangle -\langle s_u, s_v\rangle^2}=
\\
&=\sqrt{\det[\Jac^\top s\cdot \Jac s]}.
\end{align*}
where $\Jac s$ denotes the Jacobian matrix of $s$; it is a 2$\times$3 matrix with column vectors $s_u$ and $ s_v$.

The value $\jac_p s$ has the following geometric meaning:
if $P_0$ is a region in $\T_p$ and $P_1=(d_p s)(P_0)$, then
\[\area P_1=\jac_p s\cdot \area P_0.\]
This identity will become important in the definition of surface area.



\section{Surface integral and area}

Let $\Sigma$ be a smooth surface and $h\:\Sigma\to\RR$ be a smooth function.
Let us define the integral $\iint_R h$ of the function $h$ along a region $R\subset \Sigma$.
The definition will be used mostly along surfaces with boundary, but the definition can be applied to any Borel set $R\subset \Sigma$.

Recall that $\jac_ps$ is defined in the previous section.
Assume that there is a chart $(u,v)\mapsto s(u,v)$ of $\Sigma$ defined on an open set $U\subset\RR^2$ such that $R\subset s(U)$.
In this case set
\[\iint_R h\df \iint_{s^{-1}(R)} h\circ s(u,v)\cdot \jac_{(u,v)}s  \cdot du\cdot dv.\eqlbl{eq:area-def}\]


By the substitution rule (\ref{thm:mult-substitution}), the right hand side in \ref{eq:area-def} does not depend on the choice of $s$.
That is, if $s_1\:U_1\to \Sigma$ is another chart such that $s_1(U_1)\supset R$, then 
\[\iint_{s^{-1}(R)} h\circ s(u,v)\cdot \jac_{(u,v)}s  \cdot du\cdot dv=\iint_{s_1^{-1}(R)} h\circ s_1(u,v)\cdot \jac_{(u,v)}s_1  \cdot du\cdot dv.\]
In other words, the defining identity \ref{eq:area-def} makes sense.

A general region $R$ can be subdivided into regions $R_1,R_2\dots$ such that each $R_i$ lies in the image of some chart.
After that one could define the integral along $R$ as the sum
\[\iint_Rh
\df
\iint_{R_1}h+\iint_{R_2}h+\dots\]
It is straightforward to check that the value $\iint_Rh$ does not depend on the choice of such subdivision.

The area of a region $R$ in a smooth surface $\Sigma$ is defined as the surface integral 
\[\area R=\iint_R 1.\]

The following proposition provides a substitution rule for surface integral.

\begin{thm}{Area formula}\label{prop:surface-integral}
Suppose $ s\:\Sigma_0\to \Sigma_1$ is a smooth parameterization of a smooth surface $\Sigma_1$ by  a smooth surface $\Sigma_0$.
Then for any region $R\subset \Sigma_0$ and any smooth function $f\:\Sigma_1\to\RR$ we have
\[\iint_R (f\circ s)\cdot \jac  s=\int_{ s(R)}f.\]
In particular, if $f\equiv 1$, we have
\[\iint_R \jac  s=\area [ s(R)].\]

\end{thm}

\parit{Proof.}
Follows from \ref{eq:jac-composition} and the definiton of surface integral.
\qeds

\parbf{Remark.}
The notion of area of a surface is closely related to the length of a curve.
However, to define length we use a different idea --- it was defined as the least upper bound on the lengths of inscribed polygonal lines.
It turns out that an analogous definition does not work even for very simple surfaces.
The latter is shown by a classical example --- the so-called \emph{Schwarz's boot}.
This example and different approaches to the notion of area are discussed in a popular article of Vladimir Dubrovsky \cite{dubrovsky}.

\section{Normal vector and orientation}
A unit vector that is normal to $\T_p$ is usually denoted by $\Norm(p)$;
it is uniquely defined up to sign.\index{10nu@$\Norm$}

A surface $\Sigma$ is called \index{oriented surface}\emph{oriented} if it is equipped with a unit normal vector field $\Norm$;
that is, a continuous map $p\mapsto \Norm(p)$ such that $\Norm(p)\perp\T_p$ and $|\Norm(p)|=1$ for any $p$.
The choice of the field $\Norm$ is called the \index{orientation}\emph{orientation} of $\Sigma$.
A surface $\Sigma$ is called \index{orientable}\emph{orientable} if it can be oriented.
Note that each orientable surface admits two orientations: $\Norm$ and $-\Norm$.

Let $\Sigma$ be a smooth oriented surface with unit normal field $\Norm$.
The map $\Norm\:\Sigma\to \mathbb{S}^2$ defined by $p\mapsto \Norm(p)$ is called the \index{spherical map}\emph{spherical map} or \index{Gauss map}\emph{Gauss map}.

For surfaces, the spherical map plays essentially the same role as the tangent indicatrix for curves.

The M\"obius strip shown on the diagram gives an example of a nonorientable surface --- there is no choice of normal vector field that is continuous along the middle of the strip (it changes the sign if you try to go around).

\begin{wrapfigure}{o}{42 mm}
\vskip-0mm
\centering
\includegraphics{asy/moebius}
\vskip0mm
\end{wrapfigure}

Note that each surface is locally orientable.
In fact each chart $s(u,v)$ admits an orientation 
\[\Norm=
\frac{s_u\times s_v}
{\left|s_u\times s_v\right|}.\]
Indeed, the vectors $s_u$ and $s_v$ are tangent vectors at $p$; 
since they are linearly independent, their vector product does not vanish and it is perpendicular to the tangent plane.
Evidently $(u,v)\mapsto \Norm(u,v)$ is a continuous map.
Therefore $\Norm$ is a unit normal field. 

\begin{thm}{Exercise}\label{ex:implicit-orientable}
Let $h:\RR^3\to\RR$ be a smooth function with $0$ as a regular value and $\Sigma$ a surface described as a connected component of the set of solutions $h(x,y,z)=0$.
Show that $\Sigma$ is orientable.
\end{thm}

Recall that any proper surface without boundary in the Euclidean space divides it into two connected components (\ref{clm:proper-divides}).
Therefore we can choose the unit normal field on any smooth proper surface that points into one of the components of the complement.
Therefore we obtain the following observation. 

\begin{thm}{Observation}
Any smooth proper surface in the Euclidean space is oriented.
\end{thm}

In particular it follows that the M\"obius strip cannot be extended to a proper smooth surface without boundary.

\section{Sections}

\begin{thm}{Advanced exercise}\label{ex:plane-section}
Let $\Pi$ be the $(x,y)$-plane and $A \subset \Pi$ be any closed subset. Show that there is an open smooth regular surface $\Sigma$ with $\Sigma \cap \Pi = A$.
\end{thm}

The exercise above says that plane sections of a smooth regular surface might look complicated.
The following lemma makes it possible to perturb the plane so that the section becomes nice.

\begin{thm}{Lemma}\label{lem:reg-section}
Let $\Sigma$ be a smooth regular surface.
Suppose $f\:\RR^3\z\to\RR$ is a smooth function.
Then for any constant $r_0$ there is an arbitrarily close value $r$ such that 
each connected component of the intersection of the level set $L_{r}=f^{-1}\{r\}$ with
$\Sigma$ is a smooth regular curve.
\end{thm}

\parit{Proof.}
The surface $\Sigma$ can be covered by a countable set of charts $s_i\:U_i\z\to \Sigma$.
Note that the composition $f\circ s_i$ is a smooth function for any $i$.
By Sard's lemma (\ref{lem:sard}), almost all real numbers $r$ are regular values for each $f\circ s_i$.

Fix such a value $r$ sufficiently close to $r_0$ and consider the level set $L_r$ described by the equation $f(x,y,z)=r$.
Any point in the intersection $\Sigma\cap L_r$ lies in the image of one of the charts.
From above it admits a neighborhood which is a regular smooth curve;
hence the result.\qeds

\begin{thm}{Corollary}
Let $\Sigma$ be a smooth surface.
Then for any plane $\Pi$ there is a parallel plane $\Pi^{*}$ that lies arbitrary close to $\Pi$ and such that the intersection $\Sigma\cap\Pi^{*}$ is a union of disjoint smooth curves.
\end{thm}



