\chapter{Definitions}

\section*{General definition}

Few times we will need the following general definition.

A path connected subset $\Sigma$ in a metric space is called \emph{surface} (more precisely \emph{embedded surface without boundary}) 
if any point of $p\in \Sigma$ admits a neighborhood $W$ in $\Sigma$ which is \emph{homeomorphic} to an open subset in the Euclidean plane;
that is, if there is an injective continuous map $U\to W$ from an open set $U\subset \RR^2$ such that its invese $W\to U$ is also continuous.

However, as well as in the case of curves we will be mostly interested in smooth surfaces in the Euclidean space describe in the following section.

\section*{Smooth surfaces}

Recall that a function $f$ of two variables $x$ and $y$ is called \emph{smooth} if all its partial derivatives $\frac{\partial^{m+n}}{\partial x^m\partial y^n}f$ are defined and are continuous in the domain of definition of $f$. 

A path connected set $\Sigma \subset \mathbb{R}^3$ is called a \emph{smooth surface} (or more precisely \emph{smooth regular embedded surface}) if it can be described locally as a graph of a smooth function in an appropriate coordinate system.\label{page:def-smooth-surface}

More precisely, for any point $p\in \Sigma$ one can choose a coordinate system $(x,y,z)$ and a neighborhood $U\ni p$ such that
the intersection $W=U\cap \Sigma$ is formed by a graph $z=f(x,y)$ of a smooth function $f$ defined in an open domain of the $(x,y)$-plane.

\parbf{Examples.}
The simplest example of a surface is the $(x,y)$-plane 
\[\Pi=\set{(x,y,z)\in\RR^3}{z=0}.\]
The plane $\Pi$ is a surface since
it can be described as the graph of the function $f(x,y)=0$.

All other planes are surfaces as well since one can choose a coordinate system so that it becomes the $(x,y)$-plane.
We can also present a plane as a graph of a linear function 
$f(x,y)=a\cdot x+b\cdot y+c$ for some constants $a$, $b$ and $c$
(assuming the plane is not perpendicular to the $(x,y)$-plane).

A more interesting example is the unit sphere 
\[\SS^2=\set{(x,y,z)\in\RR^3}{x^2+y^2+z^2=1}.\]
This set is not the graph of any function,
but $\SS^2$ is locally a graph;
in fact it can be covered by 6 graphs:
\begin{align*}
z&=f_\pm(x,y)=\pm \sqrt{1-x^2-y^2},
\\
y&=g_\pm(x,z)=\pm \sqrt{1-x^2-z^2},
\\
x&=h_\pm(y,z)=\pm \sqrt{1-y^2-z^2};
\end{align*}
each function $f_\pm,g_\pm,h_\pm$ is defined in an open unit disc.
That is, $\SS^2$ is a smooth surface.

\parbf{More conventions.}
If the surface $\Sigma$ is formed by a closed set, then it is called \emph{complete}. %change it to proper???
For example, paraboloids %???+PICs
\[z=x^2+y^2,\quad\quad z=x^2-y^2\]
or sphere 
\[x^2+y^2+z^2=1\]
are complete surfaces, while the
open disc in a plane 
\[\set{(x,y,z)\in\RR^3}{x^2+y^2<1, z=0}\]
is a surface which is not complete.

If moreover $\Sigma$ is a compact set, then it is called \emph{closed surface}
(the term is closely related to \emph{closed curve} but has nothing to do with \emph{closed set}).

If a complete surface $\Sigma$ is noncompact, then it is called  \emph{open surface} (again the term \emph{open set} is not relevant).

For example, paraboloids 
are open surfaces, 
and sphere is closed.

A closed subset in a surface that is bounded by one or more 
curves is called \emph{surface with boundary}; in this case the collection of curves is called the \emph{boundary line} of the surface.
When we say \emph{surface} we usually mean a surface without boundary;
we may use the term \emph{surface with possibly nonempty boundary} if we need to talk about surfaces with and without boundary.

\section*{Local parametrizations}

Let $U$ be an open domain in $\RR^2$ and $s\:U\to \RR^3$ be a smooth map.
We say that $s$ is regular if its Jacobian has maximal rank;
in this case it means that the vectors $\tfrac{\partial s}{\partial u}$ and $\tfrac{\partial s}{\partial v}$ are linearly independent at any $(u,v)\in U$ or
equivalently $\tfrac{\partial s}{\partial u}\times\tfrac{\partial s}{\partial v}\ne 0$, where $\times$ denotes the vector product.

\begin{thm}{Proposition}\label{prop:graph-chart}
If $s\:U\to \RR^3$ is a smooth regular embedding of an open connected set $U\subset \RR^2$, then its image $\Sigma=s(U)$ is a smooth surface.
\end{thm}

\parit{Proof.}
Let $s(u,v)=(s_1(u,v),s_2(u,v),s_3(u,v))$.
Since $s$ is regular the Jacobian matrix
\[
\renewcommand\arraystretch{1.3}
\begin{pmatrix}
\tfrac{\partial s_1}{\partial u}&\tfrac{\partial s_1}{\partial v}\\
\tfrac{\partial s_2}{\partial u}&\tfrac{\partial s_2}{\partial v}\\
\tfrac{\partial s_3}{\partial u}&\tfrac{\partial s_3}{\partial v}
\end{pmatrix}
\]
has rank two.

Fix a point $p\in \Sigma$; by shifting the coordinate system we may assume that $p$ is the origin.
Permuting the coordinates $x,y,z$ if necessary, we may assume that 
the matrix 
\[
\renewcommand\arraystretch{1.3}
\begin{pmatrix}
\tfrac{\partial s_1}{\partial u}&\tfrac{\partial s_1}{\partial v}\\
\tfrac{\partial s_2}{\partial u}&\tfrac{\partial s_2}{\partial v}
\end{pmatrix}
\]
is invertible.
Let $\bar s\:U\to\RR^2$ be the projection of $s$ to the $(x,y)$-coordinate plane;
that is, $\bar s(u,v)=(s_1(u,v),s_2(u,v))$.
Note that the $2\times2$-matrix above is the Jacobian matrix of $\bar s$.

The inverse function theorem implies that there is a smooth regular function $h$ defined on an open set $W\ni 0$ in the $(x,y)$-plane
such that $h(0,0)=(0,0)$ and $\bar s\circ h$ is the identity map.

Note that the graph $z=s_3\circ h(x,y)$ for $(x,y)\in W$ is a subset in $\Sigma$.
Indeed, if $(u,v)\z=h(x,y)$, then $x=s_1(u,v)$ and $y=s_2(u,v)$.
Therefore the identity $z=s_3\circ h(x,y)$ can be rewritten as $(x,y,z)\z=s(u,v)$.

Clearly the graph $z=s_3\circ h(x,y)$ for $(x,y)\in W$ is open in $\Sigma$; %???why
that is, $\Gamma$ a neighborhood of $p$ in $\Sigma$ that can be described as a graph of a smooth function $f_3\circ h\:W\to\RR$.
Since $p$ is arbitrary, we get that $\Sigma$ is a surface.
\qeds

If $s$ and $\Sigma$ as in the proposition, then we say that $s$ is a \emph{parametrization} of the surface $\Sigma$. 

Not all the smooth surfaces can be described by such a parametrization;
for example the sphere $\SS^2$ cannot.
But any smooth surface $\Sigma$ admits a local parametrization; that is, any point $p\in\Sigma$ admits an open neighborhood $W\subset \Sigma$ with a smooth regular parametrization~$s$.
In this case any point in $W$ can be described by two parameters, usually denoted by $u$ and $v$, 
which are called \emph{local coordinates} at $p$.
The map $s$ is called a \emph{chart} of $\Sigma$.

If $W$ is a graph $z=h(x,y)$ then the map $s\:(u,v)\mapsto (u,v,h(u,v))$ is a chart.
Indeed, $s$ has an inverse $(u,v,h(u,v))\mapsto (u,v)$ which is continuous;
that is, $s$ is an embedding.
Further,
$\tfrac{\partial s}{\partial u}=(1,0,\tfrac{\partial h}{\partial u})$ and $\tfrac{\partial s}{\partial v}=(0,1,\tfrac{\partial h}{\partial v})$, whence $\tfrac{\partial s}{\partial u}$ and $\tfrac{\partial s}{\partial v}$ are linearly independent.

Note that from \ref{prop:graph-chart}, we obtain the following corollary.

\begin{thm}{Corollary}
A path connected set $\Sigma\subset \RR^3$ is a smooth regular surface if at any point $p\in \Sigma$ it has a local parametrization by a smooth regular map.
\end{thm}


\begin{thm}{Exercise}\label{ex:inversion}
Consider the following map 
\[s(u,v)=(\tfrac{2\cdot u}{1+u^2+v^2},\tfrac{2\cdot v}{1+u^2+v^2},\tfrac{2}{1+u^2+v^2}).\]
Show that $s$ is a chart of the unit sphere centered at $(0,0,1)$; describe the image of $s$.
\end{thm}

The map 
\[(u,v,1)\mapsto (\tfrac{2\cdot u}{1+u^2+v^2},\tfrac{2\cdot v}{1+u^2+v^2},\tfrac{2}{1+u^2+v^2})\]
is called \emph{stereographic projection}. 
Note that the point $(u,v,1)$ and its image $(\tfrac{2\cdot u}{1+u^2+v^2},\tfrac{2\cdot v}{1+u^2+v^2},\tfrac{2}{1+u^2+v^2})$ lie on one half-line starting at the origin.

\begin{wrapfigure}{r}{40 mm}
\vskip-4mm
\centering
\includegraphics{asy/torus}
\vskip-3mm
\end{wrapfigure}

Let $\gamma(t)=(x(t),y(t))$ be a plane curve.
Recall that the image of the map 
\[(t,\theta)\mapsto (x(t), y(t)\cdot\cos\theta,y(t)\cdot\sin\theta)\] 
is called \emph{surface of revolution} of the curve $\gamma$ around $x$-axis.
For fixed $t$ or $\theta$ the obtained curves are called meridian or correspondingly parallel of the surface of revolution; note that parallels are formed by circles in the plane perpendicular to the axis of rotation.

\begin{thm}{Exercise}\label{ex:revolution}
Assume $\gamma$ is a closed simple smooth regular plane curve that does not intersect $x$-axis.
Show that surface of revolution of $\gamma$ around $x$-axis is a smooth regular surface.
\end{thm}


\section*{Golbal parametrizations} 
A surface can be described by an embedding from a known surface to the space.
For example the ellipsoid
\[\Sigma_{a,b,c}=\set{(x,y,z)\in\RR^3}{\tfrac{x^2}{a^2}+\tfrac{y^2}{b^2}+\tfrac{z^2}{c^2}=1}\]
for some positive numbers $a,b,c$ can be defined as the image of the map $s\:\SS^2\to\RR^3$, defined as the restriction of the map $(x,y,z)\z\mapsto (a\cdot x, b\cdot y,c\cdot z)$ to the unit sphere $\SS^2$.

For a surface $\Sigma$, a map $s: \Sigma \to \RR ^3$ is called a 
\emph{smooth parametrized surface} if for any chart $f\:U\to \Sigma$ 
the composition $s\circ f$ is smooth and regular;
that is, all partial derivatives $\frac{\partial^{m+n}}{\partial u^m\partial v^n}(s\circ f)$ exist and are continuous in the domain of definition and the following two vectors 
$\frac{\partial}{\partial u}(s\circ f)$ and $\frac{\partial}{\partial v}(s\circ f)$ are linearly independent.

Evidently the parametric definition includes the embedded surfaces defined previously --- as the domain of parameters we can take the surface itself and the identity map as $s$.
But parametrized surfaces are more general, in particular they  might  have self-intersections.

If $\Sigma$ is a known surface for example a sphere or a plane, the paramtrized surface $s\:\Sigma\to\RR^3$ might be called by the same name.
For example, any embedding $s\:\SS^2\to\RR^3$ might be called topological sphere
and if $s$ is smooth and regular, then it might be called smooth sphere.
(A smooth regular map $s\:\SS^2\to\RR^3$ which is not necessary an embedding is called \emph{smooth regular immersion}, so we can say that $s$ describes a smooth immersed sphere.) 
Similarly an embedding $s\:\RR^2\to\RR^3$ might be called topological plane
and if $s$ is smooth it might be called smooth plane.

\section*{Implicitly defined surfaces}

\begin{thm}{Proposition}
Let $f\:\RR^3\to \RR$ be a smooth function.
Suppose that $0$ is a regular value of $f$;
that is, $\nabla_p f\ne 0$ if $f(p)=0$.
Then any path connected component $\Sigma$ of the set of solutions of the equation $f(x,y,z)=0$ is a surface.
\end{thm}

\parit{Proof.}
Fix $p\in\Sigma$.
Since $\nabla_p f\ne 0$ we have 
\[\tfrac{\partial f}{\partial x}(p)\ne 0,\quad \tfrac{\partial f}{\partial y}(p)\ne 0,\quad \text{or}\quad\tfrac{\partial f}{\partial z}(p)\ne 0.\]
We may assume $\tfrac{\partial f}{\partial z}(p)\ne 0$;
otherwise permute the coordinates $x,y,z$.

The implicit function theorem (\ref{thm:imlicit}) implies that a neighborhood of $p$ in $\Sigma$ is the graph $z=h(x,y)$ of a smooth function $h$ defined on an open domain in $\RR^2$.
It remains to apply the definition of smooth surface (page \pageref{page:def-smooth-surface}).
\qeds

\begin{thm}{Exercise}\label{ex:hyperboloinds}
Describe the set of real numbers $a$
such that the equation
\begin{align*}
x^2+y^2-z^2&=a
\end{align*}
describes a smooth regular surface.
\end{thm}

\section*{Tangent plane}

Let $z=f(x,y)$ be a local graph realization of a surface. 
Assume that a point $p=(x_p,y_p,z_p)$ lies on this graph, so $z_p=f(x_p,y_p)$.
The plane spanned by the vectors $(1,0,(\tfrac{\partial}{\partial x}f)(x_p,y_p))$ and  $(0,1,(\tfrac{\partial}{\partial y}f)(x_p,y_p))$ is called the \emph{tangent plane} of $\Sigma$ at $p$.
The tangent plane to $\Sigma$ at $p$ is usually denoted by $\T_p$ or $\T_p\Sigma$.
Vectors in $\T_p$ are called \emph{tangent vectors} of $\Sigma$ at $p$. %??? redo def

Tangent plane $\T_p$ might be considered as a linear subspace of $\RR^3$ or as a parallel plane passing thru $p$.
In the latter case it can be interpreted as the best approximation of the surface $\Sigma$ by a plane at~$p$;
it has \emph{first order of contact} with $\Sigma$ at $p$;
that is, $\rho(q)\z=o(|p-q|)$, where $q\in \Sigma$ and $\rho(q)$ denotes the distance from $q$ to $\T_p$.

\begin{thm}{Proposition}
Let $\Sigma$ be a smooth surface.
A vector $w$ is a tangent vector of $\Sigma$ at $p$ if and only if there is a curve $\gamma$ that runs in $\Sigma$ and has $w$ as a velocity vector at $p$.  
\end{thm}

Note that according to the proposition the tangent plane $\T_p\Sigma$ can be defined as the set of all velocity vectors at $p$ of smooth parameterized curves that run in $\Sigma$. 
In particular the tangent plane to a surface at a given point does not depend on the choice of its local graph representation.

\parit{Proof.}
We can assume that $\Sigma$ is a graph $z=f(x,y)$; 
otherwise pass to a local presentation of $\Sigma$ around $p$.

\parit{``Only if'' part.}
Suppose that $(x(t),y(t))$ denotes the projection of $\gamma(t)$ to the $(x,y)$-plane.
Since $\gamma$ runs in $\Sigma$, we have that
\[\gamma(t)=\bigl(x(t),y(t),f(x(t),y(t))\bigr).\]
Therefore 
\begin{align*}
\gamma'&=(x',y',\tfrac{\partial f}{\partial x}(x,y)\cdot x'+\tfrac{\partial f}{\partial y}(x,y)\cdot y')=
\\
&=x'\cdot (1,0,(\tfrac{\partial}{\partial x}f)(x,y))+y'\cdot (0,1,(\tfrac{\partial}{\partial y}f)(x,y)).
\end{align*}
That is, $\gamma'(t)\in\T_{\gamma(t)}$ for any $t$.

\parit{``If'' part.}
Without loss of generality we can assume that $p$ is the origin.
Fix a tangent vector 
\[w=a\cdot (1,0,(\tfrac{\partial}{\partial x}f)(0,0))+b\cdot(0,1,(\tfrac{\partial}{\partial y}f)(0,0))\] 
and consider the curve $\gamma(t)=(a\cdot t, b\cdot t, f(a\cdot t,b\cdot t))$.
By construction $\gamma$ runs in $\Sigma$ and the direct calculations show that $\gamma'(0)=w$.
\qeds

\begin{thm}{Exercise}\label{ex:tangent-chart}
Assume $f\:U\to\RR^3$ is a smooth regular chart of a surface $\Sigma$ and $p=f(u_0,v_0)$.
Show that the tangent plane $\T_p\Sigma$ is spanned by vectors $\tfrac{\partial f}{\partial u}(u_0,v_0)$ and $\tfrac{\partial f}{\partial v}(u_0,v_0)$.
\end{thm}

\begin{thm}{Exercise}\label{ex:tangent-normal}
Let $f:\RR^3\to\RR$ be a smooth function with a regular value $0$ and $\Sigma$ is a surface described as a connected component of the set of solutions $f(x,y,z)=0$.
Show that the tangent plane $\T_p\Sigma$ is perpendicular to the gradient $\nabla_pf$ at any point $p\in\Sigma$.
\end{thm}

\begin{thm}{Exercise}\label{ex:vertical-tangent}
Let $\Sigma$ be a smooth surface and $p\in\Sigma$.
Fix an $(x,y,z)$-coordinates.
Show that a neighborhood of $p$ in $\Sigma$ is a graph $z=f(x,y)$ of a smooth function $f$ defined on an open subset in $(x,y)$-plane if and only if the tangent plane $\T_p$ is not a \emph{vertical plane}; that is if the projection of $\T_p$ to $(x,y)$-plane does not degenerates to a line.
\end{thm}



\section*{Normal vector and orientation}
A unit vector that is normal to $\T_p$ is usually denoted by $\nu_p$;
it is uniquely defined up to sign.

A surface $\Sigma$ is called \emph{oriented} if it is equipped with a unit normal vector field $\nu$;
that is, a continuous map $p\mapsto \nu_p$ such that $\nu_p\perp\T_p$ and $|\nu_p|=1$ for any $p$.
The choice of the field $\nu$ is called \emph{orientation} on $\Sigma$.
A surface $\Sigma$ is called \emph{orientable} if it can be oriented.
Note that each orientable surface admits two orientations $\nu$ and $-\nu$. %???

\begin{wrapfigure}{r}{40 mm}
\vskip-0mm
\centering
\includegraphics{asy/moebius}
\vskip0mm
\end{wrapfigure}

M\"obius strip shown on the diagram gives an example of nonorientable surface --- there is no choice of normal vector field that is continuous along the middle of the strip, 
when you go around it changes the sign.

Note that each surface is locally orientable.
In fact each chart $f(u,v)$ admits an orientation 
\[\nu=
\frac{\tfrac{\partial f}{\partial u}\times \tfrac{\partial f}{\partial v}}
{\left|\tfrac{\partial f}{\partial u}\times \tfrac{\partial f}{\partial v}\right|}.\]
Indeed the vectors $\tfrac{\partial f}{\partial u}$ and $\tfrac{\partial f}{\partial v}$ are tangent vectors at $p$; 
since they are linearly independent, their vector product does not vanish and it is perpendicular to the tangent plane.
Therefore $\nu(u,v)$ is a unit normal vector at $f(u,v)$;
evidently $(u,v)\mapsto \nu(u,v)$ is a continuous map. 

\begin{thm}{Exercise}
Let $h:\RR^3\to\RR$ be a smooth function with a regular value $0$ and $\Sigma$ is a surface described as a connected component of the set of solutions $h(x,y,z)=0$.
Show that $\Sigma$ is orientable.
\end{thm}

\parit{Hint.} Show that $\nu=\tfrac{\nabla h}{|\nabla h|}$ defines a unit normal field on $\Sigma$.

The following claim should be intuitively obvious. 
Its proof is not at all trivial; a standard proof use the so called \emph{Alexander's duality} which is a classical technique in algebraic topology.

\begin{thm}{Claim}
The complement of any complete surface in the Euclidean space has exactly two connected components. 
\end{thm}

Note that if a complete surface is smooth, we can choose the unit normal filed on it that points into one of the components of the complement.
Therefore we obtain the following observation. 

\begin{thm}{Observation}
Any smooth complete surface in Euclidean space is oriented.
\end{thm}

In particular it follows that the M\"obius strip cannot be extended to a complete smooth surface.

\section*{Plane sections}

\begin{thm}{Advanced exercise}
Show that any closed set in the plane can appear as an intersection of this plane with a complete smooth regular surface.  
\end{thm}

As a consequence of the exercise above, the section of a smooth regular surface by a plane might look complicated.
The following lemma makes possible to perturb the plane so that the section becomes nice. 

\begin{thm}{Lemma}\label{lem:reg-section}
Let $\Sigma$ be a smooth regular surface.
Then for any plane $\Pi$ there is arbitrarily close parallel plane $\Pi'$ such that 
each connected component of the intersection $\Sigma\cap \Pi'$ is a smooth regular curve.
\end{thm}

\parit{Proof.}
Assume $\Pi$ is described by equation $f(x,y,z)=r_0$, where
\[f(x,y,z)=a\cdot x+b\cdot y+c\cdot z.\] 
The surface $\Sigma$ can be covered by a countable set of charts $s_i\:U_i\to \Sigma$.
Note that the composition $f\circ s_i$ is a smooth function.
By Sard's lemma (\ref{lem:sard}), almost all real numbers $r$ are regular values of each $f\circ s_i$.

Fix such value $r$ sufficiently close to $r_0$ and consider the plane $\Pi'$ described by the equation $f(x,y,z)=r$.
Note that $\Pi'\parallel \Pi$ and arbitrary close to it.
Any point in the intersection $\Sigma\cap\Pi'$ lies in the image of one of the charts.
From above it admits a neighborhood which is a regular smooth curve;
hence the result.\qeds


