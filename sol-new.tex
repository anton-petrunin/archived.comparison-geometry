\chapter{Semisolutions}

\parbf{Exercise \ref{ex:9}.} The image of $\gamma$ might have a shape of digit $8$ or $9$.

\parbf{Exercise \ref{ex:integral-length}.} For (\ref{ex:integral-length>}), apply the fundamental theorem of calculus for each segment in a given partition. For (\ref{ex:integral-length<}) consider a partition such that the velocity vector $\alpha'(t)$ is nearly constant on each of its segments.

\parbf{Advanced exercise \ref{adex:integral-length}.} Use theorems of Rademacher and Lusin (\ref{thm:rademacher} and \ref{thm:lusin}).

\parbf{Exercise \ref{ex:curvature-of-spherical-curve}.} Differentiate the identity $\langle\gamma(s),\gamma(s)\rangle=1$ a couple of times.

\parbf{Exercise \ref{ex:curvature-formulas}.} Prove and use the following identities: 
\begin{align*}
\gamma''(t)-\gamma''(t)^\perp&=\tfrac{\gamma'(t)}{|\gamma'(t)|}\cdot\langle\gamma''(t),\tfrac{\gamma'(t)}{|\gamma'(t)|}\rangle,
\\
|\gamma'(t)|&=\sqrt{\langle \gamma'(t),\gamma'(t)\rangle}.\
\end{align*}

\parbf{Advanced exercise \ref{ex:gamma/|gamma|}.} Assume that $\gamma$ is unit-speed; show that $|\sigma'|\le \kur+\theta'$, where $\theta(s)=\angle(\gamma(s),\gamma'(s))$.

\parbf{Exercise \ref{ex:monotonic-tc}.}
Use that exterior angle of a triangle equals to the sum of the two remote interior angles;
for the second part apply the induction on number of vertexes.

\parbf{Exercise \ref{ex:length-dist}.}
Choose a value $s_0\in[a,b]$ that splits the total curvature into two equal parts, $\theta$ in each.
Observe that $\measuredangle(\gamma'(s_0),\gamma'(s))\le \theta$ for any~$s$.
Use this inequality the same way as in the proof of the bow lemma.

\parbf{Exercise \ref{ex:tc-semicontinuous}.} Modify the proof of semi-continuity of length (\ref{thm:length-semicont}).

\parbf{Exercise \ref{ex:torsion=0}.}
Show and use that the binormal vector is constant.

\parbf{Exercise \ref{ex:evolvent-constant-slope}.}
Show that $\langle w,\alpha\rangle$ is constant if $\gamma$ makes constant angle with a fixed vector $w$ and $\alpha$ is the evolvent of $\gamma$.

\parbf{Exercise \ref{ex:cur+tor=helix}.} Use the second statement in \ref{ex:helix-torsion}.

\parbf{Advanced exercise \ref{ex:const-dist}.} Note that the function
\[\rho(\ell)=|\gamma(t+\ell)-\gamma(t)|^2=\langle \gamma(t+\ell)-\gamma(t),\gamma(t+\ell)-\gamma(t)\rangle\] 
is smooth and does not depend on $t$.
Express speed, curvature and torsion of $\gamma$ in terms of derivatives $\rho^{(n)}(0)$
and apply \ref{ex:cur+tor=helix}.

\parbf{Exercise \ref{ex:vertex-support}.} Apply the spiral lemma (\ref{lem:spiral}).

\begin{wrapfigure}{r}{35 mm}
\vskip1mm
\centering
\includegraphics{mppics/pic-66}
\vskip0mm
\end{wrapfigure}

\parbf{Exercise \ref{ex:between-parallels-1}.}
 Note that the curve lies in a figure $F$ as on the diagram.
More precisely, $F$ is formed by a rectangle with pair of bases on the lines and two half discs attached to the sides of length $2$.
Look at the right most position of $F$ that still contains the curve.

\parbf{Exercise \ref{ex:convex small}.}
Note that we can assume that $\gamma$ bounds a convex figure $F$, otherwise by \ref{prop:convex} its curvature changes the sign and therefore it has zero curvature at some point.
Choose two points $x$ and $y$ surrounded by $\gamma$ such that $|x-y|>2$,
look at the maximal lens bounded by two arcs with common chord $xy$ that lies in $F$ and apply supporting test (\ref{prop:supporting-circline}).

\parbf{Exercise \ref{ex:moon-rad}.} Note that $\gamma$ contains a simple loop; apply to it \ref{thm:moon-gen}.

\parbf{Exercise \ref{ex:moon-rad}.} Note that $\gamma$ contains a simple loop; apply to it \ref{thm:moon-gen}.

\parbf{Exercise \ref{ex:curve-crosses-circle}.}
Repeat the proof of theorem for each cyclic concatenation of an arc of $\gamma$ from $p_i$ to $p_{i+1}$ with large arc of the circle. 

\parbf{Exercise \ref{ex:inverse}.} Use the definition of osculating circle via order of contact and that inversion maps circles to circlines. 

\parbf{Exercise \ref{ex:implicit-orientable}.} Show that $\Norm=\tfrac{\nabla h}{|\nabla h|}$ defines a unit normal field on $\Sigma$.

\parbf{Exercise \ref{ex:moon-revolution}.} Use \ref{ex:line-of-curvature} and \ref{thm:moon-orginal}.

\parbf{Exercise \ref{ex:positive-gauss}.}
Consider the minimal sphere that encloses the surface.

\parbf{Exercise \ref{ex:surrounds-disc}.}
Look for a supporting spherical dome with the unit circle as the boundary.

\parbf{Exercise \ref{ex:principle-revolution}.} Use \ref{ex:line-of-curvature} and \ref{cor:meusnier}.

\parbf{Exercise \ref{ex:convex-lagunov}.}
Assume a maximal ball in $V$ touches its boundary at the points $p$ and $q$.
Consider the projection of $V$ to a plane thru $p$, $q$ and the center of the ball. 

\parbf{Exercise \ref{ex:lagunov-genus4}.} Drill an extra hole or combine two examples together.

\parbf{Exercise \ref{ex:convex-revolution}.} Use \ref{ex:principle-revolution}.

\parbf{Exercise \ref{ex:ruled=>saddle}.} Prove and use that each point $p\in\Sigma$ has a direction with vanishing normal curvature.

\parbf{Exercise \ref{ex:length-of-bry}.} Use the \ref{lem:convex-saddle} and the hemisphere lemma (\ref{lem:hemisphere}).

\parbf{Exercise \ref{ex:disc-hat}.} Observe that it is sufficient to construct a smooth parametrization of $\Delta_\eps$ by a closed hemisphere.
To do this repeat the argument in \ref{lem:gauss=sphere} with the center at a point surrounded by the boundary line of $\Delta_\eps$ in its plane.

\parbf{Exercise \ref{ex:between-parallels}.} Look for an example among the surfaces of revolution and use \ref{ex:principle-revolution}.

\parbf{Exercise \ref{ex:one-side-bernshtein}.} Look at the sections of the graph by planes parallel to the $(x,y)$-plane and to the $(x,z)$-plane, then apply Meusnier’s theorem apply Meusnier's theorem (\ref{cor:meusnier}).

\parbf{Exercise \ref{ex:convex-surf}.} Show and use that any tangent plane $\T_p$ supports $\Sigma$ at $p$.

\parbf{Exercise \ref{ex:small-gauss}.}
Note that we can assume that the surface has positive Gauss curvature, otherwise the statement is evident.
Therefore the surface bounds a convex region that contains a line segment of length~$\pi$.

Observe that the Gauss curvature of the surface of revolution of the graph $y=a\cdot \sin x$ for $x\in(0,\pi)$ cannot exceed $1$ (Use \ref{ex:curvature-graph} and \ref{cor:meusnier}).
Try to support the surface $\Sigma$ from inside by a surface of revolution of the described type with large $R$. 

\begin{figure}[h!]
\vskip-0mm
\centering
\includegraphics{asy/sin}
\vskip-0mm
\end{figure}

\parbf{Exercise \ref{ex:lasso}.} Cut the lateral surface of the mountain by a line from the cowboy to the top, unfold it on the plane and try to figure out what is the image of the strained lasso.

\parbf{Advanced exercise \ref{ex:milka}.} Show that the concatenation of the line segment $[p_t,\gamma(t)]$ and the arc $\gamma|_{[t,\ell]}$ is a shortest path in the closed region $W$ outside of $\Sigma$.

\parbf{Exercise \ref{ex:rough-bound-mountain}.} Use \ref{thm:usov} and \ref{ex:sef-intersection}.

The suggested argument does not give the optimal bound for the Lipschitz constant that guarantees that $\gamma$ is simple, but
later (see \ref{ex:sqrt(3)}) we will show that the exact bound is $\sqrt{3}=\tg\tfrac\pi3$ --- it is the same as in the exercise about mountain of with the shape of a perfect cone; see \ref{ex:lasso}.

\parbf{Exercise \ref{ex:shape-curvature-line}.}  Denote by $\Norm_1(t)$ and $\Norm_2(t)$ the unit normal vectors to $\Sigma_1$ and $\Sigma_2$ at $\gamma(t)$.
Note that $\langle \Norm_1(t),\Norm_2(t)\rangle$ is constant; take it derivative and apply \ref{thm:rodrigues}.

\parbf{Exercise \ref{ex:gauss-integral-open}.} Use \ref{lem:graph}.

\parbf{Exercise \ref{ex:parallel-transport-equator}.} Denote by $w$ the pole of the equator; show and use that $w$ is a parallel vector field along $\gamma$.

\parbf{Exercise \ref{ex:aabbccdd}.} Estimate integral of Gauss curvature bounded by a simple geodesic loop and apply \ref{ex:int-gauss=4pi}.

\parbf{Exercise \ref{ex:sqrt(3)}.} Note that it is sufficient to show that the surface has no geodesic loops.
Estimate the integral of Gauss curvature of whole surface and a disc in it surrounded by a geodesic loop.



\parbf{Exercise \ref{ex:intrinsic-diameter}.} Use \ref{lem:closest-point-projection}.



\parbf{Exercise \ref{ex:geod-semigeod}.} Note that in order to show that $\gamma''_t(s)\perp\T_{\gamma_t(s)}$, it is sufficient to show that $\langle\tfrac{\partial^2}{\partial s^2}w,\tfrac{\partial}{\partial t}w\rangle=0$.



%\parbf{Exercise \ref{}.}

%\parbf{Exercise \ref{}.}

%\parbf{Exercise \ref{}.}

%\parbf{Exercise \ref{}.}
