\parbf{\ref{ex:9}.} The image of $\gamma$ might have a shape of digit $8$ or $9$.

\parbf{\ref{aex:simple-curve}.}

\parbf{\ref{ex:L-shape}.}

\parbf{\ref{ex:cycloid}.}

\parbf{\ref{ex:y^2=x^3}.}

\parbf{\ref{ex:viviani}.}

\parbf{\ref{ex:proper-curve}.}

\parbf{\ref{ex:integral-length}.}

\parbf{\ref{ex:length-image}.}

\parbf{\ref{ex:integral-length}.} For (\ref{ex:integral-length>}), apply the fundamental theorem of calculus for each segment in a given partition. For (\ref{ex:integral-length<}) consider a partition such that the velocity vector $\alpha'(t)$ is nearly constant on each of its segments.

\parbf{\ref{adex:integral-length}.} Use theorems of Rademacher and Lusin (\ref{thm:rademacher} and \ref{thm:lusin}).

\parbf{\ref{ex:nonrectifiable-curve}.}

\parbf{\ref{ex:arc-length-helix}.}

\parbf{\ref{ex:convex-hull}.}

\parbf{\ref{ex:convex-croftons}.}

\parbf{\ref{adex:more-croftons}.}

\parbf{\ref{ex-def:length-metric}.}

\parbf{\ref{ex:intrinsic-convex}.}

\parbf{\ref{ex:antipodal}.}

\parbf{\ref{ex:bisection-of-S2}.}

\parbf{\ref{ex:flaw}.}

\parbf{\ref{adex:crofton}.}


\parbf{\ref{ex:curvature-of-spherical-curve}.} Differentiate the identity $\langle\gamma(s),\gamma(s)\rangle=1$ a couple of times.

\parbf{\ref{ex:curvature-formulas}.} Prove and use the following identities: 
\begin{align*}
\gamma''(t)-\gamma''(t)^\perp&=\tfrac{\gamma'(t)}{|\gamma'(t)|}\cdot\langle\gamma''(t),\tfrac{\gamma'(t)}{|\gamma'(t)|}\rangle,
\\
|\gamma'(t)|&=\sqrt{\langle \gamma'(t),\gamma'(t)\rangle}.\
\end{align*}

\parbf{\ref{ex:curvature-graph}.}

\parbf{\ref{ex:helix-curvature}.}

\parbf{\ref{ex:length>=2pi}.}

\parbf{\ref{ex:gamma/|gamma|}.} Assume that $\gamma$ is unit-speed; show that $|\sigma'|\le \kur+\theta'$, where $\theta(s)=\angle(\gamma(s),\gamma'(s))$.

\parbf{\ref{ex:chord-lemma-optimal}.}

\parbf{\ref{ex:monotonic-tc}.}
Use that exterior angle of a triangle equals to the sum of the two remote interior angles;
for the second part apply the induction on number of vertexes.

\parbf{\ref{ex:sef-intersection}.}

\parbf{\ref{ex:quadrisecant}.}

\parbf{\ref{ex:anti-bow}.}

\parbf{\ref{ex:length-dist}.}
Choose a value $s_0\in[a,b]$ that splits the total curvature into two equal parts, $\theta$ in each.
Observe that $\measuredangle(\gamma'(s_0),\gamma'(s))\le \theta$ for any~$s$.
Use this inequality the same way as in the proof of the bow lemma.

\parbf{\ref{ex:schwartz}.}

\parbf{\ref{ex:loop}.}

\parbf{\ref{ex:tc-semicontinuous}.} Modify the proof of semi-continuity of length (\ref{thm:length-semicont}).

\parbf{\ref{ex:gen-fenchel}.}

\parbf{\ref{ex:tc-length}.}

\parbf{\ref{ex:helix-torsion}.}

\parbf{\ref{ex:beta-from-tau+nu}.}

\parbf{\ref{ex:torsion=0}.}
Show and use that the binormal vector is constant.

\parbf{\ref{ex:frenet}.}

\parbf{\ref{ex:lancret}.}

\parbf{\ref{ex:evolvent-constant-slope}.}
Show that $\langle w,\alpha\rangle$ is constant if $\gamma$ makes constant angle with a fixed vector $w$ and $\alpha$ is the evolvent of $\gamma$.

\parbf{\ref{ex:spherical-frenet}.}

\parbf{\ref{ex:cur+tor=helix}.} Use the second statement in \ref{ex:helix-torsion}.

\parbf{\ref{ex:const-dist}.} Note that the function
\[\rho(\ell)=|\gamma(t+\ell)-\gamma(t)|^2=\langle \gamma(t+\ell)-\gamma(t),\gamma(t+\ell)-\gamma(t)\rangle\] 
is smooth and does not depend on $t$.
Express speed, curvature and torsion of $\gamma$ in terms of derivatives $\rho^{(n)}(0)$
and apply \ref{ex:cur+tor=helix}.

\parbf{\ref{ex:bike}.}

\parbf{\ref{ex:trochoids}.}

\parbf{\ref{ex:zero-tsc}.}

\parbf{\ref{ex:length'}.}

\parbf{\ref{ex:inverse}.}
Use the definition of osculating circle via order of contact and that inversion maps circles to circlines. 

\parbf{\ref{ex:evolute-of-ellipse}.}

\parbf{\ref{ex:3D-spiral}.}

\parbf{\ref{ex:double-tangent}.}

\parbf{\ref{ex:vertex-support}.} Apply the spiral lemma (\ref{lem:spiral}).

\parbf{\ref{ex:support}.}

\parbf{\ref{ex:in-circle}.}

\begin{wrapfigure}{r}{35 mm}
\vskip1mm
\centering
\includegraphics{mppics/pic-66}
\vskip0mm
\end{wrapfigure}

\parbf{\ref{ex:between-parallels-1}.}
Note that $\gamma$ lies in a figure $F$ as on the diagram.
More precisely, $F$ is formed by a rectangle with pair of bases on the lines and two half discs attached to the sides of length $2$.

Look at the right most or left most position of $F$ that still contains curve loop.

To do the second part, try to modify the suggested proof.

\parbf{\ref{ex:in-triangle}.}

\parbf{\ref{ex:lens}.}

\parbf{\ref{ex:convex small}.}
Note that we can assume that $\gamma$ bounds a convex figure $F$, otherwise by \ref{prop:convex} its curvature changes the sign and therefore it has zero curvature at some point.
Choose two points $x$ and $y$ surrounded by $\gamma$ such that $|x-y|>2$,
look at the maximal lens bounded by two arcs with common chord $xy$ that lies in $F$ and apply supporting test (\ref{prop:supporting-circline}).

\parbf{\ref{ex:line-curve-intersections}.}

\parbf{\ref{ex:moon-rad}.} Note that $\gamma$ contains a simple loop; apply to it \ref{thm:moon-orginal}.

\parbf{\ref{ex:curve-crosses-circle}.}
Repeat the proof of theorem for each cyclic concatenation of an arc of $\gamma$ from $p_i$ to $p_{i+1}$ with large arc of the circle. 

\parbf{\ref{ex:hyperboloinds}.}

\parbf{\ref{ex:inversion-chart}.}

\parbf{\ref{ex:revolution}.}

\parbf{\ref{ex:tangent-normal}.}

\parbf{\ref{ex:vertical-tangent}.}


\parbf{\ref{ex:implicit-orientable}.} Show that $\Norm=\tfrac{\nabla h}{|\nabla h|}$ defines a unit normal field on $\Sigma$.

\parbf{\ref{ex:plane-section}.}

\parbf{\ref{ex:line-of-curvature}.}

\parbf{\ref{ex:gauss+orientable}.}

\parbf{\ref{ex:normal-curvature=const}.}

\parbf{\ref{ex:shape-curvature-line}.}  Denote by $\Norm_1(t)$ and $\Norm_2(t)$ the unit normal vectors to $\Sigma_1$ and $\Sigma_2$ at $\gamma(t)$.
Note that $\langle \Norm_1(t),\Norm_2(t)\rangle$ is constant; take it derivative and apply \ref{thm:rodrigues}.

\parbf{\ref{ex:meusnier}.}

\parbf{\ref{ex:principle-revolution}.} Use \ref{ex:line-of-curvature} and \ref{cor:meusnier}.

\parbf{\ref{ex:catenoid-is-minimal}.}

\parbf{\ref{ex:helicoid-is-minimal}.}

\parbf{\ref{ex:moon-revolution}.} Use \ref{ex:line-of-curvature} and \ref{thm:moon-orginal}.

\parbf{\ref{ex:lagunov-genus4}.} Drill an extra hole or combine two examples together.

\parbf{\ref{ex:thin}.}

\parbf{\ref{ex:PI-sphere}.}

\parbf{\ref{ex:divergence-1}.}

\parbf{\ref{ex:divergence-2}.}

\parbf{\ref{ex:mean-convex}.} ???
There are mean-convex bodies bodies $V$ that are not star-shaped for which the conclusion of the exercise does not hold.
For example they can be found among bodies of revolution as shown on the picture --- one only has to check that mirror-symmetric figures $F$ and $F'$ as on the picture can be chosen in such a way that the body of revolution of $F$ is mean-convex wile the body of revolution of $F'$ has smaller surface area. 

A proof of this stronger statement can be build on an analog of comparison theorem \ref{thm:comp}(\ref{thm:comp:toponogov}) for surfaces with Gauss curvature at least~1.

\parbf{\ref{ex:catenoid-nonmin}.}

\parbf{\ref{ex:helicoid-nonmin}.}

\parbf{\ref{ex:surf-support}.}

\parbf{\ref{ex:positive-gauss-0}.}

\parbf{\ref{ex:positive-gauss}.}
Consider the minimal sphere that encloses the surface.

\parbf{\ref{ex:convex-surf}.} Show and use that any tangent plane $\T_p$ supports $\Sigma$ at $p$.

\parbf{\ref{ex:convex-lagunov}.}
Assume a maximal ball in $V$ touches its boundary at the points $p$ and $q$.
Consider the projection of $V$ to a plane thru $p$, $q$ and the center of the ball. 

\parbf{\ref{ex:section-of-convex}.}

\parbf{\ref{ex:surrounds-disc}.}
Look for a supporting spherical dome with the unit circle as the boundary.

\parbf{\ref{ex:small-gauss}.}
Note that we can assume that the surface has positive Gauss curvature, otherwise the statement is evident.
Therefore the surface bounds a convex region that contains a line segment of length~$\pi$.

\begin{figure}[h!]
\vskip-0mm
\centering
\includegraphics{asy/sin}
\vskip-0mm
\end{figure}

Observe that the Gauss curvature of the surface of revolution of the graph $y=a\cdot \sin x$ for $x\in(0,\pi)$ cannot exceed $1$ (Use \ref{ex:curvature-graph} and \ref{cor:meusnier}).
Try to support the surface $\Sigma$ from inside by a surface of revolution of the described type. 

\parit{Remark.}
In fact if Gauss curvature of $\Sigma$ is at least $1$,
then
the intrinsic diameter of $\Sigma$ can not exceed $\pi$.
The latter means that any two points in $\Sigma$ can be connected by a path that lies in $\Sigma$ and has length at most~$\pi$.

\parbf{\ref{ex:convex-proper-sphere}.}

\parbf{\ref{ex:convex-proper-plane}.}

\parbf{\ref{ex:open+convex=plane}.}

\parbf{\ref{ex:circular-cone}.}

\parbf{\ref{ex:intK}.}


\parbf{\ref{ex:convex-revolution}.} Use \ref{ex:principle-revolution}.

\parbf{\ref{ex:ruled=>saddle}.} Prove and use that each point $p\in\Sigma$ has a direction with vanishing normal curvature.


\parbf{\ref{ex:saddle-to-infty}.}

\parbf{\ref{ex:panov}.}

\parbf{\ref{ex:length-of-bry}.} Use the \ref{lem:convex-saddle} and the hemisphere lemma (\ref{lem:hemisphere}).

\parbf{\ref{ex:circular-cone-saddle}.}

\parbf{\ref{ex:disc-hat}.} Observe that it is sufficient to construct a smooth parametrization of $\Delta_\eps$ by a closed hemisphere.
To do this repeat the argument in \ref{lem:gauss=sphere} with the center at a point surrounded by the boundary line of $\Delta_\eps$ in its plane.

\parbf{\ref{ex:saddle-linear}.}

\parbf{\ref{ex:between-parallels}.} Look for an example among the surfaces of revolution and use \ref{ex:principle-revolution}.

\parbf{\ref{ex:one-side-bernshtein}.} Look at the sections of the graph by planes parallel to the $(x,y)$-plane and to the $(x,z)$-plane, then apply Meusnier’s theorem apply Meusnier's theorem (\ref{cor:meusnier}).

\parbf{\ref{ex:saddle-graph}.}

\parbf{\ref{ex:hat-convex}.}

\parbf{\ref{ex:intrinsic-diameter}.} Use \ref{lem:closest-point-projection}.

\parbf{\ref{ex:reflection-geodesic}.}

\parbf{\ref{ex:helix=geodesic}.}

\parbf{\ref{ex:asymptotic-geodesic}.}

\parbf{\ref{ex:geodesic-curvature}.}

\parbf{\ref{ex:two-min-geo}.}

\parbf{\ref{ex:min-geod+plane}.}

\parbf{\ref{ex:two-min-geod}.}

\parbf{\ref{ex:min-geod+plane}.}

\parbf{\ref{ex:milka}.} Show that the concatenation of the line segment $[p_t,\gamma(t)]$ and the arc $\gamma|_{[t,\ell]}$ is a shortest path in the closed region $W$ outside of $\Sigma$.

\parbf{\ref{ex:rho''}.}

\parbf{\ref{ex:usov-exact}.}

\parbf{\ref{ex:rough-bound-mountain}.} Use \ref{thm:usov} and \ref{ex:sef-intersection}.

The suggested argument does not give the optimal bound for the Lipschitz constant that guarantees that $\gamma$ is simple, but
later (see \ref{ex:sqrt(3)}) we will show that the exact bound is $\sqrt{3}=\tg\tfrac\pi3$ --- it is the same as in the exercise about mountain of with the shape of a perfect cone; see \ref{ex:lasso}.

\parbf{\ref{ex:parallel}.}

\parbf{\ref{ex:parallel-transport-support}.}
Observe $\Sigma_1$ supports $\Sigma_2$ at any point of $\gamma$.
Conclude that $\gamma$ identical spherical images in $\Sigma_1$ and $\Sigma_2$ and apply Observation \ref{obs:parallel=}.

\parbf{\ref{ex:holonomy=not0}.}

\parbf{\ref{ex:half-sphere-total-curvature}.}

\parbf{\ref{ex:1=geodesic-curvature}.}

\parbf{\ref{ex:geodesic-half}.}

\parbf{\ref{ex:aabbccdd}.} Estimate integral of Gauss curvature bounded by a simple geodesic loop and apply \ref{ex:int-gauss=4pi}.


\parbf{\ref{ex:sqrt(3)}.} Note that it is sufficient to show that the surface has no geodesic loops.
Estimate the integral of Gauss curvature of whole surface and a disc in it surrounded by a geodesic loop.

\parbf{\ref{ex:self-intersections}.}

\parbf{\ref{ex:lasso}.} Cut the lateral surface of the mountain by a line from the cowboy to the top, unfold it on the plane and try to figure out what is the image of the strained lasso.

Since the distance between points, can not be bigger than length of a path connecting them,
this statement implies the problem.


\parbf{\ref{ex:deformation}.}

\parbf{\ref{ex:ell-infty}.}

\parbf{\ref{ex:B2inB1}.}

\parbf{\ref{ex:shrt=>continuous}.}

\parbf{\ref{ex:close-open}.}

%\parbf{\ref{}.}

%\parbf{\ref{}.}

%\parbf{\ref{}.}

%\parbf{\ref{}.}

%\parbf{\ref{}.}
