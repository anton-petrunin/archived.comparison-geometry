\parbf{\ref{ex:ell-infty}.} Check all the conditions in the definition of metric, page \pageref{page:def:metric}.

\parbf{\ref{ex:B2inB1}}; \ref{SHORT.ex:B2inB1:a}.
Observe that $|p-q|_{\spc{X}}\le 1$. 
Apply the triangle inequality to show that $|p-x|_{\spc{X}}\le 2$ for any $x\in B[q,1]$.
Make a conclusion.

\parit{\ref{SHORT.ex:B2inB1:b}.} Take $\spc{X}$ to be a half-line $[0,\infty)$ with the standard metric; $p=0$ and $q=\tfrac45$.

\parbf{\ref{ex:shrt=>continuous}.} Show that the conditions in \ref{def:continous} hold for $\delta=\eps$.

\parbf{\ref{ex:close-open}.}
Suppose the complement $\Omega=\mathcal{X}\backslash Q$ is open.
Then for each point $p\in \Omega$ there is $\eps>0$ such that $|p-q|_{\spc{X}}>\eps$ for any $q\in Q$.
It follows that $p$ is not a limit point of any sequence $q_n\in Q$.
That is, any limit of points in $Q$ lies in $Q$ which by definition means that $Q$ is closed.

Now suppose $\Omega=\mathcal{X}\backslash Q$ is not open.
Then there is a point $p\in \Omega$ such that for any natural $n$ there is a point $q_n\in Q$ such that $|p-q_n|_{\spc{X}}\z<\tfrac 1n$; in particular $q_n\to p$ and $n\to \infty$.
Since $p\notin Q$, we get that $Q$ is not closed.

\parbf{\ref{ex:9}.} The image of $\gamma$ might have a shape of digit $8$ or $9$.

\parbf{\ref{aex:simple-curve}.}
Let $\alpha$ be a path connecting $p$ to $q$.

Passing to a subinterval if necessary,
we can assume that $\alpha(t)\ne p,q$ for $t\ne0,1$.

An open set $\Omega$ in $(0,1)$ will be called {}\emph{suitable}
if for any connected component $(a,b)$ of $\Omega$ we have $\alpha(a)=\alpha(b)$.
Show that the union of nested suitable sets is suitable.
Therefore we can find a maximal suitable set $\hat \Omega$.

Define $\beta(t)=\alpha(a)$ for any $t$ in a connected component $(a,b)\subset\Omega$.
Note that for any $x\in [0,1]$ the set $\beta^{-1}\{\beta(x)\}$ is connected.

It remains to re-parametrize $\beta$ to make it injective.
In other words we need to construct a non-decreasing surjective function $\tau\:[0,1]\z\to[0,1]$ such that 
$\tau(t_1)=\tau(t_2)$ if and only if there is a connected component $(a,b)$ such that $t_1,t_2\z\in [a,b]$.
The construction is similar to the construction of devil's staircase.


\parbf{\ref{ex:L-shape}.}
Denote the union of two half-axis by $L$.
Observe that $f(t)\to\infty$ as $t\to \infty$.
Since $f(0)=0$, the intermediate value theorem implies that $f(t)$ takes all nonnegative values for $t\ge 0$.
Use it to show that $L$ is the range of $\alpha$.

Note that the function $f$
is smooth.
Indeed, the existence of all derivatives $f^{(n)}(x)$ at $x\ne 0$ is evident and direct calculations show that $f^{(n)}(0)=0$ for all $n$.
Therefore $t\mapsto \alpha(t)=(f(t),f(-t))$ is smooth as well.

Further, show that the function $f$ is strictly increasing for $t> 0$,
and, moreover, if $0<t_0<t_1$, then $0<f(t_0)<f(t_1)$.
Use it to show that the maps $t\mapsto \alpha(t)$ is injective.

Summarizing we get that $\alpha$ is a smooth parametrization of $L$. 

Now suppose $\beta\:t\mapsto (x(t),y(t))$ is a smooth parameterization of~$L$.
Without loss of generality we may assume that $x(0)=y(0)=0$.
Note that $x(t)\ge 0$ for any $t$ therefore $x'(0)=0$.
The same way we get that $y'(0)=0$.
That is, $\beta'(0)=0$;
so $L$ does not admit a smooth regular parameterization.

\parbf{\ref{ex:cycloid}.}
Apply the definitions.
For \ref{SHORT.ex:cycloid:regular} you need to check that $\gamma'_\ell\ne 0$.
For \ref{SHORT.ex:cycloid:simple} you need to check that $\gamma_\ell(t_0)=\gamma(t_1)$ only if $t_0=t_1$.

\begin{wrapfigure}{r}{22 mm}
\vskip-0mm
\centering
\includegraphics{mppics/pic-270}
\vskip0mm
\end{wrapfigure}

\parbf{\ref{ex:y^2=x^3}.}
This is so called \emph{semicubical parabola}; it is shown on the diagram.
Try to argue similarly to \ref{ex:L-shape}.


\parbf{\ref{ex:viviani}.}
For $\ell=0$ the system describes a pair of points $(0,0,\pm1)$, so we can assume that $\ell\ne 0$.
Note that first equation desribes the unit sphere centered at the origin
and the second equation describes a cylinder over the circle in the $(x,y)$-plane with diameter with opposite points $(0,0)$ and $(0,\ell)$.


For $\ell\ne 0$,
find the gradients $\nabla f$ and $\nabla h$ for the functions
\begin{align*}
 f(x,y,z)&=x^2+y^2+z^2-1
 \\
 h(x,y,z)&=x^2+\ell\cdot x+y^2
\end{align*}
and show that they are linearly dependent only on the $x$-axis.
Conclude that for $\ell\ne\pm 1$ each connected component of the set of solutions is a smooth regular curve.

Show that 
\begin{itemize}
\item if $|\ell|<1$, then the set has two connected components with $z>0$ and $z<0$.
\item if $|\ell|\ge1$, then the set is connected.
\end{itemize}

\begin{wrapfigure}{o}{35 mm}
\centering
\begin{lpic}[t(-0mm),b(6mm),r(0mm),l(0mm)]{asy/viviani(1)}
\lbl[r]{-.5,13;$x$}
\lbl[l]{30.5,15;$y$}
\lbl[b]{14,40;$z$}
\end{lpic}
\end{wrapfigure}

Note that the condition on gradients provides only sufficient condition.
Therefore he case $\ell=\pm1$ has to be checked by hands.
In this case a neighborhood of $(\pm1,0,0)$ does not admit a smooth regular parametrization --- try to prove it. 
The case $\ell=0$ shown on the diagram.

\parit{Remark.}
In the case $\ell=\pm1$ it is called \emph{Viviani's curve}.
It admits the following smooth regular parameterization with a self-intersection:
\[t\mapsto(\pm(\cos t)^2,\cos t\cdot\sin t,\sin t).\]


\parbf{\ref{ex:proper-curve}.}
Without loss of generality we may assume that the origin does not lie on the curve.

Show that inversion of the plane $(x,y)\mapsto (\tfrac{x}{x^2+y^2},\tfrac{y}{x^2+y^2})$ maps our curve maps to a closed curve with removed origin.
Apply Jordan's theorem for the obtained curve and use the inversion again.

\parbf{\ref{ex:integral-length-0}.} Observe that if $c=\tau_0<\dots<\tau_n=d$ is a partition of $[c,d]$ if and only if $t_i=\phi(\tau_i)$ is a partition of $[a,b]$ and apply the definition of length (\ref{def:length}).

\parbf{\ref{ex:length-image}.}
Fix a partition $0=t_0<\dots <t_n=1$ of $[0,1]$.
Set $\tau_0=0$ and $\tau_i=\max\set{\tau \in[0,1]}{\beta(\tau_i)=\alpha(t_i)}$.
Show that $(\tau_i)$ is a partition of $[0,1]$;
that is, $0=\tau_0<\tau_1<\dots<\tau_n=1$.

By construction 
\begin{align*}
|\alpha(t_0)&-\alpha(t_1)|+|\alpha(t_1)-\alpha(t_2)|+\dots+|\alpha(t_{n-1})-\alpha(t_n)|=
\\
&=
|\beta(\tau_0)-\beta(\tau_1)|+|\beta(\tau_1)-\beta(\tau_2)|+\dots+|\beta(\tau_{n-1})-\beta(\tau_n)|.
\end{align*}
Since the partition $(t_i)$ is arbitrary, we get 
\[\length \beta\ge \length \alpha.\]

\parit{Remark.}
Note that the partition $(\tau_i)$ is not arbitrary, therefore the inequality might be strict; it might happen if $\beta$ runs back and forth along $\alpha$.




\parbf{\ref{ex:integral-length}.} For (\ref{ex:integral-length>}), apply the fundamental theorem of calculus for each segment in a given partition. For (\ref{ex:integral-length<}) consider a partition such that the velocity vector $\alpha'(t)$ is nearly constant on each of its segments.

\parbf{\ref{adex:integral-length}.} Use theorems of Rademacher and Lusin (\ref{thm:rademacher} and \ref{thm:lusin}).

\parbf{\ref{ex:nonrectifiable-curve}}; \ref{SHORT.ex:nonrectifiable-curve:a}.
\begin{figure}[h!]
\vskip-0mm
\centering
\includegraphics{mppics/pic-226}
\vskip0mm
\end{figure}
Look at the diagram and guess the parameterization of an arc of the snow flake by $[0,1]$.
Extend it to whole snow flake.
Show that it is indeed describes an embedding of the circle in the plane.

\parit{\ref{SHORT.ex:nonrectifiable-curve:b}.}
Suppose that $\gamma\:[0,1]\to\RR^2$ is a rectifiable curve and $\gamma_k$ be a scaled copy of $\gamma$ with factor $k>0$;
that is $\gamma_k(t)=k\cdot\gamma(t)$ for any $t$.
Show that 
\[\length\gamma_k=k\cdot\length \gamma.\]

Now suppose the arc $\gamma$ of the Koch snowflake shown on the diagram is rectifiable; denote its length by $\ell$.
Evidently $\ell>0$.
Observe that it $\gamma$ can be divided into 4 arcs each of which is a scaled copy of $\gamma$ with factor $\tfrac13$.
It follows that $\ell=\tfrac43\cdot\ell$ --- a contradiction.


\parbf{\ref{ex:arc-length-helix}.} 
We have to assume that $a\ne 0$ or $b\ne0$;
otherwise we get a constant curve.

Show that the curve has constant velocity $|\gamma'(t)|\equiv \sqrt{a^2+b^2}$.
Therefore 
\[s=\tfrac{t}{\sqrt{a^2+b^2}}\] is an arc-length parameter.


\parbf{\ref{ex:convex-hull}.}
Choose a closed polygonal line $p_1\dots p_n$ inscribed in $\beta$.
By \ref{cor:convex=>rectifiable}, we can assume that it length is arbitrary close to the length of $\beta$;
that is, given $\eps>0$ 
\[\length (p_1\dots p_n)>\length\beta-\eps.\]

Show that we may assume in addition that each point $p_i$ lies on $\alpha$.

Observe that since $\alpha$ is simple, the points $p_1,\dots,p_n$ appear on $\alpha$ in the same cyclic order;
that is, the polygonal line $p_1\dots p_n$ is also inscribed in $\alpha$.
In particular 
\[\length\alpha\ge \length (p_1\dots p_n).\]
It follows that 
\[\length\alpha>\length\beta-\eps.\]
for any $\eps>0$.
Whence 
\[\length\alpha\ge\length\beta.\]

\begin{wrapfigure}{r}{25 mm}
\vskip-0mm
\centering
\includegraphics{mppics/pic-275}
\vskip0mm
\end{wrapfigure}

If $\alpha$ has self-intersections, then the points $p_1,\dots, p_n$ might appear on $\alpha$ in a different order, say $p_{i_1},\dots,p_{i_n}$.
Apply triangle inequality to show that 
\[\length(p_{i_1}\dots p_{i_n})\ge \length (p_1\dots p_n)\]
and use it to modify the proof above.

\parbf{\ref{ex:convex-croftons}.} 
Denote by $\ell_u$ the line segment 
obtained orthogonal projection of $\gamma$ to the line in the direction $u$.
Note that $\gamma_u$ runs back and forth along $\ell_u$, we get 
\[\length\gamma_u\ge 2\cdot\length\ell_u.\]
Applying the Crofton formula, we get that 
\[\length\gamma\ge \pi\cdot \overline{\length\ell_u}.\]

In the case of equality, the curve $\gamma_u$ runs exactly back and forth along $\ell_u$ without additional zigzags for almost all (and therefore for all) $u$.

Let $K$ be a closed set bounded by $\gamma$.
Observe that the last statement implies that every line may intersect $K$ only along a closed segment.
In other words $K$ is convex.

\parbf{\ref{adex:more-croftons}.}
The proof is identical to the proof of the standard Crofton formula.
To find the coefficient one has to find average length of projection of unit vector to a line.
Which can be done by integration.
\begin{align*}
\frac1{k_a}&=\frac{1}{\area \SS^2}\cdot\int_{\SS^2} |x|;
&
\frac1{k_b}&=\frac{1}{\area \SS^2}\cdot\int_{\SS^2} \sqrt{1-x^2}.
\end{align*}
The answers are $k_a=2$ and $k_b=\tfrac4\pi$.

\parbf{\ref{ex:intrinsic-convex}.}
The ``only-if'' part is trivial.
To show the ``if'' part, assume $A$ is not convex;
that is, there are points $x,y\in A$ and a point $z\notin A$ that lies between $x$ and $y$.

Since $A$ is closed, its complement is open.
That is, the ball $B(z,\eps)$ does not intersect $A$ for some $\eps>0$.

Show that there is $\delta>0$ such that any path of length at most $|z-y|_{\RR^3}+\delta$ pass thru $B(z,\eps)$.
It follows that $|z-y|_A\ge |z-y|_{\RR^3}+\delta$, 
in particular$|z-y|_A\ne |z-y|_{\RR^3}$.

\begin{wrapfigure}{r}{30 mm}
\vskip-0mm
\centering
\includegraphics{mppics/pic-280}
\vskip0mm
\end{wrapfigure}

\parbf{\ref{ex:antipodal}.}
The spherical curve shown on the diagram does not have antipodal pairs of points.
However it has three points $x,y,z$ on one of its sides and their antipodal points $-x,-y,-z$ on the other.
Show that this property is sufficient to conclude that the curve does not lie in any hemisphere.

\parbf{\ref{ex:bisection-of-S2}.}
Assume contrary, then by the hemisphere lemma (\ref{lem:hemisphere}) $\gamma$ lies in an open hemisphere.
In particular it cannot divide $\SS^2$ into two regions of equal area --- a contradiction.

\parbf{\ref{ex:flaw}.}
The very first sentence is wrong --- it is \emph{not} sufficient to show that diameter iis at most 2.
For example an equilateral triangle with circumradius slightly above $1$ may have diameter (which is defined as the maximal distance between its points) slightly bigger than $\sqrt3$, so it can be made smaller that $2$.

On the other hand, it is easy to modify the proof of the hemisphere lemma (\ref{lem:hemisphere}) to get a correct solution.
That is, (1) choose two points $p$ and $q$ on $\gamma$ that divide it into two arcs of the same length;
(2) set $z$ to be a midpoint of $p$ and $q$,
and (3) show that $\gamma$ lies in the unit disc centered at $z$.


\parbf{\ref{adex:crofton}.}
For \ref{SHORT.adex:crofton:crofton}, modify the proof of the original Crofton formula [see page \pageref{page:crofton}].

\parit{\ref{SHORT.adex:crofton:hemisphere}.}
Assume $\length \gamma<2\cdot\pi$.
By \ref{SHORT.adex:crofton:crofton},
\[\overline{\length \gamma_u}<2\cdot\pi.\]
Therefore we can choose $u$ so that 
\[\length \gamma_u<2\cdot\pi.\]

Observe that $\gamma_u$ runs in a semicircle $h$ and therefore $\gamma$ lies in a hemisphere with $h$ as a diameter.


\parbf{\ref{ex:curvature-of-spherical-curve}.} Differentiate the identity $\langle\gamma(s),\gamma(s)\rangle=1$ a couple of times.

\parbf{\ref{ex:curvature-formulas}.} Prove and use the following identities: 
\begin{align*}
\gamma''(t)-\gamma''(t)^\perp&=\tfrac{\gamma'(t)}{|\gamma'(t)|}\cdot\langle\gamma''(t),\tfrac{\gamma'(t)}{|\gamma'(t)|}\rangle,
\\
|\gamma'(t)|&=\sqrt{\langle \gamma'(t),\gamma'(t)\rangle}.\
\end{align*}

\parbf{\ref{ex:curvature-graph}.} 
Apply \ref{ex:curvature-formulas:a} for the parameterization $t\mapsto (t,f(t))$.

\parbf{\ref{ex:helix-curvature}.}
Show that $\gamma_{a,b}''\perp \gamma'_{a,b}$ and apply \ref{ex:curvature-formulas:a}.

\parbf{\ref{ex:length>=2pi}.} Apply Fenchel's theorem.

\parbf{\ref{ex:gamma/|gamma|}.} Assume that $\gamma$ is unit-speed; show that $|\sigma'|\le \kur+\theta'$, where $\theta(s)=\measuredangle(\gamma(s),\gamma'(s))$.

\parbf{\ref{ex:chord-lemma-optimal}.} 
Set $\alpha=\measuredangle(\vec w,\vec u)$ and $\beta=\measuredangle(\vec w,\vec v)$.
Try to guess the example from the diagram.
\begin{figure}[h!]
\vskip-0mm
\centering
\includegraphics{mppics/pic-285}
\vskip0mm
\end{figure}

The shown curve is divided into three arcs: I, II, and III. 
Arc I turns from $\vec v$ to $\vec u$;
it has total curvature $\alpha$.
Analogously the arc III turns from $\vec u$ to $\vec w$  and has total curvature $\beta$. 
Arc II goes very close and almost parallel to the chord $pq$ and its total curvature can be made arbitrary small.


\parbf{\ref{ex:monotonic-tc}.}
Use that exterior angle of a triangle equals to the sum of the two remote interior angles;
for the second part apply the induction on number of vertexes.

\begin{wrapfigure}{r}{23 mm}
\vskip-0mm
\centering
\includegraphics{mppics/pic-290}
\vskip-4mm
\end{wrapfigure}

\parbf{\ref{ex:sef-intersection}.} An example for \ref{SHORT.ex:sef-intersection:<2pi} is shown on the diagram. 

\parit{\ref{SHORT.ex:sef-intersection:>pi}.} Assume $x$ is a point of self-intersection.
Show that we may choose two points $y$ and $z$ on $\gamma$ so that the triangle $xyz$ iz nondegenerate.
In particular, $\measuredangle xyz+\measuredangle yzx<\pi$, or, equivalently, 
\[\tc{xyzx}>\pi;\]
here we assume that $xyzx$ is polygonal line which is not closed. 
It remains to apply \ref{prop:inscribed-total-curvature}.

\parbf{\ref{ex:quadrisecant}.}
Observe that 
\[\tc{acbd}=4\cdot\pi,\]
here we assume that $acbd$ denotes the closed polygonal line.
It remains to apply \ref{prop:inscribed-total-curvature}.

{

\begin{wrapfigure}{r}{22 mm}
\vskip-6mm
\centering
\includegraphics{mppics/pic-255}
\vskip0mm
\end{wrapfigure}

\parbf{\ref{ex:anti-bow}.}
Start with with the curve $\gamma_1$ shown on the diagram.
To obtain $\gamma_2$, slightly unbend (that is, decrease the curvature of) the dashed arc of $\gamma_1$.

}

\parbf{\ref{ex:length-dist}.}
Choose a value $s_0\in[a,b]$ that splits the total curvature into two equal parts, $\theta$ in each.
Observe that $\measuredangle(\gamma'(s_0),\gamma'(s))\le \theta$ for any~$s$.
Use this inequality the same way as in the proof of the bow lemma.

\parbf{\ref{ex:schwartz}.}
Let $\ell=\length\gamma$.
Suppose $\ell_1<\ell<\ell_2$.
Let $\gamma_1$ be an arc of unit circle with length $\ell$.

Show that the distance between the ends of $\gamma_1$ is smallet than $|p-q|$ and apply the bow lemma (\ref{lem:bow}).


\parbf{\ref{ex:loop}.} If $\length\gamma<2\cdot\pi$, apply the bow lemma (\ref{lem:bow}) to $\gamma$ and an arc of unit circle of the same length.

\parbf{\ref{ex:tc-semicontinuous}.} Modify the proof of semi-continuity of length (\ref{thm:length-semicont}).

\parbf{\ref{ex:gen-fenchel}.}
Choose two distinct points $p$ and $q$ on $\gamma$.
Consider a \emph{diangle} $pq$ (that is, a closed polygonal line with two vertexes $p$ and $q$).
Observe that both external angles of the diangle have measure $\pi$.
Therefore the total curvature of diangle is $2\cdot \pi$.

It remains to apply the definition of total curvature for arbitrary curves (\ref{def:total-curv-poly}).

\parbf{\ref{ex:tc-length};} \ref{SHORT.ex:tc-length:rectifiable}.
Observe that if $\tc\gamma\le 1$, then
\[\length\gamma\le 2\cdot|\gamma(b)-\gamma(a)|.
\eqlbl{eq:arc<2chord}\]
Indeed, if $\tc\gamma\le 1$, then from the definition of total curvature [\ref{def:total-curv-poly}] it follows that for any polygonal line $p0\dots p_n$ inscribed in $\gamma$ the angle between any pair of edges does noe exceed $1$.
Let $q_i$ be the orthogonal projection of $p_i$ to the line of its first edge $p_0p_1$.
By the angle estimate, we have that the points $q_0,\dots,q_n$ appear on the line in the same order and 
\[|q_i-q_{i-1}|> \tfrac12\cdot|p_i-p_{i-1}|.\]
Therefore 
\begin{align*}|\gamma(b)-\gamma(a)|&=|p_n-p_0|\ge 
\\
&\ge |q_n-q_0|=
\\
&=|q_n-q_{n-1}|+\dots+|q_1-q_0|>
\\
&>\tfrac12\cdot(|p_n-p_{n-1}|+\dots+|p_1-p_0|).
\end{align*}
Recall that length of $\gamma$ is defined as the exact upper bound for the  sum $|p_n-p_{n-1}|+\dots+|p_1-p_0|$.
Therefore \ref{eq:arc<2chord} follows.

In general, if $\tc\gamma$ is bounded, we can subdivide $\gamma$ into arcs with total curvature at most $1$,
apply the above argument to each of arc and sum up the results.

\begin{wrapfigure}{r}{15 mm}
\vskip-6mm
\centering
\includegraphics{mppics/pic-295}
\vskip0mm
\end{wrapfigure}

\parit{\ref{SHORT.ex:tc-length:example}.}
A logarithmic spiral is an example;
it can be defined in polar coordinates by $\gamma(t)=(t,a\cdot\ln t)$ if $t>0$ and $\gamma(0)$ is the origin; we may assume that $t\in[0,1]$.

It remains to show that indeed, $\gamma$ has infinite total curvature, but finite length.

\begin{wrapfigure}{r}{20 mm}
\vskip-2mm
\centering
\begin{lpic}[t(-0mm),b(0mm),r(0mm),l(0mm)]{asy/helix(1)}
\lbl[br]{8,24;$\norm$}
\lbl[b]{2,26;$\bi$}
\lbl[wl]{15,25;$\tan$}
\end{lpic}
\end{wrapfigure}

\parbf{\ref{ex:helix-torsion}.} 
The arc-length parameter $s$ is already found in   \ref{ex:arc-length-helix}.
It remains to find Frenet frame and calculate curvature and torsion.
The latter can be done by straightforward calculations.

If the calculations done right, then you should see that curvature $\kur$ and torsion $\tau$ do not depend on time and given.
Moreover for any $\kur>0$ and $\tor$ one can find $a$ and $b$ so that the helix $\gamma_{a,b}$ has curvature $\kur$ and torsion~$\tor$.

One may also see it geometrically using that the helix is maped to itself by one parameter family of glide rotations around $z$-axis.
Therefore, for the $t$-parametrization, Frenet frame rotates around $z$-axis with the angular velocity~$1$.
It remains rewrite it for the arc-length parametrization and note that 
\begin{align*}
\tan(0)&=(0,\cos\theta, \sin\theta),
\\
\norm(0)&=(-1,0,0),
\\
\bi(0)&=(0,\sin\theta, -\cos\theta),
\end{align*}
where $\tg\theta\z=b/a$ if $a>0$. 

\parbf{\ref{ex:beta-from-tau+nu}.} By product rule, we get
\begin{align*}
\bi'&=(\tan\times \norm)'=
\\
&=\tan'\times \norm+\tan\times\norm'.
\end{align*}
It remains to substitute the values from \ref{eq:frenet-tau} and \ref{eq:frenet-nu} and simplify.


\parbf{\ref{ex:torsion=0}.}
Show and use that the binormal vector is constant.

\parbf{\ref{ex:frenet}.} Observe that $\tfrac{\gamma'\times\gamma''}{|\gamma'\times\gamma''|}$ is a unit vector perpendicular to the plane spanned by $\gamma'$ and $\gamma''$, so, up to sign, it has to be equal to $\bi$. It remains to check that the sign is right.

%???

\parbf{\ref{ex:lancret}}, \ref{SHORT.ex:lancret:a}.
Observe that 
$\langle \vec w,\tan\rangle'=0$.
Show that it implies that
\[\langle \vec w, \norm\rangle =0.\]

Further observe that 
$\langle \vec w,\norm\rangle'=0$.
Show that it implies that 
\[\langle \vec w, -\kur\cdot\tan+\tor\cdot \bi\rangle =0.\]

\parit{\ref{SHORT.ex:lancret:a}.}
Show that $\vec w'=0$;
it implies that $\langle \vec w,\tan\rangle=\tfrac\tor\kur$.
In particular, the velocity vector of $\gamma$ makes a constant angle with $\vec w$; that is, $\gamma$ has constant slope.

\parbf{\ref{ex:evolvent-constant-slope}.}
Show that $\langle w,\alpha\rangle$ is constant if $\gamma$ makes constant angle with a fixed vector $w$ and $\alpha$ is the evolvent of $\gamma$.

\parbf{\ref{ex:spherical-frenet}.}
Suppose $\langle \vec w,\tan\rangle$ is a constant.
Show that $\langle \vec w,\alpha\rangle'=0$.
It follows that $\langle \vec w,\alpha\rangle$ is a constant,
so $\alpha$ lies in a plane perpendicular to $\vec w$.


\parbf{\ref{ex:cur+tor=helix}.} Use the second statement in \ref{ex:helix-torsion}.

\parbf{\ref{ex:const-dist}.} Note that the function
\[\rho(\ell)=|\gamma(t+\ell)-\gamma(t)|^2=\langle \gamma(t+\ell)-\gamma(t),\gamma(t+\ell)-\gamma(t)\rangle\] 
is smooth and does not depend on $t$.
Express speed, curvature and torsion of $\gamma$ in terms of derivatives $\rho^{(n)}(0)$
and apply \ref{ex:cur+tor=helix}.

\parbf{\ref{ex:bike}.}
Without loss of generality, we may assume that $\gamma_0$ is parameterized by its arc-length.
Then
\begin{align*}
|\gamma_1'|&=|\gamma_0'+\tan'|=|\tan+\kur\cdot\norm|=
\sqrt{1+\kur^2}\ge
1=
|\gamma_0'|;
\end{align*}
that is, $|\gamma_1'(t)|\ge|\gamma_0'(t)|$ for any $t\in[a,b]$
The statement follows since 
\[\length\gamma_i=\int_a^b|\gamma_i'(t)|\cdot dt.\]





\parbf{\ref{ex:trochoids}.}
Observe that
\[\gamma'_a(t)=(1+a\cdot \cos t, -a\cdot \sin t);\]
that is $\gamma'_a$ runs clockwise along a circle with center at $(1,0)$ and radius $1$.
If $|a|>1$ then $\tan_a(t)=\gamma'_a/|\gamma'_a|$ runs clockwise and makes full turn in time $2\cdot\pi$.
It follows that if $|a|>1$, then
\[\Psi(\gamma_a)=-2\cdot\pi,\quad \tc\gamma_a=|\Psi(\gamma_a)|=2\cdot\pi.\]

If $|a|<1$, set $\theta_a=\arcsin a$.
Note that $\tan_a(t)=\gamma'_a/|\gamma'_a|$ starts with the horizontal direction $\tan_a(0)=(1,0)$, turns monotonically to angle $\theta_a$, then monotonically to $-\theta_a$ and then monotonically to back to $\tan_a(2\cdot\pi)=(1,0)$.
It follows that if $|a|>1$, then
\[\Psi(\gamma_a)=0,\quad \tc{\gamma_a}=4\cdot\theta_a.\]

In the cases $a=-1$ the velocity $\gamma'_{-1}(t)$ vanish at $t=0$ and $2\cdot\pi$.
Nevertheless, the curve admits a smooth regular parametrization --- show it.
In this case $\Psi(\gamma_{-1})=-\pi$ and $\tc{\gamma_{-1}}=\pi$.

In the cases $a=1$ the velocity $\gamma'_1(t)$ vanish at $t=\pi$.
At $t=\pi$ the curve has a cusp;
that is 
\[\lim_{t\to\pi-}\tan_1(t)=-\lim_{t\to\pi+}\tan_1(t).\]
So $\gamma_1(t)$ has undefined total signed curvature.
The curve is a joint of two smooth arcs with external angle $\pi$,
the total curvature of each arc is $\tfrac\pi2$, so 
$\tc{\gamma_{1}}=\tfrac\pi2+\pi+\tfrac\pi2=2\cdot\pi$.


\parbf{\ref{ex:zero-tsc}.}
The two marked points in the last example (for part \ref{SHORT.ex:zero-tsc:2-4}) have parallel tangent lines.

\begin{figure}[h!]
\begin{minipage}{.32\textwidth}
\centering
\includegraphics{mppics/pic-260}
\end{minipage}\hfill
\begin{minipage}{.32\textwidth}
\centering
\includegraphics{mppics/pic-261}
\end{minipage}
\hfill
\begin{minipage}{.32\textwidth}
\centering
\includegraphics{mppics/pic-262}
\end{minipage}
\end{figure}

\parbf{\ref{ex:length'}}; \ref{SHORT.ex:length':reg}.
Show that
\[
\gamma_\ell'(t)=(1-\ell\cdot\skur(t))\cdot \gamma'(t).
\]
Since $\gamma$ is regular, $\gamma'\ne0$.
Therefore if $\gamma_\ell'(t)=0$, then $\ell\cdot \skur(t)= 1$.

\parit{\ref{SHORT.ex:length':formula}.} Observe that we can assume that $\gamma$ is parameterized by its arc-length, so $\gamma'(t)=\tan(t)$.
Suppose $|\ell|<\frac{1}{\kur(t)}$ for any $t$.
Then 
\[
|\gamma_\ell'(t)|=(1-\ell\cdot\skur(t)).
\]
\begin{align*}
L(\ell)&=\int_a^b(1-\ell\cdot\skur(t))\cdot dt=
\\
&=\int_a^b 1\cdot dt-\ell\cdot \int_a^b\skur(t))\cdot dt=
\\
&=L(0)+\ell\cdot\Psi(\gamma).
\end{align*}



\parit{\ref{SHORT.ex:length':antiformula}.}
Consider the unit circle $\gamma(t)=(\cos t,\sin t)$ for $t\in[0,2\cdot\pi]$ and $\gamma_\ell$ for $\ell=2$.

\parbf{\ref{ex:inverse}.}
Use the definition of osculating circle via order of contact and that inversion maps circles to circlines. 

\parbf{\ref{ex:evolute-of-ellipse}.}
Find $\tan(t)$ and $\norm(t)$.
Use the formula in \ref{ex:curvature-formulas:b} to calculate curvature $\kur(t)$.
Apply the formula given right before the exercise.

\begin{wrapfigure}{r}{15 mm}
\vskip-0mm
\centering
\includegraphics{mppics/pic-296}
\vskip0mm
\end{wrapfigure}

\parbf{\ref{ex:3D-spiral}.} Start with a plane spiral curve as shown on the diagram.
Increase the torsion of the dashed arc without changing curvature until it a self-intersection appears.

\parbf{\ref{ex:double-tangent}.} 
Observe that if a line or circle is tangent to $\gamma$,
then it is tangent to osculating circle at the same point
and apply the spiral lemma (\ref{lem:spiral}).

%\parit{\ref{SHORT.ex:double-tangent:a}.}
%Suppose that a line $\ell$ is tangent to $\gamma$ at two points $p$ and $q$.
%Let $\sigma_p$ and $\sigma_q$ be the osculating circles of $\gamma$ at $p$ and $q$ respectively.
%Show that $\ell$ is tangent to $\sigma_p$ and $\sigma_q$.

%By \ref{lem:spiral}, we can assume that $\sigma_p$ is surrounded by $\sigma_q$.
%In particular any line that tangent to $\sigma_q$ cannot intersect $\sigma_p$;
%in particular, $\ell$ cannot be tangent to both $\sigma_p$ and $\sigma_q$ --- a contradiction. 

%\parit{\ref{SHORT.ex:double-tangent:b}.}
%If a circle $\sigma$ is tangent to $\gamma$ at three points $x$, $y$, and $z$.
%Then $\sigma$ is tangent to the osculating circles $\sigma_x$, $\sigma_y$, and $\sigma_z$.
%By \ref{lem:spiral} we can assume that $\sigma_x$ is surrounded by $\sigma_y$ and $\sigma_y$ is surrounded by $\sigma_z$.
%Observe that the first statement implies that $\sigma$ is squeezed between $\sigma_x$ and $\sigma_y$.
%The second statement implies that $\sigma$ is squeezed between $\sigma_y$ and $\sigma_z$.
%These two statements contradict each other.

\parbf{\ref{ex:vertex-support}.}
Apply the spiral lemma (\ref{lem:spiral}).

\parbf{\ref{ex:support}.} Consider the coordinate system with $p$ as the origin and $x$-axis as the common tangent line to $\gamma_1$ and $\gamma_2$.
We may assume that $\gamma_i$ are defined in $(-\eps,\eps)$ for some small $\eps>0$,
so that they run almost horizontally to the right.

Given $t\in[0,1]$ consider the curve $\gamma_t$ that is tangent and cooriented to the $x$ axis at  $\gamma_t(0)$ and has signed curvature defined by $\skur_t(s)=(1-t)\cdot\skur_0(s)+t\cdot\skur_1(s)$.
It exists by \ref{thm:fund-curves-2D}.

Fix $s\approx 0$.
Consider the curve $\alpha_s\:t\mapsto \gamma_t(s)$.
Show that $\alpha_s$ moves almost vertically up, while $\gamma_t$ moves almost horizontally to the right.
Conclude that in a small neighborhood of $p$, the curve $\gamma_1$
lies above $\gamma_0$.
Whence the statement follows.

\parbf{\ref{ex:in-circle}.} Let reduce the radius of the circle until it touches $\gamma$.
Observe that the circle supports $\gamma$ and apply \ref{prop:supporting-circline}.


\parbf{\ref{ex:between-parallels-1}, \ref{ex:in-triangle} and \ref{ex:lens}.}
Observe that one of the arcs of curvature 1 in the families shown on the diagram suppots $\gamma$ and apply \ref{prop:supporting-circline}.
\begin{figure}[h!]
\begin{minipage}{.50\textwidth}
\centering
\includegraphics{mppics/pic-265}
\end{minipage}\hfill
\begin{minipage}{.26\textwidth}
\centering
\includegraphics{mppics/pic-266}
\end{minipage}
\hfill
\begin{minipage}{.20\textwidth}
\centering
\includegraphics{mppics/pic-267}
\end{minipage}
\end{figure}
To do the second part in \ref{ex:between-parallels-1}, use shown family and another family of arcs curved in the opposite direction.

\parit{Remark.}
Exercise \ref{ex:moon-rad} is a more general. 

\parbf{\ref{ex:convex small}.}
Note that we can assume that $\gamma$ bounds a convex figure $F$, otherwise by \ref{prop:convex} its curvature changes the sign and therefore it has zero curvature at some point.
Choose two points $x$ and $y$ surrounded by $\gamma$ such that $|x-y|>2$,
look at the maximal lens bounded by two arcs with common chord $xy$ that lies in $F$ and apply supporting test (\ref{prop:supporting-circline}).

\parbf{\ref{ex:line-curve-intersections}}; \ref{SHORT.ex:line-curve-intersections:a}.
Apply the lens lemma to show that if $p_2$ lies between $p_1$ and $p_3$, then the curvature of $\gamma$ switches its sign.

\parit{\ref{SHORT.ex:line-curve-intersections:b}.} Show that in this $p_4$ lies between $p_2$ and $p_3$;
further $p_5$ lies between $p_3$ and $p_4$;
and so on.

\parit{\ref{SHORT.ex:line-curve-intersections:c}.}
According to \ref{SHORT.ex:line-curve-intersections:a}, the point $p_3$ might lie between $p_1$ and $p_2$ and then further order is determined uniquely or $p_1$ lies between $p_2$ and $p_3$.
In the latter case we have two choices, either $p_4$ lies between $p_2$ and $p_3$ and then then further order is determined uniquely or $p_2$ lies between $p_3$ and $p_4$.
In the latter case we get a choice again.

Assume we make a first choice on the step number $k$.
Without loss of generality we may assume that $p_k$ lies to the right from $p_{k-2}$.
Then we have the following order:
\[
p_{k-2},p_{p-4},\dots,p_{p-5},p_{k-3},p_k,p_{k+2},\dots,p_{k+1},p_{k-1}.
\]
The case $k=7$ shown on the diagram.

\begin{figure}[h!]
\vskip-0mm
\centering
\includegraphics{mppics/pic-29}
\vskip0mm
\end{figure}

\parbf{\ref{ex:moon-rad}.} Note that $\gamma$ contains a simple loop; apply to it \ref{thm:moon-orginal}.

\begin{wrapfigure}{r}{25 mm}
\vskip-0mm
\centering
\includegraphics{mppics/pic-305}
\vskip0mm
\end{wrapfigure}

\parbf{\ref{ex:curve-crosses-circle}.}
Repeat the proof of theorem for each cyclic concatenation of an arc of $\gamma$ from $p_i$ to $p_{i+1}$ with large arc of the circle. 

An example for the second part can be guessed from the diagram.

\parbf{\ref{ex:hyperboloinds}.} 
Denote the set of solutions by $\Phi_\ell$.

Show that $\nabla_p f=0$ if and only if $p=(0,0,0)$.
Use \ref{prop:implicit-surface} to conclude that if $\ell\ne 0$, then $\Phi_\ell$ is a union of disjoint smooth regular surfaces.

Show that $\Phi_\ell$ is connected if and only if $\ell\le 0$.
It follows that if $\ell<0$, then $\Phi_\ell$ is a smooth regular surface and if $\ell>0$ then it is not.

The case $\ell=0$ has to be done by hands --- it does not satisfy the sufficient condition in \ref{prop:implicit-surface}, but it does not solely imply that $\Phi_0$ is not a smooth surface.

Show that any neighborhood of origin in $\Phi_0$ can not be described by a graph in any coordinate system;
so by the definition (page \pageref{page:def-smooth-surface}) $\Phi_0$ is not a smooth surface.

\parbf{\ref{ex:inversion-chart}.} 
First check that the image of $s$ lies in the the unit sphere centered at $(0,0,1)$;
that is show that 
\[\left(\tfrac{2\cdot u}{1+u^2+v^2}\right)^2
+
\left(\tfrac{2\cdot v}{1+u^2+v^2}\right)^2
+\left(\tfrac{2}{1+u^2+v^2}-1\right)^2=1.\]
for any $u$ and $v$.

Further, show that the map 
\[(x,y,z)\mapsto (\tfrac{2\cdot x}{x^2+y^2+z^2},\tfrac{2\cdot y}{x^2+y^2+z^2})\]
describe the inverse map which is continuous away from the origin.
In particular, $s$ is an embedding that covers whole sphere except the origin.

It remains to show that $s$ is regular; that is, $\tfrac{\partial s}{\partial u}$ and $\tfrac{\partial s}{\partial v}$ are linearly independent at all points of the $(u,v)$-plane.

\parbf{\ref{ex:revolution}.}
Set
\[s\:(t,\theta)\mapsto (x(t), y(t)\cdot\cos\theta,y(t)\cdot\sin\theta).\]
Show that $s$ is regular; that is, $\tfrac{\partial s}{\partial t}$ and $\tfrac{\partial s}{\partial \theta}$ are linearly independent.
(It might help to observe that $\tfrac{\partial s}{\partial t}\perp\tfrac{\partial s}{\partial \theta}$).

Show that $s$ is local embedding; that is, any $(t_0,\theta_0)$ admits a neighborhood $U$ in the $(t,\theta)$-plane such that the restriction $s|_U$ has a continuous inverse.
It remains to apply \ref{cor:reg-parmeterization}.

\parbf{\ref{ex:star-shaped-disc}}; 
The solutions of these exercises are build on the following general construction known as the \emph{Moser trick}.

Suppose that $\vec u_t$ is a smooth vector field on a plane.
Consider the ordinary differential equation $x'(t)=\vec u_t(x(t))$.
Consider the map $\iota\:x(0)\mapsto x(1)$ where $x(t)$ is a solution of the equation.
The map $\iota$ is called \emph{flow} of vector field $\vec u_t$ for the time interval $[0,1]$.
Observe that according to \ref{thm:ODE} the map $\iota$ is smooth in its domain of definition.
Moreover the same holds for its inverse;
indeed $\iota^{-1}$ is the flow for of the vector field $-\vec u_{1-t}$.
That is, $\iota$ is a diffeomorphism from the domain of definition to its image. 


Therefore, in order to construct a diffeomorphism from one open subset of the plane to another it sufficient to construct a smooth vector field such that flow maps one set to the other;
such map is automatically a diffeomorphism.


\parit{\ref{SHORT.ex:plane-n}.}
Suppose $\Sigma=\RR^2\backslash{p_1,\dots,p_n}$ and $\Theta=\RR^2\backslash{q_1,\dots,q_n}$.
Choose smooth paths $\gamma_i\:[0,1]\to \RR^2$ such that $\gamma_i(0)=p_i$,
$\gamma_i(1)=q_i$, and $\gamma_i(t)\ne \gamma_j(t)$ if $i\ne j$.

Choose a smooth vector field $\vec v_t$ such that $\vec v_t(\gamma_i(t))=\gamma'_i(t)$ for any $i$ and $t$.
We can assume in addition that $\vec v_t$ vanish outside of a sufficiently large disc; it can be arranged by a multiplying the vector field to a function 
$\sigma_1(R-|x|)$; see page \pageref{page:sigma-function}.

It remains to apply the Moser trick to the constructed vector field. 

\parit{\ref{SHORT.ex:star-shaped-disc:smooth}.}
Without loss of generality we can assume that the origin belongs to both $\Sigma$ and $\Theta$.
Observe that the boundary curves of $\Sigma$ and $\Theta$ can be written in polar coordinates as 
$(\theta, f(\theta)$ and $(\theta, g(\theta)$ for smooth functions $f,g\:\SS^1\to\RR$.



Denote by $\vec u_\theta$ the unit vector written as $(\theta,1)$ in the polar coordinates.
Consider the vector field $[g(\theta)-f(\theta)]\cdot \vec u_\theta$ it is defined 
Show that there is a vector field $\vec v$ defined on $\RR^2\backslash\{0\}$ that flows $\partial \Sigma$ to $\partial \Theta$;
in fact such vector field can be found among radial fields 


Observe that the boundary curves can be written in polar coordinates as 
$(\theta, \rho_i(\theta)$ for smooth functions $\rho_i\:\SS^1\to\RR$.

???

\parbf{\ref{ex:tangent-normal}.}
Let $\gamma$ be a smooth curve in $\Sigma$.
Observe that $f\circ\gamma(t)\equiv 0$.
Differentiate this identity and apply the definition of tangent vector (\ref{def:tangent-vector}).

\parbf{\ref{ex:vertical-tangent}.}
Assume a neighborhood of $p$ in $\Sigma$ is a graph $z=f(x,y)$.
In this case $s\:(u,v)\mapsto (u,v,f(u,v))$ is a smooth chart at $p$.
Show that the plane spanned by $\tfrac{\partial s}{\partial u}$ and $\tfrac{\partial s}{\partial v}$ is not vertical;
together with \ref{def:tangent-plane} it proves the if part.

To prove the only-if part, fix a chart 
\[s\:(u,v)\mapsto(x(u,v),y(u,v),z(u,v))\] at $p$ and apply the inverse function theorem for the map $(u,v)\mapsto(x(u,v),y(u,v))$.


\parbf{\ref{ex:implicit-orientable}.} Show that $\Norm=\tfrac{\nabla h}{|\nabla h|}$ defines a unit normal field on $\Sigma$.

\parbf{\ref{ex:plane-section}.} 
Fix a closed set $A$ in the $(x,y)$-plane.
Show that there is a smooth nonnegative function $(x,y)\mapsto f(x,y)$ such that $(x,y)\in A$ if and only if $f(x,y)=0$.
Observe that the graph $z=f(x,y)$ describes a required surface.

\parbf{\ref{ex:line-of-curvature}.}
Fix a point $p$ on $\gamma$.
Since $\Sigma$ is mirror symmetric with respect to $\Pi$,
so is the tangent plane $\T_p$.

Choose $(x,y)$-coordinates on $\T_p$ so that the $x$-axis is the intersection $\Pi\cap  \T_p$.
Suppose that the osculating paraboloid is described by the graph 
\[z=\tfrac12\cdot(\ell\cdot x^2+2\cdot m\cdot x\cdot y+n\cdot y^2)\]
Since $\Sigma$ is mirror symmetric, so is the paraboloid;
that is, changing $y$ to $(-y)$ does not change the value 
$\ell\cdot x^2+2\cdot m\cdot x\cdot y+n\cdot y^2$.
In other words $m=0$, or equivalently, the $x$-axis points in the direction of curvature.

\parbf{\ref{ex:gauss+orientable}.} Note that the principle curvatures have the same sign at each point.
Therefore we can choose a unit normal $\Norm$ at each point so that both principle curvatures are positive.
Show that it defines a field on the surface.

\parbf{\ref{ex:normal-curvature=const}}; \ref{SHORT.ex:normal-curvature=const:a}.
Observe that $\Sigma$ has unit Hessian matrix at each point and apply the definition of shape operator.

\parit{\ref{SHORT.ex:normal-curvature=const:b}.} Fix a chart $s$ in $\Sigma$ and show that 
\[\tfrac{\partial }{\partial u}(s(u,v)+\nu(u,v))
=
\tfrac{\partial }{\partial v}(s(u,v)+\nu(u,v))
=
0.\]
Make a conclusion.

\parbf{\ref{ex:shape-curvature-line}.} 
We can assume that $\gamma$ is parameterized by its arc-length.
Denote by $\Norm_1(s)$ and $\Norm_2(s)$ the unit normal vectors to $\Sigma_1$ and $\Sigma_2$ at $\gamma(s)$.
Since $\gamma$ is a curvature line in $\Sigma_1$, we have 
$\Norm_1'$ is proportional to $\gamma'$;
in particular 
\[\langle\Norm_1',\Norm_2\rangle=0.\]

Note that $\langle \Norm_1(t),\Norm_2(t)\rangle$ is constant.
By taking its derivative and applying the above identity show that
\[\langle\Norm_1,\Norm_2'\rangle=0.\]
Conclude that $\Norm_2'$ is proportional to $\gamma'$
and therefore $\gamma$ is a curvature line in $\Sigma_2$.


\parbf{\ref{ex:meusnier}.} Use Meusnier's theorem (\ref{thm:meusnier}), to find center and radius of curvature of $\gamma$ in terms of normal curvature of $\gamma$ at $p$;
make a conclusion.

\parbf{\ref{ex:principle-revolution}.}
Use \ref{ex:line-of-curvature} and Meusnier's theorem (\ref{thm:meusnier}).

\parbf{\ref{ex:catenoid-is-minimal}.} Use \ref{ex:principle-revolution}.

\parbf{\ref{ex:helicoid-is-minimal}.} Apply Meusnier's theorem (\ref{thm:meusnier}) to show that the coordinate curves $\alpha_v\:u\mapsto s(u,v)$ and $\beta_u\:v\mapsto s(u,v)$ are asymptotic; that is they have vanishing normal curvature.

Observe that these two families are orthogonal to each other.
Therefore the Hessian matrix in the frame $\tfrac{\partial s}{\partial u}/|\tfrac{\partial s}{\partial u}|$ and $\tfrac{\partial s}{\partial v}/|\tfrac{\partial s}{\partial v}|$ will have zeros on the diagonal.
Apply that mean curvature is the trace of the Hessian matrix.



\parbf{\ref{ex:moon-revolution}.} Use \ref{ex:line-of-curvature} and \ref{thm:moon-orginal}.

\parbf{\ref{ex:lagunov-genus4}.} Drill an extra hole or combine two examples together.

\parbf{\ref{ex:thin}.} 
Let us define \emph{cut locus} of $V$ as a closure of the set of points $x\in V$ such that there are at least two points on $\partial V$ that minimize the distance to $x$.

Denote by $L$ the cut locus $L$ of $V$.

Choose a connected component $\Sigma$ of the boundary $\partial V$.
Show that $L$ is a smooth surface and the closest point projection $L\to \Sigma$ is a smooth regular parameterization.
In particular there is unique point on $\Sigma$ that minimize the distance to a given point $x\in L$.
It follows that there is another component $\Sigma'$ with the same property.

Finally show that $\partial V$ can not be more than two components.

\parbf{\ref{ex:PI-sphere}.}
Read about Bing's two-room house.
Try to thicken it to construct the needed example.

Assume $V$ does not contain a ball of radius $r_3$.
Show that its cut locus $L$ of $V$ is formed by a few smooth surfaces that meets by three at their boundary points.
Show that $L$ is not simply connected that is there is a loop in $L$ that can not be deformed continuously to a trivial loop.
Conclude that $V$ is not simply connected.

Finally show that if $V$ is bounded by sphere, then $V$ is simply connected --- a contradiction. 

\parbf{\ref{ex:divergence-1}}; \ref{SHORT.ex:divergence}.
Since $\div\vec k=0$, applying the divergence theorem to the domain bounded by $\Delta$ and $\Sigma$, we get that 
\[\flux_{\vec k}\Sigma=\flux_{\vec k}\Delta.\]
Since $|\vec k|=1$, we have 
\[\flux_{\vec k}\Sigma\le \area\Sigma.\]
It remains to observe that
\[\flux_{\vec k}\Delta=\area\Delta.\]


\parit{\ref{SHORT.ex:curl}.} Consider the field $\vec u=x\cdot\vec j$.
Observe that $\vec k=\curl \vec u$.
Therefore by the curl theorem (\ref{thm:curl}) we have 
\[\flux_{\vec k}\Sigma=\int_0^\ell\langle\vec u,\gamma'(t)\rangle\cdot dt=\flux_{\vec k}\Delta,\]
where $\gamma\:[0,\ell]\to\RR^3$ is the common boundary of $\Sigma$ and $\Delta$ parameterized by its arc-length and with the right choice of orientation.

The same argument as in \ref{SHORT.ex:divergence} finishes the proof.



\parbf{\ref{ex:divergence-2}.}
Observe that $\div\vec u=1$.
Applying the divergence theorem and the observation (\ref{obs:flux}) we get
\begin{align*}
\vol R&=\iiint_R\div\vec u\cdot dx\cdot dy\cdot dz
=
\flux_{\vec u}\Sigma
\le
\area\Sigma.
\end{align*}




\parbf{\ref{ex:mean-convex}}; \ref{SHORT.ex:mean-convex:u}.
We may assume that $p$ is the origin of $\RR^3$.
Let us extend the outer normal field $\Norm$ to $\Sigma$ to a field $\vec u$ defined in $\RR^3\backslash\{0\}$ by
\[\vec u(\lambda\cdot q)=\Norm(q)\]
for any $\lambda>0$ and $q\in\Sigma$.

Observe that $\vec u(\lambda\cdot q)$ is normal to the surface 
\[\lambda\cdot\Sigma=\set{\lambda\cdot q}{q\in\Sigma}.\]
By \ref{lem:div+H}, we have
\[\div \vec u=-H(\lambda\cdot q)_{\lambda\cdot\Sigma}\ge 0.\]

\parit{\ref{SHORT.ex:mean-convex:area}.} Apply the divergence theorem for the region squeezed between $\Sigma$ and $\Sigma'$.

\parit{\ref{SHORT.ex:mean-convex:wrong}.}
An example they can be found among bodies of revolution as shown on the picture.
The gutter in the middle can be chosen to be taken to be a catenoid which is a minimal surface; see \ref{ex:catenoid-is-minimal}.
\begin{figure}[h!]
\vskip-0mm
\centering
\includegraphics{mppics/pic-300}
\vskip0mm
\end{figure}
Show that if the gutter is sufficiently deep, then the surface of revolution of the dashed line can be made smaller.

\parbf{\ref{ex:area-ball-intersection}.}

\parbf{\ref{ex:catenoid-nonmin}.}
Consider the region of catenoid in the cylinder $x^2+y^2\le R^2$ for $R>1$.
It is bounded by two circles defined by the equations 
\[
\begin{cases}
x^2+y^2\le R^2,
\\
z=\pm r
\end{cases}
\]
where $\cosh r=R$.

Show that if $R$ is large, the area of this region is at least $\pi\cdot R^2$ --- the area of disc of radius $R$.

Observe that the lateral surface of cylinder bounded by the two circles is $4\cdot \pi\cdot R\cdot r$.
Since $r/R\to 0$ as $R\to infty$, we get that this surface has smaller area then region of catenoid.


\parbf{\ref{ex:helicoid-nonmin}.}
Show that for large $R$, the area of the helicoid in the cylinder defined by $|z|\le R$ and $x^2+y^2\le R^2$ exceeds the area of the surface of cylinder.
Make a conclusion.

\parbf{\ref{ex:surf-support}.}
Choose curvatures such that 
\[k_2(p)_{\Sigma_1}\z>k_2(p)_{\Sigma_2}> k_1(p)_{\Sigma_1}> k_1(p)_{\Sigma_2}\] and suppose that the first principle direction of $\Sigma_1$ coincides with the second principle direction of $\Sigma_2$ and the other way around.

\parbf{\ref{ex:positive-gauss-0} and \ref{ex:positive-gauss}.} Apply the same reasoning as in the problems \ref{ex:between-parallels-1}--\ref{ex:lens}, but use families of spheres instead.


\parbf{\ref{ex:convex-surf}.} Show and use that any tangent plane $\T_p$ supports $\Sigma$ at $p$.

\parbf{\ref{ex:convex-lagunov}.}
Assume a maximal ball in $V$ touches its boundary at the points $p$ and $q$.
Consider the projection of $V$ to a plane thru $p$, $q$ and the center of the ball. 

\parbf{\ref{ex:section-of-convex}.}
Suppose that a point $p$ lies in the intersection $\Pi\cap\Sigma$.

Show that if the tangent plane $\T_p\Sigma$ is parallel to $\Pi$,
then $p$ is an isolated point of the intersection $\Pi\cap\Sigma$.

It follows that if $\gamma$ is a connected component of the intersection $\Pi\cap\Sigma$ that is not an isolated point,
then at each point $p$ on $\gamma$ the tangent plane $\T_p\Sigma$ is transversal to $\Pi$.
Apply the implicit function theorem to show that $\gamma$ is a smooth regular curve.

Finally, observe that curvature of $\gamma$ can not be smaller than normal curvature of $\Sigma$ in the same direction.
Whence $\gamma$ has no points with vanishing curvature.
Therefore for a right choice of orientation of $\gamma$, we its signed curvature is positive.

\parbf{\ref{ex:surrounds-disc}.}
Look for a supporting spherical dome with the unit circle as the boundary.

\parbf{\ref{ex:small-gauss}.}
Note that we can assume that the surface has positive Gauss curvature, otherwise the statement is evident.
Therefore the surface bounds a convex region that contains a line segment of length~$\pi$.

\begin{figure}[h!]
\vskip-0mm
\centering
\includegraphics{asy/sin}
\vskip-0mm
\end{figure}

Observe that the Gauss curvature of the surface of revolution of the graph $y=a\cdot \sin x$ for $x\in(0,\pi)$ cannot exceed $1$ (Use \ref{ex:curvature-graph} and \ref{cor:meusnier}).
Try to support the surface $\Sigma$ from inside by a surface of revolution of the described type. 

\parit{Remark.}
In fact if Gauss curvature of $\Sigma$ is at least $1$,
then
the intrinsic diameter of $\Sigma$ can not exceed $\pi$.
The latter means that any two points in $\Sigma$ can be connected by a path that lies in $\Sigma$ and has length at most~$\pi$.

\parbf{\ref{ex:convex-proper-sphere}.}

\parbf{\ref{ex:convex-proper-plane}.}

\parbf{\ref{ex:open+convex=plane}.}
By \ref{ex:convex-proper-plane:d}, $\Sigma$ is parameterized by an open convex plane domain $\Omega$.
It remains to show that $\Omega$ can parameterize the whole plane.

We may assume that the origin of the plane leis in $\Omega$.
Show that in this case the boundary of $\Omega$ can be written in polar coordinates as $(\theta,f(\theta))$ where $f\:\SS^1\to\RR$ is a positive continuous function.
Then homeomorphism to the plane can be described in polar coordinate by changing only the radial coordinate;
for example as 
$(\theta,r)\mapsto (\theta,
\tfrac1{f(\theta)-r}-\tfrac1{f(\theta)})$.

To do the second part ane may apply \ref{ex:star-shaped-disc:nonsmooth}.


\parbf{\ref{ex:circular-cone}.}
Choose a coordinate system so that $(x,y)$-plane supports $\Sigma$ at the origin, so $\Sigma$ lies in the upper half-space.

Show that there is $\eps>0$ such that any line starting from the origin with slope at most $\eps$ may intersect $\Sigma$ only in the unit ball centered at the origin;
we may assume that $\eps$ is small, say $\eps<1$.
Consider the cone formed by half-lines from the origin with slope $\eps$ shifted down by 10 and observe that entire surface lies in this cone.



\parbf{\ref{ex:intK}}.
Choose distinct points $p,q\in\Sigma$.
Apply \ref{thm:convex-embedded} to show that the angle 
$\measuredangle(\Norm(p),p-q)$ is acute and $\measuredangle(\Norm(q),p-q)$ is obtuse.
Conclude that $\Norm(p)\ne\Norm(q)$;
that is, $\Norm\:\Sigma\to\mathbb{S}^2$ is injective.


\parit{\ref{SHORT.ex:intK:4pi}.}
Given a unit vector $\vec u$, consider a point $p\in \Sigma$ that maximize the scalar product $\langle p,\vec u\rangle$.
Show that $\Norm(p)=\vec u$.
Conclude that the spherical map $\Norm\:\Sigma\to\mathbb{S}^2$ is onto.
It follows that $\Norm\:\Sigma\to\mathbb{S}^2$ is bijection.

Applying \ref{cor:intK}, we get that 
\begin{align*}
\int_\Sigma K&=\area\mathbb{S}^2
\\
&=4\cdot\pi.
\end{align*}

\parit{\ref{SHORT.ex:intK:2pi}.} Choose an $(x,y,z)$-coordinate system provided by \ref{ex:convex-proper-plane:d}.
Observe that for any $p$ the normal vector $\Norm(p)$ forms an obtuse angle with the direction of $z$-axis.
It follows that the image $\Norm(\Sigma)$ lies in the south hemisphere.

Applying \ref{cor:intK}, we get that 
\begin{align*}
\int_\Sigma K&\le \tfrac12\cdot\area\mathbb{S}^2
\\
&=2\cdot\pi.
\end{align*}

\parbf{\ref{ex:convex-revolution}.} Use \ref{ex:principle-revolution}.

\parbf{\ref{ex:ruled=>saddle}.} Prove and use that each point $p\in\Sigma$ has a direction with vanishing normal curvature.

\parbf{\ref{ex:saddle-to-infty}.}

\parbf{\ref{ex:panov}.} Denote by $\Pi_t$ the tangent plane to $\Sigma$ at $\gamma(t)$ and by $\ell_t$ the tangent line of $\gamma$ at time $t$.

Since $\gamma$ is asymptotic, the plane $\Pi_t$ rotates around $\ell_t$ as $t$ changes.
Since $\Sigma$ is saddle, the speed of rotation cannot vanish.%
\footnote{In fact by the Beltrami--Enneper theorem, if $\gamma$ has unit speed, then the speed of rotation is $\pm\sqrt{-K}$, where $K$ is the Gauss curvature which cannot vanish on a saddle surface.}

Note that $\Pi_t$ is a graph of a linear function, say $h_t$, defined on the $(x, y)$-plane.
Denote by $\bar\ell_t$ the projection of $\ell_t$ to the $(x, y)$-plane.
The described rotation of $\Pi_t$ can be expressed algebraicaly:
the derivative $\tfrac{d}{dt}h_t(w)$ vanishes at the point $w$ if and only if $w\in \bar\ell_t$ 
and the derivative changes sign if $w$ changes the side of $\bar\ell_t$.

Denote by $\bar\gamma$ the projection of $\gamma$ to the $(x, y)$-plane.
If $\bar\gamma$ is star shaped with respect to a point $w$, then $w$ cannot cross $\bar\ell_t$.
Therefore the function $t\mapsto h_t(w)$ is monotone on $\SS^1$.
Observing that this function cannot be constant, we arrive to a contradiction.

\parit{Soruse:} The problem is discuss by Dmitri Panov \cite{panov-curves}.

\parbf{\ref{ex:length-of-bry}.} Use the \ref{lem:convex-saddle} and the hemisphere lemma (\ref{lem:hemisphere}).

\parbf{\ref{ex:circular-cone-saddle}.}
Assume $\Sigma$ is an open saddle surface that lies in a cone $K$.
Show that there is a plane $\Pi$ that cuts $\Sigma$ and cuts from $K$ a compact region.
Observe that $\Pi$ cuts from $\Sigma$ a compact region.

By \ref{lem:reg-section} one can move plane $\Pi$ slightly so that it cuts from $\Sigma$ a compact surface with boundary.
Apply \ref{lem:convex-saddle}.


\parbf{\ref{ex:disc-hat}.} Observe that it is sufficient to construct a smooth parametrization of $F_\eps$ by a closed hemisphere.

Consider the radial projection of $F_\eps$ to the sphere $\Sigma$ with the center at $p=(0,0,\eps)$;
that is a point $q\in F_\eps$ is mapped to a point $s(q)$ on the sphere that lies on the ray $pq$.

Show that $s$ is a diffeomorphism from $F_\eps$ to a south hemisphere of~$\Sigma$.

\parbf{\ref{ex:saddle-linear}.} Apply \ref{prop:hat}.

\parbf{\ref{ex:between-parallels}.} Look for an example among the surfaces of revolution and use \ref{ex:principle-revolution}.

\parbf{\ref{ex:one-side-bernshtein}.} Look at the sections of the graph by planes parallel to the $(x,y)$-plane and to the $(x,z)$-plane, then apply Meusnier’s theorem (\ref{thm:meusnier}).

\parbf{\ref{ex:saddle-graph}.}
Suppose that orthogonal projection of $\Sigma$ to the $(x,y)$-plane is not injective.
Show that there is a point $p\in\Sigma$ with vertical tangent plane;
that is $\T_p\Sigma$ is perpendicular to the $(x,y)$-plane.

Let $\Gamma_p$ be the connected component of $p$ in the intersection of $\Sigma$ and $\T_p$.
Show that (1) $\Gamma_p$ is a tree, (2) each vertex of $\Gamma_p$ has degree 4 or 1, (3) $p$ is a vertex of degree 4, and (4) each endvertex lies on the bounary of $\Sigma$.

Observe that $\T_p$ meets the boundary of $\Sigma$ at two points,
therefore $\Gamma$ has 2 endpoints.
Use it to get a contradiction.

\parbf{\ref{ex:hat-convex}.} Assume the contrary,
then there is a minimizing geodesic $\gamma\not\subset\Delta$ with ends $p$ and $q$ in $\Delta$.

Without loss of generality, we may assume that only one arc of $\gamma$ lies outside of $\Delta$.
Reflection of this arc  with respect to $\Pi$ together with the remaining part of $\gamma$ forms another curve $\hat\gamma$ from $p$ to $q$;
it runs partly along $\Sigma$ 
and partly outside $\Sigma$,
but does not get inside $\Sigma$.
Note that
\[\length\hat\gamma=\length\gamma.\]


Denote by $\bar\gamma$ the closest point projection of $\hat\gamma$ on $\Sigma$.
Note that the curve $\bar\gamma$ lies in $\Sigma$, 
it has the same ends as $\gamma$,
and by \ref{cor:shorts+convex}
\[\length\bar\gamma<\length\gamma.\]
This means that $\gamma$ is not length minimizing, 
a contradiction.

\parbf{\ref{ex:intrinsic-diameter}.} Use \ref{lem:closest-point-projection}.

\parbf{\ref{ex:reflection-geodesic}.}
Denote by $\mu$ a unit vector perpendicular to $\Pi$.
Since $\gamma$ lies in $\Pi$, we have that $\gamma''$ is parallel to $\Pi$, or equivalently $\gamma''\perp \mu$.
Since $\gamma$ is unit speed, \ref{prop:a'-pertp-a''} implies that $\gamma''\perp\gamma'$.

Since $\Sigma$ is mirror symmetric with respect to  a plane $\Pi$,
the tangent palne $\T_{\gamma(t)}\Sigma$ is also mirror symmetric with respect to  a plane $\Pi$.
It follows that $\T_{\gamma(t)}\Sigma$ is spanned by $\mu$ and $\gamma'(t)$.
Therefore $\gamma''\perp \mu$ and $\gamma''\perp\gamma'$ imply $\gamma''\perp\T_{\gamma(t)}\Sigma$;
that is, $\gamma$ is a geodesic.

\parbf{\ref{ex:helix=geodesic}.}
Show that $\Norm_{\gamma(t)}=(\cos t,\sin t, 0)$.
Calculate $\gamma''(t)$ and show that it is proportional to $\Norm_{\gamma(t)}$.
Note that the latter is equivalent to $\gamma''(t)\perp\T_{\gamma(t)}$.

\parbf{\ref{ex:asymptotic-geodesic}.} Without loss of generality, we can assume that $\gamma$ has unit speed.
By the definition of geodesic, we have that $\gamma''(s)\perp\T_{\gamma(s)}$. 
Therefore 
\[\gamma''(s)=k_n(s)\cdot\Norm_{\gamma(s)},\]
where $k_n(s)$ is the normal curvature of $\gamma$ at time $s$.
Since $\gamma$ is asymptotic, $k_n(s)\equiv 0$;
that is, $\gamma''(s)\equiv 0$, therefore $\gamma'$ is constant and $\gamma$ runs along a line segment.

\parbf{\ref{ex:two-min-geod}.}
Assume that two shortest paths $\alpha$ and $\beta$ have two common point $p$ and $q$.
Denote by $\alpha_1$ and $\beta_1$ the arcs of
$\alpha$ and $\beta$ from $p$ to $q$.
Suppose that $\alpha_1$ is distinct from $\beta_1$.

\begin{wrapfigure}{r}{38 mm}
\vskip-0mm
\centering
\includegraphics{mppics/pic-308}
\vskip0mm
\end{wrapfigure}

Note that $\alpha_1$ and $\beta_1$ are shorest paths with the same endpoints;
in particular they have the same length.
Exchanging $\alpha_1$ in $\alpha$ to $\beta_1$ produces a shortest path, say $\hat\alpha$, that is distinct from $\alpha$.
By \ref{prop:gamma''}, $\hat\alpha$ is a geodesic.

Suppose $\alpha_1$ is a proper subarc of $\alpha$;
that is, if $\alpha_1\ne\alpha$, or, equivalently, either $p$ or $q$ is not an endpoint of $\alpha$.
Then $\alpha$ and $\hat\alpha$ share one point and velocity vector at this point.
By \ref{prop:geod-existence} $\alpha$ coincides with $\hat\alpha$ --- a contradiction.

It follows that $p$ and $q$ are the end points of $\alpha$.
The same way we can show that $p$ and $q$ are the end points of $\beta$.

\parbf{\ref{ex:min-geod+plane}.}
Assume a shortest path $\alpha$ changes the sides of $\Pi$ at least twice.
In this case there is an arc $\alpha_1$ of $\alpha$ that lies on one side on $\Pi$ and has ends on $\Pi$ and these ends are distinct from the ends of $\alpha$.

\begin{wrapfigure}{r}{38 mm}
\vskip-0mm
\centering
\includegraphics{mppics/pic-310}
\vskip0mm
\end{wrapfigure}

Let us remove the arc $\alpha_1$ from $\alpha$ and exchange it to the reflection of $\alpha_1$ across $\Pi$.
Note that the obtained curve, say $\beta$, lies in the surface; it has the same length as $\alpha$, and it connects the same pair of points, say $p$ and $q$.
Therefore $\beta$ is another shortest path from $p$ to $q$ that is distinct from $\alpha$.

By \ref{prop:gamma''}, $\alpha$ and $\beta$ are geodesics.
Since $\alpha$ and $\beta$ have a common subarc, they share one point and velocity vector at this point;
by \ref{prop:geod-existence} $\alpha$ coinsides with $\beta$ --- a contradiction.



\parbf{\ref{ex:milka}.} Show that the concatenation of the line segment $[p_t,\gamma(t)]$ and the arc $\gamma|_{[t,\ell]}$ is a shortest path in the closed region $W$ outside of $\Sigma$.

\parbf{\ref{ex:rho''}.}
Equip $\Sigma$ with unit normal field $\Norm$ that points inside.
Denote by $k(t)$ the normal curvature of $\Sigma$ at $\gamma(t)$ in the direction of $\gamma'(t)$.
Since $\Sigma$ is convex $k(t)\ge 0$ for any $t$.

Since $\gamma$ is geodesic, we have $\gamma''(t)=k(t)\cdot\Norm_{\gamma(t)}$.

Since $\gamma$ has unit speed, $\langle\gamma'(t),\gamma'(t)\rangle=1$ for any $t$.

Without loss of generality, we can assume that $p$ is the origin of $\RR^3$.
Since $p$ is inside $\Sigma$, we have that $\langle\gamma(t),\Norm_{\gamma(t)}\rangle\le 0$ for any $t$.
It follows that 
\[\langle\gamma''(t),\gamma(t)\rangle=k(t)\cdot \langle\gamma(t),\Norm_{\gamma(t)}\rangle\le 0\]
for any $t$.

Applying the above estimates, we get that 
\begin{align*}
\rho''(t)
&=\langle\gamma(t),\gamma(t)\rangle''
\\
&=2\cdot\langle\gamma''(t),\gamma(t)\rangle+2\cdot\langle\gamma'(t),\gamma'(t)\rangle\le 
\\
&\le 2.
\end{align*}

\parbf{\ref{ex:usov-exact}.} 
Suppose $\gamma(t)=(x(t),y(t),z(t))$. 
Show that
\[|\gamma'' (t)| =  z''(t)\cdot\sqrt{1+ \ell ^2}\eqlbl{eq:gamma''=z''}\]
for any $t$.

Observe that $z'(t)\to\pm \tfrac\ell{\sqrt{1+ \ell ^2}}$ as $t\to\pm\infty$.
Conclude that 
\[\int_{-\infty}^{+\infty}z''(t)\cdot dt
=
\tfrac{2\cdot\ell}{\sqrt{1+ \ell ^2}}.\eqlbl{eq:int z''}\]
By \ref{eq:gamma''=z''} and \ref{eq:int z''}, we have
\begin{align*}
\tc\gamma&=\int_{-\infty}^{+\infty}|\gamma''(t)|\cdot dt=
\\
&=\sqrt{1+ \ell ^2}\cdot \int_{-\infty}^{+\infty}z''(t)\cdot dt=
\\
&=2\cdot \ell.
\end{align*}

\parbf{\ref{ex:rough-bound-mountain}.} Use \ref{thm:usov} and \ref{ex:sef-intersection}.

The suggested argument does not give the optimal bound for the Lipschitz constant that guarantees that $\gamma$ is simple, but
later (see \ref{ex:sqrt(3)}) we will show that the exact bound is $\sqrt{3}=\tg\tfrac\pi3$ --- it is the same as in the exercise about mountain of with the shape of a perfect cone; see \ref{ex:lasso}.

\parbf{\ref{ex:lasso}.} Cut the lateral surface of the mountain by a line from the cowboy to the top, unfold it on the plane and try to figure out what is the image of the strained lasso.

Since the distance between points, can not be bigger than length of a path connecting them,
this statement implies the problem.

\parbf{\ref{ex:parallel}}; \ref{SHORT.ex:parallel:a}.
Show and use that $\langle\vec v(t),\vec v'(t)\rangle=0$.

\parit{\ref{SHORT.ex:parallel:b}}
Show that $|\vec v(t)|$, $|\vec w(t)|$, and
$\langle\vec v(t),\vec w(t)\rangle=0$,
are constants; it can be done the same way as \ref{SHORT.ex:parallel:a}.
Then use the formula 
\[\langle\vec v(t),\vec w(t)\rangle=|\vec v(t)|\cdot|\vec w(t)|\cdot\cos\theta.\]

\parbf{\ref{ex:parallel-transport-support}.}
Observe $\Sigma_1$ supports $\Sigma_2$ at any point of $\gamma$.
Conclude that $\gamma$ identical spherical images in $\Sigma_1$ and $\Sigma_2$ and apply Observation \ref{obs:parallel=}.

\parbf{\ref{ex:holonomy=not0}.}
Consider triangle that coordinate octant cuts form the sphere and try to argue that parallel transport around it rotates the tangent plane by angle $\tfrac\pi 2$. 

\parbf{\ref{ex:geodesic-curvature}.}
Suppose $\tan(t),\mu(t),\Norm(t)$ is the frame as in the definition of geodesic curvature.

If $\gamma$ is a geodesic, then by \ref{lem:constant-speed}, it has constant speed.
Applying scaling, we may assume that the speed is 1.
In this case 
\[\gamma''(t)=k_g(t)\cdot \mu(t)-k_n(t)\cdot \Norm(t).\]
Since $\gamma''(t)\perp\T_{\gamma(t)}$ we get that $k_g=0$. 
That proves the ``only if'' part.

Now assume that $\gamma$ has constant speed.
Againg, applying scaling, we may assume that the speed is 1.
In this case 
\[\gamma''(t)=-k_n(t)\cdot \Norm(t).\]
Therefore $\gamma''(t)\perp\T_{\gamma(t)}$.

\parbf{\ref{ex:half-sphere-total-curvature}.}
Apply \ref{prop:pt+tgc} and \ref{prop:area-of-spher-polygon}.

\parbf{\ref{ex:1=geodesic-curvature}.}
By \ref{ex:convex-proper-sphere}, $\Sigma$ is a smooth sphere.
By Jordan theorem (\ref{thm:jordan}) the curve $\gamma$ divides $\Sigma$  into two discs.
Let us denote by $\Delta$ the disc that lies on the left from $\gamma$.

Observe that $\tgc\gamma=\length\gamma$ and apply  the Gauss--Bonnet formula (\ref{thm:gb}) for $\Delta$.

\parbf{\ref{ex:geodesic-half}.}
Apply \ref{cor:intK}, \ref{ex:intK:4pi}, and the Gauss--Bonnet formula (\ref{thm:gb}).
To prove the last statement, apply \ref{ex:bisection-of-S2}.

\parbf{\ref{ex:sqrt(3)}.} Note that it is sufficient to show that the surface has no geodesic loops.
Estimate the integral of Gauss curvature of whole surface and a disc in it surrounded by a geodesic loop.

\parbf{\ref{ex:self-intersections}}; \textit{(a)}.
Consider the 4 regions bounded by loops.
Apply Gauss--Bonnet formula (\ref{thm:gb}) to show that the integral of Gauss curvature on each of these region exceeds $\pi$.
Therefore 
\[\int_\Sigma K>4\cdot\pi.\]
The latter contradicts \ref{ex:intK:4pi}.

\parit{(b)}.
Denote by $\alpha$, $\beta$, and $\gamma$ the angles of the triangle.
Apply the Gauss--Bonnet formula (\ref{thm:gb}) to show that the loops surround regions with integral of Gauss curvature $\pi+\alpha$, $\pi+\beta$, and $\pi+\gamma$ respectively.

Apply the Gauss--Bonnet formula for the triangular region to show that $\alpha+\beta+\gamma>\pi$.
It follows that 
\[\int_\Sigma K>(\pi+\alpha)+(\pi+\beta)+(\pi+\gamma)>4\cdot\pi\]
which contradicts \ref{ex:intK:4pi}.


\parbf{\ref{ex:deformation}.}


%\parbf{\ref{}.}

%\parbf{\ref{}.}

%\parbf{\ref{}.}

%\parbf{\ref{}.}

%\parbf{\ref{}.}
