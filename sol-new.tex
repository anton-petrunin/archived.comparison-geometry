\chapter{Semisolutions}

\parbf{Exercise \ref{ex:vertex-support}}

\parit{Using the spiral lemma.}
If $\gamma$ does not have a vertex at $s$ then $k'(s)\ne 0$ and therefore the curvature of a small arc around $s$ is monotonic.
By spiral lemma the osculating circles at this arc are nested.
In particular the curve $\gamma$ crosses the osculating circle $\sigma_s$ at $s$; 
that is, $\sigma_s$ is not a local support of $\gamma$ at $s$.

We proved that \emph{if $\gamma$ does not have a vertex at $s$ then the osculating circle $\sigma_s$ is not supporting at $s$}.
The latter is equivalent to the requred statement: \emph{if the osculating circle $\sigma_s$ supports $\gamma$ at $s$, then $\gamma$ has a vertex at~$s$}.

\parit{By direct calculations.}
Assume the osculating circline $\sigma_s$ is a circle.
Then its center is $p=\gamma(s)+\tfrac1{k(s)}\cdot\nu(s)$.
Since $\sigma_s$ is supporting $\gamma$ at $s$, we have that the function
\[f(t)=\langle p-\gamma(t),p-\gamma(t)\rangle\]
has a minimum or maximum at $s$.

Note that 
\begin{align*}
f'(t)&= 2\cdot\langle p-\gamma(t),-\tau(t)\rangle;
\\
f''(s)&= 
2\cdot\langle -\tau(t),-\tau(t)\rangle
-2\cdot\langle  p-\gamma(t),k(t)\cdot\nu(t)\rangle=
\\
&=2-2\cdot\langle  p-\gamma(t),k(t)\cdot\nu(t)\rangle;
\\
f'''(s)&=-2\cdot\langle  -\tau(t),k(t)\cdot\nu(t)\rangle-
\\&\quad-
2\cdot\langle  p-\gamma(t),k'(t)\cdot\nu(t)\rangle
+2\cdot\langle  p-\gamma(t),-k^2(t)\cdot\tau(t)\rangle=
\\
&=-
2\cdot\langle  p-\gamma(t),k'(t)\cdot\nu(t)\rangle
+2\cdot\langle  p-\gamma(t),-k^2(t)\cdot\tau(t)\rangle.
\end{align*}
Therefore
\begin{align*}
f'(s)&=-2\cdot\langle \tfrac1{k(s)}\cdot\nu(s),\tau(s)\rangle=
\\
&=0.
\\
f''(s)
&=2-2\cdot\tfrac{k(s)}{k(s)}=
\\
&=0.
\\
f'''(s)&=-2\cdot\tfrac{k'(s)}{k(s)}.
\end{align*}
Therefore if $f$ has a local minimum or maximum at $s$, then $f'''(s)=0$ and therefore $k'(s)=0$.

