\parbf{\ref{ex:ell-infty}.}

\parbf{\ref{ex:B2inB1}.}

\parbf{\ref{ex:shrt=>continuous}.}

\parbf{\ref{ex:close-open}.}

\parbf{\ref{ex:9}.} The image of $\gamma$ might have a shape of digit $8$ or $9$.

\parbf{\ref{aex:simple-curve}.}
Let $\alpha$ be a path connecting $p$ to $q$.

Passing to a subinterval if necessary,
we can assume that $\alpha(t)\ne p,q$ for $t\ne0,1$.

An open set $\Omega$ in $(0,1)$ will be called {}\emph{suitable}
if for any connected component $(a,b)$ of $\Omega$ we have $\alpha(a)=\alpha(b)$.
Show that the union of nested suitable sets is suitable.
Therefore we can find a maximal suitable set $\hat \Omega$.

Define $\beta(t)=\alpha(a)$ for any $t$ in a connected component $(a,b)\subset\Omega$.
Note that for any $x\in [0,1]$ the set $\beta^{-1}\{\beta(x)\}$ is connected.

It remains to re-parametrize $\beta$ to make it injective.
In other words we need to construct a non-decreasing surjective function $\tau\:[0,1]\z\to[0,1]$ such that 
$\tau(t_1)=\tau(t_2)$ if and only if there is a connected component $(a,b)$ such that $t_1,t_2\z\in [a,b]$.
The construction is similar to the construction of devil's staircase.


\parbf{\ref{ex:L-shape}.}
Denote the union of two half-axis by $L$.
Observe that $f(t)\to\infty$ as $t\to \infty$.
Since $f(0)=0$, the intermediate value theorem implies that $f(t)$ takes all nonnegative values for $t\ge 0$.
Use it to show that $L$ is the range of $\alpha$.

Note that the function $f$
is smooth.
Indeed, the existence of all derivatives $f^{(n)}(x)$ at $x\ne 0$ is evident and direct calculations show that $f^{(n)}(0)=0$ for all $n$.
Therefore $t\mapsto \alpha(t)=(f(t),f(-t))$ is smooth as well.

Further, show that the function $f$ is strictly increasing for $t> 0$,
and, moreover, if $0<t_0<t_1$, then $0<f(t_0)<f(t_1)$.
Use it to show that the maps $t\mapsto \alpha(t)$ is injective.

Summarizing we get that $\alpha$ is a smooth parametrization of $L$. 

Now suppose $\beta\:t\mapsto (x(t),y(t))$ is a smooth parameterization of~$L$.
Without loss of generality we may assume that $x(0)=y(0)=0$.
Note that $x(t)\ge 0$ for any $t$ therefore $x'(0)=0$.
The same way we get that $y'(0)=0$.
That is, $\beta'(0)=0$;
so $L$ does not admit a smooth regular parameterization.

\parbf{\ref{ex:cycloid}.}
Apply the definitions.
For \ref{SHORT.ex:cycloid:regular} you need to check that $\gamma'_\ell\ne 0$.
For \ref{SHORT.ex:cycloid:simple} you need to check that $\gamma_\ell(t_0)=\gamma(t_1)$ only if $t_0=t_1$.

\parbf{\ref{ex:y^2=x^3}.}
This is so called \emph{semicubical parabola}; it is shown on the diagram.
Try to argue similarly to \ref{ex:L-shape}. %???+PIC


\parbf{\ref{ex:viviani}.}
For $\ell=0$ the system describes a pair of points $(0,0,\pm1)$, so we can assume that $\ell\ne 0$.
Note that first equation desribes the unit sphere centered at the origin
and the second equation describes a cylinder over the circle in the $(x,y)$-plane with diameter with opposite points $(0,0)$ and $(0,\ell)$.


For $\ell\ne 0$,
find the gradients $\nabla f$ and $\nabla h$ for the functions
\begin{align*}
 f(x,y,z)&=x^2+y^2+z^2-1
 \\
 h(x,y,z)&=x^2+\ell\cdot x+y^2
\end{align*}
and show that they are linearly dependent only on the $x$-axis.
Conclude that for $\ell\ne\pm 1$ each connected component of the set of solutions is a smooth regular curve.

Show that 
\begin{itemize}
\item if $|\ell|<1$, then the set has two connected components with $z>0$ and $z<0$.
\item if $|\ell|\ge1$, then the set is connected.
\end{itemize}

Note that the condition on gradients provides only sufficient condition.
Therefore he case $\ell=\pm1$ has to be checked by hands.
In this case a neighborhood of $(\pm1,0,0)$ does not admit a smooth regular parametrization --- try to prove it. %???+PIC

\parit{Remark.}
In the case $\ell=\pm1$ it is called \emph{Viviani's curve}.
It admits the following smooth regular parameterization with a self-intersection:
\[t\mapsto(\pm(\cos t)^2,\cos t\cdot\sin t,\sin t).\]


\parbf{\ref{ex:proper-curve}.}
Without loss of generality we may assume that the origin does not lie on the curve.

Show that inversion of the plane $(x,y)\mapsto (\tfrac{x}{x^2+y^2},\tfrac{y}{x^2+y^2})$ maps our curve maps to a closed curve with removed origin.
Apply Jordan's theorem for the obtained curve and use the inversion again.

\parbf{\ref{ex:integral-length-0}.} Observe that if $c=\tau_0<\dots<\tau_n=d$ is a partition of $[c,d]$ if and only if $t_i=\phi(\tau_i)$ is a partition of $[a,b]$ and apply the definition of length (\ref{def:length}).

\parbf{\ref{ex:length-image}.}
Fix a partition $0=t_0<\dots <t_n=1$ of $[0,1]$.
Set $\tau_0=0$ and $\tau_i=\max\set{\tau \in[0,1]}{\beta(\tau_i)=\alpha(t_i)}$.
Show that $(\tau_i)$ is a partition of $[0,1]$;
that is, $0=\tau_0<\tau_1<\dots<\tau_n=1$.

By construction 
\begin{align*}
|\alpha(t_0)&-\alpha(t_1)|+|\alpha(t_1)-\alpha(t_2)|+\dots+|\alpha(t_{n-1})-\alpha(t_n)|=
\\
&=
|\beta(\tau_0)-\beta(\tau_1)|+|\beta(\tau_1)-\beta(\tau_2)|+\dots+|\beta(\tau_{n-1})-\beta(\tau_n)|.
\end{align*}
Since the partition $(t_i)$ is arbitrary, we get 
\[\length \beta\ge \length \alpha.\]

\parit{Remark.}
Note that the partition $(\tau_i)$ is not arbitrary, therefore the inequality might be strict; it might happen if $\beta$ runs back and forth along $\alpha$.




\parbf{\ref{ex:integral-length}.} For (\ref{ex:integral-length>}), apply the fundamental theorem of calculus for each segment in a given partition. For (\ref{ex:integral-length<}) consider a partition such that the velocity vector $\alpha'(t)$ is nearly constant on each of its segments.

\parbf{\ref{adex:integral-length}.} Use theorems of Rademacher and Lusin (\ref{thm:rademacher} and \ref{thm:lusin}).

\parbf{\ref{ex:nonrectifiable-curve}.} 

\parbf{\ref{ex:arc-length-helix}.} 
We have to assume that $a\ne 0$ or $b\ne0$;
otherwise we get a constant curve.

Show that the curve has constant velocity $|\gamma'(t)|\equiv \sqrt{a^2+b^2}$.
Therefore 
\[s=\tfrac{t}{\sqrt{a^2+b^2}}\] is an arc-length parameter.


\parbf{\ref{ex:convex-hull}.}
Choose a closed polygonal line $p_1\dots p_n$ inscribed in $\beta$.
By \ref{cor:convex=>rectifiable}, we can assume that it length is arbitrary close to the length of $\beta$;
that is, given $\eps>0$ 
\[\length (p_1\dots p_n)>\length\beta-\eps.\]

Show that we may assume in addition that each point $p_i$ lies on $\alpha$.

Observe that since $\alpha$ is simple, the points $p_1,\dots,p_n$ appear on $\alpha$ in the same cyclic order;
that is, the polygonal line $p_1\dots p_n$ is also inscribed in $\alpha$.
In particular 
\[\length\alpha\ge \length (p_1\dots p_n).\]
It follows that 
\[\length\alpha>\length\beta-\eps.\]
for any $\eps>0$.
Whence 
\[\length\alpha\ge\length\beta.\]

%???+PIC
If $\alpha$ has self-intersections, then the points $p_1,\dots, p_n$ might appear on $\alpha$ in a different order, say $p_{i_1},\dots,p_{i_n}$.
Apply triangle inequality to show that 
\[\length(p_{i_1}\dots p_{i_n})\ge \length (p_1\dots p_n)\]
and use it to modify the proof above.

\parbf{\ref{ex:convex-croftons}.} 
Denote by $\ell_u$ the line segment 
obtained orthogonal projection of $\gamma$ to the line in the direction $u$.
Note that $\gamma_u$ runs back and forth along $\ell_u$, we get 
\[\length\gamma_u\ge 2\cdot\length\ell_u.\]
Applying the Crofton formula, we get that 
\[\length\gamma\ge \pi\cdot \overline{\length\ell_u}.\]

In the case of equality, the curve $\gamma_u$ runs exactly back and forth along $\ell_u$ without additional zigzags for almost all (and therefore for all) $u$.

Let $K$ be a closed set bounded by $\gamma$.
Observe that the last statement implies that every line may intersect $K$ only along a closed segment.
In other words $K$ is convex.

\parbf{\ref{adex:more-croftons}.}
The proof is identical to the proof of the standard Crofton formula.
To find the coefficient one has to find average length of projection of unit vector to a line.
Which can be done by integration.
\begin{align*}
\frac1{k_a}&=\frac{1}{\area \SS^2}\cdot\int_{\SS^2} |x|;
&
\frac1{k_b}&=\frac{1}{\area \SS^2}\cdot\int_{\SS^2} \sqrt{1-x^2}.
\end{align*}
The answers are $k_a=2$ and $k_b=\tfrac4\pi$.

\parbf{\ref{ex-def:length-metric}.}

\parbf{\ref{ex:intrinsic-convex}.}
The ``only-if'' part is trivial.
To show the ``if'' part, assume $A$ is not convex;
that is, there are points $x,y\in A$ and a point $z\notin A$ that lies between $x$ and $y$.

Since $A$ is closed, its complement is open.
That is, the ball $B(z,\eps)$ does not intersect $A$ for some $\eps>0$.

Show that there is $\delta>0$ such that any path of length at most $|z-y|_{\RR^3}+\delta$ pass thru $B(z,\eps)$.
It follows that $|z-y|_A\ge |z-y|_{\RR^3}+\delta$, 
in particular$|z-y|_A\ne |z-y|_{\RR^3}$.

\parbf{\ref{ex:antipodal}.}
The spherical curve shown on the diagram does not have antipodal pairs of points.
However it has three points $x,y,z$ on one of its sides and their antipodal points $-x,-y,-z$ on the other.
Show that it is sufficient to conclude that the curve does not lie in any hemisphere.
%???PIC

\parbf{\ref{ex:bisection-of-S2}.}
Assume contrary, then by the hemisphere lemma (\ref{lem:hemisphere}) $\gamma$ lies in an open hemisphere.
In particular it cannot divide $\SS^2$ into two regions of equal area --- a contradiction.

\parbf{\ref{ex:flaw}.}
The very first sentence is wrong --- it is \emph{not} sufficient to show that diameter iis at most 2.
For example an equilateral triangle with circumradius slightly above $1$ may have diameter (which is defined as the maximal distance between its points) slightly bigger than $\sqrt3$, so it can be made smaller that $2$.
%???+PIC

On the other hand, it is easy to modify the proof of the hemisphere lemma (\ref{lem:hemisphere}) to get a correct solution.
That is, (1) choose two points $p$ and $q$ on $\gamma$ that divide it into two arcs of the same length;
(2) set $z$ to be a midpoint of $p$ and $q$,
and (3) show that $\gamma$ lies in the unit disc centered at $z$.


\parbf{\ref{adex:crofton}.}
For \ref{SHORT.adex:crofton:crofton}, modify the proof of the original Crofton formula [see page \pageref{page:crofton}].

\parit{\ref{SHORT.adex:crofton:hemisphere}.}
Assume $\length \gamma<2\cdot\pi$.
By \ref{SHORT.adex:crofton:crofton},
\[\overline{\length \gamma_u}<2\cdot\pi.\]
Therefore we can choose $u$ so that 
\[\length \gamma_u<2\cdot\pi.\]

Observe that $\gamma_u$ runs in a semicircle $h$ and therefore $\gamma$ lies in a hemisphere with $h$ as a diameter.


\parbf{\ref{ex:curvature-of-spherical-curve}.} Differentiate the identity $\langle\gamma(s),\gamma(s)\rangle=1$ a couple of times.

\parbf{\ref{ex:curvature-formulas}.} Prove and use the following identities: 
\begin{align*}
\gamma''(t)-\gamma''(t)^\perp&=\tfrac{\gamma'(t)}{|\gamma'(t)|}\cdot\langle\gamma''(t),\tfrac{\gamma'(t)}{|\gamma'(t)|}\rangle,
\\
|\gamma'(t)|&=\sqrt{\langle \gamma'(t),\gamma'(t)\rangle}.\
\end{align*}

\parbf{\ref{ex:curvature-graph}.} 
Apply \ref{ex:curvature-formulas:a} for the parameterization $t\mapsto (t,f(t))$.

\parbf{\ref{ex:helix-curvature}.}
Show that $\gamma_{a,b}''\perp \gamma'_{a,b}$ and apply \ref{ex:curvature-formulas:a}.

\parbf{\ref{ex:length>=2pi}.} Apply Fenchel's theorem.

\parbf{\ref{ex:gamma/|gamma|}.} Assume that $\gamma$ is unit-speed; show that $|\sigma'|\le \kur+\theta'$, where $\theta(s)=\measuredangle(\gamma(s),\gamma'(s))$.

\parbf{\ref{ex:chord-lemma-optimal}.} 
Set $\alpha=\measuredangle(\vec w,\vec u)$ and $\beta=\measuredangle(\vec w,\vec v)$.
Try to guess the example from the diagram;
the shown curve is divide into three arcs. 
The first arc turns from $\vec u$ to $\vec w$;
it has total curvature $\alpha$.
Analogously the third arc turns from $\vec w$ to $\vec v$  and has total curvature $\beta$. 
The second arc goes very close and almost parallel to the chord and its total curvature can be made arbitrary small.
%???+PIC

\parbf{\ref{ex:monotonic-tc}.}
Use that exterior angle of a triangle equals to the sum of the two remote interior angles;
for the second part apply the induction on number of vertexes.

\parbf{\ref{ex:sef-intersection}.} An example for \ref{SHORT.ex:sef-intersection:<2pi} is shown on the diagram. %???+PIC

\parit{\ref{SHORT.ex:sef-intersection:>pi}.} Assume $x$ is a point of self-intersection.
Show that we may choose two points $y$ and $z$ on $\gamma$ so that the triangle $xyz$ iz nondegenerate.
In particular, $\measuredangle xyz+\measuredangle yzx<\pi$, or, equivalently, 
\[\tc{xyzx}>\pi;\]
here we assume that $xyzx$ is polygonal line which is not closed. 
It remains to apply \ref{prop:inscribed-total-curvature}.

\parbf{\ref{ex:quadrisecant}.}
Observe that 
\[\tc{acbd}=4\cdot\pi,\]
here we assume that $acbd$ denotes the closed polygonal line.
It remains to apply \ref{prop:inscribed-total-curvature}.

{

\begin{wrapfigure}{r}{22 mm}
\vskip-6mm
\centering
\includegraphics{mppics/pic-255}
\vskip0mm
\end{wrapfigure}

\parbf{\ref{ex:anti-bow}.}
Start with with the curve $\gamma_1$ shown on the diagram.
To obtain $\gamma_2$, slightly unbend (that is, decrease the curvature of) the dashed arc of $\gamma_1$.

}

\parbf{\ref{ex:length-dist}.}
Choose a value $s_0\in[a,b]$ that splits the total curvature into two equal parts, $\theta$ in each.
Observe that $\measuredangle(\gamma'(s_0),\gamma'(s))\le \theta$ for any~$s$.
Use this inequality the same way as in the proof of the bow lemma.

\parbf{\ref{ex:schwartz}.}
Let $\ell=\length\gamma$.
Suppose $\ell_1<\ell<\ell_2$.
Let $\gamma_1$ be an arc of unit circle with length $\ell$.

Show that the distance between the ends of $\gamma_1$ is smallet than $|p-q|$ and apply the bow lemma (\ref{lem:bow}).


\parbf{\ref{ex:loop}.} If $\length\gamma<2\cdot\pi$, apply the bow lemma (\ref{lem:bow}) to $\gamma$ and an arc of unit circle of the same length.

\parbf{\ref{ex:tc-semicontinuous}.} Modify the proof of semi-continuity of length (\ref{thm:length-semicont}).

\parbf{\ref{ex:gen-fenchel}.}
Choose two distinct points $p$ and $q$ on $\gamma$.
Consider a \emph{diangle} $pq$ (that is, a closed polygonal line with two vertexes $p$ and $q$).
Observe that both external angles of the diangle have measure $\pi$.
Therefore the total curvature of diangle is $2\cdot \pi$.

It remains to apply the definition of total curvature for arbitrary curves (\ref{def:total-curv-poly}).

\parbf{\ref{ex:tc-length};} \ref{SHORT.ex:tc-length:rectifiable}.
Observe that if $\tc\gamma\le 1$, then
\[\length\gamma\le 2\cdot|\gamma(b)-\gamma(a)|.
\eqlbl{eq:arc<2chord}\]
Indeed, if $\tc\gamma\le 1$, then from the definition of total curvature [\ref{def:total-curv-poly}] it follows that for any polygonal line $p0\dots p_n$ inscribed in $\gamma$ the angle between any pair of edges does noe exceed $1$.
Let $q_i$ be the orthogonal projection of $p_i$ to the line of its first edge $p_0p_1$.
By the angle estimate, we have that the points $q_0,\dots,q_n$ appear on the line in the same order and 
\[|q_i-q_{i-1}|> \tfrac12\cdot|p_i-p_{i-1}|.\]
Therefore 
\begin{align*}|\gamma(b)-\gamma(a)|&=|p_n-p_0|\ge 
\\
&\ge |q_n-q_0|=
\\
&=|q_n-q_{n-1}|+\dots+|q_1-q_0|>
\\
&>\tfrac12\cdot(|p_n-p_{n-1}|+\dots+|p_1-p_0|).
\end{align*}
Recall that length of $\gamma$ is defined as the exact upper bound for the  sum $|p_n-p_{n-1}|+\dots+|p_1-p_0|$.
Therefore \ref{eq:arc<2chord} follows.

In general, if $\tc\gamma$ is bounded, we can subdivide $\gamma$ into arcs with total curvature at most $1$,
apply the above argument to each of arc and sum up the results.

\parit{\ref{SHORT.ex:tc-length:example}.}
The logarithmic spiral $\gamma[0,1]\to\RR^2$ is an example;
it can be defined in polar coordinates by $\gamma(t)=(t,\ln t)$ if $t>0$ and $\gamma(0)$ is the origin. %???PIC

It remains to show that indeed, $\gamma$ has infinite total curvature, but finite length.


%PIC



\parbf{\ref{ex:helix-torsion}.} 
The arc-length parameter $s$ is already found in   \ref{ex:arc-length-helix}.
It remains to find Frenet frame and calculate curvature and torsion.
The latter can be done by straightforward calculations.

One may be able to see the following geometrically, without calculations (or at least use the following observation to check the calculations).
Suppose $\tg\theta\z=b/a$, then $\tan(0)\z=(0,\cos\theta, \sin\theta)$, $\norm(0)\z=(-1,0,0)$, $\bi(0)\z=(0,\sin\theta, -\cos\theta)$.
Moreover if the helix is parameterized with arc-length then its Frenet frame rotates around $z$-axis with the angular velocity $\cos\theta$. 
%???PIC

If the calculations done right, then you should see that curvature $\kur$ and torsion $\tau$ do not depend on time and given.
Moreover for any $\kur>0$ and $\tor$ one can find $a$ and $b$ so that the helix $\gamma_{a,b}$ has curvature $\kur$ and torsion $\tor$.





\parbf{\ref{ex:beta-from-tau+nu}.} By product rule, we get
\begin{align*}
\bi'&=(\tan\times \norm)'=
\\
&=\tan'\times \norm+\tan\times\norm'.
\end{align*}
It remains to substitute the values from \ref{eq:frenet-tau} and \ref{eq:frenet-nu} and simplify.


\parbf{\ref{ex:torsion=0}.}
Show and use that the binormal vector is constant.

\parbf{\ref{ex:frenet}.} Observe that $\tfrac{\gamma'\times\gamma''}{|\gamma'\times\gamma''|}$ is a unit vector perpendicular to the plane spanned by $\gamma'$ and $\gamma''$, so, up to sign, it has to be equal to $\bi$. It remains to check that the sign is right.

%???

\parbf{\ref{ex:lancret}}, \ref{SHORT.ex:lancret:a}.
Observe that 
$\langle \vec w,\tan\rangle'=0$.
Show that it implies that $\langle \vec w, \norm\rangle =0$.

Further observe that 
$\langle \vec w,\norm\rangle'=0$.
Show that it implies that $\langle \vec w, -\kur\cdot\tan+\tor\cdot \bi\rangle =0$.

\parit{\ref{SHORT.ex:lancret:a}.}
Show that $\vec w'=0$;
it implies that $\langle \vec w,\tan\rangle=\tfrac\tor\kur$.
In particular, the velocity vector of $\gamma$ makes a constant angle with $\vec w$; that is, $\gamma$ has constant slope.

\parbf{\ref{ex:evolvent-constant-slope}.}
Show that $\langle w,\alpha\rangle$ is constant if $\gamma$ makes constant angle with a fixed vector $w$ and $\alpha$ is the evolvent of $\gamma$.

\parbf{\ref{ex:spherical-frenet}.}
Suppose $\langle \vec w,\tan\rangle$ is a constant.
Show that $\langle \vec w,\alpha\rangle'=0$.
It follows that $\langle \vec w,\alpha\rangle$ is a constant,
so $\alpha$ lies in a plane perpendicular to $\vec w$.


\parbf{\ref{ex:cur+tor=helix}.} Use the second statement in \ref{ex:helix-torsion}.

\parbf{\ref{ex:const-dist}.} Note that the function
\[\rho(\ell)=|\gamma(t+\ell)-\gamma(t)|^2=\langle \gamma(t+\ell)-\gamma(t),\gamma(t+\ell)-\gamma(t)\rangle\] 
is smooth and does not depend on $t$.
Express speed, curvature and torsion of $\gamma$ in terms of derivatives $\rho^{(n)}(0)$
and apply \ref{ex:cur+tor=helix}.

\parbf{\ref{ex:bike}.}
Without loss of generality, we may assume that $\gamma_0$ is parameterized by its arc-length.
Then
\begin{align*}
|\gamma_1'|&=|\gamma_0'+\tan'|=|\tan+\kur\cdot\norm|=
\sqrt{1+\kur^2}\ge
1=
|\gamma_0'|;
\end{align*}
that is, $|\gamma_1'(t)|\ge|\gamma_0'(t)|$ for any $t\in[a,b]$
The statement follows since 
\[\length\gamma_i=\int_a^b|\gamma_i'(t)|\cdot dt.\]





\parbf{\ref{ex:trochoids}.}
Observe that
\[\gamma'_a(t)=(1+a\cdot \cos t, -a\cdot \sin t);\]
that is $\gamma'_a$ runs clockwise along a circle with center at $(1,0)$ and radius $1$.
If $|a|>1$ then $\tan_a(t)=\gamma'_a/|\gamma'_a|$ runs clockwise and makes full turn in time $2\cdot\pi$.
It follows that if $|a|>1$, then
\[\Psi(\gamma_a)=-2\cdot\pi,\quad \tc\gamma_a=|\Psi(\gamma_a)|=2\cdot\pi.\]

If $|a|<1$, set $\theta_a=\arcsin a$.
Note that $\tan_a(t)=\gamma'_a/|\gamma'_a|$ starts with the horizontal direction $\tan_a(0)=(1,0)$, turns monotonically to angle $\theta_a$, then monotonically to $-\theta_a$ and then monotonically to back to $\tan_a(2\cdot\pi)=(1,0)$.
It follows that if $|a|>1$, then
\[\Psi(\gamma_a)=0,\quad \tc{\gamma_a}=4\cdot\theta_a.\]

In the cases $a=-1$ the velocity $\gamma'_{-1}(t)$ vanish at $t=0$ and $2\cdot\pi$.
Nevertheless, the curve admits a smooth regular parametrization --- show it.
In this case $\Psi(\gamma_{-1})=-\pi$ and $\tc{\gamma_{-1}}=\pi$.

In the cases $a=1$ the velocity $\gamma'_1(t)$ vanish at $t=\pi$.
At $t=\pi$ the curve has a cusp;
that is 
\[\lim_{t\to\pi-}\tan_1(t)=-\lim_{t\to\pi+}\tan_1(t).\]
So $\gamma_1(t)$ has undefined total signed curvature.
The curve is a joint of two smooth arcs with external angle $\pi$,
the total curvature of each arc is $\tfrac\pi2$, so 
$\tc{\gamma_{1}}=\tfrac\pi2+\pi+\tfrac\pi2=2\cdot\pi$.


\parbf{\ref{ex:zero-tsc}.}
The two marked points in the last example (for part \ref{SHORT.ex:zero-tsc:2-4}) have parallel tangent lines.

\begin{figure}[h!]
\begin{minipage}{.32\textwidth}
\centering
\includegraphics{mppics/pic-260}
\end{minipage}\hfill
\begin{minipage}{.32\textwidth}
\centering
\includegraphics{mppics/pic-261}
\end{minipage}
\hfill
\begin{minipage}{.32\textwidth}
\centering
\includegraphics{mppics/pic-262}
\end{minipage}
\end{figure}

\parbf{\ref{ex:length'}}; \ref{SHORT.ex:length':reg}.
Show that
\[
\gamma_\ell'(t)=(1-\ell\cdot\skur(t))\cdot \gamma'(t).
\]
Since $\gamma$ is regular, $\gamma'\ne0$.
Therefore if $\gamma_\ell'(t)=0$, then $\ell\cdot \skur(t)= 1$.

\parit{\ref{SHORT.ex:length':formula}.} Observe that we can assume that $\gamma$ is parameterized by its arc-length, so $\gamma'(t)=\tan(t)$.
Suppose $|\ell|<\frac{1}{\kur(t)}$ for any $t$.
Then 
\[
|\gamma_\ell'(t)|=(1-\ell\cdot\skur(t)).
\]
\begin{align*}
L(\ell)&=\int_a^b(1-\ell\cdot\skur(t))\cdot dt=
\\
&=\int_a^b 1\cdot dt-\ell\cdot \int_a^b\skur(t))\cdot dt=
\\
&=L(0)+\ell\cdot\Psi(\gamma).
\end{align*}



\parit{\ref{SHORT.ex:length':antiformula}.}
Consider the unit circle $\gamma(t)=(\cos t,\sin t)$ for $t\in[0,2\cdot\pi]$ and $\gamma_\ell$ for $\ell=2$.

\parbf{\ref{ex:inverse}.}
Use the definition of osculating circle via order of contact and that inversion maps circles to circlines. 

\parbf{\ref{ex:evolute-of-ellipse}.}
Find $\tan(t)$ and $\norm(t)$.
Use the formula in \ref{ex:curvature-formulas:b} to calculate curvature $\kur(t)$.
Apply the formula given right before the exercise.


\parbf{\ref{ex:3D-spiral}.} The plane curve shown on the diagram can be approximated by a smooth regular curve with strictly monotonic and large torsion at the dashed arc. %???+PIC 

\parbf{\ref{ex:double-tangent}.} 
Observe that if a line or circle is tangent to $\gamma$,
then it is tangent to osculating circle at the same point
and apply the spiral lemma (\ref{lem:spiral}).

%\parit{\ref{SHORT.ex:double-tangent:a}.}
%Suppose that a line $\ell$ is tangent to $\gamma$ at two points $p$ and $q$.
%Let $\sigma_p$ and $\sigma_q$ be the osculating circles of $\gamma$ at $p$ and $q$ correspondingly.
%Show that $\ell$ is tangent to $\sigma_p$ and $\sigma_q$.

%By \ref{lem:spiral}, we can assume that $\sigma_p$ is surrounded by $\sigma_q$.
%In particular any line that tangent to $\sigma_q$ cannot intersect $\sigma_p$;
%in particular, $\ell$ cannot be tangent to both $\sigma_p$ and $\sigma_q$ --- a contradiction. 

%\parit{\ref{SHORT.ex:double-tangent:b}.}
%If a circle $\sigma$ is tangent to $\gamma$ at three points $x$, $y$, and $z$.
%Then $\sigma$ is tangent to the osculating circles $\sigma_x$, $\sigma_y$, and $\sigma_z$.
%By \ref{lem:spiral} we can assume that $\sigma_x$ is surrounded by $\sigma_y$ and $\sigma_y$ is surrounded by $\sigma_z$.
%Observe that the first statement implies that $\sigma$ is squeezed between $\sigma_x$ and $\sigma_y$.
%The second statement implies that $\sigma$ is squeezed between $\sigma_y$ and $\sigma_z$.
%These two statements contradict each other.

\parbf{\ref{ex:vertex-support}.}
Apply the spiral lemma (\ref{lem:spiral}).

\parbf{\ref{ex:support}.} Consider the coordinate system with $p$ as the origin and $x$-axis as the common tangent line to $\gamma_1$ and $\gamma_2$.
We may assume that $\gamma_i$ are defined in $(-\eps,\eps)$ for some small $\eps>0$,
so that they run almost horizontally to the right.

Given $t\in[0,1]$ consider the curve $\gamma_t$ that is tangent and cooriented to the $x$ axis at  $\gamma_t(0)$ and has signed curvature defined by $\skur_t(s)=(1-t)\cdot\skur_0(s)+t\cdot\skur_1(s)$.
It exists by \ref{thm:fund-curves-2D}.

Fix $s\approx 0$.
Consider the curve $\alpha_s\:t\mapsto \gamma_t(s)$.
Show that $\alpha_s$ moves almost vertically up, while $\gamma_t$ moves almost horizontally to the right.
Conclude that in a small neighborhood of $p$, the curve $\gamma_1$
lies above $\gamma_0$.
Whence the statement follows.

\parbf{\ref{ex:in-circle}.} Let reduce the radius of the circle until it touches $\gamma$.
Observe that the circle supports $\gamma$ and apply \ref{prop:supporting-circline}.


\parbf{\ref{ex:between-parallels-1}, \ref{ex:in-triangle} and \ref{ex:lens}.}
Observe that one of the arcs of curvature 1 in the families shown on the diagram suppots $\gamma$ and apply \ref{prop:supporting-circline}.
\begin{figure}[h!]
\begin{minipage}{.50\textwidth}
\centering
\includegraphics{mppics/pic-265}
\end{minipage}\hfill
\begin{minipage}{.26\textwidth}
\centering
\includegraphics{mppics/pic-266}
\end{minipage}
\hfill
\begin{minipage}{.20\textwidth}
\centering
\includegraphics{mppics/pic-267}
\end{minipage}
\end{figure}
To do the second part in \ref{ex:between-parallels-1}, use shown family and another family of arcs curved in the opposite direction.


\parbf{\ref{ex:convex small}.}
Note that we can assume that $\gamma$ bounds a convex figure $F$, otherwise by \ref{prop:convex} its curvature changes the sign and therefore it has zero curvature at some point.
Choose two points $x$ and $y$ surrounded by $\gamma$ such that $|x-y|>2$,
look at the maximal lens bounded by two arcs with common chord $xy$ that lies in $F$ and apply supporting test (\ref{prop:supporting-circline}).

\parbf{\ref{ex:line-curve-intersections}}; \ref{SHORT.ex:line-curve-intersections:a}.
Apply the lens lemma to show that if $p_2$ lies between $p_1$ and $p_3$, then the curvature of $\gamma$ switches its sign.

\parit{\ref{SHORT.ex:line-curve-intersections:b}.} Show that in this $p_4$ lies between $p_2$ and $p_3$;
further $p_5$ lies between $p_3$ and $p_4$;
and so on.

\parit{\ref{SHORT.ex:line-curve-intersections:b}.}
According to \ref{SHORT.ex:line-curve-intersections:a}, the point $p_3$ might lie between $p_1$ and $p_2$ and then further order is determined uniquely or $p_1$ lies between $p_2$ and $p_3$.
In the latter case we have two choices, either $p_4$ lies between $p_2$ and $p_3$ and then then further order is determined uniquely or $p_2$ lies between $p_3$ and $p_4$.
In the latter case we get a choice again.

Assume we make a first choice on the step number $k$.
Without loss of generality we may assume that $p_k$ lies to the right from $p_{k-2}$.
Then we have the following order:
\[
p_{k-2},p_{p-4},\dots,p_{p-5},p_{k-3},p_k,p_{k+2},\dots,p_{k+1},p_{k-1}.
\]
The case $k=7$ shown on the diagram.

\begin{figure}[h!]
\vskip-0mm
\centering
\includegraphics{mppics/pic-29}
\vskip0mm
\end{figure}

\parbf{\ref{ex:moon-rad}.} Note that $\gamma$ contains a simple loop; apply to it \ref{thm:moon-orginal}.

\parbf{\ref{ex:curve-crosses-circle}.}
Repeat the proof of theorem for each cyclic concatenation of an arc of $\gamma$ from $p_i$ to $p_{i+1}$ with large arc of the circle. 

\parbf{\ref{ex:hyperboloinds}.} 
Denote the set of solutions by $\Phi_\ell$.

Show that $\nabla_p f=0$ if and only if $p=(0,0,0)$.
Use \ref{prop:implicit-surface} to conclude that if $\ell\ne 0$, then $\Phi_\ell$ is a union of disjoint smooth regular surfaces.

Show that $\Phi_\ell$ is connected if and only if $\ell\le 0$.
It follows that if $\ell<0$, then $\Phi_\ell$ is a smooth regular surface and if $\ell>0$ then it is not.

The case $\ell=0$ has to be done by hands --- it does not satisfy the sufficient condition in \ref{prop:implicit-surface}, but it does not solely imply that $\Phi_0$ is not a smooth surface.

Show that any neighborhood of origin in $\Phi_0$ can not be described by a graph in any coordinate system;
so by the definition (page \pageref{page:def-smooth-surface}) $\Phi_0$ is not a smooth surface.

\parbf{\ref{ex:inversion-chart}.} 
First check that the image of $s$ lies in the the unit sphere centered at $(0,0,1)$;
that is show that 
\[\left(\tfrac{2\cdot u}{1+u^2+v^2}\right)^2
+
\left(\tfrac{2\cdot v}{1+u^2+v^2}\right)^2
+\left(\tfrac{2}{1+u^2+v^2}-1\right)^2=1.\]
for any $u$ and $v$.

Further, show that the map 
\[(x,y,z)\mapsto (\tfrac{2\cdot x}{x^2+y^2+z^2},\tfrac{2\cdot y}{x^2+y^2+z^2})\]
describe the inverse map which is continuous away from the origin.
In particular, $s$ is an embedding that covers whole sphere except the origin.

It remains to show that $s$ is regular; that is, $\tfrac{\partial s}{\partial u}$ and $\tfrac{\partial s}{\partial v}$ are linearly independent at all points of the $(u,v)$-plane.

\parbf{\ref{ex:revolution}.}
Set
\[s\:(t,\theta)\mapsto (x(t), y(t)\cdot\cos\theta,y(t)\cdot\sin\theta).\]
Show that $s$ is regular; that is, $\tfrac{\partial s}{\partial t}$ and $\tfrac{\partial s}{\partial \theta}$ are linearly independent.
(It might help to observe that $\tfrac{\partial s}{\partial t}\perp\tfrac{\partial s}{\partial \theta}$).

Show that $s$ is local embedding; that is, any $(t_0,\theta_0)$ admits a neighborhood $U$ in the $(t,\theta)$-plane such that the restriction $s|_U$ has a continuous inverse.
It remains to apply \ref{cor:reg-parmeterization}.

\parbf{\ref{ex:tangent-normal}.}
Let $\gamma$ be a smooth curve in $\Sigma$.
Observe that $f\circ\gamma(t)\equiv 0$.
Differentiate this identity and apply the definition of tangent vector (\ref{def:tangent-vector}).

\parbf{\ref{ex:vertical-tangent}.}
Assume a neighborhood of $p$ in $\Sigma$ is a graph $z=f(x,y)$.
In this case $s\:(u,v)\mpasto (u,v,f(u,v))$ is a smooth chart at $p$.
Show that the plane spanned by $\tfrac{\partial s}{\partial u}$ and $\tfrac{\partial s}{\partial v}$ is not vertical;
together with \ref{def:tangent-plane} it proves the if part.

To prove the only-if part, fix a chart $s\:(u,v)\mapsto(x(u,v),y(u,v),z(u,v))$ at $p$ and apply the inverse function theorem for the map $(u,v)\mapsto(x(u,v),y(u,v))$.


\parbf{\ref{ex:implicit-orientable}.} Show that $\Norm=\tfrac{\nabla h}{|\nabla h|}$ defines a unit normal field on $\Sigma$.

\parbf{\ref{ex:plane-section}.} 
Fix a closed set $A$ in the $(x,y)$-plane.
Show that there is a smooth nonnegative function $(x,y)\mapsto f(x,y)$ such that $(x,y)\in A$ if and only if $f(x,y)=0$.
Observe that the graph $z=f(x,y)$ describes a required surface.

\parbf{\ref{ex:line-of-curvature}.}
Fix a point $p$ on $\gamma$.
Since $\Sigma$ is mirror symmetric with respect to $\Pi$,
so is the tangent plane $\T_p$.

Choose $(x,y)$-coordinates on $\T_p$ so that the $x$-axis is the intersection $\Pi\cap  \T_p$.
Suppose that the osculating paraboloid is described by the graph 
\[z=\tfrac12\cdot(\ell\cdot x^2+2\cdot m\cdot x\cdot y+n\cdot y^2)\]
Since $\Sigma$ is mirror symmetric, so is the paraboloid;
that is, changing $y$ to $(-y)$ does not change the value 
$\ell\cdot x^2+2\cdot m\cdot x\cdot y+n\cdot y^2$.
In other words $m=0$, or equivalently, the $x$-axis points in the direction of curvature.

\parbf{\ref{ex:gauss+orientable}.} Note that the principle curvatures have the same sign at each point.
Therefore we can choose a unit normal $\Norm$ at each point so that both principle curvatures are positive.
Show that it defines a field on the surface.

\parbf{\ref{ex:normal-curvature=const}}; \ref{SHORT.ex:normal-curvature=const:a}.
Observe that $\Sigma$ has unit Hessian matrix at each point and apply the definition of shape operator.

\parit{\ref{SHORT.ex:normal-curvature=const:b}.} Fix a chart $s$ in $\Sigma$ and show that 
\[\tfrac{\partial }{\partial u}(s(u,v)+\nu(u,v))
=
\tfrac{\partial }{\partial v}(s(u,v)+\nu(u,v))
=
0.\]
Make a conclusion.

\parbf{\ref{ex:shape-curvature-line}.} 
We can assume that $\gamma$ is parameterized by its arc-length.
Denote by $\Norm_1(s)$ and $\Norm_2(s)$ the unit normal vectors to $\Sigma_1$ and $\Sigma_2$ at $\gamma(s)$.
Since $\gamma$ is a curvature line in $\Sigma_1$, we have 
$\Norm_1'$ is proportional to $\gamma'$;
in particular 
\[\langle\Norm_1',\Norm_2\rangle=0.\]

Note that $\langle \Norm_1(t),\Norm_2(t)\rangle$ is constant.
By taking its derivative and applying the above identity show that
\[\langle\Norm_1,\Norm_2'\rangle=0.\]
Conclude that $\Norm_2'$ is proportional to $\gamma'$
and therefore $\gamma$ is a curvature line in $\Sigma_2$.

\parbf{\ref{ex:meusnier}.}

\parbf{\ref{ex:principle-revolution}.} Use \ref{ex:line-of-curvature} and \ref{cor:meusnier}.

\parbf{\ref{ex:catenoid-is-minimal}.}

\parbf{\ref{ex:helicoid-is-minimal}.}

\parbf{\ref{ex:moon-revolution}.} Use \ref{ex:line-of-curvature} and \ref{thm:moon-orginal}.

\parbf{\ref{ex:lagunov-genus4}.} Drill an extra hole or combine two examples together.

\parbf{\ref{ex:thin}.}

\parbf{\ref{ex:PI-sphere}.}

\parbf{\ref{ex:divergence-1}.}

\parbf{\ref{ex:divergence-2}.}

\parbf{\ref{ex:mean-convex}.} ???
There are mean-convex bodies bodies $V$ that are not star-shaped for which the conclusion of the exercise does not hold.
For example they can be found among bodies of revolution as shown on the picture --- one only has to check that mirror-symmetric figures $F$ and $F'$ as on the picture can be chosen in such a way that the body of revolution of $F$ is mean-convex wile the body of revolution of $F'$ has smaller surface area. 

A proof of this stronger statement can be build on an analog of comparison theorem \ref{thm:comp}(\ref{thm:comp:toponogov}) for surfaces with Gauss curvature at least~1.

\parbf{\ref{ex:catenoid-nonmin}.}

\parbf{\ref{ex:helicoid-nonmin}.}

\parbf{\ref{ex:surf-support}.}

\parbf{\ref{ex:positive-gauss-0}.}

\parbf{\ref{ex:positive-gauss}.}
Consider the minimal sphere that encloses the surface.

\parbf{\ref{ex:convex-surf}.} Show and use that any tangent plane $\T_p$ supports $\Sigma$ at $p$.

\parbf{\ref{ex:convex-lagunov}.}
Assume a maximal ball in $V$ touches its boundary at the points $p$ and $q$.
Consider the projection of $V$ to a plane thru $p$, $q$ and the center of the ball. 

\parbf{\ref{ex:section-of-convex}.}

\parbf{\ref{ex:surrounds-disc}.}
Look for a supporting spherical dome with the unit circle as the boundary.

\parbf{\ref{ex:small-gauss}.}
Note that we can assume that the surface has positive Gauss curvature, otherwise the statement is evident.
Therefore the surface bounds a convex region that contains a line segment of length~$\pi$.

\begin{figure}[h!]
\vskip-0mm
\centering
\includegraphics{asy/sin}
\vskip-0mm
\end{figure}

Observe that the Gauss curvature of the surface of revolution of the graph $y=a\cdot \sin x$ for $x\in(0,\pi)$ cannot exceed $1$ (Use \ref{ex:curvature-graph} and \ref{cor:meusnier}).
Try to support the surface $\Sigma$ from inside by a surface of revolution of the described type. 

\parit{Remark.}
In fact if Gauss curvature of $\Sigma$ is at least $1$,
then
the intrinsic diameter of $\Sigma$ can not exceed $\pi$.
The latter means that any two points in $\Sigma$ can be connected by a path that lies in $\Sigma$ and has length at most~$\pi$.

\parbf{\ref{ex:convex-proper-sphere}.}

\parbf{\ref{ex:convex-proper-plane}.}

\parbf{\ref{ex:open+convex=plane}.}

\parbf{\ref{ex:circular-cone}.}

\parbf{\ref{ex:intK}.}


\parbf{\ref{ex:convex-revolution}.} Use \ref{ex:principle-revolution}.

\parbf{\ref{ex:ruled=>saddle}.} Prove and use that each point $p\in\Sigma$ has a direction with vanishing normal curvature.


\parbf{\ref{ex:saddle-to-infty}.}

\parbf{\ref{ex:panov}.}

\parbf{\ref{ex:length-of-bry}.} Use the \ref{lem:convex-saddle} and the hemisphere lemma (\ref{lem:hemisphere}).

\parbf{\ref{ex:circular-cone-saddle}.}

\parbf{\ref{ex:disc-hat}.} Observe that it is sufficient to construct a smooth parametrization of $\Delta_\eps$ by a closed hemisphere.
To do this repeat the argument in \ref{lem:gauss=sphere} with the center at a point surrounded by the boundary line of $\Delta_\eps$ in its plane.

\parbf{\ref{ex:saddle-linear}.}

\parbf{\ref{ex:between-parallels}.} Look for an example among the surfaces of revolution and use \ref{ex:principle-revolution}.

\parbf{\ref{ex:one-side-bernshtein}.} Look at the sections of the graph by planes parallel to the $(x,y)$-plane and to the $(x,z)$-plane, then apply Meusnier’s theorem apply Meusnier's theorem (\ref{cor:meusnier}).

\parbf{\ref{ex:saddle-graph}.}

\parbf{\ref{ex:hat-convex}.}

\parbf{\ref{ex:intrinsic-diameter}.} Use \ref{lem:closest-point-projection}.

\parbf{\ref{ex:reflection-geodesic}.}

\parbf{\ref{ex:helix=geodesic}.}

\parbf{\ref{ex:asymptotic-geodesic}.}

\parbf{\ref{ex:geodesic-curvature}.}

\parbf{\ref{ex:two-min-geo}.}

\parbf{\ref{ex:min-geod+plane}.}

\parbf{\ref{ex:two-min-geod}.}

\parbf{\ref{ex:min-geod+plane}.}

\parbf{\ref{ex:milka}.} Show that the concatenation of the line segment $[p_t,\gamma(t)]$ and the arc $\gamma|_{[t,\ell]}$ is a shortest path in the closed region $W$ outside of $\Sigma$.

\parbf{\ref{ex:rho''}.}

\parbf{\ref{ex:usov-exact}.}

\parbf{\ref{ex:rough-bound-mountain}.} Use \ref{thm:usov} and \ref{ex:sef-intersection}.

The suggested argument does not give the optimal bound for the Lipschitz constant that guarantees that $\gamma$ is simple, but
later (see \ref{ex:sqrt(3)}) we will show that the exact bound is $\sqrt{3}=\tg\tfrac\pi3$ --- it is the same as in the exercise about mountain of with the shape of a perfect cone; see \ref{ex:lasso}.

\parbf{\ref{ex:parallel}.}

\parbf{\ref{ex:parallel-transport-support}.}
Observe $\Sigma_1$ supports $\Sigma_2$ at any point of $\gamma$.
Conclude that $\gamma$ identical spherical images in $\Sigma_1$ and $\Sigma_2$ and apply Observation \ref{obs:parallel=}.

\parbf{\ref{ex:holonomy=not0}.}

\parbf{\ref{ex:half-sphere-total-curvature}.}

\parbf{\ref{ex:1=geodesic-curvature}.}

\parbf{\ref{ex:geodesic-half}.}

\parbf{\ref{ex:aabbccdd}.} Estimate integral of Gauss curvature bounded by a simple geodesic loop and apply \ref{ex:int-gauss=4pi}.


\parbf{\ref{ex:sqrt(3)}.} Note that it is sufficient to show that the surface has no geodesic loops.
Estimate the integral of Gauss curvature of whole surface and a disc in it surrounded by a geodesic loop.

\parbf{\ref{ex:self-intersections}.}

\parbf{\ref{ex:lasso}.} Cut the lateral surface of the mountain by a line from the cowboy to the top, unfold it on the plane and try to figure out what is the image of the strained lasso.

Since the distance between points, can not be bigger than length of a path connecting them,
this statement implies the problem.


\parbf{\ref{ex:deformation}.}


%\parbf{\ref{}.}

%\parbf{\ref{}.}

%\parbf{\ref{}.}

%\parbf{\ref{}.}

%\parbf{\ref{}.}
