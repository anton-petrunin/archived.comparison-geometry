\addtocounter{chapter}{-1}
\chapter{Preliminaries}

This chapter should be used as a quick reference while reading the rest of the book;
it also contains all necessary references with complete proof.

The fist section on metric spaces is an exception;
we suggest to read in before going further.

\section{Metric spaces}\label{sec:metric-spcaes}

We assume that the reader is familiar with the notion of distance in the 
Euclidean space.
In this chapter we briefly discuss its generalization and fix notations that will be used further.

The introductory part of the book by Dmitri Burago, Yuri Burago, and Sergei Ivanov \cite{burago-burago-ivanov} contains all the needed material.

\subsection*{Definitions}

\emph{Metric} is a function that returns a real value $\Dist(x,y)$ for any pair $x,y$ in a given nonempty set $\spc{X}$  and satisfies the following axioms for any triple $x,y,z$: \label{page:def:metric}
\begin{enumerate}[(a)]
\item\label{def:metric-space:a} Positiveness: 
$$\Dist(x,y)\ge 0.$$
\item\label{def:metric-space:b} $x=y$ if and only if 
$$\Dist(x,y)=0.$$
\item\label{def:metric-space:c} Symmetry: $$\Dist(x, y) = \Dist(y, x).$$
\item\label{def:metric-space:d} Triangle inequality: 
$$\Dist(x, z) \le \Dist(x, y) + \Dist(y, z).$$
\end{enumerate}

A set with a metric is called \index{metric space}\emph{metric space} and the elements of the set are called \index{point}\emph{points}.

\subsection*{Shortcut for distance}
Usually we consider only one metric on a set, therefore we can denote the metric space and its underlying set by the same letter, say $\spc{X}$.
In this case we also use the shortcut notations $\dist{x}{y}{}$ or $\dist{x}{y}{\spc{X}}$  for the {}\emph{distance $\Dist(x,y)$ from $x$ to $y$ in $\spc{X}$}.\index{10aaa@$\dist{x}{y}{}$, $\dist{x}{y}{\spc{X}}$}
For example, the triangle inequality can be written as 
$$\dist{x}{z}{\spc{X}}\le \dist{x}{y}{\spc{X}}+\dist{y}{z}{\spc{X}}.$$

The Euclidean space and plane as well as the real line will be the most important examples of metric spaces for us.
In these examples the introduced notation $\dist{x}{y}{}$ for the distance from $x$ to $y$ has perfect sense as the norm of the vector $x-y$.
However, in a general metric space the expression $x-y$ has no meaning, but we use expression $\dist{x}{y}{}$ for the distance anyway.

\subsection*{More examples}

Usually, if we say {}\emph{plane} or {}\emph{space} we mean the {}\emph{Euclidean} plane or space.
However the plane (as well as the space) admits many other metrics, for example the so-called \index{Manhattan metric}\emph{Manhattan metric} from the following exercise.

\begin{thm}{Exercise}\label{ex:ell-infty}
Consider the function
$$\Dist(p,q)=|x_1-x_2|+|y_1-y_2|,$$
where $p=(x_1,y_1)$ and $q=(x_2,y_2)$ are points in the coordinate plane $\RR^2$.
Show that $\Dist$ is a metric on $\RR^2$.
\end{thm}

Another example: the \index{discrete space}\emph{discrete space} --- an arbitrary nonempty set $\spc{X}$ with the metric defined as $\dist{x}{y}{\spc{X}}=0$ if $x=y$ and $\dist{x}{y}{\spc{X}}=1$ otherwise.

\subsection*{Subspaces}
Any subset of a metric space is also a metric space, by restricting the original metric to the subset;
the obtained metric space is called a \index{subspace of metric space}\emph{subspace}.
In particular, all subsets of the Euclidean space are metric spaces.

\subsection*{Balls}
Given a point $p$ in a metric space $\spc{X}$ and a real number $R\ge 0$, the set of points $x$ on the distance less then $R$ (at most $R$) from $p$ is called the \index{open ball}\emph{open} (respectively \index{closed ball}\emph{closed}) \emph{ball} of radius $R$ with center $p$.
The {}\emph{open ball} is denoted as $B(p,R)$ or $B(p,R)_{\spc{X}}$;
the second notation is used if we need to emphasize that the ball lies in the metric space $\spc{X}$.
Formally speaking
\[B(p,R)=B(p,R)_{\spc{X}}=\set{x\in \spc{X}}{\dist{x}{p}{\spc{X}}< R}.\]
\index{10b@$B(p,R)_{\spc{X}}$, $\bar B[p,R]_{\spc{X}}$}
Analogously, the {}\emph{closed ball} is denoted as $\bar B[p,R]$ or $\bar B[p,R]_{\spc{X}}$ and
\[\bar B[p,R]=\bar B[p,R]_{\spc{X}}=\set{x\in \spc{X}}{\dist{x}{p}{\spc{X}}\le R}.\]

\begin{thm}{Exercise}\label{ex:B2inB1}
Let $\spc{X}$ be a metric space.

\begin{subthm}{ex:B2inB1:a}
Show that if $\bar B[p,2]\subset \bar B[q,1]$ for some points $p,q\in \spc{X}$, then $\bar B[p,2]\z=\bar B[q,1]$.
\end{subthm}

\begin{subthm}{ex:B2inB1:b} Construct a metric space $\spc{X}$ with two points $p$ and $q$ such that the strict inclusion
$B(p,\tfrac32)\subset B(q,1)$ holds.
\end{subthm}

\end{thm}



\subsection*{Continuity}

\begin{thm}{Definition}
 Let ${\spc{X}}$ be a metric space.
A sequence of points $x_1, x_2, \ldots$ in ${\spc{X}}$ is called \index{convergence of points}\emph{convergent}
if there is 
$x_\infty\in {\spc{X}}$ such that $\dist{x_\infty}{x_n}{}\to 0$ as $n\to\infty$.  
That is, for every $\eps > 0$, there is a natural number $N$ such that for all $n \ge N$, we have
\[
\dist{x_\infty}{x_n}{\spc{X}}
<
\eps.
\]

In this case we say that the sequence $(x_n)$ {}\emph{converges} to $x_\infty$, 
or $x_\infty$ is the {}\emph{limit} of the sequence $(x_n)$.
Notationally, we write $x_n\to x_\infty$ as $n\to\infty$
or $x_\infty=\lim_{n\to\infty} x_n$.
\end{thm}

\begin{thm}{Definition}\label{def:continuous}
Let $\spc{X}$ and $\spc{Y}$ be metric spaces.
A map $f\:\spc{X}\to \spc{Y}$ is called \index{continuous function}\index{continuous map}\emph{continuous} if for any convergent sequence $x_n\to x_\infty$ in ${\spc{X}}$,
%the sequence $y_n\z=f(x_n)$ converges to $y_\infty=f(x_\infty)$ in $\spc{Y}$.
we have $f(x_n) \to f(x_\infty)$ in $\spc{Y}$.

Equivalently, $f\:\spc{X}\to \spc{Y}$ is continuous if for any $x\in {\spc{X}}$ and any $\eps>0$,
there is $\delta>0$ such that 
$$\dist{x}{y}{\spc{X}}<\delta\quad \text{ implies that }\quad \dist{f(x)}{f(y)}{\spc{Y}}<\eps.$$

\end{thm}

\begin{thm}{Exercise}\label{ex:shrt=>continuous}
Let ${\spc{X}}$ and $\spc{Y}$ be metric spaces $f\:\spc{X}\to \spc{Y}$ is \index{distance non-expanding map}\emph{distance non-expanding map}; that is, 
\[\dist{f(x)}{f(y)}{\spc{Y}}\le \dist{x}{y}{\spc{X}}\]
for any $x,y\in \spc{X}$.
Show that $f$ is continuous.
\end{thm}

\subsection*{Homeomorphisms}

\begin{thm}{Definition}
Let $\spc{X}$ and $\spc{Y}$ be metric spaces.
A continuous bijection $f\:\spc{X}\to \spc{Y}$ 
is called a \index{homeomorphism}\emph{homeomorphism} 
if its inverse $f^{-1}\:\spc{Y}\z\to \spc{X}$ is also continuous.

If there exists a homeomorphism $f\:\spc{X}\to \spc{Y}$,
we say that ${\spc{X}}$ is {}\emph{homeomorphic} to $\spc{Y}$,
or  $\spc{X}$ and $\spc{Y}$ are {}\emph{homeomorphic}.
\end{thm}

If a metric space $\spc{X}$ is homeomorphic to a known space, for example plane, sphere, disc, circle and so on,
we may also say that $\spc{X}$ is a \index{topological}\emph{topological} plane, sphere, disc, circle and so on.

\subsection*{Closed and open sets}

\begin{thm}{Definition}
A subset $C$ of a metric space $\spc{X}$ is called \index{closed set}\emph{closed} if whenever a sequence $(x_n)$ of points from $C$ converges in $\spc{X}$, we have that $\lim_{n\to\infty} x_n \in C$.

A set $\Omega \subset \spc{X}$ is called \index{open set}\emph{open} if for any $z\in \Omega$, 
there is $\eps>0$ such that $B(z,\eps)\subset\Omega$.
\end{thm}

\begin{thm}{Exercise}\label{ex:close-open}
Let $Q$ be a subset of a metric space $\spc{X}$.
Show that $Q$ is closed if and only if its complement $\Omega=\spc{X}\backslash Q$ is open.
\end{thm}

An open set $\Omega$ that contains a given point $p$ is called a \index{neighborhood}\emph{neighborhood of~$p$}.
A closed subset $C$ that contains $p$ together with its neighborhood is called a {}\emph{closed neighborhood of~$p$}
