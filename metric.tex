\chapter{Metric spaces}\label{app:metric-spcaes}


We assume that the reader is familiar with the notion of metric in 
Euclidean space.
Here we introduce its generalization and fix notations that will be used further.

\section*{Definitions}

\emph{Metric} is a function that returns a real value $\Dist(x,y)$ for any pair $x,y$ in a given nonempty set $\mathcal X$  and satisfies the following axioms for any triple $x,y,z$: \label{page:def:metric}
\begin{enumerate}[(a)]
\item\label{def:metric-space:a} Positiveness: 
$$\Dist(x,y)\ge 0.$$
\item\label{def:metric-space:b} $x=y$ if and only if 
$$\Dist(x,y)=0.$$
\item\label{def:metric-space:c} Symmetry: $$\Dist(x, y) = \Dist(y, x).$$
\item\label{def:metric-space:d} Triangle inequality: 
$$\Dist(x, z) \le \Dist(x, y) + \Dist(y, z).$$
\end{enumerate}\

A set with a metric is called \emph{metric space} and the elements of the set are called \emph{points}.

\parbf{Shortcut for distance.}
Usually we consider only one metric on a set, therefore we can denote the metric space and its underlying set by the same letter, say $\mathcal X$.
In this case we also use a shortcut notations $|x-y|$ or  $|x-y|_\mathcal X$ for the \emph{distance $\Dist(x,y)$ from $x$ to $y$ in $\mathcal X$}.
For example, the triangle inequality can be written as 
$$|x-z|_{\mathcal X}\le |x-y|_{\mathcal X}+|y-z|_{\mathcal X}.$$

\parbf{Examples.}
Euclidean space and plane as well as real line will be the most important examples of metric spaces for us.
In these examples the introduced notation $|x-y|$ for the distance from $x$ to $y$ has perfect sense as a norm of the vector $x-y$.
However, in general metric space the expression $x-y$ has no sense, but anyway we use expression $|x-y|$ for the distance.

If we say \emph{plane} or \emph{space} we mean \emph{Eucledean} plane or space.
However the plane (as well as the space) admits many other metrics, for example the so-called \emph{Manhattan metric} from the following exercise.

\begin{thm}{Exercise}\label{ex:ell-infty}
Consider the function
$$\Dist(p,q)=|x_p-x_q|+|y_p-y_q|,$$
where $p=(x_p,y_p)$ and $q=(x_q,y_q)$ are points in the coordinate plane $\RR^2$.
Show that $\Dist$ is a metric on $\RR^2$.
\end{thm}

Let us mention another example: the \emph{discrete space} --- arbitrary nonempty set $\mathcal X$ with the metric defined as $|x\z-y|_{\mathcal X}=0$ if $x=y$ and $|x-y|_{\mathcal X}=1$ otherwise.

\parbf{Subspaces.}
Any subset of a metric space is also a metric space, by restricting the original metric to the subset;
the obtained metric space is called a \emph{subspace}.
In particular, all subsets of Euclidean space are metric spaces.

\parbf{Balls.}
Given a point $p$ in a metric space ${\mathcal X}$ and a real number $R\ge 0$, the set of points $x$ on the distance less then $R$ (or at most $R$) from $p$ is called open (or correspondingly closed) ball of radius $R$ with center at $p$.
The \emph{open ball} is denoted as $B(p,R)$ or $B(p,R)_{\mathcal X}$;
the second notation is used if we need to emphasize that the ball lies in the metric space $\mathcal X$.
Formally speaking
\[B(p,R)=B(p,R)_{\mathcal X}=\set{x\in \mathcal X}{|x-p|_{\mathcal X}< R}.\]
Analogously, the \emph{closed ball} is denoted as $\bar B[p,R]$ or $\bar B[p,R]_{\mathcal X}$ and
\[\bar B[p,R]=\bar B[p,R]_{\mathcal X}=\set{x\in \mathcal X}{|x-p|_{\mathcal X}\le R}.\]

\begin{thm}{Exercise}\label{ex:B2inB1}
Let $\mathcal X$ be a metric space.

\begin{subthm}{ex:B2inB1:a} Show that if $\bar B[p,2]\subset \bar B[q,1]$ for some points $p,q\in \mathcal X$, then $\bar B[p,2]= \bar B[q,1]$.
\end{subthm}

\begin{subthm}{ex:B2inB1:b} Construct a metric space $\mathcal X$ with two points $p$ and $q$ such that
$B(p,\tfrac32)\subset B(q,1)$ and the inclusions is strict.
\end{subthm}

\end{thm}



\section{Continuity}

In this section we will extend standard notions from calculus to the metric spaces.

\begin{thm}{Definition}
 Let ${\mathcal X}$ be a metric space.
A sequence of points $x_1, x_2, \ldots$ in ${\mathcal X}$ is called \emph{convergent}\index{convergent}
if there is 
$x_\infty\in {\mathcal X}$ such that $|x_\infty -x_n|\to 0$ as $n\to\infty$.  
That is, for every $\eps > 0$, there is a natural number $N$ such that for all $n \ge N$, we have
\[|x_\infty-x_n|_{\mathcal X} < \eps.\]

In this case we say that the sequence $(x_n)$ \emph{converges} to $x_\infty$, 
or $x_\infty$ is the \emph{limit} of the sequence $(x_n)$.
Notationally, we write $x_n\to x_\infty$ as $n\to\infty$
or $x_\infty=\lim_{n\to\infty} x_n$.
\end{thm}

\begin{thm}{Definition}\label{def:continous}
Let ${\mathcal X}$ and ${\mathcal Y}$ be metric spaces.
A map $f\:{\mathcal X}\to {\mathcal Y}$ is called \emph{continuous} if for any convergent sequence $x_n\to x_\infty$ in ${\mathcal X}$,
%the sequence $y_n\z=f(x_n)$ converges to $y_\infty=f(x_\infty)$ in ${\mathcal Y}$.
we have $f(x_n) \to f(x_\infty)$ in ${\mathcal Y}$.

Equivalently, $f\:{\mathcal X}\to {\mathcal Y}$ is continuous if for any $x\in {\mathcal X}$ and any $\eps>0$,
there is $\delta>0$ such that 
$$|x-x'|_{\mathcal X}<\delta\ \text{ implies }\ |f(x)-f(x')|_{\mathcal Y}<\eps.$$
%$$|x-x'|_{\mathcal X}<\delta\ \Rightarrow\ |f(x)-f(x')|_{\mathcal Y}<\eps.$$

\end{thm}

\begin{thm}{Exercise}\label{ex:shrt=>continuous}
Let ${\mathcal X}$ and ${\mathcal Y}$ be metric spaces $f\:{\mathcal X}\to {\mathcal Y}$ is \emph{distance non-expanding map}; that is, 
\[|f(x)-f(x')|_{\mathcal Y}\le |x-x'|_{\mathcal X}\]
for any $x,x'\in \mathcal X$.
Show that $f$ is continuous.
\end{thm}

\begin{thm}{Definition}
Let ${\mathcal X}$ and ${\mathcal Y}$ be metric spaces.
A continous bijection $f\:{\mathcal X}\to {\mathcal Y}$ 
is called a \emph{homeomorphism}\index{homeomorphism} 
if its inverse $f^{-1}\:{\mathcal Y}\z\to {\mathcal X}$ is also continuous.

If there exists a homeomorphism $f\:{\mathcal X}\to {\mathcal Y}$,
we say that ${\mathcal X}$ is \emph{homeomorphic}\index{homeomorphic} to ${\mathcal Y}$,
or  $\mathcal X$ and ${\mathcal Y}$ are \emph{homeomorphic}.
\end{thm}

If a metric space $\mathcal X$ is homeomorphic to a known space, for example plane, sphere, disc, circle and so on,
we may also say that $\mathcal X$ is a \emph{topological} plane, sphere, disc, circle and so on.

\begin{thm}{Definition}
A subset $A$ of a metric space $\mathcal{X}$ is called \emph{closed}\index{closed set} if whenever a sequence $(x_n)$ of points from $A$ converges in $\mathcal{X}$, we have that $\lim_{n\to\infty} x_n \in A$.

A set $\Omega \subset \mathcal{X}$ is called \emph{open}\index{open set} if for any $z\in \Omega$, 
there is $\eps>0$ such that $B(z,\eps)\subset\Omega$.
\end{thm}

An open set $\Omega$ that contains a given point $p$ is called \emph{neighborhood of $p$}.

\begin{thm}{Exercise}\label{ex:close-open}
Let $Q$ be a subset of a metric space $\mathcal{X}$.
Show that $Q$ is closed if and only if its complement $\Omega=\mathcal{X}\backslash Q$ is open.
\end{thm}
