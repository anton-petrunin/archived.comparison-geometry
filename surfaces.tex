\chapter{Surfaces}

\section{Embedded surfaces}

Recall that a function $f$ of two variables $x$ and $y$ is called \emph{smooth} if all its partial derivatives $\frac{\partial^{m+n}}{\partial x^m\partial y^n}f$ are defined and continuous in the domain of definition of $f$. 

A subset $\Sigma$ is called \emph{smooth surface} (or more precisely \emph{smooth regular embedded surface}) if it can be described locally as a graph of smooth function in appropriate coordinate systems.

More precisely, any point $p\in \Sigma$ admits a neighborhood $U$ such that
in some coordinate system $(x,y,z)$, 
the intersection $W=U\cap \Sigma$ can be written as a graph $z=f(x,y)$ of a smooth function $f$ defined in an open domain of $(x,y)$-plane.

\parbf{Examples.}
For simplest example of surface is the $(x,y)$-plane 
\[\Pi=\set{(x,y,z)\in\RR^3}{z=0}.\]
It can be described as the graph of the function $f(x,y)=0$; hence $\Pi$ is a surface.

All other planes are surfaces as well since one can choose a coordinate system so that it becomes $(x,y)$-plane or we can rewrite as a graph of linear function 
$f(x,y)=a\cdot x+b\cdot y+c$ for some constants $a$, $b$ and $c$.

A more interesting example is the unit sphere 
\[\SS^2=\set{(x,y,z)\in\RR^3}{x^2+y^2+z^2=1}.\]
This set is not a graph of any function,
but $\SS^2$
can be covered by 6 graphs 
\begin{align*}
z&=f_\pm(x,y)=\pm \sqrt{1-x^2-y^2},
\\
y&=g_\pm(x,z)=\pm \sqrt{1-x^2-z^2},
\\
x&=h_\pm(y,z)=\pm \sqrt{1-y^2-z^2};
\end{align*}
each function $f_\pm,g_\pm,h_\pm$ is defined in open unit disc.
Therefore the unit sphere is a smooth surface.

\parbf{More conventions.}
If $\Sigma$ is compact, then it is called \emph{closed surface} (the term \emph{closed set} is not directly relevant).

If $\Sigma$ is closed and noncompact, then it is called  \emph{open surface} (again the term \emph{open set} is not relevant).
For example, paraboloids $z=x^2+y^2$ or $z=x^2-y^2$ are open surfaces, while
open disc in a plane $\set{(x,y,z)\in\RR^3}{x^2+y^2<1, z=0}$ is a surface which is not an open surface since this set is not closed. 

A closed subset in a surface that is bounded by one or more smooth curves is called \emph{surface with boundary}; in this case the collection of curves is called \emph{boundary line} of the surface.

\section{Immersed surfaces}

\parbf{Parametrizations.}
A surface can be described by a map from a known surface to the space.
For example the ellipsoid
\[\Sigma_{a,b,c}=\set{(x,y,z)\in\RR^3}{\tfrac{x^2}{a^2}+\tfrac{y^2}{b^2}+\tfrac{z^2}{c^2}=1}\]
for some positive numbers $a,b,c$ can be defined as the image of a map $s\:\SS^2\to\RR^3$ that is the restriction of the map $(x,y,z)\mapsto (a\cdot x, b\cdot y,c\cdot z)$ to the unit sphere $\SS^2$.
The map $s$ has to be smooth and regular as defined below. 

Assume $\SS^2$ is written locally as a graph $z=f(x,y)$ in some coordinate system.

The map $s\:\SS^2\to \RR^3$ is  smooth if the composition $s\circ f$ has all partial derivatives $\frac{\partial^{m+n}}{\partial x^m\partial y^n}(s\circ f)$ are defined and continuous in the domain of definition of $f$.

The map $s\:\SS^2\to \RR^3$ is regular
if the vectors $\frac{\partial}{\partial x}(s\circ f)$ and $\frac{\partial}{\partial y}(s\circ f)$ are linearly independent at each point of the domain of $f$.

Evidently the parametric definition includes the embedded surfaces defined above --- as a set of parameters we can take the surface itself and the identity map as $s$.

\parbf{Immersed surfaces.} The parametric definition allows the surfaces to have self-intersections and therefore more general.
The surfaces with possible self-intersections are  can called \emph{immersed}.

In the described example $\SS^2$ is the \emph{domain of parameters} of the surface.
We can say that the surface $\Sigma_{a,b,c}$ is a \emph{sphere} since it has sphere as the domain of parameters.

We may use other domains of parameters, torus or sphere with two handles or for surfaces with boundary, disc, annulus, or M\"obius band and so on.

The set of parameters can be more complicated, for example projective plane --- sphere where opposite points are identified; such set of parameters can not be realized as an embedded surface in $\RR^3$, but it can be embedded in a higher dimensional Euclidean space.

\section{Tangent plane}

Let $z=f(x,y)$ be a local graph realization of a surface. 
Assume $p=(x_p,y_p,z_p)$ lies on this graph, so $z_p=f(x_p,y_p)$.
The plane passing thru $p$ and spanned by two vectors $(\tfrac{\partial}{\partial x}f)(x_p,y_p)$ and  $(\tfrac{\partial}{\partial y}f)(x_p,y_p)$ is called \emph{tangent plane} of $\Sigma$ at $p$.
It can be interpreted as the best approximation of the surface $\Sigma$ by a plane at $p$.

It is straightforward to check that tangent plane does not depend of the local presentation of $\Sigma$ by a graph.

\begin{thm}{Proposition}
Assume the tangent of a smooth surface $\Sigma$ at point $p$ is not perpendicular to the $(x,y)$-plane.
Then a neighborhood of $p$ in $\Sigma$ can be presented as a graph of smooth function $z=f(x,y)$ defined on an open set of the $(x,y)$-plane.
\end{thm}

The proof is an immediate application of the inverse function theorem.

\parbf{Special coordinate system.}
If one choose the coordinate system so that the origin is at $p$ and the $(x,y)$-plane coincide with the tangent plane then the function $f$ satisfies the following additional properties:
\begin{align*}
f(0,0)&=0,
&
(\tfrac{\partial}{\partial x}f)(0,0)&=0,
&
(\tfrac{\partial}{\partial y}f)(0,0)&=0.
\end{align*}
The first equality holds since $p=(0,0,0)$ lies on the surface and the last two equalities mean that the tangent plane at $p$ is horizontal.

This gives almost canonical coordinate system in a neighborhood of $p$;
it is unique up to rotation of  the $(x,y)$-plane and switching the sign of $z$-coordinate.

\section{Curvatures}

\parbf{Hessian.}
Fix a point $p$ on a smooth surface $\Sigma$ and the associated special coordinate system. 

Consider the Hessian matrix 
\[M_p=\begin{pmatrix}
   (\tfrac{\partial^2}{\partial x^2}f)(0,0)
   &(\tfrac{\partial^2}{\partial x\partial y}f)(0,0)
   \\
   (\tfrac{\partial^2}{\partial y\partial x}f)(0,0)
   &(\tfrac{\partial^2}{\partial y^2}f)(0,0)
  \end{pmatrix}.
\]
This is symmetric matrix, therefore by rotation of $(x,y)$-plane, we can make it diagonal;
that is we can assume that $(\tfrac{\partial^2}{\partial x\partial y}f)(0,0)=0$.
Then the diagonal elements are called \emph{principle curvatures} of $\Sigma$ at $p$;
they defied up to sign;
They are denoted as $k_1(p)$ and $k_2(p)$.
The principle curvatures can be also defined as the eigenvalues of $M_p$.

The determinant of $M_p$ is $k_1(p)\cdot k_2(p)$;
it is called \emph{Gauss curvature} of $\Sigma$ at $p$.
The trace of $M_p$ is $k_1(p)+ k_2(p)$;
it is called \emph{mean curvature} of $\Sigma$ at $p$.

Form the discussion above, 
we get that Gauss curvature and up to sign principle curvatures and mean curvature 
do not depend only on $\Sigma$ and $p$,
but not on the choice of the coordinate system.

\begin{thm}{Exercise}\label{ex:projection}
Assume $\Sigma$ is a closed surface of with principle curvatures at most 1
and $F$ is its orthogonal projection to the plane.
Show that no circle of curvature bigger than 1 can support $F$ from left. 
\end{thm}

\begin{thm}{Exercise}
Show that any closed immersed surface has a point with positive Gauss curvature.
\end{thm}

\begin{thm}{Exercise}
Assume a closed surface $\Sigma$ bounds a convex body.
Show that $\Sigma$ is a sphere with nonnegative Gauss curvature. 
\end{thm}












