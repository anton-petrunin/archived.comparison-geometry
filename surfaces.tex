\chapter{Surfaces}

\section{Embedded surfaces}

Recall that a function $f$ of two variables $x$ and $y$ is called smooth if all its partial derivatives $\frac{\partial^{m+n}}{\partial x^m\partial y^n}f$ are defined and continuous in the domain of definition of $f$. 

A subset $\Sigma$ is called \emph{smooth surface} (or more precisely \emph{smooth regular embedded surface}) if it can be described locally as a graph of smooth function in appropriate coordinate systems.
More precisely, any point $p\in \Sigma$ admits a neighborhood $U$ such that
in some coordinate system $(x,y,z)$, 
the intersection $W=U\cap \Sigma$ can be written as a graph $z=f(x,y)$ of a smooth function $f$ defined in an open domain of $(x,y)$-plane.

If $\Sigma$ is compact, then it is called \emph{closed surface} (the term \emph{closed set} is not directly relevant).

If $\Sigma$ is closed and noncompact, then it is called  \emph{open surface} (again the term \emph{open set} is not relevant).

A closed subset in a surface that is bounded by one or more smooth curves is called \emph{surface with boundary}; in this case the collection of curves is called \emph{boundary line} of the surface.

\parbf{Examples.}
For simplest example of surface is the $(x,y)$-plane 
\[\Pi=\set{(x,y,z)\in\RR^3}{z=0}.\]
It can be described as the graph of the function $f(x,y)=0$; hence $\Pi$ is a surface.

All other planes are surfaces as well since one can choose a coordinate system so that it becomes $(x,y)$-plane or we can rewrite as a graph of linear function 
$f(x,y)=a\cdot x+b\cdot y+c$ for some constants $a$, $b$ and $c$.

A more interesting example is the unit sphere 
\[\SS^2=\set{(x,y,z)\in\RR^3}{x^2+y^2+z^2=1}.\]
This set is not a graph of any function,
but $\SS^2$
can be covered by 6 graphs 
\begin{align*}
z&=f_\pm(x,y)=\pm \sqrt{1-x^2-y^2},
\\
y&=g_\pm(x,z)=\pm \sqrt{1-x^2-z^2},
\\
x&=h_\pm(y,z)=\pm \sqrt{1-y^2-z^2};
\end{align*}
each function $f_\pm,g_\pm,h_\pm$ is defined in open unit disc.
Therefore the unit sphere is a smooth surface.

\section{Immersed surfaces}

\parbf{Parametrizations.}
A surface can be described by a map from a known surface to the space.
For example the ellipsoid
\[\Sigma_{a,b,c}=\set{(x,y,z)\in\RR^3}{\tfrac{x^2}{a^2}+\tfrac{y^2}{b^2}+\tfrac{z^2}{c^2}=1}\]
for some positive numbers $a,b,c$ can be defined as the image of a map $s\:\SS^2\to\RR^3$ that is the restriction of the map $(x,y,z)\mapsto (a\cdot x, b\cdot y,c\cdot z)$ to the unit sphere $\SS^2$.
The map $s$ has to be smooth and regular as defined below. 

Assume $\SS^2$ is written locally as a graph $z=f(x,y)$ in some coordinate system.

The map $s\:\SS^2\to \RR^3$ is  smooth if the composition $s\circ f$ has all partial derivatives $\frac{\partial^{m+n}}{\partial x^m\partial y^n}(s\circ f)$ are defined and continuous in the domain of definition of $f$.

The map $s\:\SS^2\to \RR^3$ is regular
if the vectors $\frac{\partial}{\partial x}(s\circ f)$ and $\frac{\partial}{\partial y}(s\circ f)$ are linearly independent at each point of the domain of $f$.

Evidently the parametric definition includes the embedded surfaces defined above --- as a set of parameters we can take the surface itself and the identity map as $s$.

\parbf{Immersed surfaces.} The parametric definition allows the surfaces to have self-intersections and therefore more general.
The surfaces with possible self-intersections are  can called \emph{immersed}.

In the described example $\SS^2$ is the \emph{domain of parameters} of the surface.
We can say that the surface $\Sigma_{a,b,c}$ is a \emph{sphere} since it has sphere as the domain of parameters.

We may use other domains of parameters, torus or sphere with two handles or for surfaces with boundary, disc, annulus, or M\"obius band and so on.

The set of parameters can be more complicated, for example projective plane --- sphere where opposite points are identified; such set of parameters can not be realized as an embedded surface in $\RR^3$, but it can be embedded in a higher dimensional Euclidean space.

\section{Tangent plane}

Let $z=f(x,y)$ be a local graph realization of a surface. 
Assume $p=(x_p,y_p,z_p)$ lies on this graph, so $z_p=f(x_p,y_p)$.
The plane passing thru $p$ and spanned by two vectors $(\tfrac{\partial}{\partial x}f)(x_p,y_p)$ and  $(\tfrac{\partial}{\partial y}f)(x_p,y_p)$ is called \emph{tangent plane} of $\Sigma$ at $p$.
It can be interpreted as the best approximation of the surface $\Sigma$ by a plane at $p$.

It is straightforward to check that tangent plane does not depend of the local presentation of $\Sigma$ by a graph.

The unit normal vector to the tangent plane at $p$ usually denoted by $\nu_p$; it is uniquely defined up to sign. Likely you know that 
\[\nu_p=\frac{[(\tfrac{\partial}{\partial x}f)(x_p,y_p),(\tfrac{\partial}{\partial y}f)(x_p,y_p)]}{|[(\tfrac{\partial}{\partial x}f)(x_p,y_p),(\tfrac{\partial}{\partial y}f)(x_p,y_p)]|},\]
where $[v,w]$ denotes the vector product of vectors $v$ and $w$,
but we will not need this formula.

\parbf{Special coordinate system.}
If one choose the coordinate system so that the origin is at $p$ and the $(x,y)$-plane coincide with the tangent plane then the function $f$ satisfies the following additional properties:
\begin{align*}
f(0,0)&=0,
&
(\tfrac{\partial}{\partial x}f)(0,0)&=0,
&
(\tfrac{\partial}{\partial y}f)(0,0)&=0.
\end{align*}
The first equality holds since $p=(0,0,0)$ lies on the surface and the last two equalities mean that the tangent plane at $p$ is horizontal.

This gives almost canonical coordinate system in a neighborhood of $p$;
it is unique up to rotation of  the $(x,y)$-plane and switching the sign of $z$-coordinate.

\section{Curvatures}

\parbf{Hessian.}
Fix a point $p$ on a smooth surface $\Sigma$ and the associated special coordinate system. 

Consider the Hessian matrix 
\[M_p=\begin{pmatrix}
   (\tfrac{\partial^2}{\partial x^2}f)(0,0)
   &(\tfrac{\partial^2}{\partial x\partial y}f)(0,0)
   \\
   (\tfrac{\partial^2}{\partial y\partial x}f)(0,0)
   &(\tfrac{\partial^2}{\partial y^2}f)(0,0)
  \end{pmatrix}.
\]
This is symmetric matrix, therefore by rotation of $(x,y)$-plane, we can make it diagonal;
that is we can assume that $(\tfrac{\partial^2}{\partial x\partial y}f)(0,0)=0$.
Then the diagonal elements are called \emph{principle curvatures} of $\Sigma$ at $p$;
they defied up to sign;
They are denoted as $k_1(p)$ and $k_2(p)$.
The principle curvatures can be also defined as the eigenvalues of $M_p$.

The determinant of $M_p$ is $k_1(p)\cdot k_2(p)$;
it is called \emph{Gauss curvature} of $\Sigma$ at $p$.
The trace of $M_p$ is $k_1(p)+ k_2(p)$;
it is called \emph{mean curvature} of $\Sigma$ at $p$.

Form the discussion above, 
we get that Gauss curvature and up to sign principle curvatures and mean curvature 
do not depend only on $\Sigma$ and $p$,
but not on the choice of the coordinate system.

\section{Boundaries with bounded curvature}

Note that there sets in $\RR^3$ bounded by a closed surface $\Sigma$ with principle curvatures at most 1 by absolute value
that do not contain a ball of radius 1.

\begin{wrapfigure}{r}{33 mm}
\vskip-4mm
\centering
\includegraphics{mppics/pic-34}
\vskip0mm
\end{wrapfigure}

For example the region between two spheres with large close to each other radiuses. 
This region can be made arbitrary thin and the curvature of the boundary can be made arbitrary close to zero.

\begin{thm}{Advanced exercise}
Suppose a set $V\subset \RR^3$ is bounded by a closed surface $\Sigma$ with principle curvatures at most 1 by absolute value.
Assume that $V$ does not contain a ball of radius $\tfrac1{100}$.
Show that $\Sigma$ has two components of the same topological type; 
that is, both can be written in a parametric form with the same parameter domain. 
\end{thm}


The same example would work for the curves if we allow boundary of the plane figure to be not connected.
The following question might look like a right 3-dimensional analog of the moon in a puddle problem (\ref{thm:moon}).


\begin{thm}{Question}
Assume a set $V\subset \RR^3$ is bounded by a closed connected surface $\Sigma$ of bounded curvature.
Is it true that $V$ contains a ball of radius 1?
\end{thm}

It turns out that the answer is still ``no'', the following example was constructed by Vladimir Lagunov \cite{lagunov}.

\begin{figure}%{r}{43 mm}
%\vskip-4mm
\centering
\includegraphics{mppics/pic-33}
\vskip0mm
\end{figure}

\parit{Construction.}
Let us start with a body of revolution $V_1$ with the cross section shown on the diagram.
The boundary curve of the cross section is made by 6 vertical line segments that smoothly jointed into 3 closed simple curves. 
The boundary of $V_1$ has 3 components, each of which is a sphere.

A simple computation shows that if the curvature of all curves is at most 1 then the boundary surface of $V_1$ has priniple curvatures at most 1 by absolute value.

At most of the places $V_1$ can be made arbitrary thin,
the only thick place is where all tree spheres come together;
it could be arranged that the radius of the maximal ball just a bit above 
\[\tfrac2{\sqrt{3}}-1\approx .15\ll1.\]
This the radius of the smaller circle tangent to three unit circles that tangent to each other.


It remains to modify $V_1$ to make its boundary connected without  getting larger balls inside.

Note that each sphere in the boundary contains two flat discs;
they come into pairs close lying to each other. 
Let us drill thru two of such pairs and reconnect the holes by an other body of revolution which axis is shifted but stays parallel to the axis of $V_1$.
Denote the obtained body by $V_2$; its cross section of the obtained body is shown on the diagram. 

Then repeat the operation for the other two pairs.
Denote the obtained body by $V_3$; its cross section of the obtained body is shown on the diagram.

It is easy to see that the boundary of $V_3$ is connected
and assuming that the holes are large its boundary can be made so that its principle curvatures is still at most $1$.
\qeds

In fact the boundary of $V_3$ is a sphere with two handles.
Indeed, the boundary of $V_2$ is three spheres,
when we drill a hole, we make one hole in two spheres and two holes in one shpere.
We reconnect two spheres by a tube and obtan one sphere
and connect two holes of one sphere by a tube we get a torus.
at the second operation we make a torus from the sphere and connect it to the other torus by tube.
This way we get a sphere with two handles.

\begin{thm}{Exercise}
Assume $V$ is a body of revolution in $\RR^3$ and its boundary is a connected surface with principle curvatures at most 1 by absolute value.
Show that $V$ contains a unit ball.
\end{thm}






