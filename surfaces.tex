\chapter{Surfaces}
\section{Embedded surfaces}

Recall that a function $f$ of two variables $x$ and $y$ is called \emph{smooth} if all its partial derivatives $\frac{\partial^{m+n}}{\partial x^m\partial y^n}f$ are defined and are continuous in the domain of definition of $f$. 

A subset $\Sigma \subset \mathbb{R}^3$ is called a \emph{smooth surface} (or more precisely \emph{smooth regular embedded surface}) if it can be described locally as a graph of a smooth function in an appropriate coordinate system.

More precisely, any point $p\in \Sigma$ admits a neighborhood $U$ such that
in some coordinate system $(x,y,z)$, 
the intersection $W=U\cap \Sigma$ can be written as a graph $z=f(x,y)$ of a smooth function $f$ defined in an open domain of the $(x,y)$-plane.

Once we get a local representation of the surface by a graph, we can change it using the Proposition~\ref{prop:perp} below.

\parbf{Examples.}
The simplest example of a surface is the $(x,y)$-plane 
\[\Pi=\set{(x,y,z)\in\RR^3}{z=0}.\]
The plane $\Pi$ is a surface since
it can be described as the graph of the function $f(x,y)=0$.

All other planes are surfaces as well since one can choose a coordinate system so that it becomes the $(x,y)$-plane.
We can also present a plane as a graph of a linear function 
$f(x,y)=a\cdot x+b\cdot y+c$ for some constants $a$, $b$ and $c$
if the plane is not perpendicular to the $(x,y)$-plane.

A more interesting example is the unit sphere 
\[\SS^2=\set{(x,y,z)\in\RR^3}{x^2+y^2+z^2=1}.\]
This set is not the graph of any function,
but $\SS^2$
can be covered by 6 graphs 
\begin{align*}
z&=f_\pm(x,y)=\pm \sqrt{1-x^2-y^2},
\\
y&=g_\pm(x,z)=\pm \sqrt{1-x^2-z^2},
\\
x&=h_\pm(y,z)=\pm \sqrt{1-y^2-z^2};
\end{align*}
each function $f_\pm,g_\pm,h_\pm$ is defined in an open unit disc.
Therefore the unit sphere is a smooth surface.

\parbf{More conventions.}
If the surface $\Sigma$ is formed by a closed set, then it is called \emph{complete}.
For example, paraboloids 
\[z=x^2+y^2,\quad\quad z=x^2-y^2\]
or sphere 
\[x^2+y^2+z^2=1\]
are complete surfaces, while the
open disc in a plane 
\[\set{(x,y,z)\in\RR^3}{x^2+y^2<1, z=0}\]
is a surface which is complete.


If a complete surface $\Sigma$ is compact and connected, then it is called \emph{closed surface} (the term \emph{closed set} is not directly relevant).

If a complete surface $\Sigma$ is noncompact and connected, then it is called  \emph{open surface} (again the term \emph{open set} is not relevant).

For example, paraboloids 
are open surfaces, 
and sphere is closed.

A closed subset in a surface that is bounded by one or more smooth %???smooth???
curves is called \emph{surface with boundary}; in this case the collection of curves is called the \emph{boundary line} of the surface.
When we say \emph{surface} we usually mean a surface without boundary;
we may use the term \emph{surface with possibly nonempty boundary} if we need to talk about surfaces with and without boundary.


%???complete surface???

\section{Tangent plane}

Let $z=f(x,y)$ be a local graph realization of a surface. 
Assume $p=(x_p,y_p,z_p)$ lies on this graph, so $z_p=f(x_p,y_p)$.
The plane passing thru $p$ and spanned by the vectors $(\tfrac{\partial}{\partial x}f)(x_p,y_p)$ and  $(\tfrac{\partial}{\partial y}f)(x_p,y_p)$ is called the \emph{tangent plane} of $\Sigma$ at $p$.
It can be interpreted as the best approximation of the surface $\Sigma$ by a plane at $p$.

The tangent plane to $\Sigma$ at $p$ is usually denoted by $\T_p$ or $\T_p\Sigma$.

It is straightforward to check that the tangent plane does not depend of the local presentation of $\Sigma$ by a graph.

\parbf{On local graph representations.}
The following proposition guarantees the existence of a local graph representation near a given point.

\begin{thm}{Proposition}\label{prop:perp}
Assume the tangent of a smooth surface $\Sigma$ at point $p$ is not perpendicular to the $(x,y)$-plane.
Then a neighborhood of $p$ in $\Sigma$ can be presented as a graph of smooth function $z=f(x,y)$ defined on an open set of the $(x,y)$-plane.
\end{thm}

A reader familiar with the inverse function theorem, can consider this proposition as an exercise.

\parbf{Special coordinate system.} %???tangent-normal coordinates
Fix a point $p$ in a smooth surface $\Sigma$.
Consider a coordinate system $(x,y,z)$ with origin at $p$ such that the $(x,y)$-plane coincides with $\T_p$.

According to Proposition~\ref{prop:perp}, 
we can present $\Sigma$ locally around $p$ as a graph of a function $f$.
Note that $f$ satisfies the following additional properties:
\begin{align*}
f(0,0)&=0,
&
(\tfrac{\partial}{\partial x}f)(0,0)&=0,
&
(\tfrac{\partial}{\partial y}f)(0,0)&=0.
\end{align*}
The first equality holds since $p=(0,0,0)$ lies on the graph and the last two equalities mean that the tangent plane at $p$ is horizontal.

This gives an almost canonical coordinate system in a neighborhood of $p$;
it is unique up to a rotation of  the $(x,y)$-plane and switching the sign of the $z$-coordinate.

\section{Curvatures}

\parbf{Hessian.}
Fix a point $p$ on a smooth surface $\Sigma$ and the associated special coordinate system. 

Consider the Hessian matrix 
\[M_p=\begin{pmatrix}
   (\tfrac{\partial^2}{\partial x^2}f)(0,0)
   &(\tfrac{\partial^2}{\partial x\partial y}f)(0,0)
   \\
   (\tfrac{\partial^2}{\partial y\partial x}f)(0,0)
   &(\tfrac{\partial^2}{\partial y^2}f)(0,0)
  \end{pmatrix}.
\]
This is a symmetric matrix, therefore by an appropriate rotation of the $(x,y)$-plane, we can make it diagonal;
that is, we can assume that $(\tfrac{\partial^2}{\partial x\partial y}f)(0,0)=0$.
Then the diagonal elements are called \emph{principle curvatures} of $\Sigma$ at $p$;
they are uniqueley defined up to sign;
They are denoted as $k_1(p)$ and $k_2(p)$.
The principle curvatures can be also defined as the eigenvalues of $M_p$.

The determinant of $M_p$ is $k_1(p)\cdot k_2(p)$;
it is called the \emph{Gauss curvature} of $\Sigma$ at $p$.
The trace of $M_p$ is $k_1(p)+ k_2(p)$;
it is called the \emph{mean curvature} of $\Sigma$ at $p$.

Form the discussion above, 
we get that the Gauss curvature depends only on $\Sigma$ and $p$,
and not on the choice of the coordinate system.
Up to sign, the same obsevation is true for the principle curvatures and the mean curvature. 


\begin{thm}{Exercise}\label{ex:projection}
Assume $\Sigma$ is a closed surface of with principle curvatures at most 1
and let $F$ be its orthogonal projection to a plane.
Show that no circle of curvature larger than 1 can support $F$ from the left. 
\end{thm}

\begin{thm}{Exercise}
Show that any closed immersed surface has a point with positive Gauss curvature.
\end{thm}

\begin{thm}{Exercise}
Assume a closed surface $\Sigma$ bounds a convex body.
Show that $\Sigma$ is a sphere with nonnegative Gauss curvature. 
\end{thm}

\section{Immersed surfaces}

\parbf{Parametrizations.}
A surface can be described by a map from a known surface to the space.
For example the ellipsoid
\[\Sigma_{a,b,c}=\set{(x,y,z)\in\RR^3}{\tfrac{x^2}{a^2}+\tfrac{y^2}{b^2}+\tfrac{z^2}{c^2}=1}\]
for some positive numbers $a,b,c$ can be defined as the image of the map $s\:\SS^2\to\RR^3$, defined as the restriction of the map $(x,y,z)\mapsto (a\cdot x, b\cdot y,c\cdot z)$ to the unit sphere $\SS^2$.

For a surface $\Sigma$, a map $s: \Sigma \to \RR ^3$ is called a 
\emph{parametrized surface} if it is smooth and regular in the sense defined below. 
$\Sigma$ will be called the \emph{domain of parameters} and $s$ the \emph{parametrization}.

Assume $\Sigma$ is written locally as a graph $z=f(x,y)$ in some coordinate system. Let $\tilde{f} $ be defined on the same domain as $f$ as $\tilde{f}(x,y) = (x,y, f(x,y))$.

The map $s\:\Sigma \to \RR^3$ is  smooth if for the composition $s\circ \tilde{f}$, all partial derivatives $\frac{\partial^{m+n}}{\partial x^m\partial y^n}(s\circ \tilde{f})$ exist and are continuous in the domain of definition of $f$.

The map $s\:\Sigma \to \RR^3$ is regular
if the vectors $\frac{\partial}{\partial x}(s\circ \tilde{f})$ and $\frac{\partial}{\partial y}(s\circ \tilde{f})$ are linearly independent at each point of the domain of $f$.

Evidently the parametric definition includes the embedded surfaces defined previously --- as the domain of parameters we can take the surface itself and the identity map as $s$.

\parbf{Immersed surfaces.} The parametric definition allows the surfaces to have self-intersections, hence it is more general.
The surfaces with possible self-intersections are  called \emph{immersed}.

In the described example $\SS^2$ is the \emph{domain of parameters} of the surface.
We can say that the surface $\Sigma_{a,b,c}$ is a \emph{sphere} since it has the sphere as the domain of parameters.

We may use other domains of parameters, the torus or the sphere with two handles or for surfaces with boundary, the disc, the annulus, the M\"obius band and so on.
The sphere with $n$ handles is also called the \emph{surface of genus $n$}.

The set of parameters can be more complicated, for example the projective plane --- a sphere where opposite points are identified; such set of parameters can not be realized as an embedded surface in $\RR^3$, but it can be embedded in a higher dimensional Euclidean space.
Another example is the Klein bottle --- the  nonoriented brother of the torus;
it also can not be embedded in the Euclidean space, but it can be immersed with a self-intersection along a closed smooth curve.










