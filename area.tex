\chapter{(Area)}

\section{Flux and area}

Let $\vec u$ be a vector field defined on a smooth oriented surface $\Sigma$ with unit normal filed $\Norm$.
Recall that \index{flux}\emph{flux} of $\vec u$ thru $\Sigma$ is defined by the integral 
\[\flux_{\vec u}\Sigma=\iint_\Sigma \langle \vec u,\Norm\rangle.\]\index{$\flux_{\vec u}\Sigma$}

\begin{thm}{Observation}\label{obs:flux}
Let $\vec u$ be a vector field defined on a smooth oriented surface $\Sigma$.
Assume $|\vec u|\le 1$ at every point.
Then 
\[\flux_{\vec u}\Sigma \le \area\Sigma.\]
\end{thm}

\parit{Proof.}
Since $|\vec u|\le 1$ and $|\Norm|=1$ at every point,
we have that
\[\langle\vec u,\Norm\rangle\le 1\]
Therefore 
\[\flux_{\vec u}\Sigma=\iint_\Sigma \langle\vec u,\Norm\rangle\le \iint_\Sigma1=\area\Sigma.\]
\qedsf
\section{Divergence and curl}


Consider a smooth vector field $\vec u$ defined on a domain $\Omega$ in $\RR^3$.
Recall that divergence of $\vec u$ is defined as $\div\vec u=\langle \nabla,\vec u\rangle$.
In other words, if $\vec i$, $\vec j$, and $\vec k$ denote the elements of the standard basis of  $\RR^3$ and
$\vec u=P\cdot \vec i+Q\cdot\vec j+R\cdot\vec k$
for some smooth functions $P,Q,R$,
then
\[\div\vec u= P_x+Q_y+R_z.\]\index{$\div\vec u$}
Divergence is a function on $\Omega$;
its value at a point $p$ is denoted by $\div_p\vec u$.

The following two theorems are assumed to be known:

\begin{thm}{Divergence theorem}\label{thm:div}
If a piecewise smooth surface $\Sigma$ bounds a body $V$ in $\RR^3$, then
\[\flux_{\vec u}\Sigma=\iiint_{p\in V}\div_p \vec u,\]
assuming that the orientation on $\Sigma$ is defined by a unit normal field that points out of $V$.
\end{thm}

Given a vector field $\vec u$, set $\curl \vec u\df\nabla\times \vec u$;
that is, if we write the field $\vec u$ in the standard basis 
\[\vec u=P\cdot \vec i+ Q\cdot\vec j+R\cdot \vec j,\]
then 
\[\curl \vec u
\df
\left(R_y-Q_z \right)\cdot \vec i 
+
\left(P_z-R_x\right) \cdot \vec j
+
\left (Q_x-P_y\right)\cdot \vec k \]\index{$\curl \vec u$}

\begin{thm}{Curl theorem}\label{thm:curl}
Let $\Sigma$ be a compact oriented surface bounded by a curve $\gamma$.
Assume that $\gamma\:[0,\ell]\to \RR^3$ is parameterized by its arc-length and oriented in such a way that $\Sigma$ lies on left from it.
Then, for any smooth vector field $\vec u$ defined in a neighborhood of $\Sigma$, we have
\[\flux_{\curl\vec u}\Sigma
=
\int_0^\ell\langle\vec u,\gamma'(t)\rangle\cdot dt.\]

\end{thm}


\begin{thm}{Exercise}\label{ex:divergence-1}
Let $\Sigma$ be a compact smooth surface with boundary $\partial \Sigma$ in the $(x,y)$-plane.
Denote by $\Delta$ the compact region in the $(x,y)$-plane bounded by the $\partial \Sigma$.
Suppose that $\vec i,\vec j,\vec k$ is the standard basis in the space.

\begin{subthm}{ex:divergence}
Assume that all the points of $\Sigma$ lie in the upper half-space of the $(x,y)$-plane.
Use the divergence theorem (\ref{thm:div}) and the observation for the constant vector field $\vec k$
to show that 
\[\area\Sigma\ge\area\Delta.\]
\end{subthm}

\begin{subthm}{ex:curl} 
Use the curl theorem (\ref{thm:curl}) to prove the inequality in \ref{ex:divergence} for arbitrary smooth surface $\Sigma$ with boundary $\partial \Sigma$ in the $(x,y)$-plane, without assuming that the remaining points lie in the upper half-space.
\end{subthm}

\end{thm}

Further we denote by $\vec i,\vec j,\vec k$ the standard basis in $\RR^3$.

\begin{thm}{Exercise}\label{ex:divergence-2}
Let $\Sigma$ be a smooth closed surface that lies between parallel planes on distance $1$ from the $(x,y)$-plane.
Suppose that $\Sigma$ bounds a region $R$.
Use the divergence theorem and the observation for the vector field $\vec u=z\cdot \vec k$ to  show that 
\[\area \Sigma>\vol R.\]
\end{thm}

\begin{thm}{Exercise}
Let $\Sigma$ be a smooth closed surface that lies inside the infinite cylinder $x^2+y^2\le 1$.
Suppose that $\Sigma$ bounds a region $R$.
Use the divergence theorem and the observation for the vector field $\vec i,\vec j,\vec k$ to show that  
\[\area \Sigma>2\cdot\vol R.\]
\end{thm}

\begin{thm}{Corollary}
Let $\Sigma$ and $\Sigma'$ be a compact surfaces  with identical boundary lines;
that is, $\partial \Sigma=\partial \Sigma'$.
Suppose that $\Sigma$ is oriented and its unit normal field $\Norm$
can be extended to a vector field $\vec u$ such that $|\vec u|= 1$ at every point
and $\div \vec u= 0$ at every point between $\Sigma$ and $\Sigma'$. 
Then 
\[\area\Sigma'\ge \area\Sigma.\]
\end{thm}

The vector field $\vec u$ as in the corollary is called \index{calibration field}\emph{calibration} of $\Sigma$

\section{Mean curvature and divergence}

The following lemma will be used to construct unit vector fields with vanishing or positive divergence.

\begin{thm}{Lemma}\label{lem:div+H}
Let $\Sigma$ be a smooth orented surface, $\Norm$ be the unit normal field on $\Sigma$.
Suppose that a unit vector field $\vec u$ is a smooth extension of $\Norm$ in a neighborhood of $\Sigma$.
Then at any point $p\in\Sigma$,
\[\div_p \vec u+H(p)=0,\]
where $H(p)$ is the mean curvature of $\Sigma$ at $p$.
\end{thm}

\parit{Proof.}
Consider the tangent-normal coordinates $(x,y,z)$ of $\Sigma$ at $p$ such that the $x$-  and $y$-coordinates point in the principal directions of $\Sigma$.
If $\vec i,\vec j,\vec k$ denotes the standard basis, then 
\begin{align*}
S_p(\vec i)&=k_1\cdot \vec i,
&
S_p(\vec j)&=k_2\cdot \vec j,
\end{align*}
where $\Shape_p$ stands for the shape operator of $\Sigma$ at $p$.

Applying the definition of shape operator implies that
\[
\vec u_x
=
\Norm(x)
=
-\Shape_p(\vec i),
\quad\text{and}\quad
\vec u_y
=
\Norm(y)
=
-\Shape_p(\vec j).
\]
It follows that
\[
\langle \vec u_x,\vec i\rangle =-k_1, 
\quad\text{and}\quad
\langle \vec u_y,\vec j\rangle=-k_2. \eqlbl{eq:divxy}
\]


Further, since $\vec u$ is unit, we have $\langle\vec u,\vec u\rangle\equiv 1$.
Taking derivative of this identity, we get
\begin{align*}
0&=\tfrac{\partial}{\partial z} \langle\vec u,\vec u\rangle=
\\
&=2\cdot \langle \vec u_z,\vec u\rangle.
\end{align*}
In particular,
\[\langle \vec u_z,\vec k\rangle_p=0.\eqlbl{eq:divz}\]

Let us write the field $\vec u$ in the standard basis 
$\vec u=P\cdot \vec i+Q\cdot\vec j+R\cdot\vec k$.
Since the basis $\vec i,\vec j,\vec k$ is orthonormal, the functions $P,Q,R$ can be defined by
\begin{align*}
P&=\langle\vec u,\vec i\rangle,
&
Q&=\langle\vec u,\vec j\rangle,
&
R&=\langle\vec u,\vec k\rangle.
\end{align*}
Applying the definition of divergence and using \ref{eq:divxy} and \ref{eq:divz}, we obtain
\begin{align*}
\div\vec u&=P_x+Q_y+R_z=
\\
&=\tfrac{\partial}{\partial x}\langle\vec u,\vec i \rangle
+
\tfrac{\partial}{\partial y}\langle\vec u,\vec j \rangle
+
\tfrac{\partial}{\partial z}\langle\vec u,\vec k\rangle
=
\\
&=\langle\vec u_x,\vec i \rangle
+
\langle \vec u_y,\vec j \rangle
+
\langle\vec u_z,\vec k \rangle
\\
&=-k_1(p)-k_2(p)+0
\\
&=-H(p)
\end{align*}
\qedsf

Let $V$ be a body in $\RR^3$ bounded by a closed smooth surface $\Sigma$;
assume $\Sigma$ is equipped with orientation defined by unit normal field $\Norm$ that points outside $V$.
We say that $V$ is \index{mean-convex body}\emph{mean-convex} if the mean curvature of $\Sigma$ is nonpositive.

\begin{thm}{Exercise}\label{ex:mean-convex}
Let $V$ be a mean-convex body in $\RR^3$ bounded by a closed smooth surface $\Sigma$.
Denote by $W$ the outer region of $\Sigma$, it is a complement of the interior of $V$.

\begin{subthm}{ex:mean-convex:u}
Suppose $V$ is star-shaped; that is, there is a point $p\in V$ such that for any other point $x\in V$ the line segment $[p,x]$ lies in $V$.

Construct a unit vector field $\vec u$ on $W$ such that $\div  \vec u\ge 0$ at every point in $W$ and the restriction of $\vec u$ to $\Sigma$ is a normal field that points in $W$.
\end{subthm}
 
\begin{subthm}{ex:mean-convex:area} Suppose that another body $V'$ is bounded by a closed smooth surface $\Sigma'$ and $V'\supset V$.
Use part \ref{SHORT.ex:mean-convex:u} and the divergence theorem to show that 
\[\area\Sigma'\ge \area\Sigma\]
if $V$ is star-shaped.
\end{subthm}

\begin{subthm}{ex:mean-convex:wrong} Construct a non-star-shaped mean-convex body $V$ bounded by a smooth surface such that the inequality in\ref{ex:mean-convex:u} does not hold for some body $V'\supset V$ with smooth boundary $\Sigma'$.
\end{subthm}
\end{thm}

\section{Area-minimizing surfaces}

A smooth surface $\Sigma$ is called \index{area-minimizing surface}\emph{area-minimizing} if for any compact surfaces $\Delta$ in $\Sigma$ with boundary line $\partial \Delta$ 
the following inequality 
\[\area \Delta\le \area \Delta'\]
holds for any other smooth compact surface $\Delta'$ with the same boundary line; that is, if $\partial\Delta'\z=\partial\Delta$.

\begin{thm}{Exercise}\label{ex:area-ball-intersection}
Suppose $\Sigma$ is a compact area-minimizing surface with boundary line $\partial \Sigma$.
Let $p$ be a point in $\Sigma$.
Show that if the ball $B(p,r)$ does not intersect $\partial \Sigma$, then 
\[\area [\Sigma\cap B(p,r)]\le 2\cdot \pi\cdot r^2;\]
that is, this area cannot exceed half of the area of sphere of radius $r$.
\end{thm}


Recall that a surface with vanishing mean curvature is called {}\emph{minimal}.


\begin{thm}{Proposition}\label{prop:minimizing-is-minimal}
Any area-minimizing surface is minimal.
\end{thm}



\parit{Proof.}
Assume $\Sigma$ is a surface with nonzero mean curvature at some point $p$.
Without loss of generality we may assume that it is positive,
otherwise switch the orientation.

Let $z=f(x,y)$ be a local description of $\Sigma$ in the tangent-normal coordinates at $p$.
Denote by $H(x,y)$ and $\Norm(x,y)$ the mean curvature and the unit normal vector of the graph at the point $(x,y,f(x,y))$.
Passing to a smaller domain of $f$, we can assume that 
\[H(x,y)>0\] for any $(x,y)\in\Dom f$.

Consider the vector field $\vec u$ on the domain $\Omega=\RR\times\Dom f$ defined by 
$\vec u(x,y,z)=\Norm (x,y)$.
Note that 
\[(\div \vec u)(x,y,z)+H(x,y)=0.
\eqlbl{div<0}\]
Indeed, by construction, $\vec u$ is invariant with respect to shifts of $\Omega$ up or down.
In particular the divergence $\div \vec u$ does only on $x$ and~$y$.
Therefore it is sufficient to show \ref{div<0} for points on the graph $z=f(x,y)$.
The latter follows from Lemma~\ref{lem:div+H}.

Fix a closed $\epsilon$-neighborhood $D_\epsilon$ of the origin in the $(x,y)$-plane;
we can assume that $D_\epsilon$ lies in the domain of $f$.
Choose a smooth function $(x,y)\mapsto h(x,y)$ defined on $D_\epsilon$ in its interior and vanishes on its boundary;
for example, $h=\epsilon^2-x^2-y^2$ will do.
Set
\[f_t(x,y)=f(x,y)+t\cdot h(x,y).\]
Denote by $\Delta_t$ be the graph $z=f_t(x,y)$.
Set $a(t)=\area\Delta_t$ and $b(t)\z=\flux_{\vec u} \Delta_t$.
Observe that both functions $t\mapsto a(t)$ and $t\mapsto b(t)$ are smooth.

By construction of $\vec u$, we have that $a(0)=b(0)$.
By Observation \ref{obs:flux}, we have that $a(t)\ge b(t)$ for any $t$.
Therefore
\[a'(0)=b'(0).\eqlbl{eq:a'=b'}\]

Fix $t>0$.
Let $\Theta_t$ be the domain squeezed between $\Delta$ and $\Delta_t$;
that is, 
\[\Theta_t=\set{(x,y,z)}{f(x,y)< z< f_t(x,y)}.\]

By divergence theorem, we have 
\begin{align*}
b(t)-b(0)&=\iiint_{\Theta_t}\div \vec u\cdot dx\cdot dy \cdot dz=
\\
&=-\iiint_{\Theta_t} H(x,y)\cdot dx\cdot dy \cdot dz=
\\
&=t\cdot\left[- \iint_\Delta H(x,y)\cdot h(x,y)\cdot dx\cdot dy\right].
\end{align*}
Since $H>0$ and $h>0$ in the interior of $\Delta$,
the last integral is positive.
It follows that $b(t)$ is a linear function with negative slope.
By \ref{eq:a'=b'}, $a'(0)=b'(0)< 0$.
In particular, 
\[\area\Delta_t<\area \Delta\] for small $t>0$;
that is, $\Sigma$ is not area-minimizing.
\qeds

The following two exercises show that minimal surface might be not area-minimizing.
Recall that catenoid and helicoid are minimal; see exercises \ref{ex:catenoid-is-minimal} and \ref{ex:helicoid-is-minimal}.
The following exercise state that sufficienlty large piece of these surfaces are not area-minimizing.

\begin{thm}{Exercise}\label{ex:catenoid-nonmin}
Show that the \index{catenoid}\emph{catenoid}
\[(\cosh z)^2=x^2+y^2.\]
is not area-minimizing.
\end{thm}

\begin{thm}{Exercise}\label{ex:helicoid-nonmin}
Show that the \index{helicoid}\emph{helicoid} 
\[s(u,v)=(u\cdot \sin v,u\cdot \cos v,v).\]
 is not a area-minimizing.
\end{thm}

The following theorem provides a condition on minimal surface that guarantees that it is area-minimizing.

\begin{thm}{Theorem}
Let $\Sigma$ be a graph $z=f(x,y)$ of a smooth function $f$ defined on an open convex set in the $(x,y)$-plane.
Suppose $\Sigma$ is minimal, then it is area-minimizing.

In particular, any minimal surface $\Sigma$ is \index{locally area-minimizing surface}\emph{locally area-minimizing};
that is, some neighborhood of any point $p$ in $\Sigma$ is area-minimizing.
\end{thm}

We omit its proof altho it is not hard;
it can be build on the ideas from the solutions of \ref{ex:mean-convex} and \ref{prop:minimizing-is-minimal}.

%%%%%%%%%%%%%%%%%%%%%%



\parbf{\ref{ex:divergence-1}}; \ref{SHORT.ex:divergence}.
Since $\div\vec k=0$, applying the divergence theorem to the domain bounded by $\Delta$ and $\Sigma$, we get that 
\[\flux_{\vec k}\Sigma=\flux_{\vec k}\Delta.\]
Since $|\vec k|=1$, we have 
\[\flux_{\vec k}\Sigma\le \area\Sigma.\]
It remains to observe that
\[\flux_{\vec k}\Delta=\area\Delta.\]


\parit{\ref{SHORT.ex:curl}.} Consider the field $\vec u=x\cdot\vec j$.
Observe that $\vec k=\curl \vec u$.
Therefore by the curl theorem (\ref{thm:curl}) we have 
\[\flux_{\vec k}\Sigma
=
\int_0^\ell\langle\vec u,\gamma'(t)\rangle\cdot dt
=
\flux_{\vec k}\Delta,\]
where $\gamma\:[0,\ell]\to\RR^3$ is the common boundary of $\Sigma$ and $\Delta$ parameterized by its arc-length and with the right choice of orientation.

The same argument as in \ref{SHORT.ex:divergence} finishes the proof.



\parbf{\ref{ex:divergence-2}.}
Observe that $\div\vec u=1$.
Applying the divergence theorem and the observation (\ref{obs:flux}) we get
\begin{align*}
\vol R&=\iiint_R\div\vec u\cdot dx\cdot dy\cdot dz
=
\flux_{\vec u}\Sigma
\le
\area\Sigma.
\end{align*}




\parbf{\ref{ex:mean-convex}}; \ref{SHORT.ex:mean-convex:u}.
We may assume that $p$ is the origin of $\RR^3$.
Let us extend the outer normal field $\Norm$ to $\Sigma$ to a field $\vec u$ defined in $\RR^3\backslash\{0\}$ by
\[\vec u(\lambda\cdot q)=\Norm(q)\]
for any $\lambda>0$ and $q\in\Sigma$.

Observe that $\vec u(\lambda\cdot q)$ is normal to the surface 
\[\lambda\cdot\Sigma=\set{\lambda\cdot q}{q\in\Sigma}.\]
By \ref{lem:div+H}, we have
\[\div \vec u=-H(\lambda\cdot q)_{\lambda\cdot\Sigma}\ge 0.\]

\parit{\ref{SHORT.ex:mean-convex:area}.} Apply the divergence theorem for the region squeezed between $\Sigma$ and $\Sigma'$.

\parit{\ref{SHORT.ex:mean-convex:wrong}.}
An example they can be found among bodies of revolution as shown in the picture.
The gutter in the middle can be chosen to be taken to be a catenoid which is a minimal surface; see \ref{ex:catenoid-is-minimal}.
\begin{figure}[h!]
\vskip-0mm
\centering
\includegraphics{mppics/pic-300}
\vskip0mm
\end{figure}
Show that if the gutter is sufficiently deep, then the surface of revolution of the dashed line can be made smaller.

\parbf{\ref{ex:area-ball-intersection}.}
By \ref{lem:reg-section}, varying $r$ slightly, we can assume that $\Sigma$ intersects the sphere $S(p,r)$ along a finite collection of closed curves.

These curves bound a domain $D$ with area at most $2\cdot \pi\cdot r^2$.
Assume $D$ is connected.
Since $\Sigma$ is area minimizing the area of $\Delta$ cannot be smaller than the area of $\Sigma\cap B(p,r)$;
whence the result.

If $D$ is not connected, then
it can be made connected by attaching tiny tubes between its connected components.
The extra area of these tubes can be made smaller than any given $\epsilon>0$.
Therefore we get 
\[\area[\Sigma\cap B(p,r)]\le 2\cdot \pi\cdot r^2+\epsilon.\]
The statement follows since $\epsilon>0$ is arbitrary.




\parbf{\ref{ex:catenoid-nonmin}.}
Consider the region of catenoid in the cylinder $x^2+y^2\le R^2$ for $R>1$.
It is bounded by two circles defined by the equations 
\[
\begin{cases}
x^2+y^2\le R^2,
\\
z=\pm r
\end{cases}
\]
where $\cosh r=R$.

Show that if $R$ is large, the area of this region is at least $\pi\cdot R^2$ --- the area of disc of radius $R$.

Observe that the lateral surface of cylinder bounded by the two circles is $4\cdot \pi\cdot R\cdot r$.
Since $r/R\to 0$ as $R\to \infty$, we get that this surface has smaller area then region of catenoid.


\parbf{\ref{ex:helicoid-nonmin}.}
Show that for large $R$, the area of the helicoid in the cylinder defined by $|z|\le R$ and $x^2+y^2\le R^2$ exceeds the area of the surface of cylinder.
Make a conclusion.
