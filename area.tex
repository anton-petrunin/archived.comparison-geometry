\chapter{Area}

\section*{Flux and area}

Let $\vec u$ be a vector field defined on a smooth oriented surface $\Sigma$ with unit normal filed $\Norm$.
Recall that \emph{flux} of $\vec u$ thru $\Sigma$ is defined by the integral 
\[\flux_{\vec u}\Sigma=\int_\Sigma \langle \vec u,\Norm\rangle.\]

\begin{thm}{Observation}\label{obs:flux}
Let $\vec u$ be a vector field defined on a smooth oriented surface $\Sigma$.
Assume $|\vec u|\le 1$ at every point.
Then 
\[\flux_{\vec u}\Sigma \le \area\Sigma.\]
\end{thm}

\parit{Proof.}
Since $|\vec u|\le 1$ and $|\Norm|=1$ at every point,
we have that
\[\langle\vec u,\Norm\rangle\le 1\]
Therefore 
\[\flux_{\vec u}\Sigma=\int_\Sigma \langle\vec u,\Norm\rangle\le \int_\Sigma1=\area\Sigma.\]
\qedsf
\section*{Divergence and curl}


Consider a smooth vector field $\vec u$ defined on a domain $\Omega$ in $\RR^3$.
Recall that divergence of $\vec u$ is defined as $\div\vec u=\langle \nabla,\vec u\rangle$.
In other words, if $\vec i$, $\vec j$, and $\vec k$ denote the elements of the standard basis of  $\RR^3$ and
$\vec u=P\cdot \vec i+Q\cdot\vec j+R\cdot\vec k$
for some smooth functions $P,Q,R$,
then
\[\div\vec u=\tfrac{\partial P}{\partial x}+\tfrac{\partial Q}{\partial y}+\tfrac{\partial R}{\partial z}.\]
Divergence is a function on $\Omega$;
its value at a point $p$ is denoted by $\div_p\vec u$.

The following two theorems are assumed to be known:

\begin{thm}{Divergence theorem}\label{thm:div}
If a piecewise smooth surface $\Sigma$ bounds a body $V$ in $\RR^3$, then
\[\flux_{\vec u}\Sigma=\int_{p\in V}\div_p \vec u,\]
assuming that the orientation on $\Sigma$ is defined by a unit normal field that points out of $V$.
\end{thm}

Given a vector field $\vec u$, set $\curl \vec u\df\nabla\times \vec u$;
that is, if we write the field $\vec u$ in the standard basis 
\[\vec u=P\cdot \vec i+ Q\cdot\vec j+R\cdot \vec j,\]
then 
\[\curl \vec u\df \left(\tfrac{\partial R}{\partial y}-\tfrac{\partial Q}{\partial z} \right)\cdot \vec i 
+\left(\tfrac{\partial P}{\partial z}-\tfrac{\partial R}{\partial x}\right) \cdot \vec j +\left (\tfrac{\partial Q}{\partial x}-\tfrac{\partial P}{\partial y}\right)\cdot \vec k \]

\begin{thm}{Curl theorem}\label{thm:curl}
Let $\Sigma$ be a compact oriented surface bounded by a curve $\gamma$.
Assume that $\gamma\:[0,\ell]\to \RR^3$ is parameterized by its arc-length and oriented in such a way that $\Sigma$ lies on left from it.
Then, for any smooth vector field $\vec u$ defined in a neighborhood of $\Sigma$, we have
\[\flux_{\curl\vec u}\Sigma=\int_0^\ell\langle\vec u,\gamma'(t)\rangle\cdot dt.\]

\end{thm}


\begin{thm}{Exercise}\label{ex:divergence-1}
Let $\Sigma$ be a compact smooth surface with boundary $\partial \Sigma$ in the $(x,y)$-plane.
Denote by $\Delta$ the compact region in the $(x,y)$-plane bounded by the $\partial \Sigma$.
Suppose that $\vec i,\vec j,\vec k$ is the standard basis in the the space.

\begin{subthm}{ex:divergence}
Assume that all the points of $\Sigma$ lie in the upper half-space of the $(x,y)$-plane.
Use the divergence theorem (\ref{thm:div}) and the observation for the constant vector field $\vec k$
to show that 
\[\area\Sigma\ge\area\Delta.\]
\end{subthm}

\begin{subthm}{ex:curl} 
Use the curl theorem (\ref{thm:curl}) to prove the inequality in \ref{ex:divergence} for arbitrary smooth surface $\Sigma$ with boundary $\partial \Sigma$ in the $(x,y)$-plane, without assuming that the remaining points lie in the upper half-space.
\end{subthm}

\end{thm}

Further we denote by $\vec i,\vec j,\vec k$ the standard basis in a $(x,y,z)$-coordinate system of $\RR^3$.

\begin{thm}{Exercise}\label{ex:divergence-2}
Let $\Sigma$ be a smooth closed surface that lies between parallel planes on distance $1$ from the $(x,y)$-plane.
Suppose that $\Sigma$ bounds a region $R$.
Use the divergence theorem and the observation for the vector field $\vec u=z\cdot \vec k$ to  show that 
\[\area \Sigma>\vol R.\]
\end{thm}

\begin{thm}{Exercise}
Let $\Sigma$ be a smooth closed surface that lies inside the infinite cylinder $x^2+y^2\le 1$.
Suppose that $\Sigma$ bounds a region $R$.
Use the divergence theorem and the observation for the vector field $\vec i,\vec j,\vec k$ to show that  
\[\area \Sigma>2\cdot\vol R.\]
\end{thm}

\begin{thm}{Corollary}
Let $\Sigma$ and $\Sigma'$ be a compact surfaces  with identical boundary lines;
that is, $\partial \Sigma=\partial \Sigma'$.
Suppose that $\Sigma$ is oriented and its unit normal field $\Norm$
can be extended to a vector field $\vec u$ such that $|\vec u|= 1$ at every point
and $\div \vec u= 0$ at every point between $\Sigma$ and $\Sigma'$. %???
Then 
\[\area\Sigma'\ge \area\Sigma.\]
\end{thm} %???is it needed???

The vector field $\vec u$ as in the corollary is called \emph{calibration} of $\Sigma$

\section*{Mean curvature and divergence}

The following lemma will be used to construct unit vector fields with vanishing or positive divergence.

\begin{thm}{Lemma}\label{lem:div+H}
Let $\Sigma$ be a smooth orented surface, $\Norm$ be the unit normal field on $\Sigma$.
Suppose that a unit vector field $\vec u$ is a smooth extension of $\Norm$ in a neighborhood of $\Sigma$.
Then at any point $p\in\Sigma$,
\[\div_p \vec u+H(p)=0,\]
where $H(p)$ is the mean curvature of $\Sigma$ at $p$.
\end{thm}

\parit{Proof.}
Consider the tangent-normal coordinates $(x,y,z)$ of $\Sigma$ at $p$ such that the $x$-  and $y$-coordinates point in the principle directions of $\Sigma$.
If $\vec i,\vec j,\vec k$ denotes the standard basis, then 
\begin{align*}
S_p(\vec i)&=k_1\cdot \vec i,
&
S_p(\vec j)&=k_2\cdot \vec j,
\end{align*}
where $S_p$ stands for the shape operator of $\Sigma$ at $p$.

Proposition~\ref{prop:S=-D} implies that
\[
\frac{\partial\vec u}{\partial x}=\frac{\partial\Norm}{\partial x}=-S_p(\vec i),
\quad\text{and}\quad
\frac{\partial\vec u}{\partial y}=\frac{\partial\Norm}{\partial y}=-S_p(\vec j).
\]
It follows that
\[
\langle\tfrac{\partial\vec u}{\partial x},\vec i\rangle =-k_1, 
\quad\text{and}\quad
\langle\tfrac{\partial\vec u}{\partial y},\vec j\rangle=-k_2. \eqlbl{eq:divxy}
\]


Further, since $\vec u$ is unit, we have $\langle\vec u,\vec u\rangle\equiv 1$.
Taking derivative of this identity, we get
\begin{align*}
0&=\tfrac{\partial}{\partial z} \langle\vec u,\vec u\rangle=
\\
&=2\cdot \langle\tfrac{\partial}{\partial z}\vec u,\vec u\rangle.
\end{align*}
In particular,
\[\langle\tfrac{\partial}{\partial z}\vec u,\vec k\rangle_p=0.\eqlbl{eq:divz}\]

Let us write the field $\vec u$ in the standard basis 
$\vec u=u_1\cdot \vec i+u_2\cdot\vec j+u_3\cdot\vec k$.
Since the basis $\vec i,\vec j,\vec k$ is orthonormal, the functions $u_1,u_2,u_3$ can be defined by
\begin{align*}
u_1&=\langle\vec u,\vec i\rangle,
&
u_2&=\langle\vec u,\vec j\rangle,
&
u_3&=\langle\vec u,\vec k\rangle.
\end{align*}
Applying the definition of divergence and using \ref{eq:divxy} and \ref{eq:divz}, we obtain
\begin{align*}
\div\vec u&=\tfrac{\partial u_1}{\partial x}+\tfrac{\partial u_2}{\partial y}+\tfrac{\partial u_3}{\partial z}=
\\
&=\tfrac{\partial}{\partial x}\langle\vec u,\vec i \rangle+\tfrac{\partial}{\partial y}\langle\vec u,\vec j \rangle+\tfrac{\partial}{\partial z}\langle\vec u,\vec k\rangle=
\\
&=\langle\tfrac{\partial\vec u}{\partial x},\vec i \rangle+\langle\tfrac{\partial\vec u}{\partial y},\vec j \rangle +\langle\tfrac{\partial\vec u}{\partial z},\vec k \rangle
\\
&=-k_1(p)-k_2(p)+0
\\
&=-H(p)
\end{align*}
\qedsf

Let $V$ be a body in $\RR^3$ bounded by a closed smooth surface $\Sigma$;
assume $\Sigma$ is equipped with orientation defined by unit normal field $\Norm$ that points outside $V$.
We say that $V$ is \emph{mean-convex} if the mean curvature of $\Sigma$ is nonpositive.

\begin{thm}{Exercise}\label{ex:mean-convex}
Let $V$ be a mean-convex body in $\RR^3$ bounded by a closed smooth surface $\Sigma$.
Denote by $W$ the outer region of $\Sigma$, it is a complement of the interior of $V$.

\begin{subthm}{ex:mean-convex:u}
Suppose $V$ is star-shaped; that is, there is a point $p\in V$ such that for any other point $x\in V$ the line segment $[p,x]$ lies in $V$.

Construct a unit vector field $\vec u$ on $W$ such that $\div  \vec u\ge 0$ at every point in $W$ and the restriction of $\vec u$ to $\Sigma$ is a normal field that points in $W$.
\end{subthm}
 
\begin{subthm}{ex:mean-convex:area} Suppose that another body $V'$ is bounded by a closed smooth surface $\Sigma'$ and $V'\supset V$.
Use part \ref{SHORT.ex:mean-convex:u} and the divergence theorem to show that 
\[\area\Sigma'\ge \area\Sigma\]
if $V$ is star-shaped.
\end{subthm}

\begin{subthm}{ex:mean-convex:wrong} Construct a non-star-shaped mean-convex body $V$ bounded by a smooth surface such that the inequality in\ref{ex:mean-convex:u} does not hold for some body $V'\supset V$ with smooth boundary $\Sigma'$.
\end{subthm}
\end{thm}

\section*{Area-minimizing surfaces}

A smooth surface $\Sigma$ is called \emph{area-minimizing} if for any compact surfaces $\Delta$ in $\Sigma$ with boundary line $\partial \Delta$ 
the following inequality 
\[\area \Delta\le \area \Delta'\]
holds for any other smooth compact surface $\Delta'$ with the same boundary line; that is, if $\partial\Delta'\z=\partial\Delta$.

\begin{thm}{Exercise}\label{ex:area-ball-intersection}
Suppose $\Sigma$ is a compact area-minimizing surface with boundary line $\partial \Sigma$.
Let $p$ be a point in $\Sigma$.
Show that if the ball $B(p,r)$ does not intersect $\partial \Sigma$, then 
\[\area [\Sigma\cap B(p,r)]\le 2\cdot \pi\cdot r^2;\]
that is, this area can not exceed half of the area of sphere of radius $r$.
\end{thm}


Recall that a surface with vanishing mean curvature is called \emph{minimal}.


\begin{thm}{Proposition}\label{prop:minimizing-is-minimal}
Any area-minimizing surface is minimal.
\end{thm}



\parit{Proof.}
Assume $\Sigma$ is a surface with nonzero mean curvature at some point $p$.
Without loss of generality we may assume that it is positive,
otherwise switch the orientation.

Let $z=f(x,y)$ be a local description of $\Sigma$ in the tangent-normal coordinates at $p$.
Denote by $H(x,y)$ and $\Norm(x,y)$ the mean curvature and the unit normal vector of the graph at the point $(x,y,f(x,y))$.
Passing to a smaller domain of $f$, we can assume that 
\[H(x,y)>0\] for any $(x,y)\in\Dom f$.

Consider the vector field $\vec u$ on the domain $\Omega=\RR\times\Dom f$ defined by 
$\vec u(x,y,z)=\Norm (x,y)$.
Note that 
\[(\div \vec u)(x,y,z)+H(x,y)=0.
\eqlbl{div<0}\]
Indeed, by construction, $\vec u$ is invariant with respect to shifts of $\Omega$ up or down.
In particular the divergence $\div \vec u$ does only on $x$ and~$y$.
Therefore it is sufficient to show \ref{div<0} for points on the graph $z=f(x,y)$.
The latter follows from Lemma~\ref{lem:div+H}.

Fix a closed $\eps$-neighborhood $D_\eps$ of the origin in the $(x,y)$-plane;
we can assume that $D_\eps$ lies in the domain of $f$.
Choose a smooth function $(x,y)\mapsto h(x,y)$ defined on $D_\eps$ in its interior and vanishes on its boundary;
for example, $h=\eps^2-x^2-y^2$ will do.
Set
\[f_t(x,y)=f(x,y)+t\cdot h(x,y).\]
Denote by $\Delta_t$ be the graph $z=f_t(x,y)$.
Set $a(t)=\area\Delta_t$ and $b(t)\z=\flux_{\vec u} \Delta_t$.
Observe that both functions $t\mapsto a(t)$ and $t\mapsto b(t)$ are smooth.

By construction of $\vec u$, we have that $a(0)=b(0)$.
By Observation \ref{obs:flux}, we have that $a(t)\ge b(t)$ for any $t$.
Therefore
\[a'(0)=b'(0).\eqlbl{eq:a'=b'}\]

Fix $t>0$.
Let $\Theta_t$ be the domain squeezed between $\Delta$ and $\Delta_t$;
that is, 
\[\Theta_t=\set{(x,y,z)}{f(x,y)< z< f_t(x,y)}.\]

By divergence theorem, we have 
\begin{align*}
b(t)-b(0)&=\iiint_{\Theta_t}\div \vec u\cdot dx\cdot dy \cdot dz=
\\
&=-\iiint_{\Theta_t} H(x,y)\cdot dx\cdot dy \cdot dz=
\\
&=t\cdot\left[- \iint_\Delta H(x,y)\cdot h(x,y)\cdot dx\cdot dy\right].
\end{align*}
Since $H>0$ and $h>0$ in the interior of $\Delta$,
the last integral is positive.
It follows that $b(t)$ is a linear function with negative slope.
By \ref{eq:a'=b'}, $a'(0)=b'(0)< 0$.
In particular, 
\[\area\Delta_t<\area \Delta\] for small $t>0$;
that is, $\Sigma$ is not area-minimizing.
\qeds

The following two exercises show that minimal surface might be not area-minimizing.
Recall that catenoid and helicoid are minimal; see exercises \ref{ex:catenoid-is-minimal} and \ref{ex:helicoid-is-minimal}.
The following exercise state that sufficienlty large piece of these surfaces are not area-minimizing.

\begin{thm}{Exercise}\label{ex:catenoid-nonmin}
Show that the \emph{catenoid}
\[(\cosh z)^2=x^2+y^2.\]
is not area-minimizing.
\end{thm}

\begin{thm}{Exercise}\label{ex:helicoid-nonmin}
Show that the \emph{helicoid} 
\[s(u,v)=(u\cdot \sin v,u\cdot \cos v,v).\]
 is not a area-minimizing.
\end{thm}

The following theorem provides a condition on minimal surface that guarantees that it is area-minimizing.

\begin{thm}{Theorem}
Let $\Sigma$ be a graph $z=f(x,y)$ of a smooth function $f$ defined on an open convex set in the $(x,y)$-plane.
Suppose $\Sigma$ is minimal, then it is area-minimizing.

In particular, any minimal surface $\Sigma$ is \index{locally area-minimizing surface}\emph{locally area-minimizing};
that is, some neighborhood of any point $p$ in $\Sigma$ is area-minimizing.
\end{thm}

We omit its proof altho it is not hard;
it can be build on the ideas from the solutions of \ref{ex:mean-convex} and \ref{prop:minimizing-is-minimal}.

