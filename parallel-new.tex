\chapter{Parallel transport}


\section{Parallel tangent fields}

Let $\Sigma$ be a smooth surface in the Euclidean space and $\gamma\:[a,b]\z\to \Sigma$ be a smooth curve.
A smooth vector-valued function $t\mapsto {\vec v}(t)$ is called a \index{tangent field}\emph{tangent field} on $\gamma$ if
the vector ${\vec v}(t)$ lies in the tangent plane $\T_{\gamma(t)}\Sigma$ for each $t$.

A tangent field ${\vec v}(t)$ on $\gamma$ is called \index{parallel field}\emph{parallel} if ${\vec v}'(t)\perp\T_{\gamma(t)}$ for any~$t$.

In general the family of tangent planes $\T_{\gamma(t)}\Sigma$ is not parallel.
Therefore one cannot expect to have a {}\emph{truly} parallel family ${\vec v}(t)$ with ${\vec v}'\equiv 0$.
The condition ${\vec v}'(t)\perp\T_{\gamma(t)}$ means that the family is as parallel as possible --- it rotates together with the tangent plane, but does not rotate inside the plane.

Note that by the definition of geodesic, the velocity field ${\vec v}(t)\z=\gamma'(t)$ of any geodesic $\gamma$ is parallel on $\gamma$.

\begin{thm}{Exercise}\label{ex:parallel}
Let $\Sigma$ be a smooth regular surface in the Euclidean space, 
$\gamma\:[a,b]\to \Sigma$ a smooth curve.
Suppose that ${\vec v}(t)$, $\vec w(t)$ are parallel vector fields along $\gamma$.

\begin{subthm}{ex:parallel:a} Show that $|{\vec v}(t)|$ is constant.
\end{subthm}

\begin{subthm}{ex:parallel:b} Show that the angle $\theta(t)$ between ${\vec v}(t)$ and $\vec w(t)$ is constant.
\end{subthm}

\end{thm}

\section{Parallel transport}

Let $\Sigma$ be a smooth surface in the Euclidean space and $\gamma\:[a,b]\z\to \Sigma$ be a smooth curve.
Assume $p=\gamma(a)$ and $q=\gamma(b)$.

Given a tangent vector ${\vec v}\in\T_p$ there is unique parallel field ${\vec v}(t)$ along $\gamma$ such that ${\vec v}(a)={\vec v}$.
The latter follows from \ref{thm:ODE}; the uniqueness also follows from Exercise~\ref{ex:parallel}.

The vector $\vec w={\vec v}(b)\in\T_q$ is called the \index{parallel transport}\emph{parallel transport} of ${\vec v}$ along~$\gamma$ in $\Sigma$.

The parallel transport along $\gamma$ will be denoted by $\iota_\gamma$;
so we can write $\vec w=\iota_\gamma({\vec v})$ or we can write $\vec w=\iota_\gamma({\vec v})_\Sigma$ if we need to emphasize that $\gamma$ lies in the surface $\Sigma$.
From the Exercise~\ref{ex:parallel}, it follows that parallel transport $\iota_\gamma\:\T_p\z\to\T_q$ is an isometry.
In general, the parallel transport $\iota_\gamma\:\T_p\z\to\T_q$ depends on the choice of $\gamma$; that is, for another curve $\gamma_1$ connecting $p$ to $q$ in $\Sigma$, the parallel transports $\iota_{\gamma_1}$ and $\iota_{\gamma}$ might be different.

Suppose that $\gamma_1$ and $\gamma_2$ are two smooth curves in smooth surfaces $\Sigma_1$ and $\Sigma_2$.
Denote by $\Norm_i\:\Sigma_i\to\mathbb{S}^2$ the Gauss maps of $\Sigma_1$ and $\Sigma_2$.
If $\Norm_1\circ\gamma_1(t)= \Norm_2\circ\gamma_2(t)$ for any $t$, then we say that curves $\gamma_1$ and $\gamma_2$ have {}\emph{identical spherical images} in $\Sigma_1$ and $\Sigma_2$ respectively.

In this case tangent plane $\T_{\gamma_1(t)}\Sigma_1$ is parallel to $\T_{\gamma_2(t)}\Sigma_2$ for any $t$ and so we can identify $\T_{\gamma_1(t)}\Sigma_1$ and $\T_{\gamma_2(t)}\Sigma_2$.
In particular if $\vec v(t)$ is a tangent vector field along $\gamma_1$,
then it is also a tangent vector field along $\gamma_2$.
Moreover $\vec v'(t)\perp \T_{\gamma_1(t)}\Sigma_1$ is equivalent to $\vec v'(t)\perp \T_{\gamma_2(t)}\Sigma_2$; that is, if $\vec v(t)$ is a parallel vector field along $\gamma_1$,
then it is also a parallel vector field along $\gamma_2$.

The dicussion above leads to the following observation that will play key role in the sequel.

\begin{thm}{Observation}\label{obs:parallel=}
Let $\gamma_1$ and $\gamma_2$ be two smooth curves in smooth surfaces $\Sigma_1$ and $\Sigma_2$.
Suppose that $\gamma_1$ and $\gamma_2$ have identical spherical images in $\Sigma_1$ and $\Sigma_2$ respectively.
Then the parallel transport $\iota_{\gamma_1}$ and $\iota_{\gamma_2}$ are identical. 
\end{thm}

\begin{thm}{Exercise}\label{ex:parallel-transport-support}
Let $\Sigma_1$ and $\Sigma_2$ be two surfaces with common curve~$\gamma$.
Suppose that $\Sigma_1$ bounds a region that contains $\Sigma_2$.
Show that the parallel translations along $\gamma$ in $\Sigma_1$ 
coincides the parallel translations along $\gamma$ in $\Sigma_2$. 
\end{thm}

\section{Bike wheel and projections}

In this section we describe two interpretations of parallel transport;
they might help to build right intuition, but will not help to write a rigorous proof.
The first one is physical use \emph{bike wheel} it was suggested by Mark Levi \cite{levi} and the second via orthogonal projections of tangent planes.

Think of walking along $\gamma$ and carrying a perfectly balanced bike wheel.
Imagine that you keep its axis normal to $\Sigma$ and touch only its axis.
It should be physically evident that if the wheel is non-spinning at the starting point $p$, then it will not be spinning after stopping at $q$.
(Indeed, by pushing the axis one cannot produce torque to spin the wheel.)
The map that sends the initial position of the wheel to the final position is  the parallel transport~$\iota_\gamma$.

The observation above essentially states that {}\emph{moving axis of the wheel without changing its direction does not change the direction of the wheel's spikes.}

On a more formal level, one can choose a partition $a=t_0<\dots\z<t_n=b$ of $[a,b]$
and consider the sequence of orthogonal projections $\phi_i\:\T_{\gamma(t_{i-1})}\to\T_{\gamma(t_i)}$.
For a fine partition, the composition 
\[\phi_n\circ\dots\circ\phi_1\:\T_p\z\to\T_q\]
gives an approximation of $\iota_\gamma$.

(Note that each $\phi_i$ does not increase the magnitude of a vector and neither the composition.
It is straightforward to see that if the partition is sufficiently fine, then it is almost isometry; in particular it almost preserves the magnitudes of tangent vectors.)

\begin{thm}{Exercise}\label{ex:holonomy=not0}
Construct a loop $\gamma$ in $\mathbb{S}^2$ with base at $p$ such that the parallel transport $\iota_\gamma\:\T_p\to\T_p$ is not the identity map.
\end{thm}

\section{Geodesic curvature}

Plane is the simplest example of smooth surface.
Earlier, in Section~\ref{sec:def(skur)}, we introduced signed curvature of a plane curve.
Let us introduce the so-called \index{geodesic curvature}\emph{geodesic curvature} --- an analogous notion for a smooth curve $\gamma$ in general oriented smooth surface $\Sigma$.

\begin{wrapfigure}{o}{42 mm}
\vskip-0mm
\centering
\begin{lpic}[t(-0mm),b(0mm),r(0mm),l(0mm)]{asy/paraboloid+curve(1)}
\lbl[ul]{34,14;$\tan$}
\lbl[b]{20,43;$\Norm$}
\lbl[bl]{38,35;$\mu$}
\end{lpic}
\vskip-0mm
\end{wrapfigure}

Let $\Norm\:\Sigma\to \mathbb{S}^2$ be the spherical map that defines the orientation on $\Sigma$.
Without loss of generality we can assume that $\gamma$ has unit speed.
Then for any $t$ the vectors $\Norm(t)=\Norm(\gamma(t))_\Sigma$ and the velocity vector $\tan(t)=\gamma'(t)$ are unit vectors that are normal to each other.
Denote by $\mu(t)$ the unit vector that is normal to both $\Norm(t)$ and $\tan(t)$ that points to the left from $\gamma$; that is, $\mu=\Norm\times \tan$.
Note that the triple $\tan(t),\mu(t),\Norm(t)$ is an oriented orthonormal basis for any $t$.

Since $\gamma$ is unit-speed, the acceleration $\gamma''(t)$ is perpendicular to $\tan(t)$;
therefore at any parameter value $t$, we have
\[\gamma''(t)=k_g(t)\cdot \mu(t)-k_n(t)\cdot \Norm(t),\]
for some real numbers $k_n(t)$ and $k_g(t)$.
The numbers $k_n(t)$ and $k_g(t)$ are called \index{normal curvature}\emph{normal} and \index{geodesic curvature}\emph{geodesic curvature} of $\gamma$ at $t$ respectively;
we may write $k_n(t)_\Sigma$ and $k_g(t)_\Sigma$ if we need to emphasize that we work in the surface $\Sigma$.
 
Geodesic curvature measures how much a given curve diverges from being a geodesic;
it is positive if $\gamma$ turns left and negative if $\gamma$ turns right.
In particular, by the following exercise, geodesics have vanishing geodesic curvature.

\begin{thm}{Exercise}\label{ex:geodesic-curvature}
Let $\gamma$ be a smooth regular curve in a smooth surface~$\Sigma$.
Show that $\gamma$ is a geodesic if and only if it has constant speed and vanishing geodesic curvature.
\end{thm}

\section{Total geodesic curvature}

The total geodesic curvature is defined as integral 
\[\tgc\gamma
\df
\int_{\mathbb{I}} k_g(t)\cdot dt,\]
assuming that $\gamma$ is a smooth unit-speed curve defined on the real interval $\mathbb{I}$.

Note that if $\Sigma$ is a plane and $\gamma$ lies in $\Sigma$, then geodesic curvature of $\gamma$ equals to signed curvature and therefore total geodesic curvature equals to the total signed curvature.
By that reason we use the same notation $\tgc\gamma$ as for total signed curvature; if we need to emphasize that we consider $\gamma$ as a curve in $\Sigma$, we write $\tgc\gamma_\Sigma$.

If $\gamma$ is a piecewise smooth regular curve in $\Sigma$, then
its total geodesic curvatures is defined as a sum of all total geodesic curvature of its arcs and the sum signed exterior angles of $\gamma$ at the joints.
More precisely, if $\gamma$ is a concatenation of smooth regular curves $\gamma_1,\dots,\gamma_n$, then
\[\tgc\gamma=\tgc{\gamma_1}+\dots+\tgc{\gamma_n}+\theta_1+\dots+\theta_{n-1},\]
where $\theta_i$ is the signed external angle at the joint $\gamma_i$ and $\gamma_{i+1}$; it is positive if we turn left and negative if we turn right, it is undefined if we turn to the opposite direction.
If $\gamma$ is closed, then 
\[\tgc\gamma=\tgc{\gamma_1}+\dots+\tgc{\gamma_n}+\theta_1+\dots+\theta_{n},\]
where $\theta_n$ is the signed external angle at the joint $\gamma_n$ and $\gamma_1$.

If each arc $\gamma_i$ in the concatenation is a minimizing geodesic, then $\gamma$ is called \index{broken geodesic}\emph{broken geodesic}.
In this case $\tgc{\gamma_i}=0$ for each $i$ and therefore the total geodesic curvature of $\gamma$ is the sum of its signed external angles.

\begin{thm}{Proposition}\label{prop:pt+tgc}
Assume $\gamma$ is a closed broken geodesic in a smooth oriented surface $\Sigma$ that starts and ends at the point $p$.
Then the parallel transport $\iota_\gamma\:T_p\to\T_p$ is a rotation of the plane $\T_p$ clockwise by angle $\tgc\gamma$.

Moreover, the same statement holds true for smooth closed curves and piecewise smooth curves.
\end{thm}

\parit{Proof.}
Assume $\gamma$ is a cyclic concatenation of geodesics $\gamma_1,\dots,\gamma_n$.
Fix a tangent vector ${\vec v}$ at $p$ and extend it to a parallel vector field along $\gamma$.
Since $\tan_i(t)=\gamma_i'(t)$ is parallel along $\gamma_i$, the angle $\phi_i$ from $\tan_i$ to ${\vec v}$ stays constant on each $\gamma_i$.

\begin{wrapfigure}{o}{22 mm}
\vskip-0mm
\centering
\includegraphics{mppics/pic-48}
\vskip-0mm
\end{wrapfigure}

If $\theta_i$ denotes the external angle at the vertex of switch from $\gamma_{i}$ to $\gamma_{i+1}$, we have that 
\[\phi_{i+1}=\phi_i-\theta_i \pmod{2\cdot\pi}.\]
Therefore after going around we get that 
\[\phi_{n+1}-\phi_1=-\theta_1-\dots-\theta_n=-\tgc\gamma.\]
Hence the first statement follows.

For the smooth unit-speed curve $\gamma\:[a,b]\to\Sigma$, the proof is analogous.
Denote by $\phi(t)$ is the signed angle from ${\vec v}(t)$ to $\tan(t)$.
Let us show that 
\[\phi'(t)+k_g(t)\equiv0\eqlbl{eq:phi'+kg}\]

Recall that $\mu=\mu(t)$ denotes the counterclockwise rotation of $\tan \z=\tan(t)$ by angle $\tfrac\pi2$ in $\T_{\gamma(t)}$.
Denote by $\vec w=\vec w(t)$ the counterclockwise rotation of $\vec v=\vec v(t)$ by angle $\tfrac\pi2$ in $\T_{\gamma(t)}$.
Then
\begin{align*}
\tan&=\cos\phi\cdot \vec v-\sin\phi\cdot \vec w,
\\
\mu&=\sin\phi\cdot \vec v+\cos\phi\cdot \vec w.
\end{align*}

Note that $\vec w$ is a parallel vector field along $\gamma$; that is, $\vec v'(t),\vec w'(t)\z\perp\T_{\gamma(t)}$.
Therefore $\langle\vec v',\mu\rangle\z=\langle\vec w',\mu\rangle\z=0$.
It follows that
\begin{align*}
k_g&=\langle\tan',\mu\rangle=
\\
&=-(\sin^2\phi+\cos^2\phi)\cdot \phi'.
\end{align*}
Whence \ref{eq:phi'+kg} follows.

By \ref{eq:phi'+kg} we get that 
\begin{align*}
\phi(b)-\phi(a)&=\int_a^b \phi'(t)\cdot dt=
\\
&=-\int_a^b k_g\cdot dt=
\\
&=-\tgc\gamma
\end{align*}

The case of piecewise regular smooth curve is a straightforward combination of the above two cases. 
\qeds

\chapter{Gauss--Bonnet formula}

\section{Formulation}

The following theorem was proved by Carl Friedrich Gauss \cite{gauss}
for geodesic triangles;
Pierre Bonnet and Jacques Binet independently 
generalized the statement for arbitrary curves.
A generalized formula (\ref{thm:GB-generalized}) was proved by Walther von Dyck.

\begin{thm}{Theorem}\label{thm:gb}
Let $\Delta$ be a topological disc in a smooth oriented surface $\Sigma$ bounded by a simple piecewise smooth and regular curve $\partial \Delta$.
Suppose that $\partial \Delta$ oriented in such a way that $\Delta$ lies on its left.
Then 
\[\tgc{\partial\Delta}+\iint_\Delta K=2\cdot \pi,\eqlbl{eq:g-b}\]
where $K$ denotes the Gauss curvature of $\Sigma$.
\end{thm}

We will give an informal proof of this formula in a leading partial case.
A formal computational proof will be given in Section~\ref{sec:gauss--bonnet:formal}.

Before going into the proofs, we suggest to solve the following exercises using the Gauss--Bonnet formula.

\begin{thm}{Exercise}\label{ex:1=geodesic-curvature}
 Assume $\gamma$ is a closed simple curve with constant geodesic curvature $1$ in a smooth closed surface $\Sigma$ with positive Gauss curvature.
 Show that 
 \[\length\gamma\le 2\cdot\pi;\]
that is, the length of $\gamma$ cannot exceed the length of the unit circle in the plane.  
\end{thm}


\begin{thm}{Exercise}\label{ex:geodesic-half}
Let $\gamma$ be a closed simple geodesic on a smooth closed surface $\Sigma$ with positive Gauss curvature.
Assume $\Norm\:\Sigma\to\mathbb{S}^2$ is a Gauss map.
Show that the curve $\alpha=\Norm\circ\gamma$ divides the sphere into regions of equal area.

Conclude that $\length \alpha\ge 2\cdot\pi.$
\end{thm}

\begin{thm}{Exercise}\label{ex:closed-geodesic}
Let $\gamma$ be a closed geodesic on a smooth closed surface $\Sigma$ with positive Gauss curvature.
Suppose that $R$ is one of the regions that $\gamma$ cuts from $\Sigma$.
Show that 
\[\iint_R K\le 2\cdot\pi.\]

Conclude that any two closed geodesics on $\Sigma$ have a common point.
\end{thm}

\begin{thm}{Exercise}\label{ex:self-intersections}
Let $\Sigma$ be a smooth regular sphere with positive Gauss curvature and $p\in\Sigma$. 
Suppose $\gamma$ is a closed geodesic that is covered by one chart.
Show that $\gamma$ cannot look like one the curves on the following diagrams.

\begin{figure}[h]
\begin{minipage}{.48\textwidth}
\centering
\includegraphics{mppics/pic-46}
\end{minipage}
\hfill
\begin{minipage}{.48\textwidth}
\centering
\includegraphics{mppics/pic-47}
\end{minipage}

\medskip

\begin{minipage}{.48\textwidth}
\centering
\caption*{\textit{(easy)}}
\end{minipage}\hfill
\begin{minipage}{.48\textwidth}
\centering
\caption*{\textit{(tricky)}}
\end{minipage}
\vskip-4mm
\end{figure}

\end{thm}


\begin{wrapfigure}{r}{30 mm}
\vskip-0mm
\centering
\includegraphics{mppics/pic-471}
\end{wrapfigure}

In fact $\gamma$ also cannot look like the curve on the right, but the proof requires a more advanced teqnique;
see \ref{ex:convex-polyhon+self-intersections}.

The following exercise gives the optimal bound on Lipschitz constant of a convex function that guarantees that its geodesics have no self-intersections;
compare to \ref{ex:rough-bound-mountain}.

\begin{thm}{Exercise}\label{ex:sqrt(3)}
Suppose that $f\:\RR^2\to\RR$ is a $\sqrt{3}$-Lipschitz smooth convex function.
Show that any geodesic in the graph $z=f(x,y)$ has no self-intersections.
\end{thm}

A surface $\Sigma$ is called \index{simply connected surface}\emph{simply connected} if any closed simple curve in $\Sigma$ bounds a disc.
Equivalently any closed curve in $\Sigma$ can be continuously deformed into a \index{trivial curve}\emph{trivial curve}; that is, a curve that stands at one point all the time.

Observe that a plane or a sphere are examples of simply connected surfaces, while torus or cylinder are not simply connected.

\begin{thm}{Exercise}\label{ex:unique-geod}
Suppose that $\Sigma$ is a simply connected open surface with nonpositive Gauss curvature.
\begin{subthm}{ex:unique-geod:unique}
Show that any two points in $\Sigma$ are connected by a unique geodesic.
\end{subthm}
\begin{subthm}{ex:unique-geod:diffeomorphism}
Conculude that for any point $p\in \Sigma$,
the exponential map $\exp_p$ is a diffeomorphism from the tangent plane $\T_p$ to $\Sigma$.
In particular $\Sigma$ is diffeomorphic to the plane.
\end{subthm}
\end{thm}

\section{Additivity}

Let $\Delta$ be a topological disc in a smooth oriented surface $\Sigma$ bounded by a simple piecewise smooth and regular curve $\partial \Delta$.
As before we suppose that $\partial \Delta$ oriented in such a way that $\Delta$ lies on its left.
Set
\[\GB(\Delta)=\tgc{\partial\Delta}+\iint_\Delta K-2\cdot \pi,
\eqlbl{eq:GB}\]\index{10gb@$\GB$}
where $K$ denotes the Gauss curvature of $\Sigma$.
Here $\GB$ stands for Gauss--Bonnet formula; it can be states as
\[\GB(\Delta)=0.\]

\begin{thm}{Lemma}\label{lem:GB-sum}
Suppose that the disc $\Delta$ is subdivided into two discs $\Delta_1$ and $\Delta_2$ by a curve $\delta$.
Then
\[
\GB(\Delta)=\GB(\Delta_1)+\GB(\Delta_2).
\]
\end{thm}

\begin{wrapfigure}[8]{r}{40 mm}
\vskip-4mm
\centering
\includegraphics{mppics/pic-1750}
\end{wrapfigure}

\parit{Proof.}
Let us subdivide $\partial \Delta$ into two curves $\gamma_1$ and $\gamma_2$ that share the endpoints with $\delta$ so that
\begin{itemize}
\item $\Delta_1$ is bounded by an arc $\gamma_1$  and~$\delta$.
\item Similarly, $\Delta_2$ is bounded by $\gamma_2$ and~$\delta$.
\end{itemize}

Denote by $\phi_1$, $\phi_2$, $\psi_1$, and $\psi_2$ the angles between $\delta$ and $\gamma_i$ marked on the diagram.
Further, suppose that the arcs $\gamma_1$, $\gamma_2$, and $\delta$ are oriented as on the diagram. 
Then
\begin{align*}
\tgc{\partial \Delta}&= \tgc{\gamma_1}-\tgc{\gamma_2}+(\pi-\phi_1-\phi_2)+(\pi-\psi_1-\psi_2),
\\
\tgc{\partial \Delta_1}&= \tgc{\gamma_1}-\tgc\delta+(\pi-\phi_1)+(\pi-\psi_1),
\\
\tgc{\partial \Delta_1}&= \tgc\delta-\tgc{\gamma_2}+(\pi-\phi_2)+(\pi-\psi_2),
\\
\iint_\Delta K&=\iint_{\Delta_1} K+\iint_{\Delta_2} K.
\end{align*}
It remains to plug in the results in the formulas for $\GB(\Delta)$, $\GB(\Delta_1)$, and $\GB(\Delta_2)$.
\qeds

\section{Spherical case}

Note that if $\Sigma$ is a plane, then the Gauss curvature vanished;
therefore the Gauss--Bonnet formula \ref{eq:g-b} can be written as 
\[\tgc{\partial\Delta}=2\cdot \pi,\]
and it follows from \ref{prop:total-signed-curvature}.
In other words, $\GB(\Delta)=0$ for any disc~$\Delta$ in the plane.

If $\Sigma$ is the unit sphere, then $K\equiv1$;
in this case Theorem \ref{thm:gb} can be formulated the following way:

\begin{thm}{Proposition}\label{prop:area-of-spher-polygon}
Let $P$ be a spherical polygon bounded by a simple closed broken geodesic $\partial P$.
Assume $\partial P$ is oriented such that $P$ lies on the left from $\partial P$.
Then 
\[\GB(P)=\tgc{\partial P}+\area P-2\cdot \pi=0.\]

Moreover the same formula holds true for any spherical region bounded by piecewise smooth simple closed curve.
\end{thm}

This proposition will be used in the informal proof given below.

\parit{Sketch of proof.}
Suppose that a spherical triangle $\Delta$ has angles 
$\alpha$, $\beta$, and $\gamma$.
According to \ref{lem:area-spher-triangle},
\[\area\Delta=\alpha+\beta+\gamma-\pi.\]

Recall that $\partial\Delta$ is oriented so that $\Delta$ lies on its left. 
Then its oriented external angles are  $\pi-\alpha$, $\pi-\beta$ and $\pi-\gamma$.
Therefore 
\[\tgc{\partial\Delta}=3\cdot\pi-\alpha-\beta-\gamma.\]
It follows that $\tgc{\partial\Delta}+\area \Delta=2\cdot\pi$ or, equivalently,
\[\GB(\Delta)=0.\]
 
Note that we can subdivide a given spherical polygon $P$ into triangles by dividing a polgon in two on each step.
By the additivity lemma (\ref{lem:GB-sum}), we get
\[\GB(P)=0\]
for any spherical polygon $P$.

The second statement can be proved by approximation. One has to show that the total geodesic curvature of an inscribed broken geodesic approximates the total geodesic curvature of the original curve.
We omit the proof of the latter statement goes along the same lines as \ref{ex:total-curvature=}.
\qeds


\begin{thm}{Exercise}\label{ex:half-sphere-total-curvature}
Assume $\gamma$ is a simple piecewise smooth loop on $\mathbb{S}^2$ that divides its area into two equal parts.
Denote by $p$ the base point of $\gamma$.
Show that the parallel transport $\iota_\gamma\:\T_p\mathbb{S}^2\to\T_p\mathbb{S}^2$ is the identity map.
\end{thm}



\section{Intuitive proof}

In this section we prove the Gauss--Bonnet in a partial case.
This case is leading --- the general case can be proved similarly, but one has to use the signed area counted with multiplicity.

\parit{Proof of \ref{thm:gb} for proper surfaces with positive Gauss curvature.}
By \ref{cor:intK}, in this case, we have
\[\GB(\Delta)=\tgc{\partial\Delta}+\area[\Norm(\Delta)]-2\cdot \pi.
\eqlbl{eq:gb-area}\]

Fix $p\in \partial\Delta$;
assume the loop $\alpha$ runs along $\partial\Delta$ so that $\Delta$ lies on the left from it.
Consider the parallel translation $\iota_\alpha\:\T_p\to\T_p$ along $\alpha$.
According to \ref{prop:pt+tgc}, $\iota_\alpha$ is a clockwise rotation by angle $\tgc{\alpha}_\Sigma$.

Set $\beta=\Norm\circ\alpha$.
By \ref{obs:parallel=}, the map $\iota_\alpha=\iota_\beta$ where $\beta$ is considered as a curve in the unit sphere.
In particular $\iota$ is a clockwise rotation by angle $\tgc{\beta}_{\mathbb{S}^2}$.
By \ref{prop:area-of-spher-polygon} 
\[\GB(\Norm(\Delta))=\tgc{\beta}_{\mathbb{S}^2}+\area[\Norm(\Delta)]-2\cdot \pi=0.\]
Therefore 
$\iota$ is a counterclockwise rotation by $\area[\Norm(\Delta)]$

Summarizing, the clockwise rotation by $\tgc{\alpha}_\Sigma$ is identical to a counterclockwise rotation by $\area[\Norm(\Delta)]$.
The rotations are identical if the angles are equal modulo $2\cdot\pi$.
Therefore 
\[
\begin{aligned}
\GB(\Norm(\Delta))&=\tgc{\partial\Delta}_\Sigma+\area[\Norm(\Delta)]-2\cdot \pi=
\\
&=2\cdot n \cdot \pi
\end{aligned}
\eqlbl{eq:sum=2pin}\]
for an integer $n$.

It remains to show that $n=0$.
By \ref{prop:total-signed-curvature}, this is so for a topological disc in a plane.
One can think of a general disc $\Delta$ as about a result of a continuous deformation of a plane disc.
The integer $n$ cannot change in the process of deformation since the left hand side in \ref{eq:sum=2pin} is continuous along the deformation;
whence $n=0$ for the result of the deformation.
\qeds

\section{Simple geodesic}

The following theorem provides an interesting application of Gauss--Bonnet formula;
it is proved by Stephan Cohn-Vossen \cite[Satz 9 in][]{convossen}.

\begin{thm}{Theorem}\label{thm:cohn-vossen}
Any open smooth regular surface with positive Gauss curvature has a simple two-sided infinite geodesic.
\end{thm}

\parit{Proof.}
Let $\Sigma$ be an open surface with positive Gauss curvature and $\gamma$ a two-sided infinite geodesic in $\Sigma$.

If $\gamma$ has a self-intersection, then it contains a simple loop;
that is, a restriction $\ell=\gamma|_{[a,b]}$ for some closed interval $[a,b]$ is a simple loop.

By \ref{ex:convex-proper-plane}, $\Sigma$ is parameterized by an open convex region $\Omega$ in the plane.
By Jordan's theorem (\ref{thm:jordan}), $\ell$ bounds a disc in $\Sigma$; denote it by~$\Delta$.
If $\phi$ is the angle at the base of the loop, then by Gauss--Bonnet formula,
\[\iint_\Delta K=\pi+\phi.\] 

Recall that by \ref{ex:intK:2pi}, we have
\[\iint_\Sigma K\le 2\cdot\pi.\eqlbl{intK=<2pi+}\]
Therefore $0<\phi<\pi$; that is, $\gamma$ has no concave simple loops.

Assume $\gamma$ has two simple loops, say $\ell_1$ and $\ell_2$ that bound discs $\Delta_1$ and $\Delta_2$.
Then the disks $\Delta_1$ and $\Delta_2$ have to overlap,
otherwise the curvature of $\Sigma$ would exceed $2\cdot\pi$ which contradicts \ref{intK=<2pi+}.


It follows that after leaving $\Delta_1$, the geodesic $\gamma$ has to enter it again before creating another simple loop.
\begin{figure}[h!]
\vskip-0mm
\centering
\includegraphics{mppics/pic-1550}
\end{figure}
Consider the moment when $\gamma$ enters $\Delta_1$;
two possible scenarios are shown on the picture.
On the left picture we get two nonoverlapping discs which, as we know, is impossible.
The right picture is impossible as well --- in this case we get a concave simple loop.

It follows that $\gamma$ contains only one simple loop.
This loop cuts a disk from $\Sigma$ 
and goes around it either clockwise or counterclockwise.
This way we divide all the self-intersecting geodesics on $\Sigma$
into two sets which we will call {}\emph{clockwise} and {}\emph{counterclockwise}.

Note that the geodesic $t\mapsto \gamma(t)$ is clockwise 
if and only if the same geodesic traveled backwards
$t\mapsto \gamma(-t)$
is counterclockwise.

Let us shoot a unit-speed geodesic in all directions at a given point $p\z=\gamma(0)$.
It gives a one-parameter family of geodesics $\gamma_s$ for $s\in[0,\pi]$ connecting the geodesic $t\mapsto \gamma(t)$ with
the $t\mapsto \gamma(-t)$; that is, $\gamma_0(t)\z=\gamma(t)$ and $\gamma_\pi(t)=\gamma(-t)$.

Observe that the subset of values $s\in [0,\pi]$ such that $\gamma_s$ is right (or left) is open.
That is, if $\gamma_s$ is right, then so is $\gamma_t$ for all $t$ sufficiently close to $s$.
Indeed, denote by $\phi_s$ the angle of the simple loop of $\gamma_s$.
From abve we have $0<\phi_s<\pi$.
Therefore the self-intersection at the base of the loop of $\gamma_s$ is transverse.
It follows that the self-intersection survives in $\gamma_t$ for all $t$ sufficiently close to $s$.

Since $[0,\pi]$ is connected, it cannot be subdivided into two nonempty open sets.
It follows that for some $s$, the geodesic $\gamma_s$ is neither  clockwise nor counterclockwise;
that is, $\gamma_s$ has no self-intersections.
\qeds

\begin{wrapfigure}{o}{17 mm}
\vskip-3mm
\centering
\includegraphics{mppics/pic-1575}
\end{wrapfigure}

\begin{thm}{Exercise}\label{ex:cohn-vossen}
Let $\Sigma$ be an open smooth regular surface with positive Gauss curvature.
Suppose $\alpha\:[0,1]\z\to \Sigma$ is a smooth regular loop such that $\alpha'(0)+\alpha'(1)=0$.
Show that there is a simple two-sided infinite geodesic $\gamma$ that is tangent to $\alpha$ at some point.
\end{thm}

\section{General domains}

\begin{thm}{Theorem}\label{thm:GB-generalized}
Let $\Lambda$ be a compact domain bounded by a finite collection (possibly empty) of simple piecewise smooth and regular curves $\gamma_1,\dots,\gamma_n$ in a smooth surface $\Sigma$.
Suppose that each $\gamma_i$ 
is oriented in such a way that $\Lambda$ lies on its left.
Then \[\tgc{\gamma_1}+\dots+\tgc{\gamma_n}+\iint_\Lambda K=2\cdot \pi\cdot \chi\eqlbl{eq:g-b++}\]
for an integer $\chi=\chi(\Lambda)$.

Moreover, if $\Lambda$ can be subdivided into $f$ discs by an embedded graph with   $v$ vertexes and $e$ edges then $\chi=v-e+f$.
\end{thm}

The number $\chi=\chi(\Lambda)$ is called \index{Euler characteristic}\emph{Euler characteristic} of $\Lambda$. 
Note that $\chi$ does not depend on the choice of subdivision since the left hand side in \ref{eq:g-b++} does not.

\parit{Proof.}
Suppose a graph with $v$ vertexes and $e$ edges subdivides $\Lambda$ into $f$ discs.
Apply Gauss--Bonnet formula for each disc and sum up the results.
Observe that each disc and each vertex contributes $2\cdot\pi$ and each edge contributes $-2\cdot\pi$ to the total sum.
Whence \ref{eq:g-b++} follows.
\qeds



\begin{thm}{Exercise}\label{ex:g-b-chi}
Find the integral of Gauss curvature for each of the following surfaces:

\begin{subthm}{ex:g-b-chi:torus}
Torus.
\end{subthm}

\begin{subthm}{ex:g-b-chi:moebius}
Moebius band with geodesic boundary.
\end{subthm}

\begin{subthm}{ex:g-b-chi:pair-of-pants}
Pair of pants with geodesic boundary components.
\end{subthm}

\begin{subthm}{ex:g-b-chi:two-handles}
Sphere with two handles.
\end{subthm}

\end{thm}

