\chapter{Parallel transport}

\section*{Parallel fields}

Let $\Sigma$ be a smooth surface in the Euclidean space and $\gamma\:[a,b]\z\to \Sigma$ be a smooth curve.
A smooth vector-valued function $t\mapsto {\vec v}(t)$ is called a \emph{tangent field} on $\gamma$ if
the vector ${\vec v}(t)$ lies in the tangent plane $\T_{\gamma(t)}\Sigma$ for each $t$.

A tangent field ${\vec v}(t)$ on $\gamma$ is called \emph{parallel} if ${\vec v}'(t)\perp\T_{\gamma(t)}$ for any~$t$.

In general the family of tangent planes $\T_{\gamma(t)}\Sigma$ is not parallel.
Therefore one cannot expect to have a truly parallel family ${\vec v}(t)$ with ${\vec v}'\equiv 0$.
The condition ${\vec v}'(t)\perp\T_{\gamma(t)}$ means that the family is as parallel as possible --- it rotates together with the tangent plane, but does not rotate inside the plane.

Note that by the definition of geodesic, the velocity field ${\vec v}(t)\z=\gamma'(t)$ of any geodesic $\gamma$ is parallel on $\gamma$.

\begin{thm}{Exercise}\label{ex:parallel}
Let $\Sigma$ be a smooth regular surface in the Euclidean space, 
$\gamma\:[a,b]\to \Sigma$ a smooth curve 
and ${\vec v}(t)$, ${\vec w}(t)$ parallel vector fields along $\gamma$.

\begin{subthm}{ex:parallel:a} Show that $|{\vec v}(t)|$ is constant.
\end{subthm}

\begin{subthm}{ex:parallel:b} Show that the angle $\theta(t)$ between ${\vec v}(t)$ and ${\vec w}(t)$ is constant.
\end{subthm}

\end{thm}

\section*{Parallel transport}

Assume $p=\gamma(a)$ and $q=\gamma(b)$.
Given a tangent vector ${\vec v}\in\T_p$ there is unique parallel field ${\vec v}(t)$ along $\gamma$ such that ${\vec v}(a)={\vec v}$.
The latter follows from \ref{thm:ODE}; the uniqueness also follows from Exercise~\ref{ex:parallel}.

The vector $\vec w={\vec v}(b)\in\T_q$ is called the \emph{parallel transport} of ${\vec v}$ along~$\gamma$ in $\Sigma$.

The parallel transport is denoted by $\iota_\gamma$;
so we can write $\vec w=\iota_\gamma({\vec v})$ or we can write $\vec w=\iota_\gamma({\vec v})_\Sigma$ if we need to emphasize that we consider surface $\Sigma$.
From the Exercise~\ref{ex:parallel}, it follows that parallel transport $\iota_\gamma\:\T_p\z\to\T_q$ is an an isometry.
In general, the parallel transport $\iota_\gamma\:\T_p\z\to\T_q$ depends on the choice of $\gamma$; that is, for another curve $\gamma_1$ connecting $p$ to $q$ in $\Sigma$, the parallel transport $\iota_{\gamma_1}\:\T_p\to\T_q$ might be different.

Suppose that $\gamma_1$ and $\gamma_2$ are two smooth curves in smooth surfaces $\Sigma_1$ and $\Sigma_2$.
Denote by $\Norm_i\:\Sigma_i\to\SS^2$ the Gauss maps of $\Sigma_1$ and $\Sigma_2$.
If $\Norm_1\circ\gamma_1(t)= \Norm_2\circ\gamma_2(t)$ for any $t$, then we say that curves $\gamma_1$ and $\gamma_2$ have \emph{identical spherical images} in $\Sigma_1$ and $\Sigma_2$ correspondingly.

In this case tangent plane $\T_{\gamma_1(t)}\Sigma_1$ is parallel to $\T_{\gamma_2(t)}\Sigma_2$ for any $t$ and so we can identify $\T_{\gamma_1(t)}\Sigma_1$ and $\T_{\gamma_2(t)}\Sigma_2$.
In particular if $\vec v(t)$ is a tangent vector field along $\gamma_1$,
then it is also a tangent vector field along $\gamma_2$.
Moreover $\vec v'(t)\perp \T_{\gamma_1(t)}\Sigma_1$ is equivalent to $\vec v'(t)\perp \T_{\gamma_2(t)}\Sigma_2$; that is, if $\vec v(t)$ is a parallel vector field along $\gamma_1$,
then it is also a parallel vector field along $\gamma_2$.

The dicussion above leads to the following observation that will play key role in the sequel.

\begin{thm}{Observation}\label{obs:parallel=}
Let $\gamma_1$ and $\gamma_2$ be two smooth curves in smooth surfaces $\Sigma_1$ and $\Sigma_2$.
Suppose that $\gamma_1$ and $\gamma_2$ have identical spherical images in $\Sigma_1$ and $\Sigma_2$ correspondingly.
Then the parallel transport $\iota_{\gamma_1}$ and $\iota_{\gamma_2}$ are identical. 
\end{thm}

\begin{thm}{Exercise}\label{ex:parallel-transport-support}
Let $\Sigma_1$ and $\Sigma_2$ be two surfaces with common curve~$\gamma$.
Suppose that $\Sigma_1$ bounds a region that contains $\Sigma_2$.
Show that the parallel translations along $\gamma$ in $\Sigma_1$ 
coincides the parallel translations along $\gamma$ in $\Sigma_2$. 
\end{thm}

The following physical interpretation of parallel translation was suggested by Mark Levi \cite{levi};
it might help to build right intuition.

Think of walking along $\gamma$ and carrying a perfectly balanced bike wheel keeping its axis normal to $\Sigma$ and touching only the axis.
It should be physically evident that if the wheel is non-spinning at the starting point $p$, then it will not be spinning after stopping at $q$.
(Indeed, by pushing the axis one cannot produce torque to spin the wheel.)
The map that sends the initial position of the wheel to the final position is  the parallel transport~$\iota_\gamma$.

The observation above essentially states that \emph{moving axis of the wheel without changing its direction does not change the direction of the wheel's spikes.}

On a more formal level, one can choose a partition $a=t_0<\dots\z<t_n=b$ of $[a,b]$
and consider the sequence of orthogonal projections $\phi_i\:\T_{\gamma(t_{i-1})}\to\T_{\gamma(t_i)}$.
For a fine partition, the composition 
\[\phi_n\circ\dots\circ\phi_1\:T_p\z\to\T_q\]
gives an approximation of $\iota_\gamma$.
Each $\phi_i$ does not increase the magnitude of a vector and neither the composition.
It is straightforward to see that if if the partition is sufficiently fine, then it is almost isometry; in particular it almost preserves the magnitudes of tangent vectors.

\begin{thm}{Exercise}\label{ex:holonomy=not0}
Construct a loop $\gamma$ in $\SS^2$ with base at $p$ such that the parallel transport $\iota_\gamma\:\T_p\to\T_p$ is not the identity map.
\end{thm}

\section*{Geodesic curvature}

Plane is the simplest example of smooth surface.
Earlier we introduced signed curvature of a plane curve.
Let us introduce the so called \emph{geodesic curvature} --- an analogous notion for a smooth curve $\gamma$ in general oriented smooth surface $\Sigma$.

%???PIC

Let $\Norm\:\Sigma\to \SS^2$ be the spherical map that defines the orientation on $\Sigma$.
Without loss of generality we can assume that $\gamma$ has unit speed.
Then for any $t$ the vectors $\Norm(t)=\Norm(\gamma(t))_\Sigma$ and the velocity vector $\tan(t)=\gamma'(t)$ are unit vectors that are normal to each other.
Denote by $\mu(t)$ the unit vector that is normal to both $\Norm(t)$ and $\tan(t)$ that points to the left from $\gamma$; that is, $\mu=\Norm\times \tan$.
Note that the triple $\tan(t),\mu(t),\Norm(t)$ is an oriented orthonormal basis for any $t$.

Since $\gamma$ is unit-speed, the acceleration $\gamma''(t)$ is perpendicular to $\tan(t)$;
therefore at any parameter value $t$, we have
\[\gamma''(t)=k_g(t)\cdot \mu(t)-k_n(t)\cdot \Norm(t),\]
for some real numbers $k_n(t)$ and $k_g(t)$.
The numbers $k_n(t)$ and $k_g(t)$ are called \emph{normal} and \emph{geodesic curvature} of $\gamma$ at $t$ correspondingly;
we may write $k_n(t)_\Sigma$ and $k_g(t)_\Sigma$ if we need to emphasize that we work in the surface $\Sigma$.

Note that the geodesic curvature vanishes if $\gamma$ is a geodesic. 
It measures how much a given curve diverges from being a geodesic;
it is positive if $\gamma$ turns left and negative if $\gamma$ turns right.

\begin{thm}{Exercise}\label{ex:geodesic-curvature}
Let $\gamma$ be a smooth regular curve in a smooth surface~$\Sigma$.
Show that $\gamma$ is a geodesic if and only if it has constant speed and vanishing geodesic curvature.
\end{thm}

\section*{Total geodesic curvature}

The total geodesic curvature is defined as integral 
\[\tgc\gamma\df \int_{\mathbb{I}} k_g(t)\cdot dt,\]
assuming that $\gamma$ is a smooth unit-speed curve defined on the real interval $\mathbb{I}$.

Note that if $\Sigma$ is a plane and $\gamma$ lies in $\Sigma$, then geodesic curvature of $\gamma$ equals to signed curvature and therefore total geodesic curvature equals to the total signed curvature.
By that reason we use the same notation $\tgc\gamma$ as for total signed curvature; if we need to emphasize that we consider $\gamma$ as a curve in $\Sigma$, we write $\tgc\gamma_\Sigma$.

If $\gamma$ is a piecewise smooth regular curve in $\Sigma$, then
its total geodesic curvature is defined as a sum of all total geodesic curvature of its arcs and the sum signed exterior angles of $\gamma$ at the joints.
More precisely, if $\gamma$ is a concatenation of smooth regular curves $\gamma_1,\dots,\gamma_n$, then
\[\tgc\gamma=\tgc{\gamma_1}+\dots+\tgc{\gamma_n}+\theta_1+\dots+\theta_{n-1},\]
where $\theta_i$ is the signed external angle at the joint $\gamma_i$ and $\gamma_{i+1}$; it is positive if we turn left and negative if we turn right, it is undefined if we turn to the opposite direction.
If $\gamma$ is closed, then 
\[\tgc\gamma=\tgc{\gamma_1}+\dots+\tgc{\gamma_n}+\theta_1+\dots+\theta_{n},\]
where $\theta_n$ is the signed external angle at the joint $\gamma_n$ and $\gamma_1$.

If each arc $\gamma_i$ in the concatenation is a geodesic, then $\gamma$ is called \emph{broken geodesic}.
In this case $\tgc{\gamma_i}=0$ for each $i$ and therefore the total geodesic curvature of $\gamma$ is the sum of its signed external angles.

\begin{thm}{Proposition}\label{prop:pt+tgc}
Assume $\gamma$ is a closed broken geodesic in a smooth oriented surface $\Sigma$ that starts and ends at the point $p$.
Then the parallel transport $\iota_\gamma\:T_p\to\T_p$ is a rotation of the the plane $\T_p$ clockwise by angle $\tgc\gamma$.

Moreover, the same statement holds for smooth closed curves and piecewise smooth curves.
\end{thm}

\begin{wrapfigure}{o}{22 mm}
\vskip-0mm
\centering
\includegraphics{mppics/pic-48}
\vskip-0mm
\end{wrapfigure}

\parit{Proof.}
Assume $\gamma$ is a cyclic concatenation of geodesics $\gamma_1,\dots,\gamma_n$.
Fix a tangent vector ${\vec v}$ at $p$ and extend it to a parallel vector field along $\gamma$.
Since ${\vec w}_i(t)=\gamma_i'(t)$ is parallel along $\gamma_i$, the angle $\phi_i$ between ${\vec v}$ and ${\vec w}_i$ stays constant on each $\gamma_i$.

If $\theta_i$ denotes the external angle at this vertex of switch from $\gamma_{i}$ to $\gamma_{i+1}$, we have that 
\[\phi_{i+1}=\phi_i-\theta_i \pmod{2\cdot\pi}.\]
Therefore after going around we get that 
\[\phi_{n+1}-\phi_1=-\theta_1-\dots-\theta_n=-\tgc\gamma.\]
Hence the the first statement follows.

For the smooth unit-speed curve $\gamma\:[a,b]\to\Sigma$, the proof is analogous.
If $\phi(t)$ denotes the angle between ${\vec v}(t)$ and ${\vec w}(t)=\gamma'(t)$, then 
\[\phi'(t)+k_g(t)\equiv0\]
Whence the angle of rotation 
\begin{align*}
\phi(b)-\phi(a)&=\int_a^b \phi'(t)\cdot dt=
\\
&=-\int_a^b k_g\cdot dt=
\\
&=-\tgc\gamma
\end{align*}

The case of piecewise regular smooth curve is a straightforward combination of the above two cases. 
\qeds


\section*{Spherical area}

If the contour $\partial\Delta$ of a spherical triangle with angles $\alpha$, $\beta$ and $\gamma$ is oriented such that the triangle lies on the left, then its external angles are  $\pi-\alpha$, $\pi-\beta$ and $\pi-\gamma$.
Therefore the total geodesic curvature of $\partial\Delta$ is $\tgc{\partial\Delta}=3\cdot\pi-\alpha-\beta-\gamma$.
The identity \ref{eq:area(Delta)} can be rewritten as 
\[\tgc{\partial\Delta}+\area\Delta=2\cdot \pi.
\eqlbl{eq:sphere-gauss-bonnet}\]

The formula \ref{eq:sphere-gauss-bonnet} holds for an arbitrary spherical polygon bounded by a simple broken geodesic.
The latter can be proved by triangulating the poygon, applying the formula for each triangle in the triangulation and summing up the results.

\begin{wrapfigure}{o}{42 mm}
\vskip-0mm
\centering
\includegraphics{mppics/pic-45}
\vskip-0mm
\end{wrapfigure}

If a spherical polygon is $P$ divided in two polygons $Q$ and $R$ by polygonal line between vertexes $v$ and $w$
then 
\[\tgc{\partial P}+2\cdot\pi =\tgc{\partial Q}+\tgc{\partial R}.\]
Indeed, for the internal angles $Q$ and $R$ at $v$ are $\alpha$ and $\beta$,
then their external angles are $\pi-\alpha$ and $\pi-\beta$ respectfully.
The internal angle of $P$ in this case is $\alpha+\beta$ and its external angle is $\pi-\alpha-\beta$
Clearly we have that 
\[(\pi-\alpha)+(\pi-\beta)=(\pi-\alpha-\beta)+\pi;\]
that is, the sum of external angles of $Q$ and $R$ at $v$ is $\pi$ plus the external angle of $P$ at $v$. 
The same holds for the external angles at $w$ and the rest of the external angles of $P$ appear once on $Q$ or $R$.
Therefore if the formula \ref{eq:sphere-gauss-bonnet} holds for $Q$ and $R$,
then it holds for~$P$.

The following proposition gives a spherical analog of \ref{prop:total-signed-curvature}.

\begin{thm}{Proposition}\label{prop:area-of-spher-polygon}
Let $P$ be a spherical polygon bounded by a simple closed broken geodesic $\partial P$.
Assume $\partial P$ is oriented such that $P$ lies on the left from $\partial P$.
Then 
\[\tgc{\partial P}+\area P=2\cdot \pi.\]

Moreover the same formula holds for any spherical region $P$ bounded by piecewise smooth simple closed curve $\partial P$.
\end{thm}

\parit{Sketch of proof.}
The proof of the first statement is given above. 

The second statement can be proved by approximation. One has to show that the total geodesic curvature of an inscribed broken geodesic approximates the total geodesic curvature of the original curve.
We omit the proof of the latter statement, but it can be done along the same lines as \ref{thm:total-curvature=}.
\qeds


\begin{thm}{Exercise}\label{ex:half-sphere-total-curvature}
Assume $\gamma$ is a simple piecewise smooth loop on $\SS^2$ that divides its area into two equal parts.
Denote by $p$ the base point of $\gamma$.
Show that $\iota_\gamma\:\T_p\SS^2\to\T_p\SS^2$ is the identity map.
\end{thm}

\section*{Gauss--Bonnet formula}


\begin{thm}{Theorem}\label{thm:gb}
Let $\Delta$ be a topological disc in a smooth oriented surface $\Sigma$ bounded by a simple piecewise smooth and regular curve $\partial \Delta$ that is oriented in such a way that $\Delta$ lies on its left.
Then 
\[\tgc{\partial\Delta}+\int_\Delta K=2\cdot \pi,\eqlbl{eq:g-b}\]
where $K$ denotes the Gauss curvature of $\Sigma$.
\end{thm}

For geodesic triangles this theorem was proved by Carl Friedrich Gauss \cite{gauss};
Pierre Bonnet and Jacques Binet independently generalized the statement for arbitrary curves. 

Note that if $\Sigma$ is a plane, then the Gauss curvatue vanished;
therefore the statement of theorem follows from \ref{prop:total-signed-curvature}.

If $\Sigma$ is the unit sphere, then $K\equiv1$. Therefore formula \ref{eq:g-b} can be rewritten as 
\[\tgc{\partial\Delta}+\area\Delta=2\cdot \pi,\]
which follows from \ref{prop:area-of-spher-polygon}.

We will give an informal proof of \ref{thm:gb} in a partial case based on the bike wheel interpretation described above.
We suppose that it is intuitively clear that moving the axis of the wheel without changing its direction does not change the direction of the wheel's spikes.

More precisely, assume we keep the axis of a non-spinning bike wheel and perform the following two experiments:
\begin{enumerate}[(i)]
\item We move it around and bring the axis back to the original position. 
As a result the wheel might turn by some angle; let us measure this angle.

\item
We move the direction of the axis the same way as before without moving the center of the wheel.
After that we measure the angle of rotation.
\end{enumerate}
Then the resulting angles in these two experiments is the same. 

Consider a surface $\Sigma$ with a Gauss map $\Norm\:\Sigma\to \SS^2$.
Note that for any point $p$ on $\Sigma$, the tangent plane $\T_p\Sigma$ is parallel to the tangent plane $\T_{\Norm(p)}\SS^2$; so we can identify these tangent spaces.
From the experiments above, we get the following:

\begin{thm}{Lemma}\label{lem:spherical-image}
Suppose $\alpha$ is a piecewise smooth regular curve in a smooth regular surface $\Sigma$ which has a Gauss map $\Norm\:\Sigma\to \SS^2$.
Then the parallel transport along $\alpha$ in $\Sigma$ coincides with the parallel transport along the curve $\beta=\Norm\circ\alpha$ in $\SS^2$.
\end{thm}

\parit{Proof of partial case of \ref{thm:gb}.}
We will prove the formula for proper surface $\Sigma$ with positive Gauss curvature.
In this case, by \ref{thm:spherical-image} the formula can be rewritten as 
\[\tgc{\partial\Delta}+\area[\Norm(\Delta)]=2\cdot \pi.\eqlbl{eq:gb-area}\]
The general case can be proved similarly, but one has to use the area formula (\ref{thm:area-formula}) and oriented area surrounded by a spherical curve.

Fix $p\in \partial\Delta$;
assume the loop $\alpha$ runs along $\partial\Delta$ so that $\Delta$ lies on the left from it.
Consider the parallel translation $\iota\:\T_p\to\T_p$ along $\alpha$.
According to \ref{prop:pt+tgc}, $\iota$ is a clockwise rotation by angle $\tgc{\alpha}_\Sigma$.

Set $\beta=\Norm\circ\alpha$.
According to \ref{lem:spherical-image}, $\iota$ is also parallel translation along $\beta$ in $\SS^2$.
In particular $\iota$ is a clockwise rotation by angle $\tgc{\beta}_{\SS^2}$.
By \ref{prop:area-of-spher-polygon} 
\[\tgc{\beta}_{\SS^2}+\area[\Norm(\Delta)]=2\cdot \pi.\]
Therefore 
$\iota$ is a counterclockwise rotation by $\area[\Norm(\Delta)]$

Summarizing, the clockwise rotation by $\tgc{\alpha}_\Sigma$ is identical to a counterclockwise rotation by $\area[\Norm(\Delta)]$.
The rotations are identical if the angles are equal modulo $2\cdot\pi$.
Therefore 
\[\tgc{\partial\Delta}_\Sigma+\area[\Norm(\Delta)]=2\cdot \pi\cdot n\eqlbl{eq:sum=2pin}\]
for an integer $n$.

It remains to show that $n=1$.
By \ref{eq:sum=2pin}, this is so for a topological disc in a plane.
One can think of a general disc $\Delta$ as about a result of a continuous deformation of a plane disc.
The integer $n$ cannot change in the process of deformation since the left hand side in \ref{eq:sum=2pin} is continuous along the deformation.

Let us redo the last argument more formally.

First assume that $\Delta$ lies in a local graph realization $z=f(x,y)$ of $\Sigma$.
Consider one parameter family $\Sigma_t$ of graphs $z=t\cdot f(x,y)$ and denote by $\Delta_t$ the corresponding disc in $\Sigma_t$, so $\Delta_1=\Delta$ amd $\Delta_0$ is its projection to the $(x,y)$-plane.
Since $\Sigma_0$ is a plane domain, we have $\area[\Norm_0(\Delta_0)]=0$.
Therefore by \ref{prop:total-signed-curvature} we gave 
\[\tgc{\partial\Delta_0}_{\Sigma_0}+\area[\Norm_0(\Delta_0)]=2\cdot \pi.\]

Note that 
\[\tgc{\partial\Delta_t}_{\Sigma_t}+\area[\Norm_t(\Delta_t)]\]
depends continuously on $t$.
According to \ref{eq:sum=2pin}, its value is a multiple of $2\cdot\pi$;
therefore it has to be constant.
Whence the Gauss--Bonnet formula follows.

If $\Delta$ does not lie in one graph, then one could divide it into smaller discs, apply the formula for each and sum up the result.
The proof is done along the same lines as \ref{prop:area-of-spher-polygon}.
\qeds



\begin{thm}{Exercise}\label{ex:1=geodesic-curvature}
 Assume $\gamma$ is a closed simple curve with constant geodesic curvature $1$ in a smooth closed surface $\Sigma$ with positive Gauss curvature.
 Show that 
 \[\length\gamma\le 2\cdot\pi;\]
that is, the length of $\gamma$ cannot exceed the length of the unit circle in the plane.  
\end{thm}


\begin{thm}{Exercise}\label{ex:geodesic-half}
Let $\gamma$ be a closed simple geodesic on a smooth closed surface $\Sigma$ with positive Gauss curvature.
Assume $\Norm\:\Sigma\to\SS^2$ is a Gauss map.
Show that the curve $\alpha=\Norm\circ\gamma$ divides the sphere into regions of equal area.

Conclude that
\[\length \alpha\ge 2\cdot\pi.\]
\end{thm}


\begin{thm}{Exercise}\label{ex:self-intersections}
Let $\Sigma$ be a smooth regular sphere with positive Gauss curvature and $p\in\Sigma$. 
Suppose $\gamma$ is a closed geodesic that is covered by one chart.
Can it happen that in this chart, the curve $\gamma$ looks like one the curves on the following diagrams?

\begin{figure}[h]
\begin{minipage}{.48\textwidth}
\centering
\includegraphics{mppics/pic-46}
\end{minipage}\hfill
\begin{minipage}{.48\textwidth}
\centering
\includegraphics{mppics/pic-47}
\end{minipage}


\medskip

\begin{minipage}{.48\textwidth}
\centering
\caption*{\textit{(a)}}
\end{minipage}\hfill
\begin{minipage}{.48\textwidth}
\centering
\caption*{\textit{(b)}}
\end{minipage}
\vskip-4mm
\end{figure}

\end{thm}


The following exercise gives the optimal bound on Lipschitz constant of a convex function that guarantees that its geodesics have no self-intersections;
compare to \ref{ex:rough-bound-mountain}.

\begin{thm}{Exercise}\label{ex:sqrt(3)}
Suppose that $f\:\RR^2\to\RR$ is a $\sqrt{3}$-Lipschitz smooth convex function.
Show that any geodesic in the surface defined by the graph $z=f(x,y)$ has no self-intersections.
\end{thm}


We start to study the intrinsic geometry of surfaces.
The following exercise should help you to be in the right mood for this;
it might look like a tedious problem in calculus, but actually it is an easy problem in geometry.

\begin{wrapfigure}{r}{33 mm}
\vskip-0mm
\centering
\includegraphics{mppics/pic-77}
\vskip-0mm
\end{wrapfigure}

\begin{thm}{Exercise}\label{ex:lasso}
There is a mountain of frictionless ice with the shape of a perfect cone with a circular base.
A cowboy is at the bottom and he wants to climb the mountain.
So, he throws up his lasso which slips neatly over the top of the cone, he pulls it tight and starts to climb.
If the angle of inclination $\theta$ is large, there is no problem; the lasso grips tight and up he goes.
On the other hand if $\theta$ is small, the lasso slips off as soon as the cowboy pulls on it.

What is the critical angle $\theta_0$ at which the cowboy can no longer climb the ice-mountain?
\end{thm}


\section*{The remarkable theorem}

Let $\Sigma_1$ and $\Sigma_2$ be two smooth regular surfaces in the Euclidean space.
A map $f\:\Sigma_1\to \Sigma_2$ is called  length-preserving if for any curve $\gamma_1$ in $\Sigma_1$ the curve $\gamma_2=f\circ\gamma_1$ in $\Sigma_2$ has the same length. %???it is sufficient to consider smooth only curves???
If in addition $f$ is smooth and bijective, then it is called \emph{intrinsic isometry}. 

A simple example of intrinsic isometry can obtained by warping a plane into a cylinder.
The following exercise produce slightly more interesting example.

\begin{thm}{Exercise}\label{ex:deformation}
Suppose $\gamma(t)=(x(t),y(t))$ is a smooth unit-speed curve in the plane such that $y(t)=a\cdot \cos t$.
Let $\Sigma_\gamma$ be the surface of revolution of $\gamma$ around the $x$-axis.
Show that a small open domain in $\Sigma_\gamma$ admits a smooth length-pereserving map to the unit sphere.

Conclude that any round disc $\Delta$ in $\SS^2$ of intrinsic radius smaller than $\tfrac\pi2$ admits a smooth length preserving deformation; that is, there is one parameter family of surfaces with boundary $\Delta_t$, such that $\Delta_0=\Delta$ and $\Delta_t$ is not congruent to $\Delta_0$ for any $t\ne0$.\footnote{In fact any disc in $\SS^2$ of intrinsic radius smaller than $\pi$ admits a smooth length preserving deformation. %???REF
}
\end{thm}


\begin{thm}{Theorem}\label{thm:remarkable}
Suppose $f\:\Sigma_1\to \Sigma_2$ is an intrinsic isometry between two smooth regular surfaces in  the Euclidean space; $p_1\in \Sigma_1$ and $p_2=f(p_1)\in \Sigma_1$.
Then 
\[G(p_1)_{\Sigma_1}=G(p_2)_{\Sigma_2};\]
that is, the Gauss curvature of $\Sigma_1$ at $p_1$ is the same as the Gauss curvature of $\Sigma_2$ at $p_2$.
\end{thm}

This theorem was proved by Carl Friedrich Gauss \cite{gauss} who called it \emph{Remarkable theorem} (Theorema Egregium).
The theorem is indeed remarkable because the Gauss curvature is defined as a product of principle curvatures which might be different at these points; however, according to the theorem, their product can not change.

In fact Gauss curvature of the surface at the given point can be found \emph{intrinsically},
by measuring the lengths of curves in the surface.
For example, Gauss curvature $G(p)$ in the following formula for the circumference $c(r)$ of a geodesic circle centered at $p$ in a surface: 
\[c(r)=2\cdot\pi\cdot r-\tfrac\pi3\cdot G(p)\cdot r^3+o(r^3).\]

Note that the theorem implies there is no smooth length-preserving map that sends an open region in the unit sphere to the plane.%
\footnote{There are plenty of non-smooth length-preserving maps from the sphere to the plane; see \cite{petrunin-yashinski} and the references there in.}
It follows since the Gauss curvature of the plane is zero and the unit sphere has Gauss curvature 1. 
In other words, there is no map of a region on Earth without distortion.

\parit{Proof.}
Set $g_1=G(p_1)_{\Sigma_1}$ and $g_2=G(p_2)_{\Sigma_2}$;
we need to show that 
\[g_1=g_2.\eqlbl{eq:g=g}\]

Suppose $\Delta_1$ is a small geodesic triangle in $\Sigma_1$ that contains $p_1$.
Set $\Delta_2=f(\Delta_1)$.
We may assume that the Gauss curvature is almost constant in $\Delta_1$ and $\Delta_2$;
that is, given $\eps>0$, we can assume that 
\[
\begin{aligned}
|G(x_1)_{\Sigma_1}-g_1|&<\eps,
\\
|G(x_2)_{\Sigma_2}-g_2|&<\eps
\end{aligned}
\eqlbl{eq:almost=}\]
for any $x_1\in \Delta_1$ and $x_2\in \Delta_2$.

Since $f$ is length-preserving the triangles $\Delta_2$ is geodesic and
\[\area\Delta_1=\area\Delta_2.\eqlbl{eq:area=}\]
Moreover, triangles $\Delta_1$ and $\Delta_2$ have the same corresponding angles; denote them by $\alpha$, $\beta$ and $\gamma$.

By Gauss--Bonnet formula, we get that 
\[\iint_{\Delta_1}G_{\Sigma_1}=\alpha+\beta+\gamma-\pi=\iint_{\Delta_2}G_{\Sigma_2}.\eqlbl{eq:gauss-int=}\]

By \ref{eq:almost=}, 
\begin{align*}
\left|g_1-\frac1{\area\Delta_1}\cdot\iint_{\Delta_1}G_{\Sigma_1}\right|&<\eps,
\\
\left|g_2-\frac1{\area\Delta_2}\cdot\iint_{\Delta_2}G_{\Sigma_2}\right|&<\eps.
\end{align*}
By \ref{eq:area=} and \ref{eq:gauss-int=},
\[\frac1{\area\Delta_1}\cdot\iint_{\Delta_1}G_{\Sigma_1}
=
\frac1{\area\Delta_2}\cdot\iint_{\Delta_2}G_{\Sigma_2},\]
therefore
\[|g_1-g_2|<2\cdot\eps.\]
Since $\eps>0$ is arbitrary, \ref{eq:g=g} follows.
\qeds


\section*{Simple geodesic}

The following theorem provides an interesting application of Gauss--Bonnet formula;
it is proved by Stephan Cohn-Vossen \cite[Satz 9 in][]{convossen}.


\begin{thm}{Theorem}\label{thm:cohn-vossen}
Any open smooth regular surface with positive Gauss curvature has a simple two-sided infinite geodesic.
\end{thm}

\begin{thm}{Lemma}\label{lem:graph}
Suppose $\Sigma$ is an open surface in with positive Gauss curvature in the Euclidean space.
Then there is a convex function $f$ defined on a convex open region of $(x,y)$-plane 
such that $\Sigma$ can be presented as a graph $z=f(x,y)$ in some $(x,y,z)$-coordinate system of the Euclidean space.

Moreover 
\[\iint_\Sigma G\le 2\cdot\pi.\eqlbl{eq:int=<2pi}\]

\end{thm}

\parit{Proof.}
The surface $\Sigma$ is a boundary of an unbounded closed convex set $K$.

Fix $p\in \Sigma$ and consider a sequence of points $x_n$ such that $|x_n-p|\z\to \infty$ as $n\to \infty$.
Set $u_n=\tfrac{x_n-p}{|x_n-p|}$; the unit vector in the direction from $p$ to $x_n$.
Since the unit sphere is compact, we can pass to a subsequence of $(x_n)$ such that $u_n$ converges to a unit vector $u$.

Note that for any $q\in \Sigma$, the directions $v_n=\tfrac{x_n-q}{|x_n-q|}$ converge to $u$ as well.
The half-line from $q$ in the direction of $u$ lies in $K$.
Indeed any point on the half-line is a limit of points on the line segments $[q,x_n]$;
since $K$ is closed, all of these poins lie in $K$.


Let us choose the $z$-axis in the direction of $u$.
Note that line segments can not lie in $\Sigma$, otherwise its Gauss curvature would vanish.
It follows that any vertical line can intersect $\Sigma$ at most at one point.
That is, $\Sigma$ is a graph of a function $z=f(x,y)$.
Since $K$ is convex, the function $f$ is convex and it is defined in a region $\Omega$ which is convex.
The domain $\Omega$ is the projection of $\Sigma$ to the $(x,y)$-plane.
This projection is injective and by the inverse function theorem, it maps open sets in $\Sigma$ to open sets in the plane;
hence $\Omega$ is open.

It follows that the outer normal vectors to $\Sigma$ at any point, points to the south hemisphere $\SS^2_-=\set{(x,y,z)\in\SS^2}{z< 0}$.
Therefore the area of the spherical image of $\Sigma$ is at most $\area\SS^2_-= 2\cdot\pi$.
The area of this image is the integral of the Gauss curvature along $\Sigma$.
That is,
\begin{align*}
\iint_{\Sigma}G&=\area[\Norm(\Sigma)]\le 
\\
&\le \area\SS^2_-=
\\
&=2\cdot\pi,
\end{align*}
where $\Norm(p)$ denotes the outer unit normal vector at $p$.
Hence \ref{eq:int=<2pi} follows.
\qeds

\parit{Proof of \ref{thm:cohn-vossen}.}
Let $\Sigma$ be an open surface in with positive Gauss curvature and $\gamma$ a two-sided infinite geodesic in $\Sigma$.
The following is the key statement in the proof.

\begin{thm}{Claim}
The geodesic $\gamma$ contains at most one simple loop.
\end{thm}

Assume $\gamma$ has a simple loop $\ell$.
By Lemma \ref{lem:graph}, $\Sigma$ is parameterized by a open convex region $\Omega$ in the plane;
therefore $\ell$ bounds a disc in $\Sigma$; denote it by $\Delta$.
If $\phi$ is the angle at the base of the loop, then by Gauss--Bonnet,
\[\iint_\Delta G=\pi+\phi.\] 
By Lemma \ref{lem:graph}, $\phi<\pi$; that is, $\gamma$ has no concave simple loops 

Assume $\gamma$ has two simple loops, say $\ell_1$ and $\ell_2$ that bound discs $\Delta_1$ and $\Delta_2$.
Then the disks $\Delta_1$ and $\Delta_2$ have to overlap,
otherwise the curvature of $\Sigma$ would exceed $2\cdot\pi$.

We may assume that $\Delta_1\not\subset \Delta_2$; the loop $\ell_2$ appears after $\ell_1$ on $\gamma$ and there are no other simple loops between them.
In this case, after going around $\ell_1$ and before closing $\ell_2$, the curve $\gamma$ must enter $\Delta_1$ creating a concave loop.
The latter contradicts the above observation.

If a geodesic $\gamma$ has a self-intersection,
then it contains a simple loop.
From above, there is only one such loop;
it cuts a disk from $\Sigma$ 
and goes around it either clockwise or counterclockwise.
This way we divide all the self-intersecting geodesics 
into two sets which we will call {}\emph{clockwise} and {}\emph{counterclockwise}.

Note that the geodesic $t\mapsto \gamma(t)$ is clockwise 
if and only if the same geodesic traveled backwards
$t\mapsto \gamma(-t)$
is counterclockwise.
By shooting unit-speed geodesics in all directions at a given point $p=\gamma(0)$,
we get a one parameter family of geodesics $\gamma_s$ for $s\in[0,\pi]$ connecting the geodesic $t\mapsto \gamma(t)$ with
the $t\mapsto \gamma(-t)$; that is, $\gamma_0(t)\z=\gamma(t)$ and $\gamma_\pi(t)=\gamma(-t)$. 
It follows that there are geodesics 
which aren't clockwise nor counterclockwise.
Those geodesics have no self-intersections.\qeds
