\chapter{Curvatures}
\label{chap:surface-curvature}

\section{Tangent-normal coordinates} \label{sec:lmn}

Fix a point $p$ in a smooth oriented surface $\Sigma$.
Consider a coordinate system $(x,y,z)$ with origin at $p$ such that the $(x,y)$-plane coincides with $\T_p$ and the $z$-axis points in the direction of the normal vector $\Norm(p)$.
By \ref{ex:vertical-tangent}, we can present $\Sigma$ locally at $p$ as a graph $z=f(x,y)$ of a smooth function. 
Note that 
\begin{align*}
f(0,0)&=0,
&
f_x(0,0)&=0,
&
f_y(0,0)&=0.
\end{align*}
The first equality holds since $p=(0,0,0)$ lies on the graph and the other two equalities mean that the tangent plane at $p$ is horizontal.


\begin{wrapfigure}[7]{o}{42 mm}
\vskip-4mm
\centering
\includegraphics{asy/paraboloid}
\vskip-3mm
\end{wrapfigure}

Set 
\begin{align*}
\ell&=f_{xx}(0,0),
\\
m&=f_{xy}(0,0)=f_{yx}(0,0),
\\
n&=f_{yy}(0,0).
\end{align*}
The \index{Taylor series}\emph{Taylor series} 
for $f$ at $(0,0)$ up to the second order term can be then written as
\[f(x,y)=\tfrac12(\ell\cdot x^2+2\cdot m\cdot x\cdot y+n\cdot y^2)+o(x^2+y^2).\]
Note that values $\ell$, $m$, and $n$ are completely determined by this equation.\index{10lmn@$\ell$, $m$, $n$}
The so-called \index{osculating!paraboloid}\emph{osculating paraboloid}
\[z=\tfrac12\cdot(\ell\cdot x^2+2\cdot m\cdot x\cdot y+n\cdot y^2)\]
has \index{order of contact}\emph{second order of contact} with $\Sigma$ at $p$.

Note that 
\[\ell\cdot x^2+2\cdot m\cdot x\cdot y+n\cdot y^2=\langle M_p\cdot (\begin{smallmatrix}
x\\y
\end{smallmatrix}), (\begin{smallmatrix}
x\\y
\end{smallmatrix})\rangle,\]
where $M_p$ is the so-called \index{Hessian matrix}\emph{Hessian matrix} of $f$ at $(0,0)$,\index{10m@$M_p$}
\[M_p=\begin{pmatrix}
   \ell
   &m
   \\
   m
   &n
  \end{pmatrix}.
\eqlbl{eq:hessian}
\]


\section{Principal curvatures}

Note that tangent-normal coordinates give an almost canonical coordinate system in a neighborhood of $p$;
it is unique up to a rotation of  the $(x,y)$-plane.
Rotating the $(x,y)$-plane results in rewriting   
the matrix $M_p$ in the new basis.

Since the Hessian matrix $M_p$ is symmetric, by the spectral theorem (\ref{thm:spectral}) it is diagonalizable by orthogonal matrices.
That is, by rotating the $(x,y)$-plane we can assume that $m=0$ in \ref{eq:hessian}; see \ref{thm:spectral}.
In this case
\[M_p=\begin{pmatrix}
   k_1
   &0
   \\
   0
   &k_2
  \end{pmatrix},
\]
the diagonal components $k_1$ and $k_2$ of $M_p$ are called the \index{principal curvatures and directions}\emph{principal curvatures} of $\Sigma$ at $p$;\index{10k@$k_1$, $k_2$}
they might also be denoted as $k_1(p)$ and $k_2(p)$, or $k_1(p)_\Sigma$ and $k_2(p)_\Sigma$;
if we need to emphasize that we compute them at the point $p$ for the surface $\Sigma$.
We will always assume that $k_1\le k_2$.


Note that if $x=f(x,y)$ is a local graph representation of $\Sigma$ in these coordinates, then 
\[f(x,y)=\tfrac12\cdot(k_1\cdot x^2+k_2\cdot y^2)+o(x^2+y^2).\]

The principal curvatures can be also defined as the eigenvalues of the Hessian matrix $M_p$.
The eigendirections of $M_p$  are called the {}\emph{principal directions} of $\Sigma$ at~$p$.
Note that if $k_1(p)\ne k_2(p)$, then $p$ has exactly two principal directions, which are perpendicular to each other; if $k_1(p)\z= k_2(p)$ then all tangent directions at $p$ are principal.

Note that if we revert the orientation of $\Sigma$, then the principal curvatures switch their signs and indexes.

A smooth regular curve on a surface $\Sigma$ that always runs in the principal directions is called a \index{line of curvature}\emph{line of curvature} of $\Sigma$.  

\begin{thm}{Exercise}\label{ex:line-of-curvature}
Assume that a smooth surface $\Sigma$ is mirror symmetric with respect to  a plane $\Pi$.
Suppose that $\Sigma$ and $\Pi$ intersect along a smooth regular curve~$\gamma$.
Show that $\gamma$ is a line of curvature of $\Sigma$.
\end{thm}

\section{More curvatures}\label{sec:More curvatures}

Let $p$ be a point on an oriented smooth surface $\Sigma$.

The product 
\[K(p)=k_1(p)\cdot k_2(p)\]
is called the \index{10k@$K$}\index{Gauss curvature}\emph{Gauss curvature} at $p$.
We may denote it by $K$, $K(p)$, or $K(p)_\Sigma$ if we need to specify the point $p$ and the surface $\Sigma$.

Since the determinant is equal to the product of the eigenvalues, we get
\[K=\ell\cdot n-m^2,\]
where 
$M_p\z=
(\begin{smallmatrix}
\ell&m
\\
m&n
\end{smallmatrix}
)
$ is the Hessian matrix.

\begin{thm}{Exercise}\label{ex:gauss+orientable}
Show that any surface with positive Gauss curvature is orientable. 
\end{thm}

The sum \index{10h@$H$}
\[H(p)=k_1(p)+ k_2(p)\] 
is called the \index{mean curvature}\emph{mean curvature}\footnote{Some authors define the mean curvature as $\tfrac12\cdot(k_1(p)+ k_2(p))$ --- the mean value of the principal curvatures. It suits the name better, but it is not as convenient when it comes to calculations.} at $p$.
We may also denote it by $H(p)_\Sigma$.
The mean curvature can be also interpreted as the trace of the Hessian matrix $M_p\z=
(\begin{smallmatrix}
\ell&m
\\
m&n
\end{smallmatrix}
)$;
that is,
\[H=\ell+n\] 

A surface with vanishing mean curvature is called \index{minimal surface}\emph{minimal}.

Note that reverting the orientation of $\Sigma$ does not change the Gauss curvature, but changes the sign of the mean curvature.
In particular, the Gauss curvature is well defined  for a nonoriented surfaces.

\section{Shape operator}

In the following definitions we use the notion of directional derivative defined in \ref{def:directional-derivative}.

Let $p$ be a point on a smooth surface $\Sigma$ with orientation defined by the unit normal field $\Norm$.
Given $\vec w\in \T_p$,
its \emph{shape operator} is defined by
\[\Shape_p\vec w=-D_{\vec w}\Norm.\]
Equivalently, the shape operator can be defined by
\[\Shape=-d\Norm,\eqlbl{eq:shape=-L}\] 
where $d\Norm$ denotes the differential of the spherical map $\Norm\:\Sigma\to\mathbb{S}^2$; that is, $d_p\Norm(\vec v)=(D_{\vec v}\Norm)(p)$.

Recall that $d_p\Norm$ is a linear map $\T_p\Sigma\to \T_{\Norm(p)}\SS^2$.
Note that $\T_p\Sigma$ coincides with $\T_{\Norm(p)}\SS^2$ --- both of them are normal subspaces to $\Norm(p)$.
Therefore $\Shape_p$ is indeed a linear operator $\T_p\to \T_p$ (the latter also follows from \ref{thm:shape-chart}).

For a point $p\in \Sigma$ the shape operator of a tangent vector $\vec w\in \T_p$ will be denoted by $\Shape\vec w$ if it is clear from the context which base point $p$ and which surface we are working with;
otherwise we may use the notations 
\[\Shape_p(\vec w)\quad\text{or}\quad \Shape_p(\vec w)_\Sigma\]
as the situation requires.  

\begin{thm}{Theorem}\label{thm:shape-chart}
Suppose that $(u,v)\mapsto s(u,v)$ is a smooth map to a smooth surface $\Sigma$ with unit normal field $\Norm$.
Then 
\begin{align*}
\langle \Shape(s_u), s_u\rangle 
&=\langle s_{uu},\Norm\rangle,
&
\langle \Shape(s_v), s_u\rangle 
&=\langle s_{uv},\Norm\rangle,
\\
\langle \Shape(s_u), s_v\rangle 
&=\langle s_{uv},\Norm\rangle,
&
\langle \Shape(s_v), s_v\rangle 
&=\langle s_{vv},\Norm\rangle,
\\
\langle \Shape(s_u), \Norm\rangle 
&=0,
&
\langle \Shape(s_v), \Norm\rangle 
&=0
\end{align*}
for any $(u,v)$.

\end{thm}

\parit{Proof.}
We will use the shortcut $\Norm=\Norm(u,v)$ for $\Norm(s(u,v))$,
so 
\[
\begin{aligned}
\Shape(s_u)&=-D_{s_u}\Norm=-\Norm_u,
&
\Shape(s_v)&=-D_{s_v}\Norm=-\Norm_v.
\end{aligned}
\eqlbl{eq:shape=norm_u}
\]

Note that $\Norm$ is a unit vector orthogonal to $s_u$ and $s_v$;
therefore
\begin{align*}
\langle \Norm,s_u\rangle&\equiv0,
&
\langle \Norm,s_v\rangle&\equiv0,
&
\langle \Norm,\Norm\rangle&\equiv1.
\end{align*}
Taking partial derivatives of these two identities we get
\begin{align*}
\langle \Norm_u,s_u\rangle+\langle \Norm,s_{uu}\rangle&=0,
&
\langle \Norm_v,s_u\rangle+\langle \Norm,s_{uv}\rangle&=0,
\\
\langle \Norm_u,s_v\rangle+\langle \Norm,s_{uv}\rangle&=0,
&
\langle \Norm_v,s_v\rangle+\langle \Norm,s_{vv}\rangle&=0,
\\
2\cdot\langle \Norm_u,\Norm\rangle&=0,
&
2\cdot\langle \Norm_v,\Norm\rangle&=0.
\end{align*}
It remains to plug in the expressions from \ref{eq:shape=norm_u}.
\qeds

\begin{thm}{Exercise}\label{ex:self-adjoint}
Show that the shape operator is \index{self-adjoint operator}\emph{self-adjoint}; that is,
\[\langle \Shape\vec u,\vec v\rangle=\langle \vec u,\Shape\vec v\rangle\]
for any $\vec u,\vec v\in\T_p$.
\end{thm}

Let us denote by $\vec i$, $\vec j$ and $\vec k$ the standard basis in the $(x,y,z)$-coordinates.\index{10i@$\vec i$, $\vec j$, $\vec k$}
Recall that the components $\ell$, $m$, and $n$ of the Hessian matrix are defined in Section~\ref{sec:lmn}.

\begin{thm}{Corollary}\label{cor:Shape(ij)}
Let $z=f(x,y)$ be a local representation of a smooth surface $\Sigma$ in the tangent-normal coordinates at $p$.
Suppose that its the Hessian matrix at $p$ is $(\begin{smallmatrix}
\ell&m\\ m&n
\end{smallmatrix})$.
Then 
\begin{align*}
\Shape\vec i&=\ell\cdot \vec i+m\cdot \vec j,
&
\Shape\vec j&=m\cdot \vec i+n\cdot\vec j;
\end{align*}
that is, the multiplication by the Hessian matrix at $p$ describes its shape operator.

In particular, the principal curvatures of $\Sigma$ at $p$ are the eigenvalues of $\Shape_p$ and the principal directions are the eigendirections of $\Shape_p$.
\end{thm}

Since the Hessian matrix is symmetric, the corollary also implies that $\Shape$ is self-adjoint which gives another way to solve \ref{ex:self-adjoint}.

\parit{Proof.}
Note that $s\:(u,v)\mapsto (u,v,f(u,v))$ is a chart of $\Sigma$ that covers $p$.
Further note that 
\begin{align*}
s_u(0,0)&=\vec i,
&
s_v(0,0)&=\vec j,
&
\Norm(0,0)&=\vec k,
\\
s_{uu}(0,0)&=\ell\cdot \vec k,
&
s_{uv}(0,0)&=m\cdot \vec k,
&
s_{vv}(0,0)&=n\cdot \vec k.
\end{align*}
It remains to apply \ref{thm:shape-chart}.
\qeds

\begin{thm}{Corollary}\label{cor:intK}
Let $\Sigma$ be a smooth surface with orientation defined by a unit normal field $\Norm$.
Suppose the spherical map $\Norm\:\Sigma\to\mathbb{S}^2$ is injective.
Then 
\[\iint_\Sigma|K|=\area[\Norm(\Sigma)].\]
\end{thm}

\parit{Proof.}
Observe that the tangent planes $\T_p\Sigma=\T_{\Norm(p)}\mathbb{S}^2$ are parallel for any $p\in\Sigma$.
Indeed both of these planes are perpendicular to $\Norm(p)$. 


Choose an orthonormal basis of $\T_p$ consisting of principal directions,
so the shape operator can be expressed by the matrix 
$(\begin{smallmatrix}
   k_1
   &0
   \\
   0
   &k_2
  \end{smallmatrix})$.

Since $\Shape_p=-d_p\Norm$, \ref{cor:Shape(ij)} implies that
\[\jac_p\Norm=|\det(\begin{smallmatrix}
   k_1
   &0
   \\
   0
   &k_2
  \end{smallmatrix})|=|K(p)|.\]
By the area formula (\ref{prop:surface-integral}), the statement follows.
\qeds


\begin{thm}{Exercise}\label{ex:normal-curvature=const}
Let $\Sigma$ be a smooth surface with orientation defined by a unit normal field $\Norm$.
Suppose that $\Sigma$ has unit principal curvatures at any point.

\begin{subthm}{ex:normal-curvature=const:a} Show that $\Shape_p(\vec w)=\vec w$ for any $p\in\Sigma$ and $\vec w\in \T_p\Sigma$.
\end{subthm}

\begin{subthm}{ex:normal-curvature=const:b}Show that $p+\Norm(p)$ is constant; that is, the point $c=p+\Norm(p)$ does not depend on $p\in\Sigma$.
Conclude that $\Sigma$ is a subset of the unit sphere centered at $c$.
\end{subthm}

\end{thm}

We define the \index{angle between surfaces}\emph{angle} between two oriented surfaces at a point of their intersection $p$ as the angle between their normal vectors at $p$.
The following exercise is a result by Ferdinand Joachimsthal \cite{joachimsthal} generalized by Ossian Bonnet \cite{bonnet}.

\begin{thm}{Exercise}\label{ex:shape-curvature-line}
Assume that two smooth oriented surfaces $\Sigma_1$ and $\Sigma_2$ intersect at constant angle along a smooth regular curve $\gamma$.
Show that if $\gamma$ is a curvature line in $\Sigma_1$, then it is also a curvature line in $\Sigma_2$.

Conclude that if a smooth surface $\Sigma$ intersects a plane or sphere at constant angle along a smooth regular curve $\gamma$,
then $\gamma$ is a curvature line of~$\Sigma$.
\end{thm}

\begin{thm}{Exercise}\label{ex:equidistant}
Let $\Sigma$ be a closed smooth surface with orientation defined by a unit normal field $\Norm$.
Assume $t$ is sufficiently close to zero.

\begin{subthm}{ex:equidistant:smooth}
Show that the set of points of the form $p+t\cdot \Norm(p)$ with $p\in\Sigma$ forms a smooth surface; denote it by $\Sigma_t$.
\end{subthm}

\begin{subthm}{ex:equidistant:area}
Let $a(t)$ be the area of $\Sigma_t$.
Show that 
\[a'(0)=-\iint_\Sigma H,\]
where $H$ denotes mean curvature of $\Sigma$.
\end{subthm}



\end{thm}

