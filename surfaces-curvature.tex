\chapter{Curvatures}

\section{Tangent-normal coordinates} \label{sec:lmn}

Fix a point $p$ in a smooth oriented surface $\Sigma$.
Consider a coordinate system $(x,y,z)$ with origin at $p$ such that the $(x,y)$-plane coincides with $\T_p$ and the $z$-axis in the direction of the normal vector $\Norm_p$.
By \ref{ex:vertical-tangent}, we can present $\Sigma$ locally around $p$ as a graph of a smooth function~$f$. 
Note that $f$ satisfies the following additional properties:
\begin{align*}
f(0,0)&=0,
&
f_x(0,0)&=0,
&
f_y(0,0)&=0.
\end{align*}
The first equality holds since $p=(0,0,0)$ lies on the graph and the last two equalities mean that the tangent plane at $p$ is horizontal.

\begin{wrapfigure}[7]{o}{42 mm}
\vskip-4mm
\centering
\includegraphics{asy/paraboloid}
\vskip-3mm
\end{wrapfigure}

Set 
\begin{align*}
\ell&=f_{xx}(0,0),
\\
m&=f_{xy}(0,0)=f_{yx}(0,0),
\\
n&=f_{yy}(0,0).
\end{align*}
The \index{Taylor series}\emph{Taylor series} %???Taylor series
for $f$ at $(0,0)$ up to the second order term can be then written as
\[f(x,y)=\tfrac12(\ell\cdot x^2+2\cdot m\cdot x\cdot y+n\cdot y^2)+o(x^2+y^2).\]
Note that values $\ell$, $m$, and $n$ are completely determined by this equation.
The so-called \index{osculating paraboloid}\emph{osculating paraboloid}
\[z=\tfrac12\cdot(\ell\cdot x^2+2\cdot m\cdot x\cdot y+n\cdot y^2)\]
gives the best approximation of the surface at $p$;
it has \index{order of contact}\emph{second order of contact} with $\Sigma$ at $p$.

Note that 
\[\ell\cdot x^2+2\cdot m\cdot x\cdot y+n\cdot y^2=\langle M_p\cdot (\begin{smallmatrix}
x\\y
\end{smallmatrix}), (\begin{smallmatrix}
x\\y
\end{smallmatrix})\rangle,\]
where
\[M_p=\begin{pmatrix}
   \ell
   &m
   \\
   m
   &n
  \end{pmatrix};
\eqlbl{eq:hessian}
\]
it is called the \index{Hessian matrix}\emph{Hessian matrix} of $f$ at $(0,0)$.\index{$M_p$}

\section{Principle curvatures}

Note that tangent-normal coordinates give an almost canonical coordinate system in a neighborhood of $p$;
it is unique up to a rotation of  the $(x,y)$-plane.
Rotating the $(x,y)$-plane results in the rewriting   
the matrix $M_p$ in the new basis.

Since the Hessian matrix $M_p$ is symmetric, it is diagonalizable by orthogonal matrices.
That is, by rotating the $(x,y)$-plane we can assume that $m=0$ in \ref{eq:hessian}; see \ref{thm:spectral}.
In this case
\[M_p=\begin{pmatrix}
   k_1
   &0
   \\
   0
   &k_2
  \end{pmatrix},
\]
the diagonal components $k_1$ and $k_2$ of $M_p$ are called \index{principle curvatures}\emph{principle curvatures} of $\Sigma$ at $p$;\index{$k_1$}\index{$k_2$}
they might also be denoted as $k_1(p)$ and $k_2(p)$, or $k_1(p)_\Sigma$ and $k_2(p)_\Sigma$;
if we need to emphasise that we calculate it at the point $p$ for the surface $\Sigma$.
We will always assume that $k_1\le k_2$.


Note that if $x=f(x,y)$ is a local graph representation of $\Sigma$ in these coordinates, then 
\[f(x,y)=\tfrac12\cdot(k_1\cdot x^2+k_2\cdot y^2)+o(x^2+y^2).\]

The principle curvatures can be also defined as the eigenvalues of the Hessian matrix $M_p$.
The eigendirections of $M_p$  are called \index{principle directions}\emph{principle directions} of $\Sigma$ at~$p$.
Note that if $k_1(p)\ne k_2(p)$, then $p$ has exactly two principle directions, which are perpendicular to each other; if $k_1(p)\z= k_2(p)$ then all tangent directions at $p$ are principle.

Note that if we revert the orientation of $\Sigma$, then the principle curvatures at each point switch their signs and indexes.

A smooth regular curve on a surface $\Sigma$ that always runs in the principle directions is called a \index{line of curvature}\emph{line of curvature} of $\Sigma$.  

\begin{thm}{Exercise}\label{ex:line-of-curvature}
Assume that a smooth surface $\Sigma$ is mirror symmetric with respect to  a plane $\Pi$.
Suppose that $\Sigma$ and $\Pi$ intersect along a curve $\gamma$.
Show that $\gamma$ is a line of curvature of $\Sigma$.
\end{thm}


\section{More curvatures}

Fix an oriented smooth surface $\Sigma$ and a point $p\in\Sigma$.

The product 
\[K=k_1(p)\cdot k_2(p)\]
is called \index{$K$}\index{Gauss curvature}\emph{Gauss curvature} at $p$.
We may denote it by $K(p)$ or $K(p)_\Sigma$ if we need to specify the point $p$ and the surface $\Sigma$.

Since the product of principle values equals to determinant,
the Gauss curvature equals to the determinant of the Hessian matrix 
$M_p\z=
(\begin{smallmatrix}
\ell&m
\\
m&n
\end{smallmatrix}
)
$;
that is,
\[K=\ell\cdot n-m^2.\]

\begin{thm}{Exercise}\label{ex:gauss+orientable}
Show that any surface with positive Gauss curvature is orientable. 
\end{thm}

The sum \index{$H$}
\[H(p)=k_1(p)+ k_2(p)\] 
is called \index{mean curvature}\emph{mean curvature}\footnote{Some authors define mean curvature as $\tfrac12\cdot(k_1(p)+ k_2(p))$ --- the mean value of the principle curvatures. It is suits the name better, but not as convenient when it comes to calculations.} at $p$.
We may also denote it by $H(p)_\Sigma$.
The mean curvature can be also interpreted as the trace of the Hessian matrix $M_p$. 

Note that reverting orientaion of $\Sigma$ does not change Gauss curvatre and changes sign of mean curvature.
In particular, the Gauss curvature is well defined for nonoriented surface $\Sigma$ and $p$.

A surface with vanishing mean curvature is called \index{minimal surface}\emph{minimal};
the reason for such a name will be given in Proposition~\ref{prop:minimizing-is-minimal}.

\section{Shape operator}

Let $p$ be a point on a smooth oriented surface $\Sigma$.
Suppose $z=f(x,y)$ is a local description of $\Sigma$ in a tangent-normal coordinates at $p$
and 
\[M_p=\begin{pmatrix}
   \ell
   &m
   \\
   m
   &n
  \end{pmatrix};
\]
is the Hessian matrix of $f$ at $(0,0)$; the components $\ell$, $m$, and $n$ are defined in Section~\ref{sec:lmn}.

The multiplication by the Hessian matrix defies the so called \index{shape operator}\emph{shape operator}
\[\Shape\:(\begin{smallmatrix}
x\\y
\end{smallmatrix})
\mapsto
M_p\cdot(\begin{smallmatrix}
x\\y
\end{smallmatrix});\]\index{$\Shape$}
it is a linear operator $\Shape\:\T_p\to \T_p$.
For a point $p\in \Sigma$ the shape operator of a tangent vector $\vec w\in \T_p$ will be denoted by $\Shape(\vec w)$ if it is clear from the context which base point $p$ and which surface we are working with;
otherwise we may use notations 
\[\Shape_p(\vec w)\quad\text{or even}\quad \Shape_p(\vec w)_\Sigma.\]


Since $M_p$ is symmetric, $\Shape$ is \index{self-adjoint operator}\emph{self-adjoint}; that is
\[\langle \Shape(\vec v),\vec w\rangle=\langle \vec v,\Shape(\vec w)\rangle\]
for any $\vec v,\vec w\in\T_p$.
Note also that principle curvatures of $\Sigma$ at $p$ are the eigenvalues of $\Shape_p$ and the principle directions are the directions of principle vectors of $\Shape_p$.

The shape operator $\Shape_p$ is defined using the Hessian matrix $M_p$, that depends on the choice of basis in $\T_p$;
the following proposition implies in particular that $\Shape_p$ does not depend on the choice of basis. 

\begin{thm}{Proposition}\label{prop:shape=D2}
Let $p$ be a point on a smooth oriented surface $\Sigma$.
Suppose $\Sigma$ is described locally as a graph $z=f(x,y)$ in a tangent-normal coordinates at $p$.
Then
\[\langle \Shape(\vec v),\vec w\rangle=D_{\vec w}D_{\vec v}f(0,0)\]
for any $\vec v,\vec w\in\T_p$.
Moreover $\Shape\:\T_p\to\T_p$ is a unique linear operator that satisfies the above condition.
\end{thm}

Here $D_{\vec v}f$ denoted directional derivative of $f$ along vector $\vec v$;
that is, if $\phi(t)=f(q+\vec v\cdot t)$, then $D_{\vec v}f(q)=\phi'(0)$.

\parit{Proof.} 
Suppose $\vec v=(\begin{smallmatrix}
a\\b
\end{smallmatrix})$
and 
$\vec v=(\begin{smallmatrix}
c\\d
\end{smallmatrix})$, then 
\begin{align*}
D_{\vec v}&=a\cdot\tfrac{\partial}{\partial x}+ b\cdot\tfrac{\partial}{\partial y},
&
D_{\vec w}&=c\cdot\tfrac{\partial}{\partial x}+ d\cdot\tfrac{\partial}{\partial y}.
\end{align*}
Therefore 
\begin{align*}
D_{\vec w}D_{\vec v}f&=
a\cdot c\cdot\tfrac{\partial^2 f}{\partial^2 x}
+b\cdot c\cdot\tfrac{\partial^2 f}{\partial x\partial y}
+a\cdot d\cdot\tfrac{\partial^2 f}{\partial y\partial x}
+b\cdot d\cdot\tfrac{\partial^2 f}{\partial^2 y}
\intertext{evaluating this expression at $(0,0)$ we get}
D_{\vec w}D_{\vec v}f(0,0)&=a\cdot c\cdot\ell
+b\cdot c\cdot m
+a\cdot d\cdot m
+b\cdot d\cdot n=
\\
&=\langle M_p\cdot \vec v,\vec w\rangle=\langle \vec v,M_p\cdot \vec w\rangle=
\\
&=\langle \Shape(\vec v),\vec w\rangle=\langle \vec v,\Shape(\vec w)\rangle.
\end{align*}
\qedsf

\begin{thm}{Corollary}\label{cor:S(ij)}
Let  $p$ be a point on a smooth oriented surface $\Sigma$.
Suppose $\Sigma$ is described locally as a graph $z=f(x,y)$ in a tangent-normal coordinates at $p$.
Denote by $\vec i$, $\vec j$ and $\vec k$ the standard basis in the $(x,y,z)$-coordinates.
Then
\begin{align*}
\langle \Shape(\vec i),\vec i\rangle&=\ell,
&
\langle \Shape(\vec i),\vec j\rangle&=m,
&
\langle \Shape(\vec i),\vec k\rangle&=0,
\\
\langle \Shape(\vec j),\vec i\rangle&=m,
&
\langle \Shape(\vec j),\vec j\rangle&=n,
&
\langle \Shape(\vec j),\vec k\rangle&=0,
\end{align*}
where $\ell$, $m$, and $n$ are the components of the Hessian matrix of $f$ at $(0,0)$ defined in Section~\ref{sec:lmn}.
\end{thm}

\parit{Proof.} 
Note that 
$D_{\vec i}=\tfrac{\partial}{\partial x}$ and $D_{\vec j}=\tfrac{\partial}{\partial y}$.
It remains to use \ref{prop:shape=D2} and the expressions for $\ell$, $m$, and $n$; see Section~\ref{sec:lmn}.
\qeds

In the following proposition we use the notion of directional derivative defined in \ref{def:directional-derivative}.

\begin{thm}{Proposition}\label{prop:S=-D}
Let $\Sigma$ be a smooth surface with unit normal field $\Norm$.
Suppose $p\in \Sigma$ and $\Shape\:\T_p\to\T_p$ is the shape operator at $p$.
Then 
\[\Shape(\vec w)=-D_{\vec w}\Norm
\eqlbl{eq:shape=-dNorm}\]
for any $\vec w\in \T_p$.
Equivalently 
\[\Shape=-d_p\Norm,\eqlbl{eq:shape=-L}\] 
where $d_p\Norm$ denotes differential of the spherical map $\nu\:\Sigma\to\mathbb{S}^2$ at $p$; that is, $d_p\Norm(\vec v)=(D_{\vec v}\Norm)(p)$.
\end{thm}



The reason for minus sign in \ref{eq:shape=-dNorm} and \ref{eq:shape=-L}is the same as in the formula  
for curvature of plane curve in its Frenet frame: $\norm'=-\skur\cdot\tan$.
The proof is done by straightforward calculations.


\parit{Proof of \ref{prop:S=-D}.}
Let $\Sigma$ be a smooth surface with unit normal field $\Norm$.
Suppose $(u,v)\mapsto s(u,v)$ is a local chart of $\Sigma$ at $p$.
Denote by $\Norm(u,v)$ the unit normal vector at $s(u,v)$.

Since the vectors $s_u(u,v)$ and $s_v(u,v)$ are tangent to $\Sigma$ at $s(u,v)$ and $\Norm(u,v)$ is a unit normal vector, we get that
\begin{align*}
1&=\langle\Norm,\Norm\rangle,
&
0&=\langle s_u,\Norm\rangle,
&
0&=\langle s_v,\Norm\rangle.
\end{align*}

Taking partial derivatives of these there identities and applying the product rule,
we get the following six:
\begin{align*}
0&=\tfrac{\partial}{\partial u}\langle\Norm ,\Norm \rangle
=
2\cdot \langle\Norm_u,\Norm\rangle,
\\
0
&=\tfrac{\partial}{\partial v}\langle\Norm,\Norm\rangle
=
2\cdot \langle\Norm_v,\Norm\rangle,
\\
0&=\tfrac{\partial}{\partial u}
\langle s_u,\Norm\rangle
=
\langle s_{uu},\Norm\rangle
+
\langle s_u,\Norm_u\rangle,
\\
0&=
\tfrac{\partial}{\partial v}\langle s_u,\Norm\rangle
=
\langle s_{uv},\Norm\rangle
+
\langle s_u,\Norm_v\rangle
\\
0
&=
\tfrac{\partial}{\partial u}
\langle s_v,\Norm\rangle
=
\langle s_{uv},\Norm\rangle
+
\langle s_v,\Norm_u\rangle,
\\
0
&=
\tfrac{\partial}{\partial v}
\langle s_v,\Norm\rangle
=
\langle s_{vv},\Norm\rangle
+
\langle s_v, \Norm_v\rangle.
\end{align*}

Now suppose $z=f(x,y)$ be a local description of $\Sigma$ in the tangent-normal coordinates at $p$.
Note that 
\[s(u,v)=(u,v,f(u,v))\]
describes a chart of $\Sigma$ at $p$.

Denote by $\vec i$, $\vec j$ and $\vec k$ the standard basis in the $(x,y,z)$-coordinates.
Note that $s(0,0)=p$ and 
\begin{align*}
s_u(0,0)&=\vec i,
&
s_v(0,0)&=\vec j,
&
\nu(0,0)&=\vec k,
\intertext{In particular $D_{\vec i}\Norm=\Norm_u(0,0)$ and $D_{\vec j}\Norm=\Norm_v(0,0)$. Further,}
s_{uu}(0,0)&=\ell\cdot \vec k,
&
s_{uv}(0,0)&=m\cdot \vec k,
&
s_{vv}(0,0)&=n\cdot \vec k,
\end{align*}
where $\ell$, $m$, and $n$ are the components of the Hessian matrix of $f$ at $(0,0)$; see Section~\ref{sec:lmn}.

Evaluating the above 6 identities at $(u,v)\z=(0,0)$, we get that
\begin{align*}
\langle -D_{\vec i}\nu ,\vec i\rangle&=\ell,
&
\langle -D_{\vec i},\vec j\rangle&=m,
&
\langle -D_{\vec i}\nu,\vec k\rangle&=0,
\\
\langle -D_{\vec j}\nu,\vec i\rangle&=m,
&
\langle -D_{\vec j}\nu,\vec j\rangle&=n,
&
\langle -D_{\vec j}\nu,\vec k\rangle&=0,
\end{align*}
That is, $-D_{\vec i}\nu$ and $-D_{\vec j}\nu$ satisfy the same equalities as $\Shape(\vec i)$ and $\Shape(\vec j)$ in \ref{cor:S(ij)}.
Since these equalities define $S$ completely, \ref{eq:shape=-dNorm} follows.
\qeds

\begin{thm}{Corollary}\label{cor:intK}
Let $\Sigma$ be a smooth surface with orientation defined by unit normal field $\Norm$.
Suppose the spherical map $\Norm\:\Sigma\to\mathbb{S}^2$ is injective.
Then 
\[\iint_\Sigma|K|=\area[\nu(\Sigma)].\]
\end{thm}

\parit{Proof.}
Observe that the tangent planes $\T_p\Sigma=\T_{\Norm(p)}\mathbb{S}^2$ are parallel for any $p\in\Sigma$.
Indeed both of these planes are perpendicular to $\Norm(p)$. 


Choose an orthonormal basis of $\T_p$ with in principle directions,
so the shape operator can be expressed by matrix 
$(\begin{smallmatrix}
   k_1
   &0
   \\
   0
   &k_2
  \end{smallmatrix})$.

Since by \ref{prop:S=-D}, $\Shape_p=-d_p\Norm$, we have
\[\jac_p\Norm=|\det(\begin{smallmatrix}
   k_1
   &0
   \\
   0
   &k_2
  \end{smallmatrix})|=|K(p)|.\]
By \ref{prop:surface-integral}, the statement follows.
\qeds


\begin{thm}{Exercise}\label{ex:shape-chart}
Suppose that $(u,v)\mapsto s(u,v)$ is a chart of a smooth surface $\Sigma$ with unit normal field $\Norm$.
Show that 
\begin{align*}
\langle \Shape(s_u), s_u\rangle 
&=\langle s_{uu},\Norm\rangle,
&
\langle \Shape(s_u), s_v\rangle 
&=\langle s_{uv},\Norm\rangle,
\\
\langle \Shape(s_v), s_u\rangle 
&=\langle s_{uv},\Norm\rangle,
&
\langle \Shape(s_v), s_v\rangle 
&=\langle s_{vv},\Norm\rangle
\end{align*}
for any $(u,v)$.

\end{thm}

\begin{thm}{Exercise}\label{ex:normal-curvature=const}
Let $\Sigma$ be a smooth oriented surface with the unit normal field $\Norm$.
Suppose that $\Sigma$ has unit principle curvatures at any point.

\begin{subthm}{ex:normal-curvature=const:a} Show that $\Shape_p(\vec w)=\vec w$ for any $p\in\Sigma$ and $\vec w\in \T_p\Sigma$.
\end{subthm}

\begin{subthm}{ex:normal-curvature=const:b}Show that $p+\Norm_p$ is constant; that is, the point $c=p+\Norm_p$ does not depend on $p\in\Sigma$.
Conclude that $\Sigma$ is a part of the unit sphere centered at $c$.
\end{subthm}

\end{thm}

\begin{thm}{Exercise}\label{ex:shape-curvature-line}
Assume that smooth surfaces $\Sigma_1$ and $\Sigma_2$ intersect at constant angle along a smooth regular curve $\gamma$.
Show that if $\gamma$ is a curvature line in $\Sigma_1$, then it is also a curvature line in $\Sigma_2$.

Conclude that if a smooth surface $\Sigma$ intersects a plane or sphere at constant angle along a smooth curve $\gamma$,
then $\gamma$ is a curvature line of~$\Sigma$.
\end{thm}
