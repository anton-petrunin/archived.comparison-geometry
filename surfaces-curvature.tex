\chapter{Curvatures}

\section*{Tangent-normal coordinates} 

Fix a point $p$ in a smooth oriented surface $\Sigma$.
Consider a coordinate system $(x,y,z)$ with origin at $p$ such that the $(x,y)$-plane coincides with $\T_p$ and the $z$-axis in the direction of the normal vector $\Norm_p$.
By \ref{ex:vertical-tangent}, we can present $\Sigma$ locally around $p$ as a graph of a function $f$. 
Note that $f$ satisfies the following additional properties:
\begin{align*}
f(0,0)&=0,
&
(\tfrac{\partial}{\partial x}f)(0,0)&=0,
&
(\tfrac{\partial}{\partial y}f)(0,0)&=0.
\end{align*}
The first equality holds since $p=(0,0,0)$ lies on the graph and the last two equalities mean that the tangent plane at $p$ is horizontal.

\begin{wrapfigure}[8]{o}{40 mm}
\vskip-0mm
\centering
\includegraphics{asy/paraboloid}
\vskip-3mm
\end{wrapfigure}

Set \label{page:lmn}
\begin{align*}
\ell&=(\tfrac{\partial^2}{\partial x^2}f)(0,0),
\\
m&=(\tfrac{\partial^2}{\partial x\partial y}f)(0,0)=(\tfrac{\partial^2}{\partial y\partial x}f)(0,0),
\\
n&=(\tfrac{\partial^2}{\partial y^2}f)(0,0).
\end{align*}
The Taylor series for $f$ at $(0,0)$ up to the second oreder term can be then written as
\[f(x,y)=\tfrac12(\ell\cdot x^2+2\cdot m\cdot x\cdot y+n\cdot y^2)+o(x^2+y^2).\]
Note that values $\ell$, $m$, and $n$ are completely determined by this equation.
The so-called \emph{osculating paraboloid}
\[z=\tfrac12\cdot(\ell\cdot x^2+2\cdot m\cdot x\cdot y+n\cdot y^2)\]
gives the best approximation of the surface at $p$;
it has \emph{second order of contact} with $\Sigma$ at $p$.

Note that 
\[\ell\cdot x^2+2\cdot m\cdot x\cdot y+n\cdot y^2=\langle M_p\cdot (\begin{smallmatrix}
x\\y
\end{smallmatrix}), (\begin{smallmatrix}
x\\y
\end{smallmatrix})\rangle,\]
where
\[M_p=\begin{pmatrix}
   \ell
   &m
   \\
   m
   &n
  \end{pmatrix};
\eqlbl{eq:hessian}
\]
it is called the Hessian matrix of $f$ at $(0,0)$.

\section*{Principle curvatures}

Note that tangent-normal coordinates give an almost canonical coordinate system in a neighborhood of $p$;
it is unique up to a rotation of  the $(x,y)$-plane.
Rotating the $(x,y)$-plane is equivalent to changing its the basis, which results in the rewriting   
the matrix $M_p$ in this new basis.

Since the Hessian matrix $M_p$ is symmetric, it is diagonalizable by orthogonal matrices.
That is, by rotating the $(x,y)$-plane we can assume that $m=0$ in \ref{eq:hessian}. %???+lin algebra material
In this case the diagonal components of $M_p$ are called \emph{principle curvatures} of $\Sigma$ at $p$;
they are uniquely defined up to sign;
they are denoted as $k_1(p)$ and $k_2(p)$, or $k_1(p)_\Sigma$ and $k_2(p)_\Sigma$ if we need to emphasize that these are the curvatures of the surface $\Sigma$.
We will always assume that $k_1\le k_2$.

Note that if $x=f(x,y)$ is a local graph representation of $\Sigma$ in these coordinates, then 
\[f(x,y)=\tfrac12\cdot(k_1\cdot x^2+k_2\cdot y^2)+o(x^2+y^2).\]

The principle curvatures can be also defined as the eigenvalues of $M_p$;
the eigendirections of $M_p$ are called \emph{principle directions} of $\Sigma$ at~$p$.
Note that if $k_1(p)\ne k_2(p)$, then $p$ has exactly two principle directions, which are perpendicular to each other; if $k_1(p)= k_2(p)$ then all tangent directions at $p$ are principle.

Note that if we revert the orientation of $\Sigma$, then the principle curvatures at each point switch their signs and indexes.

A smooth regular curve on a surface $\Sigma$ that always runs in the principle directions is called a \emph{line of curvature} of $\Sigma$.  

\begin{thm}{Exercise}\label{ex:line-of-curvature}
Assume that a smooth surface $\Sigma$ is mirror symmetric with respect to  a plane $\Pi$.
Suppose that $\Sigma$ and $\Pi$ intersect along a curve $\gamma$.
Show that $\gamma$ is a line of curvature of $\Sigma$.
\end{thm}

\section*{Shape operator}

Let $p$ be a point on a smooth oriented surface $\Sigma$.
Suppose $\Sigma$ is described locally as a graph $z=f(x,y)$ in a tangent-normal coordinates at $p$
and 
\[M_p=\begin{pmatrix}
   \ell
   &m
   \\
   m
   &n
  \end{pmatrix};
\]
is the Hessian matrix of $f$ at $(0,0)$; that is, the components $\ell$, $m$, and $n$ are as on page \pageref{page:lmn}.

The multiplication by the Hessian matrix defies the so called \emph{shape operator}
\[S\:(\begin{smallmatrix}
x\\y
\end{smallmatrix})
\mapsto
M_p\cdot(\begin{smallmatrix}
x\\y
\end{smallmatrix});\]
it is a linear operator $S\:\T_p\to \T_p$.
For a point $p\in \Sigma$ the shape operator of a tangent vector $\vec w\in \T_p$ will be denoted by $S(\vec w)$ if it is clear from the context which base point $p$ and which surface we are working with;
otherwise we may use notations 
\[S_p(\vec w)\quad\text{or even}\quad S_p(\vec w)_\Sigma.\]

Since $M_p$ is symmetric, $S$ is \emph{self-adjoint}; that is
\[\langle S(\vec v),\vec w\rangle=\langle \vec v,S(\vec w)\rangle\]
for any $\vec v,\vec w\in\T_p$.
Note also that principle curvatures of $\Sigma$ at $p$ are the eigenvalues of $S_p$ and the principle directions are the directions of principle vectors of $S_p$.

\begin{thm}{Proposition}\label{prop:shape=D2}
Let $p$ be a point on a smooth oriented surface $\Sigma$.
Suppose $\Sigma$ is described locally as a graph $z=f(x,y)$ in a tangent-normal coordinates at $p$.
Then
\[\langle S(\vec v),\vec w\rangle=D_{\vec w}D_{\vec v}f(0,0)\]
for any $\vec v,\vec w\in\T_p$.
Moreover $S$ is unique linear operator $\T_p\to\T_p$ that satisfies the above condition.
\end{thm}

Here $D_{\vec v}f$ denoted directional derivative of $f$ along vector $\vec v$;
that is, if $\phi(t)=f(q+\vec v\cdot t)$, then $D_{\vec v}f(q)=\phi'(0)$.

\parit{Proof.} 
Suppose $\vec v=(\begin{smallmatrix}
a\\b
\end{smallmatrix})$
and 
$\vec v=(\begin{smallmatrix}
c\\d
\end{smallmatrix})$, then 
\begin{align*}
D_{\vec v}&=a\cdot\tfrac{\partial}{\partial x}+ b\cdot\tfrac{\partial}{\partial y},
&
D_{\vec w}&=c\cdot\tfrac{\partial}{\partial x}+ d\cdot\tfrac{\partial}{\partial y}.
\end{align*}
Therefore 
\begin{align*}
D_{\vec w}D_{\vec v}f&=
a\cdot c\cdot\tfrac{\partial^2 f}{\partial^2 x}
+b\cdot c\cdot\tfrac{\partial^2 f}{\partial x\partial y}
+a\cdot d\cdot\tfrac{\partial^2 f}{\partial y\partial x}
+b\cdot d\cdot\tfrac{\partial^2 f}{\partial^2 y}
\intertext{evaluating this expression at $(0,0)$ we get}
D_{\vec w}D_{\vec v}f(0,0)&=a\cdot c\cdot\ell
+b\cdot c\cdot m
+a\cdot d\cdot m
+b\cdot d\cdot n=
\\
&=\langle M_p\cdot \vec v,\vec w\rangle=\langle \vec v,M_p\cdot \vec w\rangle=
\\
&=\langle S(\vec v),\vec w\rangle=\langle \vec v,S(\vec w)\rangle.
\end{align*}
\qedsf

\begin{thm}{Corollary}\label{cor:S(ij)}
Let  $p$ be a point on a smooth oriented surface $\Sigma$.
Suppose $\Sigma$ is described locally as a graph $z=f(x,y)$ in a tangent-normal coordinates at $p$.
Denote by $\vec i$, $\vec j$ and $\vec k$ the standard basis in the $(x,y,z)$-coordinates.
Then
\begin{align*}
\langle S(\vec i),\vec i\rangle&=\ell,
&
\langle S(\vec i),\vec j\rangle&=m,
&
\langle S(\vec i),\vec k\rangle&=0,
\\
\langle S(\vec j),\vec i\rangle&=m,
&
\langle S(\vec j),\vec j\rangle&=n,
&
\langle S(\vec j),\vec k\rangle&=0,
\end{align*}
where $\ell$, $m$, and $n$ are the components of the Hessian matrix of $f$ at $(0,0)$ defined on page \pageref{page:lmn}.
\end{thm}

\parit{Proof.} 
Note that 
\[D_{\vec i}=\tfrac{\partial}{\partial x}\quad\text{and}\quad D_{\vec j}=\tfrac{\partial}{\partial y}.\]
It remains to use \ref{prop:shape=D2} and the expressions for $\ell$, $m$, and $n$ on page~\pageref{page:lmn}.
\qeds

In the following proposition we use the notion of directional derivative defined in \ref{def:directional-derivative}.

\begin{thm}{Proposition}\label{prop:S=-D}
Let $\Sigma$ be a smooth surface with unit normal field $\Norm$.
Suppose $p\in \Sigma$ and $S\:\T_p\to\T_p$ is the shape operator at $p$.
Then 
\[S(\vec w)=-D_{\vec w}\nu
\eqlbl{eq:shape=-dNorm}\]
for any $\vec w\in \T_p$.
\end{thm}


The reason for minus sign in \ref{eq:shape=-dNorm} is the same as in the formula  
for curvature of plane curve in its Frenet frame: $\norm'=-\skur\cdot\tan$.
This proposition will be proved in the next section. 

\begin{thm}{Exercise}\label{ex:normal-curvature=const}
Let $\Sigma$ be a smooth oriented surface with the unit normal field $\Norm$.
Suppose that $\Sigma$ has unit principle curvatures at any point.
\begin{enumerate}[(a)]
 \item Show that $S_p(w)=w$ for any $p\in\Sigma$ and $w\in \T_p\Sigma$.
 \item Show that $p+\Norm_p$ is constant; that is, the point $c=p+\Norm_p$ does not depend on $p\in\Sigma$.
 Conclude that $\Sigma$ is a part of the unit sphere centered at $c$.
\end{enumerate}

\end{thm}

\begin{thm}{Exercise}\label{ex:shape-curvature-line}
Assume that smooth surfaces $\Sigma_1$ and $\Sigma_2$ intersect at constant angle along a smooth regular curve $\gamma$.
Show that if $\gamma$ is a curvature line in $\Sigma_1$, then it is also a curvature line in $\Sigma_2$.

Conclude that if a smooth surface $\Sigma$ intersects a plane or sphere along a smooth curve $\gamma$,
then $\gamma$ is a curvature line of $\Sigma$.
\end{thm}

\section*{Proof of \ref{prop:S=-D}*} 


\parit{Proof of \ref{prop:S=-D}.}
Let $\Sigma$ be a smooth surface with unit normal field $\Norm$.
Suppose $(u,v)\mapsto s(u,v)$ is a local chart of $\Sigma$ at $p$.
Since $\Norm$ is unit we have the identity
\[1=\langle\Norm{\circ} s,\Norm{\circ} s\rangle.\]
Note that the vectors $\tfrac{\partial s}{\partial u}$ and $\tfrac{\partial s}{\partial v}$ tangent at their base point; therefore we have two more identities:
\begin{align*}
 0&=\langle\tfrac{\partial}{\partial u} s,\Norm{\circ} s\rangle,
 &
 0&=\langle\tfrac{\partial}{\partial u} s,\Norm\rangle.
\end{align*}
Taking partial derivatives of these there identities and applying the product rule,
we get the following six identities:
\begin{align*}
0&=\tfrac{\partial}{\partial u}\langle\Norm{\circ} s,\Norm{\circ} s\rangle
=
2\cdot \langle\tfrac{\partial}{\partial u}\Norm{\circ} s,\Norm{\circ} s\rangle,
\\
0&=\tfrac{\partial}{\partial v}\langle\Norm{\circ} s,\Norm{\circ} s\rangle=2\cdot \langle\tfrac{\partial}{\partial v}\Norm{\circ} s,\Norm{\circ} s\rangle,
\\
0&=\tfrac{\partial}{\partial u}\langle\tfrac{\partial}{\partial u} s,\Norm{\circ} s\rangle=\langle\tfrac{\partial^2}{\partial u^2} s,\Norm{\circ} s\rangle+\langle\tfrac{\partial}{\partial u} s,\tfrac{\partial}{\partial u}\Norm{\circ} s\rangle,
\\
0&=\tfrac{\partial}{\partial v}\langle\tfrac{\partial}{\partial u} s,\Norm{\circ} s\rangle=\langle\tfrac{\partial^2}{\partial v\partial u} s,\Norm{\circ} s\rangle+\langle\tfrac{\partial}{\partial u} s,\tfrac{\partial}{\partial v}\Norm{\circ} s\rangle
\\
0&=\tfrac{\partial}{\partial u}\langle\tfrac{\partial}{\partial v} s,\Norm{\circ} s\rangle=\langle\tfrac{\partial^2}{\partial u\partial v} s,\Norm{\circ} s\rangle+\langle\tfrac{\partial}{\partial v} s,\tfrac{\partial}{\partial u}\Norm{\circ} s\rangle,
\\
0&=\tfrac{\partial}{\partial v}\langle\tfrac{\partial}{\partial v} s,\Norm{\circ} s\rangle=\langle\tfrac{\partial^2}{\partial v^2} s,\Norm{\circ} s\rangle
+
\langle\tfrac{\partial}{\partial v} s,\tfrac{\partial}{\partial v}\Norm{\circ} s\rangle.
\end{align*}

Now suppose $z=f(x,y)$ be a local description of $\Sigma$ in the tangent-normal coordinates at $p$.
Note that 
\[s(u,v)=(u,v,f(u,v))\]
describes a chart of $\Sigma$ at $p$.

Denote by $\vec i$, $\vec j$ and $\vec k$ the standard basis in the $(x,y,z)$-coordinates.
Note that $s(0,0)=p$ and 
\begin{align*}
\tfrac{\partial}{\partial u} s(0,0)&=\vec i,
&
\tfrac{\partial}{\partial v} s(0,0)&=\vec j,
&
\nu\circ s(0,0)&=\vec k,
\intertext{In particular $D_{\vec i}\nu=\tfrac{\partial}{\partial u}\Norm{\circ} s(0,0)$ and $D_{\vec j}\nu=\tfrac{\partial}{\partial v}\Norm{\circ} s(0,0)$. Further,}
\tfrac{\partial^2}{\partial u^2} s(0,0)&=\ell\cdot \vec k,
&
\tfrac{\partial^2}{\partial v\partial u} s(0,0)&=m\cdot \vec k,
&
\tfrac{\partial^2}{\partial v^2} s(0,0)&=n\cdot \vec k,
\end{align*}
where $\ell$, $m$, and $n$ are the components of the Hessian matrix of $f$ at $(0,0)$ defined on page \pageref{page:lmn}.

Evaluating the above 6 identities at $(u,v)\z=(0,0)$, we get that
\begin{align*}
\langle -D_{\vec i}\nu ,\vec i\rangle&=\ell,
&
\langle -D_{\vec i},\vec j\rangle&=m,
&
\langle -D_{\vec i}\nu,\vec k\rangle&=0,
\\
\langle -D_{\vec j}\nu,\vec i\rangle&=m,
&
\langle -D_{\vec j}\nu,\vec j\rangle&=n,
&
\langle -D_{\vec j}\nu,\vec k\rangle&=0,
\end{align*}
That is, $-D_{\vec i}\nu$ and $-D_{\vec j}\nu$ satisfy the same equalities as $S(\vec i)$ in \ref{cor:S(ij)}.
Note that these equalities define $S$ completely; 
therefore \ref{eq:shape=-dNorm} follows.
\qeds


\section*{More curvatures}

Fix an oriented smooth surface $\Sigma$ and a point $p\in\Sigma$.

The product 
\[K(p)=k_1(p)\cdot k_2(p)\]
is called Gauss curvature at $p$.
We may denote it by $K(p)_\Sigma$ if we need to emphasize that this is curvature of $\Sigma$.
The Gauss curvature can be also interpreted as the determinant of the Hessian matrix $M_p$, or, equivalently,  as the determinant of the shape operator $S_p$.

The sum 
\[H(p)=k_1(p)+ k_2(p)\] 
is called \emph{mean curvature}\footnote{Some authors define mean curvature as $\tfrac12\cdot(k_1(p)+ k_2(p))$ --- the mean value of the principle curvatures. It is suits the name better, but not as convenient when it comes to calculations.} at $p$.
We may denote it by $H(p)_\Sigma$ if we need to emphasize that this is curvature of $\Sigma$.
The mean curvature can be also interpreted as the trace of the Hessian matrix $M_p$, or, equivalently,  as the trace of the shape operator $S_p$. 

Note that the Gauss curvature depends only on $\Sigma$ and $p$.
The same is true up to sign for the mean curvature --- it changes the sign if we revert the orientation of the surface.

\begin{thm}{Exercise}\label{ex:gauss+orientable}
Show that any surface with positive Gauss curvature is orientable. 
%Show that any surface with positive mean curvature is orientable.
\end{thm}



