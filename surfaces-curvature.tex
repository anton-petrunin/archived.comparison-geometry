\chapter{Curvatures}

\section*{Tangent-normal coordinates} 

Fix a point $p$ in a smooth surface $\Sigma$.
Consider a coordinate system $(x,y,z)$ with origin at $p$ such that the $(x,y)$-plane coincides with $\T_p$.
By \ref{ex:vertical-tangent}, we can present $\Sigma$ locally around $p$ as a graph of a function $f$. 
Note that $f$ satisfies the following additional properties:
\begin{align*}
f(0,0)&=0,
&
(\tfrac{\partial}{\partial x}f)(0,0)&=0,
&
(\tfrac{\partial}{\partial y}f)(0,0)&=0.
\end{align*}
The first equality holds since $p=(0,0,0)$ lies on the graph and the last two equalities mean that the tangent plane at $p$ is horizontal.

\begin{wrapfigure}{r}{40 mm}
\vskip-0mm
\centering
\includegraphics{asy/paraboloid}
\vskip-3mm
\end{wrapfigure}

Consider the Hessian matrix 
\[M_p=\begin{pmatrix}
   \ell
   &m
   \\
   m
   &n
  \end{pmatrix},
\eqlbl{eq:hessian}
\]
where 
\begin{align*}
\ell&=(\tfrac{\partial^2}{\partial x^2}f)(0,0),
\\
m&=(\tfrac{\partial^2}{\partial x\partial y}f)(0,0)=(\tfrac{\partial^2}{\partial y\partial x}f)(0,0),
\\
n&=(\tfrac{\partial^2}{\partial y^2}f)(0,0).
\end{align*}

The components of the matrix describe the surface at up to the second order at $p$.
In fact the so-called \emph{osculating paraboloid}
\[z=\tfrac12(\ell\cdot x^2+2\cdot m\cdot x\cdot y+n\cdot y^2)\]
gives the best approximation of the surface at $p$;
it has \emph{second order of contact} with $\Sigma$ at $p$.

Given two vectors $v,w$ in the $(x,y)$-plane, consider the value 
\[\II_p(v,w)\df(D_wD_vf)(0,0),\]
where $D$ denotes the directional derivative.
The function $(v,w)\z\mapsto \II_p(v,w)$ is called the \emph{second fundamental form} at $p$;\label{page:second fundamental form}
it takes two tangent vectors $v$ and $w$ at $p$ and spits out the real number $\II_p(v,w)$.

The second fundamental form can be written in terms of the Hessian matrix.
Indeed if 
$w=(\begin{smallmatrix}a\\b
\end{smallmatrix})$ 
and 
$v=(\begin{smallmatrix}c\\d
\end{smallmatrix})$, then 
\[D_w=a\cdot \tfrac\partial{\partial x}+b\cdot \tfrac\partial{\partial y}\quad\text{and}\quad D_v=c\cdot \tfrac\partial{\partial x}+d\cdot \tfrac\partial{\partial y}.\]
Therefore 
\[\begin{aligned}
\II_p(w,v)&\df(D_wD_vf)(0,0)=
\\
&=a\cdot c\cdot \ell+(a\cdot d+ b\cdot c)\cdot m +b\cdot d\cdot n=
\\
&=\langle M_p\cdot w,v\rangle=
\\
&=\langle M_p\cdot v,w\rangle.
\end{aligned}
\eqlbl{eq:DwDv}\]
Note that from \ref{eq:DwDv} it follows that $\II_p$ is symmetric; that is,
\[\II_p(v,w)=\II_p(w,v)\eqlbl{eq:II=II}\]
for any two tangent vectors $v,w\in \T_p$.

\section*{Principle curvatures}

Note that tangent-normal coordinates give an almost canonical coordinate system in a neighborhood of $p$;
it is unique up to a rotation of  the $(x,y)$-plane and a switching of the sign of the $z$-coordinate.
Rotating the $(x,y)$-plane is equivalent to changing its orthonormal basis, which results in the rewriting   
the Hessian matrix in this new basis.

Since Hessian matrix is symmetric, it is diagonalizable by orthogonal matrices. %???
That is, by rotating the $(x,y)$-plane we can assume that $m=0$ in \ref{eq:hessian}. %???+lin algebra material
In this case the diagonal components of $M_p$ are called \emph{principle curvatures} of $\Sigma$ at $p$;
they are uniquely defined up to sign;
they are denoted as $k_1(p)$ and $k_2(p)$, or $k_1(p)_\Sigma$ and $k_2(p)_\Sigma$ if we need to emphasize that these are the curvatures of the surface $\Sigma$.
We will always assume that $k_1\le k_2$.

The principle curvatures can be also defined as the eigenvalues of $M_p$;
the eigendirections of $M_p$ are called \emph{principle directions} of $\Sigma$ at~$p$.
Note that if $k_1(p)\ne k_2(p)$ then $p$ has exactly two principle directions, which are perpendicular to each other.

Note that if we revert the orientation of $\Sigma$, then the principle curvatures at each point switch their signs and indexes.

\section*{Normal curvature}

%???+PIC

Assume we choose the coordinates in the $(x,y)$-plane so that the Hessian matrix is diagonalized, we can assume that
\[M_p=\begin{pmatrix}
   k_1(p)
   &0
   \\
   0
   &k_2(p)
  \end{pmatrix}.
\]
According to \ref{eq:DwDv}, the second directional derivative
for a vector $w=(\begin{smallmatrix}a\\b
\end{smallmatrix})$ in the $(x,y)$-plane can be written as 
\[
(D^2_wf)(0,0)=a^2\cdot k_1(p) +b^2\cdot k_2(p).
\]

If $w$ is unit, then the second directional derivative $D^2_wf(0,0)$ can be intepreted as the signed curvature of the curve formed by the intersection of $\Sigma$ with the plane thru $p$ spanned by $\nu_p$ and $w$.
In this case $\II_p(w,w)=D^2_wf(0,0)$ is called \emph{normal curvature} in the direction $w$;
it is denoted by $k_w(p)$ or $k_w(p)_\Sigma$.

Since $|w|=1$, we have $a^2+b^2=1$ which implies the following:

\begin{thm}{Observation}\label{obs:k1-k2}
For any point $p$ on an oriented smooth surface $\Sigma$,
the principle curvatures $k_1(p)$ and $k_2(p)$ are correspondingly minimum and maximum of the normal curvatures at $p$.
Moreover, if $\theta$ is the angle between a unit vector $w\in\T_p$ and the first principle direction at $p$, then 
\[k_w(p)=k_1(p)\cdot\cos^2\theta+k_2(p)\cdot\sin^2\theta.\]

\end{thm}

The last identity is the so-called \emph{Euler's formula}.

A smooth regular curve on a surface $\Sigma$ that always runs in the principle directions is called a \emph{line of curvature} of $\Sigma$.  

\begin{thm}{Exercise}\label{ex:line-of-curvature}
Assume that a smooth surface $\Sigma$ is mirror symmetric with respect to  a plane $\Pi$.
Suppose that $\Sigma$ and $\Pi$ intersect along a curve $\gamma$.
Show that $\gamma$ is a line of curvature of $\Sigma$.
\end{thm}

\begin{thm}{Exercise}\label{ex:moon-revolution}
Assume $V$ is a body of revolution in $\RR^3$ and its boundary is a smooth surface with principle curvatures at most 1 in absolute value.
Show that $V$ contains a unit ball.
\end{thm}

\parit{Hint:} Use \ref{ex:line-of-curvature} and \ref{thm:moon-orginal}.

\section*{More curvatures}

Fix an oriented smooth surface $\Sigma$ and a point $p\in\Sigma$.

The product 
\[K(p)=k_1(p)\cdot k_2(p)\]
is called Gauss curvature at $p$.
We may denote it by $K(p)_\Sigma$ if we need to emphasize that this is curvature of $\Sigma$.
The Gauss curvature can be also interpreted as determinant of the Hessian matrix $M_p$.

The sum 
\[H(p)=\tfrac12\cdot(k_1(p)+ k_2(p))\] %???  back later
is called mean curvature at $p$.
We may denote it by $H(p)_\Sigma$ if we need to emphasize that this is curvature of $\Sigma$.
The mean curvature can be also interpreted as half of the trace of the Hessian matrix $M_p$. %???  back later

Note that the Gauss curvature depends only on $\Sigma$ and $p$,
and not on the choice of the coordinate system.
The same is true up to sign for the mean curvature --- it changes the sign if we revert the orientation of the surface.

\begin{thm}{Exercise}\label{ex:gauss+orientable}
Show that any surface with positive Gauss curvature is orientable. 
\end{thm}


\section*{Supporting surfaces}

Assume two oriented surfaces $\Sigma_1$ and $\Sigma_2$ have a common point $p$.
If there is a neighborhood $U$ of $p$ such that $\Sigma_1\cap U$ lies on one side from $\Sigma_2$ in $U$, then we say that $\Sigma_2$ \emph{locally supports} $\Sigma_1$ at $p$.

\begin{thm}{Exercise}\label{ex:T=T}
Let $\Sigma_1$ and $\Sigma_2$ be two smooth surfaces.
Assume $\Sigma_2$ \emph{locally supports} $\Sigma_1$ at a point $p$.
Show that $\T_p\Sigma_1=\T_p\Sigma_2$;
that is, the tangent planes of $\Sigma_1$ and $\Sigma_2$ at $p$ coincide.
\end{thm}

By the exercise, we can assume that $\Sigma_1$ and $\Sigma_2$ are cooriented at $p$;
that is, they have common unit normal vector at $p$.
If not we can revert the orientation of one of the surfaces.

If $\Sigma_1$ and $\Sigma_2$ are cooriented at $p$,
then we can say that $\Sigma_1$ locally supports $\Sigma_2$ from \emph{inside} or from \emph{outside},
assuming that the normal vector points \emph{inside} the domain bounded by surface $\Sigma_2$ in $U$.

More precisely, we can use for $\Sigma_1$ and $\Sigma_2$ one tangent-normal coordinate system at $p$,
assuming that the $z$-axis points in the direction of the unit normal vector $\nu_p$ to both surfaces.
This way we write $\Sigma_1$ and $\Sigma_2$ locally as graphs: $z=f_1(x,y)$  and $z=f_2(x,y)$ correspondingly.
Then $\Sigma_1$ locally supports $\Sigma_2$ from inside (from outside)  if $f_1(x,y)\ge f_2(x,y)$ (correspondingly $f_1(x,y)\le f_2(x,y)$) for $(x,y)$ in a sufficiently small neighborhood of the origin.

Note that $\Sigma_1$ locally supports $\Sigma_2$ from inside at the point $p$ is equivalent to $\Sigma_2$ locally supports $\Sigma_1$ from outside.
Further if we revert the orientation of both surfaces then supporting from inside becomes supporting from outside and the other way around.


\begin{thm}{Proposition}\label{prop:surf-support}
Let $\Sigma_1$ and $\Sigma_2$ be oriented surfaces.
Assume $\Sigma_1$ locally supports $\Sigma_2$ from inside at the point $p$ (equivalently $\Sigma_2$ locally supports $\Sigma_1$ from outside).
Then $k_1(p)_{\Sigma_1}\ge k_1(p)_{\Sigma_2}$ and $k_2(p)_{\Sigma_1}\z\ge k_2(p)_{\Sigma_2}$.
\end{thm}

\begin{thm}{Exercise}\label{ex:surf-support}
Give an example of two surfaces $\Sigma_1$ and $\Sigma_2$ that have common point $p$ with common unit normal vector $\nu_p$ such that 
$k_1(p)_{\Sigma_1}> k_1(p)_{\Sigma_2}$ and $k_2(p)_{\Sigma_1}\z> k_2(p)_{\Sigma_2}$, but $\Sigma_1$ does not support $\Sigma_2$ locally at $p$.
\end{thm}


\parit{Proof.} We can assume that $\Sigma_1$ and $\Sigma_2$ are graphs $z=f_1(x,y)$  and $z=f_2(x,y)$ in a common tangent-normal coordinates at $p$, so we have $f_1\ge f_2$.

\begin{wrapfigure}{o}{40 mm}
\vskip-4mm
\centering
\includegraphics{mppics/pic-80}
\vskip-0mm
\end{wrapfigure}

Fix a unit vector $w\in \T_p\Sigma_1=\T_p\Sigma_2$.
Consider the plane $\Pi$ passing thru $p$ and spanned by the normal vector $\nu_p$ and $w$.
Let $\gamma_1$ and $\gamma_2$ be the curves of intersection of $\Sigma_1$ and $\Sigma_2$ with $\Pi$.

Let us orient $\Pi$ so that $\nu_p$ points to the left from $w$.
Further, let us parametrize both curves so that they are running in the direction of $w$ at $p$ and therefore cooriented.
In this case the curve $\gamma_1$ supports the curve $\gamma_2$ from the right. %??? make \ref{prop:supporting-circline} once it is generalized


Therefore we have the following inequality for the normal curvatures of $\Sigma_1$ and $\Sigma_2$ at $p$ in the direction of $w$:
\[k_w(p)_{\Sigma_1}\ge k_w(p)_{\Sigma_2}.\eqlbl{kw>=kw}\]

According to \ref{obs:k1-k2},
\[k_1(p)_{\Sigma_i}=\min\set{k_w(p)_{\Sigma_i}}{w\in\T_p, |w|=1}\]
for $i=1,2$.
Choose $w$ so that $k_1(p)_{\Sigma_1}=k_w(p)_{\Sigma_1}$.
Then by \ref{kw>=kw}, we have that
\begin{align*}
k_1(p)_{\Sigma_1}&=k_w(p)_{\Sigma_1}\ge
\\
&\ge k_w(p)_{\Sigma_2}\ge
\\
&\ge\min\set{k_w(p)_{\Sigma_2}}{}=
\\
&=k_1(p)_{\Sigma_2};
\end{align*}
that is, $k_1(p)_{\Sigma_1}\ge k_1(p)_{\Sigma_2}$.

Similarly, by \ref{obs:k1-k2}, we have that
\[k_2(p)_{\Sigma_i}=\max\set{k_w(p)_{\Sigma_i}}{}.\]
Let us fix $w$ so that $k_2(p)_{\Sigma_2}=k_w(p)_{\Sigma_2}$.
Then 
\begin{align*}
k_2(p)_{\Sigma_2}&=k_w(p)_{\Sigma_2}\le
\\
&\le k_w(p)_{\Sigma_1}\le
\\
&\le\max\set{k_w(p)_{\Sigma_1}}{}=
\\
&=k_2(p)_{\Sigma_1};
\end{align*}
that is, $k_2(p)_{\Sigma_1}\ge k_2(p)_{\Sigma_2}$.
\qeds

\begin{thm}{Corollary}\label{cor:surf-support}
Let $\Sigma_1$ and $\Sigma_2$ be oriented surfaces.
Assume $\Sigma_1$ locally supports $\Sigma_2$ from inside at the point $p$.
Then
\begin{enumerate}[(a)]
\item\label{cor:surf-support:mean} $H(p)_{\Sigma_1}\ge H(p)_{\Sigma_2}$;
\item\label{cor:surf-support:gauss} If $k_1(p)_{\Sigma_2}\ge 0$, then $K(p)_{\Sigma_1}\ge K(p)_{\Sigma_2}$.
\end{enumerate}
 
\end{thm}

\parit{Proof.}
By (\ref{prop:surf-support}), we get that $k_1(p)_{\Sigma_1}\ge k_1(p)_{\Sigma_2}$ and $k_2(p)_{\Sigma_2}\ge k_2(p)_{\Sigma_2}$.
Therefore part (\ref{cor:surf-support:mean}) follows since 
\[H(p)_{\Sigma_i}=\tfrac12\cdot(k_1(p)_{\Sigma_i}+k_2(p)_{\Sigma_i})).\]


\parit{(\ref{cor:surf-support:gauss}).} Since $k_2(p)_{\Sigma_i}\ge k_1(p)_{\Sigma_i}$ and $k_1(p)_{\Sigma_2}\ge 0$, we get that all the principle curvatures 
$k_1(p)_{\Sigma_1}$, $k_1(p)_{\Sigma_1}$ and $k_2(p)_{\Sigma_1}$ and $k_2(p)_{\Sigma_2}$ are nonnegative.
Whence
\begin{align*}
K(p)_{\Sigma_1}&=k_1(p)_{\Sigma_1}\cdot k_2(p)_{\Sigma_1}\ge 
\\
&\ge k_1(p)_{\Sigma_2}\cdot k_2(p)_{\Sigma_2}=
\\
&=K(p)_{\Sigma_2}.
\end{align*}
\qedsf

\begin{thm}{Exercise}\label{ex:positive-gauss}
Show that any closed surface has a point with positive Gauss curvature. %??? change to the surface in a unit ball
\end{thm}


\parit{Hint.}
Consider the minimal sphere that encloses the surface.

\begin{thm}{Exercise}\label{ex:surrounds-disc}
Assume that a closed surface $\Sigma$ surrounds a unit disc.
Show that Gauss curvature of $\Sigma$ is at most 1 at some point. 

Try to prove the same assuming that $\Sigma$ surrounds a unit circle only.
\end{thm} %move in positive curvature???

\parit{Hint.}
Look for a supporting spherical dome with the unit circle as the boundary.

\section*{Curve in a surface}

Recall that the second fundamental form $\II_p$ is defined on page \pageref{page:second fundamental form}.

\begin{thm}{Proposition}\label{prop:gamma''=II}
Suppose $\gamma$ is a smooth curve in a smooth oriented surface $\Sigma$ with a unit normal field $\nu$.
Then, the following identity holds for any time parameter $t$:
\[\langle \gamma''(t),\nu_{\gamma(t)}\rangle=\II_{\gamma(t)}(\gamma'(t),\gamma'(t)).\]

\end{thm}

\parit{Proof.} 
Fix a parameteer value $t_0$; set $p=\gamma(t_0)$, $v=\gamma'(t_0)$ and $a\z=\gamma''(t_0)$;
so we need to show that
\[\langle a,\nu_{p}\rangle=\II_p(v,v).\eqlbl{a-nu-II}\]
Let $z=f(x,y)$ be the local representaion of $\Sigma$ in the tangent-normal coordinates at $p$;
we assume that $\nu$ points in the direction of~$\nu_p$.

Without loss of generality may assume that $\gamma$ runs in the graph $z=f(x,y)$;
so 
\[\gamma(t)=\left(x(t),y(t),f(x(t),y(t))\right).\]
Then
\begin{align*}
\gamma'&=(x',y',\tfrac{\partial f}{\partial x}\cdot x'+\tfrac{\partial f}{\partial y}\cdot y');
\\
\gamma''
&=
{\small(x'',
y'',
 \tfrac{\partial^2 f}{\partial x^2}\cdot (x')^2
+
2\cdot \tfrac{\partial^2 f}{\partial x\partial y}\cdot x'\cdot y'
+
\tfrac{\partial^2 f}{\partial y^2}\cdot (y')^2
+
\tfrac{\partial f}{\partial x}\cdot x''
+
\tfrac{\partial f}{\partial y}\cdot y'')}.
\end{align*}

Recall that $p=\gamma(t_0)=(0,0,0)$ and
\begin{align*}
f(0,0)&=0,
&
\tfrac{\partial f}{\partial x}(0,0)&=0,
&
\tfrac{\partial f}{\partial y}(0,0)&=0.
\end{align*}

Therefore 
\begin{align*}
v&=\left(x',y',0\right)(t_0);
\\
a&=\left(x'',y'',
\tfrac{\partial^2 f}{\partial x^2}\cdot (x')^2
+
2\cdot \tfrac{\partial^2 f}{\partial x\partial y}\cdot x'\cdot y'
+
\tfrac{\partial^2 f}{\partial y^2}\cdot (y')^2\right)(t_0).
\end{align*}

Note that 
\[\II_p(v,v)=\left(\tfrac{\partial^2 f}{\partial x^2}\cdot (x')^2
+
2\cdot \tfrac{\partial^2 f}{\partial x\partial y}\cdot x'\cdot y'
+
\tfrac{\partial^2 f}{\partial y^2}\cdot (y')^2\right)(t_0);\]
that is, the $z$-coordinate of the acceleration $a$ equals $\II_p(v,v)$ which is equivalent to~\ref{a-nu-II}.
\qeds



\begin{thm}{Corollary}\label{cor:meusnier}
Let $\gamma$ be a regular smooth curve that runs in a smooth surface $\Sigma$.
Suppose $p=\gamma(t_0)$ and $w=\gamma'(t_0)$.
Denote by $\alpha$ the angle between the unit normal to $\Sigma$ at $p$ and the unit normal vector in the Frenet frame of $\gamma$.
Then the following identity holds for the curvature $\kur(t_0)_\gamma$ of $\gamma$ at $p$ and the normal curvature $k_w(p)$ of $\Sigma$ at $p$ in the direction of $w$:  
\[ \kur(t_0)_\gamma\cdot\cos\alpha=k_{w}(p).\]

\end{thm}


\parit{Proof.}
Denote by $\nu_\Sigma$ the unit normal vector to $\Sigma$ at $p$ 
and by $\nu_\gamma$ the unit normal vector in the Frenet frame of $\gamma$.
Note that $\cos\alpha=\langle\nu_\Sigma,\nu_\gamma\rangle$.

Applying \ref{prop:gamma''=II}, we get that
\begin{align*}
k_{w}(p)&=\II_p(w,w) =
\\
&=\langle\gamma'',\nu_\Sigma\rangle=
\\
&=\kur(t_0)_\gamma\cdot\langle\nu_\gamma,\nu_\Sigma\rangle=
\\
&=\kur(t_0)_\gamma\cdot\cos\alpha.
\end{align*}
\qedsf

The corollary above, as well as the statement in the following exercise are proved by Jean Baptiste Meusnier \cite{meusnier}.

\begin{thm}{Exercise}\label{ex:meusnier}
Let $\Sigma$ be a smooth surface, $p\in\Sigma$ and $w\in \T_p\Sigma$ is a unit vector.
Assume that $k_w(p)\ne 0$; that is the normal curvature of $\Sigma$ at $p$ in the direction of $w$ does not vanish.

Show that the osculating circles at $p$ of smooth regular curves in $\Sigma$ that run in the direction $w$ sweep out a sphere. 
\end{thm}

\begin{thm}{Exercise}\label{ex:principle-revolution}
Let $\gamma(t)=(x(t),y(t))$ be a smooth unit-speed simple plane curve in the upper half-plane.
Suppose that $\Sigma$ is the surface of revolution of $\gamma$ with respect to the $x$-axis.

Express the principle curvatures of $\Sigma$ at $(x(t),y(t),0)$ in terms of $y(t)$, $y'(t)$ and $y''(t)$.
Conclude that $-\tfrac{y''(t)}{y(t)}$ is the Gauss curvature of $\Sigma$ at $(x(t),y(t),0)$.
\end{thm}

\textit{Hint:} Use \ref{ex:line-of-curvature} and \ref{cor:meusnier}.

\begin{thm}{Exercise}
Assume that a regular smooth curve $\gamma$ lies in a surface of positive Gauss curvature.
Show that curvature of $\gamma$ does not vanish at any value.
\end{thm}
