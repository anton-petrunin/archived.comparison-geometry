\chapter{Shortest paths}

\section{Intrinsic geometry}

We start to study the \index{intrinsic geometry}\emph{intrinsic geometry} of surfaces.
A property is called \index{intrinsic property}\emph{intrinsic} if it can be checked measuring things inside the surface, for example length of curves or angles between the curves that lie in the surface.
Otherwise, if a definition of property essentially use the ambient space, then it is called \index{extrinsic}\emph{extrinsic}.

For instance, mean curvature as well as Gauss curvature are defined via principle curvatures, which are extrinsic.
Later (\ref{thm:remarkable}) it will be shown that \emph{remarkably} Gauss curvature is actually intrinsic --- it can be calculated based on measurements inside the surface.
The mean curvature is not intrinsic, for example intrinsic geometry of the $(x,y)$-plane is not distinguishable from the intrinsic geometry of the graph $z=(x+y)^2$,
while the mean curvature of former vanish at all points, the mean curvature of the latter does not vanish, say at $p=(0,0,1)$.  

The following exercise should help you to be in the right mood;
it might look like a tedious problem in calculus, but actually it is an easy problem in geometry.
We learned this problem from Joel Fine, who attributed it so Frederic Bourgeois \cite{fine}.

\begin{wrapfigure}[6]{r}{33 mm}
\vskip-6mm
\centering
\includegraphics{mppics/pic-77}
\vskip-0mm
\end{wrapfigure}

\begin{thm}{Exercise}\label{ex:lasso}
А cowboy stands at the bottom of a frictionless ice-mountain formed by a cone with a circular base with the angle  of inclination~$\theta$.
He wants to climb the mountain;
he throws up his lasso which slips neatly over the tip of the cone, pulls it tight and starts to climb.
%If the angle of inclination $\theta$ is large, there is no problem; the lasso grips tight and up he goes.
%On the other hand if $\theta$ is small, the lasso slips off as soon as the cowboy pulls on it.

What is the critical angle $\theta$ at which the cowboy can no longer climb the ice-mountain?
\end{thm}

\section{Definition}

Let $p$ and $q$ be two points on a surface $\Sigma$.
Recall that $|p-q|_\Sigma$ denotes the induced length distance from $p$ to $q$;
that is, the greatest lower bound on lengths of paths in $\Sigma$ from $p$ to $q$.

Note that if $\Sigma$ is smooth, then any two points in $\Sigma$ can be joined by a piecewise smooth path.
Since any such path is rectifiable, the value $|p-q|_\Sigma$ is finite for any pair $p,q\in\Sigma$.

A path $\gamma$ from $p$ to $q$ in $\Sigma$ that minimizes the length is called a \index{shortest path}\emph{shortest path} from $p$ to $q$.

The image of a shortest path between $p$ and $q$ in $\Sigma$ is usually denoted by $[pq]$ or by $[pq]_\Sigma$.
In general there might be no shortest path between two given points on the surface
and it might be many of them;
this is shown in the following two examples.

Usually, if we write $[pq]_\Sigma$, then we assume that a shortest path exists and we made a choice of one of them.

{

\begin{wrapfigure}{r}{28 mm}
\vskip-6mm
\centering
\includegraphics{asy/sphere}
\end{wrapfigure}

\parbf{Nonuniqueness.} There are plenty of shortest paths between the poles on a round sphere --- each meridian is a shortest path.

\parbf{Nonexistence.} Let $\Sigma$ be the $(x,y)$-plane with removed origin.
Consider two points $p=(1,0,0)$ and $q=(-1,0,0)$ in $\Sigma$.

}

\begin{thm}{Claim}
There is no shortest path from $p$ to $q$ in $\Sigma$.
\end{thm}

\begin{wrapfigure}{r}{28 mm}
\vskip-4mm
\centering
\includegraphics{mppics/pic-79}
\end{wrapfigure}

\parit{Proof.}
Note that $|p-q|_\Sigma=2$. 
Indeed, given $\eps\z>0$, consider the point $s_\eps=(0,\eps,0)$.
Observe that the polygonal path $ps_\eps q$ lies in $\Sigma$ and its length $2\cdot\sqrt{1+\eps^2}$ approaches $2$ as $\eps\to0$.
It follows that $|p-q|_\Sigma\le 2$.
Since $|p-q|_\Sigma\ge |p-q|_{\RR^3}=2$, we get $|p-q|_\Sigma=2$.


It follows that a shortest path from $p$ to $q$, if it exists, must have length 2.
By triangle inequality any curve of length 2 from $p$ to $q$ must run along the line segment $[pq]$;
in particular it must pass thru the origin.
Since the origin does not lie in $\Sigma$, there is no shortest from $p$ to $q$ in $\Sigma$ 
\qeds

\begin{thm}{Proposition}\label{prop:shortest-paths-exist}
Any two points in a proper smooth surface can be joined by a shortest path. 
\end{thm}

\parit{Proof.}
Fix a proper smooth surface $\Sigma$ with two points $p$ and $q$.
Set $\ell=|p-q|_\Sigma$.

By the definition of induced length metric (Section~\ref{sec:Length metric}),
there is a sequence of paths $\gamma_n$ from $p$ to $q$ in $\Sigma$ such that
\[\length\gamma_n\to \ell\quad\text{as}\quad n\to \infty.\]

Without loss of generality, we may assume that $\length\gamma_n<\ell+1$ for any $n$ and each $\gamma_n$ is parametrized proportional to its arc-length.
In particular each path $\gamma_n\:[0,1]\to\Sigma$ is $(\ell+1)$-Lipschitz; 
that is,
\[|\gamma(t_0)-\gamma(t_1)|\le (\ell+1)\cdot|t_0-t_1|\]
for any $t_0,t_1\in[0,1]$.

Note that the image of $\gamma_n$ lies in the closed ball $\bar B[p,\ell+1]$ for any $n$.
It follows that the coordinate functions of $\gamma_n$ are uniformly equicontinuous and uniformly bounded.
By Arzel\'{a}--Ascoli theorem (\ref{lem:equicontinuous})
 there is a converging subsequence of $\gamma_n$ and its limit, say $\gamma_\infty\:[0,1]\to\RR^3$, is continuous;
that is, $\gamma_\infty$ is a path.
Evidently $\gamma_\infty$ runs from $p$ to $q$;
in particular
\[\length\gamma_\infty\ge \ell.\]
Since $\Sigma$ is a closed set, $\gamma_\infty$ lies in $\Sigma$.
Finally, since length is semicontinuous (\ref{thm:length-semicont}), we get that
\[\length\gamma_\infty\le \ell.\]
Therefore $\length\gamma_\infty= \ell$ or, equivalently, $\gamma_\infty$ is a shortest path from $p$ to $q$.
\qeds

\section{Closest point projection}\label{sec:closest-point-projection}

\begin{thm}{Lemma}\label{lem:closest-point-projection}
Let $R$ be a closed convex set in $\RR^3$.
Then for every point $p\in\RR^3$ there is a unique point $\bar p\in R$ that minimizes the distance to $R$;
that is, $|p-\bar p|\le |p-x|$ for any point $x\in R$.

Moreover the map $p\mapsto \bar p$ is short;
that is,
\[|p-q|\ge|\bar p-\bar q| \eqlbl{eq:short-cpp}\]
for any pair of points $p,q\in \RR^3$.
\end{thm}

The map $p\mapsto \bar p$ is called the \label{closest point projection}\index{closest point projection}\emph{closest point projection};
it maps the Euclidean space to $R$.
Note that if $p\in R$, then $\bar p=p$.

\parit{Proof.}
Fix a point $p$ and set 
\[\ell=\inf\set{|p-x|}{x\in R}.\]
Choose a sequence $x_n\in R$ such that $|p-x_n|\to \ell$ as $n\to\infty$.

Without loss of generality, we can assume that all the points $x_n$ lie in a ball of radius $\ell+1$ centered at~$p$.
Therefore we can pass to a \index{partial limit}\emph{partial limit} $\bar p$ of $x_n$; that is, $\bar p$ is a limit of a subsequence of $x_n$.
Since $R$ is closed, $\bar p\in R$.
By construction 
\begin{align*}
|p-\bar p|&=\lim_{n\to\infty}|p-x_n|=\ell.
\end{align*}
Hence the existence follows.

{

\begin{wrapfigure}{r}{22 mm}
\vskip-0mm
\centering
\includegraphics{mppics/pic-40}
\vskip-0mm
\end{wrapfigure}

Assume there are two distinct points $\bar p, \bar p'\in R$ that minimize the distance to $p$.
Since $R$ is convex, their midpoint $m=\tfrac12\cdot (\bar p+\bar p')$ lies in~$R$.
Note that $|p-\bar p|\z=|p-\bar p'|=\ell$; that is, the triangle $[p\bar p\bar p']$ is isosceles and therefore the triangle $[p\bar p m]$ is right with the right angle at~$m$.
Since a leg of a right triangle is shorter than its hypotenuse, we have $|p-m|<\ell$ --- a contradiction. 

It remains to prove \ref{eq:short-cpp}.
We can assume that $\bar p\ne\bar q$, otherwise there is nothing to prove.

}

Note that if $\measuredangle \hinge{\bar p}{p}{\bar q}< \tfrac\pi2$, then $\dist{p}{z}{}\z<\dist{p}{\bar p}{}$ for some point $x\in [\bar p\bar q]$.
Since $[\bar p\bar q]\subset K$,
the latter is impossible.

\begin{wrapfigure}{l}{37 mm}
\vskip-0mm
\centering
\includegraphics{mppics/pic-41}
\vskip-0mm
\end{wrapfigure}

Therefore $p=\bar p$ or $\measuredangle \hinge{\bar p}{p}{\bar q}\ge \tfrac\pi2$.
In both cases the orthogonal projection of $p$ to the line $\bar p\bar q$ lies behind $\bar p$, or coincides with $\bar p$.
The same way we show that the orthogonal projection of $q$ to the line $\bar p\bar q$ lies behind $\bar q$, or coincides with $\bar q$.
It implies that the orthogonal projection of the line segment $[pq]$ to the line $\bar p\bar q$ contains the line segment $[\bar p\bar q]$.
In particular 
\[|p-q|\ge |\bar p-\bar q|.\]
\qedsf

\begin{thm}{Corollary}\label{cor:shorts+convex}
Assume a surface $\Sigma$ bounds a closed convex region $R$ and  $p,q \in \Sigma$.
Denote by $W$ the outer closed region of $\Sigma$; in other words $W$ is the union of $\Sigma$ and the complement of $R$.
Then 
\[\length\gamma\ge |p-q|_\Sigma\]
for any path $\gamma$ in $W$ from $p$ to $q$.
Moreover if  $\gamma$ does not lie in $\Sigma$, then the inequality is strict.
\end{thm}

\parit{Proof.}
The first part of the corollary follows from the lemma and the definition of length.
Indeed consider the closest point projection $\bar\gamma$ of~$\gamma$.
Note that $\bar\gamma$ lies in $\Sigma$ and connects $p$ to $q$ therefore 
\[\length\bar\gamma\ge |p-q|_\Sigma.\]

To prove the first statement, it is remains to show that 
\[\length\gamma\ge\length\bar\gamma.\eqlbl{bar-gamma=<gamma}\]

Consider a polygonal line $\bar p_0\dots \bar p_n$ inscribed in $\bar\gamma$.
Let $p_0\dots p_n$ be the corresponding polygonal line inscribed in in $\gamma$;
that is $p_i=\gamma(t_i)$ if $\bar p_i=\bar\gamma(t_i)$.
By \ref{lem:closest-point-projection} $|p_i-p_{i-1}|\z\ge|\bar p_i-\bar p_{i-1}|$ for any $i$.
Therefore 
\[\length p_0\dots p_n\ge \length \bar p_0\dots \bar p_n.\]
Taking least upper bound of each side of the inequality for all inscribed polygonal lines $p_0\dots p_n$ in $\gamma$, we get \ref{bar-gamma=<gamma}.\

\begin{wrapfigure}{o}{37 mm}
\vskip-0mm
\centering
\includegraphics{mppics/pic-82}
\vskip-0mm
\end{wrapfigure}

It remains to prove the second statement.
Suppose that there is a point $w\z=\gamma(t_1)\notin\Sigma$;
note that $w\notin R$.
By the separation lemma (\ref{lem:separation}) there is a plane $\Pi$ that cuts $w$ from $\Sigma$.
The curve $\gamma$ must intersect $\Pi$ at two points: one point before $t_1$ and one after.
Let $a=\gamma(t_0)$ and $b=\gamma(t_2)$ be these points.
Note that the arc of $\gamma$ from $a$ to $b$ is strictly longer that $|a-b|$;
indeed its length is at least $|a-w|+|w-b|$ and $|a-w|+|w-b|>|a-b|$ since $w\notin[ab]$.

Remove from $\gamma$ the arc from $a$ to $b$ and glue in the line segment $[ab]$;
denote the obtained curve by $\gamma_1$. 
From above, we have that
\[\length\gamma>\length \gamma_1\]
Note that $\gamma_1$ runs in $W$.
Therefore by the first part of the corollary, we have
\[\length \gamma_1\ge |p-q|_\Sigma.\]
Whence the second statement follows.
\qeds

\begin{thm}{Exercise}\label{ex:length-dist-conv}
Suppose $\Sigma$ is a proper smooth surface with positive Gauss curvature and $\Norm$ is the unit normal field on $\Sigma$.
Show that for any two points $p,q\in \Sigma$ we have the following inequality:
\[|p-q|_\Sigma\le 2\cdot \frac{|p-q|}{|\Norm(p)+\Norm(q)|}.\]

\end{thm}


\begin{wrapfigure}{r}{27 mm}
\vskip-10mm
\centering
\includegraphics{mppics/pic-240}
\end{wrapfigure}

\begin{thm}{Exercise}\label{ex:hat-convex}
Suppose $\Sigma$ is a closed smooth surface that bounds a convex region $R$ 
in $\RR^3$
and $\Pi$ is a plane that cuts a hat $\Delta$ from $\Sigma$.
Assume that the reflection of the interior of $\Delta$ across $\Pi$ lies in the interior of $R$.
Show that $\Delta$ is \index{convex set}\emph{convex} with respect to the intrinsic metric  of $\Sigma$;
that is, 
if both ends of a shortest path in $\Sigma$ 
lie in $\Delta$,
then the entire path lies in $\Delta$.
\end{thm}


Let us define the \index{intrinsic diameter}\emph{intrinsic diameter} of a closed surface $\Sigma$ as the least upper bound on the lengths of shortest paths in the surface.

\begin{thm}{Exercise}\label{ex:intrinsic-diameter}
Assume that a closed smooth surface $\Sigma$ with positive Gauss curvature lies in a unit ball.

\begin{subthm}{} Show that the intrinsic diameter of $\Sigma$ cannot exceed $\pi$.
 
\end{subthm}

\begin{subthm}{}
Show that the area of $\Sigma$ cannot exceed $4\cdot \pi$.
\end{subthm}

\end{thm}

\chapter{Geodesics}


\section{Definition}

A smooth curve $\gamma$ on a smooth surface $\Sigma$ is called \index{geodesic}\emph{geodesic} if for any~$t$, the acceleration $\gamma''(t)$ is perpendicular to the tangent plane $\T_{\gamma(t)}$.

Physically, geodesics can be understood as the trajectories of a particle that slides on $\Sigma$ without friction.
Indeed, since there is no friction, the force that keeps the particle on $\Sigma$ must be perpendicular to $\Sigma$.
Therefore, by the second Newton's laws of motion,
we get that the acceleration $\gamma''$ is perpendicular to $\T_{\gamma(t)}$.

\begin{thm}{Exercise}\label{ex:helix=geodesic}
Let $\Sigma$ be the cylindrical surface described by the equation $x^2\z+y^2=1$.
Show that one turn of helix $\gamma\:[0,2\cdot\pi]\to \Sigma$ defined by $\gamma(t)=(\cos t,\sin t, t)$
is a geodesic, but not a shortest path on $\Sigma$.
\end{thm}

\begin{thm}{Exercise}\label{ex:reflection-geodesic}
Assume that a smooth surface $\Sigma$ is mirror symmetric with respect to  a plane $\Pi$.
Suppose that $\Sigma$ and $\Pi$ intersect along a smooth regular curve $\gamma$.
Show that $\gamma$ parametrized by its arc-length is a geodesic on $\Sigma$.
\end{thm}

Recall that asymptotic line is defined in Section~\ref{sec:saddle}.

\begin{thm}{Exercise}\label{ex:asymptotic-geodesic}
Suppose that a curve $\gamma$ is a geodesic and, at the same time, is an asymptotic line on a smooth surface $\Sigma$.
Show that $\gamma$ is a line segment.
\end{thm}

\section{Existence and uniqueness}

The following lemma, exercise, and proposition can be interpreted physically as follows.
The lemma follows from the conservation of energy.

\begin{thm}{Lemma}\label{lem:constant-speed}
Any geodesic has constant speed.

More precisely, if $\gamma$ is a geodesic on a smooth surface, then $|\gamma'|$ is constant.
\end{thm}

\parit{Proof.} 
Since $\gamma'(t)$ is a tangent vector at $\gamma(t)$,
we have that $\gamma''(t)\z\perp\gamma'(t)$, or equivalently $\langle\gamma'',\gamma'\rangle=0$ for any $t$.
Whence 
\begin{align*}
\langle\gamma',\gamma'\rangle'&=2\cdot \langle\gamma'',\gamma'\rangle=0.
\end{align*}
That is, $|\gamma'|^2=\langle\gamma',\gamma'\rangle$ is constant.
\qeds

The statement in the following exercise is called \emph{Clairaut's relation};
it can be obtained from the lemma above and the conservation angular momentum.

\begin{thm}{Exercise}\label{ex:clairaut}
Let $\gamma$ be a geodesic on a smooth surface of revolution.
Suppose that $r(t)$ denotes the distance from $\gamma(t)$ to the axis of rotation
and $\theta(t)$ --- the angle between $\gamma'(t)$ and the latitudinal circle thru $\gamma(t)$. 

Show that the value $r(t)\cdot \cos\theta(t)$ is constant. 
\end{thm}


The following proposition gives smooth dependence of trajectory of a particle depending on its initial position and velocity.


\begin{thm}{Proposition}\label{prop:geod-existence} 
Let $\Sigma$ be  a smooth surface without boundary.
Given a tangent vector ${\vec v}$ to $\Sigma$ at a point $p$
there is a unique geodesic $\gamma\:\mathbb{I}\to \Sigma$ defined on a maximal open interval $\mathbb{I}\ni 0$ that starts at $p$ with velocity vector ${\vec v}$;
that is, $\gamma(0)=p$ and $\gamma'(0)={\vec v}$.

Moreover
\begin{subthm}{prop:geod-existence:smooth} the map $(p,{\vec v},t)\mapsto \gamma(t)$ is smooth in its domain of definition.
\end{subthm}

\begin{subthm}{prop:geod-existence:whole} if $\Sigma$ is proper, then $\mathbb{I}=\RR$; that is, the maximal interval is whole real line.
\end{subthm}

\end{thm}

A surface that satisfies the conclusion of \ref{SHORT.prop:geod-existence:whole} for any tangent vector ${\vec v}$ is called \index{geodesically complete}\emph{geodesically complete}.
So part \ref{SHORT.prop:geod-existence:whole} says that any proper surface is geodesically complete.
The latter statement is a part of the \index{Hopf--Rinow theorem}\emph{Hopf--Rinow theorem} \cite{hopf-rinow} which also provides a converse;
moreover, 
\emph{if there is a point $p\in \Sigma$ such that for any tangent vector ${\vec v}\in\T_p\Sigma$ there is a both side infinite geodesic $\gamma$ with $\gamma'(0)={\vec v}$, then $\Sigma$ is proper.}

The proof of this proposition on the existence theorem for an initial value problem (\ref{thm:ODE}).

\begin{thm}{Lemma}\label{lem:geodesic=2nd-order}
Let $f$ be  a smooth function defined on an open domain in $\RR^2$.
A smooth curve $t\mapsto \gamma(t)=(x(t),y(t),z(t))$ is the geodesic in a graph $z=f(x,y)$ if and only if $z(t)=f(x(t),y(t))$ for any $t$ and the functions $t\mapsto x(t)$ and $t\mapsto y(t)$
satisfy a differential equation
\[
\begin{cases}
x''=g(x,y,x',y'),
\\
y''=h(x,y,x',y'),
\end{cases}
\]
where the functions $g$ and $h$ are smooth functions of four variables that determined by $f$.
\end{thm}

The proof of the lemma is done by means of direct calculations.

\parit{Proof.} In the following calculations, we often omit the arguments --- we may write $x$ instead of $x(t)$  and $f$ instead of $f(x,y)$ or $f(x(t),y(t))$ and so on.

First let us calulate $z''(t)$ in terms of $f$, $x(t)$, and $y(t)$.
\[
\begin{aligned}
z''&=f(x,y)''=
\\
&=\left(f_x\cdot x'+ f_y\cdot y'\right)'=
\\
&=
f_{xx}\cdot (x')^2
+
f_x\cdot x''
+
f_{yy}\cdot (y')^2
+
f_y\cdot y''.
\end{aligned}
\eqlbl{eq:def-geod}
\]
Now observe that the equation 
\[\gamma''(t)\perp\T_{\gamma(t)}\] 
means that 
$\gamma''$ is perpendicular to two basis vectors in $\T_{\gamma(t)}$.
Therefore the vector equation \ref{eq:def-geod} can be rewritten as the following system of two real equations
\[
\begin{cases}
\langle \gamma'',s_x\rangle=0,
\\
\langle\gamma'',s_y\rangle=0,
\end{cases}
\]
where $s(x,y)\df (x,y,f(x,y))$, $x=x(t)$, and $y=y(t)$.

Observe that 
$s_x=(1,0, f_x)$ 
and 
$s_y=(0,1, f_y)$.
Since $\gamma''\z=(x'',y'',z'')$, we can rewrite the system the following way.
\[
\begin{cases}
x''+ f_x\cdot z''=0,
\\
y''+ f_y\cdot z''=0,
\end{cases}
\]
It remains use expression \ref{eq:z''} for $z''$, combine like terms and simplify.
\qeds


\parit{Proof of \ref{prop:geod-existence}.}
Let $z=f(x,y)$ be a description of $\Sigma$ in a tangent-normal coordinates at $p$.
By Lemma \ref{lem:geodesic=2nd-order} the condition $\gamma''(t)\perp\T_{\gamma(t)}$ can be written as a second order differential equation.
Applying the existence and uniqueness of the initial value problem (\ref{thm:ODE}) we get existence and uniqueness of geodesic $\gamma$ in a in a small interval $(-\eps,\eps)$ for some $\eps>0$.

Let us extend $\gamma$ to a maximal open interval $\mathbb{I}$.
Suppose there is another geodesic $\gamma_1$ with the same initial data that is defined on a maximal open interval $\mathbb{I}_1$.
Suppose $\gamma_1$ splits from $\gamma$ at some time $t_0>0$;
that is, $\gamma_1$ coincides with $\gamma$ on the interval $[0,t_0)$, but they are different on the interval $[0,t_0+\eps)$ for any $\eps>0$.
By continuity $\gamma_1(t_0)=\gamma(t_0)$ and $\gamma'(t_0)=\gamma'(t_0)$.
Applying uniqueness of the initial value problem (\ref{thm:ODE}) again, we get that $\gamma_1$ coincides with $\gamma$ in a small neighborhood of $t_0$ --- a contradiction.

The case $t_0<0$ can be proved along the same lines.
It follows that $\gamma_1=\gamma$;
in particular, $\mathbb{I}_1=\mathbb{I}$.

Part \ref{SHORT.prop:geod-existence:smooth} follows since the solution of the initial value problem depends smoothly on the initial data (\ref{thm:ODE}).

Suppose \ref{SHORT.prop:geod-existence:whole} does not hold;
that is, the maximal interval $\mathbb{I}$ is a proper subset of the real line $\RR$.
Without loss of generality we may assume that $b=\sup\mathbb{I}<\infty$.
(If not switch the direction of $\gamma$.)

By \ref{lem:constant-speed} $|\gamma'|$ is constant, in particular $t\mapsto \gamma(t)$ is a uniformly continuous function.
Therefore  the limit point
\[q=\lim_{t\to b}\gamma(t)\] 
is defined.
Since $\Sigma$ is a proper surface, $q\in \Sigma$. 

Applying the argument above in a tangent-normal coordinates at $q$ shows that $\gamma$ can be extended as a geodesic behind $q$.
Therefore $\mathbb{I}$ is not a maximal interval --- a contradiction.
\qeds

\begin{thm}{Exercise}\label{ex:round-torus}
Let $\Sigma$ be a smooth torus of revolution; that is,
a smooth surface of revolution with closed generatrix.
Show that any closed geodesic on $\Sigma$ is noncontractible.

(In other words, if $s\:\RR^2\to \Sigma$ is the natural bi-periodic parameterization of $\Sigma$, then
there is no closed curve $\gamma$ in $\RR^2$ such that $s\circ\gamma$ is a geodesic.)
\end{thm}


\section{Exponential map}\label{sec:exp}

Let $\Sigma$ be a smooth regular surface and $p\in \Sigma$.
Given a tangent vector ${\vec v}\in \T_p$, consider a geodesic $\gamma_{\vec v}$ in $\Sigma$ that runs from $p$ with the initial velocity ${\vec v}$;  
that is, $\gamma(0)=p$ and $\gamma'(0)={\vec v}$.

The map 
\[\exp_p\:\vec v\mapsto \gamma_{\vec v}(1)\]
is called \index{exponential map}\emph{exponential}.%
\footnote{There is a good reason to call this map {}\emph{exponential}, but it is far from the subject.}
By \ref{prop:geod-existence}, the map $\exp_p\:\T_p\to \Sigma$ is smooth and defined in a neighborhood of zero in $\T_p$;
moreover, if $\Sigma$ is proper, then $\exp_p$ is defined on the whole space $\T_p$.

Note that the exponential map $\exp_p$ 
is defined on the tangent plane $\T_p$, which is a smooth surfaces,
and its target is another smooth surface $\Sigma$.
Observe that one can identify the plane $\T_p$
with its tangent plane $\T_0\T_p$ so the differenial $d_0(\exp_p)\:\vec v\mapsto D_{\vec v}\exp_p$ maps $\T_p$ to itself.
Further note that by the definition of exponential map we have that this differential is the identity map; that is, $d_0\exp_p(\vec v)=
\vec v$ for any $\vec v\in \T_p$.

Summarizing, we get the following statement:

\begin{thm}{Observation}\label{obs:d(exp)=1}
Let $\Sigma$ be a smooth surface and $p\in \Sigma$.
Then 
\begin{subthm}{}
$\exp_p$ is a smooth map and its domain contains a neighborhood of the origin in $\T_p$, 
\end{subthm}

\begin{subthm}{}
the differential $d_0(\exp_p)\:\T_p\to \T_p$ is the identity map.
\end{subthm}

\end{thm}

In fact it is easy to see that $\Dom(\exp_p)$ --- the domain of definition of $\exp_p$ --- is an open \index{star-shaped}\emph{star-shaped} region of $\T_p$;
the latter means that if $\vec v\in \Dom(\exp_p)$, then $\lambda\cdot\vec v\in \Dom(\exp_p)$ for any $0\le \lambda\le 1$.
Note also that \ref{prop:geod-existence:whole} implies that \emph{if $\Sigma$ is proper, then $\Dom(\exp_p)=\T_p$}.


\begin{thm}{Proposition}\label{prop:exp}
Let $\Sigma$ be smooth surface (without boundary) and $p\in \Sigma$.
Then there is $r_p>0$ such that
the exponential map $\exp_p$ is defined on the open ball $B=B(0,r_p)_{\T_p}$
and the restriction $\exp_p|_B$ is a smooth regular parametrization of a neighborhood of $p$ in $\Sigma$.

Moreover we have a {}\emph{local control} on $r_p$;
that is, for any $q\in \Sigma$ there is $\eps>0$ such that if $|q-p|_\Sigma<\eps$, then $r_p\ge\eps$.
\end{thm}

The proof of the proposition uses the observation and the inverse function theorem (\ref{thm:inverse}).

\parit{Proof.}
Let $z=f(x,y)$ be a local graph representation of $\Sigma$ in the tangent-normal coordinates at $p$.
Note that $(x,y)$-plane coincides with the tangent plane $\T_p$.

Denote by $s$ the composition of  the exponential map $\exp_p$ with the orthogonal projection $(x,y,z)\mapsto (x,y)$.
By \ref{obs:d(exp)=1}, the differential $d_0s$ is the identity;
in other words, this map has identity Jacobian matrix at $0$.
Applying the inverse function theorem (\ref{thm:inverse}) we get the first part of the proposition.

The second part can be proved along the same lines, using the second part the inverse function theorem (\ref{thm:inverse});
it guarantees that the size of the neighborhood in $\T_p$ for all points $p$ sufficiently close to~$q$.
\qeds

Given $p\in \Sigma$, the least upper bound on $r_p$ that satisfies \ref{prop:exp} is called \index{injectivity radius}\emph{injectivity radius} of $\Sigma$ at $p$;
it is denoted by $\inj(p)$.
The proposition states that {}\emph{injectivity radius is positive and locally bounded away from zero}.

In fact the function {}\emph{$\inj\:\Sigma\to (0,\infty]$ is continuous};
the latter was proved by Wilhelm Klingenberg \cite[5.4]{gromoll-klingenberg-meyer}. 

The proof of the following statement will be indicated in \ref{ex:inj-rad}.

\begin{thm}{Proposition}\label{prop:inj-rad}
Let $\Sigma$ be smooth surface (without boundary) and $p\in \Sigma$.
If $\exp_p$ is injective in $B(0,r)_{\T_p}$, then the restriction $\exp_p|_{B(0,r)}$ is a diffeomorphism to its image.

In other words, the injectivity radius at $p$ can be defined as the least upper bound on $r$ such that $\exp_p$ is injective in the ball $B(0,r)_{\T_p}$.
\end{thm}

\section{Shortest paths are geodesics}

\begin{thm}{Proposition}\label{prop:gamma''}
Let $\Sigma$ be a smooth regular surface.
Then any shortest path $\gamma$ in $\Sigma$ parametrized proportional to its arc-length is a geodesic in $\Sigma$.
In particular $\gamma$ is a smooth curve.

A partial converse to the first statement also holds: a sufficiently short arc of any geodesic is a shortest path.
More precisely, any point $p$ in $\Sigma$ has a neighborhood $U$ such that any geodesic that lies completely in $U$ is a shortest path.
\end{thm}

A geodesic might not form a shortest path, but if this is the case, then it is called a \index{minimizing geodesic}\emph{minimizing geodesic}.
Note that according to the proposition, any shortest path is a reparametrization of a minimizing geodesic.

A formal proof will be given much latter; see Section~\ref{sec:proof-of-gamma''}.

The following informal physical explanation might be sufficiently convincing.
In fact, if one assumes that $\gamma$ is smooth, then it is easy to convert this explanation into a rigorous proof.

\parit{Informal explanation.}
Let us think about a shortest path $\gamma$ as of stable position of a stretched elastic thread that is forced to lie on a frictionless surface.
Since it is frictionless, the force density $\vec n=\vec n(t)$ that keeps $\gamma$ in the surface must be proportional to the normal vector to the surface at $\gamma(t)$.

The tension in the thread has to be the same at all points (otherwise the thread would move back or forth and it would not be stable).
Denote  the tension by $\tau$.

We can assume that $\gamma$ has unit speed;
in this case the net force from tension to the arc $\gamma_{[t_0,t_1]}$ is $\tau\cdot(\gamma'(t_1)-\gamma'(t_0)$.
Hence the density of net force from tension at $t_0$ is 
\begin{align*}
\vec f(t_0)&=\lim_{t_1\to t_0}\tau\cdot\frac{\gamma'(t_1)-\gamma'(t_0)}{t_1-t_0}=
\\
&=\tau\cdot\gamma''(t_0).
\end{align*}
According to the second Newton's law of motion, we have 
$\vec f+\vec n=0$.
The latter implies that  $\gamma''(t)\perp\T_{\gamma(t)}\Sigma$.
\qeds

\begin{thm}{Corollary}
Let $\Sigma$ be a smooth regular surface, $p\in\Sigma$ and $r\z\le \inj(p)$.
Then  the exponential map $\exp_p$ is a diffeomorphism from $B(0,r)_{\T_p}$ to $B(p,r)_\Sigma$.
\end{thm}

\parit{Proof.}
By the definition of injectivity radius, the restriction of $\exp_p$ to $B={B(0,r)_{\T_p}}$ is a diffeomorphism to its image $\exp_p(B)$.

Evidently $B(p,r)_\Sigma\supset\exp_p(B)$.
By \ref{prop:gamma''}, $B(p,r)_\Sigma\subset\exp_p(B)$, hence the result.
\qeds

According to the corollary, the restriction $\exp_p|_{\T_p}$ admits an inverse map that is called \emph{logarithmic map at $p$};
it is denoted by \[\log_p\:B(p,r)_\Sigma\to B(0,r)_{\T_p}.\]

Note that according to the proposition above, any shortest path parametrized by its arc-length is a smooth curve.
This observation should help to solve in the following two exercises.


\begin{thm}{Exercise}\label{ex:two-min-geod}
Show that two shortest paths can cross each other at most once.
More precisely, if two shortest paths have two distinct common points $p$ and $q$, then either these points are the ends of both shortest paths or both shortest paths contain an arc from $p$ to $q$.

Show by example that nonoverlapping geodesics can cross each other an arbitrary number of times.
\end{thm}

\begin{thm}{Exercise}\label{ex:min-geod+plane}
Assume that a smooth regular surface $\Sigma$ is mirror symmetric with respect to a plane $\Pi$.
Show that no shortest path $\alpha$ in $\Sigma$ can {}\emph{cross} $\Pi$ more than once.


In other words, if you travel along $\alpha$, then you change the sides of $\Pi$ at most once. 
\end{thm}

{

\begin{wrapfigure}{r}{40 mm}
\vskip-8mm
\centering
\includegraphics{mppics/pic-250}
\vskip-0mm
\end{wrapfigure}

\begin{thm}{Advanced exercise}\label{ex:milka}
Let $\Sigma$ be a smooth closed strictly convex surface 
in $\RR^3$ 
and $\gamma\:[0,\ell]\z\to \Sigma$ be a unit-speed minimizing geodesic.
Set $p\z=\gamma(0)$, $q=\gamma(\ell)$ and 
$$p^t=\gamma(t)-t\cdot\gamma'(t),$$ 
where $\gamma'(t)$ denotes the velocity vector of $\gamma$ at $t$.

Show that for any $t\in (0,\ell)$,
one {}\emph{cannot see}  $q$ from $p^t$;
that is, the line segment $[p^tq]$ intersects $\Sigma$ at a point distinct from $q$.

Show that the statement does not hold without assuming that $\gamma$ is minimizing.
\end{thm}

}

\section{Liberman's lemma}

The following lemma is a smooth analog of lemma proved by Joseph Liberman \cite{liberman}.

\begin{thm}{Liberman's lemma}\label{lem:liberman}
Let $f$ be a smooth convex function  defined on an open subset of the plane.
Suppose that $t\mapsto \gamma(t)\z=(x(t),y(t),z(t))$ is a unit-speed geodesic on the graph $z=f(x,y)$.
Then $t\mapsto z(t)$ is a convex function; that is, $z''(t)\ge 0$ for any $t$.
\end{thm}

\parit{Proof.}
Choose the orientation on the graph so that the unit normal vector $\Norm$ always points up;
that is, $\Norm$ has positive $z$-coordinate.
Let us use shortcut $\Norm(t)$ for $\Norm(\gamma(t))$.

Since $\gamma$ is a geodesic, we have $\gamma''(t)\perp\T_{\gamma(t)}$,
or equivalently $\gamma''(t)$ is proportional to $\Norm(t)$ for any $t$.
Further 
\[\gamma''=k\cdot\Norm,\]
where $k=k(t)$ is the normal curvature at $\gamma(t)$ in the direction of $\gamma'(t)$.

Therefore
\[z''=k\cdot\cos\theta,
\eqlbl{eq:z''}\]
where $\theta=\theta(t)$ denotes the angle between $\Norm(t)$ and the $z$-axis.

Since $\Norm$ points up, we have $\theta(t)<\tfrac\pi2$, or equivalently
\[\cos\theta>0.\]

Since $f$ is convex, we have that the tangent plane supports the graph from below at any point;
in particular $k(t)\ge 0$ for any $t$.
It follows that the right hand side in \ref{eq:z''} is nonnegative;
whence the statement follows.
\qeds

\begin{thm}{Exercise}\label{ex:rho''}
Assume $\gamma$ is a unit-speed geodesic on a smooth convex surface $\Sigma$ and a point $p$ lies in the interior of a convex set bounded by $\Sigma$.
Set $\rho(t)=|p-\gamma(t)|^2$.
Show that $\rho''(t)\le 2$ for any $t$.
\end{thm}



\section{Total curvature of geodesics}

Recall that $\tc\gamma$ denotes the total curvature of curve $\gamma$.

\begin{thm}{Exercise}\label{ex:tc-spherical-image}
Let $\gamma$ be a geodesic on an oriented smooth surface $\Sigma$
with unit normal field $\Norm$.
Show that 
\[\length(\Norm\circ\gamma)\ge \tc\gamma.\]
\end{thm}


\begin{thm}{Theorem}\label{thm:usov}
Assume $\Sigma$ is a graph $z=f(x,y)$ of a convex $\ell$-Lipschitz function $f$ defined on an open set in the $(x,y)$-plane.
Then the total curvature of any geodesic in $\Sigma$ is at most $2\cdot \ell$.
\end{thm}

This theorem proved by Vladimir Usov \cite{usov},
an amusing generalization was found by David Berg \cite{berg}.

\parit{Proof.}
Let $t\mapsto\gamma(t)=(x(t),y(t),z(t))$ be a unit-speed geodesic on $\Sigma$.
According to Liberman's lemma, the function
$t\mapsto z(t)$ is convex.

Since the slope of $f$ is at most $\ell$, we have
\[|z'(t)|\le \tfrac{\ell}{\sqrt{1+\ell^2}}.\]
If $\gamma$ is defined on the interval $[a,b]$, then
\[
\begin{aligned}
\int_a^b z''(t)&=z'(b)-z'(a)\le 
\\
&\le 2\cdot \tfrac{\ell}{\sqrt{1+\ell^2}}.
\end{aligned}
\eqlbl{eq:intz''}
\]

Further, note that $z''$ is the projection of $\gamma''$ to the $z$-axis.
Since $f$ is $\ell$-Lipschitz, the tangent plane $\T_{\gamma (t)} \Sigma$ cannot have slope greater than $\ell$ for any $t$.
Because $\gamma ''$ is perpendicular to that plane, we have that
\[|\gamma'' (t)|  \le  z''(t)\cdot\sqrt{1+ \ell ^2}.\]

By \ref{eq:intz''}, we get that
\begin{align*}
\tc\gamma&=\int_a^b|\gamma'' (t)|\cdot dt\le 
\\
&\le \sqrt{1+ \ell ^2}\cdot  \int_a^b z''(t)\cdot dt\le 
\\
&\le 2\cdot \ell.
\end{align*}
\qedsf

\begin{thm}{Exercise}\label{ex:usov-exact}
Note that the graph $z=\ell\cdot\sqrt{x^2+y^2}$ with removed origin is a smooth surface; denote it by $\Sigma$.
Show that any both side infinite geodesic $\gamma$ in $\Sigma$ has total curvature exactly $2\cdot \ell$.
\end{thm}

Note that the last exercise implies that the estimate in the Usov's theorem is optimal. Smooth the function $f(x,y)=\ell\cdot\sqrt{x^2+y^2}$ in a small neighborhood of the origin keeping it convex and $\ell$-Lipschitz,
and note that we can assume that the geodesic $\gamma$ does not enter the smoothed part of the graph.


\begin{thm}{Exercise}\label{ex:rough-bound-mountain}
Assume $f$ is a convex $\tfrac32$-Lipschitz function defined on the $(x,y)$-plane.
Show that any geodesic $\gamma$ on the graph $z=f(x,y)$ is simple;
that is, it has no self-intersections.

Construct a convex $2$-Lipschitz function defined on the $(x,y)$-plane
with a nonsimple geodesic $\gamma$ on its graph $z=f(x,y)$.
\end{thm}


\begin{thm}{Theorem}\label{thm:tc-of-mingeod}
Suppose a smooth surface $\Sigma$ bounds a convex set $K$ in the Euclidean space.
Assume $B(0,\eps)\subset K\subset B(0,1)$.
Then the total curvatures of any shortest path in $\Sigma$ can be bounded in terms of~$\eps$. 
\end{thm}

The following exercise will guide you thru the proof of the theorem. 

\begin{wrapfigure}{r}{48 mm}
\vskip-0mm
\centering
\includegraphics{mppics/pic-83}
\vskip-0mm
\end{wrapfigure}

\begin{thm}{Exercise}\label{ex:bound-tc}
Let $\Sigma$ be as in the theorem and $\gamma$ be a unit-speed shortest path in $\Sigma$.
Denote by $\Norm_p$ the unit normal vector that points outside of $\Sigma$;
denote by $\theta_p$ the angle between $\Norm_p$ and the direction from the origin to a point $p\in\Sigma$.
Set $\rho(t)\z=|\gamma(t)|^2$; denote by $k(t)$ the curvature of $\gamma$ at $t$.

\begin{subthm}{}
Show that $\cos\theta_p\ge \eps$ for any $p\z\in \Sigma$.
\end{subthm}

\begin{subthm}{}
 Show that $|\rho'(t)|\le 2$ for any $t$.
\end{subthm}

\begin{subthm}{}
 Show that 
\[\rho''(t)=2-2\cdot k(t)\cdot \cos \theta_{\gamma(t)}\cdot |\gamma(t)|\]
for any $t$.
\end{subthm}

\begin{subthm}{}
 Use the closest-point projection from the unit sphere to $\Sigma$ to show that 
\[\length \gamma\le \pi.\]
\end{subthm}

\begin{subthm}{}
 Use the statements above to conclude that 
\[\tc\gamma\le \frac{100}{\eps^2}.\]
\end{subthm}

\end{thm}

Note that the obtained bound on total curvature goes to infinity as $\eps\to 0$.
In fact there is a bound that is independent of $\eps$ \cite{lebedeva-petrunin}.
(According to Exercise~\ref{ex:tc-spherical-image}, it would also follow if the length of spherical image of $\gamma$ can be bounded above; 
that is, if $\length(\Norm\circ\gamma)\le C$ for a universal constant $C$.
The latter was conjectured by Aleksei Pogorelov \cite{pogorelov};
counterexamples to the different forms of this conjecture were found 
by Viktor Zalgaller in \cite{zalgaller},
Anatoliy Milka in \cite{milka}
and Vladimir Usov in \cite{usov};
these results were partly rediscovered later 
by J\'{a}nos Pach \cite{pach}.)
