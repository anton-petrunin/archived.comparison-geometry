\chapter{Geodesics}

The following exercise might look like a hard problem in calculus, but actually it is an easy problem in geometry.

%???angle of inclination

\begin{wrapfigure}{r}{43 mm}
\vskip-0mm
\centering
\includegraphics{mppics/pic-77}
\vskip-0mm
\end{wrapfigure}

\begin{thm}{Exercise}\label{ex:lasso}
There is a mountain of frictionless ice with the shape of a perfect cone with a circular base.
A cowboy is at the bottom and he wants to climb the mountain.
So, he throws up his lasso which slips neatly over the top of the cone, he pulls it tight and starts to climb.
If the angle of inclination $\theta$ is large, there is no problem; the lasso grips tight and up he goes.
On the other hand if the angle of inclination $\theta$ is small, the lasso slips off as soon as the cowboy pulls on it.

What is the critical angle $\theta_0$ at which the cowboy can no longer climb the ice-mountain?
\end{thm}




