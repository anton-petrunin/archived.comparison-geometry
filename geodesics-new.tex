\chapter{Geodesics}

We start to study the intrinsic geometry of surfaces.
The following exercise should help you to be in the right mood for this;
it might look like a tedious problem in calculus, but actually it is an easy problem in geometry.

%???angle of inclination

\begin{wrapfigure}{r}{33 mm}
\vskip-0mm
\centering
\includegraphics{mppics/pic-77}
\vskip-0mm
\end{wrapfigure}

\begin{thm}{Exercise}\label{ex:lasso}
There is a mountain of frictionless ice with the shape of a perfect cone with a circular base.
A cowboy is at the bottom and he wants to climb the mountain.
So, he throws up his lasso which slips neatly over the top of the cone, he pulls it tight and starts to climb.
If the angle of inclination $\theta$ is large, there is no problem; the lasso grips tight and up he goes.
On the other hand if $\theta$ is small, the lasso slips off as soon as the cowboy pulls on it.

What is the critical angle $\theta_0$ at which the cowboy can no longer climb the ice-mountain?
\end{thm}

\parit{Hint:} Cut the lateral surface of the mountain by a line from the cowboy to the top, unfold it on the plane and try to figure out what is the image of the strained lasso.

\section*{Shortest paths}

Let $p$ and $q$ be two points on a surface $\Sigma$.
Recall that $|p-q|_\Sigma$ denotes the induced length distance from $p$ to $q$;
that is, the exact lower bound on lengths of paths in $\Sigma$ from $p$ to $q$.

Note that if $\Sigma$ is smooth, then any two points in $\Sigma$ can be joined by a piecewise smooth path.
Since any such path is rectifiable, the value $|p-q|_\Sigma$ is finite for any pair of points $p,q\in\Sigma$.

A path $\gamma$ from $p$ to $q$ in $\Sigma$ that minimize the length is called a \emph{shortest path} from $p$ to $q$.

The image of a shortest path between $p$ and $q$ in $\Sigma$ is usually denoted by $[p,q]_\Sigma$.
In general there might be no shortest path between two given points on the surface
and it might many of them;
this is shown in the following two examples.
However if we write $[p,q]_\Sigma$, then we assume that a shortest path exists and we made a choice of one of them.

\begin{wrapfigure}{r}{28 mm}
\vskip-4mm
\centering
\includegraphics{asy/sphere}
\vskip-3mm
\end{wrapfigure}

\parbf{Nonuniqueness.} There plenty of shortest paths between the poles on the sphere --- each meridian is a shortest path.

\parbf{Nonexistence.} Let $\Sigma$ be the $(x,y)$-plane with removed origin.
Consider two points $p=(1,0,0)$ and $q=(-1,0,0)$ in $\Sigma$.

Note that $|p-q|_\Sigma=2$. 
Indeed, given $\eps\z>0$, consider the point $s_\eps=(0,\eps,0)$.
Note that the polygonal path $ps_\eps q$ lies in $\Sigma$ and its length $2\cdot\sqrt{1+\eps^2}$ approaches $2$ as $\eps\to0$.
It follows that $|p-q|_\Sigma\le 2$.
On the other hand $|p-q|_\Sigma\ge |p-q|_{\RR^3}=2$; that is, $|p-q|_\Sigma= 2$.

\begin{wrapfigure}{r}{28 mm}
\vskip-0mm
\centering
\includegraphics{mppics/pic-79}
\vskip-0mm
\end{wrapfigure}

Therefore a shortest path from $p$ to $q$ (if it exists) must have length 2.
By triangle inequality any curve of length 2 from $p$ to $q$ must run along the line segment $[p,q]$;
in particular it must pass thru the origin.
Since the origin does not lie in $\Sigma$, there is no shortest from $p$ to $q$ in $\Sigma$ 

\begin{thm}{Proposition}
Any two points in a complete smooth surface can be joined by a shortest path. 
\end{thm}

\parit{Proof.}
Fix a complete smooth surface $\Sigma$ with two points $p$ and $q$.
Set $\ell=|p-q|_\Sigma$.

By the definition of induced length metric,
there is a sequence of paths $\gamma_n$ from $p$ to $q$ in $\Sigma$ such that
\[\length\gamma_n\to \ell\quad\text{as}\quad n\to \infty.\]

Without loss of generality, we may assume that $\length\gamma_n<\ell+1$ for any $n$ and each $\gamma_n$ is parameterized proportional to its arc length.
In particular each path $\gamma_n\:[0,1]\to\Sigma$ is $(\ell+1)$-Lipschitz; 
that is,
\[|\gamma(t_0)-\gamma(t_1)|\le (\ell+1)\cdot|t_0-t_1|\]
for any $t_0,t_1\in[0,1]$.
Further the image of $\gamma_n$ lies in the closed ball $\bar B[p,\ell+1]$ for any $n$.
It follows that the coordinate functions of $\gamma_n$ are uniformly equicontinuous and uniformly bounded.
By \ref{lem:equicontinuous}, we can pass to a converging subsequence of $\gamma_n$;
denote by $\gamma_\infty\:[0,1]\to\RR^3$ its limit.
As a limit of uniformly continuous sequence, $\gamma_\infty$ is continuous;
that is, $\gamma_\infty$ is a path.
Evidently $\gamma_\infty$ runs from $p$ to $q$.
Since $\Sigma$ is a closed set, $\gamma_\infty$ lies in $\Sigma$.
Finally, by \ref{thm:length-semicont}, 
\[\gamma_\infty\le \ell;\]
that is, $\gamma_\infty$ is a shortest path from $p$ to $q$.\qeds

\section*{Closest point projection}

\begin{thm}{Lemma}\label{lem:closest-point-projection}
Let $R$ be a closed convex set in $\RR^3$.
Then for every point $p\in\RR^3$ there is unique point $\bar p\in R$ that minimizes the distance $|p-x|$ among all points $x\in R$.

Moreover the map $p\mapsto \bar p$ is short;
that is,
\[|p-q|\ge|\bar p-\bar q| \eqlbl{eq:short-cpp}\]
for any pair of points $p,q\in \RR^3$.
\end{thm}

The map $p\mapsto \bar p$ is called the \emph{closest point projection};
it maps the Euclidean space to $R$.
Note that if $p\in R$, then $\bar p=p$.

\parit{Proof.}
Fix a point $p$ and set 
\[\ell=\inf\set{|p-x|}{x\in R}.\]
Choose a sequence $x_n\in R$ such that $|p-x_n|\to \ell$ as $n\to\infty$.

\begin{wrapfigure}{i}{22 mm}
\vskip-0mm
\centering
\includegraphics{mppics/pic-40}
\vskip-0mm
\end{wrapfigure}

Without loss of generality, we can assume that all the points $x_n$ lie in a ball or radius $\ell+1$ centered at~$p$.
Therefore we can pass to a partial limit $\bar p$ of $x_n$; that is, $\bar p$ is a limit of a subsequence of $x_n$.
Since $R$ is closed $\bar p\in R$.
By construction 
\begin{align*}
|p-\bar p|&=\lim_{n\to\infty}|p-x_n|=
\\
&=\ell.
\end{align*}
Hence the existence follows.

Assume there are two distinct points $\bar p, \bar p'\in R$ that minimize the distance to $p$.
Since $R$ is convex, their midpoint $m=\tfrac12\cdot (\bar p+\bar p')$ lies in~$R$.
Note that $|p-\bar p|\z=|p-\bar p'|=\ell$; that is $\triangle p\bar p\bar p'$ is isosceles and therefore $\triangle p\bar p m$ is right with the right angle at~$m$.
Since a leg of a right triangle is shorter than its hypotenuse, we have $|p-m|<\ell$ --- a contradiction. 

It remains to prove inequality \ref{eq:short-cpp}.

\begin{wrapfigure}{o}{37 mm}
\vskip-0mm
\centering
\includegraphics{mppics/pic-41}
\vskip-0mm
\end{wrapfigure}

We can assume that $\bar p\ne\bar q$, otherwise there is nothing to prove.
Note that if $p\ne \bar p$ (that is, if $p\notin R$), 
then $\angle p \bar p \bar q$ is right or obtuse.
Otherwise there would be a point $x$ on the line segment $[\bar q,\bar p]$ that is closer to $p$ than $\bar p$.
Since $R$ is convex, the line segment $[\bar q,\bar p]$ and therefore $x$ lie in $R$.
Hence $\bar p$ is not closest to $p$ --- a contradiction.

The same way we can show that  if $q\z\ne \bar q$, then $\angle q \bar q \bar p$ is right or obtuse.

We have to consider the following 4 cases:
(1) $p\ne \bar p$ and $q\ne \bar q$,
(2) $p= \bar p$ and $q\ne \bar q$,
(3) $p\ne \bar p$ and $q= \bar q$,
(4) $p= \bar p$ and $q= \bar q$.
In all these cases the obtained angle estimates imply that the orthogonal projection of the line segment $[p,q]$ to the line $\bar p\bar q$ contains the line segment $[\bar p,\bar q]$.
In particular 
\[|p-q|\ge |\bar p-\bar q|.\]
\qedsf

\begin{thm}{Corollary}
Assume a surface $\Sigma$ bounds a closed convex region $R$ and  $p,q \in \Sigma$.
Denote by $W$ the outer closed region of $\Sigma$; in other words $W$ is the union of $\Sigma$ and the complement of $R$.
Then for any curve $\gamma$ in $W$ that runs from $p$ to $q$ we have
\[\length\gamma\ge |p-q|_\Sigma.\]
Moreover if  $\gamma$ does not lie in $\Sigma$, then the inequality is strict.
\end{thm}

\parit{Proof.}
The first part of the corollary follows from the lemma and the definition of length.
Indeed consider the closest point projection $\bar\gamma$ of~$\gamma$.
Note that $\bar\gamma$ lies in $\Sigma$ and connects $p$ to $q$ therefore 
\[\length\bar\gamma\ge |p-q|_\Sigma.\]

Consider an inscribed polygonal line $p_0\dots p_n$ in $\gamma$.
Denote by $\bar p_i$ the closest point projection of $p_i$ to $R$.
Note that the polygonal line  $\bar p_0\dots \bar p_n$ is inscribed in $\bar\gamma$;
moreover any inscribed polygonal line in $\bar\gamma$ can appear this way.
By \ref{lem:closest-point-projection} $|p_i-p_{i-1}|\z\ge|\bar p_i-\bar p_{i-1}|$ for any $i$.
Therefore 
\[\length p_0\dots p_n\ge \length \bar p_0\dots \bar p_n.\]
Taking least upper bound of each side of the inequality for all inscribed polygonal lines $p_0\dots p_n$ in $\gamma$, we get
\[\length\gamma\ge\length\bar\gamma.\]
Whence the first statement follows.

\begin{wrapfigure}{o}{37 mm}
\vskip-0mm
\centering
\includegraphics{mppics/pic-82}
\vskip-0mm
\end{wrapfigure}

To prove the second statement, note that if $s=\gamma(t_1)\notin\Sigma$,
then $s\notin R$.
Hence there is a plane $\Pi$ that cuts $s$ from $\Sigma$.
The curve $\gamma$ must intersect at least at two points: one point before $t_1$ and one after;
let $a=\gamma(t_0)$ and $b=\gamma(t_2)$ be these points.
Note that the arc of $\gamma$ from $a$ to $b$ is strictly longer that $|a-b|$;
indeed on the way $\gamma$ visits $s$ that is not on the plane $\Pi$ and therefore not on the lie segment $[a,b]$.

Remove from $\gamma$ the arc from $a$ to $b$ and glue in the line segment $[a,b]$;
denote the obtained curve by $\gamma_1$. 
From above,
\[\length\gamma>\length \gamma_1\]
Note that $\gamma_1$ runs in $W$.
Therefore by the first part of corollary, we have
\[\length \gamma_1\ge |p-q|_\Sigma.\]
Whence the second statement follows.
\qeds

\begin{wrapfigure}{r}{25 mm}
\begin{lpic}[t(-0 mm),b(-4 mm),r(0 mm),l(0 mm)]{pics/convex-hat(1)}
\lbl{15,8.6;$\Sigma$}
\lbl[w]{5,18.6;$\Delta$}
\end{lpic}
\end{wrapfigure}

\begin{thm}{Exercise}\label{ex:hat-convex}
Suppose $\Sigma$ is a complete smooth surface that bounds a convex region $R$ 
in $\RR^3$
and $\Pi$ is a plane that cuts a hat $\Delta$ from $\Sigma$.
Assume that the reflection of the interior of $\Delta$ with respect to $\Pi$ lies in the interior of $R$.
Show that $\Delta$ is \index{convex set}\emph{convex} with respect to the intrinsic metric  of $\Sigma$;
that is, 
if both ends of a shortest path in $\Sigma$ 
lie in $\Delta$,
then the entire geodesic lies in $\Delta$.
\end{thm}


Let us define the \emph{intrinsic diameter} of a closed surface $\Sigma$ as the exact upper bound on the lengths of shortest paths in the surface.

\begin{thm}{Exercise}\label{ex:intrinsic-diameter}
Assume that a closed smooth surface $\Sigma$ with positive Gauss curvature lies in a unit ball.
Show that the intrinsic diameter of $\Sigma$ cannot exceed $\pi$.
\end{thm}

\parit{Hint:} Use \ref{lem:closest-point-projection}.


\section*{Geodesics}

A smooth curve $\gamma$ on a smooth surface $\Sigma$ is called \emph{geodesic} if its acceleration $\gamma''(t)$ is perpendicular to the tangent plane $\T_{\gamma(t)}$ for each~$t$.

Geodesics can be understood as the trajectories of a particle that slides on $\Sigma$ without friction.
In this case the force that keeps the particle on $\Sigma$ must be perpendicular to $\Sigma$.
By the second Newton's laws of motion,
we get that the acceleration $\gamma''$ is perpendicular to $\T_{\gamma(t)}$.

\begin{thm}{Exercise}\label{ex:reflection-geodesic}
Assume that a smooth surface $\Sigma$ is mirror symmetric with respect to  a plane $\Pi$.
Suppose that $\Sigma$ and $\Pi$ intersect along a curve $\gamma$.
Show that $\gamma$ is a geodesic of $\Sigma$.
\end{thm}

\begin{thm}{Lemma}\label{lem:constant-speed}
Any geodesic $\gamma$ has constant speed; that is $|\gamma'(t)|$ is constant.
\end{thm}

\parit{Proof.} 
Since $\gamma'(t)$ is a tangent vector at $\gamma(t)$,
we have that $\gamma''(t)\z\perp\gamma'(t)$, or equivalently $\langle\gamma'',\gamma'\rangle=0$ for any $t$.
Whence 
\begin{align*}
\langle\gamma',\gamma'\rangle'&=2\cdot \langle\gamma'',\gamma'\rangle=
\\
&=0.
\end{align*}
That is $|\gamma'(t)|^2=\langle\gamma'(t),\gamma'(t)\rangle$ is constant.
\qeds

\begin{thm}{Proposition}\label{prop:geod-existance} 
Given a tangent vector $v$ to a smooth surface $\Sigma$ at a point $p$
There is a unique geodesic $\gamma\:\mathbb{I}\to \Sigma$ defined on a maximal open interval $\mathbb{I}\ni 0$ that starts at $p$ with velocity vector $v$;
that is, $\gamma(0)=p$ and $\gamma'(0)=v$.

Moreover
\begin{enumerate}[(a)]
\item\label{prop:geod-existance:smooth} the map $(p,v,t)\mapsto \gamma(t)$ is smooth in its domain of definition.
\item\label{prop:geod-existance:whole} if $\Sigma$ is complete, then $\mathbb{I}=\RR$; that is, the maximal interval is whole real line.
\end{enumerate}

\end{thm}

\parit{Sketch of proof.}
The first part of the proposition and part (\ref{prop:geod-existance:smooth}) follows from existence and uniqueness of a solution of initial value problem (\ref{thm:ODE}).
One only needs to rewrite the condition $\gamma''(t)\perp\T_{\gamma(t)}$ as a differential equation 
$\gamma''(t)=\II_{\gamma(t)}(\gamma'(t),\gamma'(t))$.

The part (\ref{prop:geod-existance:whole}) follows from \ref{lem:constant-speed}.
Indeed by \ref{thm:ODE}, if the maximal interval is not whole real line, then the curve $\gamma$ must escape to infinity. %???
But the later is impossible since $\gamma$ runs with constant speed.
\qeds

\section*{Exponential map}

Let $\Sigma$ be smooth regular surface and $p\in \Sigma$.
Given a tangent vector $v\in \T_p$ consider a geodesic $\gamma_v$ in $\Sigma$ that runs from $p$ with the initial velocity $v$;  
that is, $\gamma(0)=p$ and $\gamma'(0)=v$.

The point $q=\gamma_v(1)$ is called \emph{exponential map} of $v$, or briefly $q=\exp_pv$.
(There is a reason to call this map \emph{exponential}, but it will takes us to far from the subject.)
By \ref{prop:geod-existance}, the map $\exp_p\:\T_p\to \Sigma$ is smooth and defined in a neighborhood of zero in $\T_p$;
moreover, if $\Sigma$ is compete, then $\exp_p$ is defined on the whole space $\T_p$.

Note that the Jacobian of $\exp_p$ at zero is the identity matrix.
Indeed, let $z=f(x,y)$ be a local graph representation of $\Sigma$ in the tangent-normal coordinates.
The tangent plane at $p$ is the $(x,y)$-plane.
Let $\gamma_x$ and $\gamma_y$ be the geodesics starting from $p$ in the directions
$(1,0,0)$
and
$(0,1,0)$
correspondingly.
A general tangent vector can be written as $v=(x,y,0)$.
Note that $\tfrac{\partial\exp_p}{\partial x}(0,0)=\gamma_x'(0)=(1,0,0)$ and  
$\tfrac{\partial\exp_p}{\partial x}(0,0)=\gamma_y'(0)=(0,1,0)$.
That is, the Jacobian matrix of $\exp_p$ at $(0,0)$ is
\[\begin{pmatrix}
1&0
\\
0&1
\\
0&0
  \end{pmatrix}.
\]
It follows that the Jacobian matrix of the projection of $\exp_p$ to the $(x,y)$-plane has is the identity matrix.
Therefore by the inverse function theorem (\ref{thm:inverse}), we get the following statement:

\begin{thm}{Proposition}\label{prop:exp}
Let $\Sigma$ be smooth surface and $p\in \Sigma$.
Then the exponential map $\exp_p\:\T_p\to \Sigma$ is a smooth regular parametrization of a neighborhood of $p$ in $\Sigma$ by a neighborhood of $0$ in the tangent plane~$\T_p$.

Moreover for any $p\in \Sigma$ there is $\eps>0$ such that for any $x\in \Sigma$ such that $|x-p|_\Sigma<\eps$ the map 
$\exp_x\:\T_x\to \Sigma$ is a smooth regular parametrization of the $\eps$-neighborhood of $x$ in $\Sigma$ by the $\eps$-neighborhood of zero in the tangent plane~$\T_x$. %???
\end{thm}

\section*{Shortest paths are geodesics}


\begin{thm}{Claim}\label{clm:gamma''}
Let $\Sigma$ be a smooth regular surface.
Then any shortest path $\gamma$ in $\Sigma$ parameterized proportional to its length is a geodesic in $\Sigma$.
In particular $\gamma$ is a smooth curve.

A partial converse to the first statement also holds: a sufficiently short arc of any geodesic is a shortest path.
More precisely, given a smooth surface $\Sigma$ there is a positive function $\rho$ on $\Sigma$ such that 
if a geodesic $\gamma$ starts at $p\in \Sigma$ and has length at most $\rho(p)$ then it is a shortest path.
\end{thm}

A geodesic might not form a shortest path, but if this is the case, then it is called \emph{minimizing geodesic}.
Note that according to the claim, any shortest path is a reparametrization of a minimizing geodesic.

This claim provides connection between intrinsic geometry of the surface and its extrinsic geometry.
This connection will be important latter; in particular it will play the key role in the proof of the so called remarkable theorem. %???+ref(see \ref{thm:remarkable}).

Intrinsic means that it can be expressed in terms of measuring things inside the surface, for example length of curves or angles between the curves that lie in the surface.
Extrinsic means that we have to use ambient space in order to measure it.

For instance shortest path $\gamma$ is an object of intrinsic geometry of the surface $\Sigma$,
while definition of geodesic is not intrinsic --- it requires the second derivative $\gamma''$ which needs the ambient space.
Note that there is a smooth bijection between the cylinder $z=x^2$ and the plane $z=0$ that preserves the lengths of all curves; in other words cylinder can be \emph{unfolded} on the plane.
Such a bijection sends geodesics in the cylinder to geodesics on the plane and the other way around; however a geodesic on the cylinder might have nonvanishing second derivative while geodesics on the plane are straight lines with vanishing second derivative.



\parit{Informal sketch.}
The smoothness should be intuitively obvious; at least the curve should be twice differentiable otherwise it can be shortened.

Let us give an informal physical explanation why $\gamma''(t)\perp\T_{\gamma(t)}\Sigma$.
One may think about the geodesic $\gamma$ as of stable position of a stretched elastic thread that is forced to lie on a frictionless surface.
Since it is frictionless, the force density $N(t)$ that keeps the geodesic $\gamma$ in the surface must be proportional to the normal vector to the surface at $\gamma(t)$.

The tension in the thread has to be the same at all points (otherwise the thread would move back or forth and it would not be stable).
Denote by $T$ the tension.
We can assume that $\gamma$ has unit speed,
In this case the net force from tension to the arc $\gamma_{[t_0,t_1]}$ is $T\cdot(\gamma'(t_1)-\gamma'(t_0)$.
Hence the density of net force from tension at $t$ is $F(t)=T\cdot\gamma''(t)$.
According to the second Newton's law of motion, we have 
\[F(t)+N(t)=0;\]
which implies that  $\gamma''(t)$ is perpendicular to $\T_{\gamma(t)}\Sigma$.

Fix a point $p\in\Sigma$.
Let $\eps>0$ be as in \ref{prop:exp}.
Assume a geodesic $\gamma$ of length less than $\eps$ from $p$ to $q$ does not minimize the length between its endpoints.
Then there is a shortest path from $p$ to $q$, which becomes a geodesic if parameterized by its arc length.
That is there are two geodesics from $p$ to $q$ of length smaller than $\eps$.
In other words there are two vectors $v,w\in\T_p$ such that $|v|<\eps$, $|w|<\eps$ and 
$q=\exp_pv=\exp_pw$.
But according to \ref{prop:exp}, the exponential map is injective in $\eps$-neighborhood of zero --- a contradiction.\qeds

\begin{thm}{Exercise}\label{ex:two-min-geod}
Show that two shortest paths all 4 ends of which are distinct can not have more than one point of intersections.

Show by example that distinct geodesics can cross each other arbitrary number of times.
\end{thm}

\begin{thm}{Exercise}\label{ex:min-geod+plane}
Assume that a smooth regular surface $\Sigma$ is mirror symmetric with respect to a plane $\Pi$.
Show that no shortest path in $\Sigma$ can cross $\Pi$ more than once.
\end{thm}

{

\begin{wrapfigure}[12]{r}{39 mm}
\begin{lpic}[t(-0 mm),b(-4 mm),r(0 mm),l(0 mm)]{pics/unbend(1)}
\lbl[t]{11.5,29;$p$}
\lbl[r]{10.5,37;$p_t$}
\lbl[t]{32.5,26;$q$}
\lbl[t]{26,30;$\gamma(t)$}
\lbl{20,13;{\Large $\Sigma$}}
\end{lpic}
\end{wrapfigure}

\begin{thm}{Advanced exercise}
Let $\Sigma$ be a smooth closed strictly convex surface 
in $\RR^3$ 
and $\gamma\:[0,\ell]\z\to \Sigma$ be a unit-speed minimizing geodesic.
Set $p\z=\gamma(0)$, $q=\gamma(\ell)$ and 
$$p_t=\gamma(t)-t\cdot\gamma'(t),$$ 
where $\gamma'(t)$ denotes the velocity vector of $\gamma$ at $t$.

Show that for any $t\in (0,\ell)$,
one {}\emph{cannot see}  $q$ from $p_t$;
that is, the line segment $[p_t,q]$ intersects $\Sigma$ at a point distinct from $q$.

Show that the statement is does not hold if $\gamma$ fails to me minimizing.
\end{thm}

\parit{Hint:} Show that the concatenation of the line segment $[p_t,\gamma(t)]$ and the arc $\gamma|_{[t,\ell]}$ is a shortest path in the closed region $W$ outside of $\Sigma$.

}

\section*{Liberman's lemma}

The following lemma is a smooth analog of lemma proved by Joseph Liberman \cite{liberman}.

\begin{thm}{Liberman's lemma}\label{lem:liberman}
Assume $\gamma$ is a geodesic on the graph $z=f(x,y)$ of a smooth convex function $f$ defined on an open subset of the plane.
Suppose that $\gamma(t)=(x(t),y(t),z(t))$
Then $t\mapsto z(t)$ is a convex function; that is $z''(t)\ge 0$ for any $t$.
\end{thm}

\parit{Proof.}
Choose the orientation on the graph so that the unit normal vector $\nu$ always points up;
that is, it has positive $z$-coordinate.

Since $\gamma$ is a geodesic, we have $\gamma''(t)\perp\T_{\gamma(t)}$,
or equivalently $\gamma''(t)$ is proportional to $\nu_{\gamma(t)}$ for any $t$.
By \ref{prop:gamma''=II}, we have
\[\langle \gamma''(t),\nu_{\gamma(t)}\rangle=\II_{\gamma(t)}(\gamma'(t),\gamma'(t));\]
hence
\[\gamma''(t)=\nu_{\gamma(t)}\cdot \II_{\gamma(t)}(\gamma'(t),\gamma'(t))\]
for any $t$.

Therefore
\[z''(t)=\cos(\theta_\gamma(t))\cdot\nu_{\gamma(t)}\cdot \II_{\gamma(t)}(\gamma'(t),\gamma'(t)),\eqlbl{eq:z''}\]
where $\theta_\gamma(t)$ denotes the angle between $\nu_{\gamma(t)}$ and the $z$-axis.

Since $\nu$ points up, we have $\theta_\gamma(t)<\tfrac\pi2$, or equivalently
\[\cos(\theta_\gamma(t))>0\]
for any $t$.
 
Since $f$ is convex, we have that tangent plan supports the graph from below at any point;
in particular $\II_{\gamma(t)}(\gamma'(t),\gamma'(t))\ge 0$.
It follows that the right hand side in \ref{eq:z''} is nonnegative;
whence the statement follows.
\qeds

\begin{thm}{Exercise}
Assume $\gamma$ is a unit-speed geodesic on a smooth convex surface $\Sigma$ and $p$ in the interior of a convex set bounded by $\Sigma$.
Set $\rho(t)=|p-\gamma(t)|^2$.
Show that $\rho''(t)\le 2$ for any $t$.
\end{thm}



\section*{Bound on total curvature}

\begin{thm}{Theorem}\label{thm:usov}
Assume $\Sigma$ is a graph $z=f(x,y)$ of a convex $\ell$-Lipschitz function $f$ defined on an open set in the $(x,y)$-plane.
Then the total curvature of any geodesic in $\Sigma$ is at most $2\cdot \ell$.
\end{thm}

The above theorem was proved by Vladimir Usov \cite{usov},
later David Berg \cite{berg} pointed out that the same proof works for geodesics in closed epigraphs of $\ell$-Lipschitz functions which are not necessary convex; that is, sets of the type 
\[W=\set{(x,y,z)\in\RR^3}{z\ge f(x,y)}\]

\parit{Proof.}
Let $\gamma(t)=(x(t),y(t),z(t))$ be a unit speed geodesic on $\Sigma$.
According to Liberman's lemma 
$z(t)$ is convex.

Since the slope of $f$ is at most $\ell$, we have
\[|z'(t)|\le \tfrac{\ell}{\sqrt{1+\ell^2}}.\]
If $\gamma$ is defined on the interval $[a,b]$, then
\begin{align*}
\int_a^b |z''(t)|&=z'(a)-z'(b)\le 
\\
&\le 2\cdot \tfrac{\ell}{\sqrt{1+\ell^2}}.
\end{align*}

Further, note that $z''$ is the projection of $\gamma''$ to the $z$-axis.
Since $f$ is $\ell$-Lipschitz, the tangent plane $T_{\gamma (t)} \Sigma$ cannot have slope greater than $\ell$ for any $t$.
Because $\gamma ''$ is perpendicular to that plane, 
\[|\gamma'' (t)|  \le \sqrt{1+ \ell ^2} \cdot|z''(t)| .\]

Recall that $\tc\gamma$ denotes the total curvature of curve $\gamma$.
It follows that 
\begin{align*}
\tc\gamma&=\int_a^b|\gamma'' (t)|\cdot dt\le 
\\
&\le \sqrt{1+ \ell ^2}\cdot  \int_a^b |z''(t)|\cdot dt\le 
\\
&\le 2\cdot \ell.
\end{align*}
\qedsf

\begin{thm}{Exercise}
Note that the graph $z=\ell\cdot\sqrt{x^2+y^2}$ with removed origin is a smooth surface; denote it by $\Sigma$.
Show that it has an both side infinite geodesic $\gamma$ with total curvature exactly $2\cdot \ell$.
\end{thm}

Note that the function $f(x,y)=\ell\cdot\sqrt{x^2+y^2}$ is $\ell$-Lipschitz.
The graph $z=f(x,y)$ in the exercise can be smoothed in a neighborhood of the origin while keeping it convex.
It follows that the estimate in the Usov's theorem is optimal.


\begin{thm}{Exercise}
Assume $f$ is a convex function $\tfrac32$-Lipschitz function defined on the $(x,y)$-plane.
Show that any geodesic $\gamma$ on the graph $z=f(x,y)$ is simple;
that is, it has no self-intersections.

Construct a convex function $2$-Lipschitz function defined on the $(x,y)$-plane
with a nonsimple geodesic $\gamma$ on its graph $z=f(x,y)$.
\end{thm}

\parit{Hint:} Use the theorem and \ref{ex:sef-intersection}.
The suggested argument does not give the optimal bound for the Lipschitz constant that guarantee that $\gamma$ is simple, but
later we will show that the exact bound is $\sqrt{3}=\tan\tfrac\pi3$ --- it is the same as in the case of cone; compare to \ref{ex:lasso}.  %+??? ref

\begin{thm}{Theorem}
Suppose a smooth surface $\Sigma$ bounds a convex set $K$ in the Euclidean space.
Assume $B(0,\eps)\subset K\subset B(0,1)$.
Then the total curvatures of any shortest path in $\Sigma$ can be bounded in terms of~$\eps$. 
\end{thm}

\begin{wrapfigure}{o}{48 mm}
\vskip-4mm
\centering
\includegraphics{mppics/pic-83}
\vskip-0mm
\end{wrapfigure}

The following exercise will guide you thru the proof of the theorem. 

\begin{thm}{Exercise}
Let $\Sigma$ be as in the theorem and $\gamma$ be a unit-speed shortest path in $\Sigma$.
Denote by $\nu_p$ the unit normal vector that points outside of $\Sigma$;
denote by $\theta_p$ the angle between $\nu_p$ and the direction from the origin to a point $p\in\Sigma$.
Set $\rho(t)\z=|\gamma(t)|^2$; let $k(t)$ be the curvature of $\gamma$ at $t$.
\begin{enumerate}[(a)]
\item Show that $\cos\theta_p\ge \eps$ for any $p\in \Sigma$.
\item Show that $|\rho'(t)|\le 2$ for any $t$.
\item Show that 
\[\rho''(t)=2-2\cdot k(t)\cdot \cos \theta_{\gamma(t)}\cdot |\gamma(t)|\]
for any $t$.
\item Use the closest-point projection from the unit sphere to $\Sigma$ to show that 
\[\length \gamma\le \pi.\]
\item Use the the statements above to conclude that 
\[\tc\gamma\le \frac{100}{\eps^2}.\]
\end{enumerate}
\end{thm}

Note that our bound on total curvature given above goes to infinity as $\eps\to 0$,
but in fact there is a bound independent of $\eps$;
it is good of any closed convex surface \cite{lebedeva-petrunin}.
