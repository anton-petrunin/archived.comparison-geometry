\chapter{Shortest paths}




\section{Definition}

Let $p$ and $q$ be two points on a surface $\Sigma$.
Recall that $|p-q|_\Sigma$ denotes the induced length distance from $p$ to $q$;
that is, the exact lower bound on lengths of paths in $\Sigma$ from $p$ to $q$.

Note that if $\Sigma$ is smooth, then any two points in $\Sigma$ can be joined by a piecewise smooth path.
Since any such path is rectifiable, the value $|p-q|_\Sigma$ is finite for any pair $p,q\in\Sigma$.

A path $\gamma$ from $p$ to $q$ in $\Sigma$ that minimizes the length is called a \emph{shortest path} from $p$ to $q$.

The image of a shortest path between $p$ and $q$ in $\Sigma$ is usually denoted by $[pq]$ or by $[pq]_\Sigma$.
In general there might be no shortest path between two given points on the surface
and it might be many of them;
this is shown in the following two examples.

Usually, if we write $[pq]_\Sigma$, then we assume that a shortest path exists and we made a choice of one of them.

\begin{wrapfigure}{r}{28 mm}
\vskip-4mm
\centering
\includegraphics{asy/sphere}
\bigskip
\includegraphics{mppics/pic-79}
\end{wrapfigure}

\parbf{Nonuniqueness.} There are plenty of shortest paths between the poles on the sphere --- each meridian is a shortest path.

\parbf{Nonexistence.} Let $\Sigma$ be the $(x,y)$-plane with removed origin.
Consider two points $p=(1,0,0)$ and $q=(-1,0,0)$ in $\Sigma$.

Note that $|p-q|_\Sigma=2$. 
Indeed, given $\eps\z>0$, consider the point $s_\eps=(0,\eps,0)$.
Observe that the polygonal path $ps_\eps q$ lies in $\Sigma$ and its length $2\cdot\sqrt{1+\eps^2}$ approaches $2$ as $\eps\to0$.
It follows that $|p-q|_\Sigma\le 2$.
On the other hand $|p-q|_\Sigma\ge |p-q|_{\RR^3}=2$; therefore $|p-q|_\Sigma= 2$.


It follows that a shortest path from $p$ to $q$, if it exists, must have length 2.
By triangle inequality any curve of length 2 from $p$ to $q$ must run along the line segment $[pq]$;
in particular it must pass thru the origin.
Since the origin does not lie in $\Sigma$, there is no shortest from $p$ to $q$ in $\Sigma$ 

\begin{thm}{Proposition}\label{prop:shortest-paths-exist}
Any two points in a proper smooth surface can be joined by a shortest path. 
\end{thm}

\parit{Proof.}
Fix a proper smooth surface $\Sigma$ with two points $p$ and $q$.
Set $\ell=|p-q|_\Sigma$.

By the definition of induced length metric,
there is a sequence of paths $\gamma_n$ from $p$ to $q$ in $\Sigma$ such that
\[\length\gamma_n\to \ell\quad\text{as}\quad n\to \infty.\]

Without loss of generality, we may assume that $\length\gamma_n<\ell+1$ for any $n$ and each $\gamma_n$ is parameterized proportional to its arc-length.
In particular each path $\gamma_n\:[0,1]\to\Sigma$ is $(\ell+1)$-Lipschitz; 
that is,
\[|\gamma(t_0)-\gamma(t_1)|\le (\ell+1)\cdot|t_0-t_1|\]
for any $t_0,t_1\in[0,1]$.

Note that the image of $\gamma_n$ lies in the closed ball $\bar B[p,\ell+1]$ for any $n$.
It follows that the coordinate functions of $\gamma_n$ are uniformly equicontinuous and uniformly bounded.
By \ref{lem:equicontinuous}, we can pass to a converging subsequence of $\gamma_n$;
denote by $\gamma_\infty\:[0,1]\to\RR^3$ its limit.

As a limit of uniformly continuous sequence, $\gamma_\infty$ is continuous;
that is, $\gamma_\infty$ is a path.
Evidently $\gamma_\infty$ runs from $p$ to $q$;
in particular
\[\length\gamma_\infty\ge \ell.\]
Since $\Sigma$ is a closed set, $\gamma_\infty$ lies in $\Sigma$.
Finally, by \ref{thm:length-semicont}, 
\[\length\gamma_\infty\le \ell.\]
That is, $\gamma_\infty= \ell$ or, equivalently, $\gamma_\infty$ is a shortest path from $p$ to $q$.
\qeds

\section{Closest point projection}

\begin{thm}{Lemma}\label{lem:closest-point-projection}
Let $R$ be a closed convex set in $\RR^3$.
Then for every point $p\in\RR^3$ there is a unique point $\bar p\in R$ that minimizes the distance to $R$;
that is, $|p-\bar p|\le |p-x|$ for any point $x\in R$.

Moreover the map $p\mapsto \bar p$ is short;
that is,
\[|p-q|\ge|\bar p-\bar q| \eqlbl{eq:short-cpp}\]
for any pair of points $p,q\in \RR^3$.
\end{thm}

The map $p\mapsto \bar p$ is called the \emph{closest point projection};
it maps the Euclidean space to $R$.
Note that if $p\in R$, then $\bar p=p$.

\parit{Proof.}
Fix a point $p$ and set 
\[\ell=\inf\set{|p-x|}{x\in R}.\]
Choose a sequence $x_n\in R$ such that $|p-x_n|\to \ell$ as $n\to\infty$.

\begin{wrapfigure}{o}{22 mm}
\vskip-0mm
\centering
\includegraphics{mppics/pic-40}
\vskip-0mm
\end{wrapfigure}

Without loss of generality, we can assume that all the points $x_n$ lie in a ball of radius $\ell+1$ centered at~$p$.
Therefore we can pass to a partial limit $\bar p$ of $x_n$; that is, $\bar p$ is a limit of a subsequence of $x_n$.
Since $R$ is closed, $\bar p\in R$.
By construction 
\begin{align*}
|p-\bar p|&=\lim_{n\to\infty}|p-x_n|=
\\
&=\ell.
\end{align*}
Hence the existence follows.

Assume there are two distinct points $\bar p, \bar p'\in R$ that minimize the distance to $p$.
Since $R$ is convex, their midpoint $m=\tfrac12\cdot (\bar p+\bar p')$ lies in~$R$.
Note that $|p-\bar p|\z=|p-\bar p'|=\ell$; that is, the triangle $[p\bar p\bar p']$ is isosceles and therefore the triangle $[p\bar p m]$ is right with the right angle at~$m$.
Since a leg of a right triangle is shorter than its hypotenuse, we have $|p-m|<\ell$ --- a contradiction. 

It remains to prove inequality \ref{eq:short-cpp}.

\begin{wrapfigure}{o}{37 mm}
\vskip-0mm
\centering
\includegraphics{mppics/pic-41}
\vskip-0mm
\end{wrapfigure}

We can assume that $\bar p\ne\bar q$, otherwise there is nothing to prove.
Note that if $p\ne \bar p$ (that is, if $p\notin R$), 
then $\angle p \bar p \bar q$ is right or obtuse.
Otherwise there would be a point $x$ on the line segment $[\bar q\bar p]$ that is closer to $p$ than $\bar p$.
Since $R$ is convex, the line segment $[\bar q\bar p]$ and therefore $x$ lie in $R$.
Hence $\bar p$ is not closest to $p$ --- a contradiction.

The same way we can show that  if $q\z\ne \bar q$, then $\angle q \bar q \bar p$ is right or obtuse.

We have to consider the following 4 cases:
(1) $p\ne \bar p$ and $q\ne \bar q$,
(2) $p= \bar p$ and $q\ne \bar q$,
(3) $p\ne \bar p$ and $q= \bar q$,
(4) $p= \bar p$ and $q= \bar q$.
In all these cases the obtained angle estimates imply that the orthogonal projection of the line segment $[pq]$ to the line $\bar p\bar q$ contains the line segment $[\bar p\bar q]$.
In particular 
\[|p-q|\ge |\bar p-\bar q|.\]
\qedsf

\begin{thm}{Corollary}\label{cor:shorts+convex}
Assume a surface $\Sigma$ bounds a closed convex region $R$ and  $p,q \in \Sigma$.
Denote by $W$ the outer closed region of $\Sigma$; in other words $W$ is the union of $\Sigma$ and the complement of $R$.
Then 
\[\length\gamma\ge |p-q|_\Sigma\]
for any path $\gamma$ in $W$ from $p$ to $q$.
Moreover if  $\gamma$ does not lie in $\Sigma$, then the inequality is strict.
\end{thm}

\parit{Proof.}
The first part of the corollary follows from the lemma and the definition of length.
Indeed consider the closest point projection $\bar\gamma$ of~$\gamma$.
Note that $\bar\gamma$ lies in $\Sigma$ and connects $p$ to $q$ therefore 
\[\length\bar\gamma\ge |p-q|_\Sigma.\]

To prove the first statemnt, it is sufficient to show that 
\[\length\gamma\ge\length\bar\gamma.\eqlbl{bar-gamma=<gamma}\]

Consider an inscribed polygonal line $p_0\dots p_n$ in $\gamma$.
Denote by $\bar p_i$ the closest point projection of $p_i$ to $R$.
Note that the polygonal line  $\bar p_0\dots \bar p_n$ is inscribed in $\bar\gamma$;
moreover any inscribed polygonal line in $\bar\gamma$ can appear this way.
By \ref{lem:closest-point-projection} $|p_i-p_{i-1}|\z\ge|\bar p_i-\bar p_{i-1}|$ for any $i$.
Therefore 
\[\length p_0\dots p_n\ge \length \bar p_0\dots \bar p_n.\]
Taking least upper bound of each side of the inequality for all inscribed polygonal lines $p_0\dots p_n$ in $\gamma$, we get \ref{bar-gamma=<gamma}.\

\begin{wrapfigure}{o}{37 mm}
\vskip-0mm
\centering
\includegraphics{mppics/pic-82}
\vskip-0mm
\end{wrapfigure}

It remains to prove the second statement.
Suppose that there is a point $s\z=\gamma(t_1)\notin\Sigma$;
note that $s\notin R$.
By the separation lemma (\ref{lem:separation}) there is a plane $\Pi$ that cuts $s$ from $\Sigma$.
The curve $\gamma$ must intersect at least at two points: one point before $t_1$ and one after;
let $a=\gamma(t_0)$ and $b=\gamma(t_2)$ be these points.
Note that the arc of $\gamma$ from $a$ to $b$ is strictly longer that $|a-b|$;
indeed its length is at least $|a-s|+|s-b|$ and $|a-s|+|s-b|>|a-b|$ since $s\notin[ab]$.

Remove from $\gamma$ the arc from $a$ to $b$ and glue in the line segment $[ab]$;
denote the obtained curve by $\gamma_1$. 
From above, we have that
\[\length\gamma>\length \gamma_1\]
Note that $\gamma_1$ runs in $W$.
Therefore by the first part of corollary, we have
\[\length \gamma_1\ge |p-q|_\Sigma.\]
Whence the second statement follows.
\qeds

\begin{wrapfigure}{r}{27 mm}
\vskip0mm
\centering
\includegraphics{mppics/pic-240}
\end{wrapfigure}

\begin{thm}{Exercise}\label{ex:hat-convex}
Suppose $\Sigma$ is a closed smooth surface that bounds a convex region $R$ 
in $\RR^3$
and $\Pi$ is a plane that cuts a hat $\Delta$ from $\Sigma$.
Assume that the reflection of the interior of $\Delta$ across $\Pi$ lies in the interior of $R$.
Show that $\Delta$ is \index{convex set}\emph{convex} with respect to the intrinsic metric  of $\Sigma$;
that is, 
if both ends of a shortest path in $\Sigma$ 
lie in $\Delta$,
then the entire path lies in $\Delta$.
\end{thm}


Let us define the \emph{intrinsic diameter} of a closed surface $\Sigma$ as the exact upper bound on the lengths of shortest paths in the surface.

\begin{thm}{Exercise}\label{ex:intrinsic-diameter}
Assume that a closed smooth surface $\Sigma$ with positive Gauss curvature lies in a unit ball.

\begin{subthm}{} Show that the intrinsic diameter of $\Sigma$ cannot exceed $\pi$.
 
\end{subthm}

\begin{subthm}{}
Show that the area of $\Sigma$ cannot exceed $4\cdot \pi$.
\end{subthm}

\end{thm}

\chapter{Geodesics}


\section{Definition}

A smooth curve $\gamma$ on a smooth surface $\Sigma$ is called \emph{geodesic} if for any~$t$, the acceleration $\gamma''(t)$ is perpendicular to the tangent plane $\T_{\gamma(t)}$.

\begin{thm}{Exercise}\label{ex:helix=geodesic}
Show that the helix 
\[\gamma(t)=(\cos t,\sin t, a\cdot t)\]
is a geodesic on the cylindrical surface described by the equation $x^2\z+y^2=1$.
\end{thm}

\begin{thm}{Exercise}\label{ex:reflection-geodesic}
Assume that a smooth surface $\Sigma$ is mirror symmetric with respect to  a plane $\Pi$.
Suppose that $\Sigma$ and $\Pi$ intersect along a smooth regular curve $\gamma$.
Show that $\gamma$ parameterized by its arc-length is a geodesic on $\Sigma$.
\end{thm}

Recall that asymptotic line is defined on page~\pageref{page:asymptotic line}.

\begin{thm}{Exercise}\label{ex:asymptotic-geodesic}
Suppose that a curve $\gamma$ is a geodesic and, at the same time, is an asymptotic line on a smooth surface $\Sigma$.
Show that $\gamma$ is a line segment.
\end{thm}

Physically, geodesics can be understood as the trajectories of a particle that slides on $\Sigma$ without friction.
Indeed, since there is no friction, the force that keeps the particle on $\Sigma$ must be perpendicular to $\Sigma$.
Therefore, by the second Newton's laws of motion,
we get that the acceleration $\gamma''$ is perpendicular to $\T_{\gamma(t)}$.

\section{Existence}

The following lemma and proposition can be also interpreted physically;
lemma follow from the conservation of energy and the proposition gives smooth dependence of trajectory of a particle depending on its initial position and velocity.

\begin{thm}{Lemma}\label{lem:constant-speed}
Any geodesic has constant speed.

More precisely, if $\gamma$ is a geodesic on a smooth surface, then $|\gamma'|$ is constant.
\end{thm}

\parit{Proof.} 
Since $\gamma'(t)$ is a tangent vector at $\gamma(t)$,
we have that $\gamma''(t)\z\perp\gamma'(t)$, or equivalently $\langle\gamma'',\gamma'\rangle=0$ for any $t$.
Whence 
\begin{align*}
\langle\gamma',\gamma'\rangle'&=2\cdot \langle\gamma'',\gamma'\rangle=
\\
&=0.
\end{align*}
That is, $|\gamma'(t)|^2=\langle\gamma'(t),\gamma'(t)\rangle$ is constant.
\qeds

\begin{thm}{Proposition}\label{prop:geod-existence} 
Let $\Sigma$ be  a smooth surface without boundary.
Given a tangent vector ${\vec v}$ to $\Sigma$ at a point $p$
there is a unique geodesic $\gamma\:\mathbb{I}\to \Sigma$ defined on a maximal open interval $\mathbb{I}\ni 0$ that starts at $p$ with velocity vector ${\vec v}$;
that is, $\gamma(0)=p$ and $\gamma'(0)={\vec v}$.

Moreover
\begin{subthm}{prop:geod-existence:smooth} the map $(p,{\vec v},t)\mapsto \gamma(t)$ is smooth in its domain of definition.
\end{subthm}

\begin{subthm}{prop:geod-existence:whole} if $\Sigma$ is proper, then $\mathbb{I}=\RR$; that is, the maximal interval is whole real line.
\end{subthm}

\end{thm}

The proof of this proposition relies on the existence and uniqueness of the initial value problem (\ref{thm:ODE}).

\begin{thm}{Lemma}\label{lem:geodesic=2nd-order}
Let $f$ be  a smooth function defined on an open domain in $\RR^2$.
A smooth curve $t\mapsto \gamma(t)=(x(t),y(t),z(t))$ is the geodesic in a graph $z=f(x,y)$ if and only if $z(t)=f(x(t),y(t))$ for any $t$ and the functions $t\mapsto x(t)$ and $t\mapsto y(t)$
satisfy a differential equation
\[
\begin{cases}
x''=g(x,y,x',y'),
\\
y''=h(x,y,x',y'),
\end{cases}
\]
where the functions $g$ and $h$ are smooth functions of four variables that determined by $f$.
\end{thm}

The proof of the lemma is done by means of direct calculations.

\parit{Proof.} In the following calculations, we often omit the arguments --- we may write $x$ instead of $x(t)$  and $f$ instead of $f(x,y)$ or $f(x(t),y(t))$ and so on.

First let us calulate $z''(t)$ in terms of $f$, $x(t)$, and $y(t)$.
\[
\begin{aligned}
z''(t)&=f(x(t),y(t))''=
\\
&=\left(\tfrac{\partial f}{\partial x}\cdot x'+\tfrac{\partial f}{\partial y}\cdot y'\right)'=
\\
&=\tfrac{\partial^2 f}{\partial x^2}\cdot (x')^2+\tfrac{\partial f}{\partial x}\cdot x''+\tfrac{\partial^2 f}{\partial y^2}\cdot (y')^2+\tfrac{\partial f}{\partial y}\cdot y''.
\end{aligned}
\]
Now observe that the equation 
\[\gamma''(t)\perp\T_{\gamma(t)}\eqlbl{eq:def-geod}\] 
means that 
$\gamma''$ is perpendicular to two basis vectors in $\T_{\gamma(t)}$.
Therefore the vector equation \ref{eq:def-geod} can be rewritten as the following system of two real equations
\[
\begin{cases}
\langle \gamma''(t),\tfrac{\partial s}{\partial x}\rangle=0,
\\
\langle\gamma''(t),\tfrac{\partial s}{\partial y}\rangle=0,
\end{cases}
\]
where $s(x,y)\df (x,y,f(x,y))$, $x=x(t)$, and $y=y(t)$.

Observe that $\frac{\partial s}{\partial x}=(1,0,\tfrac{\partial f}{\partial x})$ and $\frac{\partial s}{\partial y}=(0,1,\tfrac{\partial f}{\partial y})$.
Since $\gamma''\z=(x'',y'',z'')$, we can rewrite the system the following way.
\[
\begin{cases}
x''+\tfrac{\partial f}{\partial x}\cdot z''=0,
\\
y''+\tfrac{\partial f}{\partial y}\cdot z''=0,
\end{cases}
\]
It remains use expression \ref{eq:z''} for $z''$, combine like terms and simplify.
\qeds


\parit{Proof of \ref{prop:geod-existence}.}
Let $z=f(x,y)$ be a description of $\Sigma$ in a tangent-normal coordinates at $p$.
By Lemma \ref{lem:geodesic=2nd-order} the condition $\gamma''(t)\perp\T_{\gamma(t)}$ can be written as a second order differential equation.
Applying the existence and uniqueness of the initial value problem (\ref{thm:ODE}) we get existence and uniqueness of geodesic $\gamma$ in a in a small interval $(-\eps,\eps)$ for some $\eps>0$.

Let us extend $\gamma$ to a maximal open interval $\mathbb{I}$.
Suppose there is another geodesic $\gamma_1$ with the same initial data that is defined on a maximal open interval $\mathbb{I}_1$.
Suppose $\gamma_1$ splits from $\gamma$ at some time $t_0>0$;
that is, $\gamma_1$ coincides with $\gamma$ on the interval $[0,t_0)$, but they are different on the interval $[0,t_0+\eps)$ for any $\eps>0$.
By continuity $\gamma_1(t_0)=\gamma(t_0)$ and $\gamma'(t_0)=\gamma'(t_0)$.
Applying uniqueness of the initial value problem (\ref{thm:ODE}) again, we get that $\gamma_1$ coincides with $\gamma$ in a small neighborhood of $t_0$ --- a contradiction.

The case $t_0<0$ can be proved along the same lines.
It follows that $\gamma_1$ coincides with $\gamma$.

Part \ref{SHORT.prop:geod-existence:smooth} follows since the solution of the initial value problem depends smoothly on the initial data (\ref{thm:ODE}).

Suppose \ref{SHORT.prop:geod-existence:whole} does not hold;
that is, the maximal interval $\mathbb{I}$ is a proper subset of the real line $\RR$.
Without loss of generality we may assume that $b=\sup\mathbb{I}<\infty$.

By \ref{lem:constant-speed} $|\gamma'|$ is constant, in particular $t\mapsto \gamma(t)$ is a uniformly continuous function.
Therefore  the limit 
\[q=\lim_{t\to b}\gamma(t)\] 
is defined.
Since $\Sigma$ is a proper surface, $q\in \Sigma$. 

Applying the argument above in a tangent-normal coordinates at $q$ shows that $\gamma$ can be extended as a geodesic behind $q$.
Therefore $\mathbb{I}$ is not a maximal interval --- a contradiction.
\qeds




\section{Exponential map}\label{sec:exp}

Let $\Sigma$ be smooth regular surface and $p\in \Sigma$.
Given a tangent vector ${\vec v}\in \T_p$ consider a geodesic $\gamma_{\vec v}$ in $\Sigma$ that runs from $p$ with the initial velocity ${\vec v}$;  
that is, $\gamma(0)=p$ and $\gamma'(0)={\vec v}$.

The point $q=\gamma_{\vec v}(1)$ is called \emph{exponential map} of ${\vec v}$, or briefly $q=\exp_p{\vec v}$.
(There is a reason to call this map \emph{exponential}, but it will take us too far from the subject.)
By \ref{prop:geod-existence}, the map $\exp_p\:\T_p\to \Sigma$ is smooth and defined in a neighborhood of zero in $\T_p$;
moreover, if $\Sigma$ is proper, then $\exp_p$ is defined on the whole space $\T_p$.

Note that the exponential map $\exp_p$ 
is defined on the tangent plane $\T_p$, which is a smooth surfaces,
and its target is another smooth surface $\Sigma$.
Observe that one can identify the plane $\T_p$
with its tangent plane $\T_0\T_p$ so the linearization $L_0\exp_p$ maps $\T_p$ to itself.
Further note that by the definition of exponential map we have that $L_0\exp_p(\vec v)=
\vec v$ for any $\vec v\in \T_p$.

The above discussion is summarized in the following statement. 

\begin{thm}{Observation}\label{obs:d(exp)=1}
Let $\Sigma$ be a smooth surface and $p\in \Sigma$.
Then the exponential map $\exp_p\:\T_p\to \Sigma$ is smooth and its linearization $L_0(\exp_p)\:\T_p\to \T_p$ is the identity map. %%%????linearization
\end{thm}


\begin{thm}{Proposition}\label{prop:exp}
Let $\Sigma$ be smooth surface and $p\in \Sigma$.
Then the exponential map $\exp_p\:\T_p\to \Sigma$ is a smooth regular parametrization of a neighborhood of $p$ in $\Sigma$ by a neighborhood of $0$ in the tangent plane~$\T_p$.

Moreover for any $p\in \Sigma$ there is $\eps>0$ such that for any $q\in \Sigma$ such that $|q-p|_\Sigma<\eps$ the map 
$\exp_q\:\T_q\to \Sigma$ is a smooth regular parametrization of the $\eps$-neighborhood of $q$ in $\Sigma$ by the $\eps$-neighborhood of zero in the tangent plane~$\T_q$. %???
\end{thm}

The proposition follows from the observation and the inverse function theorem (\ref{thm:inverse}). %???+Jacobian matrix

\parit{Proof.}
Let $z=f(x,y)$ be a local graph representation of $\Sigma$ in the tangent-normal coordinates at $p$.
Note that $(x,y)$-plane coincides with the tangent plane $\T_p$.

Denote by $s$ be the composition of  the exponential map $\exp_p$ with the orthogonal projection $(x,y,z)\mapsto (x,y)$.
By \ref{obs:d(exp)=1}, the linearization $L_0s$ is the identity.
Applying the inverse function theorem (\ref{thm:inverse}) we get the first part of the proposition.

The second part can be proved along the same lines, using the second part the inverse function theorem (\ref{thm:inverse});
it guarantees that the size of the neighborhood in $\T_q$ for all points $q$ sufficiently close to~$p$.
\qeds

\section{Shortest paths are geodesics}

\begin{thm}{Proposition}\label{prop:gamma''}
Let $\Sigma$ be a smooth regular surface.
Then any shortest path $\gamma$ in $\Sigma$ parameterized proportional to its arc-length is a geodesic in $\Sigma$.
In particular $\gamma$ is a smooth curve.

A partial converse to the first statement also holds: a sufficiently short arc of any geodesic is a shortest path.
More precisely, any point $p$ in $\Sigma$ has a neighborhood $U$ such that any geodesic that lies completely in $U$ is a shortest path.
\end{thm}

A geodesic might not form a shortest path, but if this is the case, then it is called a \emph{minimizing geodesic}.
Note that according to the proposition, any shortest path is a reparametrization of a minimizing geodesic.

A formal proof will be given much latter; see page \pageref{page:proof-of-gamma''}. 
The following informal physical explanation might be sufficiently convincing.
In fact, if one assumes that $\gamma$ is smooth, then it is easy to convert this explanation into a rigorous proof.

\parit{Informal explanation.}
Let us think about a shortest path $\gamma$ as of stable position of a stretched elastic thread that is forced to lie on a frictionless surface.
Since it is frictionless, the force density $N(t)$ that keeps the geodesic $\gamma$ in the surface must be proportional to the normal vector to the surface at $\gamma(t)$.

The tension in the thread has to be the same at all points (otherwise the thread would move back or forth and it would not be stable).
Denote  the tension by $T$.

We can assume that $\gamma$ has unit speed;
in this case the net force from tension to the arc $\gamma_{[t_0,t_1]}$ is $T\cdot(\gamma'(t_1)-\gamma'(t_0)$.
Hence the density of net force from tension at $t_0$ is 
\begin{align*}
F(t_0)&=\lim_{t_1\to t_0}T\cdot\frac{\gamma'(t_1)-\gamma'(t_0)}{t_1-t_0}=
\\
&=T\cdot\gamma''(t_0).
\end{align*}
According to the second Newton's law of motion, we have 
\[F(t)+N(t)=0;\]
which implies that  $\gamma''(t)\perp\T_{\gamma(t)}\Sigma$.
\qeds

Note that according to the proposition above, any shortest path parameterized by its arc-length is a smooth curve.
This observation should help to solve in the following two exercises.

\begin{thm}{Exercise}\label{ex:two-min-geod}
Show that two shortest paths can cross each other at most once.
More precisely, if two shortest paths have two distinct common points $p$ and $q$, then either these points are the ends of both shortest paths or both shortest paths contain an arc from $p$ to $q$.

Show by example that nonoverlapping geodesics can cross each other an arbitrary number of times.
\end{thm}

\begin{thm}{Exercise}\label{ex:min-geod+plane}
Assume that a smooth regular surface $\Sigma$ is mirror symmetric with respect to a plane $\Pi$.
Show that no shortest path $\alpha$ in $\Sigma$ can \emph{cross} $\Pi$ more than once.


In other words, if you travel along $\alpha$, then you change the sides of $\Pi$ at most once. 
\end{thm}

{

\begin{wrapfigure}{r}{48 mm}
\vskip-4mm
\centering
\includegraphics{mppics/pic-250}
\vskip-0mm
\end{wrapfigure}

\begin{thm}{Advanced exercise}\label{ex:milka}
Let $\Sigma$ be a smooth closed strictly convex surface 
in $\RR^3$ 
and $\gamma\:[0,\ell]\z\to \Sigma$ be a unit-speed minimizing geodesic.
Set $p\z=\gamma(0)$, $q=\gamma(\ell)$ and 
$$p_t=\gamma(t)-t\cdot\gamma'(t),$$ 
where $\gamma'(t)$ denotes the velocity vector of $\gamma$ at $t$.

Show that for any $t\in (0,\ell)$,
one {}\emph{cannot see}  $q$ from $p_t$;
that is, the line segment $[p_tq]$ intersects $\Sigma$ at a point distinct from $q$.

Show that the statement does not hold without assuming that $\gamma$ is minimizing.
\end{thm}

}

\section{Liberman's lemma}

The following lemma is a smooth analog of lemma proved by Joseph Liberman \cite{liberman}.

\begin{thm}{Liberman's lemma}\label{lem:liberman}
Assume $\gamma$ is a unit-speed geodesic on the graph $z=f(x,y)$ of a smooth convex function $f$ defined on an open subset of the plane.
Suppose that $\gamma(t)=(x(t),y(t),z(t))$
Then $t\mapsto z(t)$ is a convex function; that is, $z''(t)\ge 0$ for any $t$.
\end{thm}

\parit{Proof.}
Choose the orientation on the graph so that the unit normal vector $\Norm$ always points up;
that is, it has positive $z$-coordinate.

Since $\gamma$ is a geodesic, we have $\gamma''(t)\perp\T_{\gamma(t)}$,
or equivalently $\gamma''(t)$ is proportional to $\Norm_{\gamma(t)}$ for any $t$.
Further 
\[\gamma''(t)=k(t)\cdot\Norm_{\gamma(t)},\]
where $k(t)$ is the normal curvature at $\gamma(t)$ in the direction of $\gamma'(t)$.

Therefore
\[z''(t)=\cos(\theta_\gamma(t))\cdot k(t)\cdot\Norm_{\gamma(t)},\eqlbl{eq:z''}\]
where $\theta_\gamma(t)$ denotes the angle between $\Norm_{\gamma(t)}$ and the $z$-axis.

Since $\Norm$ points up, we have $\theta_\gamma(t)<\tfrac\pi2$, or equivalently
\[\cos(\theta_\gamma(t))>0\]
for any $t$.

Since $f$ is convex, we have that tangent plane supports the graph from below at any point;
in particular $k(t)\ge 0$ for any $t$.
It follows that the right hand side in \ref{eq:z''} is nonnegative;
whence the statement follows.
\qeds

\begin{thm}{Exercise}\label{ex:rho''}
Assume $\gamma$ is a unit-speed geodesic on a smooth convex surface $\Sigma$ and $p$ in the interior of a convex set bounded by $\Sigma$.
Set $\rho(t)=|p-\gamma(t)|^2$.
Show that $\rho''(t)\le 2$ for any $t$.
\end{thm}



\section{Total curvature of geodesics}

\begin{thm}{Theorem}\label{thm:usov}
Assume $\Sigma$ is a graph $z=f(x,y)$ of a convex $\ell$-Lipschitz function $f$ defined on an open set in the $(x,y)$-plane.
Then the total curvature of any geodesic in $\Sigma$ is at most $2\cdot \ell$.
\end{thm}

The above theorem was proved by Vladimir Usov \cite{usov},
an amusing generalization was found by David Berg \cite{berg}.

\parit{Proof.}
Let $\gamma(t)=(x(t),y(t),z(t))$ be a unit-speed geodesic on $\Sigma$.
According to Liberman's lemma 
$z(t)$ is convex.

Since the slope of $f$ is at most $\ell$, we have
\[|z'(t)|\le \tfrac{\ell}{\sqrt{1+\ell^2}}.\]
If $\gamma$ is defined on the interval $[a,b]$, then
\begin{align*}
\int_a^b z''(t)&=z'(b)-z'(a)\le 
\\
&\le 2\cdot \tfrac{\ell}{\sqrt{1+\ell^2}}.
\end{align*}

Further, note that $z''$ is the projection of $\gamma''$ to the $z$-axis.
Since $f$ is $\ell$-Lipschitz, the tangent plane $\T_{\gamma (t)} \Sigma$ cannot have slope greater than $\ell$ for any $t$.
Because $\gamma ''$ is perpendicular to that plane, we have that
\[|\gamma'' (t)|  \le  z''(t)\cdot\sqrt{1+ \ell ^2}.\]

Recall that $\tc\gamma$ denotes the total curvature of curve $\gamma$.
It follows that 
\begin{align*}
\tc\gamma&=\int_a^b|\gamma'' (t)|\cdot dt\le 
\\
&\le \sqrt{1+ \ell ^2}\cdot  \int_a^b z''(t)\cdot dt\le 
\\
&\le 2\cdot \ell.
\end{align*}
\qedsf

\begin{thm}{Exercise}\label{ex:usov-exact}
Note that the graph $z=\ell\cdot\sqrt{x^2+y^2}$ with removed origin is a smooth surface; denote it by $\Sigma$.
Show that any both side infinite geodesic $\gamma$ has total curvature exactly $2\cdot \ell$.
\end{thm}

Note that the function $f(x,y)=\ell\cdot\sqrt{x^2+y^2}$ is $\ell$-Lipschitz.
The graph $z=f(x,y)$ in the exercise can be smoothed in a neighborhood of the origin while keeping it convex.
It follows that the estimate in the Usov's theorem is optimal.


\begin{thm}{Exercise}\label{ex:rough-bound-mountain}
Assume $f$ is a convex $\tfrac32$-Lipschitz function defined on the $(x,y)$-plane.
Show that any geodesic $\gamma$ on the graph $z=f(x,y)$ is simple;
that is, it has no self-intersections.

Construct a convex $2$-Lipschitz function defined on the $(x,y)$-plane
with a nonsimple geodesic $\gamma$ on its graph $z=f(x,y)$.
\end{thm}


\begin{thm}{Theorem}\label{thm:tc-of-mingeod}
Suppose a smooth surface $\Sigma$ bounds a convex set $K$ in the Euclidean space.
Assume $B(0,\eps)\subset K\subset B(0,1)$.
Then the total curvatures of any shortest path in $\Sigma$ can be bounded in terms of~$\eps$. 
\end{thm}

\begin{wrapfigure}{r}{48 mm}
\vskip-4mm
\centering
\includegraphics{mppics/pic-83}
\vskip-0mm
\end{wrapfigure}

The following exercise will guide you thru the proof of the theorem. 

\begin{thm}{Exercise}\label{ex:bound-tc}
Let $\Sigma$ be as in the theorem and $\gamma$ be a unit-speed shortest path in $\Sigma$.
Denote by $\Norm_p$ the unit normal vector that points outside of $\Sigma$;
denote by $\theta_p$ the angle between $\Norm_p$ and the direction from the origin to a point $p\in\Sigma$.
Set $\rho(t)\z=|\gamma(t)|^2$; let $k(t)$ be the curvature of $\gamma$ at $t$.
\begin{enumerate}[(a)]
\item Show that $\cos\theta_p\ge \eps$ for any $p\in \Sigma$.
\item Show that $|\rho'(t)|\le 2$ for any $t$.
\item Show that 
\[\rho''(t)=2-2\cdot k(t)\cdot \cos \theta_{\gamma(t)}\cdot |\gamma(t)|\]
for any $t$.
\item Use the closest-point projection from the unit sphere to $\Sigma$ to show that 
\[\length \gamma\le \pi.\]
\item Use the statements above to conclude that 
\[\tc\gamma\le \frac{100}{\eps^2}.\]
\end{enumerate}
\end{thm}

Note that the obtained bound on total curvature goes to infinity as $\eps\to 0$.
In fact there is a bound independent of $\eps$ \cite{lebedeva-petrunin}.
