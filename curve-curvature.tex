\chapter{Curvature of curves}

\section{Total curvature}\label{sec:total-curvature-smooth}

Here we introduce the so called \emph{total curvature of curve}.
In general term \emph{curvature} is used for something that measures how 
a geometric object deviates from being a straight;
total curvature is not an exception --- as you will see if the total curvature of a curve is vanishing then the curve runs along a straight line.


Let $\alpha\:[a,b]\to\RR^3$ be a \emph{smooth} \emph{regular} curve --- smooth means that
the velocity vector $\alpha'(t)$ is defined and continuous with respect to $t$ and regular means that $\alpha'(t)\ne 0$ for any $t$.
If the curve $\alpha$ is closed then we assume in addition that $\alpha'(a)=\alpha'(b)$.

Denote by $\tau(t)$ the unit vector in the direction of $\alpha'(t)$;
that is, $\tau(t)=\tfrac{\alpha'(t)}{|\alpha'(t)|}$.
The $\tau\:[a,b]\to\mathbb{S}^2$ is an other curve which is called \emph{tangent indicatrix} of $\alpha$.
The length of $\tau$ is called \emph{total curvature of}~$\alpha$;
that is,
\[\tc\alpha:=\length\tau.\]

\begin{thm}{Exercise}\label{ex:fenchel}
Show that 
\[\tc\alpha\ge 2\cdot \pi\]
for any smooth closed regular curve $\alpha$.

Moreover, the equality holds if and only if $\alpha$ is a closed and convex curve that lies in a plane.
\end{thm}

The above exercise is the so called Fenchel's theorem.


\section{On inscribed polygonal lines}

The total curvature of a polygonal line is defined as the sum of its external angles.
More precisely, assume $p_0,\dots,p_n$ are the vertexes of a polygonal line.
The external angle at the vertex $p_i$ is defined as $\alpha_i=\pi-\measuredangle p_{i-1}p_ip_{i+1}$.
The the total curvature of the polygonal line $p_0\dots p_n$ is defined as the sum
\[\alpha_1+\dots+\alpha_{n-1};\]
it is defined if the polygonal line is \emph{nondegenerate}; that is, $p_{i-1}\ne p_i$ for any $i$.

If the polygonal line $p_0\dots p_n$ is closed; that is $p_0=p_n$ you add one more angle 
\[\alpha_0+\alpha_1+\dots+\alpha_{n-1},\]
where $\alpha_0=\pi-\measuredangle p_{n}p_0p_{1}$.

Let us define tangent indicatrix for a polygonal line $p_0\dots p_n$ as a spherical polygonal line (each edge is an arc of big circle of the sphere) that vertexes are unit vectors $\xi_1,\dots,\xi_n$ in the directions of $p_1-p_0$, $p_2-p_1,\dots, p_n-p_{n-1}$ correspondingly;
if the  polygonal line is closed then we add one more vertex $\xi_0$ in the directions of $p_0-p_n$ and two more edges $\xi_0\xi_1$ and $\xi_n\xi_0$ so the indicatrix of closed polygonal line is a closed spherical polygonal line.

Note that the total curvature of polygonal line is the length of its tangent indicatrix.

\begin{thm}{Exercise}\label{ex:monotonic-tc}
Let $a,b,c,d$ and $x$ be distinct points in $\RR^3$.
Show that 
\[\tc abcd\ge \tc abxcd.\]
\end{thm}

\begin{thm}{Exercise}\label{ex:poly-fenchel}
Use Exercise~\ref{ex:monotonic-tc} to prove an analog of Fenchel's theorem (Exercise~\ref{ex:fenchel}) for closed polygonal lines.
\end{thm}


Let  $\alpha\:[a,b]\to \RR^3$ be a curve and  $a=t_0<\dots<t_n=b$ a partition.
Set $p_i=\alpha(t_i)$.
Then the polygonal line $p_0\dots p_n$ is called inscribed in~$\alpha$. 

We gave two definitions of total curvature:
the first one is given in Section~\ref{sec:total-curvature-smooth} via tangent indicatrix --- it works for smooth regular curves;
the second, via external angles --- it works for polygonal lines.
The latter can be used to define total curvature of arbitrary nonconstant curves.

\begin{thm}{Definition}\label{def:total-curv-poly}
The total curvature of a nonconstant curve $\alpha$ is the exact upper bound on the total curvatures of inscribed nondegenerate polygonal lines.
\end{thm}

The following definition tells that these two definitions agree.

\begin{thm}{Theorem}\label{thm:total-curvature=}
For smooth regular curves the two definitions of total curvature agree;
that is, for any regular curve, the length of its tangent indicatrix is equal to the exact upper bound on the total curvatures of inscribed nondegenerate polygonal lines.
\end{thm}

Note that from the theorem and Exercise~\ref{ex:poly-fenchel}, we get a generalization of Fenchel's theorem (Exercise~\ref{ex:fenchel}).

\begin{thm}{Lemma}\label{lem:uvw}
Let $\alpha\:[a,b]\to\RR^3$ be a smooth regular curve.
Consider three unit vectors $\lambda$, $\mu$ and $\nu$ in the directions of
$\alpha'(a)$, $\alpha(b)-\alpha(a)$ and $\alpha'(b)$ correspondingly.
Then 
\[\tc\alpha\ge\measuredangle (\lambda,\mu)+\measuredangle (\mu,\nu).\]
\end{thm}

\begin{wrapfigure}{r}{45 mm}
\vskip-7mm
\centering
\includegraphics{mppics/pic-12}
\vskip0mm
\end{wrapfigure}

\parit{Proof.}
The tangent indicatrix $\tau$ runs from $\lambda$ to $\nu$ in the unit sphere~$\mathbb{S}^2$.

Note that $\tau$ can not be separated from $\mu$ by an equator.
Indeed the vector 
\[\alpha(b)-\alpha(a)=\int_a^b\alpha'(t)\cdot dt\]
points in the same direction as $\mu$.
Therefore if the indicatrix $\tau=\tfrac{\alpha'}{|\alpha'|}$ lies in a hemisphere then $\mu$ lies in the same hemisphere. 

Fix an equator $\ell$ in general position.
If $\ell$ intersects the spherical polygonal line $\lambda \mu \nu$ at one point, then $\ell$ separates $\lambda$ from $\nu$ and therefore it must intersect $\tau$.
If $\ell$ intersects the spherical polygonal line $\lambda \mu \nu$ at two points, then $\ell$ separates $\mu$ from $\lambda$ and $\nu$ and therefore it must intersect $\tau$ at two points --- $\tau$ must cross $\ell$ and then come back.
It follows that for almost all equators the number of intesections with the spherical polygonal line $\lambda \mu\nu$ can not exceed the number of intersections with $\tau$.
By the spherical Crofton formula (\ref{thm:crofton-sphere}), $\tau$ is longer than the spherical polygonal line $\lambda \mu\nu$.
But the polygonal line $\lambda \mu\nu$ has length $\measuredangle (\lambda,\mu)+\measuredangle (\mu,\nu)$, hence the result.
\qeds
 

\parit{Proof of \ref{thm:total-curvature=}.}
Let $\alpha\:[a,b]\to\RR^3$ be a smooth curve.
Fix a partition $a\z=t_0<\dots<t_n=b$ and consider the corresponding inscribed polygonal line $\beta=w_0\dots w_n$.
Let $\chi=\xi_1\dots\xi_n$ be its tangent indicatrix --- this is a spherical polygonal line;
we assume that $\chi(t_i)=\xi_i$ and it has constant speed on each arc.

Consider a sequence of finer and finer partitions, denote by $\beta_n$ and $\chi_n$ the correspoponging inscribed polygonal line and its tangent indicatrix;
since $\alpha$ is smooth, the $\chi_n$ converges pointwise to the $\tau$ --- the thangent indicatrix of $\alpha$.
By semi-continuity of length functional, we get that 
\begin{align*}
\tc\alpha&=\length \tau\le  
\\
&\le \liminf_{n\to\infty}\length \chi_n\le
\\
&\le \sup\{\tc\beta\},
\end{align*}
where the list upper bound is taken for all partitions and corresponding inscribed polygonal lines $\beta$.

It remains to prove that
\[\tc\alpha\ge \tc\beta,\eqlbl{eq:tc-alpha<tc-beta}\]
for any polygonal line $\beta$ inscribed in $\alpha$.
Let $\zeta_i$ be the unit vector in the direction of $\alpha'(t_i)$.
Consider the spherical polygonal line $\gamma=\zeta_0 \xi_1 \zeta_1 \xi_2\dots \xi_n \zeta_n$;
recall that $\beta=\xi_0\dots \xi_n$.
By triangle inequality, $\length \gamma\ge \length \beta$.
By Lemma~\ref{lem:uvw}, $\tc\alpha\ge\length\gamma$, 
hence \ref{eq:tc-alpha<tc-beta} follows.
\qeds
