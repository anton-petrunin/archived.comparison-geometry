\chapter{Curvature of curves}

\section{Total curvature}

Here we introduce the so called \emph{total curvature of curve}.
In general term \emph{curvature} is used for something that measures how 
a geometric object deviates from being a straight;
total curvature is not an exception --- as you will see if the total curvature of a curve is vanishing then the curve runs along a straight line.


Let $\alpha\:[a,b]\to\RR^3$ be a \emph{smooth} \emph{regular} curve --- smooth means that
the velocity vector $\alpha'(t)$ is defined and continuous with respect to $t$ and regular means that $\alpha'(t)\ne 0$ for any $t$.
If the curve $\alpha$ is closed then we assume in addition that $\alpha'(a)=\alpha'(b)$.

Denote by $\tau(t)$ the unit vector in the direction of $\alpha'(t)$;
that is, $\tau(t)=\tfrac{\alpha'(t)}{|\alpha'(t)|}$.
The $\tau\:[a,b]\to\mathbb{S}^2$ is an other curve which is called \emph{tangent indicatrix} of $\alpha$.
The length of $\tau$ is called \emph{total curvature of}~$\alpha$;
that is,
\[\tc\alpha:=\length\tau.\]

\begin{thm}{Exercise}\label{ex:fenchel}
Show that 
\[\tc\alpha\ge 2\cdot \pi\]
for any smooth closed regular curve $\alpha$.

Moreover, the equality holds if and only if $\alpha$ is a closed and convex curve that lies in a plane.
\end{thm}

The above exercise is the so called Fenchel's theorem.

\section{On inscribed broken lines}

The total curvature of a polygonal line is defined as the sum of its external angles.
More precisely, assume $p_0,\dots,p_n$ are the vertexes of a polygonal line then.
The external angle at the vertex $p_i$ is defined as $\alpha_i=\pi-\measuredangle p_{i-1}p_ip_{i+1}$.
The the total curvature of the polygonal line $p_0\dots p_n$ is defined as the sum
\[\alpha_1+\dots+\alpha_{n-1}.\]

If the polygonal line $p_0\dots p_n$ is closed; that is $p_0=p_n$ you add one more angle 
\[\alpha_0+\alpha_1+\dots+\alpha_{n-1},\]
where $\alpha_0=\pi-\measuredangle p_{n}p_0p_{1}$.

\begin{thm}{Exercise}\label{ex:monotonic-tc}
Let $a,b,c,d$ and $x$ be distinct points in $\RR^3$.
Show that 
\[\tc abcd\ge \tc abxcd.\]
\end{thm}

\begin{thm}{Exercise}
Use Exercise~{ex:monotonic-tc} to prove an analog of Fenchel's theorem (Exercise~\ref{ex:fenchel}) for closed polygonal lines.
\end{thm}


Let  $\alpha\:[a,b]\to \RR^3$ be a curve and  $a=t_0<\dots<t_n=b$ a partition.
Set $p_i=\alpha(t_i)$.
Then the polygonal line $p_0\dots p_n$ is called inscribed in~$\alpha$.

We gave two definitions of total curvature:
the first, via tangent indicatrix --- it works for smooth regular curves;
the second, via external angles --- it works for polygonal lines.
The following definition tells that these two definitions agree.

\begin{thm}{Theorem}
Total curvature of a smooth regular curve equals to the exact upper bound on the total curvature of inscribed polygonal lines; if the original curve is closed then the inscribed polygonal line is assumed to be closed as well. 
\end{thm}
