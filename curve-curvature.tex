\chapter{Total curvature}

\section{Smooth regular curves}\label{sec:total-curvature-smooth}

Here we introduce the so called \emph{total curvature of curve}.
In general term \emph{curvature} is used for something that measures how 
a geometric object deviates from being a straight;
total curvature is not an exception --- as you will see if the total curvature of a curve is vanishing then the curve runs along a straight line.


Let $\alpha\:[a,b]\to\RR^3$ be a \emph{smooth} \emph{regular} curve --- smooth means that
the velocity vector $\alpha'(t)$ is defined and continuous with respect to $t$ and regular means that $\alpha'(t)\ne 0$ for any $t$.
If the curve $\alpha$ is closed then we assume in addition that $\alpha'(a)=\alpha'(b)$.

Denote by $\tau(t)$ the unit vector in the direction of $\alpha'(t)$;
that is, $\tau(t)=\tfrac{\alpha'(t)}{|\alpha'(t)|}$.
The $\tau\:[a,b]\to\mathbb{S}^2$ is an other curve which is called \emph{tangent indicatrix} of $\alpha$.
The length of $\tau$ is called \emph{total curvature of}~$\alpha$;
that is,
\[\tc\alpha:=\length\tau.\]

\begin{thm}{Exercise}\label{ex:fenchel}
Show that 
\[\tc\alpha\ge 2\cdot \pi\]
for any smooth closed regular curve $\alpha$.

Moreover, the equality holds if and only if $\alpha$ is a closed and convex curve that lies in a plane.
\end{thm}

The above exercise is the so called Fenchel's theorem.


\section{General definition}

The total curvature of a polygonal line is defined as the sum of its external angles.

More precisely, 
for a polygonal line $\beta=p_0\dots p_n$,
the external angle at the vertex $p_i$ is defined as $\alpha_i=\pi-\measuredangle p_{i-1}p_ip_{i+1}$.
The total curvature of the polygonal line $\beta=p_0\dots p_n$ is defined as the sum
\[\tc\beta=\alpha_1+\dots+\alpha_{n-1};\]
it is defined if the polygonal line is \emph{nondegenerate}; that is, $p_{i-1}\ne p_i$ for any $i$.

If the polygonal line $p_0\dots p_n$ is closed; that is $p_0=p_n$ you add one more angle 
\[\alpha_0+\alpha_1+\dots+\alpha_{n-1},\]
where $\alpha_0=\pi-\measuredangle p_{n}p_0p_{1}$.

One can define the tangent indicatrix for a polygonal line $\beta$ as a spherical polygonal line (each edge is an arc of big circle of the sphere) that vertexes are unit vectors $\xi_1,\dots,\xi_n$ in the directions of $p_1-p_0$, $p_2-p_1,\dots, p_n-p_{n-1}$ correspondingly;
if the  polygonal line is closed then we add one more vertex $\xi_0$ in the directions of $p_0-p_n$ and two more edges $\xi_0\xi_1$ and $\xi_n\xi_0$ so the indicatrix of closed polygonal line is a closed spherical polygonal line.

Note that the total curvature of polygonal line is the length of its tangent indicatrix.

\begin{thm}{Exercise}\label{ex:monotonic-tc}
Let $a,b,c,d$ and $x$ be distinct points in $\RR^3$.
Show that 
\[\tc abcd\ge \tc abxcd.\]
\end{thm}

\begin{thm}{Exercise}\label{ex:poly-fenchel}
Use Exercise~\ref{ex:monotonic-tc} to prove an analog of Fenchel's theorem (Exercise~\ref{ex:fenchel}) for closed polygonal lines.
\end{thm}


Let  $\alpha\:[a,b]\to \RR^3$ be a curve and  $a=t_0<\dots<t_n=b$ a partition.
Set $p_i=\alpha(t_i)$.
Then the polygonal line $p_0\dots p_n$ is called inscribed in~$\alpha$. 

We gave two definitions of total curvature:
the first one is given in Section~\ref{sec:total-curvature-smooth} via tangent indicatrix --- it works for smooth regular curves;
the second, via external angles --- it works for polygonal lines.
The latter can be used to define total curvature of arbitrary curves.

\begin{thm}{Definition}\label{def:total-curv-poly}
The total curvature of a nonconstant curve $\alpha$ is the exact upper bound on the total curvatures of inscribed nondegenerate polygonal lines;
if the curve is closed then we assume that the inscribed polygonal lines are closed as well.
\end{thm}

We need to assume that the curve is nonconstant, otherwise it does not admit inscribed polygonal lines that are not trivial.

\begin{thm}{Exercise}
Show that the total curvature is lower semi-continuous with respect to pointwise convergence of curves.
That is, if a sequence
of curves $\alpha_n\:[a,b]\to \RR^3$ converges pointwise 
to a curve $\alpha_\infty\:[a,b]\z\to \RR^3$, then 
\[\liminf_{n\to\infty} \tc\alpha_n \ge \tc\alpha_\infty.\]
\end{thm}

\parit{Hint:} Modify the proof of semi-continuity of length (Theorem~\ref{thm:length-semicont}).


The following definition tells that these two definitions agree.

\begin{thm}{Theorem}\label{thm:total-curvature=}
For smooth regular curves the two definitions of total curvature agree;
that is, for any regular curve, the length of its tangent indicatrix is equal to the exact upper bound on the total curvatures of inscribed nondegenerate polygonal lines.
\end{thm}

Note that from the theorem and Exercise~\ref{ex:poly-fenchel}, we get a generalization of Fenchel's theorem (Exercise~\ref{ex:fenchel}) --- it works for arbtrary closed curves, not necessary smooth and regular.

\begin{thm}{Lemma}\label{lem:uvw}
Let $\alpha\:[a,b]\to\RR^3$ be a smooth regular curve.
Consider three unit vectors $\lambda$, $\mu$ and $\nu$ in the directions of
$\alpha'(a)$, $\alpha(b)-\alpha(a)$ and $\alpha'(b)$ correspondingly.
Then 
\[\tc\alpha\ge\measuredangle (\lambda,\mu)+\measuredangle (\mu,\nu).\]
\end{thm}

\begin{wrapfigure}{r}{45 mm}
\vskip-7mm
\centering
\includegraphics{mppics/pic-12}
\vskip0mm
\end{wrapfigure}

\parit{Proof.}
The tangent indicatrix $\tau$ runs from $\lambda$ to $\nu$ in the unit sphere~$\mathbb{S}^2$.

Note that $\tau$ can not be separated from $\mu$ by an equator.
Indeed the vector 
\[\alpha(b)-\alpha(a)=\int_a^b\alpha'(t)\cdot dt\]
points in the same direction as $\mu$.
Therefore if the indicatrix $\tau=\tfrac{\alpha'}{|\alpha'|}$ lies in a hemisphere then $\mu$ lies in the same hemisphere. 

Fix an equator $\ell$ in general position.
If $\ell$ intersects the spherical polygonal line $\lambda \mu \nu$ at one point, then $\ell$ separates $\lambda$ from $\nu$ and therefore it must intersect $\tau$.
If $\ell$ intersects the spherical polygonal line $\lambda \mu \nu$ at two points, then $\ell$ separates $\mu$ from $\lambda$ and $\nu$ and therefore it must intersect $\tau$ at two points --- $\tau$ must cross $\ell$ and then come back.
It follows that for almost all equators the number of intesections with the spherical polygonal line $\lambda \mu\nu$ can not exceed the number of intersections with $\tau$.
By the spherical Crofton formula (\ref{thm:crofton-sphere}), $\tau$ is longer than the spherical polygonal line $\lambda \mu\nu$.
But the polygonal line $\lambda \mu\nu$ has length $\measuredangle (\lambda,\mu)+\measuredangle (\mu,\nu)$, hence the result.
\qeds

Let us sketch an alternative proof of the lemma which is built on Fenchel's theorem. 

\parit{An alternative proof of the lemma.}
Note that the curve $\alpha$ can be extended to a smooth regular closed curve $\hat\alpha$ by an arc $\beta$ that starts from $\alpha(b)$ in the same direction as $\alpha$ turns and joins the segment $[\alpha(b),\alpha(a)]$ runs along the segment close to $\alpha(a)$ turns and joints $\alpha$ smoothly at $\alpha(a)$.

Note that the total curvature of $\beta$ can be made arbitrary close to $2\cdot\pi -\measuredangle (\lambda,\mu)-\measuredangle (\mu,\nu)$.
Indeed, $\beta$ needs a bit more than $\pi -\measuredangle (\mu,\nu)$ to turn an join the segment $[\alpha(b),\alpha(a)]$ and bit more than $\pi -\measuredangle (\lambda,\mu)$ to turn an join the segment $\alpha$.

By Fenchel's theorem,
\[\tc\hat\alpha\ge 2\cdot \pi.\]
Evidently 
\[\tc\hat\alpha=\tc\alpha+\tc\beta,\]
hence the lemma follows.
\qeds


\parit{Proof of \ref{thm:total-curvature=}.}
Let $\alpha\:[a,b]\to\RR^3$ be a smooth curve.
Fix a partition $a\z=t_0<\dots<t_n=b$ and consider the corresponding inscribed polygonal line $\beta=w_0\dots w_n$.
Let $\chi=\xi_1\dots\xi_n$ be its tangent indicatrix --- this is a spherical polygonal line;
we assume that $\chi(t_i)=\xi_i$ and it has constant speed on each arc.

Consider a sequence of finer and finer partitions, denote by $\beta_n$ and $\chi_n$ the correspoponging inscribed polygonal line and its tangent indicatrix;
since $\alpha$ is smooth, the $\chi_n$ converges pointwise to the $\tau$ --- the thangent indicatrix of $\alpha$.
By semi-continuity of length functional, we get that 
\begin{align*}
\tc\alpha&=\length \tau\le  
\\
&\le \liminf_{n\to\infty}\length \chi_n=
\\
&= \liminf_{n\to\infty}\tc \beta_n\le
\\
&\le \sup\{\tc\beta\},
\end{align*}
where the list upper bound is taken for all partitions and corresponding inscribed polygonal lines $\beta$.

It remains to prove that
\[\tc\alpha\ge \tc\beta,\eqlbl{eq:tc-alpha<tc-beta}\]
for any polygonal line $\beta$ inscribed in $\alpha$.
Let $\zeta_i$ be the unit vector in the direction of $\alpha'(t_i)$.
Consider the spherical polygonal line $\gamma\z=\zeta_0 \xi_1 \zeta_1 \xi_2\dots \xi_n \zeta_n$;
recall that $\chi=\xi_0\dots \xi_n$.
By triangle inequality, 
\[\length \gamma\ge \length \chi=\tc\beta.\]
By Lemma~\ref{lem:uvw}, 
\[\tc\alpha\ge\length\gamma,\] 
hence \ref{eq:tc-alpha<tc-beta} follows.
\qeds




\section{Crofton again}

Given a curve $\alpha$ in $\RR^3$ and a unit vector $u$, denote by $\alpha_{u^\perp}$ 
and $\alpha_u$ the projection of $\alpha$ to the plane perpendicular to $u$ and the line parallel to $u$ correspondingly.

To prove the following proposition, apply the spherical Crofton formula to the tangent indicatrix of $\alpha$.

\begin{thm}{Proposition}\label{prop:tc-crofton}
Let $\alpha$ be a polygonal line in $\RR^3$.
Show that 
\begin{align*}
\tc\alpha
&=\overline{\tc\alpha_{u^\perp}}=
\\
&=\overline{\tc\alpha_u}.
\end{align*}
\end{thm}

Note that since the curve $\alpha_u$ runs back and forth along one line.
Each change of its direction contributes $\pi$ to the total curvature of $\alpha_u$.
Therefore the total curvature of $\alpha_u$ is $n\cdot\pi$, where $n$ is the number of switches of the direction.
Since $n$ has to be even, $\tc\alpha_u$ may take values $2\cdot\pi$, $4\cdot\pi$, $6\cdot\pi$ and so on.

\begin{thm}{Exercise}
Use the proposition and the observation above to give yet an other proof of  Fenchel's theorem (Exercise~\ref{ex:fenchel}).
\end{thm}

\section{F\'ary--Milnor theorem}

\begin{thm}{Theorem}
The total curvature of any nontrival knot is at least $4\cdot\pi$. 
\end{thm}

The famous F\'ary--Milnor theorem states that the inequality is strict;
that is, the total curvature of any nontrival knot \emph{exceeds} $4\cdot\pi$.
It is easy to construct a trefoil knot with total curvature arbitrary close to $4\cdot\pi$;
therefore this result is optimal.
The question was raised by Karol Borsuk \cite{borsuk} and answered independently by Istv\'an F\'ary and John Milnor \cite{fary, milnor}. 

In the proof we will use the following fact: \emph{if a height function has only one local minimum and one local maximum on a closed simple polygonal line then the line is a trivial knot.}
It is easy to prove assuming that we gave the definitions of nontrivial knot.
Roughly, if a height function has only one local minimum and one local maximum, then at each intermediate height, there are
exactly two points of the curve.
Connecting each such pair with a straight segment, we obtain a disk bounded by the knot.
Therefore the know is trivial.


A standard introduction to knot theory defines knots as simple closed polygonal lines, so in the proof we use this agreement; alternatively one can define knot as a closed smooth regular curve, but this approach requires more work.

\parit{Proof.}
Let $\alpha$ be a simple closed polygonal line.
Assume its total curvature is less that $4\cdot\pi$.
Then by Proposition~\ref{prop:tc-crofton}, 
\[\tc\alpha_u<4\cdot\pi\]
for some unit vector $u$.
Moreover, we can assume that $u$ points in a generic direction;
that is, $u$ is not perpendicular to any edge of $\alpha$.

The total curvature of $\alpha_u$ is $\pi$ times the number of turns of $\alpha_u$
which has to be an even number.
It follows that number of turns of $\alpha_u$ is at most $2$;
it can not be less than 2 for generic direction and therefore it is $2$.
That is, if we rotate the space so that $u$ pints to the top,
than the height function has exactly one minimum and one maximum;
by the fact stated above $\alpha$ is a trivial knot --- hence the result.
\qeds

Let us give a sketch of another proof, based on the original idea of Istv\'an F\'ary.

\begin{wrapfigure}{r}{30 mm}
\vskip-0mm
\centering
\includegraphics{mppics/pic-13}
\vskip0mm
\end{wrapfigure}

\parit{Alternative proof.}
Assume that the knot is a polygonal line.

Consider a projection of knot to a plane in general position.
That is, we assume that the self-intersections of the projection are at most double and the projection of each edge does not degenerate.
The obtained closed polygonal line $\beta=p_1p_2\dots p_n$ divides the plane into domains, one is unbounded, denote by $U$ and the others are bounded.

Let us show that if the knot is nontrivial then there is a bounded domain $D$ that does not share a borderline with unbounded.

First note that all domains can be colored in a cheeseboard order;
that is, they can be colored in black and white in such a way that domains with common border  line get different colors.
If the unbounded domain gets white color and every other domain is black then one can untie the knot by flipping these domains one by one.

\begin{wrapfigure}{r}{30 mm}
\vskip-0mm
\centering
\includegraphics{mppics/pic-14}
\vskip0mm
\end{wrapfigure}

Therefore among the bounded domains there is a white domain, denote it by $D$.
The domain $D$ can not have a borderline with $U$ since they have the same color.
Fix a point $o$ in this domain.

Set 
\begin{align*}
\alpha_i&=\pi-\measuredangle p_{i-1}p_ip_{i+1},
\\
\beta_i&=\measuredangle p_i o p_{i+1},
\\
\gamma_i&=\measuredangle o p_i p_{i+1}.
\end{align*}
Here indexes are taken modulo $n$; in particular, $n+1$ means $1$.


Note that $\alpha_i$ is the external angle at $p_i$;
therefore 
\[\tc\beta= \alpha_1+\dots+\alpha_n\]

\begin{figure}[h]
\vskip-0mm
\centering
\includegraphics{mppics/pic-15}
\vskip0mm
\end{figure}

Direct calculations show that 
\[\alpha_i\ge \beta_{i-1}+\gamma_{i-1}-\gamma_i.\]
(Two pictures with the same angles, we have equality on the first one and strict inequality on the second one.)

It follows that 
\[\alpha_1+\dots+\alpha_n\ge \beta_1+\dots+\beta_n.\]

The last sum 
is the total angle at  which $\beta$ seen from $o$ counted with multiplicity. 
The boundary of $D$ contributes at least $2\cdot\pi$ to this sum and the boundary of $U$ contributes an other $2\cdot\pi$;
since their boundaries do not overlap we get 
\[\beta_1+\dots+\beta_n\ge 4\cdot\pi,\]
hence the result.
\qeds



\begin{thm}{Exercise}
Construct a closed smooth simple curve with total curvature arbitrary close to $2\cdot\pi$ such that its projection to any plane has at least $10$ self-intersections.   
\end{thm}

\section{DNA inequality}


\begin{thm}{Theorem}
Let $\alpha$ be a closed curve that lies in a unit disc.
Then 
\[\tc\alpha\ge \length\alpha.\]
\end{thm}

Note that if $\length\alpha\le2\cdot\pi$, then Fenchel's theorem gives a better estimate,
for longer curves it gives something new.

\parit{Proof.}
Assume $\alpha$ is a polygonal line.

Fix a unit vector $u$.
Note that the curve $\alpha_u$ can run at most length 2 in one direction;
therefore the number of turns has to be at least $\tfrac12\cdot\length\alpha$.
Since each turn of $\alpha_u$ contributes $\pi$ to its total curvature, we get
\[\tc\alpha_u\ge \tfrac\pi2\cdot\length\alpha_u.\]

The same inequality holds for the average values of left and right hand sides;
that is,
\[\overline{\tc\alpha_u}
\ge \tfrac\pi2\cdot\overline{\length\alpha_u}.\]
It remains to apply the Crofton's formula and Proposition~\ref{prop:tc-crofton}.

It remains to reduce the general case to polygonal lines.
Let us inscribe a polygonal line $\beta$ in $\alpha$ with length sufficiently close to length of $\alpha$;
that is, given $\eps>0$, we choose an inscribed polygonal line $\beta$ such that 
\[\length \alpha<\length \beta+\eps.\]
By the definition of total curvature (\ref{def:total-curv-poly}) and from above
\begin{align*}
\tc\alpha&\ge\tc\beta\ge
\\
&\ge \length\beta>
\\
&>\length\alpha-\eps.
\end{align*}
The statement follows since $\eps$ is arbitrary positive number. 
\qeds

\parit{Alternative proof.}
Note that it is sufficient to consider a closed polygonal lines $\beta=p_0p_1\dots p_n-1$ in the unit disc.
We assume that $p_n=p_0$, $p_{n+1}=p_1$ and so on.
Denote by $\alpha_i$ the external angle at $p_i$.

Denote by $o$ the center of the disc.
Consider a sequence of triangles 
$\triangle q_0q_1s_0\cong \triangle p_0p_1o$,
$\triangle q_1q_2s_0\cong \triangle p_1p_2o$ 
and so on such that the points $q_0,q_1\dots$ lie on one line in the same order and the points $s_i$ lie on one side from this line.

Note that 
\[|s_n-s_0|=\length \beta.\]
Therefore 
\[|s_0-s_1|+\dots+|s_{n-1}-s_n|\ge \length \beta.\]

Note that 
\[|q_i-s_{i-1}|=|q_i-s_i|=|p_i-o|\le 1\]
and
\[\measuredangle s_{i-1}q_is_i\le \alpha_i\]
for each $i$.
Therefore 
\[|s_{i-1}-s_i|<\alpha_i\]
for each $i$.

Therefore
\begin{align*}
\tc \beta
&=\alpha_1+\dots+\alpha_n\ge
\\
&\ge |s_{0}-s_1|+\dots |s_{n-1}-s_n|\ge 
\\
&\ge\length \beta.
\end{align*}
Hence the result.
\qeds

\section{Curves of finite total curvature}

\begin{thm}{Exercise} 
Assume that a curve $\alpha\:[a,b]\to\RR^3$ has finite total curvature.
 Show that $\alpha$ is rectifiable.
\end{thm}

We say that a curve $\alpha\:[a,b]\to\RR^3$ \emph{does not stop} if $\alpha$ is not constant on any subinterval of $[a,b]$. 

\begin{thm}{Exercise} 
Assume that the curve $\alpha$ does not stop and its the total curvature is less than $\pi$.
Show that $\alpha$ is simple; that is it has no self-intersections.
\end{thm}

\begin{thm}{Exercise-definition} 
Assume that a curve $\alpha\:[a,b]\to\RR^3$ does not stop and has finite total curvature.
Show that the direction of exit and entrance is defined for any point.

That is for any $t_0\in [a,b)$ the unit vector  
\[v(\eps)=\frac{\alpha(t_0+\eps)-\alpha(t_0)}{|\alpha(t_0+\eps)-\alpha(t_0)|}\] converges as $\eps\to0^+$;
its limit is called the direction of exit and it will be denoted by $\alpha^+(t_0)$

Analogously, for any $t_0\in (a,b]$ the unit vector  
\[w(\eps)=\frac{\alpha(t_0-\eps)-\alpha(t_0)}{|\alpha(t_0-\eps)-\alpha(t_0)|}\] 
converges as $\eps\to0^+$;
its limit is called the direction of entrance and it will be denoted by $\alpha^-(t_0)$.
\end{thm}

\begin{thm}{Exercise} 
Assume that a curve $\alpha\:[a,b]\to\RR^3$ does not stop and has finite total curvature.
Show that 
\[\alpha^+(t)=-\alpha^-(t)\]
at all $t\in[a,b]$ except a countable subset.
\end{thm}

\begin{thm}{Exercise} 
Assume a sequence of curves $\alpha_n\:[a,b]\to\RR^3$ converges to a curve $\alpha_\infty\:[a,b]\to\RR^3$ and
\[\lim_{n\to\infty}\length\alpha_n>\length\alpha_\infty.\]
Show that 
\[\tc\alpha_n\to\infty\quad\text{as}\quad n\to\infty.\]

\end{thm}



