\chapter{Total curvature}

\section{Smooth regular curves}\label{sec:total-curvature-smooth}

Here we introduce the so called \emph{total curvature of a curve}.
In general the term \emph{curvature} is used for something that measures how much 
a geometric object deviates from being straight;
total curvature is not an exception --- as you will see, if the total curvature of a curve is zero, then the curve runs along a straight line.


Let $\alpha\:[a,b]\to\RR^3$ be a \emph{smooth} \emph{regular} curve --- smooth means that
the velocity vector $\alpha'(t)$ is defined and is continuous with respect to $t$, and regular means that $\alpha'(t)\ne 0$ for all $t$.
If the curve $\alpha$ is closed then we assume in addition that $\alpha'(a)=\alpha'(b)$.

Denote by $\tau(t)$ the unit vector in the direction of $\alpha'(t)$;
that is, $\tau(t)=\tfrac{\alpha'(t)}{|\alpha'(t)|}$.
Then $\tau\:[a,b]\to\mathbb{S}^2$ is another curve which is called \emph{tangent indicatrix} of $\alpha$.
The length of $\tau$ is called the \emph{total curvature of}~$\alpha$;
that is,
\[\tc\alpha:=\length\tau.\]

\begin{thm}{Exercise}\label{ex:fenchel}
Show that 
\[\tc\alpha\ge 2\cdot \pi\]
for any smooth closed regular curve $\alpha$.

Moreover, the equality holds if and only if $\alpha$ is a closed convex curve lying in a plane.
\end{thm}

The above exercise is the so called Fenchel's theorem.


\section{General definition}

The total curvature of a polygonal line is defined as the sum of its external angles.

More precisely, 
for a polygonal line $\beta=p_0\dots p_n$,
the external angle at the vertex $p_i$ is defined as $\alpha_i=\pi-\measuredangle p_{i-1}p_ip_{i+1}$.
The total curvature of the polygonal line $\beta=p_0\dots p_n$ is defined as the sum
\[\tc\beta=\alpha_1+\dots+\alpha_{n-1};\]
it is defined if the polygonal line is \emph{nondegenerate}; that is, $p_{i-1}\ne p_i$ for any $i$.

If the polygonal line $p_0\dots p_n$ is closed; that is $p_0=p_{n+1}$ you add two more angles 
\[\alpha_0+\alpha_1+\dots+\alpha_{n-1} + \alpha_n ,\]
where $\alpha_0=\pi-\measuredangle p_{n}p_0p_{1}$ and $\alpha _n = \pi - \measuredangle p_{n-1} p_n p_0$.

One can define the tangent indicatrix of a polygonal line $\beta$ as a spherical polygonal line (each edge is an arc of a big circle in the sphere) whose vertexes are the unit vectors $\xi_1,\dots,\xi_n$ in the directions of $p_1-p_0$, $p_2-p_1,\dots, p_n-p_{n-1}$ correspondingly;
if the  polygonal line is closed then we add one more vertex $\xi_0$ in the direction of $p_0-p_{n}$ and two more edges $\xi_0\xi_1$ and $\xi_n\xi_0$ so the indicatrix of a closed polygonal line is a closed spherical polygonal line.

Note that the total curvature of a polygonal line is the length of its tangent indicatrix.

\begin{thm}{Exercise}\label{ex:monotonic-tc}
Let $a,b,c,d$ and $x$ be distinct points in $\RR^3$.
Show that 
\[\tc abcd \leq \tc abxcd.\]
\end{thm}

\begin{thm}{Exercise}\label{ex:poly-fenchel}
Use Exercise~\ref{ex:monotonic-tc} to prove an analog of Fenchel's theorem (Exercise~\ref{ex:fenchel}) for closed polygonal lines.
\end{thm}


We gave two definitions of total curvature:
the first one is given in Section~\ref{sec:total-curvature-smooth} via the tangent indicatrix --- it works for smooth regular curves;
the second is via external angles --- it works for polygonal lines.
The latter can be used to define total curvature of arbitrary curves.


Let  $\alpha\:[a,b]\to \RR^3$ be a curve and  $a=t_0<\dots<t_n=b$ a partition.
Set $p_i=\alpha(t_i)$.
Then the polygonal line $p_0\dots p_n$ is said to be inscribed in~$\alpha$. 

\begin{thm}{Definition}\label{def:total-curv-poly}
The total curvature of a nonconstant curve $\alpha$ is the exact upper bound on the total curvatures of inscribed nondegenerate polygonal lines;
if $\alpha$ is closed then we assume that the inscribed polygonal lines are closed as well.
\end{thm}

We need to assume that the curve is nonconstant, otherwise it does not admit inscribed polygonal lines that are not trivial.

\begin{thm}{Exercise}
Show that the total curvature is lower semi-continuous with respect to pointwise convergence of curves.
That is, if a sequence
of curves $\alpha_n\:[a,b]\to \RR^3$ converges pointwise 
to a curve $\alpha_\infty\:[a,b]\z\to \RR^3$, then 
\[\liminf_{n\to\infty} \tc\alpha_n \ge \tc\alpha_\infty.\]
\end{thm}

\parit{Hint:} Modify the proof of semi-continuity of length (Theorem~\ref{thm:length-semicont}).


The following definition tells us that the two definitions agree.

\begin{thm}{Theorem}\label{thm:total-curvature=}
For smooth regular curves the two definitions of total curvature agree;
that is, for any regular curve, the length of its tangent indicatrix is equal to the exact upper bound of the total curvatures of inscribed nondegenerate polygonal lines.
\end{thm}

Note that from the theorem and Exercise~\ref{ex:poly-fenchel}, we get a generalization of Fenchel's theorem (Exercise~\ref{ex:fenchel}) --- it works for arbtrary closed curves, not necessary smooth and regular.

\begin{thm}{Lemma}\label{lem:uvw}
Let $\alpha\:[a,b]\to\RR^3$ be a smooth regular curve.
Consider three unit vectors $\lambda$, $\mu$ and $\nu$ in the directions of
$\alpha'(a)$, $\alpha(b)-\alpha(a)$ and $\alpha'(b)$ correspondingly.
Then 
\[\tc\alpha\ge\measuredangle (\lambda,\mu)+\measuredangle (\mu,\nu).\]
\end{thm}

\begin{wrapfigure}{r}{45 mm}
\vskip-7mm
\centering
\includegraphics{mppics/pic-12}
\vskip0mm
\end{wrapfigure}

\parit{Proof.}
The tangent indicatrix $\tau$ runs from $\lambda$ to $\nu$ in the unit sphere~$\mathbb{S}^2$.

Note that $\tau$ can not be separated from $\mu$ by an equator.
Indeed the vector 
\[\alpha(b)-\alpha(a)=\int_a^b\alpha'(t)\cdot dt\]
points in the same direction as $\mu$.
Therefore if the indicatrix $\tau=\tfrac{\alpha'}{|\alpha'|}$ lies in a hemisphere then $\mu$ lies in the same hemisphere. 

Fix an equator $\ell$ in general position.
If $\ell$ intersects the spherical polygonal line $\lambda \mu \nu$ at one point, then $\ell$ separates $\lambda$ from $\nu$ and therefore it must intersect $\tau$.
If $\ell$ intersects the spherical polygonal line $\lambda \mu \nu$ at two points, then $\ell$ separates $\mu$ from $\lambda$ and $\nu$ and therefore it must intersect $\tau$ at least twice --- $\tau$ must cross $\ell$ and then come back.
It follows that for almost all equators the number of intesections with the spherical polygonal line $\lambda \mu\nu$ can not exceed the number of intersections with $\tau$.
By the spherical Crofton formula (\ref{thm:crofton-sphere}), $\tau$ is longer than the spherical polygonal line $\lambda \mu\nu$.
But the polygonal line $\lambda \mu\nu$ has length $\measuredangle (\lambda,\mu)+\measuredangle (\mu,\nu)$, hence the result.
\qeds

Let us sketch an alternative proof of the lemma which is built on Fenchel's theorem. 

\parit{Alternative proof of the lemma.}
Note that the curve $\alpha$ can be extended to a smooth regular closed curve $\hat\alpha$ by an arc $\beta$ that starts from $\alpha(b)$ in the same direction as $\alpha$. Then turns and joins the segment $[\alpha(b),\alpha(a)]$, runs along the segment until it is close to $\alpha(a)$ turns and smoothly joints $\alpha$ at $\alpha(a)$.

Note that the total curvature of $\beta$ can be made arbitrarily close to $2\cdot\pi -\measuredangle (\lambda,\mu)-\measuredangle (\mu,\nu)$.
Indeed, $\beta$ needs a bit more than $\pi -\measuredangle (\mu,\nu)$ to turn an join the segment $[\alpha(b),\alpha(a)]$ and bit more than $\pi -\measuredangle (\lambda,\mu)$ to turn an join the segment $\alpha$.

By Fenchel's theorem,
\[\tc\hat\alpha\ge 2\cdot \pi.\]
Evidently 
\[\tc\hat\alpha=\tc\alpha+\tc\beta,\]
hence the lemma follows.
\qeds


\parit{Proof of \ref{thm:total-curvature=}.}
Let $\alpha\:[a,b]\to\RR^3$ be a smooth curve.
Fix a partition $a\z=t_0<\dots<t_n=b$ and consider the corresponding inscribed polygonal line $\beta=w_0\dots w_n$.
Let $\chi=\xi_1\dots\xi_n$ be its tangent indicatrix --- this is a spherical polygonal line;
we assume that $\chi(t_i)=\xi_i$ and it has constant speed on each arc.

Consider a sequence of finer and finer partitions, denote by $\beta_n$ and $\chi_n$ the corresponding inscribed polygonal line and its tangent indicatrix;
since $\alpha$ is smooth, the $\chi_n$ converge pointwise to $\tau$ --- the thangent indicatrix of $\alpha$.
By semi-continuity of the length functional, we get  
\begin{align*}
\tc\alpha&=\length \tau\le  
\\
&\le \liminf_{n\to\infty}\length \chi_n=
\\
&= \liminf_{n\to\infty}\tc \beta_n\le
\\
&\le \sup\{\tc\beta\},
\end{align*}
where the last supremum is taken over all partitions and their corresponding inscribed polygonal lines $\beta$.

It remains to prove that
\[\tc\alpha\ge \tc\beta,\eqlbl{eq:tc-alpha<tc-beta}\]
for any polygonal line $\beta$ inscribed in $\alpha$.
Let $\zeta_i$ be the unit vector in the direction of $\alpha'(t_i)$.
Consider the spherical polygonal line $\gamma\z=\zeta_0 \xi_1 \zeta_1 \xi_2\dots \xi_n \zeta_n$;
recall that $\chi=\xi_0\dots \xi_n$.
By the triangle inequality, 
\[\length \gamma\ge \length \chi=\tc\beta.\]
By Lemma~\ref{lem:uvw}, 
\[\tc\alpha\ge\length\gamma,\] 
hence \ref{eq:tc-alpha<tc-beta} follows.
\qeds




\section{Crofton again}

Given a curve $\alpha$ in $\RR^3$ and a unit vector $u$, denote by $\alpha_{u^\perp}$ 
and $\alpha_u$ the projection of $\alpha$ to the plane perpendicular to $u$ and the line parallel to $u$ correspondingly.

To prove the following proposition, apply the spherical Crofton formula to the tangent indicatrix of $\alpha$.

\begin{thm}{Proposition}\label{prop:tc-crofton}
Let $\alpha$ be a polygonal line in $\RR^3$.
Show that 
\begin{align*}
\tc\alpha
&=\overline{\tc\alpha_{u^\perp}}=
\\
&=\overline{\tc\alpha_u}.
\end{align*}
\end{thm}

Note that since the curve $\alpha_u$ runs back and forth along one line, 
every time it changes direction contributes $\pi$ to the total curvature of $\alpha_u$.
Therefore the total curvature of $\alpha_u$ is $n\cdot\pi$, where $n$ is the number of changes of direction.
Since $n$ has to be even, $\tc\alpha_u$ may take values $2\cdot\pi$, $4\cdot\pi$, $6\cdot\pi$ and so on.

\begin{thm}{Exercise}
Use the proposition and the observation above to give yet another proof of  Fenchel's theorem (Exercise~\ref{ex:fenchel}).
\end{thm}



\section{DNA inequality}


\begin{thm}{Theorem}\label{thm:DNA}
Let $\alpha$ be a closed curve that lies in a unit disc.
Then 
\[\tc\alpha\ge \length\alpha.\]
\end{thm}

Note that if $\length\alpha\le2\cdot\pi$, then Fenchel's theorem gives a better estimate,
for longer curves this gives something new.

\parit{Proof.}
Assume $\alpha$ is a polygonal line.

Fix a unit vector $u$.
Note that the curve $\alpha_u$ can run at most length 2 in one direction;
therefore the number of turns has to be at least $\tfrac12\cdot\length\alpha$.
Since each turn of $\alpha_u$ contributes $\pi$ to its total curvature, we get
\[\tc\alpha_u\ge \tfrac\pi2\cdot\length\alpha_u.\]

The same inequality holds for the average values of left and right hand sides;
that is,
\[\overline{\tc\alpha_u}
\ge \tfrac\pi2\cdot\overline{\length\alpha_u}.\]
Applying the Crofton's formula and Proposition~\ref{prop:tc-crofton} we get the result.

It remains to reduce the general case to polygonal lines. 
Given $\eps>0$, we choose an inscribed polygonal line $\beta$ such that 
\[\length \alpha<\length \beta+\eps.\]
By the definition of total curvature (\ref{def:total-curv-poly}) and from the first part of the proof
\begin{align*}
\tc\alpha&\ge\tc\beta\ge
\\
&\ge \length\beta>
\\
&>\length\alpha-\eps.
\end{align*}
The statement follows since $\eps$ was arbitrary. 
\qeds

\parit{Alternative proof.}
The same argument as above shows that it is sufficient to consider a closed polygonal line $\beta=p_0p_1\dots p_{n-1}$ in the unit disc.
We assume that $p_n=p_0$, $p_{n+1}=p_1$ and so on.
Denote by $\alpha_i$ the external angle at $p_i$.

Denote by $o$ the center of the disc.
Consider a sequence of triangles 
\[\triangle q_0q_1s_0\cong \triangle p_0p_1o,\ \triangle q_1q_2s_1\cong \triangle p_1p_2o,\dots\]
such that the points $q_0,q_1\dots$ lie on one line in that order and all the $s_i$'s lie on one side from this line.

\begin{figure}[h]
\vskip-0mm
\centering
\includegraphics{mppics/pic-16}
\vskip0mm
\end{figure}

Note that 
\[|s_n-s_0|=\length \beta.\]
Therefore 
\[|s_0-s_1|+\dots+|s_{n-1}-s_n|\ge \length \beta.\]

Note that 
\[|q_i-s_{i-1}|=|q_i-s_i|=|p_i-o|\le 1\]
and
\[\measuredangle s_{i-1}q_is_i\le \alpha_i\]
for each $i$.
Therefore 
\[|s_{i-1}-s_i|<\alpha_i\]
for each $i$.

It follows that
\begin{align*}
\tc \beta
&=\alpha_1+\dots+\alpha_n\ge
\\
&\ge |s_{0}-s_1|+\dots |s_{n-1}-s_n|\ge 
\\
&\ge\length \beta.
\end{align*}
Hence the result.
\qeds

With minor modifications both proofs given above work in the 3-dimensional case (an in higher dimensions).
The following more general result was proved by Jeffrey Lagarias and Thomas Richardson in \cite{lagarias-richardso}, an other proof is given by Alexander Nazarov and Fedor Petrov in \cite{nazarov-petrov}.

\begin{thm}{Theorem}
Let $\alpha$ be a closed curve that lies in a convex plane figure bounded by a curve $\gamma$
Then the average curvature of $\alpha$ is not less than the average curvature of $\gamma$.
Since  $\tc\gamma=2\cdot\pi$, it can be written as
\[\frac{\tc\alpha}{\length\alpha}\ge \frac{2\cdot\pi}{\length\gamma}.\]

\end{thm}



\section{Curves of finite total curvature}

\begin{thm}{Exercise} 
Assume that a curve $\alpha\:[a,b]\to\RR^3$ has finite total curvature.
 Show that $\alpha$ is rectifiable.
\end{thm}

We say that a curve $\alpha\:[a,b]\to\RR^3$ \emph{does not stop} if $\alpha$ is not constant on any subinterval of $[a,b]$. 

\begin{thm}{Exercise} 
Assume that the curve $\alpha$ does not stop and its total curvature is less than $\pi$.
Show that $\alpha$ is simple; that is, it has no self-intersections.
\end{thm}

\begin{thm}{Exercise-definition} 
Assume that a curve $\alpha\:[a,b]\to\RR^3$ does not stop and has finite total curvature.
Show that the direction of exit and entrance is defined for any point.

That is for any $t_0\in [a,b)$ the unit vector  
\[v(\eps)=\frac{\alpha(t_0+\eps)-\alpha(t_0)}{|\alpha(t_0+\eps)-\alpha(t_0)|}\] converges as $\eps\to0^+$;
its limit is called the direction of exit and it will be denoted by $\alpha^+(t_0)$

Analogously, for any $t_0\in (a,b]$ the unit vector  
\[w(\eps)=\frac{\alpha(t_0-\eps)-\alpha(t_0)}{|\alpha(t_0-\eps)-\alpha(t_0)|}\] 
converges as $\eps\to0^+$;
its limit is called the direction of entrance and it will be denoted by $\alpha^-(t_0)$.
\end{thm}

\begin{thm}{Exercise} 
Assume that a curve $\alpha\:[a,b]\to\RR^3$ does not stop and has finite total curvature.
Show that 
\[\alpha^+(t)=-\alpha^-(t)\]
at all $t\in[a,b]$ except possibly on a countable subset.
\end{thm}

\begin{thm}{Exercise} 
Assume a sequence of curves $\alpha_n\:[a,b]\to\RR^3$ converges uniformly to a curve $\alpha_\infty\:[a,b]\to\RR^3$ and
\[ \liminf_{n\to\infty}\length\alpha_n>\length\alpha_\infty.\]
Show that 
\[\tc\alpha_n\to\infty\quad\text{as}\quad n\to\infty.\]

\end{thm}


\section{Total signed curvature}

Let us define the \emph{total signed curvature} of a polygonal line in the plane as the sum of the signed external angles;
the external angle has positive sign if the line turns left and negative sign if the line turns right; the signed external angle is undefined if a pair of adjacent edges overlap;
that is if at one vertex the polygonal line turns in the exact opposite direction.
In particular the total signed curvature is defined for any simple polygonal line in the plane.

\begin{thm}{Exercise}\label{ex:2kpi}
Assume that the total signed curvature of a closed polygonal line in the plane is defined.
Show that it is a multiple of $2\cdot\pi$.
\end{thm}


\begin{thm}{Exercise}\label{ex:pm2pi}
Show that the total signed curvature of any closed simple polygonal line in the plane is $\pm2\cdot\pi$; 
if the bounded region lies on the left from the curve then it is $+2\cdot\pi$ and if it lies on the right then it is $-2\cdot\pi$.
\end{thm}
