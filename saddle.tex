\chapter{Saddle surfaces}

\section*{Definitions}

A surface is called \emph{saddle} if its Gauss curvature at each point is nonpositive;
in other words principle curvatures at each point have opposite signs or one of them is zero.

\begin{wrapfigure}{r}{40 mm}
\vskip-0mm
\centering
\includegraphics{asy/saddle}
\vskip0mm
\end{wrapfigure}

If the Gauss curvature is negative at each point,
then the surface is called \emph{strictly saddle};
equivalently it means that the principle curvatures have opposite signs at each point.
Note that in this case tangent plane is does not support the surface even locally --- moving along the surface in the principle directions at a given point, one gets above and below the tangent plane at this point.  


\begin{thm}{Exercise}\label{ex:convex-revolution}
Let $f\:\RR\to\RR$ be a smooth positive function.
Show that the surface of revolution of the graph $y\z=f(x)$ around the $x$-axis
 is saddle if and only if $f$ is convex; that is if $f''(x)\ge0$ for any $x$.
\end{thm}

\parit{Hint:} Use \ref{ex:principle-revolution}.

A surface $\Sigma$ is called \emph{ruled} if for every point $p\in \Sigma$ there is a line segment $\ell_p\subset \Sigma_p$ thru $p$ that is infinite or has its endpoint(s) on the boundary line of $\Sigma$.

\begin{thm}{Exercise}
Show that any ruled surface $\Sigma$ is saddle.
\end{thm}

\parit{Hint:} Prove and use that each point $p\in\Sigma$ has a direction with vanishing normal curvature.

\begin{thm}{Exercise}
Assume $\Sigma$ is complete saddle surface.
Show that for any point $p\in \Sigma$ there is a curve $\gamma\:[0,\infty)\to\Sigma$ that starts at $p$
such that $t\mapsto|\gamma(t)|$ is an increasing function and $|\gamma(t)|\to\infty$ as $t\to\infty$.
\end{thm}

A tangent direction on a smooth surface with vanishing normal curvature is called \emph{asymptotic}.
A smooth regular curve that always run in an asymptotic direction is called
\index{asymptotic line}\emph{asymptotic line}.\label{page:asymptotic line}

\begin{thm}{Advanced exercise}
Let $\Sigma\subset \RR^3$ be the graph $z\z=f(x,y)$
of a smooth function $f$ 
and $\gamma$ be a closed smooth asymptotic line in $\Sigma$.
Assume $\Sigma$ is strictly saddle in a neighborhood of $\gamma$.
Show that the projection of $\gamma$ to the $(x, y)$-plane cannot be star-shaped.
\end{thm}

\section*{Hats}

Note that a closed surface cannot be saddle.
Indeed consider a smallest sphere that contains a closed surface $\Sigma$ inside;
it supports $\Sigma$ at some point $p$ and at this point the principle curvature must have the same sign.
The following more general statement  is proved using the same idea.

\begin{thm}{Lemma}\label{lem:convex-saddle}
Assume $\Sigma$ is a compact saddle surface and its boundary line lies in a convex closed region $R$.
Then $\Sigma\subset R$.
\end{thm}

\begin{wrapfigure}{o}{50 mm}
\vskip-0mm
\centering
\includegraphics{mppics/pic-73}
\vskip0mm
\end{wrapfigure}

\parit{Proof.}
Assume contrary; that is, there is point $p\in \Sigma$ that does not lie in $R$.
Let $\Pi$ be a plane that separates $p$ from $R$; it exists by \ref{lem:separation}.
Denote by $\Sigma'$ the part of $\Sigma$ that lies with $p$ on the same side from $\Pi$.

Since $\Sigma$ is compact, it is surrounded by a sphere $S$;
let $\sigma$ be the circle of intersection of $S$ and $\Pi$.
Consider the smallest spherical dome $\Sigma_0$ with boundary $\sigma$ that includes $\Sigma'$.

Note that $\Sigma_0$ supports $\Sigma$ at some point~$q$.
Without loss of generality we may assume that $\Sigma_0$ and $\Sigma$ are cooriented at $q$ and $\Sigma_0$ has positive principle curvatures.
In this case $\Sigma_0$ supports $\Delta$ from outside.
By \ref{cor:surf-support}, $K(q)_\Sigma\ge K(q)_{\Sigma_0}>0$ --- a contradiction.
\qeds

Note that if we assume that $\Sigma$ is strictly saddle, then we could arrive to a contradiction by taking a point $q$ on the same side with $p$ and on the maximal distance from $\Pi$.

\begin{thm}{Exercise}\label{ex:length-of-bry}
Let $\Delta$ be a smooth regular saddle disc and $p\in \Delta$.
Assume that the boundary line $\partial \Delta$ lies in the unit sphere centered at~$p$.
Show that $\length (\partial \Delta)\ge 2\cdot\pi$.
\end{thm}

\parit{Hint:} Use the lemma above and the hemisphere lemma (\ref{lem:hemisphere}).

\begin{thm}{Exercise}\label{ex:circular-cone-saddle}
Show that an open saddle surface
can not lie inside of an infinite circular cone. 
\end{thm}

A disc $\Delta$ in a surface $\Sigma$ is called a \emph{hat} of $\Sigma$
if its boundary line $\partial\Delta$ lies in a plane $\Pi$ and the remaining points of $\Delta$ lie on one side of $\Pi$.

\begin{thm}{Proposition}\label{prop:hat}
A smooth surface $\Sigma$ is saddle if and only if it has no hats.
\end{thm}

Note that a saddle surface can contain a closed plane curve.
For example the hyperboloid $x^2+y^2-z^2=1$ contains the unit circle in the $(x,y)$-plane centered at the origin.
However, a plane curve can not bound a disc (as well any compact set) in a saddle surface.

\parit{Proof.}
Since plane is a convex set, the ``only if'' part follows from \ref{lem:convex-saddle};
it remains to prove the ``if'' part.

Assume $\Sigma$ is not saddle; that is, it has a point $p$ with strictly positive Gauss curvature;
or equivalently, the principle curvatures $k_1(p)$ and $k_2(p)$ have the same sign.

Let $z=f(x,y)$ be a graph representation of $\Sigma$ in the tangent-normal coordinates at $p$.
Without loss of generality we may assume that both principle curvatures are positive,
or equivalently the 
\[D^2_wf(0,0)=\II_p(w,w)>0\] 
for any unit tangent vector $w\in\T_p\Sigma$ (which is the $(x,y)$-plane).

Since the set of unit vectors is compact, we have that 
\[D^2_wf(0,0)>\eps\]
for some fixed $\eps>0$ and any unit tangent vector $w\in\T_p\Sigma$.
By continuity of the function $(x,y,w)\mapsto D^2_wf(x,y)$,
we have that $D^2_wf(x,y)>0$ for $(x,y)$ in a neighborhood of the origin.
That is, $f$ is a strictly convex function in a neighborhood of the origin in the $(x,y)$-plane.
In particular the set 
\[\Delta_\eps=\set{(x,y,f(x,y))\in \RR^3}{f(x,y))\le \eps}\]
is a disc for sufficiently small $\eps>0$ (see \ref{ex:disc-hat}).
Note that its boundary line lies on the plane $z=\eps$ and whole disc lies below it;
that is, $\Delta_\eps$ is a hat.
\qeds

Note that we proved the following lemma; it will be useful later. %??? move in curvature

\begin{thm}{Lemma}\label{lem:gauss+=>convexity}
Let $z=f(x,y)$ be the local description of a smooth surface $\Sigma$ in a tangent-normal coordinates at some point $p\in\Sigma$.
Assume both principle curvatures of $\Sigma$ are positive at $p$.
Then the function $f$ is strictly convex in a neighborhood of the origin.
\end{thm}

In the proof above we assumed that the statement in following exercise is evident.

\begin{thm}{Exercise}\label{ex:disc-hat}
Let $\Delta_\eps$ be as in the proof above.
Show that $\Delta_\eps$ is a smooth disc; that is, $\Delta_\eps$ is the image of regular embedding $\DD\to \RR^3$, where $\DD=\set{(x,y)\in\RR^2}{x^2+y^2\le 1}$.
\end{thm}

\parit{Hint:} Observe that it is sufficient to construct a smooth parametrization of $\Delta_\eps$ by a closed hemisphere.
To do this repeat the argument in \ref{lem:gauss=sphere} with the center at a point surrounded by the boundary line of $\Delta_\eps$ in its plane.

\begin{thm}{Exercise}\label{ex:saddle-linear}
Let $T\:\RR^3\to\RR^3$ be a linear transformation; that is, $T(x,y,z)=(x,y,z)\cdot A$ for a invertible $3{\times}3$-matrix $A$. 
Show that for any saddle surface $\Sigma$ the image $T(\Sigma)$ is also a saddle surface.
\end{thm}

\section*{Saddle graphs}

The following theorem was proved by Sergei Bernstein \cite{bernstein}.

\begin{thm}{Theorem}\label{thm:bernshtein}
Let $f\:\RR^2\to\RR$ be a smooth function.
Assume its graph $z=f(x,y)$ is a strictly saddle surface in $\RR^3$.
Then $f$ is not bounded;
that is, there is no constant $C$ such that 
$|f(x,y)|\le C$ for any $(x,y)\in\RR^2$.
\end{thm}

Before going into the proof let us discuss some examples.

Note that the theorem states that a saddle graph can not lie between parallel horizontal planes;
applying \ref{ex:saddle-linear} we get that saddle graph can not lie between parallel planes,
not necessarily horizontal.
The following exercise shows that the theorem does not hold for saddle surface which are not graphs.


\begin{thm}{Exercise}\label{ex:between-parallels}
Construct a complete strictly saddle surface that lies between parallel planes.
\end{thm}

\parit{Hint:} Look for an example among the surfaces of revolution and use \ref{ex:principle-revolution}.


The following exercise shows that there are saddle graphs with functions bounded on one side; that is, both (upper and lower) bounds are needed in the proof of Bernshtein's theorem.

\begin{thm}{Exercise}\label{ex:one-side-bernshtein}
Show that there are positive functions with strictly saddle graphs.
In fact the graph
$z=\exp(x-y^2)$
is strictly saddle.
\end{thm}

\parit{Hint:} Look at two section of the graph by planes parallel to $(x,y)$-plane and to $(x,z)$-plane and apply Meusnier's theorem (\ref{cor:meusnier}).

Note that according to \ref{lem:convex-saddle}, there are no complete saddle surfaces in a parallelepiped that boundary line lies on one of its faces.
The following lemma gives an analogous statement for a parallelepiped with an infinite side.

\begin{thm}{Lemma}\label{lem:region}
There is no complete strictly saddle smooth surface 
with the boundary line in a plane
that lies on bounded distance from a line.
\end{thm}


\parit{Proof.}
Note that in a suitable coordinate system, the statement can be reformulated the following way:
\emph{There is no complete strictly saddle smooth surface 
with the boundary line in the $(x,y)$-plane
that lies in a region of the following form:}
\[R=\set{(x,y,z)\in\RR^3}{0\le z\le r, 0\le y\le r}.\]
Further we will prove this statement.

Assume contrary, let $\Sigma$ be such a surface.
Consider the projection $\hat \Sigma$ of $\Sigma$ to the $(x,z)$-plane.
It lies in the upper half-plane and below the line $z=r$.

Consider the open upper half-plane $H=\set{(x,z)\in \RR^2}{z> 0}$. 
Let $\Theta$ be the connected component of the complement $H\backslash \hat \Sigma$ that contains all the points above the line $z=r$.

\begin{figure}[h!]
\vskip-0mm
\centering
\includegraphics{mppics/pic-74}
\vskip0mm
\end{figure}

Note that $\Theta$ is convex.
If not then a line segment $[pq]$ for some $p,q\in \Theta$ cuts from $\hat\Sigma$ a compact piece.
Consider the plane $\Pi$ thru $[pq]$ that is perpendicular to the $(x,z)$-plane.
Note that $\pi$ cuts from $\Sigma$ a compact region $\Delta$.
By general position argument \ref{lem:reg-section}, 
we can assume that $\Delta$ is a compact surface with boundary line in $\Pi$ and the remaining part of $\Delta$ lies on one side from $\Pi$.
Since the plane $\Pi$ is a convex, this statement contradicts \ref{lem:convex-saddle}.

Summarizing, $\Theta$ is an open convex set of $H$ that contains all points above $z=r$.
By convexity, together with any point $w$, the set $\Theta$ contains all points on the half-lines that point up from it. 
Whence it contain all points with $z$-coordinate larger than the $z$-coordinate of $w$.
\begin{figure}[h!]
\vskip-0mm
\centering
\includegraphics{mppics/pic-75}
\vskip0mm
\end{figure}
Since $\Theta$ is open it can be described by inequality $z>r_0$.
It follows that the plane $z=r_0$ supports $\Sigma$ at some point (in fact at many points).
By \ref{prop:surf-support}, the latter is impossible --- a contradiction.
\qeds

\parit{Proof of \ref{thm:bernshtein}.}
Denote by $\Sigma$ the graph $z=f(x,y)$.
Assume contrary; that is, $\Sigma$ lies between two planes $z=\pm C$.

Note $f$ can not be constant.
It follows that the tangent plane $\T_p$ at some point $p\in\Sigma$ is not horizontal.

Denote by $\Sigma^+$ the part of $\Sigma$ that lies above $\T_p$.
Note that it has at least two connected components which are approaching $p$ from both sides 
in the principle direction with positive principle curvature.
Indeed if there would be a curve that runs in $\Sigma^+$ and approaches $p$ from both sides then it would cut a disc from $\Sigma$ with boundary line above $\T_p$ and some points below it;
the latter contradicts \ref{lem:convex-saddle}.

\begin{figure}[h!]
\vskip-0mm
\centering
\includegraphics{mppics/pic-76}
\caption*{The surface $\Sigma$ seeing from above.}
\vskip0mm
\end{figure}

Summarizing, $\Sigma^+$ has at least two connected components, denote them by $\Sigma^+_0$ and $\Sigma^+_1$.
Let $z=h(x,y)=a\cdot x+b\cdot y+c$ be the equation of $\T_p$.
Note that $\Sigma^+$ contains all points in the region
\[R_-=\set{(x,y,f(x,y))\in\Sigma}{h(x,y)< C}\] 
which is a connected set and no points in 
\[R_+=\set{(x,y,f(x,y))\in\Sigma}{h(x,y)> C}\]
Whence one of the connected components, say $\Sigma^+_0$, lies in 
\[R_0=\set{(x,y,f(x,y))\in\Sigma}{|h(x,y)|\le  C}.\]
This set lies on a bounded distance from the line of intersection of $\T_p$ with the $(x,y)$-plane.

Moving the plane $\T_p$ little up, we can cut from $\Sigma^+_0$ is a complete surface with boundary line lying in this plane (see \ref{lem:reg-section}).
The obtained surface is still on a bounded distance to a line
which is impossible by \ref{lem:region}.
\qeds

The following exercise gives a condition that guarantees that a saddle surface is a graph;
it can be used in combination with Bernshtein's theorem.

\begin{thm}{Advanced exercise}
Let $\Sigma$ be a smooth saddle disk in $\RR^3$.
Assume that the orthogonal projection to the $(x,y)$-plane
maps the boundary line of $\Sigma$
injectively to a convex closed curve.
Show that the orthogonal projection to $(x,y)$-plane is injective on $\Sigma$.

In particular, $\Sigma$ is the graph $z=f(x,y)$ of a function $f$ defined on a convex figure in the $(x,y)$-plane.
\end{thm}


\section*{Remarks}

Note that Bernstein's theorem and the lemma in its proof do not hold for saddle surfaces;
counterexamples can be found among infinite cylinders over a smooth regular curves.
In fact it can be shown that these are only counterexamples;
the proof is based on the same idea, but more technical.

By \ref{prop:hat}, saddle surfaces can be defined as smooth surfaces without hats.
This definition can be used for arbitrary surfaces not necessarily smooth.
Some results, for example Bernshtein's characterization of saddle graphs can be extended to generalized saddle surface, but this class of surfaces is far from being understood.
Some nontrivial properties were proved by in Samuil Shefel \cite{shefel} see also \cite[Capter 4]{akp}.
