\chapter{Saddle surfaces}

\section*{Definitions}

A surface is called \emph{saddle} if its Gauss curvature at each point is nonpositive;
in other words principle curvatures at each point have opposite signs or one of them is zero.

If the Gauss curvature is negative at each point,
then the surface is called \emph{strictly saddle}.


\begin{thm}{Exercise}\label{ex:convex-revolution}
Let $f\:\RR\to\RR$ be a smooth positive function.
Show that the surface of revolution of the graph $y\z=f(x)$ around the $x$-axis
 is saddle if and only if $f$ is convex; that is if $f''(x)\ge0$ for any $x$.
\end{thm}

\parit{Hint:} Use \ref{ex:principle-revolution}.

A surface $\Sigma$ is called \emph{ruled} if for every point $p\in \Sigma$ there is a line segment $\ell_p\subset \Sigma_p$ thru $p$ that is infinite or has its endpoint(s) on the boundary line of $\Sigma$.

\begin{thm}{Exercise}
Show that any ruled surface $\Sigma$ is saddle.
\end{thm}

\parit{Hint:} Prove and use that each point $p\in\Sigma$ has a direction with vanishing normal curvature.

A tangent direction on a smooth surface with vanishing normal curvature is called \emph{asymptotic}.
A smooth regular curve that always run in an asymptotic direction is called
\index{asymptotic line}\emph{asymptotic line}.

\begin{thm}{Advanced exercise}
Let $\Sigma\subset \RR^3$ be the graph $z\z=f(x,y)$
of a smooth function $f$ 
and $\gamma$ be a closed smooth asymptotic line in $\Sigma$.
Assume $\Sigma$ is strictly saddle in a neighborhood of $\gamma$.
Show that the projection of $\gamma$ to the $(x, y)$-plane cannot be star-shaped.
\end{thm}

\section*{Hats}

Note that a closed surface can not be saddle.
Indeed consider a smallest sphere that contains a closed surface $\Sigma$ inside;
it supports $\Sigma$ at some point $p$ and at this point the principle curvature must have the same sign.
The following more general statement  is proved using the same idea.

\begin{thm}{Lemma}\label{lem:convex-saddle}
Assume $\Sigma$ is a compact saddle surface and its boundary line lies in a convex closed set $F$.
Then $\Sigma\subset F$.
\end{thm}

\begin{wrapfigure}{o}{50 mm}
\vskip-0mm
\centering
\includegraphics{mppics/pic-73}
\vskip0mm
\end{wrapfigure}

\parit{Proof.}
Assume contrary; that is, there is point $p\in \Sigma$ that does not lie in $F$.
By \ref{lem:separation}, there is a plane $\Pi$ that separates $p$ from $F$.
Denote by $\Sigma'$ the part of $\Sigma$ that lies with $p$ on the same side from $\Pi$.

Since $\Sigma$ is compact, it is surrounded by a sphere $S$ centered at $\Pi$;
let $\sigma$ be the circle of intersection of $S$ and $\Pi$.
Consider the smallest spherical dome $\Sigma_0$ with boundary $\sigma$ that includes $\Sigma'$.

Note that $\Sigma_0$ supports $\Sigma$ at some point~$q$.
Without loss of generality we may assume that $\Sigma_0$ and $\Sigma$ are cooriented at $q$ and $\Sigma_0$ has positive principle curvatures.
In this case $\Sigma_0$ supports $\Delta$ from outside.
By \ref{cor:surf-support}, $K(q)_\Sigma\ge K(q)_{\Sigma_0}$.
The statement follows since $K(q)_{\Sigma_0}$ is positive --- a contradiction.
\qeds

Note that if we assume that $\Sigma$ is strictly saddle, then we could arrive to a contradiction by taking a point $q$ on the same side with $p$ and on the maximal distance from $\Pi$.

A disc $\Delta$ in a surface $\Sigma$ is called a \emph{hat} of $\Sigma$
if its boundary line $\partial\Delta$ lies in a plane $\Pi$ and the remaining points of $\Delta$ lie on one side of $\Pi$.

\begin{thm}{Proposition}\label{prop:hat}
A smooth surface $\Sigma$ is saddle if and only if it has no hats.
\end{thm}

\parit{Proof.}
Since plane is a convex set, the ``only if'' part follows from \ref{lem:convex-saddle};
it remains to prove the ``if'' part.

Assume $\Sigma$ is not saddle; that is, it has a point $p$ with strictly positive Gauss curvature;
or equivalently, the principle curvatures $k_1(p)$ and $k_2(p)$ have the same sign.

Let $z=f(x,y)$ be a graph representation of $\Sigma$ in the tangent-normal coordinates at $p$.
Without loss of generality we may assume that both principle curvatures are positive,
or equivalently the 
\[D^2_wf(0,0)=\II_p(w,w)>0\] 
for any nonzero tangent vector $w\in\T_p\Sigma$ (which is the $(x,y)$-plane).

It follows that $D^2_wf(x,y)>0$ for $(x,y)$ in a neighborhood of the origin.
Whence $f$ is a strictly convex function in a neighborhood of the origin in the $(x,y)$-plane.
In particular the set 
\[\Delta_\eps=\set{(x,y,f(x,y))\in \RR^3}{f(x,y))\le \eps}\]
is a disc for sufficiently small $\eps>0$. %???
Note that its boundary line lies on the plane $z=\eps$ and whole disc lies below it;
that is, $\Delta_\eps$ is a hat.
\qeds

\begin{thm}{Exercise}
Let $\Delta_\eps$ be as in the proof above.
Show that $\Delta_\eps$ is a smooth disc; that is, $\Delta_\eps$ is the image of regular embedding $\DD\to \RR^3$, where $\DD=\set{(x,y)\in\RR^2}{x^2+y^2\le 1}$.
\end{thm}


\begin{thm}{Exercise}
Some saddle surfaces contain a closed plane curves;
for example the hyperboloid $x^2+y^2-z^2=1$ contains the unit circle in the $(x,y)$-plane centered at the origin.

Why does not it contradict \ref{prop:hat}?
\end{thm}

\begin{thm}{Exercise}\label{ex:saddle-linear}
Let $T\:\RR^3\to\RR^3$ be a linear transformation; that is, $T(x,y,z)=(x,y,z)\cdot A$ for a invertible $3{\times}3$-matrix $A$. 
Show that for any saddle surface $\Sigma$ the image $T(\Sigma)$ is also a saddle surface.
\end{thm}

\section*{Saddle graphs}

The following theorem was proved by Sergei Bernstein \cite{bernstein}.

\begin{thm}{Bernshtein's theorem}\label{thm:bernshtein}
Let $f\:\RR^2\to\RR$ be a smooth function.
Assume its graph $z=f(x,y)$ is a strictly saddle surface $\RR^3$.
Then $f$ is not bounded;
that, is there is no constant $C$ such that 
$|f(x,y)|\le C$ for any $(x,y)\in\RR^2$.
\end{thm}

\parit{The proof of \ref{thm:bernshtein} is coming.}

Note that the theorem states that a saddle graph can not lie between parallel horizontal planes;
applying \ref{ex:saddle-linear} we get that saddle graph can not lie between parallel planes,
not necessary horizontal.
The following exercise shows that the theorem does not hold for saddle surface which are not graphs.


\begin{thm}{Exercise}\label{ex:between-parallels}
Construct a complete strictly saddle surface that lies between parallel planes.
\end{thm}

\parit{Hint:} Look for an example among the surfaces of revolution and use \ref{ex:principle-revolution}.


The following exercise shows that there are saddle graphs with with functions bounded on one side; that is, both --- upper and lower bounds are needed in the proof of Bernshtein's theorem.

\begin{thm}{Exercise}\label{ex:one-side-bernshtein}
Show that there are positive functions with strictly saddle graphs.
In fact the graph
$z=\exp(x-y^2)$
is strictly saddle.
\end{thm}

\parit{Hint:} Look at two section of the graph by planes parallel to $(x,y)$-plane and to $(x,z)$-plane and apply Meusnier's theorem (\ref{cor:meusnier}).


