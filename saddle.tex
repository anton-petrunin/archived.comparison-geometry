\chapter{Saddle surfaces}

A surface is called \emph{saddle} if its Gauss curvature at each point is nonpositive;
in other words principle curvatures at each point have opposite signs or one of them is zero.

Note that a closed surface can not be saddle.
Indeed consider a smallest sphere that contains a closed surface $\Sigma$ inside;
it supports $\Sigma$ at some point $p$ and at this point the principle curvature must have the same sign.
The following exercise can be solved using the same idea.

\begin{thm}{Exercise}
Show that a smooth surface $\Sigma$ is saddle if and only if it has no hats;
that is no disc $\Delta$ in $\Sigma$ that boundary lies in a plane and the remaining points of the $\Delta$ lie on one side of the plane.  
\end{thm}

A surface $\Sigma$ is called \emph{ruled} if thru every point of $\Sigma$ there is a straight line that lies on $\Sigma$.

\begin{thm}{Exercise}
Show that any ruled surface is saddle.
\end{thm}

