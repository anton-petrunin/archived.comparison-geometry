\chapter{Semigeodesic charts}\label{chap:semigeodesic}

This chapter contains formal computational proofs of several statements discussed above.

\section*{Polar coordinates}

The property of exponential map in \ref{prop:exp} can be used to define \index{polar coordinates}\emph{polar coordinates} in a smooth surface $\Sigma$ with respect to a point $p\in \Sigma$.

Namely, fix polar coordinates $(\theta,r)$ on tangent plane $\T_p$.
If $\vec v\in \T_p$ has coordinates $(\theta,r)$,
then we say that $q=\exp_p\vec v$ is the point in $\Sigma$ with the polar coordinates $(\theta,r)$, or briefly $q\z=w_p(r,\theta)$.
Note that according to Proposition~\ref{prop:exp} polar coordinates behave usual way in a neighborhood of $p$;
that is, there is $r_0>0$ such that if $r_1,r_2<r_0$ then $w_p(r_1,\theta_1) = w_p(r_2,\theta_2)$ if and only if
$r_1=r_2=0$ or $r_1=r_2$ and $\theta_1=\theta_2+2\cdot n\cdot\pi$ for an integer $n$.

The following statement is known as \index{Gauss lemma}\emph{Gauss lemma},
it will play a key role in the formal proof that shortest paths are geodesics; see \ref{prop:gamma''}.

\begin{thm}{Lemma}\label{lem:palar-perp}
Let $w_p(\theta,r)$ describes polar coordinates in a neingborhood of point $p$ of smooth surface $\Sigma$.
Then
\[\tfrac{\partial w_p}{\partial\theta}\perp\tfrac{\partial w_p}{\partial r}\]
for any $r$ and $\theta$.
\end{thm}


\parit{Proof.}
Note that by the definition of exponential map, for a fixed $\theta$, the curve $\gamma_\theta(t)=w(\theta,t)$ is a unit-speed geodesic that starts at $p$;
in particular we have the following two identities:
\begin{enumerate}[(i)]
\item Since the geodesic has unit speed we have $|\tfrac{\partial}{\partial r}w_p|=|\gamma_\theta'(r)|=1$.
In particular,
 \[\tfrac{\partial}{\partial \theta}\langle\tfrac{\partial w_p}{\partial r},\tfrac{\partial w_p}{\partial r}\rangle=0\]
\item Since $\gamma_\theta$ is a geodesic, we have $\tfrac{\partial^2w_p}{\partial r^2}=\gamma''_\theta(r)\perp\T_{\gamma_\theta(r)}$ and therefore 
\[\langle\tfrac{\partial^2w_p}{\partial r^2},\tfrac{\partial w_p}{\partial \theta}\rangle=0\]
\end{enumerate}
It follows that
\begin{align*}
\tfrac{\partial}{\partial r}
\langle\tfrac{\partial w_p}{\partial\theta},\tfrac{\partial w_p}{\partial r}\rangle
&=\langle\tfrac{\partial^2 w_p}{\partial\theta\partial r},\tfrac{\partial w_p}{\partial r}\rangle
+
\langle\tfrac{\partial w_p}{\partial\theta},\tfrac{\partial^2 w_p}{\partial r^2}\rangle=
\\
&=\tfrac12\cdot \tfrac{\partial}{\partial \theta}\langle\tfrac{\partial w_p}{\partial r},\tfrac{\partial w_p}{\partial r}\rangle=
\\
&=0;
\end{align*}
that is, for fixed $\theta$, the value $\langle\tfrac{\partial w_p}{\partial\theta},\tfrac{\partial w_p}{\partial r}\rangle$ does not depend on $r$.

Note that $w_p(\theta,0)=p$ for any $\theta$.
Therefore
$\tfrac{\partial w_p}{\partial\theta}=0$
and in particular 
\[\langle\tfrac{\partial w_p}{\partial\theta},\tfrac{\partial w_p}{\partial r}\rangle=0\]
if $r=0$ and therefore it holds for any $r$.
\qeds

\section*{Shortest paths are geodesics: a formal proof}

Let us use the construction of polar coordinates and the Gauss lemma \ref{lem:palar-perp} to prove Proposition~\ref{prop:gamma''}.

\parit{Proof of \ref{prop:gamma''}.}\label{page:proof-of-gamma''}
Let $\gamma\:[0,\ell]\to\Sigma$ be a shortest path parameterized by arc-length.
Suppose $\ell=\length\gamma$ is sufficiently small, so $\gamma$ can be described in the polar coordinates at $p$;
say $\gamma(t)=w_p(\theta(t),r(t))$ for some functions $t\mapsto \theta(t)$ and $t\mapsto r(t)$ such that $r(0)=0$. %???problem

Note that by chain rule, we have
\[\gamma'(t)=\tfrac{\partial w_p}{\partial \theta}\cdot \theta'(t)+\tfrac{\partial w_p}{\partial r}\cdot r'(t)
\eqlbl{eq:chain(gamma)}\]
if the left side is defined.
By Gauss lemma \ref{lem:palar-perp}, $\tfrac{\partial w_p}{\partial \theta}\perp\tfrac{\partial w_p}{\partial r}$ and by definition of polar coordinates $|\tfrac{\partial w_p}{\partial r}|=1$.
Therefore \ref{eq:chain(gamma)} implies
\[|\gamma'(t)|\ge r'(t).\eqlbl{eq:|gamma'|=r'}\]
for any $t$ where $\gamma'(t)$ is defined.

Since $\gamma$ parameterized by arc-length, we have 
\[|\gamma(t_2)-\gamma(t_1)|\le |t_2-t_1|.\]
In particular, $\gamma$ is Lipschitz.
Therefore by Rademacher's theorem (\ref{thm:rademacher}) the derivative $\gamma'$ is defined almost everywhere.
By~\ref{adex:integral-length:a}, we have that
\begin{align*}
\length\gamma&=\int_0^\ell|\gamma'(t)|\cdot dt\ge
\\
&\ge\int_0^\ell r'(t)\cdot dt=
\\
&=r(\ell).
\end{align*}

Note that by the definition of polar coordinates, there is a geodesic of length $r(\ell)$ from $p=\gamma(0)$ to $q=\gamma(\ell)$.
Since $\gamma$ is a shortest path, we have therefore $r(\ell)=\ell$ and moreover $r(t)=t$ for any $t$.
This equality holds if and only if we have equality in \ref{eq:|gamma'|=r'} for almost all $t$.
The latter implies that $\gamma$ is a geodesic.

It remains to prove the partial converse.

Fix a point $p\in\Sigma$.
Let $\eps>0$ be as in \ref{prop:exp}.
Assume a geodesic $\gamma$ of length less than $\eps$ from $p$ to $q$ does not minimize the length between its endpoints.
Then there is a shortest path from $p$ to $q$, which becomes a geodesic if parameterized by its arc-length.
That is, there are two geodesics from $p$ to $q$ of length smaller than $\eps$.
In other words there are two vectors ${\vec v},\vec w\in\T_p$ such that $|{\vec v}|<\eps$, $|\vec w|<\eps$ and 
$q=\exp_p\vec v=\exp_p\vec w$.
But according to \ref{prop:exp}, the exponential map is injective in $\eps$-neighborhood of zero --- a contradiction.\qeds

\section*{Gauss curvature}

A chart $(u,v)\mapsto s(u,v)$ of smooth surface $\Sigma$ is called \index{semigeodesic chart}\emph{semigeodesic} if $|\tfrac{\partial s}{\partial u}|=1$ and $\tfrac{\partial s}{\partial u}\perp\tfrac{\partial s}{\partial v}$ for any $(u,v)$.

A chart $(u,v)\mapsto s(u,v)$ is called \emph{orthogonal} if $\tfrac{\partial s}{\partial u}\perp\tfrac{\partial s}{\partial v}$ for any $(u,v)$.
Note that any semigeodesic chart is orthogonal.

\begin{thm}{Proposition}\label{prop:K(semigeodesic)}
Any point $p$ on a smooth surface $\Sigma$ can be covered by a semigeodesic chart $(u,v)\mapsto s(u,v)$.

Moreover, the Gauss curvature $K=K(u,v)$ can be expressed as
\[K=-\tfrac1b\cdot\tfrac{\partial^2 b}{\partial u^2},\]
where $b=b(u,v)\df|\tfrac{\partial s}{\partial v}|$.
\end{thm}

\parit{Proof.}
By Gauss lemma (\ref{lem:palar-perp}) polar coordinates is an example of a semigeodesic chart.
It remains to find a point $q\ne p$ such that polar coordinates on $\Sigma$ with center at $q$ cover $p$;
any sufficiently close point will do the trick. %??? needs bit more

The last statement follows from \ref{lem:K(orthogonal)}.
Indeed, since the chart is semigeodesic it is orthogonal  and 
$a=a(u,v)\df|\tfrac{\partial s}{\partial u}|\equiv 1$.
\qeds

\begin{thm}{Lemma}\label{lem:K(orthogonal)}
Let $(u,v)\mapsto s(u,v)$ be a orthogonal chart of smooth surface $\Sigma$.
Then 
\[K=-\tfrac1{a\cdot b}\cdot\left(
\tfrac{\partial }{\partial u}
(\tfrac1a\cdot\tfrac{\partial }{\partial u}b)
+
\tfrac{\partial }{\partial v}
(\tfrac1b\cdot\tfrac{\partial }{\partial v}a)
\right)\eqlbl{eq:K(orthonormal)}\]
where 
$a=a(u,v)\df|\tfrac{\partial s}{\partial u}|$,
$b=b(u,v)\df|\tfrac{\partial s}{\partial v}|$,
and
$K=K(u,v)$ denotes the Gauss curvature of $\Sigma$ at $s(u,v)$.
\end{thm}

\parit{Proof.}
Set $\vec u(u,v)\df\tfrac1a\cdot\tfrac{\partial s}{\partial u}$
and 
$\vec v(u,v)\df\tfrac1b\cdot \tfrac{\partial s}{\partial v}$.
Recall that $\Norm(u,v)$ denotes the unit normal vector to $\Sigma$ at $s(u,v)$.

Observe that $\vec u, \vec v,\Norm$ form an orthonormal frame for any $(u,v)$.
Without loss of generality we may assume that the frame $\vec u, \vec v,\Norm$ is oriented; that is, $\Norm=\vec u\times \vec v$ for any $(u,v)$.

Suppose that $\ell=\ell(u,v)$, $m=m(u,v)$, and $n=n(u,v)$ be the components of the matrix describing the shape operator in the frame $\vec u, \vec v$.
Since $\vec u, \vec v$ is an orthonormal frame, by \ref{ex:shape-chart} we have
\[
\begin{aligned}
\tfrac1{a^2}
\cdot
\langle \tfrac{\partial^2s}{\partial u^2},\Norm\rangle
=
\langle S(\vec u),\vec u\rangle
&=\ell,
&
\tfrac1{a\cdot b}
\cdot
\langle\tfrac{\partial^2s}{\partial u\partial v},\Norm\rangle
=
\langle S(\vec u),\vec v\rangle&=m,
\\
\tfrac1{a\cdot b}
\cdot
\langle\tfrac{\partial^2s}{\partial v\partial u},\Norm\rangle
=
\langle S(\vec v),\vec u\rangle&=m,
&
\tfrac1{b^2}
\cdot
\langle \tfrac{\partial^2s}{\partial v^2},\Norm\rangle
=
\langle S(\vec v),\vec v\rangle&=n.
\end{aligned}
\eqlbl{eq:shape-lmn}
\]

First let us show that the statement follows from the following identities:
\[
\begin{aligned}
\tfrac{\partial}{\partial u}\vec u
&=
-\tfrac1{b}\cdot \tfrac{\partial a}{\partial v}
\cdot
\vec v 
+
a\cdot \ell\cdot \Norm
,
&
\tfrac{\partial}{\partial u}\vec v
&=
\tfrac1{b}\cdot \tfrac{\partial a}{\partial v}
\cdot \vec u
+
a\cdot m\cdot \Norm
\\
\tfrac{\partial}{\partial v}\vec u
&=
-\tfrac1{a}\cdot \tfrac{\partial b}{\partial u}\cdot\vec v
+
b\cdot m\cdot \Norm
,
&
\tfrac{\partial}{\partial v}\vec v
&=\tfrac1{a}\cdot \tfrac{\partial b}{\partial u}\cdot\vec u
+
b\cdot n\cdot \Norm
\end{aligned}
\eqlbl{eq:key-orthogonal}
\]

Indeed, recall that the Gauss curvature equals to the determinant of the matrix $
(\begin{smallmatrix}
\ell&m\\n&n
\end{smallmatrix}
)
$;
that is, $K=\ell\cdot n-m^2$.
Therefore 
\begin{align*}
\langle\tfrac{\partial}{\partial u}\vec u,\tfrac{\partial}{\partial v}\vec v\rangle 
-
\langle\tfrac{\partial}{\partial v}\vec u,\tfrac{\partial}{\partial u}\vec v\rangle
&=a\cdot b\cdot(\ell\cdot n-m^2)=
\\
&=a\cdot b\cdot K.
\intertext{On the other hand}
\langle\tfrac{\partial}{\partial u}\vec u,\tfrac{\partial}{\partial v}\vec v\rangle 
-
\langle\tfrac{\partial}{\partial v}\vec u,\tfrac{\partial}{\partial u}\vec v\rangle
&= 
\left(\tfrac{\partial}{\partial v}\langle\tfrac{\partial}{\partial u}\vec u,\vec v\rangle -\langle\tfrac{\partial^2}{\partial u\partial v}\vec u,\vec v\rangle\right)-
\\
&-
\left(\tfrac{\partial}{\partial u}\langle\tfrac{\partial}{\partial v}\vec u,\vec v\rangle-\langle\tfrac{\partial^2}{\partial u\partial v}\vec u,\vec v\rangle\right)=
\\
&=\tfrac{\partial}{\partial v}(-\tfrac1{b}\cdot \tfrac{\partial a}{\partial v})
-
\tfrac{\partial}{\partial u}(\tfrac1{a}\cdot \tfrac{\partial b}{\partial u})
\end{align*}
The identity \ref{eq:K(orthonormal)} follows since the left hand sides in the last two equations are identical.

It remains to prove \ref{eq:key-orthogonal}.
Since the frame $\vec u, \vec v,\Norm$ is orthonormal,
first two vector identities are equivalent to the following six real identities:
\[
\begin{aligned}
\langle\tfrac{\partial}{\partial u}\vec u,\vec u\rangle
&=0,
&
\langle\tfrac{\partial}{\partial u}\vec v,\vec u\rangle
&=
\tfrac1{b}\cdot \tfrac{\partial a}{\partial v},
\\
\langle\tfrac{\partial}{\partial u}\vec u,\vec v\rangle
&=-\tfrac1{b}\cdot \tfrac{\partial a}{\partial v},
&
\langle\tfrac{\partial}{\partial u}\vec v,\vec v\rangle
&=0,
\\
\langle\tfrac{\partial}{\partial u}\vec u,\Norm\rangle
&=a\cdot \ell,
&
\langle\tfrac{\partial}{\partial u}\vec v,\Norm\rangle
&=
a\cdot m.
\end{aligned}
\eqlbl{eq:key-orthogonal/2}
\]

By taking partial derivatives of the identities
$\langle\vec u,\vec u\rangle=1$,
$\langle\vec u,\vec v\rangle=0$, and
$\langle\vec v,\vec v\rangle=1$,
we get 
\begin{align*}
\langle\tfrac{\partial}{\partial u}\vec u,\vec u\rangle
&=0,
&
\langle\tfrac{\partial}{\partial u}\vec v,\vec v\rangle
&=0,
&
\langle\tfrac{\partial}{\partial u}\vec v,\vec u\rangle
&=
-
\langle\vec v,\tfrac{\partial}{\partial u}\vec u\rangle.
\end{align*}


Further, observe that
\[
\begin{aligned}
\tfrac{\partial}{\partial u}\vec v
&=
\tfrac{\partial}{\partial v}
(\tfrac1b\cdot \tfrac{\partial s}{\partial v})=
\\
&=\tfrac1b\cdot\tfrac{\partial^2s}{\partial u\partial v}
+
\tfrac{\partial}{\partial u}(\tfrac1b)
\cdot
\tfrac{\partial s}{\partial v}.
\end{aligned}
\eqlbl{eq:dv/du}
\]
Therefore 
\begin{align*}
\langle\tfrac{\partial}{\partial u}\vec v,\vec u\rangle
&=
\tfrac1{a\cdot b}\cdot \langle\tfrac{\partial^2s}{\partial v\partial u},\tfrac{\partial s}{\partial u}\rangle
=
\\
&=\tfrac1{2\cdot a\cdot b}\cdot \tfrac{\partial}{\partial v}\langle\tfrac{\partial s}{\partial u},\tfrac{\partial s}{\partial u}\rangle=
\\
&=\tfrac1{2\cdot a\cdot b}\cdot \tfrac{\partial a^2}{\partial v}=
\\
&=\tfrac1{b}\cdot \tfrac{\partial a}{\partial v}
\intertext{and}
\langle\vec v,\tfrac{\partial}{\partial u}\vec u\rangle
&=-\tfrac1{b}\cdot \tfrac{\partial a}{\partial v}.
\end{align*}
Since
$0=\tfrac{\partial}{\partial u}\langle\vec v,\vec u\rangle=
\langle\tfrac{\partial}{\partial u}\vec v,\vec u\rangle
+
\langle\vec v,\tfrac{\partial}{\partial u}\vec u\rangle$,
we get
\begin{align*}
\langle\tfrac{\partial}{\partial u}\vec v,\vec u\rangle
=
-
\langle\vec v,\tfrac{\partial}{\partial u}\vec u\rangle
&=\tfrac1{b}\cdot \tfrac{\partial a}{\partial v}.
\end{align*}

Applying \ref{eq:shape-lmn} and \ref{eq:dv/du}, we get
\begin{align*}
\langle\tfrac{\partial}{\partial u}\vec u,\Norm\rangle
&=
\tfrac1{a}\cdot \langle\tfrac{\partial^2s}{\partial u^2},\Norm\rangle
=a\cdot \ell,
\\
\langle\tfrac{\partial}{\partial u}\vec v,\Norm\rangle
&=
\tfrac1{a}\cdot \langle\tfrac{\partial^2s}{\partial u\partial v},\Norm\rangle
=a\cdot m,
\end{align*}
that implies the last two equalities in \ref{eq:key-orthogonal/2}.
Therefore the first two identities in \ref{eq:key-orthogonal} are proved.
The other two identities in \ref{eq:key-orthogonal} can be proved along the same lines.
\qeds

\begin{thm}{Exericse}
Suppose that $(u,v)\mapsto s(u,v)$ is a \index{conformal chart}\emph{conformal chart};
that is, $\tfrac{\partial s}{\partial u}\perp\tfrac{\partial s}{\partial v}$ and $|\tfrac{\partial s}{\partial u}|=|\tfrac{\partial s}{\partial v}|$ for any $(u,v)$.
Show that Gauss curvature can be expressed as 
\[K=-\frac{\triangle (\ln b)}{b^2},\]
where $\triangle$ denotes the \emph{laplacian}; that is $\triangle=\tfrac{\partial^2}{\partial u^2}+\tfrac{\partial^2}{\partial v^2}$.
\end{thm}


\section*{The remarkable theorem}

Let $\Sigma_1$ and $\Sigma_2$ be two smooth regular surfaces in the Euclidean space.
A map $f\:\Sigma_1\to \Sigma_2$ is called  length-preserving if for any curve $\gamma_1$ in $\Sigma_1$ the curve $\gamma_2=f\circ\gamma_1$ in $\Sigma_2$ has the same length. %???it is sufficient to consider smooth only curves???
If in addition $f$ is smooth and bijective, then it is called \emph{intrinsic isometry}. 

A simple example of intrinsic isometry can obtained by warping a plane into a cylinder.
The following exercise produce slightly more interesting example.

\begin{thm}{Exercise}\label{ex:deformation}
Suppose $\gamma(t)=(x(t),y(t))$ is a smooth unit-speed curve in the plane such that $y(t)=a\cdot \cos t$.
Let $\Sigma_\gamma$ be the surface of revolution of $\gamma$ around the $x$-axis.
Show that a small open domain in $\Sigma_\gamma$ admits a smooth length-pereserving map to the unit sphere.

Conclude that any round disc $\Delta$ in $\SS^2$ of intrinsic radius smaller than $\tfrac\pi2$ admits a smooth length preserving deformation; that is, there is one parameter family of surfaces with boundary $\Delta_t$, such that $\Delta_0=\Delta$ and $\Delta_t$ is not congruent to $\Delta_0$ for any $t\ne0$.\footnote{In fact any disc in $\SS^2$ of intrinsic radius smaller than $\pi$ admits a smooth length preserving deformation. %???REF
}
\end{thm}


\begin{thm}{Theorem}\label{thm:remarkable}
Suppose $f\:\Sigma_1\to \Sigma_2$ is an intrinsic isometry between two smooth regular surfaces in  the Euclidean space; $p_1\in \Sigma_1$ and $p_2=f(p_1)\in \Sigma_1$.
Then 
\[K(p_1)_{\Sigma_1}=K(p_2)_{\Sigma_2};\]
that is, the Gauss curvature of $\Sigma_1$ at $p_1$ is the same as the Gauss curvature of $\Sigma_2$ at $p_2$.
\end{thm}

This theorem was proved by Carl Friedrich Gauss \cite{gauss} who called it \emph{Remarkable theorem} (Theorema Egregium).
The theorem is indeed remarkable because the Gauss curvature is defined as a product of principle curvatures which might be different at these points; however, according to the theorem, their product can not change.
In other words, the Gaussian curvature is an \emph{intrinsic invariant}.

In fact Gauss curvature of the surface at the given point can be found \emph{intrinsically},
by measuring the lengths of curves in the surface.
For example, Gauss curvature $K(p)$ in the following formula for the circumference $c(r)$ of a geodesic circle centered at $p$ in a surface: 
\[c(r)=2\cdot\pi\cdot r-\tfrac\pi3\cdot K(p)\cdot r^3+o(r^3).\]

Note that the theorem implies there is no smooth length-preserving map that sends an open region in the unit sphere to the plane.%
\footnote{There are plenty of non-smooth length-preserving maps from the sphere to the plane; see \cite{petrunin-yashinski} and the references there in.}
It follows since the Gauss curvature of the plane is zero and the unit sphere has Gauss curvature 1. 
In other words, there is no map of a region on Earth without distortion.

\parit{Proof.}
Choose a chart $(u,v)\mapsto s_1(u,v)$ on $\Sigma_1$ and set
$s_2 f\circ s_1$.
Note that $s_2$ is a chart of $\Sigma_2$ and 
\begin{align*}
\langle\tfrac{\partial s_1}{\partial u},\tfrac{\partial s_1}{\partial u}\rangle
&=
\langle\tfrac{\partial s_2}{\partial u},\tfrac{\partial s_2}{\partial u}\rangle,
\\
\langle\tfrac{\partial s_1}{\partial u},\tfrac{\partial s_1}{\partial v}\rangle
&=
\langle\tfrac{\partial s_2}{\partial u},\tfrac{\partial s_2}{\partial v}\rangle,
\\
\langle\tfrac{\partial s_1}{\partial v},\tfrac{\partial s_1}{\partial v}\rangle
&=
\langle\tfrac{\partial s_2}{\partial v},\tfrac{\partial s_2}{\partial v}\rangle
\end{align*}
at any $(u,v)$.
Indeed the first and the third identity hold since otherwise $f$ does not preserve length of coordinate lines $\gamma\:t\mapsto s(t,v)$ or  $\gamma\:t\z\mapsto s(u,t)$.
Taking this into account, the second identity hold since otherwise $f$ does not preserve length of coordinate lines $\gamma\:t\z\mapsto s(t,c-t)$ for some constant $c$.

It follows that if $s_1$ is a semigeodesic chart of $\Sigma_1$,
then $s_2$ is a semigeodesic chart of $\Sigma_2$.
It remains to apply \ref{prop:K(semigeodesic)}.
\qeds


\section*{Gauss--Bonnet formula: a formal proof}\label{page:gauss--bonnet:formal}

The following proof relies on two statements.

First, the following identity in polar coordinates $(r,\theta)\mapsto w_p(r,\theta)$ on a smooth surface:
\[K=\tfrac1b\cdot\tfrac{\partial^2 b}{\partial \theta^2},\]
where $b=|\tfrac{\partial w_p}{\partial \theta}|$; see \ref{lem:K(orthogonal)}.

Second, is the Green formula which can be formulated the following way.
Let $\gamma$ be a piecewise smooth simple closed curve in a plane that bounds a compact region $D$.
Suppose that $\gamma$ is oriented in such a way that $D$ lies on the left from $\gamma$.
Then for any two smooth functions $P$ and $Q$ defined on $D$ we have
\[\iint_D (\tfrac{\partial Q}{\partial x}-\tfrac{\partial P}{\partial y})\cdot dx\cdot dy=\int_\gamma (P\cdot dx+Q\cdot dy).\]

Note that Green and similar to Gauss--Bonnet formulas are similar --- they relate the integral along a disc and its boundary curve.
So it should be not surprising Green helps to prove Gauss--Bonnet.

\begin{thm}{Lemma}
Let $\vec u,\vec v,\vec w$ be two tangent vector fields along a piecewise smooth curve $\gamma$ on a smooth surface $\Sigma$.
Suppose that $\gamma$ is parametrized by length and
at each point the frame $\vec u(s),\vec v(s)$ is oriented and orthonormal for any $s$;
that is, $|\vec u|=|\vec v|=1$ and $\vec v$ obtained from $\vec u$ by counterclockwise rotation of the tangent plane at each point.
Further suppose that $\vec w$ is a parallel vector field.
Then there is a smooth function $\phi(s)$ such that $\vec u(s)$ is a counterclockwise rotation of $\vec w(s)$ and
\[\phi'(s)=\langle\vec u'(s),\vec v(s)\rangle\]
for any $s$.

Moreover if $\gamma\:[a,b]\to\Sigma$ is a loop based at a point $p$ then the parallel transport $\iota\:\T_p\to\T_p$ along $\gamma$ is a counterclockwise roatation by angle
\[\omega=\phi(a)-\phi(b)=-\int_a^b\phi'(s)\cdot ds.\]
\end{thm}

\begin{thm}{Corollary}
Suppose that $\gamma$ is a piecewise smooth loop on a smooth surface $\Sigma$.
\end{thm}

\section*{Rauch comparison}

The following proposition is a baby case of the so called \emph{Rauch comparison theorem}.

\begin{thm}{Proposition}
Suppose that $p$ is a point on a smooth surface $\Sigma$ and $\exp_p\:\T_p\to \Sigma$ is the exponential map.
Given a curve $\tilde\gamma$ in the tangent plane $\T_p$, set 
\[\gamma=\exp_p\circ\tilde\gamma;\]
note that $\gamma$ is a curve in $\Sigma$.

\begin{subthm}{}
If $\Sigma$ has nonnegative Gauss curvature, then $\exp_p$ is length nonexpanding in a small neighborhood of $0$;
that is, there is $r_0>0$ such that $\tilde\gamma$ lies in a $r_0$ neighborhood of $0$ in the tangent plane $\T_p$, then
\[\length \gamma\le \length \tilde\gamma.\]
\end{subthm}

\begin{subthm}{}
If $\Sigma$ has nonpositive Gauss curvature, then $\exp_p$ is length noncontractin in a small neighborhood of $0$;
that is, there is $r_0>0$ such that $\tilde\gamma$ lies in a $r_0$ neighborhood of $0$ in the tangent plane $\T_p$, then
\[\length \gamma\ge \length \tilde\gamma.\]
\end{subthm}

\end{thm}

\parit{Proof.}
Without loss of generality we may assume that $\tilde\gamma\:[a,b]\to \T_p$ is parameterized by length;
in particular it is a Lischitz curve.

Suppose $t\mapsto (\theta(t),r(t})$ is the curve $\tilde\gamma$ in polar coordinates.
Note that $\gamma(t)=w_p(\theta(t),r(t})$; that is, $\gamma$ has polar coordinates $(\theta(t),r(t})$ on $\Sigma$.
Set $b(\theta,r)\df|\tfrac{\partial w_p}{\partial \theta}|$.

By \ref{prop:K(semigeodesic)}
\[\tfrac{\partial^2 b}{\partial u^2}=-K\cdot b.\]
If $K\ge 0$, then $r\mapsto b(\theta,r)$ is concave
and
if $K\le 0$, then $r\mapsto b(\theta,r)$ is convex for any fixed $\theta$.
Observe that $b(\theta,0)=0$ and $\tfrac{\partial b}{\partial \theta}(\theta,0)=1$.
Therefor $b(\theta,r)\le r$ if $K\ge 0$ and $b(\theta,r)\ge r$ if $K\le 0$.

Note that
\begin{align*}
\length\tilde\gamma&=\int_a^b\sqrt{r'(t)^2+r(t)^2\cdot\theta'(t)^2}.
\intertext{Applying \ref{lem:palar-perp}, we get}
\length\gamma&=\int_a^b\sqrt{r'(t)^2+b(\theta(t),r(t))^2\cdot\theta'(t)^2}.
\end{align*}
Whence the statement follows.
\qeds
